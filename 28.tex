\begin{center}\section{ሱራቱ አልቀሰስ -  \textarabic{سوره  القصص}}\end{center}
\begin{longtable}{%
  @{}
    p{.5\textwidth}
  @{~~~}
    p{.5\textwidth}
    @{}
}
ቢስሚላሂ አራህመኒ ራሂይም &  \mytextarabic{بِسْمِ ٱللَّهِ ٱلرَّحْمَـٰنِ ٱلرَّحِيمِ}\\
1.\  & \mytextarabic{ طسٓمٓ ﴿١﴾}\\
2.\  & \mytextarabic{تِلْكَ ءَايَـٰتُ ٱلْكِتَـٰبِ ٱلْمُبِينِ ﴿٢﴾}\\
3.\  & \mytextarabic{نَتْلُوا۟ عَلَيْكَ مِن نَّبَإِ مُوسَىٰ وَفِرْعَوْنَ بِٱلْحَقِّ لِقَوْمٍۢ يُؤْمِنُونَ ﴿٣﴾}\\
4.\  & \mytextarabic{إِنَّ فِرْعَوْنَ عَلَا فِى ٱلْأَرْضِ وَجَعَلَ أَهْلَهَا شِيَعًۭا يَسْتَضْعِفُ طَآئِفَةًۭ مِّنْهُمْ يُذَبِّحُ أَبْنَآءَهُمْ وَيَسْتَحْىِۦ نِسَآءَهُمْ ۚ إِنَّهُۥ كَانَ مِنَ ٱلْمُفْسِدِينَ ﴿٤﴾}\\
5.\  & \mytextarabic{وَنُرِيدُ أَن نَّمُنَّ عَلَى ٱلَّذِينَ ٱسْتُضْعِفُوا۟ فِى ٱلْأَرْضِ وَنَجْعَلَهُمْ أَئِمَّةًۭ وَنَجْعَلَهُمُ ٱلْوَٟرِثِينَ ﴿٥﴾}\\
6.\  & \mytextarabic{وَنُمَكِّنَ لَهُمْ فِى ٱلْأَرْضِ وَنُرِىَ فِرْعَوْنَ وَهَـٰمَـٰنَ وَجُنُودَهُمَا مِنْهُم مَّا كَانُوا۟ يَحْذَرُونَ ﴿٦﴾}\\
7.\  & \mytextarabic{وَأَوْحَيْنَآ إِلَىٰٓ أُمِّ مُوسَىٰٓ أَنْ أَرْضِعِيهِ ۖ فَإِذَا خِفْتِ عَلَيْهِ فَأَلْقِيهِ فِى ٱلْيَمِّ وَلَا تَخَافِى وَلَا تَحْزَنِىٓ ۖ إِنَّا رَآدُّوهُ إِلَيْكِ وَجَاعِلُوهُ مِنَ ٱلْمُرْسَلِينَ ﴿٧﴾}\\
8.\  & \mytextarabic{فَٱلْتَقَطَهُۥٓ ءَالُ فِرْعَوْنَ لِيَكُونَ لَهُمْ عَدُوًّۭا وَحَزَنًا ۗ إِنَّ فِرْعَوْنَ وَهَـٰمَـٰنَ وَجُنُودَهُمَا كَانُوا۟ خَـٰطِـِٔينَ ﴿٨﴾}\\
9.\  & \mytextarabic{وَقَالَتِ ٱمْرَأَتُ فِرْعَوْنَ قُرَّتُ عَيْنٍۢ لِّى وَلَكَ ۖ لَا تَقْتُلُوهُ عَسَىٰٓ أَن يَنفَعَنَآ أَوْ نَتَّخِذَهُۥ وَلَدًۭا وَهُمْ لَا يَشْعُرُونَ ﴿٩﴾}\\
10.\  & \mytextarabic{وَأَصْبَحَ فُؤَادُ أُمِّ مُوسَىٰ فَـٰرِغًا ۖ إِن كَادَتْ لَتُبْدِى بِهِۦ لَوْلَآ أَن رَّبَطْنَا عَلَىٰ قَلْبِهَا لِتَكُونَ مِنَ ٱلْمُؤْمِنِينَ ﴿١٠﴾}\\
11.\  & \mytextarabic{وَقَالَتْ لِأُخْتِهِۦ قُصِّيهِ ۖ فَبَصُرَتْ بِهِۦ عَن جُنُبٍۢ وَهُمْ لَا يَشْعُرُونَ ﴿١١﴾}\\
12.\  & \mytextarabic{۞ وَحَرَّمْنَا عَلَيْهِ ٱلْمَرَاضِعَ مِن قَبْلُ فَقَالَتْ هَلْ أَدُلُّكُمْ عَلَىٰٓ أَهْلِ بَيْتٍۢ يَكْفُلُونَهُۥ لَكُمْ وَهُمْ لَهُۥ نَـٰصِحُونَ ﴿١٢﴾}\\
13.\  & \mytextarabic{فَرَدَدْنَـٰهُ إِلَىٰٓ أُمِّهِۦ كَىْ تَقَرَّ عَيْنُهَا وَلَا تَحْزَنَ وَلِتَعْلَمَ أَنَّ وَعْدَ ٱللَّهِ حَقٌّۭ وَلَـٰكِنَّ أَكْثَرَهُمْ لَا يَعْلَمُونَ ﴿١٣﴾}\\
14.\  & \mytextarabic{وَلَمَّا بَلَغَ أَشُدَّهُۥ وَٱسْتَوَىٰٓ ءَاتَيْنَـٰهُ حُكْمًۭا وَعِلْمًۭا ۚ وَكَذَٟلِكَ نَجْزِى ٱلْمُحْسِنِينَ ﴿١٤﴾}\\
15.\  & \mytextarabic{وَدَخَلَ ٱلْمَدِينَةَ عَلَىٰ حِينِ غَفْلَةٍۢ مِّنْ أَهْلِهَا فَوَجَدَ فِيهَا رَجُلَيْنِ يَقْتَتِلَانِ هَـٰذَا مِن شِيعَتِهِۦ وَهَـٰذَا مِنْ عَدُوِّهِۦ ۖ فَٱسْتَغَٰثَهُ ٱلَّذِى مِن شِيعَتِهِۦ عَلَى ٱلَّذِى مِنْ عَدُوِّهِۦ فَوَكَزَهُۥ مُوسَىٰ فَقَضَىٰ عَلَيْهِ ۖ قَالَ هَـٰذَا مِنْ عَمَلِ ٱلشَّيْطَٰنِ ۖ إِنَّهُۥ عَدُوٌّۭ مُّضِلٌّۭ مُّبِينٌۭ ﴿١٥﴾}\\
16.\  & \mytextarabic{قَالَ رَبِّ إِنِّى ظَلَمْتُ نَفْسِى فَٱغْفِرْ لِى فَغَفَرَ لَهُۥٓ ۚ إِنَّهُۥ هُوَ ٱلْغَفُورُ ٱلرَّحِيمُ ﴿١٦﴾}\\
17.\  & \mytextarabic{قَالَ رَبِّ بِمَآ أَنْعَمْتَ عَلَىَّ فَلَنْ أَكُونَ ظَهِيرًۭا لِّلْمُجْرِمِينَ ﴿١٧﴾}\\
18.\  & \mytextarabic{فَأَصْبَحَ فِى ٱلْمَدِينَةِ خَآئِفًۭا يَتَرَقَّبُ فَإِذَا ٱلَّذِى ٱسْتَنصَرَهُۥ بِٱلْأَمْسِ يَسْتَصْرِخُهُۥ ۚ قَالَ لَهُۥ مُوسَىٰٓ إِنَّكَ لَغَوِىٌّۭ مُّبِينٌۭ ﴿١٨﴾}\\
19.\  & \mytextarabic{فَلَمَّآ أَنْ أَرَادَ أَن يَبْطِشَ بِٱلَّذِى هُوَ عَدُوٌّۭ لَّهُمَا قَالَ يَـٰمُوسَىٰٓ أَتُرِيدُ أَن تَقْتُلَنِى كَمَا قَتَلْتَ نَفْسًۢا بِٱلْأَمْسِ ۖ إِن تُرِيدُ إِلَّآ أَن تَكُونَ جَبَّارًۭا فِى ٱلْأَرْضِ وَمَا تُرِيدُ أَن تَكُونَ مِنَ ٱلْمُصْلِحِينَ ﴿١٩﴾}\\
20.\  & \mytextarabic{وَجَآءَ رَجُلٌۭ مِّنْ أَقْصَا ٱلْمَدِينَةِ يَسْعَىٰ قَالَ يَـٰمُوسَىٰٓ إِنَّ ٱلْمَلَأَ يَأْتَمِرُونَ بِكَ لِيَقْتُلُوكَ فَٱخْرُجْ إِنِّى لَكَ مِنَ ٱلنَّـٰصِحِينَ ﴿٢٠﴾}\\
21.\  & \mytextarabic{فَخَرَجَ مِنْهَا خَآئِفًۭا يَتَرَقَّبُ ۖ قَالَ رَبِّ نَجِّنِى مِنَ ٱلْقَوْمِ ٱلظَّـٰلِمِينَ ﴿٢١﴾}\\
22.\  & \mytextarabic{وَلَمَّا تَوَجَّهَ تِلْقَآءَ مَدْيَنَ قَالَ عَسَىٰ رَبِّىٓ أَن يَهْدِيَنِى سَوَآءَ ٱلسَّبِيلِ ﴿٢٢﴾}\\
23.\  & \mytextarabic{وَلَمَّا وَرَدَ مَآءَ مَدْيَنَ وَجَدَ عَلَيْهِ أُمَّةًۭ مِّنَ ٱلنَّاسِ يَسْقُونَ وَوَجَدَ مِن دُونِهِمُ ٱمْرَأَتَيْنِ تَذُودَانِ ۖ قَالَ مَا خَطْبُكُمَا ۖ قَالَتَا لَا نَسْقِى حَتَّىٰ يُصْدِرَ ٱلرِّعَآءُ ۖ وَأَبُونَا شَيْخٌۭ كَبِيرٌۭ ﴿٢٣﴾}\\
24.\  & \mytextarabic{فَسَقَىٰ لَهُمَا ثُمَّ تَوَلَّىٰٓ إِلَى ٱلظِّلِّ فَقَالَ رَبِّ إِنِّى لِمَآ أَنزَلْتَ إِلَىَّ مِنْ خَيْرٍۢ فَقِيرٌۭ ﴿٢٤﴾}\\
25.\  & \mytextarabic{فَجَآءَتْهُ إِحْدَىٰهُمَا تَمْشِى عَلَى ٱسْتِحْيَآءٍۢ قَالَتْ إِنَّ أَبِى يَدْعُوكَ لِيَجْزِيَكَ أَجْرَ مَا سَقَيْتَ لَنَا ۚ فَلَمَّا جَآءَهُۥ وَقَصَّ عَلَيْهِ ٱلْقَصَصَ قَالَ لَا تَخَفْ ۖ نَجَوْتَ مِنَ ٱلْقَوْمِ ٱلظَّـٰلِمِينَ ﴿٢٥﴾}\\
26.\  & \mytextarabic{قَالَتْ إِحْدَىٰهُمَا يَـٰٓأَبَتِ ٱسْتَـْٔجِرْهُ ۖ إِنَّ خَيْرَ مَنِ ٱسْتَـْٔجَرْتَ ٱلْقَوِىُّ ٱلْأَمِينُ ﴿٢٦﴾}\\
27.\  & \mytextarabic{قَالَ إِنِّىٓ أُرِيدُ أَنْ أُنكِحَكَ إِحْدَى ٱبْنَتَىَّ هَـٰتَيْنِ عَلَىٰٓ أَن تَأْجُرَنِى ثَمَـٰنِىَ حِجَجٍۢ ۖ فَإِنْ أَتْمَمْتَ عَشْرًۭا فَمِنْ عِندِكَ ۖ وَمَآ أُرِيدُ أَنْ أَشُقَّ عَلَيْكَ ۚ سَتَجِدُنِىٓ إِن شَآءَ ٱللَّهُ مِنَ ٱلصَّـٰلِحِينَ ﴿٢٧﴾}\\
28.\  & \mytextarabic{قَالَ ذَٟلِكَ بَيْنِى وَبَيْنَكَ ۖ أَيَّمَا ٱلْأَجَلَيْنِ قَضَيْتُ فَلَا عُدْوَٟنَ عَلَىَّ ۖ وَٱللَّهُ عَلَىٰ مَا نَقُولُ وَكِيلٌۭ ﴿٢٨﴾}\\
29.\  & \mytextarabic{۞ فَلَمَّا قَضَىٰ مُوسَى ٱلْأَجَلَ وَسَارَ بِأَهْلِهِۦٓ ءَانَسَ مِن جَانِبِ ٱلطُّورِ نَارًۭا قَالَ لِأَهْلِهِ ٱمْكُثُوٓا۟ إِنِّىٓ ءَانَسْتُ نَارًۭا لَّعَلِّىٓ ءَاتِيكُم مِّنْهَا بِخَبَرٍ أَوْ جَذْوَةٍۢ مِّنَ ٱلنَّارِ لَعَلَّكُمْ تَصْطَلُونَ ﴿٢٩﴾}\\
30.\  & \mytextarabic{فَلَمَّآ أَتَىٰهَا نُودِىَ مِن شَـٰطِئِ ٱلْوَادِ ٱلْأَيْمَنِ فِى ٱلْبُقْعَةِ ٱلْمُبَٰرَكَةِ مِنَ ٱلشَّجَرَةِ أَن يَـٰمُوسَىٰٓ إِنِّىٓ أَنَا ٱللَّهُ رَبُّ ٱلْعَـٰلَمِينَ ﴿٣٠﴾}\\
31.\  & \mytextarabic{وَأَنْ أَلْقِ عَصَاكَ ۖ فَلَمَّا رَءَاهَا تَهْتَزُّ كَأَنَّهَا جَآنٌّۭ وَلَّىٰ مُدْبِرًۭا وَلَمْ يُعَقِّبْ ۚ يَـٰمُوسَىٰٓ أَقْبِلْ وَلَا تَخَفْ ۖ إِنَّكَ مِنَ ٱلْءَامِنِينَ ﴿٣١﴾}\\
32.\  & \mytextarabic{ٱسْلُكْ يَدَكَ فِى جَيْبِكَ تَخْرُجْ بَيْضَآءَ مِنْ غَيْرِ سُوٓءٍۢ وَٱضْمُمْ إِلَيْكَ جَنَاحَكَ مِنَ ٱلرَّهْبِ ۖ فَذَٟنِكَ بُرْهَـٰنَانِ مِن رَّبِّكَ إِلَىٰ فِرْعَوْنَ وَمَلَإِي۟هِۦٓ ۚ إِنَّهُمْ كَانُوا۟ قَوْمًۭا فَـٰسِقِينَ ﴿٣٢﴾}\\
33.\  & \mytextarabic{قَالَ رَبِّ إِنِّى قَتَلْتُ مِنْهُمْ نَفْسًۭا فَأَخَافُ أَن يَقْتُلُونِ ﴿٣٣﴾}\\
34.\  & \mytextarabic{وَأَخِى هَـٰرُونُ هُوَ أَفْصَحُ مِنِّى لِسَانًۭا فَأَرْسِلْهُ مَعِىَ رِدْءًۭا يُصَدِّقُنِىٓ ۖ إِنِّىٓ أَخَافُ أَن يُكَذِّبُونِ ﴿٣٤﴾}\\
35.\  & \mytextarabic{قَالَ سَنَشُدُّ عَضُدَكَ بِأَخِيكَ وَنَجْعَلُ لَكُمَا سُلْطَٰنًۭا فَلَا يَصِلُونَ إِلَيْكُمَا ۚ بِـَٔايَـٰتِنَآ أَنتُمَا وَمَنِ ٱتَّبَعَكُمَا ٱلْغَٰلِبُونَ ﴿٣٥﴾}\\
36.\  & \mytextarabic{فَلَمَّا جَآءَهُم مُّوسَىٰ بِـَٔايَـٰتِنَا بَيِّنَـٰتٍۢ قَالُوا۟ مَا هَـٰذَآ إِلَّا سِحْرٌۭ مُّفْتَرًۭى وَمَا سَمِعْنَا بِهَـٰذَا فِىٓ ءَابَآئِنَا ٱلْأَوَّلِينَ ﴿٣٦﴾}\\
37.\  & \mytextarabic{وَقَالَ مُوسَىٰ رَبِّىٓ أَعْلَمُ بِمَن جَآءَ بِٱلْهُدَىٰ مِنْ عِندِهِۦ وَمَن تَكُونُ لَهُۥ عَـٰقِبَةُ ٱلدَّارِ ۖ إِنَّهُۥ لَا يُفْلِحُ ٱلظَّـٰلِمُونَ ﴿٣٧﴾}\\
38.\  & \mytextarabic{وَقَالَ فِرْعَوْنُ يَـٰٓأَيُّهَا ٱلْمَلَأُ مَا عَلِمْتُ لَكُم مِّنْ إِلَـٰهٍ غَيْرِى فَأَوْقِدْ لِى يَـٰهَـٰمَـٰنُ عَلَى ٱلطِّينِ فَٱجْعَل لِّى صَرْحًۭا لَّعَلِّىٓ أَطَّلِعُ إِلَىٰٓ إِلَـٰهِ مُوسَىٰ وَإِنِّى لَأَظُنُّهُۥ مِنَ ٱلْكَـٰذِبِينَ ﴿٣٨﴾}\\
39.\  & \mytextarabic{وَٱسْتَكْبَرَ هُوَ وَجُنُودُهُۥ فِى ٱلْأَرْضِ بِغَيْرِ ٱلْحَقِّ وَظَنُّوٓا۟ أَنَّهُمْ إِلَيْنَا لَا يُرْجَعُونَ ﴿٣٩﴾}\\
40.\  & \mytextarabic{فَأَخَذْنَـٰهُ وَجُنُودَهُۥ فَنَبَذْنَـٰهُمْ فِى ٱلْيَمِّ ۖ فَٱنظُرْ كَيْفَ كَانَ عَـٰقِبَةُ ٱلظَّـٰلِمِينَ ﴿٤٠﴾}\\
41.\  & \mytextarabic{وَجَعَلْنَـٰهُمْ أَئِمَّةًۭ يَدْعُونَ إِلَى ٱلنَّارِ ۖ وَيَوْمَ ٱلْقِيَـٰمَةِ لَا يُنصَرُونَ ﴿٤١﴾}\\
42.\  & \mytextarabic{وَأَتْبَعْنَـٰهُمْ فِى هَـٰذِهِ ٱلدُّنْيَا لَعْنَةًۭ ۖ وَيَوْمَ ٱلْقِيَـٰمَةِ هُم مِّنَ ٱلْمَقْبُوحِينَ ﴿٤٢﴾}\\
43.\  & \mytextarabic{وَلَقَدْ ءَاتَيْنَا مُوسَى ٱلْكِتَـٰبَ مِنۢ بَعْدِ مَآ أَهْلَكْنَا ٱلْقُرُونَ ٱلْأُولَىٰ بَصَآئِرَ لِلنَّاسِ وَهُدًۭى وَرَحْمَةًۭ لَّعَلَّهُمْ يَتَذَكَّرُونَ ﴿٤٣﴾}\\
44.\  & \mytextarabic{وَمَا كُنتَ بِجَانِبِ ٱلْغَرْبِىِّ إِذْ قَضَيْنَآ إِلَىٰ مُوسَى ٱلْأَمْرَ وَمَا كُنتَ مِنَ ٱلشَّـٰهِدِينَ ﴿٤٤﴾}\\
45.\  & \mytextarabic{وَلَـٰكِنَّآ أَنشَأْنَا قُرُونًۭا فَتَطَاوَلَ عَلَيْهِمُ ٱلْعُمُرُ ۚ وَمَا كُنتَ ثَاوِيًۭا فِىٓ أَهْلِ مَدْيَنَ تَتْلُوا۟ عَلَيْهِمْ ءَايَـٰتِنَا وَلَـٰكِنَّا كُنَّا مُرْسِلِينَ ﴿٤٥﴾}\\
46.\  & \mytextarabic{وَمَا كُنتَ بِجَانِبِ ٱلطُّورِ إِذْ نَادَيْنَا وَلَـٰكِن رَّحْمَةًۭ مِّن رَّبِّكَ لِتُنذِرَ قَوْمًۭا مَّآ أَتَىٰهُم مِّن نَّذِيرٍۢ مِّن قَبْلِكَ لَعَلَّهُمْ يَتَذَكَّرُونَ ﴿٤٦﴾}\\
47.\  & \mytextarabic{وَلَوْلَآ أَن تُصِيبَهُم مُّصِيبَةٌۢ بِمَا قَدَّمَتْ أَيْدِيهِمْ فَيَقُولُوا۟ رَبَّنَا لَوْلَآ أَرْسَلْتَ إِلَيْنَا رَسُولًۭا فَنَتَّبِعَ ءَايَـٰتِكَ وَنَكُونَ مِنَ ٱلْمُؤْمِنِينَ ﴿٤٧﴾}\\
48.\  & \mytextarabic{فَلَمَّا جَآءَهُمُ ٱلْحَقُّ مِنْ عِندِنَا قَالُوا۟ لَوْلَآ أُوتِىَ مِثْلَ مَآ أُوتِىَ مُوسَىٰٓ ۚ أَوَلَمْ يَكْفُرُوا۟ بِمَآ أُوتِىَ مُوسَىٰ مِن قَبْلُ ۖ قَالُوا۟ سِحْرَانِ تَظَـٰهَرَا وَقَالُوٓا۟ إِنَّا بِكُلٍّۢ كَـٰفِرُونَ ﴿٤٨﴾}\\
49.\  & \mytextarabic{قُلْ فَأْتُوا۟ بِكِتَـٰبٍۢ مِّنْ عِندِ ٱللَّهِ هُوَ أَهْدَىٰ مِنْهُمَآ أَتَّبِعْهُ إِن كُنتُمْ صَـٰدِقِينَ ﴿٤٩﴾}\\
50.\  & \mytextarabic{فَإِن لَّمْ يَسْتَجِيبُوا۟ لَكَ فَٱعْلَمْ أَنَّمَا يَتَّبِعُونَ أَهْوَآءَهُمْ ۚ وَمَنْ أَضَلُّ مِمَّنِ ٱتَّبَعَ هَوَىٰهُ بِغَيْرِ هُدًۭى مِّنَ ٱللَّهِ ۚ إِنَّ ٱللَّهَ لَا يَهْدِى ٱلْقَوْمَ ٱلظَّـٰلِمِينَ ﴿٥٠﴾}\\
51.\  & \mytextarabic{۞ وَلَقَدْ وَصَّلْنَا لَهُمُ ٱلْقَوْلَ لَعَلَّهُمْ يَتَذَكَّرُونَ ﴿٥١﴾}\\
52.\  & \mytextarabic{ٱلَّذِينَ ءَاتَيْنَـٰهُمُ ٱلْكِتَـٰبَ مِن قَبْلِهِۦ هُم بِهِۦ يُؤْمِنُونَ ﴿٥٢﴾}\\
53.\  & \mytextarabic{وَإِذَا يُتْلَىٰ عَلَيْهِمْ قَالُوٓا۟ ءَامَنَّا بِهِۦٓ إِنَّهُ ٱلْحَقُّ مِن رَّبِّنَآ إِنَّا كُنَّا مِن قَبْلِهِۦ مُسْلِمِينَ ﴿٥٣﴾}\\
54.\  & \mytextarabic{أُو۟لَـٰٓئِكَ يُؤْتَوْنَ أَجْرَهُم مَّرَّتَيْنِ بِمَا صَبَرُوا۟ وَيَدْرَءُونَ بِٱلْحَسَنَةِ ٱلسَّيِّئَةَ وَمِمَّا رَزَقْنَـٰهُمْ يُنفِقُونَ ﴿٥٤﴾}\\
55.\  & \mytextarabic{وَإِذَا سَمِعُوا۟ ٱللَّغْوَ أَعْرَضُوا۟ عَنْهُ وَقَالُوا۟ لَنَآ أَعْمَـٰلُنَا وَلَكُمْ أَعْمَـٰلُكُمْ سَلَـٰمٌ عَلَيْكُمْ لَا نَبْتَغِى ٱلْجَٰهِلِينَ ﴿٥٥﴾}\\
56.\  & \mytextarabic{إِنَّكَ لَا تَهْدِى مَنْ أَحْبَبْتَ وَلَـٰكِنَّ ٱللَّهَ يَهْدِى مَن يَشَآءُ ۚ وَهُوَ أَعْلَمُ بِٱلْمُهْتَدِينَ ﴿٥٦﴾}\\
57.\  & \mytextarabic{وَقَالُوٓا۟ إِن نَّتَّبِعِ ٱلْهُدَىٰ مَعَكَ نُتَخَطَّفْ مِنْ أَرْضِنَآ ۚ أَوَلَمْ نُمَكِّن لَّهُمْ حَرَمًا ءَامِنًۭا يُجْبَىٰٓ إِلَيْهِ ثَمَرَٰتُ كُلِّ شَىْءٍۢ رِّزْقًۭا مِّن لَّدُنَّا وَلَـٰكِنَّ أَكْثَرَهُمْ لَا يَعْلَمُونَ ﴿٥٧﴾}\\
58.\  & \mytextarabic{وَكَمْ أَهْلَكْنَا مِن قَرْيَةٍۭ بَطِرَتْ مَعِيشَتَهَا ۖ فَتِلْكَ مَسَـٰكِنُهُمْ لَمْ تُسْكَن مِّنۢ بَعْدِهِمْ إِلَّا قَلِيلًۭا ۖ وَكُنَّا نَحْنُ ٱلْوَٟرِثِينَ ﴿٥٨﴾}\\
59.\  & \mytextarabic{وَمَا كَانَ رَبُّكَ مُهْلِكَ ٱلْقُرَىٰ حَتَّىٰ يَبْعَثَ فِىٓ أُمِّهَا رَسُولًۭا يَتْلُوا۟ عَلَيْهِمْ ءَايَـٰتِنَا ۚ وَمَا كُنَّا مُهْلِكِى ٱلْقُرَىٰٓ إِلَّا وَأَهْلُهَا ظَـٰلِمُونَ ﴿٥٩﴾}\\
60.\  & \mytextarabic{وَمَآ أُوتِيتُم مِّن شَىْءٍۢ فَمَتَـٰعُ ٱلْحَيَوٰةِ ٱلدُّنْيَا وَزِينَتُهَا ۚ وَمَا عِندَ ٱللَّهِ خَيْرٌۭ وَأَبْقَىٰٓ ۚ أَفَلَا تَعْقِلُونَ ﴿٦٠﴾}\\
61.\  & \mytextarabic{أَفَمَن وَعَدْنَـٰهُ وَعْدًا حَسَنًۭا فَهُوَ لَـٰقِيهِ كَمَن مَّتَّعْنَـٰهُ مَتَـٰعَ ٱلْحَيَوٰةِ ٱلدُّنْيَا ثُمَّ هُوَ يَوْمَ ٱلْقِيَـٰمَةِ مِنَ ٱلْمُحْضَرِينَ ﴿٦١﴾}\\
62.\  & \mytextarabic{وَيَوْمَ يُنَادِيهِمْ فَيَقُولُ أَيْنَ شُرَكَآءِىَ ٱلَّذِينَ كُنتُمْ تَزْعُمُونَ ﴿٦٢﴾}\\
63.\  & \mytextarabic{قَالَ ٱلَّذِينَ حَقَّ عَلَيْهِمُ ٱلْقَوْلُ رَبَّنَا هَـٰٓؤُلَآءِ ٱلَّذِينَ أَغْوَيْنَآ أَغْوَيْنَـٰهُمْ كَمَا غَوَيْنَا ۖ تَبَرَّأْنَآ إِلَيْكَ ۖ مَا كَانُوٓا۟ إِيَّانَا يَعْبُدُونَ ﴿٦٣﴾}\\
64.\  & \mytextarabic{وَقِيلَ ٱدْعُوا۟ شُرَكَآءَكُمْ فَدَعَوْهُمْ فَلَمْ يَسْتَجِيبُوا۟ لَهُمْ وَرَأَوُا۟ ٱلْعَذَابَ ۚ لَوْ أَنَّهُمْ كَانُوا۟ يَهْتَدُونَ ﴿٦٤﴾}\\
65.\  & \mytextarabic{وَيَوْمَ يُنَادِيهِمْ فَيَقُولُ مَاذَآ أَجَبْتُمُ ٱلْمُرْسَلِينَ ﴿٦٥﴾}\\
66.\  & \mytextarabic{فَعَمِيَتْ عَلَيْهِمُ ٱلْأَنۢبَآءُ يَوْمَئِذٍۢ فَهُمْ لَا يَتَسَآءَلُونَ ﴿٦٦﴾}\\
67.\  & \mytextarabic{فَأَمَّا مَن تَابَ وَءَامَنَ وَعَمِلَ صَـٰلِحًۭا فَعَسَىٰٓ أَن يَكُونَ مِنَ ٱلْمُفْلِحِينَ ﴿٦٧﴾}\\
68.\  & \mytextarabic{وَرَبُّكَ يَخْلُقُ مَا يَشَآءُ وَيَخْتَارُ ۗ مَا كَانَ لَهُمُ ٱلْخِيَرَةُ ۚ سُبْحَـٰنَ ٱللَّهِ وَتَعَـٰلَىٰ عَمَّا يُشْرِكُونَ ﴿٦٨﴾}\\
69.\  & \mytextarabic{وَرَبُّكَ يَعْلَمُ مَا تُكِنُّ صُدُورُهُمْ وَمَا يُعْلِنُونَ ﴿٦٩﴾}\\
70.\  & \mytextarabic{وَهُوَ ٱللَّهُ لَآ إِلَـٰهَ إِلَّا هُوَ ۖ لَهُ ٱلْحَمْدُ فِى ٱلْأُولَىٰ وَٱلْءَاخِرَةِ ۖ وَلَهُ ٱلْحُكْمُ وَإِلَيْهِ تُرْجَعُونَ ﴿٧٠﴾}\\
71.\  & \mytextarabic{قُلْ أَرَءَيْتُمْ إِن جَعَلَ ٱللَّهُ عَلَيْكُمُ ٱلَّيْلَ سَرْمَدًا إِلَىٰ يَوْمِ ٱلْقِيَـٰمَةِ مَنْ إِلَـٰهٌ غَيْرُ ٱللَّهِ يَأْتِيكُم بِضِيَآءٍ ۖ أَفَلَا تَسْمَعُونَ ﴿٧١﴾}\\
72.\  & \mytextarabic{قُلْ أَرَءَيْتُمْ إِن جَعَلَ ٱللَّهُ عَلَيْكُمُ ٱلنَّهَارَ سَرْمَدًا إِلَىٰ يَوْمِ ٱلْقِيَـٰمَةِ مَنْ إِلَـٰهٌ غَيْرُ ٱللَّهِ يَأْتِيكُم بِلَيْلٍۢ تَسْكُنُونَ فِيهِ ۖ أَفَلَا تُبْصِرُونَ ﴿٧٢﴾}\\
73.\  & \mytextarabic{وَمِن رَّحْمَتِهِۦ جَعَلَ لَكُمُ ٱلَّيْلَ وَٱلنَّهَارَ لِتَسْكُنُوا۟ فِيهِ وَلِتَبْتَغُوا۟ مِن فَضْلِهِۦ وَلَعَلَّكُمْ تَشْكُرُونَ ﴿٧٣﴾}\\
74.\  & \mytextarabic{وَيَوْمَ يُنَادِيهِمْ فَيَقُولُ أَيْنَ شُرَكَآءِىَ ٱلَّذِينَ كُنتُمْ تَزْعُمُونَ ﴿٧٤﴾}\\
75.\  & \mytextarabic{وَنَزَعْنَا مِن كُلِّ أُمَّةٍۢ شَهِيدًۭا فَقُلْنَا هَاتُوا۟ بُرْهَـٰنَكُمْ فَعَلِمُوٓا۟ أَنَّ ٱلْحَقَّ لِلَّهِ وَضَلَّ عَنْهُم مَّا كَانُوا۟ يَفْتَرُونَ ﴿٧٥﴾}\\
76.\  & \mytextarabic{۞ إِنَّ قَـٰرُونَ كَانَ مِن قَوْمِ مُوسَىٰ فَبَغَىٰ عَلَيْهِمْ ۖ وَءَاتَيْنَـٰهُ مِنَ ٱلْكُنُوزِ مَآ إِنَّ مَفَاتِحَهُۥ لَتَنُوٓأُ بِٱلْعُصْبَةِ أُو۟لِى ٱلْقُوَّةِ إِذْ قَالَ لَهُۥ قَوْمُهُۥ لَا تَفْرَحْ ۖ إِنَّ ٱللَّهَ لَا يُحِبُّ ٱلْفَرِحِينَ ﴿٧٦﴾}\\
77.\  & \mytextarabic{وَٱبْتَغِ فِيمَآ ءَاتَىٰكَ ٱللَّهُ ٱلدَّارَ ٱلْءَاخِرَةَ ۖ وَلَا تَنسَ نَصِيبَكَ مِنَ ٱلدُّنْيَا ۖ وَأَحْسِن كَمَآ أَحْسَنَ ٱللَّهُ إِلَيْكَ ۖ وَلَا تَبْغِ ٱلْفَسَادَ فِى ٱلْأَرْضِ ۖ إِنَّ ٱللَّهَ لَا يُحِبُّ ٱلْمُفْسِدِينَ ﴿٧٧﴾}\\
78.\  & \mytextarabic{قَالَ إِنَّمَآ أُوتِيتُهُۥ عَلَىٰ عِلْمٍ عِندِىٓ ۚ أَوَلَمْ يَعْلَمْ أَنَّ ٱللَّهَ قَدْ أَهْلَكَ مِن قَبْلِهِۦ مِنَ ٱلْقُرُونِ مَنْ هُوَ أَشَدُّ مِنْهُ قُوَّةًۭ وَأَكْثَرُ جَمْعًۭا ۚ وَلَا يُسْـَٔلُ عَن ذُنُوبِهِمُ ٱلْمُجْرِمُونَ ﴿٧٨﴾}\\
79.\  & \mytextarabic{فَخَرَجَ عَلَىٰ قَوْمِهِۦ فِى زِينَتِهِۦ ۖ قَالَ ٱلَّذِينَ يُرِيدُونَ ٱلْحَيَوٰةَ ٱلدُّنْيَا يَـٰلَيْتَ لَنَا مِثْلَ مَآ أُوتِىَ قَـٰرُونُ إِنَّهُۥ لَذُو حَظٍّ عَظِيمٍۢ ﴿٧٩﴾}\\
80.\  & \mytextarabic{وَقَالَ ٱلَّذِينَ أُوتُوا۟ ٱلْعِلْمَ وَيْلَكُمْ ثَوَابُ ٱللَّهِ خَيْرٌۭ لِّمَنْ ءَامَنَ وَعَمِلَ صَـٰلِحًۭا وَلَا يُلَقَّىٰهَآ إِلَّا ٱلصَّـٰبِرُونَ ﴿٨٠﴾}\\
81.\  & \mytextarabic{فَخَسَفْنَا بِهِۦ وَبِدَارِهِ ٱلْأَرْضَ فَمَا كَانَ لَهُۥ مِن فِئَةٍۢ يَنصُرُونَهُۥ مِن دُونِ ٱللَّهِ وَمَا كَانَ مِنَ ٱلْمُنتَصِرِينَ ﴿٨١﴾}\\
82.\  & \mytextarabic{وَأَصْبَحَ ٱلَّذِينَ تَمَنَّوْا۟ مَكَانَهُۥ بِٱلْأَمْسِ يَقُولُونَ وَيْكَأَنَّ ٱللَّهَ يَبْسُطُ ٱلرِّزْقَ لِمَن يَشَآءُ مِنْ عِبَادِهِۦ وَيَقْدِرُ ۖ لَوْلَآ أَن مَّنَّ ٱللَّهُ عَلَيْنَا لَخَسَفَ بِنَا ۖ وَيْكَأَنَّهُۥ لَا يُفْلِحُ ٱلْكَـٰفِرُونَ ﴿٨٢﴾}\\
83.\  & \mytextarabic{تِلْكَ ٱلدَّارُ ٱلْءَاخِرَةُ نَجْعَلُهَا لِلَّذِينَ لَا يُرِيدُونَ عُلُوًّۭا فِى ٱلْأَرْضِ وَلَا فَسَادًۭا ۚ وَٱلْعَـٰقِبَةُ لِلْمُتَّقِينَ ﴿٨٣﴾}\\
84.\  & \mytextarabic{مَن جَآءَ بِٱلْحَسَنَةِ فَلَهُۥ خَيْرٌۭ مِّنْهَا ۖ وَمَن جَآءَ بِٱلسَّيِّئَةِ فَلَا يُجْزَى ٱلَّذِينَ عَمِلُوا۟ ٱلسَّيِّـَٔاتِ إِلَّا مَا كَانُوا۟ يَعْمَلُونَ ﴿٨٤﴾}\\
85.\  & \mytextarabic{إِنَّ ٱلَّذِى فَرَضَ عَلَيْكَ ٱلْقُرْءَانَ لَرَآدُّكَ إِلَىٰ مَعَادٍۢ ۚ قُل رَّبِّىٓ أَعْلَمُ مَن جَآءَ بِٱلْهُدَىٰ وَمَنْ هُوَ فِى ضَلَـٰلٍۢ مُّبِينٍۢ ﴿٨٥﴾}\\
86.\  & \mytextarabic{وَمَا كُنتَ تَرْجُوٓا۟ أَن يُلْقَىٰٓ إِلَيْكَ ٱلْكِتَـٰبُ إِلَّا رَحْمَةًۭ مِّن رَّبِّكَ ۖ فَلَا تَكُونَنَّ ظَهِيرًۭا لِّلْكَـٰفِرِينَ ﴿٨٦﴾}\\
87.\  & \mytextarabic{وَلَا يَصُدُّنَّكَ عَنْ ءَايَـٰتِ ٱللَّهِ بَعْدَ إِذْ أُنزِلَتْ إِلَيْكَ ۖ وَٱدْعُ إِلَىٰ رَبِّكَ ۖ وَلَا تَكُونَنَّ مِنَ ٱلْمُشْرِكِينَ ﴿٨٧﴾}\\
88.\  & \mytextarabic{وَلَا تَدْعُ مَعَ ٱللَّهِ إِلَـٰهًا ءَاخَرَ ۘ لَآ إِلَـٰهَ إِلَّا هُوَ ۚ كُلُّ شَىْءٍ هَالِكٌ إِلَّا وَجْهَهُۥ ۚ لَهُ ٱلْحُكْمُ وَإِلَيْهِ تُرْجَعُونَ ﴿٨٨﴾}\\
\end{longtable}
\clearpage