\begin{center}\section{ሱራቱ ያሲን -  \textarabic{سوره  يس}}\end{center}
\begin{longtable}{%
  @{}
    p{.5\textwidth}
  @{~~~}
    p{.5\textwidth}
    @{}
}
ቢስሚላሂ አራህመኒ ራሂይም &  \mytextarabic{بِسْمِ ٱللَّهِ ٱلرَّحْمَـٰنِ ٱلرَّحِيمِ}\\
1.\  & \mytextarabic{ يسٓ ﴿١﴾}\\
2.\  & \mytextarabic{وَٱلْقُرْءَانِ ٱلْحَكِيمِ ﴿٢﴾}\\
3.\  & \mytextarabic{إِنَّكَ لَمِنَ ٱلْمُرْسَلِينَ ﴿٣﴾}\\
4.\  & \mytextarabic{عَلَىٰ صِرَٰطٍۢ مُّسْتَقِيمٍۢ ﴿٤﴾}\\
5.\  & \mytextarabic{تَنزِيلَ ٱلْعَزِيزِ ٱلرَّحِيمِ ﴿٥﴾}\\
6.\  & \mytextarabic{لِتُنذِرَ قَوْمًۭا مَّآ أُنذِرَ ءَابَآؤُهُمْ فَهُمْ غَٰفِلُونَ ﴿٦﴾}\\
7.\  & \mytextarabic{لَقَدْ حَقَّ ٱلْقَوْلُ عَلَىٰٓ أَكْثَرِهِمْ فَهُمْ لَا يُؤْمِنُونَ ﴿٧﴾}\\
8.\  & \mytextarabic{إِنَّا جَعَلْنَا فِىٓ أَعْنَـٰقِهِمْ أَغْلَـٰلًۭا فَهِىَ إِلَى ٱلْأَذْقَانِ فَهُم مُّقْمَحُونَ ﴿٨﴾}\\
9.\  & \mytextarabic{وَجَعَلْنَا مِنۢ بَيْنِ أَيْدِيهِمْ سَدًّۭا وَمِنْ خَلْفِهِمْ سَدًّۭا فَأَغْشَيْنَـٰهُمْ فَهُمْ لَا يُبْصِرُونَ ﴿٩﴾}\\
10.\  & \mytextarabic{وَسَوَآءٌ عَلَيْهِمْ ءَأَنذَرْتَهُمْ أَمْ لَمْ تُنذِرْهُمْ لَا يُؤْمِنُونَ ﴿١٠﴾}\\
11.\  & \mytextarabic{إِنَّمَا تُنذِرُ مَنِ ٱتَّبَعَ ٱلذِّكْرَ وَخَشِىَ ٱلرَّحْمَـٰنَ بِٱلْغَيْبِ ۖ فَبَشِّرْهُ بِمَغْفِرَةٍۢ وَأَجْرٍۢ كَرِيمٍ ﴿١١﴾}\\
12.\  & \mytextarabic{إِنَّا نَحْنُ نُحْىِ ٱلْمَوْتَىٰ وَنَكْتُبُ مَا قَدَّمُوا۟ وَءَاثَـٰرَهُمْ ۚ وَكُلَّ شَىْءٍ أَحْصَيْنَـٰهُ فِىٓ إِمَامٍۢ مُّبِينٍۢ ﴿١٢﴾}\\
13.\  & \mytextarabic{وَٱضْرِبْ لَهُم مَّثَلًا أَصْحَـٰبَ ٱلْقَرْيَةِ إِذْ جَآءَهَا ٱلْمُرْسَلُونَ ﴿١٣﴾}\\
14.\  & \mytextarabic{إِذْ أَرْسَلْنَآ إِلَيْهِمُ ٱثْنَيْنِ فَكَذَّبُوهُمَا فَعَزَّزْنَا بِثَالِثٍۢ فَقَالُوٓا۟ إِنَّآ إِلَيْكُم مُّرْسَلُونَ ﴿١٤﴾}\\
15.\  & \mytextarabic{قَالُوا۟ مَآ أَنتُمْ إِلَّا بَشَرٌۭ مِّثْلُنَا وَمَآ أَنزَلَ ٱلرَّحْمَـٰنُ مِن شَىْءٍ إِنْ أَنتُمْ إِلَّا تَكْذِبُونَ ﴿١٥﴾}\\
16.\  & \mytextarabic{قَالُوا۟ رَبُّنَا يَعْلَمُ إِنَّآ إِلَيْكُمْ لَمُرْسَلُونَ ﴿١٦﴾}\\
17.\  & \mytextarabic{وَمَا عَلَيْنَآ إِلَّا ٱلْبَلَـٰغُ ٱلْمُبِينُ ﴿١٧﴾}\\
18.\  & \mytextarabic{قَالُوٓا۟ إِنَّا تَطَيَّرْنَا بِكُمْ ۖ لَئِن لَّمْ تَنتَهُوا۟ لَنَرْجُمَنَّكُمْ وَلَيَمَسَّنَّكُم مِّنَّا عَذَابٌ أَلِيمٌۭ ﴿١٨﴾}\\
19.\  & \mytextarabic{قَالُوا۟ طَٰٓئِرُكُم مَّعَكُمْ ۚ أَئِن ذُكِّرْتُم ۚ بَلْ أَنتُمْ قَوْمٌۭ مُّسْرِفُونَ ﴿١٩﴾}\\
20.\  & \mytextarabic{وَجَآءَ مِنْ أَقْصَا ٱلْمَدِينَةِ رَجُلٌۭ يَسْعَىٰ قَالَ يَـٰقَوْمِ ٱتَّبِعُوا۟ ٱلْمُرْسَلِينَ ﴿٢٠﴾}\\
21.\  & \mytextarabic{ٱتَّبِعُوا۟ مَن لَّا يَسْـَٔلُكُمْ أَجْرًۭا وَهُم مُّهْتَدُونَ ﴿٢١﴾}\\
22.\  & \mytextarabic{وَمَا لِىَ لَآ أَعْبُدُ ٱلَّذِى فَطَرَنِى وَإِلَيْهِ تُرْجَعُونَ ﴿٢٢﴾}\\
23.\  & \mytextarabic{ءَأَتَّخِذُ مِن دُونِهِۦٓ ءَالِهَةً إِن يُرِدْنِ ٱلرَّحْمَـٰنُ بِضُرٍّۢ لَّا تُغْنِ عَنِّى شَفَـٰعَتُهُمْ شَيْـًۭٔا وَلَا يُنقِذُونِ ﴿٢٣﴾}\\
24.\  & \mytextarabic{إِنِّىٓ إِذًۭا لَّفِى ضَلَـٰلٍۢ مُّبِينٍ ﴿٢٤﴾}\\
25.\  & \mytextarabic{إِنِّىٓ ءَامَنتُ بِرَبِّكُمْ فَٱسْمَعُونِ ﴿٢٥﴾}\\
26.\  & \mytextarabic{قِيلَ ٱدْخُلِ ٱلْجَنَّةَ ۖ قَالَ يَـٰلَيْتَ قَوْمِى يَعْلَمُونَ ﴿٢٦﴾}\\
27.\  & \mytextarabic{بِمَا غَفَرَ لِى رَبِّى وَجَعَلَنِى مِنَ ٱلْمُكْرَمِينَ ﴿٢٧﴾}\\
28.\  & \mytextarabic{۞ وَمَآ أَنزَلْنَا عَلَىٰ قَوْمِهِۦ مِنۢ بَعْدِهِۦ مِن جُندٍۢ مِّنَ ٱلسَّمَآءِ وَمَا كُنَّا مُنزِلِينَ ﴿٢٨﴾}\\
29.\  & \mytextarabic{إِن كَانَتْ إِلَّا صَيْحَةًۭ وَٟحِدَةًۭ فَإِذَا هُمْ خَـٰمِدُونَ ﴿٢٩﴾}\\
30.\  & \mytextarabic{يَـٰحَسْرَةً عَلَى ٱلْعِبَادِ ۚ مَا يَأْتِيهِم مِّن رَّسُولٍ إِلَّا كَانُوا۟ بِهِۦ يَسْتَهْزِءُونَ ﴿٣٠﴾}\\
31.\  & \mytextarabic{أَلَمْ يَرَوْا۟ كَمْ أَهْلَكْنَا قَبْلَهُم مِّنَ ٱلْقُرُونِ أَنَّهُمْ إِلَيْهِمْ لَا يَرْجِعُونَ ﴿٣١﴾}\\
32.\  & \mytextarabic{وَإِن كُلٌّۭ لَّمَّا جَمِيعٌۭ لَّدَيْنَا مُحْضَرُونَ ﴿٣٢﴾}\\
33.\  & \mytextarabic{وَءَايَةٌۭ لَّهُمُ ٱلْأَرْضُ ٱلْمَيْتَةُ أَحْيَيْنَـٰهَا وَأَخْرَجْنَا مِنْهَا حَبًّۭا فَمِنْهُ يَأْكُلُونَ ﴿٣٣﴾}\\
34.\  & \mytextarabic{وَجَعَلْنَا فِيهَا جَنَّـٰتٍۢ مِّن نَّخِيلٍۢ وَأَعْنَـٰبٍۢ وَفَجَّرْنَا فِيهَا مِنَ ٱلْعُيُونِ ﴿٣٤﴾}\\
35.\  & \mytextarabic{لِيَأْكُلُوا۟ مِن ثَمَرِهِۦ وَمَا عَمِلَتْهُ أَيْدِيهِمْ ۖ أَفَلَا يَشْكُرُونَ ﴿٣٥﴾}\\
36.\  & \mytextarabic{سُبْحَـٰنَ ٱلَّذِى خَلَقَ ٱلْأَزْوَٟجَ كُلَّهَا مِمَّا تُنۢبِتُ ٱلْأَرْضُ وَمِنْ أَنفُسِهِمْ وَمِمَّا لَا يَعْلَمُونَ ﴿٣٦﴾}\\
37.\  & \mytextarabic{وَءَايَةٌۭ لَّهُمُ ٱلَّيْلُ نَسْلَخُ مِنْهُ ٱلنَّهَارَ فَإِذَا هُم مُّظْلِمُونَ ﴿٣٧﴾}\\
38.\  & \mytextarabic{وَٱلشَّمْسُ تَجْرِى لِمُسْتَقَرٍّۢ لَّهَا ۚ ذَٟلِكَ تَقْدِيرُ ٱلْعَزِيزِ ٱلْعَلِيمِ ﴿٣٨﴾}\\
39.\  & \mytextarabic{وَٱلْقَمَرَ قَدَّرْنَـٰهُ مَنَازِلَ حَتَّىٰ عَادَ كَٱلْعُرْجُونِ ٱلْقَدِيمِ ﴿٣٩﴾}\\
40.\  & \mytextarabic{لَا ٱلشَّمْسُ يَنۢبَغِى لَهَآ أَن تُدْرِكَ ٱلْقَمَرَ وَلَا ٱلَّيْلُ سَابِقُ ٱلنَّهَارِ ۚ وَكُلٌّۭ فِى فَلَكٍۢ يَسْبَحُونَ ﴿٤٠﴾}\\
41.\  & \mytextarabic{وَءَايَةٌۭ لَّهُمْ أَنَّا حَمَلْنَا ذُرِّيَّتَهُمْ فِى ٱلْفُلْكِ ٱلْمَشْحُونِ ﴿٤١﴾}\\
42.\  & \mytextarabic{وَخَلَقْنَا لَهُم مِّن مِّثْلِهِۦ مَا يَرْكَبُونَ ﴿٤٢﴾}\\
43.\  & \mytextarabic{وَإِن نَّشَأْ نُغْرِقْهُمْ فَلَا صَرِيخَ لَهُمْ وَلَا هُمْ يُنقَذُونَ ﴿٤٣﴾}\\
44.\  & \mytextarabic{إِلَّا رَحْمَةًۭ مِّنَّا وَمَتَـٰعًا إِلَىٰ حِينٍۢ ﴿٤٤﴾}\\
45.\  & \mytextarabic{وَإِذَا قِيلَ لَهُمُ ٱتَّقُوا۟ مَا بَيْنَ أَيْدِيكُمْ وَمَا خَلْفَكُمْ لَعَلَّكُمْ تُرْحَمُونَ ﴿٤٥﴾}\\
46.\  & \mytextarabic{وَمَا تَأْتِيهِم مِّنْ ءَايَةٍۢ مِّنْ ءَايَـٰتِ رَبِّهِمْ إِلَّا كَانُوا۟ عَنْهَا مُعْرِضِينَ ﴿٤٦﴾}\\
47.\  & \mytextarabic{وَإِذَا قِيلَ لَهُمْ أَنفِقُوا۟ مِمَّا رَزَقَكُمُ ٱللَّهُ قَالَ ٱلَّذِينَ كَفَرُوا۟ لِلَّذِينَ ءَامَنُوٓا۟ أَنُطْعِمُ مَن لَّوْ يَشَآءُ ٱللَّهُ أَطْعَمَهُۥٓ إِنْ أَنتُمْ إِلَّا فِى ضَلَـٰلٍۢ مُّبِينٍۢ ﴿٤٧﴾}\\
48.\  & \mytextarabic{وَيَقُولُونَ مَتَىٰ هَـٰذَا ٱلْوَعْدُ إِن كُنتُمْ صَـٰدِقِينَ ﴿٤٨﴾}\\
49.\  & \mytextarabic{مَا يَنظُرُونَ إِلَّا صَيْحَةًۭ وَٟحِدَةًۭ تَأْخُذُهُمْ وَهُمْ يَخِصِّمُونَ ﴿٤٩﴾}\\
50.\  & \mytextarabic{فَلَا يَسْتَطِيعُونَ تَوْصِيَةًۭ وَلَآ إِلَىٰٓ أَهْلِهِمْ يَرْجِعُونَ ﴿٥٠﴾}\\
51.\  & \mytextarabic{وَنُفِخَ فِى ٱلصُّورِ فَإِذَا هُم مِّنَ ٱلْأَجْدَاثِ إِلَىٰ رَبِّهِمْ يَنسِلُونَ ﴿٥١﴾}\\
52.\  & \mytextarabic{قَالُوا۟ يَـٰوَيْلَنَا مَنۢ بَعَثَنَا مِن مَّرْقَدِنَا ۜ ۗ هَـٰذَا مَا وَعَدَ ٱلرَّحْمَـٰنُ وَصَدَقَ ٱلْمُرْسَلُونَ ﴿٥٢﴾}\\
53.\  & \mytextarabic{إِن كَانَتْ إِلَّا صَيْحَةًۭ وَٟحِدَةًۭ فَإِذَا هُمْ جَمِيعٌۭ لَّدَيْنَا مُحْضَرُونَ ﴿٥٣﴾}\\
54.\  & \mytextarabic{فَٱلْيَوْمَ لَا تُظْلَمُ نَفْسٌۭ شَيْـًۭٔا وَلَا تُجْزَوْنَ إِلَّا مَا كُنتُمْ تَعْمَلُونَ ﴿٥٤﴾}\\
55.\  & \mytextarabic{إِنَّ أَصْحَـٰبَ ٱلْجَنَّةِ ٱلْيَوْمَ فِى شُغُلٍۢ فَـٰكِهُونَ ﴿٥٥﴾}\\
56.\  & \mytextarabic{هُمْ وَأَزْوَٟجُهُمْ فِى ظِلَـٰلٍ عَلَى ٱلْأَرَآئِكِ مُتَّكِـُٔونَ ﴿٥٦﴾}\\
57.\  & \mytextarabic{لَهُمْ فِيهَا فَـٰكِهَةٌۭ وَلَهُم مَّا يَدَّعُونَ ﴿٥٧﴾}\\
58.\  & \mytextarabic{سَلَـٰمٌۭ قَوْلًۭا مِّن رَّبٍّۢ رَّحِيمٍۢ ﴿٥٨﴾}\\
59.\  & \mytextarabic{وَٱمْتَـٰزُوا۟ ٱلْيَوْمَ أَيُّهَا ٱلْمُجْرِمُونَ ﴿٥٩﴾}\\
60.\  & \mytextarabic{۞ أَلَمْ أَعْهَدْ إِلَيْكُمْ يَـٰبَنِىٓ ءَادَمَ أَن لَّا تَعْبُدُوا۟ ٱلشَّيْطَٰنَ ۖ إِنَّهُۥ لَكُمْ عَدُوٌّۭ مُّبِينٌۭ ﴿٦٠﴾}\\
61.\  & \mytextarabic{وَأَنِ ٱعْبُدُونِى ۚ هَـٰذَا صِرَٰطٌۭ مُّسْتَقِيمٌۭ ﴿٦١﴾}\\
62.\  & \mytextarabic{وَلَقَدْ أَضَلَّ مِنكُمْ جِبِلًّۭا كَثِيرًا ۖ أَفَلَمْ تَكُونُوا۟ تَعْقِلُونَ ﴿٦٢﴾}\\
63.\  & \mytextarabic{هَـٰذِهِۦ جَهَنَّمُ ٱلَّتِى كُنتُمْ تُوعَدُونَ ﴿٦٣﴾}\\
64.\  & \mytextarabic{ٱصْلَوْهَا ٱلْيَوْمَ بِمَا كُنتُمْ تَكْفُرُونَ ﴿٦٤﴾}\\
65.\  & \mytextarabic{ٱلْيَوْمَ نَخْتِمُ عَلَىٰٓ أَفْوَٟهِهِمْ وَتُكَلِّمُنَآ أَيْدِيهِمْ وَتَشْهَدُ أَرْجُلُهُم بِمَا كَانُوا۟ يَكْسِبُونَ ﴿٦٥﴾}\\
66.\  & \mytextarabic{وَلَوْ نَشَآءُ لَطَمَسْنَا عَلَىٰٓ أَعْيُنِهِمْ فَٱسْتَبَقُوا۟ ٱلصِّرَٰطَ فَأَنَّىٰ يُبْصِرُونَ ﴿٦٦﴾}\\
67.\  & \mytextarabic{وَلَوْ نَشَآءُ لَمَسَخْنَـٰهُمْ عَلَىٰ مَكَانَتِهِمْ فَمَا ٱسْتَطَٰعُوا۟ مُضِيًّۭا وَلَا يَرْجِعُونَ ﴿٦٧﴾}\\
68.\  & \mytextarabic{وَمَن نُّعَمِّرْهُ نُنَكِّسْهُ فِى ٱلْخَلْقِ ۖ أَفَلَا يَعْقِلُونَ ﴿٦٨﴾}\\
69.\  & \mytextarabic{وَمَا عَلَّمْنَـٰهُ ٱلشِّعْرَ وَمَا يَنۢبَغِى لَهُۥٓ ۚ إِنْ هُوَ إِلَّا ذِكْرٌۭ وَقُرْءَانٌۭ مُّبِينٌۭ ﴿٦٩﴾}\\
70.\  & \mytextarabic{لِّيُنذِرَ مَن كَانَ حَيًّۭا وَيَحِقَّ ٱلْقَوْلُ عَلَى ٱلْكَـٰفِرِينَ ﴿٧٠﴾}\\
71.\  & \mytextarabic{أَوَلَمْ يَرَوْا۟ أَنَّا خَلَقْنَا لَهُم مِّمَّا عَمِلَتْ أَيْدِينَآ أَنْعَـٰمًۭا فَهُمْ لَهَا مَـٰلِكُونَ ﴿٧١﴾}\\
72.\  & \mytextarabic{وَذَلَّلْنَـٰهَا لَهُمْ فَمِنْهَا رَكُوبُهُمْ وَمِنْهَا يَأْكُلُونَ ﴿٧٢﴾}\\
73.\  & \mytextarabic{وَلَهُمْ فِيهَا مَنَـٰفِعُ وَمَشَارِبُ ۖ أَفَلَا يَشْكُرُونَ ﴿٧٣﴾}\\
74.\  & \mytextarabic{وَٱتَّخَذُوا۟ مِن دُونِ ٱللَّهِ ءَالِهَةًۭ لَّعَلَّهُمْ يُنصَرُونَ ﴿٧٤﴾}\\
75.\  & \mytextarabic{لَا يَسْتَطِيعُونَ نَصْرَهُمْ وَهُمْ لَهُمْ جُندٌۭ مُّحْضَرُونَ ﴿٧٥﴾}\\
76.\  & \mytextarabic{فَلَا يَحْزُنكَ قَوْلُهُمْ ۘ إِنَّا نَعْلَمُ مَا يُسِرُّونَ وَمَا يُعْلِنُونَ ﴿٧٦﴾}\\
77.\  & \mytextarabic{أَوَلَمْ يَرَ ٱلْإِنسَـٰنُ أَنَّا خَلَقْنَـٰهُ مِن نُّطْفَةٍۢ فَإِذَا هُوَ خَصِيمٌۭ مُّبِينٌۭ ﴿٧٧﴾}\\
78.\  & \mytextarabic{وَضَرَبَ لَنَا مَثَلًۭا وَنَسِىَ خَلْقَهُۥ ۖ قَالَ مَن يُحْىِ ٱلْعِظَـٰمَ وَهِىَ رَمِيمٌۭ ﴿٧٨﴾}\\
79.\  & \mytextarabic{قُلْ يُحْيِيهَا ٱلَّذِىٓ أَنشَأَهَآ أَوَّلَ مَرَّةٍۢ ۖ وَهُوَ بِكُلِّ خَلْقٍ عَلِيمٌ ﴿٧٩﴾}\\
80.\  & \mytextarabic{ٱلَّذِى جَعَلَ لَكُم مِّنَ ٱلشَّجَرِ ٱلْأَخْضَرِ نَارًۭا فَإِذَآ أَنتُم مِّنْهُ تُوقِدُونَ ﴿٨٠﴾}\\
81.\  & \mytextarabic{أَوَلَيْسَ ٱلَّذِى خَلَقَ ٱلسَّمَـٰوَٟتِ وَٱلْأَرْضَ بِقَـٰدِرٍ عَلَىٰٓ أَن يَخْلُقَ مِثْلَهُم ۚ بَلَىٰ وَهُوَ ٱلْخَلَّٰقُ ٱلْعَلِيمُ ﴿٨١﴾}\\
82.\  & \mytextarabic{إِنَّمَآ أَمْرُهُۥٓ إِذَآ أَرَادَ شَيْـًٔا أَن يَقُولَ لَهُۥ كُن فَيَكُونُ ﴿٨٢﴾}\\
83.\  & \mytextarabic{فَسُبْحَـٰنَ ٱلَّذِى بِيَدِهِۦ مَلَكُوتُ كُلِّ شَىْءٍۢ وَإِلَيْهِ تُرْجَعُونَ ﴿٨٣﴾}\\
\end{longtable}
\clearpage