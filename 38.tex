\begin{center}\section{ሱራቱ ሷድ -  \textarabic{سوره  ص}}\end{center}
\begin{longtable}{%
  @{}
    p{.5\textwidth}
  @{~~~}
    p{.5\textwidth}
    @{}
}
ቢስሚላሂ አራህመኒ ራሂይም &  \mytextarabic{بِسْمِ ٱللَّهِ ٱلرَّحْمَـٰنِ ٱلرَّحِيمِ}\\
1.\  & \mytextarabic{ صٓ ۚ وَٱلْقُرْءَانِ ذِى ٱلذِّكْرِ ﴿١﴾}\\
2.\  & \mytextarabic{بَلِ ٱلَّذِينَ كَفَرُوا۟ فِى عِزَّةٍۢ وَشِقَاقٍۢ ﴿٢﴾}\\
3.\  & \mytextarabic{كَمْ أَهْلَكْنَا مِن قَبْلِهِم مِّن قَرْنٍۢ فَنَادَوا۟ وَّلَاتَ حِينَ مَنَاصٍۢ ﴿٣﴾}\\
4.\  & \mytextarabic{وَعَجِبُوٓا۟ أَن جَآءَهُم مُّنذِرٌۭ مِّنْهُمْ ۖ وَقَالَ ٱلْكَـٰفِرُونَ هَـٰذَا سَـٰحِرٌۭ كَذَّابٌ ﴿٤﴾}\\
5.\  & \mytextarabic{أَجَعَلَ ٱلْءَالِهَةَ إِلَـٰهًۭا وَٟحِدًا ۖ إِنَّ هَـٰذَا لَشَىْءٌ عُجَابٌۭ ﴿٥﴾}\\
6.\  & \mytextarabic{وَٱنطَلَقَ ٱلْمَلَأُ مِنْهُمْ أَنِ ٱمْشُوا۟ وَٱصْبِرُوا۟ عَلَىٰٓ ءَالِهَتِكُمْ ۖ إِنَّ هَـٰذَا لَشَىْءٌۭ يُرَادُ ﴿٦﴾}\\
7.\  & \mytextarabic{مَا سَمِعْنَا بِهَـٰذَا فِى ٱلْمِلَّةِ ٱلْءَاخِرَةِ إِنْ هَـٰذَآ إِلَّا ٱخْتِلَـٰقٌ ﴿٧﴾}\\
8.\  & \mytextarabic{أَءُنزِلَ عَلَيْهِ ٱلذِّكْرُ مِنۢ بَيْنِنَا ۚ بَلْ هُمْ فِى شَكٍّۢ مِّن ذِكْرِى ۖ بَل لَّمَّا يَذُوقُوا۟ عَذَابِ ﴿٨﴾}\\
9.\  & \mytextarabic{أَمْ عِندَهُمْ خَزَآئِنُ رَحْمَةِ رَبِّكَ ٱلْعَزِيزِ ٱلْوَهَّابِ ﴿٩﴾}\\
10.\  & \mytextarabic{أَمْ لَهُم مُّلْكُ ٱلسَّمَـٰوَٟتِ وَٱلْأَرْضِ وَمَا بَيْنَهُمَا ۖ فَلْيَرْتَقُوا۟ فِى ٱلْأَسْبَٰبِ ﴿١٠﴾}\\
11.\  & \mytextarabic{جُندٌۭ مَّا هُنَالِكَ مَهْزُومٌۭ مِّنَ ٱلْأَحْزَابِ ﴿١١﴾}\\
12.\  & \mytextarabic{كَذَّبَتْ قَبْلَهُمْ قَوْمُ نُوحٍۢ وَعَادٌۭ وَفِرْعَوْنُ ذُو ٱلْأَوْتَادِ ﴿١٢﴾}\\
13.\  & \mytextarabic{وَثَمُودُ وَقَوْمُ لُوطٍۢ وَأَصْحَـٰبُ لْـَٔيْكَةِ ۚ أُو۟لَـٰٓئِكَ ٱلْأَحْزَابُ ﴿١٣﴾}\\
14.\  & \mytextarabic{إِن كُلٌّ إِلَّا كَذَّبَ ٱلرُّسُلَ فَحَقَّ عِقَابِ ﴿١٤﴾}\\
15.\  & \mytextarabic{وَمَا يَنظُرُ هَـٰٓؤُلَآءِ إِلَّا صَيْحَةًۭ وَٟحِدَةًۭ مَّا لَهَا مِن فَوَاقٍۢ ﴿١٥﴾}\\
16.\  & \mytextarabic{وَقَالُوا۟ رَبَّنَا عَجِّل لَّنَا قِطَّنَا قَبْلَ يَوْمِ ٱلْحِسَابِ ﴿١٦﴾}\\
17.\  & \mytextarabic{ٱصْبِرْ عَلَىٰ مَا يَقُولُونَ وَٱذْكُرْ عَبْدَنَا دَاوُۥدَ ذَا ٱلْأَيْدِ ۖ إِنَّهُۥٓ أَوَّابٌ ﴿١٧﴾}\\
18.\  & \mytextarabic{إِنَّا سَخَّرْنَا ٱلْجِبَالَ مَعَهُۥ يُسَبِّحْنَ بِٱلْعَشِىِّ وَٱلْإِشْرَاقِ ﴿١٨﴾}\\
19.\  & \mytextarabic{وَٱلطَّيْرَ مَحْشُورَةًۭ ۖ كُلٌّۭ لَّهُۥٓ أَوَّابٌۭ ﴿١٩﴾}\\
20.\  & \mytextarabic{وَشَدَدْنَا مُلْكَهُۥ وَءَاتَيْنَـٰهُ ٱلْحِكْمَةَ وَفَصْلَ ٱلْخِطَابِ ﴿٢٠﴾}\\
21.\  & \mytextarabic{۞ وَهَلْ أَتَىٰكَ نَبَؤُا۟ ٱلْخَصْمِ إِذْ تَسَوَّرُوا۟ ٱلْمِحْرَابَ ﴿٢١﴾}\\
22.\  & \mytextarabic{إِذْ دَخَلُوا۟ عَلَىٰ دَاوُۥدَ فَفَزِعَ مِنْهُمْ ۖ قَالُوا۟ لَا تَخَفْ ۖ خَصْمَانِ بَغَىٰ بَعْضُنَا عَلَىٰ بَعْضٍۢ فَٱحْكُم بَيْنَنَا بِٱلْحَقِّ وَلَا تُشْطِطْ وَٱهْدِنَآ إِلَىٰ سَوَآءِ ٱلصِّرَٰطِ ﴿٢٢﴾}\\
23.\  & \mytextarabic{إِنَّ هَـٰذَآ أَخِى لَهُۥ تِسْعٌۭ وَتِسْعُونَ نَعْجَةًۭ وَلِىَ نَعْجَةٌۭ وَٟحِدَةٌۭ فَقَالَ أَكْفِلْنِيهَا وَعَزَّنِى فِى ٱلْخِطَابِ ﴿٢٣﴾}\\
24.\  & \mytextarabic{قَالَ لَقَدْ ظَلَمَكَ بِسُؤَالِ نَعْجَتِكَ إِلَىٰ نِعَاجِهِۦ ۖ وَإِنَّ كَثِيرًۭا مِّنَ ٱلْخُلَطَآءِ لَيَبْغِى بَعْضُهُمْ عَلَىٰ بَعْضٍ إِلَّا ٱلَّذِينَ ءَامَنُوا۟ وَعَمِلُوا۟ ٱلصَّـٰلِحَـٰتِ وَقَلِيلٌۭ مَّا هُمْ ۗ وَظَنَّ دَاوُۥدُ أَنَّمَا فَتَنَّـٰهُ فَٱسْتَغْفَرَ رَبَّهُۥ وَخَرَّ رَاكِعًۭا وَأَنَابَ ۩ ﴿٢٤﴾}\\
25.\  & \mytextarabic{فَغَفَرْنَا لَهُۥ ذَٟلِكَ ۖ وَإِنَّ لَهُۥ عِندَنَا لَزُلْفَىٰ وَحُسْنَ مَـَٔابٍۢ ﴿٢٥﴾}\\
26.\  & \mytextarabic{يَـٰدَاوُۥدُ إِنَّا جَعَلْنَـٰكَ خَلِيفَةًۭ فِى ٱلْأَرْضِ فَٱحْكُم بَيْنَ ٱلنَّاسِ بِٱلْحَقِّ وَلَا تَتَّبِعِ ٱلْهَوَىٰ فَيُضِلَّكَ عَن سَبِيلِ ٱللَّهِ ۚ إِنَّ ٱلَّذِينَ يَضِلُّونَ عَن سَبِيلِ ٱللَّهِ لَهُمْ عَذَابٌۭ شَدِيدٌۢ بِمَا نَسُوا۟ يَوْمَ ٱلْحِسَابِ ﴿٢٦﴾}\\
27.\  & \mytextarabic{وَمَا خَلَقْنَا ٱلسَّمَآءَ وَٱلْأَرْضَ وَمَا بَيْنَهُمَا بَٰطِلًۭا ۚ ذَٟلِكَ ظَنُّ ٱلَّذِينَ كَفَرُوا۟ ۚ فَوَيْلٌۭ لِّلَّذِينَ كَفَرُوا۟ مِنَ ٱلنَّارِ ﴿٢٧﴾}\\
28.\  & \mytextarabic{أَمْ نَجْعَلُ ٱلَّذِينَ ءَامَنُوا۟ وَعَمِلُوا۟ ٱلصَّـٰلِحَـٰتِ كَٱلْمُفْسِدِينَ فِى ٱلْأَرْضِ أَمْ نَجْعَلُ ٱلْمُتَّقِينَ كَٱلْفُجَّارِ ﴿٢٨﴾}\\
29.\  & \mytextarabic{كِتَـٰبٌ أَنزَلْنَـٰهُ إِلَيْكَ مُبَٰرَكٌۭ لِّيَدَّبَّرُوٓا۟ ءَايَـٰتِهِۦ وَلِيَتَذَكَّرَ أُو۟لُوا۟ ٱلْأَلْبَٰبِ ﴿٢٩﴾}\\
30.\  & \mytextarabic{وَوَهَبْنَا لِدَاوُۥدَ سُلَيْمَـٰنَ ۚ نِعْمَ ٱلْعَبْدُ ۖ إِنَّهُۥٓ أَوَّابٌ ﴿٣٠﴾}\\
31.\  & \mytextarabic{إِذْ عُرِضَ عَلَيْهِ بِٱلْعَشِىِّ ٱلصَّـٰفِنَـٰتُ ٱلْجِيَادُ ﴿٣١﴾}\\
32.\  & \mytextarabic{فَقَالَ إِنِّىٓ أَحْبَبْتُ حُبَّ ٱلْخَيْرِ عَن ذِكْرِ رَبِّى حَتَّىٰ تَوَارَتْ بِٱلْحِجَابِ ﴿٣٢﴾}\\
33.\  & \mytextarabic{رُدُّوهَا عَلَىَّ ۖ فَطَفِقَ مَسْحًۢا بِٱلسُّوقِ وَٱلْأَعْنَاقِ ﴿٣٣﴾}\\
34.\  & \mytextarabic{وَلَقَدْ فَتَنَّا سُلَيْمَـٰنَ وَأَلْقَيْنَا عَلَىٰ كُرْسِيِّهِۦ جَسَدًۭا ثُمَّ أَنَابَ ﴿٣٤﴾}\\
35.\  & \mytextarabic{قَالَ رَبِّ ٱغْفِرْ لِى وَهَبْ لِى مُلْكًۭا لَّا يَنۢبَغِى لِأَحَدٍۢ مِّنۢ بَعْدِىٓ ۖ إِنَّكَ أَنتَ ٱلْوَهَّابُ ﴿٣٥﴾}\\
36.\  & \mytextarabic{فَسَخَّرْنَا لَهُ ٱلرِّيحَ تَجْرِى بِأَمْرِهِۦ رُخَآءً حَيْثُ أَصَابَ ﴿٣٦﴾}\\
37.\  & \mytextarabic{وَٱلشَّيَـٰطِينَ كُلَّ بَنَّآءٍۢ وَغَوَّاصٍۢ ﴿٣٧﴾}\\
38.\  & \mytextarabic{وَءَاخَرِينَ مُقَرَّنِينَ فِى ٱلْأَصْفَادِ ﴿٣٨﴾}\\
39.\  & \mytextarabic{هَـٰذَا عَطَآؤُنَا فَٱمْنُنْ أَوْ أَمْسِكْ بِغَيْرِ حِسَابٍۢ ﴿٣٩﴾}\\
40.\  & \mytextarabic{وَإِنَّ لَهُۥ عِندَنَا لَزُلْفَىٰ وَحُسْنَ مَـَٔابٍۢ ﴿٤٠﴾}\\
41.\  & \mytextarabic{وَٱذْكُرْ عَبْدَنَآ أَيُّوبَ إِذْ نَادَىٰ رَبَّهُۥٓ أَنِّى مَسَّنِىَ ٱلشَّيْطَٰنُ بِنُصْبٍۢ وَعَذَابٍ ﴿٤١﴾}\\
42.\  & \mytextarabic{ٱرْكُضْ بِرِجْلِكَ ۖ هَـٰذَا مُغْتَسَلٌۢ بَارِدٌۭ وَشَرَابٌۭ ﴿٤٢﴾}\\
43.\  & \mytextarabic{وَوَهَبْنَا لَهُۥٓ أَهْلَهُۥ وَمِثْلَهُم مَّعَهُمْ رَحْمَةًۭ مِّنَّا وَذِكْرَىٰ لِأُو۟لِى ٱلْأَلْبَٰبِ ﴿٤٣﴾}\\
44.\  & \mytextarabic{وَخُذْ بِيَدِكَ ضِغْثًۭا فَٱضْرِب بِّهِۦ وَلَا تَحْنَثْ ۗ إِنَّا وَجَدْنَـٰهُ صَابِرًۭا ۚ نِّعْمَ ٱلْعَبْدُ ۖ إِنَّهُۥٓ أَوَّابٌۭ ﴿٤٤﴾}\\
45.\  & \mytextarabic{وَٱذْكُرْ عِبَٰدَنَآ إِبْرَٰهِيمَ وَإِسْحَـٰقَ وَيَعْقُوبَ أُو۟لِى ٱلْأَيْدِى وَٱلْأَبْصَـٰرِ ﴿٤٥﴾}\\
46.\  & \mytextarabic{إِنَّآ أَخْلَصْنَـٰهُم بِخَالِصَةٍۢ ذِكْرَى ٱلدَّارِ ﴿٤٦﴾}\\
47.\  & \mytextarabic{وَإِنَّهُمْ عِندَنَا لَمِنَ ٱلْمُصْطَفَيْنَ ٱلْأَخْيَارِ ﴿٤٧﴾}\\
48.\  & \mytextarabic{وَٱذْكُرْ إِسْمَـٰعِيلَ وَٱلْيَسَعَ وَذَا ٱلْكِفْلِ ۖ وَكُلٌّۭ مِّنَ ٱلْأَخْيَارِ ﴿٤٨﴾}\\
49.\  & \mytextarabic{هَـٰذَا ذِكْرٌۭ ۚ وَإِنَّ لِلْمُتَّقِينَ لَحُسْنَ مَـَٔابٍۢ ﴿٤٩﴾}\\
50.\  & \mytextarabic{جَنَّـٰتِ عَدْنٍۢ مُّفَتَّحَةًۭ لَّهُمُ ٱلْأَبْوَٟبُ ﴿٥٠﴾}\\
51.\  & \mytextarabic{مُتَّكِـِٔينَ فِيهَا يَدْعُونَ فِيهَا بِفَـٰكِهَةٍۢ كَثِيرَةٍۢ وَشَرَابٍۢ ﴿٥١﴾}\\
52.\  & \mytextarabic{۞ وَعِندَهُمْ قَـٰصِرَٰتُ ٱلطَّرْفِ أَتْرَابٌ ﴿٥٢﴾}\\
53.\  & \mytextarabic{هَـٰذَا مَا تُوعَدُونَ لِيَوْمِ ٱلْحِسَابِ ﴿٥٣﴾}\\
54.\  & \mytextarabic{إِنَّ هَـٰذَا لَرِزْقُنَا مَا لَهُۥ مِن نَّفَادٍ ﴿٥٤﴾}\\
55.\  & \mytextarabic{هَـٰذَا ۚ وَإِنَّ لِلطَّٰغِينَ لَشَرَّ مَـَٔابٍۢ ﴿٥٥﴾}\\
56.\  & \mytextarabic{جَهَنَّمَ يَصْلَوْنَهَا فَبِئْسَ ٱلْمِهَادُ ﴿٥٦﴾}\\
57.\  & \mytextarabic{هَـٰذَا فَلْيَذُوقُوهُ حَمِيمٌۭ وَغَسَّاقٌۭ ﴿٥٧﴾}\\
58.\  & \mytextarabic{وَءَاخَرُ مِن شَكْلِهِۦٓ أَزْوَٟجٌ ﴿٥٨﴾}\\
59.\  & \mytextarabic{هَـٰذَا فَوْجٌۭ مُّقْتَحِمٌۭ مَّعَكُمْ ۖ لَا مَرْحَبًۢا بِهِمْ ۚ إِنَّهُمْ صَالُوا۟ ٱلنَّارِ ﴿٥٩﴾}\\
60.\  & \mytextarabic{قَالُوا۟ بَلْ أَنتُمْ لَا مَرْحَبًۢا بِكُمْ ۖ أَنتُمْ قَدَّمْتُمُوهُ لَنَا ۖ فَبِئْسَ ٱلْقَرَارُ ﴿٦٠﴾}\\
61.\  & \mytextarabic{قَالُوا۟ رَبَّنَا مَن قَدَّمَ لَنَا هَـٰذَا فَزِدْهُ عَذَابًۭا ضِعْفًۭا فِى ٱلنَّارِ ﴿٦١﴾}\\
62.\  & \mytextarabic{وَقَالُوا۟ مَا لَنَا لَا نَرَىٰ رِجَالًۭا كُنَّا نَعُدُّهُم مِّنَ ٱلْأَشْرَارِ ﴿٦٢﴾}\\
63.\  & \mytextarabic{أَتَّخَذْنَـٰهُمْ سِخْرِيًّا أَمْ زَاغَتْ عَنْهُمُ ٱلْأَبْصَـٰرُ ﴿٦٣﴾}\\
64.\  & \mytextarabic{إِنَّ ذَٟلِكَ لَحَقٌّۭ تَخَاصُمُ أَهْلِ ٱلنَّارِ ﴿٦٤﴾}\\
65.\  & \mytextarabic{قُلْ إِنَّمَآ أَنَا۠ مُنذِرٌۭ ۖ وَمَا مِنْ إِلَـٰهٍ إِلَّا ٱللَّهُ ٱلْوَٟحِدُ ٱلْقَهَّارُ ﴿٦٥﴾}\\
66.\  & \mytextarabic{رَبُّ ٱلسَّمَـٰوَٟتِ وَٱلْأَرْضِ وَمَا بَيْنَهُمَا ٱلْعَزِيزُ ٱلْغَفَّٰرُ ﴿٦٦﴾}\\
67.\  & \mytextarabic{قُلْ هُوَ نَبَؤٌا۟ عَظِيمٌ ﴿٦٧﴾}\\
68.\  & \mytextarabic{أَنتُمْ عَنْهُ مُعْرِضُونَ ﴿٦٨﴾}\\
69.\  & \mytextarabic{مَا كَانَ لِىَ مِنْ عِلْمٍۭ بِٱلْمَلَإِ ٱلْأَعْلَىٰٓ إِذْ يَخْتَصِمُونَ ﴿٦٩﴾}\\
70.\  & \mytextarabic{إِن يُوحَىٰٓ إِلَىَّ إِلَّآ أَنَّمَآ أَنَا۠ نَذِيرٌۭ مُّبِينٌ ﴿٧٠﴾}\\
71.\  & \mytextarabic{إِذْ قَالَ رَبُّكَ لِلْمَلَـٰٓئِكَةِ إِنِّى خَـٰلِقٌۢ بَشَرًۭا مِّن طِينٍۢ ﴿٧١﴾}\\
72.\  & \mytextarabic{فَإِذَا سَوَّيْتُهُۥ وَنَفَخْتُ فِيهِ مِن رُّوحِى فَقَعُوا۟ لَهُۥ سَـٰجِدِينَ ﴿٧٢﴾}\\
73.\  & \mytextarabic{فَسَجَدَ ٱلْمَلَـٰٓئِكَةُ كُلُّهُمْ أَجْمَعُونَ ﴿٧٣﴾}\\
74.\  & \mytextarabic{إِلَّآ إِبْلِيسَ ٱسْتَكْبَرَ وَكَانَ مِنَ ٱلْكَـٰفِرِينَ ﴿٧٤﴾}\\
75.\  & \mytextarabic{قَالَ يَـٰٓإِبْلِيسُ مَا مَنَعَكَ أَن تَسْجُدَ لِمَا خَلَقْتُ بِيَدَىَّ ۖ أَسْتَكْبَرْتَ أَمْ كُنتَ مِنَ ٱلْعَالِينَ ﴿٧٥﴾}\\
76.\  & \mytextarabic{قَالَ أَنَا۠ خَيْرٌۭ مِّنْهُ ۖ خَلَقْتَنِى مِن نَّارٍۢ وَخَلَقْتَهُۥ مِن طِينٍۢ ﴿٧٦﴾}\\
77.\  & \mytextarabic{قَالَ فَٱخْرُجْ مِنْهَا فَإِنَّكَ رَجِيمٌۭ ﴿٧٧﴾}\\
78.\  & \mytextarabic{وَإِنَّ عَلَيْكَ لَعْنَتِىٓ إِلَىٰ يَوْمِ ٱلدِّينِ ﴿٧٨﴾}\\
79.\  & \mytextarabic{قَالَ رَبِّ فَأَنظِرْنِىٓ إِلَىٰ يَوْمِ يُبْعَثُونَ ﴿٧٩﴾}\\
80.\  & \mytextarabic{قَالَ فَإِنَّكَ مِنَ ٱلْمُنظَرِينَ ﴿٨٠﴾}\\
81.\  & \mytextarabic{إِلَىٰ يَوْمِ ٱلْوَقْتِ ٱلْمَعْلُومِ ﴿٨١﴾}\\
82.\  & \mytextarabic{قَالَ فَبِعِزَّتِكَ لَأُغْوِيَنَّهُمْ أَجْمَعِينَ ﴿٨٢﴾}\\
83.\  & \mytextarabic{إِلَّا عِبَادَكَ مِنْهُمُ ٱلْمُخْلَصِينَ ﴿٨٣﴾}\\
84.\  & \mytextarabic{قَالَ فَٱلْحَقُّ وَٱلْحَقَّ أَقُولُ ﴿٨٤﴾}\\
85.\  & \mytextarabic{لَأَمْلَأَنَّ جَهَنَّمَ مِنكَ وَمِمَّن تَبِعَكَ مِنْهُمْ أَجْمَعِينَ ﴿٨٥﴾}\\
86.\  & \mytextarabic{قُلْ مَآ أَسْـَٔلُكُمْ عَلَيْهِ مِنْ أَجْرٍۢ وَمَآ أَنَا۠ مِنَ ٱلْمُتَكَلِّفِينَ ﴿٨٦﴾}\\
87.\  & \mytextarabic{إِنْ هُوَ إِلَّا ذِكْرٌۭ لِّلْعَـٰلَمِينَ ﴿٨٧﴾}\\
88.\  & \mytextarabic{وَلَتَعْلَمُنَّ نَبَأَهُۥ بَعْدَ حِينٍۭ ﴿٨٨﴾}\\
\end{longtable}
\clearpage