\begin{center}\section{ሱራቱ አልዋቂያ -  \textarabic{سوره  الواقعة}}\end{center}
\begin{longtable}{%
  @{}
    p{.5\textwidth}
  @{~~~}
    p{.5\textwidth}
    @{}
}
ቢስሚላሂ አራህመኒ ራሂይም &  \mytextarabic{بِسْمِ ٱللَّهِ ٱلرَّحْمَـٰنِ ٱلرَّحِيمِ}\\
1.\  & \mytextarabic{ إِذَا وَقَعَتِ ٱلْوَاقِعَةُ ﴿١﴾}\\
2.\  & \mytextarabic{لَيْسَ لِوَقْعَتِهَا كَاذِبَةٌ ﴿٢﴾}\\
3.\  & \mytextarabic{خَافِضَةٌۭ رَّافِعَةٌ ﴿٣﴾}\\
4.\  & \mytextarabic{إِذَا رُجَّتِ ٱلْأَرْضُ رَجًّۭا ﴿٤﴾}\\
5.\  & \mytextarabic{وَبُسَّتِ ٱلْجِبَالُ بَسًّۭا ﴿٥﴾}\\
6.\  & \mytextarabic{فَكَانَتْ هَبَآءًۭ مُّنۢبَثًّۭا ﴿٦﴾}\\
7.\  & \mytextarabic{وَكُنتُمْ أَزْوَٟجًۭا ثَلَـٰثَةًۭ ﴿٧﴾}\\
8.\  & \mytextarabic{فَأَصْحَـٰبُ ٱلْمَيْمَنَةِ مَآ أَصْحَـٰبُ ٱلْمَيْمَنَةِ ﴿٨﴾}\\
9.\  & \mytextarabic{وَأَصْحَـٰبُ ٱلْمَشْـَٔمَةِ مَآ أَصْحَـٰبُ ٱلْمَشْـَٔمَةِ ﴿٩﴾}\\
10.\  & \mytextarabic{وَٱلسَّٰبِقُونَ ٱلسَّٰبِقُونَ ﴿١٠﴾}\\
11.\  & \mytextarabic{أُو۟لَـٰٓئِكَ ٱلْمُقَرَّبُونَ ﴿١١﴾}\\
12.\  & \mytextarabic{فِى جَنَّـٰتِ ٱلنَّعِيمِ ﴿١٢﴾}\\
13.\  & \mytextarabic{ثُلَّةٌۭ مِّنَ ٱلْأَوَّلِينَ ﴿١٣﴾}\\
14.\  & \mytextarabic{وَقَلِيلٌۭ مِّنَ ٱلْءَاخِرِينَ ﴿١٤﴾}\\
15.\  & \mytextarabic{عَلَىٰ سُرُرٍۢ مَّوْضُونَةٍۢ ﴿١٥﴾}\\
16.\  & \mytextarabic{مُّتَّكِـِٔينَ عَلَيْهَا مُتَقَـٰبِلِينَ ﴿١٦﴾}\\
17.\  & \mytextarabic{يَطُوفُ عَلَيْهِمْ وِلْدَٟنٌۭ مُّخَلَّدُونَ ﴿١٧﴾}\\
18.\  & \mytextarabic{بِأَكْوَابٍۢ وَأَبَارِيقَ وَكَأْسٍۢ مِّن مَّعِينٍۢ ﴿١٨﴾}\\
19.\  & \mytextarabic{لَّا يُصَدَّعُونَ عَنْهَا وَلَا يُنزِفُونَ ﴿١٩﴾}\\
20.\  & \mytextarabic{وَفَـٰكِهَةٍۢ مِّمَّا يَتَخَيَّرُونَ ﴿٢٠﴾}\\
21.\  & \mytextarabic{وَلَحْمِ طَيْرٍۢ مِّمَّا يَشْتَهُونَ ﴿٢١﴾}\\
22.\  & \mytextarabic{وَحُورٌ عِينٌۭ ﴿٢٢﴾}\\
23.\  & \mytextarabic{كَأَمْثَـٰلِ ٱللُّؤْلُؤِ ٱلْمَكْنُونِ ﴿٢٣﴾}\\
24.\  & \mytextarabic{جَزَآءًۢ بِمَا كَانُوا۟ يَعْمَلُونَ ﴿٢٤﴾}\\
25.\  & \mytextarabic{لَا يَسْمَعُونَ فِيهَا لَغْوًۭا وَلَا تَأْثِيمًا ﴿٢٥﴾}\\
26.\  & \mytextarabic{إِلَّا قِيلًۭا سَلَـٰمًۭا سَلَـٰمًۭا ﴿٢٦﴾}\\
27.\  & \mytextarabic{وَأَصْحَـٰبُ ٱلْيَمِينِ مَآ أَصْحَـٰبُ ٱلْيَمِينِ ﴿٢٧﴾}\\
28.\  & \mytextarabic{فِى سِدْرٍۢ مَّخْضُودٍۢ ﴿٢٨﴾}\\
29.\  & \mytextarabic{وَطَلْحٍۢ مَّنضُودٍۢ ﴿٢٩﴾}\\
30.\  & \mytextarabic{وَظِلٍّۢ مَّمْدُودٍۢ ﴿٣٠﴾}\\
31.\  & \mytextarabic{وَمَآءٍۢ مَّسْكُوبٍۢ ﴿٣١﴾}\\
32.\  & \mytextarabic{وَفَـٰكِهَةٍۢ كَثِيرَةٍۢ ﴿٣٢﴾}\\
33.\  & \mytextarabic{لَّا مَقْطُوعَةٍۢ وَلَا مَمْنُوعَةٍۢ ﴿٣٣﴾}\\
34.\  & \mytextarabic{وَفُرُشٍۢ مَّرْفُوعَةٍ ﴿٣٤﴾}\\
35.\  & \mytextarabic{إِنَّآ أَنشَأْنَـٰهُنَّ إِنشَآءًۭ ﴿٣٥﴾}\\
36.\  & \mytextarabic{فَجَعَلْنَـٰهُنَّ أَبْكَارًا ﴿٣٦﴾}\\
37.\  & \mytextarabic{عُرُبًا أَتْرَابًۭا ﴿٣٧﴾}\\
38.\  & \mytextarabic{لِّأَصْحَـٰبِ ٱلْيَمِينِ ﴿٣٨﴾}\\
39.\  & \mytextarabic{ثُلَّةٌۭ مِّنَ ٱلْأَوَّلِينَ ﴿٣٩﴾}\\
40.\  & \mytextarabic{وَثُلَّةٌۭ مِّنَ ٱلْءَاخِرِينَ ﴿٤٠﴾}\\
41.\  & \mytextarabic{وَأَصْحَـٰبُ ٱلشِّمَالِ مَآ أَصْحَـٰبُ ٱلشِّمَالِ ﴿٤١﴾}\\
42.\  & \mytextarabic{فِى سَمُومٍۢ وَحَمِيمٍۢ ﴿٤٢﴾}\\
43.\  & \mytextarabic{وَظِلٍّۢ مِّن يَحْمُومٍۢ ﴿٤٣﴾}\\
44.\  & \mytextarabic{لَّا بَارِدٍۢ وَلَا كَرِيمٍ ﴿٤٤﴾}\\
45.\  & \mytextarabic{إِنَّهُمْ كَانُوا۟ قَبْلَ ذَٟلِكَ مُتْرَفِينَ ﴿٤٥﴾}\\
46.\  & \mytextarabic{وَكَانُوا۟ يُصِرُّونَ عَلَى ٱلْحِنثِ ٱلْعَظِيمِ ﴿٤٦﴾}\\
47.\  & \mytextarabic{وَكَانُوا۟ يَقُولُونَ أَئِذَا مِتْنَا وَكُنَّا تُرَابًۭا وَعِظَـٰمًا أَءِنَّا لَمَبْعُوثُونَ ﴿٤٧﴾}\\
48.\  & \mytextarabic{أَوَءَابَآؤُنَا ٱلْأَوَّلُونَ ﴿٤٨﴾}\\
49.\  & \mytextarabic{قُلْ إِنَّ ٱلْأَوَّلِينَ وَٱلْءَاخِرِينَ ﴿٤٩﴾}\\
50.\  & \mytextarabic{لَمَجْمُوعُونَ إِلَىٰ مِيقَـٰتِ يَوْمٍۢ مَّعْلُومٍۢ ﴿٥٠﴾}\\
51.\  & \mytextarabic{ثُمَّ إِنَّكُمْ أَيُّهَا ٱلضَّآلُّونَ ٱلْمُكَذِّبُونَ ﴿٥١﴾}\\
52.\  & \mytextarabic{لَءَاكِلُونَ مِن شَجَرٍۢ مِّن زَقُّومٍۢ ﴿٥٢﴾}\\
53.\  & \mytextarabic{فَمَالِـُٔونَ مِنْهَا ٱلْبُطُونَ ﴿٥٣﴾}\\
54.\  & \mytextarabic{فَشَـٰرِبُونَ عَلَيْهِ مِنَ ٱلْحَمِيمِ ﴿٥٤﴾}\\
55.\  & \mytextarabic{فَشَـٰرِبُونَ شُرْبَ ٱلْهِيمِ ﴿٥٥﴾}\\
56.\  & \mytextarabic{هَـٰذَا نُزُلُهُمْ يَوْمَ ٱلدِّينِ ﴿٥٦﴾}\\
57.\  & \mytextarabic{نَحْنُ خَلَقْنَـٰكُمْ فَلَوْلَا تُصَدِّقُونَ ﴿٥٧﴾}\\
58.\  & \mytextarabic{أَفَرَءَيْتُم مَّا تُمْنُونَ ﴿٥٨﴾}\\
59.\  & \mytextarabic{ءَأَنتُمْ تَخْلُقُونَهُۥٓ أَمْ نَحْنُ ٱلْخَـٰلِقُونَ ﴿٥٩﴾}\\
60.\  & \mytextarabic{نَحْنُ قَدَّرْنَا بَيْنَكُمُ ٱلْمَوْتَ وَمَا نَحْنُ بِمَسْبُوقِينَ ﴿٦٠﴾}\\
61.\  & \mytextarabic{عَلَىٰٓ أَن نُّبَدِّلَ أَمْثَـٰلَكُمْ وَنُنشِئَكُمْ فِى مَا لَا تَعْلَمُونَ ﴿٦١﴾}\\
62.\  & \mytextarabic{وَلَقَدْ عَلِمْتُمُ ٱلنَّشْأَةَ ٱلْأُولَىٰ فَلَوْلَا تَذَكَّرُونَ ﴿٦٢﴾}\\
63.\  & \mytextarabic{أَفَرَءَيْتُم مَّا تَحْرُثُونَ ﴿٦٣﴾}\\
64.\  & \mytextarabic{ءَأَنتُمْ تَزْرَعُونَهُۥٓ أَمْ نَحْنُ ٱلزَّٰرِعُونَ ﴿٦٤﴾}\\
65.\  & \mytextarabic{لَوْ نَشَآءُ لَجَعَلْنَـٰهُ حُطَٰمًۭا فَظَلْتُمْ تَفَكَّهُونَ ﴿٦٥﴾}\\
66.\  & \mytextarabic{إِنَّا لَمُغْرَمُونَ ﴿٦٦﴾}\\
67.\  & \mytextarabic{بَلْ نَحْنُ مَحْرُومُونَ ﴿٦٧﴾}\\
68.\  & \mytextarabic{أَفَرَءَيْتُمُ ٱلْمَآءَ ٱلَّذِى تَشْرَبُونَ ﴿٦٨﴾}\\
69.\  & \mytextarabic{ءَأَنتُمْ أَنزَلْتُمُوهُ مِنَ ٱلْمُزْنِ أَمْ نَحْنُ ٱلْمُنزِلُونَ ﴿٦٩﴾}\\
70.\  & \mytextarabic{لَوْ نَشَآءُ جَعَلْنَـٰهُ أُجَاجًۭا فَلَوْلَا تَشْكُرُونَ ﴿٧٠﴾}\\
71.\  & \mytextarabic{أَفَرَءَيْتُمُ ٱلنَّارَ ٱلَّتِى تُورُونَ ﴿٧١﴾}\\
72.\  & \mytextarabic{ءَأَنتُمْ أَنشَأْتُمْ شَجَرَتَهَآ أَمْ نَحْنُ ٱلْمُنشِـُٔونَ ﴿٧٢﴾}\\
73.\  & \mytextarabic{نَحْنُ جَعَلْنَـٰهَا تَذْكِرَةًۭ وَمَتَـٰعًۭا لِّلْمُقْوِينَ ﴿٧٣﴾}\\
74.\  & \mytextarabic{فَسَبِّحْ بِٱسْمِ رَبِّكَ ٱلْعَظِيمِ ﴿٧٤﴾}\\
75.\  & \mytextarabic{۞ فَلَآ أُقْسِمُ بِمَوَٟقِعِ ٱلنُّجُومِ ﴿٧٥﴾}\\
76.\  & \mytextarabic{وَإِنَّهُۥ لَقَسَمٌۭ لَّوْ تَعْلَمُونَ عَظِيمٌ ﴿٧٦﴾}\\
77.\  & \mytextarabic{إِنَّهُۥ لَقُرْءَانٌۭ كَرِيمٌۭ ﴿٧٧﴾}\\
78.\  & \mytextarabic{فِى كِتَـٰبٍۢ مَّكْنُونٍۢ ﴿٧٨﴾}\\
79.\  & \mytextarabic{لَّا يَمَسُّهُۥٓ إِلَّا ٱلْمُطَهَّرُونَ ﴿٧٩﴾}\\
80.\  & \mytextarabic{تَنزِيلٌۭ مِّن رَّبِّ ٱلْعَـٰلَمِينَ ﴿٨٠﴾}\\
81.\  & \mytextarabic{أَفَبِهَـٰذَا ٱلْحَدِيثِ أَنتُم مُّدْهِنُونَ ﴿٨١﴾}\\
82.\  & \mytextarabic{وَتَجْعَلُونَ رِزْقَكُمْ أَنَّكُمْ تُكَذِّبُونَ ﴿٨٢﴾}\\
83.\  & \mytextarabic{فَلَوْلَآ إِذَا بَلَغَتِ ٱلْحُلْقُومَ ﴿٨٣﴾}\\
84.\  & \mytextarabic{وَأَنتُمْ حِينَئِذٍۢ تَنظُرُونَ ﴿٨٤﴾}\\
85.\  & \mytextarabic{وَنَحْنُ أَقْرَبُ إِلَيْهِ مِنكُمْ وَلَـٰكِن لَّا تُبْصِرُونَ ﴿٨٥﴾}\\
86.\  & \mytextarabic{فَلَوْلَآ إِن كُنتُمْ غَيْرَ مَدِينِينَ ﴿٨٦﴾}\\
87.\  & \mytextarabic{تَرْجِعُونَهَآ إِن كُنتُمْ صَـٰدِقِينَ ﴿٨٧﴾}\\
88.\  & \mytextarabic{فَأَمَّآ إِن كَانَ مِنَ ٱلْمُقَرَّبِينَ ﴿٨٨﴾}\\
89.\  & \mytextarabic{فَرَوْحٌۭ وَرَيْحَانٌۭ وَجَنَّتُ نَعِيمٍۢ ﴿٨٩﴾}\\
90.\  & \mytextarabic{وَأَمَّآ إِن كَانَ مِنْ أَصْحَـٰبِ ٱلْيَمِينِ ﴿٩٠﴾}\\
91.\  & \mytextarabic{فَسَلَـٰمٌۭ لَّكَ مِنْ أَصْحَـٰبِ ٱلْيَمِينِ ﴿٩١﴾}\\
92.\  & \mytextarabic{وَأَمَّآ إِن كَانَ مِنَ ٱلْمُكَذِّبِينَ ٱلضَّآلِّينَ ﴿٩٢﴾}\\
93.\  & \mytextarabic{فَنُزُلٌۭ مِّنْ حَمِيمٍۢ ﴿٩٣﴾}\\
94.\  & \mytextarabic{وَتَصْلِيَةُ جَحِيمٍ ﴿٩٤﴾}\\
95.\  & \mytextarabic{إِنَّ هَـٰذَا لَهُوَ حَقُّ ٱلْيَقِينِ ﴿٩٥﴾}\\
96.\  & \mytextarabic{فَسَبِّحْ بِٱسْمِ رَبِّكَ ٱلْعَظِيمِ ﴿٩٦﴾}\\
\end{longtable}
\clearpage