\begin{center}\section{ሱራቱ አተውባ -  \textarabic{سوره  التوبة}}\end{center}
\begin{longtable}{%
  @{}
    p{.5\textwidth}
  @{~~~}
    p{.5\textwidth}
    @{}
}
1.\  & \mytextarabic{بَرَآءَةٌۭ مِّنَ ٱللَّهِ وَرَسُولِهِۦٓ إِلَى ٱلَّذِينَ عَـٰهَدتُّم مِّنَ ٱلْمُشْرِكِينَ ﴿١﴾}\\
2.\  & \mytextarabic{فَسِيحُوا۟ فِى ٱلْأَرْضِ أَرْبَعَةَ أَشْهُرٍۢ وَٱعْلَمُوٓا۟ أَنَّكُمْ غَيْرُ مُعْجِزِى ٱللَّهِ ۙ وَأَنَّ ٱللَّهَ مُخْزِى ٱلْكَـٰفِرِينَ ﴿٢﴾}\\
3.\  & \mytextarabic{وَأَذَٟنٌۭ مِّنَ ٱللَّهِ وَرَسُولِهِۦٓ إِلَى ٱلنَّاسِ يَوْمَ ٱلْحَجِّ ٱلْأَكْبَرِ أَنَّ ٱللَّهَ بَرِىٓءٌۭ مِّنَ ٱلْمُشْرِكِينَ ۙ وَرَسُولُهُۥ ۚ فَإِن تُبْتُمْ فَهُوَ خَيْرٌۭ لَّكُمْ ۖ وَإِن تَوَلَّيْتُمْ فَٱعْلَمُوٓا۟ أَنَّكُمْ غَيْرُ مُعْجِزِى ٱللَّهِ ۗ وَبَشِّرِ ٱلَّذِينَ كَفَرُوا۟ بِعَذَابٍ أَلِيمٍ ﴿٣﴾}\\
4.\  & \mytextarabic{إِلَّا ٱلَّذِينَ عَـٰهَدتُّم مِّنَ ٱلْمُشْرِكِينَ ثُمَّ لَمْ يَنقُصُوكُمْ شَيْـًۭٔا وَلَمْ يُظَـٰهِرُوا۟ عَلَيْكُمْ أَحَدًۭا فَأَتِمُّوٓا۟ إِلَيْهِمْ عَهْدَهُمْ إِلَىٰ مُدَّتِهِمْ ۚ إِنَّ ٱللَّهَ يُحِبُّ ٱلْمُتَّقِينَ ﴿٤﴾}\\
5.\  & \mytextarabic{فَإِذَا ٱنسَلَخَ ٱلْأَشْهُرُ ٱلْحُرُمُ فَٱقْتُلُوا۟ ٱلْمُشْرِكِينَ حَيْثُ وَجَدتُّمُوهُمْ وَخُذُوهُمْ وَٱحْصُرُوهُمْ وَٱقْعُدُوا۟ لَهُمْ كُلَّ مَرْصَدٍۢ ۚ فَإِن تَابُوا۟ وَأَقَامُوا۟ ٱلصَّلَوٰةَ وَءَاتَوُا۟ ٱلزَّكَوٰةَ فَخَلُّوا۟ سَبِيلَهُمْ ۚ إِنَّ ٱللَّهَ غَفُورٌۭ رَّحِيمٌۭ ﴿٥﴾}\\
6.\  & \mytextarabic{وَإِنْ أَحَدٌۭ مِّنَ ٱلْمُشْرِكِينَ ٱسْتَجَارَكَ فَأَجِرْهُ حَتَّىٰ يَسْمَعَ كَلَـٰمَ ٱللَّهِ ثُمَّ أَبْلِغْهُ مَأْمَنَهُۥ ۚ ذَٟلِكَ بِأَنَّهُمْ قَوْمٌۭ لَّا يَعْلَمُونَ ﴿٦﴾}\\
7.\  & \mytextarabic{كَيْفَ يَكُونُ لِلْمُشْرِكِينَ عَهْدٌ عِندَ ٱللَّهِ وَعِندَ رَسُولِهِۦٓ إِلَّا ٱلَّذِينَ عَـٰهَدتُّمْ عِندَ ٱلْمَسْجِدِ ٱلْحَرَامِ ۖ فَمَا ٱسْتَقَـٰمُوا۟ لَكُمْ فَٱسْتَقِيمُوا۟ لَهُمْ ۚ إِنَّ ٱللَّهَ يُحِبُّ ٱلْمُتَّقِينَ ﴿٧﴾}\\
8.\  & \mytextarabic{كَيْفَ وَإِن يَظْهَرُوا۟ عَلَيْكُمْ لَا يَرْقُبُوا۟ فِيكُمْ إِلًّۭا وَلَا ذِمَّةًۭ ۚ يُرْضُونَكُم بِأَفْوَٟهِهِمْ وَتَأْبَىٰ قُلُوبُهُمْ وَأَكْثَرُهُمْ فَـٰسِقُونَ ﴿٨﴾}\\
9.\  & \mytextarabic{ٱشْتَرَوْا۟ بِـَٔايَـٰتِ ٱللَّهِ ثَمَنًۭا قَلِيلًۭا فَصَدُّوا۟ عَن سَبِيلِهِۦٓ ۚ إِنَّهُمْ سَآءَ مَا كَانُوا۟ يَعْمَلُونَ ﴿٩﴾}\\
10.\  & \mytextarabic{لَا يَرْقُبُونَ فِى مُؤْمِنٍ إِلًّۭا وَلَا ذِمَّةًۭ ۚ وَأُو۟لَـٰٓئِكَ هُمُ ٱلْمُعْتَدُونَ ﴿١٠﴾}\\
11.\  & \mytextarabic{فَإِن تَابُوا۟ وَأَقَامُوا۟ ٱلصَّلَوٰةَ وَءَاتَوُا۟ ٱلزَّكَوٰةَ فَإِخْوَٟنُكُمْ فِى ٱلدِّينِ ۗ وَنُفَصِّلُ ٱلْءَايَـٰتِ لِقَوْمٍۢ يَعْلَمُونَ ﴿١١﴾}\\
12.\  & \mytextarabic{وَإِن نَّكَثُوٓا۟ أَيْمَـٰنَهُم مِّنۢ بَعْدِ عَهْدِهِمْ وَطَعَنُوا۟ فِى دِينِكُمْ فَقَـٰتِلُوٓا۟ أَئِمَّةَ ٱلْكُفْرِ ۙ إِنَّهُمْ لَآ أَيْمَـٰنَ لَهُمْ لَعَلَّهُمْ يَنتَهُونَ ﴿١٢﴾}\\
13.\  & \mytextarabic{أَلَا تُقَـٰتِلُونَ قَوْمًۭا نَّكَثُوٓا۟ أَيْمَـٰنَهُمْ وَهَمُّوا۟ بِإِخْرَاجِ ٱلرَّسُولِ وَهُم بَدَءُوكُمْ أَوَّلَ مَرَّةٍ ۚ أَتَخْشَوْنَهُمْ ۚ فَٱللَّهُ أَحَقُّ أَن تَخْشَوْهُ إِن كُنتُم مُّؤْمِنِينَ ﴿١٣﴾}\\
14.\  & \mytextarabic{قَـٰتِلُوهُمْ يُعَذِّبْهُمُ ٱللَّهُ بِأَيْدِيكُمْ وَيُخْزِهِمْ وَيَنصُرْكُمْ عَلَيْهِمْ وَيَشْفِ صُدُورَ قَوْمٍۢ مُّؤْمِنِينَ ﴿١٤﴾}\\
15.\  & \mytextarabic{وَيُذْهِبْ غَيْظَ قُلُوبِهِمْ ۗ وَيَتُوبُ ٱللَّهُ عَلَىٰ مَن يَشَآءُ ۗ وَٱللَّهُ عَلِيمٌ حَكِيمٌ ﴿١٥﴾}\\
16.\  & \mytextarabic{أَمْ حَسِبْتُمْ أَن تُتْرَكُوا۟ وَلَمَّا يَعْلَمِ ٱللَّهُ ٱلَّذِينَ جَٰهَدُوا۟ مِنكُمْ وَلَمْ يَتَّخِذُوا۟ مِن دُونِ ٱللَّهِ وَلَا رَسُولِهِۦ وَلَا ٱلْمُؤْمِنِينَ وَلِيجَةًۭ ۚ وَٱللَّهُ خَبِيرٌۢ بِمَا تَعْمَلُونَ ﴿١٦﴾}\\
17.\  & \mytextarabic{مَا كَانَ لِلْمُشْرِكِينَ أَن يَعْمُرُوا۟ مَسَـٰجِدَ ٱللَّهِ شَـٰهِدِينَ عَلَىٰٓ أَنفُسِهِم بِٱلْكُفْرِ ۚ أُو۟لَـٰٓئِكَ حَبِطَتْ أَعْمَـٰلُهُمْ وَفِى ٱلنَّارِ هُمْ خَـٰلِدُونَ ﴿١٧﴾}\\
18.\  & \mytextarabic{إِنَّمَا يَعْمُرُ مَسَـٰجِدَ ٱللَّهِ مَنْ ءَامَنَ بِٱللَّهِ وَٱلْيَوْمِ ٱلْءَاخِرِ وَأَقَامَ ٱلصَّلَوٰةَ وَءَاتَى ٱلزَّكَوٰةَ وَلَمْ يَخْشَ إِلَّا ٱللَّهَ ۖ فَعَسَىٰٓ أُو۟لَـٰٓئِكَ أَن يَكُونُوا۟ مِنَ ٱلْمُهْتَدِينَ ﴿١٨﴾}\\
19.\  & \mytextarabic{۞ أَجَعَلْتُمْ سِقَايَةَ ٱلْحَآجِّ وَعِمَارَةَ ٱلْمَسْجِدِ ٱلْحَرَامِ كَمَنْ ءَامَنَ بِٱللَّهِ وَٱلْيَوْمِ ٱلْءَاخِرِ وَجَٰهَدَ فِى سَبِيلِ ٱللَّهِ ۚ لَا يَسْتَوُۥنَ عِندَ ٱللَّهِ ۗ وَٱللَّهُ لَا يَهْدِى ٱلْقَوْمَ ٱلظَّـٰلِمِينَ ﴿١٩﴾}\\
20.\  & \mytextarabic{ٱلَّذِينَ ءَامَنُوا۟ وَهَاجَرُوا۟ وَجَٰهَدُوا۟ فِى سَبِيلِ ٱللَّهِ بِأَمْوَٟلِهِمْ وَأَنفُسِهِمْ أَعْظَمُ دَرَجَةً عِندَ ٱللَّهِ ۚ وَأُو۟لَـٰٓئِكَ هُمُ ٱلْفَآئِزُونَ ﴿٢٠﴾}\\
21.\  & \mytextarabic{يُبَشِّرُهُمْ رَبُّهُم بِرَحْمَةٍۢ مِّنْهُ وَرِضْوَٟنٍۢ وَجَنَّـٰتٍۢ لَّهُمْ فِيهَا نَعِيمٌۭ مُّقِيمٌ ﴿٢١﴾}\\
22.\  & \mytextarabic{خَـٰلِدِينَ فِيهَآ أَبَدًا ۚ إِنَّ ٱللَّهَ عِندَهُۥٓ أَجْرٌ عَظِيمٌۭ ﴿٢٢﴾}\\
23.\  & \mytextarabic{يَـٰٓأَيُّهَا ٱلَّذِينَ ءَامَنُوا۟ لَا تَتَّخِذُوٓا۟ ءَابَآءَكُمْ وَإِخْوَٟنَكُمْ أَوْلِيَآءَ إِنِ ٱسْتَحَبُّوا۟ ٱلْكُفْرَ عَلَى ٱلْإِيمَـٰنِ ۚ وَمَن يَتَوَلَّهُم مِّنكُمْ فَأُو۟لَـٰٓئِكَ هُمُ ٱلظَّـٰلِمُونَ ﴿٢٣﴾}\\
24.\  & \mytextarabic{قُلْ إِن كَانَ ءَابَآؤُكُمْ وَأَبْنَآؤُكُمْ وَإِخْوَٟنُكُمْ وَأَزْوَٟجُكُمْ وَعَشِيرَتُكُمْ وَأَمْوَٟلٌ ٱقْتَرَفْتُمُوهَا وَتِجَٰرَةٌۭ تَخْشَوْنَ كَسَادَهَا وَمَسَـٰكِنُ تَرْضَوْنَهَآ أَحَبَّ إِلَيْكُم مِّنَ ٱللَّهِ وَرَسُولِهِۦ وَجِهَادٍۢ فِى سَبِيلِهِۦ فَتَرَبَّصُوا۟ حَتَّىٰ يَأْتِىَ ٱللَّهُ بِأَمْرِهِۦ ۗ وَٱللَّهُ لَا يَهْدِى ٱلْقَوْمَ ٱلْفَـٰسِقِينَ ﴿٢٤﴾}\\
25.\  & \mytextarabic{لَقَدْ نَصَرَكُمُ ٱللَّهُ فِى مَوَاطِنَ كَثِيرَةٍۢ ۙ وَيَوْمَ حُنَيْنٍ ۙ إِذْ أَعْجَبَتْكُمْ كَثْرَتُكُمْ فَلَمْ تُغْنِ عَنكُمْ شَيْـًۭٔا وَضَاقَتْ عَلَيْكُمُ ٱلْأَرْضُ بِمَا رَحُبَتْ ثُمَّ وَلَّيْتُم مُّدْبِرِينَ ﴿٢٥﴾}\\
26.\  & \mytextarabic{ثُمَّ أَنزَلَ ٱللَّهُ سَكِينَتَهُۥ عَلَىٰ رَسُولِهِۦ وَعَلَى ٱلْمُؤْمِنِينَ وَأَنزَلَ جُنُودًۭا لَّمْ تَرَوْهَا وَعَذَّبَ ٱلَّذِينَ كَفَرُوا۟ ۚ وَذَٟلِكَ جَزَآءُ ٱلْكَـٰفِرِينَ ﴿٢٦﴾}\\
27.\  & \mytextarabic{ثُمَّ يَتُوبُ ٱللَّهُ مِنۢ بَعْدِ ذَٟلِكَ عَلَىٰ مَن يَشَآءُ ۗ وَٱللَّهُ غَفُورٌۭ رَّحِيمٌۭ ﴿٢٧﴾}\\
28.\  & \mytextarabic{يَـٰٓأَيُّهَا ٱلَّذِينَ ءَامَنُوٓا۟ إِنَّمَا ٱلْمُشْرِكُونَ نَجَسٌۭ فَلَا يَقْرَبُوا۟ ٱلْمَسْجِدَ ٱلْحَرَامَ بَعْدَ عَامِهِمْ هَـٰذَا ۚ وَإِنْ خِفْتُمْ عَيْلَةًۭ فَسَوْفَ يُغْنِيكُمُ ٱللَّهُ مِن فَضْلِهِۦٓ إِن شَآءَ ۚ إِنَّ ٱللَّهَ عَلِيمٌ حَكِيمٌۭ ﴿٢٨﴾}\\
29.\  & \mytextarabic{قَـٰتِلُوا۟ ٱلَّذِينَ لَا يُؤْمِنُونَ بِٱللَّهِ وَلَا بِٱلْيَوْمِ ٱلْءَاخِرِ وَلَا يُحَرِّمُونَ مَا حَرَّمَ ٱللَّهُ وَرَسُولُهُۥ وَلَا يَدِينُونَ دِينَ ٱلْحَقِّ مِنَ ٱلَّذِينَ أُوتُوا۟ ٱلْكِتَـٰبَ حَتَّىٰ يُعْطُوا۟ ٱلْجِزْيَةَ عَن يَدٍۢ وَهُمْ صَـٰغِرُونَ ﴿٢٩﴾}\\
30.\  & \mytextarabic{وَقَالَتِ ٱلْيَهُودُ عُزَيْرٌ ٱبْنُ ٱللَّهِ وَقَالَتِ ٱلنَّصَـٰرَى ٱلْمَسِيحُ ٱبْنُ ٱللَّهِ ۖ ذَٟلِكَ قَوْلُهُم بِأَفْوَٟهِهِمْ ۖ يُضَٰهِـُٔونَ قَوْلَ ٱلَّذِينَ كَفَرُوا۟ مِن قَبْلُ ۚ قَـٰتَلَهُمُ ٱللَّهُ ۚ أَنَّىٰ يُؤْفَكُونَ ﴿٣٠﴾}\\
31.\  & \mytextarabic{ٱتَّخَذُوٓا۟ أَحْبَارَهُمْ وَرُهْبَٰنَهُمْ أَرْبَابًۭا مِّن دُونِ ٱللَّهِ وَٱلْمَسِيحَ ٱبْنَ مَرْيَمَ وَمَآ أُمِرُوٓا۟ إِلَّا لِيَعْبُدُوٓا۟ إِلَـٰهًۭا وَٟحِدًۭا ۖ لَّآ إِلَـٰهَ إِلَّا هُوَ ۚ سُبْحَـٰنَهُۥ عَمَّا يُشْرِكُونَ ﴿٣١﴾}\\
32.\  & \mytextarabic{يُرِيدُونَ أَن يُطْفِـُٔوا۟ نُورَ ٱللَّهِ بِأَفْوَٟهِهِمْ وَيَأْبَى ٱللَّهُ إِلَّآ أَن يُتِمَّ نُورَهُۥ وَلَوْ كَرِهَ ٱلْكَـٰفِرُونَ ﴿٣٢﴾}\\
33.\  & \mytextarabic{هُوَ ٱلَّذِىٓ أَرْسَلَ رَسُولَهُۥ بِٱلْهُدَىٰ وَدِينِ ٱلْحَقِّ لِيُظْهِرَهُۥ عَلَى ٱلدِّينِ كُلِّهِۦ وَلَوْ كَرِهَ ٱلْمُشْرِكُونَ ﴿٣٣﴾}\\
34.\  & \mytextarabic{۞ يَـٰٓأَيُّهَا ٱلَّذِينَ ءَامَنُوٓا۟ إِنَّ كَثِيرًۭا مِّنَ ٱلْأَحْبَارِ وَٱلرُّهْبَانِ لَيَأْكُلُونَ أَمْوَٟلَ ٱلنَّاسِ بِٱلْبَٰطِلِ وَيَصُدُّونَ عَن سَبِيلِ ٱللَّهِ ۗ وَٱلَّذِينَ يَكْنِزُونَ ٱلذَّهَبَ وَٱلْفِضَّةَ وَلَا يُنفِقُونَهَا فِى سَبِيلِ ٱللَّهِ فَبَشِّرْهُم بِعَذَابٍ أَلِيمٍۢ ﴿٣٤﴾}\\
35.\  & \mytextarabic{يَوْمَ يُحْمَىٰ عَلَيْهَا فِى نَارِ جَهَنَّمَ فَتُكْوَىٰ بِهَا جِبَاهُهُمْ وَجُنُوبُهُمْ وَظُهُورُهُمْ ۖ هَـٰذَا مَا كَنَزْتُمْ لِأَنفُسِكُمْ فَذُوقُوا۟ مَا كُنتُمْ تَكْنِزُونَ ﴿٣٥﴾}\\
36.\  & \mytextarabic{إِنَّ عِدَّةَ ٱلشُّهُورِ عِندَ ٱللَّهِ ٱثْنَا عَشَرَ شَهْرًۭا فِى كِتَـٰبِ ٱللَّهِ يَوْمَ خَلَقَ ٱلسَّمَـٰوَٟتِ وَٱلْأَرْضَ مِنْهَآ أَرْبَعَةٌ حُرُمٌۭ ۚ ذَٟلِكَ ٱلدِّينُ ٱلْقَيِّمُ ۚ فَلَا تَظْلِمُوا۟ فِيهِنَّ أَنفُسَكُمْ ۚ وَقَـٰتِلُوا۟ ٱلْمُشْرِكِينَ كَآفَّةًۭ كَمَا يُقَـٰتِلُونَكُمْ كَآفَّةًۭ ۚ وَٱعْلَمُوٓا۟ أَنَّ ٱللَّهَ مَعَ ٱلْمُتَّقِينَ ﴿٣٦﴾}\\
37.\  & \mytextarabic{إِنَّمَا ٱلنَّسِىٓءُ زِيَادَةٌۭ فِى ٱلْكُفْرِ ۖ يُضَلُّ بِهِ ٱلَّذِينَ كَفَرُوا۟ يُحِلُّونَهُۥ عَامًۭا وَيُحَرِّمُونَهُۥ عَامًۭا لِّيُوَاطِـُٔوا۟ عِدَّةَ مَا حَرَّمَ ٱللَّهُ فَيُحِلُّوا۟ مَا حَرَّمَ ٱللَّهُ ۚ زُيِّنَ لَهُمْ سُوٓءُ أَعْمَـٰلِهِمْ ۗ وَٱللَّهُ لَا يَهْدِى ٱلْقَوْمَ ٱلْكَـٰفِرِينَ ﴿٣٧﴾}\\
38.\  & \mytextarabic{يَـٰٓأَيُّهَا ٱلَّذِينَ ءَامَنُوا۟ مَا لَكُمْ إِذَا قِيلَ لَكُمُ ٱنفِرُوا۟ فِى سَبِيلِ ٱللَّهِ ٱثَّاقَلْتُمْ إِلَى ٱلْأَرْضِ ۚ أَرَضِيتُم بِٱلْحَيَوٰةِ ٱلدُّنْيَا مِنَ ٱلْءَاخِرَةِ ۚ فَمَا مَتَـٰعُ ٱلْحَيَوٰةِ ٱلدُّنْيَا فِى ٱلْءَاخِرَةِ إِلَّا قَلِيلٌ ﴿٣٨﴾}\\
39.\  & \mytextarabic{إِلَّا تَنفِرُوا۟ يُعَذِّبْكُمْ عَذَابًا أَلِيمًۭا وَيَسْتَبْدِلْ قَوْمًا غَيْرَكُمْ وَلَا تَضُرُّوهُ شَيْـًۭٔا ۗ وَٱللَّهُ عَلَىٰ كُلِّ شَىْءٍۢ قَدِيرٌ ﴿٣٩﴾}\\
40.\  & \mytextarabic{إِلَّا تَنصُرُوهُ فَقَدْ نَصَرَهُ ٱللَّهُ إِذْ أَخْرَجَهُ ٱلَّذِينَ كَفَرُوا۟ ثَانِىَ ٱثْنَيْنِ إِذْ هُمَا فِى ٱلْغَارِ إِذْ يَقُولُ لِصَـٰحِبِهِۦ لَا تَحْزَنْ إِنَّ ٱللَّهَ مَعَنَا ۖ فَأَنزَلَ ٱللَّهُ سَكِينَتَهُۥ عَلَيْهِ وَأَيَّدَهُۥ بِجُنُودٍۢ لَّمْ تَرَوْهَا وَجَعَلَ كَلِمَةَ ٱلَّذِينَ كَفَرُوا۟ ٱلسُّفْلَىٰ ۗ وَكَلِمَةُ ٱللَّهِ هِىَ ٱلْعُلْيَا ۗ وَٱللَّهُ عَزِيزٌ حَكِيمٌ ﴿٤٠﴾}\\
41.\  & \mytextarabic{ٱنفِرُوا۟ خِفَافًۭا وَثِقَالًۭا وَجَٰهِدُوا۟ بِأَمْوَٟلِكُمْ وَأَنفُسِكُمْ فِى سَبِيلِ ٱللَّهِ ۚ ذَٟلِكُمْ خَيْرٌۭ لَّكُمْ إِن كُنتُمْ تَعْلَمُونَ ﴿٤١﴾}\\
42.\  & \mytextarabic{لَوْ كَانَ عَرَضًۭا قَرِيبًۭا وَسَفَرًۭا قَاصِدًۭا لَّٱتَّبَعُوكَ وَلَـٰكِنۢ بَعُدَتْ عَلَيْهِمُ ٱلشُّقَّةُ ۚ وَسَيَحْلِفُونَ بِٱللَّهِ لَوِ ٱسْتَطَعْنَا لَخَرَجْنَا مَعَكُمْ يُهْلِكُونَ أَنفُسَهُمْ وَٱللَّهُ يَعْلَمُ إِنَّهُمْ لَكَـٰذِبُونَ ﴿٤٢﴾}\\
43.\  & \mytextarabic{عَفَا ٱللَّهُ عَنكَ لِمَ أَذِنتَ لَهُمْ حَتَّىٰ يَتَبَيَّنَ لَكَ ٱلَّذِينَ صَدَقُوا۟ وَتَعْلَمَ ٱلْكَـٰذِبِينَ ﴿٤٣﴾}\\
44.\  & \mytextarabic{لَا يَسْتَـْٔذِنُكَ ٱلَّذِينَ يُؤْمِنُونَ بِٱللَّهِ وَٱلْيَوْمِ ٱلْءَاخِرِ أَن يُجَٰهِدُوا۟ بِأَمْوَٟلِهِمْ وَأَنفُسِهِمْ ۗ وَٱللَّهُ عَلِيمٌۢ بِٱلْمُتَّقِينَ ﴿٤٤﴾}\\
45.\  & \mytextarabic{إِنَّمَا يَسْتَـْٔذِنُكَ ٱلَّذِينَ لَا يُؤْمِنُونَ بِٱللَّهِ وَٱلْيَوْمِ ٱلْءَاخِرِ وَٱرْتَابَتْ قُلُوبُهُمْ فَهُمْ فِى رَيْبِهِمْ يَتَرَدَّدُونَ ﴿٤٥﴾}\\
46.\  & \mytextarabic{۞ وَلَوْ أَرَادُوا۟ ٱلْخُرُوجَ لَأَعَدُّوا۟ لَهُۥ عُدَّةًۭ وَلَـٰكِن كَرِهَ ٱللَّهُ ٱنۢبِعَاثَهُمْ فَثَبَّطَهُمْ وَقِيلَ ٱقْعُدُوا۟ مَعَ ٱلْقَـٰعِدِينَ ﴿٤٦﴾}\\
47.\  & \mytextarabic{لَوْ خَرَجُوا۟ فِيكُم مَّا زَادُوكُمْ إِلَّا خَبَالًۭا وَلَأَوْضَعُوا۟ خِلَـٰلَكُمْ يَبْغُونَكُمُ ٱلْفِتْنَةَ وَفِيكُمْ سَمَّٰعُونَ لَهُمْ ۗ وَٱللَّهُ عَلِيمٌۢ بِٱلظَّـٰلِمِينَ ﴿٤٧﴾}\\
48.\  & \mytextarabic{لَقَدِ ٱبْتَغَوُا۟ ٱلْفِتْنَةَ مِن قَبْلُ وَقَلَّبُوا۟ لَكَ ٱلْأُمُورَ حَتَّىٰ جَآءَ ٱلْحَقُّ وَظَهَرَ أَمْرُ ٱللَّهِ وَهُمْ كَـٰرِهُونَ ﴿٤٨﴾}\\
49.\  & \mytextarabic{وَمِنْهُم مَّن يَقُولُ ٱئْذَن لِّى وَلَا تَفْتِنِّىٓ ۚ أَلَا فِى ٱلْفِتْنَةِ سَقَطُوا۟ ۗ وَإِنَّ جَهَنَّمَ لَمُحِيطَةٌۢ بِٱلْكَـٰفِرِينَ ﴿٤٩﴾}\\
50.\  & \mytextarabic{إِن تُصِبْكَ حَسَنَةٌۭ تَسُؤْهُمْ ۖ وَإِن تُصِبْكَ مُصِيبَةٌۭ يَقُولُوا۟ قَدْ أَخَذْنَآ أَمْرَنَا مِن قَبْلُ وَيَتَوَلَّوا۟ وَّهُمْ فَرِحُونَ ﴿٥٠﴾}\\
51.\  & \mytextarabic{قُل لَّن يُصِيبَنَآ إِلَّا مَا كَتَبَ ٱللَّهُ لَنَا هُوَ مَوْلَىٰنَا ۚ وَعَلَى ٱللَّهِ فَلْيَتَوَكَّلِ ٱلْمُؤْمِنُونَ ﴿٥١﴾}\\
52.\  & \mytextarabic{قُلْ هَلْ تَرَبَّصُونَ بِنَآ إِلَّآ إِحْدَى ٱلْحُسْنَيَيْنِ ۖ وَنَحْنُ نَتَرَبَّصُ بِكُمْ أَن يُصِيبَكُمُ ٱللَّهُ بِعَذَابٍۢ مِّنْ عِندِهِۦٓ أَوْ بِأَيْدِينَا ۖ فَتَرَبَّصُوٓا۟ إِنَّا مَعَكُم مُّتَرَبِّصُونَ ﴿٥٢﴾}\\
53.\  & \mytextarabic{قُلْ أَنفِقُوا۟ طَوْعًا أَوْ كَرْهًۭا لَّن يُتَقَبَّلَ مِنكُمْ ۖ إِنَّكُمْ كُنتُمْ قَوْمًۭا فَـٰسِقِينَ ﴿٥٣﴾}\\
54.\  & \mytextarabic{وَمَا مَنَعَهُمْ أَن تُقْبَلَ مِنْهُمْ نَفَقَـٰتُهُمْ إِلَّآ أَنَّهُمْ كَفَرُوا۟ بِٱللَّهِ وَبِرَسُولِهِۦ وَلَا يَأْتُونَ ٱلصَّلَوٰةَ إِلَّا وَهُمْ كُسَالَىٰ وَلَا يُنفِقُونَ إِلَّا وَهُمْ كَـٰرِهُونَ ﴿٥٤﴾}\\
55.\  & \mytextarabic{فَلَا تُعْجِبْكَ أَمْوَٟلُهُمْ وَلَآ أَوْلَـٰدُهُمْ ۚ إِنَّمَا يُرِيدُ ٱللَّهُ لِيُعَذِّبَهُم بِهَا فِى ٱلْحَيَوٰةِ ٱلدُّنْيَا وَتَزْهَقَ أَنفُسُهُمْ وَهُمْ كَـٰفِرُونَ ﴿٥٥﴾}\\
56.\  & \mytextarabic{وَيَحْلِفُونَ بِٱللَّهِ إِنَّهُمْ لَمِنكُمْ وَمَا هُم مِّنكُمْ وَلَـٰكِنَّهُمْ قَوْمٌۭ يَفْرَقُونَ ﴿٥٦﴾}\\
57.\  & \mytextarabic{لَوْ يَجِدُونَ مَلْجَـًٔا أَوْ مَغَٰرَٰتٍ أَوْ مُدَّخَلًۭا لَّوَلَّوْا۟ إِلَيْهِ وَهُمْ يَجْمَحُونَ ﴿٥٧﴾}\\
58.\  & \mytextarabic{وَمِنْهُم مَّن يَلْمِزُكَ فِى ٱلصَّدَقَـٰتِ فَإِنْ أُعْطُوا۟ مِنْهَا رَضُوا۟ وَإِن لَّمْ يُعْطَوْا۟ مِنْهَآ إِذَا هُمْ يَسْخَطُونَ ﴿٥٨﴾}\\
59.\  & \mytextarabic{وَلَوْ أَنَّهُمْ رَضُوا۟ مَآ ءَاتَىٰهُمُ ٱللَّهُ وَرَسُولُهُۥ وَقَالُوا۟ حَسْبُنَا ٱللَّهُ سَيُؤْتِينَا ٱللَّهُ مِن فَضْلِهِۦ وَرَسُولُهُۥٓ إِنَّآ إِلَى ٱللَّهِ رَٰغِبُونَ ﴿٥٩﴾}\\
60.\  & \mytextarabic{۞ إِنَّمَا ٱلصَّدَقَـٰتُ لِلْفُقَرَآءِ وَٱلْمَسَـٰكِينِ وَٱلْعَـٰمِلِينَ عَلَيْهَا وَٱلْمُؤَلَّفَةِ قُلُوبُهُمْ وَفِى ٱلرِّقَابِ وَٱلْغَٰرِمِينَ وَفِى سَبِيلِ ٱللَّهِ وَٱبْنِ ٱلسَّبِيلِ ۖ فَرِيضَةًۭ مِّنَ ٱللَّهِ ۗ وَٱللَّهُ عَلِيمٌ حَكِيمٌۭ ﴿٦٠﴾}\\
61.\  & \mytextarabic{وَمِنْهُمُ ٱلَّذِينَ يُؤْذُونَ ٱلنَّبِىَّ وَيَقُولُونَ هُوَ أُذُنٌۭ ۚ قُلْ أُذُنُ خَيْرٍۢ لَّكُمْ يُؤْمِنُ بِٱللَّهِ وَيُؤْمِنُ لِلْمُؤْمِنِينَ وَرَحْمَةٌۭ لِّلَّذِينَ ءَامَنُوا۟ مِنكُمْ ۚ وَٱلَّذِينَ يُؤْذُونَ رَسُولَ ٱللَّهِ لَهُمْ عَذَابٌ أَلِيمٌۭ ﴿٦١﴾}\\
62.\  & \mytextarabic{يَحْلِفُونَ بِٱللَّهِ لَكُمْ لِيُرْضُوكُمْ وَٱللَّهُ وَرَسُولُهُۥٓ أَحَقُّ أَن يُرْضُوهُ إِن كَانُوا۟ مُؤْمِنِينَ ﴿٦٢﴾}\\
63.\  & \mytextarabic{أَلَمْ يَعْلَمُوٓا۟ أَنَّهُۥ مَن يُحَادِدِ ٱللَّهَ وَرَسُولَهُۥ فَأَنَّ لَهُۥ نَارَ جَهَنَّمَ خَـٰلِدًۭا فِيهَا ۚ ذَٟلِكَ ٱلْخِزْىُ ٱلْعَظِيمُ ﴿٦٣﴾}\\
64.\  & \mytextarabic{يَحْذَرُ ٱلْمُنَـٰفِقُونَ أَن تُنَزَّلَ عَلَيْهِمْ سُورَةٌۭ تُنَبِّئُهُم بِمَا فِى قُلُوبِهِمْ ۚ قُلِ ٱسْتَهْزِءُوٓا۟ إِنَّ ٱللَّهَ مُخْرِجٌۭ مَّا تَحْذَرُونَ ﴿٦٤﴾}\\
65.\  & \mytextarabic{وَلَئِن سَأَلْتَهُمْ لَيَقُولُنَّ إِنَّمَا كُنَّا نَخُوضُ وَنَلْعَبُ ۚ قُلْ أَبِٱللَّهِ وَءَايَـٰتِهِۦ وَرَسُولِهِۦ كُنتُمْ تَسْتَهْزِءُونَ ﴿٦٥﴾}\\
66.\  & \mytextarabic{لَا تَعْتَذِرُوا۟ قَدْ كَفَرْتُم بَعْدَ إِيمَـٰنِكُمْ ۚ إِن نَّعْفُ عَن طَآئِفَةٍۢ مِّنكُمْ نُعَذِّبْ طَآئِفَةًۢ بِأَنَّهُمْ كَانُوا۟ مُجْرِمِينَ ﴿٦٦﴾}\\
67.\  & \mytextarabic{ٱلْمُنَـٰفِقُونَ وَٱلْمُنَـٰفِقَـٰتُ بَعْضُهُم مِّنۢ بَعْضٍۢ ۚ يَأْمُرُونَ بِٱلْمُنكَرِ وَيَنْهَوْنَ عَنِ ٱلْمَعْرُوفِ وَيَقْبِضُونَ أَيْدِيَهُمْ ۚ نَسُوا۟ ٱللَّهَ فَنَسِيَهُمْ ۗ إِنَّ ٱلْمُنَـٰفِقِينَ هُمُ ٱلْفَـٰسِقُونَ ﴿٦٧﴾}\\
68.\  & \mytextarabic{وَعَدَ ٱللَّهُ ٱلْمُنَـٰفِقِينَ وَٱلْمُنَـٰفِقَـٰتِ وَٱلْكُفَّارَ نَارَ جَهَنَّمَ خَـٰلِدِينَ فِيهَا ۚ هِىَ حَسْبُهُمْ ۚ وَلَعَنَهُمُ ٱللَّهُ ۖ وَلَهُمْ عَذَابٌۭ مُّقِيمٌۭ ﴿٦٨﴾}\\
69.\  & \mytextarabic{كَٱلَّذِينَ مِن قَبْلِكُمْ كَانُوٓا۟ أَشَدَّ مِنكُمْ قُوَّةًۭ وَأَكْثَرَ أَمْوَٟلًۭا وَأَوْلَـٰدًۭا فَٱسْتَمْتَعُوا۟ بِخَلَـٰقِهِمْ فَٱسْتَمْتَعْتُم بِخَلَـٰقِكُمْ كَمَا ٱسْتَمْتَعَ ٱلَّذِينَ مِن قَبْلِكُم بِخَلَـٰقِهِمْ وَخُضْتُمْ كَٱلَّذِى خَاضُوٓا۟ ۚ أُو۟لَـٰٓئِكَ حَبِطَتْ أَعْمَـٰلُهُمْ فِى ٱلدُّنْيَا وَٱلْءَاخِرَةِ ۖ وَأُو۟لَـٰٓئِكَ هُمُ ٱلْخَـٰسِرُونَ ﴿٦٩﴾}\\
70.\  & \mytextarabic{أَلَمْ يَأْتِهِمْ نَبَأُ ٱلَّذِينَ مِن قَبْلِهِمْ قَوْمِ نُوحٍۢ وَعَادٍۢ وَثَمُودَ وَقَوْمِ إِبْرَٰهِيمَ وَأَصْحَـٰبِ مَدْيَنَ وَٱلْمُؤْتَفِكَـٰتِ ۚ أَتَتْهُمْ رُسُلُهُم بِٱلْبَيِّنَـٰتِ ۖ فَمَا كَانَ ٱللَّهُ لِيَظْلِمَهُمْ وَلَـٰكِن كَانُوٓا۟ أَنفُسَهُمْ يَظْلِمُونَ ﴿٧٠﴾}\\
71.\  & \mytextarabic{وَٱلْمُؤْمِنُونَ وَٱلْمُؤْمِنَـٰتُ بَعْضُهُمْ أَوْلِيَآءُ بَعْضٍۢ ۚ يَأْمُرُونَ بِٱلْمَعْرُوفِ وَيَنْهَوْنَ عَنِ ٱلْمُنكَرِ وَيُقِيمُونَ ٱلصَّلَوٰةَ وَيُؤْتُونَ ٱلزَّكَوٰةَ وَيُطِيعُونَ ٱللَّهَ وَرَسُولَهُۥٓ ۚ أُو۟لَـٰٓئِكَ سَيَرْحَمُهُمُ ٱللَّهُ ۗ إِنَّ ٱللَّهَ عَزِيزٌ حَكِيمٌۭ ﴿٧١﴾}\\
72.\  & \mytextarabic{وَعَدَ ٱللَّهُ ٱلْمُؤْمِنِينَ وَٱلْمُؤْمِنَـٰتِ جَنَّـٰتٍۢ تَجْرِى مِن تَحْتِهَا ٱلْأَنْهَـٰرُ خَـٰلِدِينَ فِيهَا وَمَسَـٰكِنَ طَيِّبَةًۭ فِى جَنَّـٰتِ عَدْنٍۢ ۚ وَرِضْوَٟنٌۭ مِّنَ ٱللَّهِ أَكْبَرُ ۚ ذَٟلِكَ هُوَ ٱلْفَوْزُ ٱلْعَظِيمُ ﴿٧٢﴾}\\
73.\  & \mytextarabic{يَـٰٓأَيُّهَا ٱلنَّبِىُّ جَٰهِدِ ٱلْكُفَّارَ وَٱلْمُنَـٰفِقِينَ وَٱغْلُظْ عَلَيْهِمْ ۚ وَمَأْوَىٰهُمْ جَهَنَّمُ ۖ وَبِئْسَ ٱلْمَصِيرُ ﴿٧٣﴾}\\
74.\  & \mytextarabic{يَحْلِفُونَ بِٱللَّهِ مَا قَالُوا۟ وَلَقَدْ قَالُوا۟ كَلِمَةَ ٱلْكُفْرِ وَكَفَرُوا۟ بَعْدَ إِسْلَـٰمِهِمْ وَهَمُّوا۟ بِمَا لَمْ يَنَالُوا۟ ۚ وَمَا نَقَمُوٓا۟ إِلَّآ أَنْ أَغْنَىٰهُمُ ٱللَّهُ وَرَسُولُهُۥ مِن فَضْلِهِۦ ۚ فَإِن يَتُوبُوا۟ يَكُ خَيْرًۭا لَّهُمْ ۖ وَإِن يَتَوَلَّوْا۟ يُعَذِّبْهُمُ ٱللَّهُ عَذَابًا أَلِيمًۭا فِى ٱلدُّنْيَا وَٱلْءَاخِرَةِ ۚ وَمَا لَهُمْ فِى ٱلْأَرْضِ مِن وَلِىٍّۢ وَلَا نَصِيرٍۢ ﴿٧٤﴾}\\
75.\  & \mytextarabic{۞ وَمِنْهُم مَّنْ عَـٰهَدَ ٱللَّهَ لَئِنْ ءَاتَىٰنَا مِن فَضْلِهِۦ لَنَصَّدَّقَنَّ وَلَنَكُونَنَّ مِنَ ٱلصَّـٰلِحِينَ ﴿٧٥﴾}\\
76.\  & \mytextarabic{فَلَمَّآ ءَاتَىٰهُم مِّن فَضْلِهِۦ بَخِلُوا۟ بِهِۦ وَتَوَلَّوا۟ وَّهُم مُّعْرِضُونَ ﴿٧٦﴾}\\
77.\  & \mytextarabic{فَأَعْقَبَهُمْ نِفَاقًۭا فِى قُلُوبِهِمْ إِلَىٰ يَوْمِ يَلْقَوْنَهُۥ بِمَآ أَخْلَفُوا۟ ٱللَّهَ مَا وَعَدُوهُ وَبِمَا كَانُوا۟ يَكْذِبُونَ ﴿٧٧﴾}\\
78.\  & \mytextarabic{أَلَمْ يَعْلَمُوٓا۟ أَنَّ ٱللَّهَ يَعْلَمُ سِرَّهُمْ وَنَجْوَىٰهُمْ وَأَنَّ ٱللَّهَ عَلَّٰمُ ٱلْغُيُوبِ ﴿٧٨﴾}\\
79.\  & \mytextarabic{ٱلَّذِينَ يَلْمِزُونَ ٱلْمُطَّوِّعِينَ مِنَ ٱلْمُؤْمِنِينَ فِى ٱلصَّدَقَـٰتِ وَٱلَّذِينَ لَا يَجِدُونَ إِلَّا جُهْدَهُمْ فَيَسْخَرُونَ مِنْهُمْ ۙ سَخِرَ ٱللَّهُ مِنْهُمْ وَلَهُمْ عَذَابٌ أَلِيمٌ ﴿٧٩﴾}\\
80.\  & \mytextarabic{ٱسْتَغْفِرْ لَهُمْ أَوْ لَا تَسْتَغْفِرْ لَهُمْ إِن تَسْتَغْفِرْ لَهُمْ سَبْعِينَ مَرَّةًۭ فَلَن يَغْفِرَ ٱللَّهُ لَهُمْ ۚ ذَٟلِكَ بِأَنَّهُمْ كَفَرُوا۟ بِٱللَّهِ وَرَسُولِهِۦ ۗ وَٱللَّهُ لَا يَهْدِى ٱلْقَوْمَ ٱلْفَـٰسِقِينَ ﴿٨٠﴾}\\
81.\  & \mytextarabic{فَرِحَ ٱلْمُخَلَّفُونَ بِمَقْعَدِهِمْ خِلَـٰفَ رَسُولِ ٱللَّهِ وَكَرِهُوٓا۟ أَن يُجَٰهِدُوا۟ بِأَمْوَٟلِهِمْ وَأَنفُسِهِمْ فِى سَبِيلِ ٱللَّهِ وَقَالُوا۟ لَا تَنفِرُوا۟ فِى ٱلْحَرِّ ۗ قُلْ نَارُ جَهَنَّمَ أَشَدُّ حَرًّۭا ۚ لَّوْ كَانُوا۟ يَفْقَهُونَ ﴿٨١﴾}\\
82.\  & \mytextarabic{فَلْيَضْحَكُوا۟ قَلِيلًۭا وَلْيَبْكُوا۟ كَثِيرًۭا جَزَآءًۢ بِمَا كَانُوا۟ يَكْسِبُونَ ﴿٨٢﴾}\\
83.\  & \mytextarabic{فَإِن رَّجَعَكَ ٱللَّهُ إِلَىٰ طَآئِفَةٍۢ مِّنْهُمْ فَٱسْتَـْٔذَنُوكَ لِلْخُرُوجِ فَقُل لَّن تَخْرُجُوا۟ مَعِىَ أَبَدًۭا وَلَن تُقَـٰتِلُوا۟ مَعِىَ عَدُوًّا ۖ إِنَّكُمْ رَضِيتُم بِٱلْقُعُودِ أَوَّلَ مَرَّةٍۢ فَٱقْعُدُوا۟ مَعَ ٱلْخَـٰلِفِينَ ﴿٨٣﴾}\\
84.\  & \mytextarabic{وَلَا تُصَلِّ عَلَىٰٓ أَحَدٍۢ مِّنْهُم مَّاتَ أَبَدًۭا وَلَا تَقُمْ عَلَىٰ قَبْرِهِۦٓ ۖ إِنَّهُمْ كَفَرُوا۟ بِٱللَّهِ وَرَسُولِهِۦ وَمَاتُوا۟ وَهُمْ فَـٰسِقُونَ ﴿٨٤﴾}\\
85.\  & \mytextarabic{وَلَا تُعْجِبْكَ أَمْوَٟلُهُمْ وَأَوْلَـٰدُهُمْ ۚ إِنَّمَا يُرِيدُ ٱللَّهُ أَن يُعَذِّبَهُم بِهَا فِى ٱلدُّنْيَا وَتَزْهَقَ أَنفُسُهُمْ وَهُمْ كَـٰفِرُونَ ﴿٨٥﴾}\\
86.\  & \mytextarabic{وَإِذَآ أُنزِلَتْ سُورَةٌ أَنْ ءَامِنُوا۟ بِٱللَّهِ وَجَٰهِدُوا۟ مَعَ رَسُولِهِ ٱسْتَـْٔذَنَكَ أُو۟لُوا۟ ٱلطَّوْلِ مِنْهُمْ وَقَالُوا۟ ذَرْنَا نَكُن مَّعَ ٱلْقَـٰعِدِينَ ﴿٨٦﴾}\\
87.\  & \mytextarabic{رَضُوا۟ بِأَن يَكُونُوا۟ مَعَ ٱلْخَوَالِفِ وَطُبِعَ عَلَىٰ قُلُوبِهِمْ فَهُمْ لَا يَفْقَهُونَ ﴿٨٧﴾}\\
88.\  & \mytextarabic{لَـٰكِنِ ٱلرَّسُولُ وَٱلَّذِينَ ءَامَنُوا۟ مَعَهُۥ جَٰهَدُوا۟ بِأَمْوَٟلِهِمْ وَأَنفُسِهِمْ ۚ وَأُو۟لَـٰٓئِكَ لَهُمُ ٱلْخَيْرَٰتُ ۖ وَأُو۟لَـٰٓئِكَ هُمُ ٱلْمُفْلِحُونَ ﴿٨٨﴾}\\
89.\  & \mytextarabic{أَعَدَّ ٱللَّهُ لَهُمْ جَنَّـٰتٍۢ تَجْرِى مِن تَحْتِهَا ٱلْأَنْهَـٰرُ خَـٰلِدِينَ فِيهَا ۚ ذَٟلِكَ ٱلْفَوْزُ ٱلْعَظِيمُ ﴿٨٩﴾}\\
90.\  & \mytextarabic{وَجَآءَ ٱلْمُعَذِّرُونَ مِنَ ٱلْأَعْرَابِ لِيُؤْذَنَ لَهُمْ وَقَعَدَ ٱلَّذِينَ كَذَبُوا۟ ٱللَّهَ وَرَسُولَهُۥ ۚ سَيُصِيبُ ٱلَّذِينَ كَفَرُوا۟ مِنْهُمْ عَذَابٌ أَلِيمٌۭ ﴿٩٠﴾}\\
91.\  & \mytextarabic{لَّيْسَ عَلَى ٱلضُّعَفَآءِ وَلَا عَلَى ٱلْمَرْضَىٰ وَلَا عَلَى ٱلَّذِينَ لَا يَجِدُونَ مَا يُنفِقُونَ حَرَجٌ إِذَا نَصَحُوا۟ لِلَّهِ وَرَسُولِهِۦ ۚ مَا عَلَى ٱلْمُحْسِنِينَ مِن سَبِيلٍۢ ۚ وَٱللَّهُ غَفُورٌۭ رَّحِيمٌۭ ﴿٩١﴾}\\
92.\  & \mytextarabic{وَلَا عَلَى ٱلَّذِينَ إِذَا مَآ أَتَوْكَ لِتَحْمِلَهُمْ قُلْتَ لَآ أَجِدُ مَآ أَحْمِلُكُمْ عَلَيْهِ تَوَلَّوا۟ وَّأَعْيُنُهُمْ تَفِيضُ مِنَ ٱلدَّمْعِ حَزَنًا أَلَّا يَجِدُوا۟ مَا يُنفِقُونَ ﴿٩٢﴾}\\
93.\  & \mytextarabic{۞ إِنَّمَا ٱلسَّبِيلُ عَلَى ٱلَّذِينَ يَسْتَـْٔذِنُونَكَ وَهُمْ أَغْنِيَآءُ ۚ رَضُوا۟ بِأَن يَكُونُوا۟ مَعَ ٱلْخَوَالِفِ وَطَبَعَ ٱللَّهُ عَلَىٰ قُلُوبِهِمْ فَهُمْ لَا يَعْلَمُونَ ﴿٩٣﴾}\\
94.\  & \mytextarabic{يَعْتَذِرُونَ إِلَيْكُمْ إِذَا رَجَعْتُمْ إِلَيْهِمْ ۚ قُل لَّا تَعْتَذِرُوا۟ لَن نُّؤْمِنَ لَكُمْ قَدْ نَبَّأَنَا ٱللَّهُ مِنْ أَخْبَارِكُمْ ۚ وَسَيَرَى ٱللَّهُ عَمَلَكُمْ وَرَسُولُهُۥ ثُمَّ تُرَدُّونَ إِلَىٰ عَـٰلِمِ ٱلْغَيْبِ وَٱلشَّهَـٰدَةِ فَيُنَبِّئُكُم بِمَا كُنتُمْ تَعْمَلُونَ ﴿٩٤﴾}\\
95.\  & \mytextarabic{سَيَحْلِفُونَ بِٱللَّهِ لَكُمْ إِذَا ٱنقَلَبْتُمْ إِلَيْهِمْ لِتُعْرِضُوا۟ عَنْهُمْ ۖ فَأَعْرِضُوا۟ عَنْهُمْ ۖ إِنَّهُمْ رِجْسٌۭ ۖ وَمَأْوَىٰهُمْ جَهَنَّمُ جَزَآءًۢ بِمَا كَانُوا۟ يَكْسِبُونَ ﴿٩٥﴾}\\
96.\  & \mytextarabic{يَحْلِفُونَ لَكُمْ لِتَرْضَوْا۟ عَنْهُمْ ۖ فَإِن تَرْضَوْا۟ عَنْهُمْ فَإِنَّ ٱللَّهَ لَا يَرْضَىٰ عَنِ ٱلْقَوْمِ ٱلْفَـٰسِقِينَ ﴿٩٦﴾}\\
97.\  & \mytextarabic{ٱلْأَعْرَابُ أَشَدُّ كُفْرًۭا وَنِفَاقًۭا وَأَجْدَرُ أَلَّا يَعْلَمُوا۟ حُدُودَ مَآ أَنزَلَ ٱللَّهُ عَلَىٰ رَسُولِهِۦ ۗ وَٱللَّهُ عَلِيمٌ حَكِيمٌۭ ﴿٩٧﴾}\\
98.\  & \mytextarabic{وَمِنَ ٱلْأَعْرَابِ مَن يَتَّخِذُ مَا يُنفِقُ مَغْرَمًۭا وَيَتَرَبَّصُ بِكُمُ ٱلدَّوَآئِرَ ۚ عَلَيْهِمْ دَآئِرَةُ ٱلسَّوْءِ ۗ وَٱللَّهُ سَمِيعٌ عَلِيمٌۭ ﴿٩٨﴾}\\
99.\  & \mytextarabic{وَمِنَ ٱلْأَعْرَابِ مَن يُؤْمِنُ بِٱللَّهِ وَٱلْيَوْمِ ٱلْءَاخِرِ وَيَتَّخِذُ مَا يُنفِقُ قُرُبَٰتٍ عِندَ ٱللَّهِ وَصَلَوَٟتِ ٱلرَّسُولِ ۚ أَلَآ إِنَّهَا قُرْبَةٌۭ لَّهُمْ ۚ سَيُدْخِلُهُمُ ٱللَّهُ فِى رَحْمَتِهِۦٓ ۗ إِنَّ ٱللَّهَ غَفُورٌۭ رَّحِيمٌۭ ﴿٩٩﴾}\\
100.\  & \mytextarabic{وَٱلسَّٰبِقُونَ ٱلْأَوَّلُونَ مِنَ ٱلْمُهَـٰجِرِينَ وَٱلْأَنصَارِ وَٱلَّذِينَ ٱتَّبَعُوهُم بِإِحْسَـٰنٍۢ رَّضِىَ ٱللَّهُ عَنْهُمْ وَرَضُوا۟ عَنْهُ وَأَعَدَّ لَهُمْ جَنَّـٰتٍۢ تَجْرِى تَحْتَهَا ٱلْأَنْهَـٰرُ خَـٰلِدِينَ فِيهَآ أَبَدًۭا ۚ ذَٟلِكَ ٱلْفَوْزُ ٱلْعَظِيمُ ﴿١٠٠﴾}\\
101.\  & \mytextarabic{وَمِمَّنْ حَوْلَكُم مِّنَ ٱلْأَعْرَابِ مُنَـٰفِقُونَ ۖ وَمِنْ أَهْلِ ٱلْمَدِينَةِ ۖ مَرَدُوا۟ عَلَى ٱلنِّفَاقِ لَا تَعْلَمُهُمْ ۖ نَحْنُ نَعْلَمُهُمْ ۚ سَنُعَذِّبُهُم مَّرَّتَيْنِ ثُمَّ يُرَدُّونَ إِلَىٰ عَذَابٍ عَظِيمٍۢ ﴿١٠١﴾}\\
102.\  & \mytextarabic{وَءَاخَرُونَ ٱعْتَرَفُوا۟ بِذُنُوبِهِمْ خَلَطُوا۟ عَمَلًۭا صَـٰلِحًۭا وَءَاخَرَ سَيِّئًا عَسَى ٱللَّهُ أَن يَتُوبَ عَلَيْهِمْ ۚ إِنَّ ٱللَّهَ غَفُورٌۭ رَّحِيمٌ ﴿١٠٢﴾}\\
103.\  & \mytextarabic{خُذْ مِنْ أَمْوَٟلِهِمْ صَدَقَةًۭ تُطَهِّرُهُمْ وَتُزَكِّيهِم بِهَا وَصَلِّ عَلَيْهِمْ ۖ إِنَّ صَلَوٰتَكَ سَكَنٌۭ لَّهُمْ ۗ وَٱللَّهُ سَمِيعٌ عَلِيمٌ ﴿١٠٣﴾}\\
104.\  & \mytextarabic{أَلَمْ يَعْلَمُوٓا۟ أَنَّ ٱللَّهَ هُوَ يَقْبَلُ ٱلتَّوْبَةَ عَنْ عِبَادِهِۦ وَيَأْخُذُ ٱلصَّدَقَـٰتِ وَأَنَّ ٱللَّهَ هُوَ ٱلتَّوَّابُ ٱلرَّحِيمُ ﴿١٠٤﴾}\\
105.\  & \mytextarabic{وَقُلِ ٱعْمَلُوا۟ فَسَيَرَى ٱللَّهُ عَمَلَكُمْ وَرَسُولُهُۥ وَٱلْمُؤْمِنُونَ ۖ وَسَتُرَدُّونَ إِلَىٰ عَـٰلِمِ ٱلْغَيْبِ وَٱلشَّهَـٰدَةِ فَيُنَبِّئُكُم بِمَا كُنتُمْ تَعْمَلُونَ ﴿١٠٥﴾}\\
106.\  & \mytextarabic{وَءَاخَرُونَ مُرْجَوْنَ لِأَمْرِ ٱللَّهِ إِمَّا يُعَذِّبُهُمْ وَإِمَّا يَتُوبُ عَلَيْهِمْ ۗ وَٱللَّهُ عَلِيمٌ حَكِيمٌۭ ﴿١٠٦﴾}\\
107.\  & \mytextarabic{وَٱلَّذِينَ ٱتَّخَذُوا۟ مَسْجِدًۭا ضِرَارًۭا وَكُفْرًۭا وَتَفْرِيقًۢا بَيْنَ ٱلْمُؤْمِنِينَ وَإِرْصَادًۭا لِّمَنْ حَارَبَ ٱللَّهَ وَرَسُولَهُۥ مِن قَبْلُ ۚ وَلَيَحْلِفُنَّ إِنْ أَرَدْنَآ إِلَّا ٱلْحُسْنَىٰ ۖ وَٱللَّهُ يَشْهَدُ إِنَّهُمْ لَكَـٰذِبُونَ ﴿١٠٧﴾}\\
108.\  & \mytextarabic{لَا تَقُمْ فِيهِ أَبَدًۭا ۚ لَّمَسْجِدٌ أُسِّسَ عَلَى ٱلتَّقْوَىٰ مِنْ أَوَّلِ يَوْمٍ أَحَقُّ أَن تَقُومَ فِيهِ ۚ فِيهِ رِجَالٌۭ يُحِبُّونَ أَن يَتَطَهَّرُوا۟ ۚ وَٱللَّهُ يُحِبُّ ٱلْمُطَّهِّرِينَ ﴿١٠٨﴾}\\
109.\  & \mytextarabic{أَفَمَنْ أَسَّسَ بُنْيَـٰنَهُۥ عَلَىٰ تَقْوَىٰ مِنَ ٱللَّهِ وَرِضْوَٟنٍ خَيْرٌ أَم مَّنْ أَسَّسَ بُنْيَـٰنَهُۥ عَلَىٰ شَفَا جُرُفٍ هَارٍۢ فَٱنْهَارَ بِهِۦ فِى نَارِ جَهَنَّمَ ۗ وَٱللَّهُ لَا يَهْدِى ٱلْقَوْمَ ٱلظَّـٰلِمِينَ ﴿١٠٩﴾}\\
110.\  & \mytextarabic{لَا يَزَالُ بُنْيَـٰنُهُمُ ٱلَّذِى بَنَوْا۟ رِيبَةًۭ فِى قُلُوبِهِمْ إِلَّآ أَن تَقَطَّعَ قُلُوبُهُمْ ۗ وَٱللَّهُ عَلِيمٌ حَكِيمٌ ﴿١١٠﴾}\\
111.\  & \mytextarabic{۞ إِنَّ ٱللَّهَ ٱشْتَرَىٰ مِنَ ٱلْمُؤْمِنِينَ أَنفُسَهُمْ وَأَمْوَٟلَهُم بِأَنَّ لَهُمُ ٱلْجَنَّةَ ۚ يُقَـٰتِلُونَ فِى سَبِيلِ ٱللَّهِ فَيَقْتُلُونَ وَيُقْتَلُونَ ۖ وَعْدًا عَلَيْهِ حَقًّۭا فِى ٱلتَّوْرَىٰةِ وَٱلْإِنجِيلِ وَٱلْقُرْءَانِ ۚ وَمَنْ أَوْفَىٰ بِعَهْدِهِۦ مِنَ ٱللَّهِ ۚ فَٱسْتَبْشِرُوا۟ بِبَيْعِكُمُ ٱلَّذِى بَايَعْتُم بِهِۦ ۚ وَذَٟلِكَ هُوَ ٱلْفَوْزُ ٱلْعَظِيمُ ﴿١١١﴾}\\
112.\  & \mytextarabic{ٱلتَّٰٓئِبُونَ ٱلْعَـٰبِدُونَ ٱلْحَـٰمِدُونَ ٱلسَّٰٓئِحُونَ ٱلرَّٟكِعُونَ ٱلسَّٰجِدُونَ ٱلْءَامِرُونَ بِٱلْمَعْرُوفِ وَٱلنَّاهُونَ عَنِ ٱلْمُنكَرِ وَٱلْحَـٰفِظُونَ لِحُدُودِ ٱللَّهِ ۗ وَبَشِّرِ ٱلْمُؤْمِنِينَ ﴿١١٢﴾}\\
113.\  & \mytextarabic{مَا كَانَ لِلنَّبِىِّ وَٱلَّذِينَ ءَامَنُوٓا۟ أَن يَسْتَغْفِرُوا۟ لِلْمُشْرِكِينَ وَلَوْ كَانُوٓا۟ أُو۟لِى قُرْبَىٰ مِنۢ بَعْدِ مَا تَبَيَّنَ لَهُمْ أَنَّهُمْ أَصْحَـٰبُ ٱلْجَحِيمِ ﴿١١٣﴾}\\
114.\  & \mytextarabic{وَمَا كَانَ ٱسْتِغْفَارُ إِبْرَٰهِيمَ لِأَبِيهِ إِلَّا عَن مَّوْعِدَةٍۢ وَعَدَهَآ إِيَّاهُ فَلَمَّا تَبَيَّنَ لَهُۥٓ أَنَّهُۥ عَدُوٌّۭ لِّلَّهِ تَبَرَّأَ مِنْهُ ۚ إِنَّ إِبْرَٰهِيمَ لَأَوَّٰهٌ حَلِيمٌۭ ﴿١١٤﴾}\\
115.\  & \mytextarabic{وَمَا كَانَ ٱللَّهُ لِيُضِلَّ قَوْمًۢا بَعْدَ إِذْ هَدَىٰهُمْ حَتَّىٰ يُبَيِّنَ لَهُم مَّا يَتَّقُونَ ۚ إِنَّ ٱللَّهَ بِكُلِّ شَىْءٍ عَلِيمٌ ﴿١١٥﴾}\\
116.\  & \mytextarabic{إِنَّ ٱللَّهَ لَهُۥ مُلْكُ ٱلسَّمَـٰوَٟتِ وَٱلْأَرْضِ ۖ يُحْىِۦ وَيُمِيتُ ۚ وَمَا لَكُم مِّن دُونِ ٱللَّهِ مِن وَلِىٍّۢ وَلَا نَصِيرٍۢ ﴿١١٦﴾}\\
117.\  & \mytextarabic{لَّقَد تَّابَ ٱللَّهُ عَلَى ٱلنَّبِىِّ وَٱلْمُهَـٰجِرِينَ وَٱلْأَنصَارِ ٱلَّذِينَ ٱتَّبَعُوهُ فِى سَاعَةِ ٱلْعُسْرَةِ مِنۢ بَعْدِ مَا كَادَ يَزِيغُ قُلُوبُ فَرِيقٍۢ مِّنْهُمْ ثُمَّ تَابَ عَلَيْهِمْ ۚ إِنَّهُۥ بِهِمْ رَءُوفٌۭ رَّحِيمٌۭ ﴿١١٧﴾}\\
118.\  & \mytextarabic{وَعَلَى ٱلثَّلَـٰثَةِ ٱلَّذِينَ خُلِّفُوا۟ حَتَّىٰٓ إِذَا ضَاقَتْ عَلَيْهِمُ ٱلْأَرْضُ بِمَا رَحُبَتْ وَضَاقَتْ عَلَيْهِمْ أَنفُسُهُمْ وَظَنُّوٓا۟ أَن لَّا مَلْجَأَ مِنَ ٱللَّهِ إِلَّآ إِلَيْهِ ثُمَّ تَابَ عَلَيْهِمْ لِيَتُوبُوٓا۟ ۚ إِنَّ ٱللَّهَ هُوَ ٱلتَّوَّابُ ٱلرَّحِيمُ ﴿١١٨﴾}\\
119.\  & \mytextarabic{يَـٰٓأَيُّهَا ٱلَّذِينَ ءَامَنُوا۟ ٱتَّقُوا۟ ٱللَّهَ وَكُونُوا۟ مَعَ ٱلصَّـٰدِقِينَ ﴿١١٩﴾}\\
120.\  & \mytextarabic{مَا كَانَ لِأَهْلِ ٱلْمَدِينَةِ وَمَنْ حَوْلَهُم مِّنَ ٱلْأَعْرَابِ أَن يَتَخَلَّفُوا۟ عَن رَّسُولِ ٱللَّهِ وَلَا يَرْغَبُوا۟ بِأَنفُسِهِمْ عَن نَّفْسِهِۦ ۚ ذَٟلِكَ بِأَنَّهُمْ لَا يُصِيبُهُمْ ظَمَأٌۭ وَلَا نَصَبٌۭ وَلَا مَخْمَصَةٌۭ فِى سَبِيلِ ٱللَّهِ وَلَا يَطَـُٔونَ مَوْطِئًۭا يَغِيظُ ٱلْكُفَّارَ وَلَا يَنَالُونَ مِنْ عَدُوٍّۢ نَّيْلًا إِلَّا كُتِبَ لَهُم بِهِۦ عَمَلٌۭ صَـٰلِحٌ ۚ إِنَّ ٱللَّهَ لَا يُضِيعُ أَجْرَ ٱلْمُحْسِنِينَ ﴿١٢٠﴾}\\
121.\  & \mytextarabic{وَلَا يُنفِقُونَ نَفَقَةًۭ صَغِيرَةًۭ وَلَا كَبِيرَةًۭ وَلَا يَقْطَعُونَ وَادِيًا إِلَّا كُتِبَ لَهُمْ لِيَجْزِيَهُمُ ٱللَّهُ أَحْسَنَ مَا كَانُوا۟ يَعْمَلُونَ ﴿١٢١﴾}\\
122.\  & \mytextarabic{۞ وَمَا كَانَ ٱلْمُؤْمِنُونَ لِيَنفِرُوا۟ كَآفَّةًۭ ۚ فَلَوْلَا نَفَرَ مِن كُلِّ فِرْقَةٍۢ مِّنْهُمْ طَآئِفَةٌۭ لِّيَتَفَقَّهُوا۟ فِى ٱلدِّينِ وَلِيُنذِرُوا۟ قَوْمَهُمْ إِذَا رَجَعُوٓا۟ إِلَيْهِمْ لَعَلَّهُمْ يَحْذَرُونَ ﴿١٢٢﴾}\\
123.\  & \mytextarabic{يَـٰٓأَيُّهَا ٱلَّذِينَ ءَامَنُوا۟ قَـٰتِلُوا۟ ٱلَّذِينَ يَلُونَكُم مِّنَ ٱلْكُفَّارِ وَلْيَجِدُوا۟ فِيكُمْ غِلْظَةًۭ ۚ وَٱعْلَمُوٓا۟ أَنَّ ٱللَّهَ مَعَ ٱلْمُتَّقِينَ ﴿١٢٣﴾}\\
124.\  & \mytextarabic{وَإِذَا مَآ أُنزِلَتْ سُورَةٌۭ فَمِنْهُم مَّن يَقُولُ أَيُّكُمْ زَادَتْهُ هَـٰذِهِۦٓ إِيمَـٰنًۭا ۚ فَأَمَّا ٱلَّذِينَ ءَامَنُوا۟ فَزَادَتْهُمْ إِيمَـٰنًۭا وَهُمْ يَسْتَبْشِرُونَ ﴿١٢٤﴾}\\
125.\  & \mytextarabic{وَأَمَّا ٱلَّذِينَ فِى قُلُوبِهِم مَّرَضٌۭ فَزَادَتْهُمْ رِجْسًا إِلَىٰ رِجْسِهِمْ وَمَاتُوا۟ وَهُمْ كَـٰفِرُونَ ﴿١٢٥﴾}\\
126.\  & \mytextarabic{أَوَلَا يَرَوْنَ أَنَّهُمْ يُفْتَنُونَ فِى كُلِّ عَامٍۢ مَّرَّةً أَوْ مَرَّتَيْنِ ثُمَّ لَا يَتُوبُونَ وَلَا هُمْ يَذَّكَّرُونَ ﴿١٢٦﴾}\\
127.\  & \mytextarabic{وَإِذَا مَآ أُنزِلَتْ سُورَةٌۭ نَّظَرَ بَعْضُهُمْ إِلَىٰ بَعْضٍ هَلْ يَرَىٰكُم مِّنْ أَحَدٍۢ ثُمَّ ٱنصَرَفُوا۟ ۚ صَرَفَ ٱللَّهُ قُلُوبَهُم بِأَنَّهُمْ قَوْمٌۭ لَّا يَفْقَهُونَ ﴿١٢٧﴾}\\
128.\  & \mytextarabic{لَقَدْ جَآءَكُمْ رَسُولٌۭ مِّنْ أَنفُسِكُمْ عَزِيزٌ عَلَيْهِ مَا عَنِتُّمْ حَرِيصٌ عَلَيْكُم بِٱلْمُؤْمِنِينَ رَءُوفٌۭ رَّحِيمٌۭ ﴿١٢٨﴾}\\
129.\  & \mytextarabic{فَإِن تَوَلَّوْا۟ فَقُلْ حَسْبِىَ ٱللَّهُ لَآ إِلَـٰهَ إِلَّا هُوَ ۖ عَلَيْهِ تَوَكَّلْتُ ۖ وَهُوَ رَبُّ ٱلْعَرْشِ ٱلْعَظِيمِ ﴿١٢٩﴾}\\
\end{longtable}
\clearpage