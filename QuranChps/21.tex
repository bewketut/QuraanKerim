%% License: BSD style (Berkley) (i.e. Put the Copyright owner's name always)
%% Writer and Copyright (to): Bewketu(Bilal) Tadilo (2016-17)
\shadowbox{\section{\LR{\textamharic{ሱራቱ አልአንቢያ -}  \RL{سوره  الأنبياء}}}}

  
    
  
    
    

\nopagebreak
  بِسمِ ٱللَّهِ ٱلرَّحمَـٰنِ ٱلرَّحِيمِ
  ٱقتَرَبَ لِلنَّاسِ حِسَابُهُم وَهُم فِى غَفلَةٍۢ مُّعرِضُونَ ﴿١﴾
 مَا يَأتِيهِم مِّن ذِكرٍۢ مِّن رَّبِّهِم مُّحدَثٍ إِلَّا ٱستَمَعُوهُ وَهُم يَلعَبُونَ ﴿٢﴾
 لَاهِيَةًۭ قُلُوبُهُم ۗ وَأَسَرُّوا۟ ٱلنَّجوَى ٱلَّذِينَ ظَلَمُوا۟ هَل هَـٰذَآ إِلَّا بَشَرٌۭ مِّثلُكُم ۖ أَفَتَأتُونَ ٱلسِّحرَ وَأَنتُم تُبصِرُونَ ﴿٣﴾
 قَالَ رَبِّى يَعلَمُ ٱلقَولَ فِى ٱلسَّمَآءِ وَٱلأَرضِ ۖ وَهُوَ ٱلسَّمِيعُ ٱلعَلِيمُ ﴿٤﴾
 بَل قَالُوٓا۟ أَضغَٰثُ أَحلَـٰمٍۭ بَلِ ٱفتَرَىٰهُ بَل هُوَ شَاعِرٌۭ فَليَأتِنَا بِـَٔايَةٍۢ كَمَآ أُرسِلَ ٱلأَوَّلُونَ ﴿٥﴾
 مَآ ءَامَنَت قَبلَهُم مِّن قَريَةٍ أَهلَكنَـٰهَآ ۖ أَفَهُم يُؤمِنُونَ ﴿٦﴾
 وَمَآ أَرسَلنَا قَبلَكَ إِلَّا رِجَالًۭا نُّوحِىٓ إِلَيهِم ۖ فَسـَٔلُوٓا۟ أَهلَ ٱلذِّكرِ إِن كُنتُم لَا تَعلَمُونَ ﴿٧﴾
 وَمَا جَعَلنَـٰهُم جَسَدًۭا لَّا يَأكُلُونَ ٱلطَّعَامَ وَمَا كَانُوا۟ خَـٰلِدِينَ ﴿٨﴾
 ثُمَّ صَدَقنَـٰهُمُ ٱلوَعدَ فَأَنجَينَـٰهُم وَمَن نَّشَآءُ وَأَهلَكنَا ٱلمُسرِفِينَ ﴿٩﴾
 لَقَد أَنزَلنَآ إِلَيكُم كِتَـٰبًۭا فِيهِ ذِكرُكُم ۖ أَفَلَا تَعقِلُونَ ﴿١٠﴾
 وَكَم قَصَمنَا مِن قَريَةٍۢ كَانَت ظَالِمَةًۭ وَأَنشَأنَا بَعدَهَا قَومًا ءَاخَرِينَ ﴿١١﴾
 فَلَمَّآ أَحَسُّوا۟ بَأسَنَآ إِذَا هُم مِّنهَا يَركُضُونَ ﴿١٢﴾
 لَا تَركُضُوا۟ وَٱرجِعُوٓا۟ إِلَىٰ مَآ أُترِفتُم فِيهِ وَمَسَـٰكِنِكُم لَعَلَّكُم تُسـَٔلُونَ ﴿١٣﴾
 قَالُوا۟ يَـٰوَيلَنَآ إِنَّا كُنَّا ظَـٰلِمِينَ ﴿١٤﴾
 فَمَا زَالَت تِّلكَ دَعوَىٰهُم حَتَّىٰ جَعَلنَـٰهُم حَصِيدًا خَـٰمِدِينَ ﴿١٥﴾
 وَمَا خَلَقنَا ٱلسَّمَآءَ وَٱلأَرضَ وَمَا بَينَهُمَا لَـٰعِبِينَ ﴿١٦﴾
 لَو أَرَدنَآ أَن نَّتَّخِذَ لَهوًۭا لَّٱتَّخَذنَـٰهُ مِن لَّدُنَّآ إِن كُنَّا فَـٰعِلِينَ ﴿١٧﴾
 بَل نَقذِفُ بِٱلحَقِّ عَلَى ٱلبَٰطِلِ فَيَدمَغُهُۥ فَإِذَا هُوَ زَاهِقٌۭ ۚ وَلَكُمُ ٱلوَيلُ مِمَّا تَصِفُونَ ﴿١٨﴾
 وَلَهُۥ مَن فِى ٱلسَّمَـٰوَٟتِ وَٱلأَرضِ ۚ وَمَن عِندَهُۥ لَا يَستَكبِرُونَ عَن عِبَادَتِهِۦ وَلَا يَستَحسِرُونَ ﴿١٩﴾
 يُسَبِّحُونَ ٱلَّيلَ وَٱلنَّهَارَ لَا يَفتُرُونَ ﴿٢٠﴾
 أَمِ ٱتَّخَذُوٓا۟ ءَالِهَةًۭ مِّنَ ٱلأَرضِ هُم يُنشِرُونَ ﴿٢١﴾
 لَو كَانَ فِيهِمَآ ءَالِهَةٌ إِلَّا ٱللَّهُ لَفَسَدَتَا ۚ فَسُبحَـٰنَ ٱللَّهِ رَبِّ ٱلعَرشِ عَمَّا يَصِفُونَ ﴿٢٢﴾
 لَا يُسـَٔلُ عَمَّا يَفعَلُ وَهُم يُسـَٔلُونَ ﴿٢٣﴾
 أَمِ ٱتَّخَذُوا۟ مِن دُونِهِۦٓ ءَالِهَةًۭ ۖ قُل هَاتُوا۟ بُرهَـٰنَكُم ۖ هَـٰذَا ذِكرُ مَن مَّعِىَ وَذِكرُ مَن قَبلِى ۗ بَل أَكثَرُهُم لَا يَعلَمُونَ ٱلحَقَّ ۖ فَهُم مُّعرِضُونَ ﴿٢٤﴾
 وَمَآ أَرسَلنَا مِن قَبلِكَ مِن رَّسُولٍ إِلَّا نُوحِىٓ إِلَيهِ أَنَّهُۥ لَآ إِلَـٰهَ إِلَّآ أَنَا۠ فَٱعبُدُونِ ﴿٢٥﴾
 وَقَالُوا۟ ٱتَّخَذَ ٱلرَّحمَـٰنُ وَلَدًۭا ۗ سُبحَـٰنَهُۥ ۚ بَل عِبَادٌۭ مُّكرَمُونَ ﴿٢٦﴾
 لَا يَسبِقُونَهُۥ بِٱلقَولِ وَهُم بِأَمرِهِۦ يَعمَلُونَ ﴿٢٧﴾
 يَعلَمُ مَا بَينَ أَيدِيهِم وَمَا خَلفَهُم وَلَا يَشفَعُونَ إِلَّا لِمَنِ ٱرتَضَىٰ وَهُم مِّن خَشيَتِهِۦ مُشفِقُونَ ﴿٢٨﴾
 ۞ وَمَن يَقُل مِنهُم إِنِّىٓ إِلَـٰهٌۭ مِّن دُونِهِۦ فَذَٟلِكَ نَجزِيهِ جَهَنَّمَ ۚ كَذَٟلِكَ نَجزِى ٱلظَّـٰلِمِينَ ﴿٢٩﴾
 أَوَلَم يَرَ ٱلَّذِينَ كَفَرُوٓا۟ أَنَّ ٱلسَّمَـٰوَٟتِ وَٱلأَرضَ كَانَتَا رَتقًۭا فَفَتَقنَـٰهُمَا ۖ وَجَعَلنَا مِنَ ٱلمَآءِ كُلَّ شَىءٍ حَىٍّ ۖ أَفَلَا يُؤمِنُونَ ﴿٣٠﴾
 وَجَعَلنَا فِى ٱلأَرضِ رَوَٟسِىَ أَن تَمِيدَ بِهِم وَجَعَلنَا فِيهَا فِجَاجًۭا سُبُلًۭا لَّعَلَّهُم يَهتَدُونَ ﴿٣١﴾
 وَجَعَلنَا ٱلسَّمَآءَ سَقفًۭا مَّحفُوظًۭا ۖ وَهُم عَن ءَايَـٰتِهَا مُعرِضُونَ ﴿٣٢﴾
 وَهُوَ ٱلَّذِى خَلَقَ ٱلَّيلَ وَٱلنَّهَارَ وَٱلشَّمسَ وَٱلقَمَرَ ۖ كُلٌّۭ فِى فَلَكٍۢ يَسبَحُونَ ﴿٣٣﴾
 وَمَا جَعَلنَا لِبَشَرٍۢ مِّن قَبلِكَ ٱلخُلدَ ۖ أَفَإِي۟ن مِّتَّ فَهُمُ ٱلخَـٰلِدُونَ ﴿٣٤﴾
 كُلُّ نَفسٍۢ ذَآئِقَةُ ٱلمَوتِ ۗ وَنَبلُوكُم بِٱلشَّرِّ وَٱلخَيرِ فِتنَةًۭ ۖ وَإِلَينَا تُرجَعُونَ ﴿٣٥﴾
 وَإِذَا رَءَاكَ ٱلَّذِينَ كَفَرُوٓا۟ إِن يَتَّخِذُونَكَ إِلَّا هُزُوًا أَهَـٰذَا ٱلَّذِى يَذكُرُ ءَالِهَتَكُم وَهُم بِذِكرِ ٱلرَّحمَـٰنِ هُم كَـٰفِرُونَ ﴿٣٦﴾
 خُلِقَ ٱلإِنسَـٰنُ مِن عَجَلٍۢ ۚ سَأُو۟رِيكُم ءَايَـٰتِى فَلَا تَستَعجِلُونِ ﴿٣٧﴾
 وَيَقُولُونَ مَتَىٰ هَـٰذَا ٱلوَعدُ إِن كُنتُم صَـٰدِقِينَ ﴿٣٨﴾
 لَو يَعلَمُ ٱلَّذِينَ كَفَرُوا۟ حِينَ لَا يَكُفُّونَ عَن وُجُوهِهِمُ ٱلنَّارَ وَلَا عَن ظُهُورِهِم وَلَا هُم يُنصَرُونَ ﴿٣٩﴾
 بَل تَأتِيهِم بَغتَةًۭ فَتَبهَتُهُم فَلَا يَستَطِيعُونَ رَدَّهَا وَلَا هُم يُنظَرُونَ ﴿٤٠﴾
 وَلَقَدِ ٱستُهزِئَ بِرُسُلٍۢ مِّن قَبلِكَ فَحَاقَ بِٱلَّذِينَ سَخِرُوا۟ مِنهُم مَّا كَانُوا۟ بِهِۦ يَستَهزِءُونَ ﴿٤١﴾
 قُل مَن يَكلَؤُكُم بِٱلَّيلِ وَٱلنَّهَارِ مِنَ ٱلرَّحمَـٰنِ ۗ بَل هُم عَن ذِكرِ رَبِّهِم مُّعرِضُونَ ﴿٤٢﴾
 أَم لَهُم ءَالِهَةٌۭ تَمنَعُهُم مِّن دُونِنَا ۚ لَا يَستَطِيعُونَ نَصرَ أَنفُسِهِم وَلَا هُم مِّنَّا يُصحَبُونَ ﴿٤٣﴾
 بَل مَتَّعنَا هَـٰٓؤُلَآءِ وَءَابَآءَهُم حَتَّىٰ طَالَ عَلَيهِمُ ٱلعُمُرُ ۗ أَفَلَا يَرَونَ أَنَّا نَأتِى ٱلأَرضَ نَنقُصُهَا مِن أَطرَافِهَآ ۚ أَفَهُمُ ٱلغَٰلِبُونَ ﴿٤٤﴾
 قُل إِنَّمَآ أُنذِرُكُم بِٱلوَحىِ ۚ وَلَا يَسمَعُ ٱلصُّمُّ ٱلدُّعَآءَ إِذَا مَا يُنذَرُونَ ﴿٤٥﴾
 وَلَئِن مَّسَّتهُم نَفحَةٌۭ مِّن عَذَابِ رَبِّكَ لَيَقُولُنَّ يَـٰوَيلَنَآ إِنَّا كُنَّا ظَـٰلِمِينَ ﴿٤٦﴾
 وَنَضَعُ ٱلمَوَٟزِينَ ٱلقِسطَ لِيَومِ ٱلقِيَـٰمَةِ فَلَا تُظلَمُ نَفسٌۭ شَيـًۭٔا ۖ وَإِن كَانَ مِثقَالَ حَبَّةٍۢ مِّن خَردَلٍ أَتَينَا بِهَا ۗ وَكَفَىٰ بِنَا حَـٰسِبِينَ ﴿٤٧﴾
 وَلَقَد ءَاتَينَا مُوسَىٰ وَهَـٰرُونَ ٱلفُرقَانَ وَضِيَآءًۭ وَذِكرًۭا لِّلمُتَّقِينَ ﴿٤٨﴾
 ٱلَّذِينَ يَخشَونَ رَبَّهُم بِٱلغَيبِ وَهُم مِّنَ ٱلسَّاعَةِ مُشفِقُونَ ﴿٤٩﴾
 وَهَـٰذَا ذِكرٌۭ مُّبَارَكٌ أَنزَلنَـٰهُ ۚ أَفَأَنتُم لَهُۥ مُنكِرُونَ ﴿٥٠﴾
 ۞ وَلَقَد ءَاتَينَآ إِبرَٰهِيمَ رُشدَهُۥ مِن قَبلُ وَكُنَّا بِهِۦ عَـٰلِمِينَ ﴿٥١﴾
 إِذ قَالَ لِأَبِيهِ وَقَومِهِۦ مَا هَـٰذِهِ ٱلتَّمَاثِيلُ ٱلَّتِىٓ أَنتُم لَهَا عَـٰكِفُونَ ﴿٥٢﴾
 قَالُوا۟ وَجَدنَآ ءَابَآءَنَا لَهَا عَـٰبِدِينَ ﴿٥٣﴾
 قَالَ لَقَد كُنتُم أَنتُم وَءَابَآؤُكُم فِى ضَلَـٰلٍۢ مُّبِينٍۢ ﴿٥٤﴾
 قَالُوٓا۟ أَجِئتَنَا بِٱلحَقِّ أَم أَنتَ مِنَ ٱللَّٰعِبِينَ ﴿٥٥﴾
 قَالَ بَل رَّبُّكُم رَبُّ ٱلسَّمَـٰوَٟتِ وَٱلأَرضِ ٱلَّذِى فَطَرَهُنَّ وَأَنَا۠ عَلَىٰ ذَٟلِكُم مِّنَ ٱلشَّـٰهِدِينَ ﴿٥٦﴾
 وَتَٱللَّهِ لَأَكِيدَنَّ أَصنَـٰمَكُم بَعدَ أَن تُوَلُّوا۟ مُدبِرِينَ ﴿٥٧﴾
 فَجَعَلَهُم جُذَٟذًا إِلَّا كَبِيرًۭا لَّهُم لَعَلَّهُم إِلَيهِ يَرجِعُونَ ﴿٥٨﴾
 قَالُوا۟ مَن فَعَلَ هَـٰذَا بِـَٔالِهَتِنَآ إِنَّهُۥ لَمِنَ ٱلظَّـٰلِمِينَ ﴿٥٩﴾
 قَالُوا۟ سَمِعنَا فَتًۭى يَذكُرُهُم يُقَالُ لَهُۥٓ إِبرَٰهِيمُ ﴿٦٠﴾
 قَالُوا۟ فَأتُوا۟ بِهِۦ عَلَىٰٓ أَعيُنِ ٱلنَّاسِ لَعَلَّهُم يَشهَدُونَ ﴿٦١﴾
 قَالُوٓا۟ ءَأَنتَ فَعَلتَ هَـٰذَا بِـَٔالِهَتِنَا يَـٰٓإِبرَٰهِيمُ ﴿٦٢﴾
 قَالَ بَل فَعَلَهُۥ كَبِيرُهُم هَـٰذَا فَسـَٔلُوهُم إِن كَانُوا۟ يَنطِقُونَ ﴿٦٣﴾
 فَرَجَعُوٓا۟ إِلَىٰٓ أَنفُسِهِم فَقَالُوٓا۟ إِنَّكُم أَنتُمُ ٱلظَّـٰلِمُونَ ﴿٦٤﴾
 ثُمَّ نُكِسُوا۟ عَلَىٰ رُءُوسِهِم لَقَد عَلِمتَ مَا هَـٰٓؤُلَآءِ يَنطِقُونَ ﴿٦٥﴾
 قَالَ أَفَتَعبُدُونَ مِن دُونِ ٱللَّهِ مَا لَا يَنفَعُكُم شَيـًۭٔا وَلَا يَضُرُّكُم ﴿٦٦﴾
 أُفٍّۢ لَّكُم وَلِمَا تَعبُدُونَ مِن دُونِ ٱللَّهِ ۖ أَفَلَا تَعقِلُونَ ﴿٦٧﴾
 قَالُوا۟ حَرِّقُوهُ وَٱنصُرُوٓا۟ ءَالِهَتَكُم إِن كُنتُم فَـٰعِلِينَ ﴿٦٨﴾
 قُلنَا يَـٰنَارُ كُونِى بَردًۭا وَسَلَـٰمًا عَلَىٰٓ إِبرَٰهِيمَ ﴿٦٩﴾
 وَأَرَادُوا۟ بِهِۦ كَيدًۭا فَجَعَلنَـٰهُمُ ٱلأَخسَرِينَ ﴿٧٠﴾
 وَنَجَّينَـٰهُ وَلُوطًا إِلَى ٱلأَرضِ ٱلَّتِى بَٰرَكنَا فِيهَا لِلعَـٰلَمِينَ ﴿٧١﴾
 وَوَهَبنَا لَهُۥٓ إِسحَـٰقَ وَيَعقُوبَ نَافِلَةًۭ ۖ وَكُلًّۭا جَعَلنَا صَـٰلِحِينَ ﴿٧٢﴾
 وَجَعَلنَـٰهُم أَئِمَّةًۭ يَهدُونَ بِأَمرِنَا وَأَوحَينَآ إِلَيهِم فِعلَ ٱلخَيرَٰتِ وَإِقَامَ ٱلصَّلَوٰةِ وَإِيتَآءَ ٱلزَّكَوٰةِ ۖ وَكَانُوا۟ لَنَا عَـٰبِدِينَ ﴿٧٣﴾
 وَلُوطًا ءَاتَينَـٰهُ حُكمًۭا وَعِلمًۭا وَنَجَّينَـٰهُ مِنَ ٱلقَريَةِ ٱلَّتِى كَانَت تَّعمَلُ ٱلخَبَٰٓئِثَ ۗ إِنَّهُم كَانُوا۟ قَومَ سَوءٍۢ فَـٰسِقِينَ ﴿٧٤﴾
 وَأَدخَلنَـٰهُ فِى رَحمَتِنَآ ۖ إِنَّهُۥ مِنَ ٱلصَّـٰلِحِينَ ﴿٧٥﴾
 وَنُوحًا إِذ نَادَىٰ مِن قَبلُ فَٱستَجَبنَا لَهُۥ فَنَجَّينَـٰهُ وَأَهلَهُۥ مِنَ ٱلكَربِ ٱلعَظِيمِ ﴿٧٦﴾
 وَنَصَرنَـٰهُ مِنَ ٱلقَومِ ٱلَّذِينَ كَذَّبُوا۟ بِـَٔايَـٰتِنَآ ۚ إِنَّهُم كَانُوا۟ قَومَ سَوءٍۢ فَأَغرَقنَـٰهُم أَجمَعِينَ ﴿٧٧﴾
 وَدَاوُۥدَ وَسُلَيمَـٰنَ إِذ يَحكُمَانِ فِى ٱلحَرثِ إِذ نَفَشَت فِيهِ غَنَمُ ٱلقَومِ وَكُنَّا لِحُكمِهِم شَـٰهِدِينَ ﴿٧٨﴾
 فَفَهَّمنَـٰهَا سُلَيمَـٰنَ ۚ وَكُلًّا ءَاتَينَا حُكمًۭا وَعِلمًۭا ۚ وَسَخَّرنَا مَعَ دَاوُۥدَ ٱلجِبَالَ يُسَبِّحنَ وَٱلطَّيرَ ۚ وَكُنَّا فَـٰعِلِينَ ﴿٧٩﴾
 وَعَلَّمنَـٰهُ صَنعَةَ لَبُوسٍۢ لَّكُم لِتُحصِنَكُم مِّنۢ بَأسِكُم ۖ فَهَل أَنتُم شَـٰكِرُونَ ﴿٨٠﴾
 وَلِسُلَيمَـٰنَ ٱلرِّيحَ عَاصِفَةًۭ تَجرِى بِأَمرِهِۦٓ إِلَى ٱلأَرضِ ٱلَّتِى بَٰرَكنَا فِيهَا ۚ وَكُنَّا بِكُلِّ شَىءٍ عَـٰلِمِينَ ﴿٨١﴾
 وَمِنَ ٱلشَّيَـٰطِينِ مَن يَغُوصُونَ لَهُۥ وَيَعمَلُونَ عَمَلًۭا دُونَ ذَٟلِكَ ۖ وَكُنَّا لَهُم حَـٰفِظِينَ ﴿٨٢﴾
 ۞ وَأَيُّوبَ إِذ نَادَىٰ رَبَّهُۥٓ أَنِّى مَسَّنِىَ ٱلضُّرُّ وَأَنتَ أَرحَمُ ٱلرَّٟحِمِينَ ﴿٨٣﴾
 فَٱستَجَبنَا لَهُۥ فَكَشَفنَا مَا بِهِۦ مِن ضُرٍّۢ ۖ وَءَاتَينَـٰهُ أَهلَهُۥ وَمِثلَهُم مَّعَهُم رَحمَةًۭ مِّن عِندِنَا وَذِكرَىٰ لِلعَـٰبِدِينَ ﴿٨٤﴾
 وَإِسمَـٰعِيلَ وَإِدرِيسَ وَذَا ٱلكِفلِ ۖ كُلٌّۭ مِّنَ ٱلصَّـٰبِرِينَ ﴿٨٥﴾
 وَأَدخَلنَـٰهُم فِى رَحمَتِنَآ ۖ إِنَّهُم مِّنَ ٱلصَّـٰلِحِينَ ﴿٨٦﴾
 وَذَا ٱلنُّونِ إِذ ذَّهَبَ مُغَٰضِبًۭا فَظَنَّ أَن لَّن نَّقدِرَ عَلَيهِ فَنَادَىٰ فِى ٱلظُّلُمَـٰتِ أَن لَّآ إِلَـٰهَ إِلَّآ أَنتَ سُبحَـٰنَكَ إِنِّى كُنتُ مِنَ ٱلظَّـٰلِمِينَ ﴿٨٧﴾
 فَٱستَجَبنَا لَهُۥ وَنَجَّينَـٰهُ مِنَ ٱلغَمِّ ۚ وَكَذَٟلِكَ نُۨجِى ٱلمُؤمِنِينَ ﴿٨٨﴾
 وَزَكَرِيَّآ إِذ نَادَىٰ رَبَّهُۥ رَبِّ لَا تَذَرنِى فَردًۭا وَأَنتَ خَيرُ ٱلوَٟرِثِينَ ﴿٨٩﴾
 فَٱستَجَبنَا لَهُۥ وَوَهَبنَا لَهُۥ يَحيَىٰ وَأَصلَحنَا لَهُۥ زَوجَهُۥٓ ۚ إِنَّهُم كَانُوا۟ يُسَـٰرِعُونَ فِى ٱلخَيرَٰتِ وَيَدعُونَنَا رَغَبًۭا وَرَهَبًۭا ۖ وَكَانُوا۟ لَنَا خَـٰشِعِينَ ﴿٩٠﴾
 وَٱلَّتِىٓ أَحصَنَت فَرجَهَا فَنَفَخنَا فِيهَا مِن رُّوحِنَا وَجَعَلنَـٰهَا وَٱبنَهَآ ءَايَةًۭ لِّلعَـٰلَمِينَ ﴿٩١﴾
 إِنَّ هَـٰذِهِۦٓ أُمَّتُكُم أُمَّةًۭ وَٟحِدَةًۭ وَأَنَا۠ رَبُّكُم فَٱعبُدُونِ ﴿٩٢﴾
 وَتَقَطَّعُوٓا۟ أَمرَهُم بَينَهُم ۖ كُلٌّ إِلَينَا رَٰجِعُونَ ﴿٩٣﴾
 فَمَن يَعمَل مِنَ ٱلصَّـٰلِحَـٰتِ وَهُوَ مُؤمِنٌۭ فَلَا كُفرَانَ لِسَعيِهِۦ وَإِنَّا لَهُۥ كَـٰتِبُونَ ﴿٩٤﴾
 وَحَرَٰمٌ عَلَىٰ قَريَةٍ أَهلَكنَـٰهَآ أَنَّهُم لَا يَرجِعُونَ ﴿٩٥﴾
 حَتَّىٰٓ إِذَا فُتِحَت يَأجُوجُ وَمَأجُوجُ وَهُم مِّن كُلِّ حَدَبٍۢ يَنسِلُونَ ﴿٩٦﴾
 وَٱقتَرَبَ ٱلوَعدُ ٱلحَقُّ فَإِذَا هِىَ شَـٰخِصَةٌ أَبصَـٰرُ ٱلَّذِينَ كَفَرُوا۟ يَـٰوَيلَنَا قَد كُنَّا فِى غَفلَةٍۢ مِّن هَـٰذَا بَل كُنَّا ظَـٰلِمِينَ ﴿٩٧﴾
 إِنَّكُم وَمَا تَعبُدُونَ مِن دُونِ ٱللَّهِ حَصَبُ جَهَنَّمَ أَنتُم لَهَا وَٟرِدُونَ ﴿٩٨﴾
 لَو كَانَ هَـٰٓؤُلَآءِ ءَالِهَةًۭ مَّا وَرَدُوهَا ۖ وَكُلٌّۭ فِيهَا خَـٰلِدُونَ ﴿٩٩﴾
 لَهُم فِيهَا زَفِيرٌۭ وَهُم فِيهَا لَا يَسمَعُونَ ﴿١٠٠﴾
 إِنَّ ٱلَّذِينَ سَبَقَت لَهُم مِّنَّا ٱلحُسنَىٰٓ أُو۟لَـٰٓئِكَ عَنهَا مُبعَدُونَ ﴿١٠١﴾
 لَا يَسمَعُونَ حَسِيسَهَا ۖ وَهُم فِى مَا ٱشتَهَت أَنفُسُهُم خَـٰلِدُونَ ﴿١٠٢﴾
 لَا يَحزُنُهُمُ ٱلفَزَعُ ٱلأَكبَرُ وَتَتَلَقَّىٰهُمُ ٱلمَلَـٰٓئِكَةُ هَـٰذَا يَومُكُمُ ٱلَّذِى كُنتُم تُوعَدُونَ ﴿١٠٣﴾
 يَومَ نَطوِى ٱلسَّمَآءَ كَطَىِّ ٱلسِّجِلِّ لِلكُتُبِ ۚ كَمَا بَدَأنَآ أَوَّلَ خَلقٍۢ نُّعِيدُهُۥ ۚ وَعدًا عَلَينَآ ۚ إِنَّا كُنَّا فَـٰعِلِينَ ﴿١٠٤﴾
 وَلَقَد كَتَبنَا فِى ٱلزَّبُورِ مِنۢ بَعدِ ٱلذِّكرِ أَنَّ ٱلأَرضَ يَرِثُهَا عِبَادِىَ ٱلصَّـٰلِحُونَ ﴿١٠٥﴾
 إِنَّ فِى هَـٰذَا لَبَلَـٰغًۭا لِّقَومٍ عَـٰبِدِينَ ﴿١٠٦﴾
 وَمَآ أَرسَلنَـٰكَ إِلَّا رَحمَةًۭ لِّلعَـٰلَمِينَ ﴿١٠٧﴾
 قُل إِنَّمَا يُوحَىٰٓ إِلَىَّ أَنَّمَآ إِلَـٰهُكُم إِلَـٰهٌۭ وَٟحِدٌۭ ۖ فَهَل أَنتُم مُّسلِمُونَ ﴿١٠٨﴾
 فَإِن تَوَلَّوا۟ فَقُل ءَاذَنتُكُم عَلَىٰ سَوَآءٍۢ ۖ وَإِن أَدرِىٓ أَقَرِيبٌ أَم بَعِيدٌۭ مَّا تُوعَدُونَ ﴿١٠٩﴾
 إِنَّهُۥ يَعلَمُ ٱلجَهرَ مِنَ ٱلقَولِ وَيَعلَمُ مَا تَكتُمُونَ ﴿١١٠﴾
 وَإِن أَدرِى لَعَلَّهُۥ فِتنَةٌۭ لَّكُم وَمَتَـٰعٌ إِلَىٰ حِينٍۢ ﴿١١١﴾
 قَـٰلَ رَبِّ ٱحكُم بِٱلحَقِّ ۗ وَرَبُّنَا ٱلرَّحمَـٰنُ ٱلمُستَعَانُ عَلَىٰ مَا تَصِفُونَ ﴿١١٢﴾
 
