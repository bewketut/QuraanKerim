%% License: BSD style (Berkley) (i.e. Put the Copyright owner's name always)
%% Writer and Copyright (to): Bewketu(Bilal) Tadilo (2016-17)
\shadowbox{\section{\LR{\textamharic{ሱራቱ አልአህዛብ -}  \RL{سوره  الأحزاب}}}}

  
    
  
    
    

\nopagebreak
  بِسمِ ٱللَّهِ ٱلرَّحمَـٰنِ ٱلرَّحِيمِ
  يَـٰٓأَيُّهَا ٱلنَّبِىُّ ٱتَّقِ ٱللَّهَ وَلَا تُطِعِ ٱلكَـٰفِرِينَ وَٱلمُنَـٰفِقِينَ ۗ إِنَّ ٱللَّهَ كَانَ عَلِيمًا حَكِيمًۭا ﴿١﴾
 وَٱتَّبِع مَا يُوحَىٰٓ إِلَيكَ مِن رَّبِّكَ ۚ إِنَّ ٱللَّهَ كَانَ بِمَا تَعمَلُونَ خَبِيرًۭا ﴿٢﴾
 وَتَوَكَّل عَلَى ٱللَّهِ ۚ وَكَفَىٰ بِٱللَّهِ وَكِيلًۭا ﴿٣﴾
 مَّا جَعَلَ ٱللَّهُ لِرَجُلٍۢ مِّن قَلبَينِ فِى جَوفِهِۦ ۚ وَمَا جَعَلَ أَزوَٟجَكُمُ ٱلَّٰٓـِٔى تُظَـٰهِرُونَ مِنهُنَّ أُمَّهَـٰتِكُم ۚ وَمَا جَعَلَ أَدعِيَآءَكُم أَبنَآءَكُم ۚ ذَٟلِكُم قَولُكُم بِأَفوَٟهِكُم ۖ وَٱللَّهُ يَقُولُ ٱلحَقَّ وَهُوَ يَهدِى ٱلسَّبِيلَ ﴿٤﴾
 ٱدعُوهُم لِءَابَآئِهِم هُوَ أَقسَطُ عِندَ ٱللَّهِ ۚ فَإِن لَّم تَعلَمُوٓا۟ ءَابَآءَهُم فَإِخوَٟنُكُم فِى ٱلدِّينِ وَمَوَٟلِيكُم ۚ وَلَيسَ عَلَيكُم جُنَاحٌۭ فِيمَآ أَخطَأتُم بِهِۦ وَلَـٰكِن مَّا تَعَمَّدَت قُلُوبُكُم ۚ وَكَانَ ٱللَّهُ غَفُورًۭا رَّحِيمًا ﴿٥﴾
 ٱلنَّبِىُّ أَولَىٰ بِٱلمُؤمِنِينَ مِن أَنفُسِهِم ۖ وَأَزوَٟجُهُۥٓ أُمَّهَـٰتُهُم ۗ وَأُو۟لُوا۟ ٱلأَرحَامِ بَعضُهُم أَولَىٰ بِبَعضٍۢ فِى كِتَـٰبِ ٱللَّهِ مِنَ ٱلمُؤمِنِينَ وَٱلمُهَـٰجِرِينَ إِلَّآ أَن تَفعَلُوٓا۟ إِلَىٰٓ أَولِيَآئِكُم مَّعرُوفًۭا ۚ كَانَ ذَٟلِكَ فِى ٱلكِتَـٰبِ مَسطُورًۭا ﴿٦﴾
 وَإِذ أَخَذنَا مِنَ ٱلنَّبِيِّۦنَ مِيثَـٰقَهُم وَمِنكَ وَمِن نُّوحٍۢ وَإِبرَٰهِيمَ وَمُوسَىٰ وَعِيسَى ٱبنِ مَريَمَ ۖ وَأَخَذنَا مِنهُم مِّيثَـٰقًا غَلِيظًۭا ﴿٧﴾
 لِّيَسـَٔلَ ٱلصَّـٰدِقِينَ عَن صِدقِهِم ۚ وَأَعَدَّ لِلكَـٰفِرِينَ عَذَابًا أَلِيمًۭا ﴿٨﴾
 يَـٰٓأَيُّهَا ٱلَّذِينَ ءَامَنُوا۟ ٱذكُرُوا۟ نِعمَةَ ٱللَّهِ عَلَيكُم إِذ جَآءَتكُم جُنُودٌۭ فَأَرسَلنَا عَلَيهِم رِيحًۭا وَجُنُودًۭا لَّم تَرَوهَا ۚ وَكَانَ ٱللَّهُ بِمَا تَعمَلُونَ بَصِيرًا ﴿٩﴾
 إِذ جَآءُوكُم مِّن فَوقِكُم وَمِن أَسفَلَ مِنكُم وَإِذ زَاغَتِ ٱلأَبصَـٰرُ وَبَلَغَتِ ٱلقُلُوبُ ٱلحَنَاجِرَ وَتَظُنُّونَ بِٱللَّهِ ٱلظُّنُونَا۠ ﴿١٠﴾
 هُنَالِكَ ٱبتُلِىَ ٱلمُؤمِنُونَ وَزُلزِلُوا۟ زِلزَالًۭا شَدِيدًۭا ﴿١١﴾
 وَإِذ يَقُولُ ٱلمُنَـٰفِقُونَ وَٱلَّذِينَ فِى قُلُوبِهِم مَّرَضٌۭ مَّا وَعَدَنَا ٱللَّهُ وَرَسُولُهُۥٓ إِلَّا غُرُورًۭا ﴿١٢﴾
 وَإِذ قَالَت طَّآئِفَةٌۭ مِّنهُم يَـٰٓأَهلَ يَثرِبَ لَا مُقَامَ لَكُم فَٱرجِعُوا۟ ۚ وَيَستَـٔذِنُ فَرِيقٌۭ مِّنهُمُ ٱلنَّبِىَّ يَقُولُونَ إِنَّ بُيُوتَنَا عَورَةٌۭ وَمَا هِىَ بِعَورَةٍ ۖ إِن يُرِيدُونَ إِلَّا فِرَارًۭا ﴿١٣﴾
 وَلَو دُخِلَت عَلَيهِم مِّن أَقطَارِهَا ثُمَّ سُئِلُوا۟ ٱلفِتنَةَ لَءَاتَوهَا وَمَا تَلَبَّثُوا۟ بِهَآ إِلَّا يَسِيرًۭا ﴿١٤﴾
 وَلَقَد كَانُوا۟ عَـٰهَدُوا۟ ٱللَّهَ مِن قَبلُ لَا يُوَلُّونَ ٱلأَدبَٰرَ ۚ وَكَانَ عَهدُ ٱللَّهِ مَسـُٔولًۭا ﴿١٥﴾
 قُل لَّن يَنفَعَكُمُ ٱلفِرَارُ إِن فَرَرتُم مِّنَ ٱلمَوتِ أَوِ ٱلقَتلِ وَإِذًۭا لَّا تُمَتَّعُونَ إِلَّا قَلِيلًۭا ﴿١٦﴾
 قُل مَن ذَا ٱلَّذِى يَعصِمُكُم مِّنَ ٱللَّهِ إِن أَرَادَ بِكُم سُوٓءًا أَو أَرَادَ بِكُم رَحمَةًۭ ۚ وَلَا يَجِدُونَ لَهُم مِّن دُونِ ٱللَّهِ وَلِيًّۭا وَلَا نَصِيرًۭا ﴿١٧﴾
 ۞ قَد يَعلَمُ ٱللَّهُ ٱلمُعَوِّقِينَ مِنكُم وَٱلقَآئِلِينَ لِإِخوَٟنِهِم هَلُمَّ إِلَينَا ۖ وَلَا يَأتُونَ ٱلبَأسَ إِلَّا قَلِيلًا ﴿١٨﴾
 أَشِحَّةً عَلَيكُم ۖ فَإِذَا جَآءَ ٱلخَوفُ رَأَيتَهُم يَنظُرُونَ إِلَيكَ تَدُورُ أَعيُنُهُم كَٱلَّذِى يُغشَىٰ عَلَيهِ مِنَ ٱلمَوتِ ۖ فَإِذَا ذَهَبَ ٱلخَوفُ سَلَقُوكُم بِأَلسِنَةٍ حِدَادٍ أَشِحَّةً عَلَى ٱلخَيرِ ۚ أُو۟لَـٰٓئِكَ لَم يُؤمِنُوا۟ فَأَحبَطَ ٱللَّهُ أَعمَـٰلَهُم ۚ وَكَانَ ذَٟلِكَ عَلَى ٱللَّهِ يَسِيرًۭا ﴿١٩﴾
 يَحسَبُونَ ٱلأَحزَابَ لَم يَذهَبُوا۟ ۖ وَإِن يَأتِ ٱلأَحزَابُ يَوَدُّوا۟ لَو أَنَّهُم بَادُونَ فِى ٱلأَعرَابِ يَسـَٔلُونَ عَن أَنۢبَآئِكُم ۖ وَلَو كَانُوا۟ فِيكُم مَّا قَـٰتَلُوٓا۟ إِلَّا قَلِيلًۭا ﴿٢٠﴾
 لَّقَد كَانَ لَكُم فِى رَسُولِ ٱللَّهِ أُسوَةٌ حَسَنَةٌۭ لِّمَن كَانَ يَرجُوا۟ ٱللَّهَ وَٱليَومَ ٱلءَاخِرَ وَذَكَرَ ٱللَّهَ كَثِيرًۭا ﴿٢١﴾
 وَلَمَّا رَءَا ٱلمُؤمِنُونَ ٱلأَحزَابَ قَالُوا۟ هَـٰذَا مَا وَعَدَنَا ٱللَّهُ وَرَسُولُهُۥ وَصَدَقَ ٱللَّهُ وَرَسُولُهُۥ ۚ وَمَا زَادَهُم إِلَّآ إِيمَـٰنًۭا وَتَسلِيمًۭا ﴿٢٢﴾
 مِّنَ ٱلمُؤمِنِينَ رِجَالٌۭ صَدَقُوا۟ مَا عَـٰهَدُوا۟ ٱللَّهَ عَلَيهِ ۖ فَمِنهُم مَّن قَضَىٰ نَحبَهُۥ وَمِنهُم مَّن يَنتَظِرُ ۖ وَمَا بَدَّلُوا۟ تَبدِيلًۭا ﴿٢٣﴾
 لِّيَجزِىَ ٱللَّهُ ٱلصَّـٰدِقِينَ بِصِدقِهِم وَيُعَذِّبَ ٱلمُنَـٰفِقِينَ إِن شَآءَ أَو يَتُوبَ عَلَيهِم ۚ إِنَّ ٱللَّهَ كَانَ غَفُورًۭا رَّحِيمًۭا ﴿٢٤﴾
 وَرَدَّ ٱللَّهُ ٱلَّذِينَ كَفَرُوا۟ بِغَيظِهِم لَم يَنَالُوا۟ خَيرًۭا ۚ وَكَفَى ٱللَّهُ ٱلمُؤمِنِينَ ٱلقِتَالَ ۚ وَكَانَ ٱللَّهُ قَوِيًّا عَزِيزًۭا ﴿٢٥﴾
 وَأَنزَلَ ٱلَّذِينَ ظَـٰهَرُوهُم مِّن أَهلِ ٱلكِتَـٰبِ مِن صَيَاصِيهِم وَقَذَفَ فِى قُلُوبِهِمُ ٱلرُّعبَ فَرِيقًۭا تَقتُلُونَ وَتَأسِرُونَ فَرِيقًۭا ﴿٢٦﴾
 وَأَورَثَكُم أَرضَهُم وَدِيَـٰرَهُم وَأَموَٟلَهُم وَأَرضًۭا لَّم تَطَـُٔوهَا ۚ وَكَانَ ٱللَّهُ عَلَىٰ كُلِّ شَىءٍۢ قَدِيرًۭا ﴿٢٧﴾
 يَـٰٓأَيُّهَا ٱلنَّبِىُّ قُل لِّأَزوَٟجِكَ إِن كُنتُنَّ تُرِدنَ ٱلحَيَوٰةَ ٱلدُّنيَا وَزِينَتَهَا فَتَعَالَينَ أُمَتِّعكُنَّ وَأُسَرِّحكُنَّ سَرَاحًۭا جَمِيلًۭا ﴿٢٨﴾
 وَإِن كُنتُنَّ تُرِدنَ ٱللَّهَ وَرَسُولَهُۥ وَٱلدَّارَ ٱلءَاخِرَةَ فَإِنَّ ٱللَّهَ أَعَدَّ لِلمُحسِنَـٰتِ مِنكُنَّ أَجرًا عَظِيمًۭا ﴿٢٩﴾
 يَـٰنِسَآءَ ٱلنَّبِىِّ مَن يَأتِ مِنكُنَّ بِفَـٰحِشَةٍۢ مُّبَيِّنَةٍۢ يُضَٰعَف لَهَا ٱلعَذَابُ ضِعفَينِ ۚ وَكَانَ ذَٟلِكَ عَلَى ٱللَّهِ يَسِيرًۭا ﴿٣٠﴾
 ۞ وَمَن يَقنُت مِنكُنَّ لِلَّهِ وَرَسُولِهِۦ وَتَعمَل صَـٰلِحًۭا نُّؤتِهَآ أَجرَهَا مَرَّتَينِ وَأَعتَدنَا لَهَا رِزقًۭا كَرِيمًۭا ﴿٣١﴾
 يَـٰنِسَآءَ ٱلنَّبِىِّ لَستُنَّ كَأَحَدٍۢ مِّنَ ٱلنِّسَآءِ ۚ إِنِ ٱتَّقَيتُنَّ فَلَا تَخضَعنَ بِٱلقَولِ فَيَطمَعَ ٱلَّذِى فِى قَلبِهِۦ مَرَضٌۭ وَقُلنَ قَولًۭا مَّعرُوفًۭا ﴿٣٢﴾
 وَقَرنَ فِى بُيُوتِكُنَّ وَلَا تَبَرَّجنَ تَبَرُّجَ ٱلجَٰهِلِيَّةِ ٱلأُولَىٰ ۖ وَأَقِمنَ ٱلصَّلَوٰةَ وَءَاتِينَ ٱلزَّكَوٰةَ وَأَطِعنَ ٱللَّهَ وَرَسُولَهُۥٓ ۚ إِنَّمَا يُرِيدُ ٱللَّهُ لِيُذهِبَ عَنكُمُ ٱلرِّجسَ أَهلَ ٱلبَيتِ وَيُطَهِّرَكُم تَطهِيرًۭا ﴿٣٣﴾
 وَٱذكُرنَ مَا يُتلَىٰ فِى بُيُوتِكُنَّ مِن ءَايَـٰتِ ٱللَّهِ وَٱلحِكمَةِ ۚ إِنَّ ٱللَّهَ كَانَ لَطِيفًا خَبِيرًا ﴿٣٤﴾
 إِنَّ ٱلمُسلِمِينَ وَٱلمُسلِمَـٰتِ وَٱلمُؤمِنِينَ وَٱلمُؤمِنَـٰتِ وَٱلقَـٰنِتِينَ وَٱلقَـٰنِتَـٰتِ وَٱلصَّـٰدِقِينَ وَٱلصَّـٰدِقَـٰتِ وَٱلصَّـٰبِرِينَ وَٱلصَّـٰبِرَٰتِ وَٱلخَـٰشِعِينَ وَٱلخَـٰشِعَـٰتِ وَٱلمُتَصَدِّقِينَ وَٱلمُتَصَدِّقَـٰتِ وَٱلصَّـٰٓئِمِينَ وَٱلصَّـٰٓئِمَـٰتِ وَٱلحَـٰفِظِينَ فُرُوجَهُم وَٱلحَـٰفِظَـٰتِ وَٱلذَّٰكِرِينَ ٱللَّهَ كَثِيرًۭا وَٱلذَّٰكِرَٰتِ أَعَدَّ ٱللَّهُ لَهُم مَّغفِرَةًۭ وَأَجرًا عَظِيمًۭا ﴿٣٥﴾
 وَمَا كَانَ لِمُؤمِنٍۢ وَلَا مُؤمِنَةٍ إِذَا قَضَى ٱللَّهُ وَرَسُولُهُۥٓ أَمرًا أَن يَكُونَ لَهُمُ ٱلخِيَرَةُ مِن أَمرِهِم ۗ وَمَن يَعصِ ٱللَّهَ وَرَسُولَهُۥ فَقَد ضَلَّ ضَلَـٰلًۭا مُّبِينًۭا ﴿٣٦﴾
 وَإِذ تَقُولُ لِلَّذِىٓ أَنعَمَ ٱللَّهُ عَلَيهِ وَأَنعَمتَ عَلَيهِ أَمسِك عَلَيكَ زَوجَكَ وَٱتَّقِ ٱللَّهَ وَتُخفِى فِى نَفسِكَ مَا ٱللَّهُ مُبدِيهِ وَتَخشَى ٱلنَّاسَ وَٱللَّهُ أَحَقُّ أَن تَخشَىٰهُ ۖ فَلَمَّا قَضَىٰ زَيدٌۭ مِّنهَا وَطَرًۭا زَوَّجنَـٰكَهَا لِكَى لَا يَكُونَ عَلَى ٱلمُؤمِنِينَ حَرَجٌۭ فِىٓ أَزوَٟجِ أَدعِيَآئِهِم إِذَا قَضَوا۟ مِنهُنَّ وَطَرًۭا ۚ وَكَانَ أَمرُ ٱللَّهِ مَفعُولًۭا ﴿٣٧﴾
 مَّا كَانَ عَلَى ٱلنَّبِىِّ مِن حَرَجٍۢ فِيمَا فَرَضَ ٱللَّهُ لَهُۥ ۖ سُنَّةَ ٱللَّهِ فِى ٱلَّذِينَ خَلَوا۟ مِن قَبلُ ۚ وَكَانَ أَمرُ ٱللَّهِ قَدَرًۭا مَّقدُورًا ﴿٣٨﴾
 ٱلَّذِينَ يُبَلِّغُونَ رِسَـٰلَـٰتِ ٱللَّهِ وَيَخشَونَهُۥ وَلَا يَخشَونَ أَحَدًا إِلَّا ٱللَّهَ ۗ وَكَفَىٰ بِٱللَّهِ حَسِيبًۭا ﴿٣٩﴾
 مَّا كَانَ مُحَمَّدٌ أَبَآ أَحَدٍۢ مِّن رِّجَالِكُم وَلَـٰكِن رَّسُولَ ٱللَّهِ وَخَاتَمَ ٱلنَّبِيِّۦنَ ۗ وَكَانَ ٱللَّهُ بِكُلِّ شَىءٍ عَلِيمًۭا ﴿٤٠﴾
 يَـٰٓأَيُّهَا ٱلَّذِينَ ءَامَنُوا۟ ٱذكُرُوا۟ ٱللَّهَ ذِكرًۭا كَثِيرًۭا ﴿٤١﴾
 وَسَبِّحُوهُ بُكرَةًۭ وَأَصِيلًا ﴿٤٢﴾
 هُوَ ٱلَّذِى يُصَلِّى عَلَيكُم وَمَلَـٰٓئِكَتُهُۥ لِيُخرِجَكُم مِّنَ ٱلظُّلُمَـٰتِ إِلَى ٱلنُّورِ ۚ وَكَانَ بِٱلمُؤمِنِينَ رَحِيمًۭا ﴿٤٣﴾
 تَحِيَّتُهُم يَومَ يَلقَونَهُۥ سَلَـٰمٌۭ ۚ وَأَعَدَّ لَهُم أَجرًۭا كَرِيمًۭا ﴿٤٤﴾
 يَـٰٓأَيُّهَا ٱلنَّبِىُّ إِنَّآ أَرسَلنَـٰكَ شَـٰهِدًۭا وَمُبَشِّرًۭا وَنَذِيرًۭا ﴿٤٥﴾
 وَدَاعِيًا إِلَى ٱللَّهِ بِإِذنِهِۦ وَسِرَاجًۭا مُّنِيرًۭا ﴿٤٦﴾
 وَبَشِّرِ ٱلمُؤمِنِينَ بِأَنَّ لَهُم مِّنَ ٱللَّهِ فَضلًۭا كَبِيرًۭا ﴿٤٧﴾
 وَلَا تُطِعِ ٱلكَـٰفِرِينَ وَٱلمُنَـٰفِقِينَ وَدَع أَذَىٰهُم وَتَوَكَّل عَلَى ٱللَّهِ ۚ وَكَفَىٰ بِٱللَّهِ وَكِيلًۭا ﴿٤٨﴾
 يَـٰٓأَيُّهَا ٱلَّذِينَ ءَامَنُوٓا۟ إِذَا نَكَحتُمُ ٱلمُؤمِنَـٰتِ ثُمَّ طَلَّقتُمُوهُنَّ مِن قَبلِ أَن تَمَسُّوهُنَّ فَمَا لَكُم عَلَيهِنَّ مِن عِدَّةٍۢ تَعتَدُّونَهَا ۖ فَمَتِّعُوهُنَّ وَسَرِّحُوهُنَّ سَرَاحًۭا جَمِيلًۭا ﴿٤٩﴾
 يَـٰٓأَيُّهَا ٱلنَّبِىُّ إِنَّآ أَحلَلنَا لَكَ أَزوَٟجَكَ ٱلَّٰتِىٓ ءَاتَيتَ أُجُورَهُنَّ وَمَا مَلَكَت يَمِينُكَ مِمَّآ أَفَآءَ ٱللَّهُ عَلَيكَ وَبَنَاتِ عَمِّكَ وَبَنَاتِ عَمَّٰتِكَ وَبَنَاتِ خَالِكَ وَبَنَاتِ خَـٰلَـٰتِكَ ٱلَّٰتِى هَاجَرنَ مَعَكَ وَٱمرَأَةًۭ مُّؤمِنَةً إِن وَهَبَت نَفسَهَا لِلنَّبِىِّ إِن أَرَادَ ٱلنَّبِىُّ أَن يَستَنكِحَهَا خَالِصَةًۭ لَّكَ مِن دُونِ ٱلمُؤمِنِينَ ۗ قَد عَلِمنَا مَا فَرَضنَا عَلَيهِم فِىٓ أَزوَٟجِهِم وَمَا مَلَكَت أَيمَـٰنُهُم لِكَيلَا يَكُونَ عَلَيكَ حَرَجٌۭ ۗ وَكَانَ ٱللَّهُ غَفُورًۭا رَّحِيمًۭا ﴿٥٠﴾
 ۞ تُرجِى مَن تَشَآءُ مِنهُنَّ وَتُـٔوِىٓ إِلَيكَ مَن تَشَآءُ ۖ وَمَنِ ٱبتَغَيتَ مِمَّن عَزَلتَ فَلَا جُنَاحَ عَلَيكَ ۚ ذَٟلِكَ أَدنَىٰٓ أَن تَقَرَّ أَعيُنُهُنَّ وَلَا يَحزَنَّ وَيَرضَينَ بِمَآ ءَاتَيتَهُنَّ كُلُّهُنَّ ۚ وَٱللَّهُ يَعلَمُ مَا فِى قُلُوبِكُم ۚ وَكَانَ ٱللَّهُ عَلِيمًا حَلِيمًۭا ﴿٥١﴾
 لَّا يَحِلُّ لَكَ ٱلنِّسَآءُ مِنۢ بَعدُ وَلَآ أَن تَبَدَّلَ بِهِنَّ مِن أَزوَٟجٍۢ وَلَو أَعجَبَكَ حُسنُهُنَّ إِلَّا مَا مَلَكَت يَمِينُكَ ۗ وَكَانَ ٱللَّهُ عَلَىٰ كُلِّ شَىءٍۢ رَّقِيبًۭا ﴿٥٢﴾
 يَـٰٓأَيُّهَا ٱلَّذِينَ ءَامَنُوا۟ لَا تَدخُلُوا۟ بُيُوتَ ٱلنَّبِىِّ إِلَّآ أَن يُؤذَنَ لَكُم إِلَىٰ طَعَامٍ غَيرَ نَـٰظِرِينَ إِنَىٰهُ وَلَـٰكِن إِذَا دُعِيتُم فَٱدخُلُوا۟ فَإِذَا طَعِمتُم فَٱنتَشِرُوا۟ وَلَا مُستَـٔنِسِينَ لِحَدِيثٍ ۚ إِنَّ ذَٟلِكُم كَانَ يُؤذِى ٱلنَّبِىَّ فَيَستَحىِۦ مِنكُم ۖ وَٱللَّهُ لَا يَستَحىِۦ مِنَ ٱلحَقِّ ۚ وَإِذَا سَأَلتُمُوهُنَّ مَتَـٰعًۭا فَسـَٔلُوهُنَّ مِن وَرَآءِ حِجَابٍۢ ۚ ذَٟلِكُم أَطهَرُ لِقُلُوبِكُم وَقُلُوبِهِنَّ ۚ وَمَا كَانَ لَكُم أَن تُؤذُوا۟ رَسُولَ ٱللَّهِ وَلَآ أَن تَنكِحُوٓا۟ أَزوَٟجَهُۥ مِنۢ بَعدِهِۦٓ أَبَدًا ۚ إِنَّ ذَٟلِكُم كَانَ عِندَ ٱللَّهِ عَظِيمًا ﴿٥٣﴾
 إِن تُبدُوا۟ شَيـًٔا أَو تُخفُوهُ فَإِنَّ ٱللَّهَ كَانَ بِكُلِّ شَىءٍ عَلِيمًۭا ﴿٥٤﴾
 لَّا جُنَاحَ عَلَيهِنَّ فِىٓ ءَابَآئِهِنَّ وَلَآ أَبنَآئِهِنَّ وَلَآ إِخوَٟنِهِنَّ وَلَآ أَبنَآءِ إِخوَٟنِهِنَّ وَلَآ أَبنَآءِ أَخَوَٟتِهِنَّ وَلَا نِسَآئِهِنَّ وَلَا مَا مَلَكَت أَيمَـٰنُهُنَّ ۗ وَٱتَّقِينَ ٱللَّهَ ۚ إِنَّ ٱللَّهَ كَانَ عَلَىٰ كُلِّ شَىءٍۢ شَهِيدًا ﴿٥٥﴾
 إِنَّ ٱللَّهَ وَمَلَـٰٓئِكَتَهُۥ يُصَلُّونَ عَلَى ٱلنَّبِىِّ ۚ يَـٰٓأَيُّهَا ٱلَّذِينَ ءَامَنُوا۟ صَلُّوا۟ عَلَيهِ وَسَلِّمُوا۟ تَسلِيمًا ﴿٥٦﴾
 إِنَّ ٱلَّذِينَ يُؤذُونَ ٱللَّهَ وَرَسُولَهُۥ لَعَنَهُمُ ٱللَّهُ فِى ٱلدُّنيَا وَٱلءَاخِرَةِ وَأَعَدَّ لَهُم عَذَابًۭا مُّهِينًۭا ﴿٥٧﴾
 وَٱلَّذِينَ يُؤذُونَ ٱلمُؤمِنِينَ وَٱلمُؤمِنَـٰتِ بِغَيرِ مَا ٱكتَسَبُوا۟ فَقَدِ ٱحتَمَلُوا۟ بُهتَـٰنًۭا وَإِثمًۭا مُّبِينًۭا ﴿٥٨﴾
 يَـٰٓأَيُّهَا ٱلنَّبِىُّ قُل لِّأَزوَٟجِكَ وَبَنَاتِكَ وَنِسَآءِ ٱلمُؤمِنِينَ يُدنِينَ عَلَيهِنَّ مِن جَلَـٰبِيبِهِنَّ ۚ ذَٟلِكَ أَدنَىٰٓ أَن يُعرَفنَ فَلَا يُؤذَينَ ۗ وَكَانَ ٱللَّهُ غَفُورًۭا رَّحِيمًۭا ﴿٥٩﴾
 ۞ لَّئِن لَّم يَنتَهِ ٱلمُنَـٰفِقُونَ وَٱلَّذِينَ فِى قُلُوبِهِم مَّرَضٌۭ وَٱلمُرجِفُونَ فِى ٱلمَدِينَةِ لَنُغرِيَنَّكَ بِهِم ثُمَّ لَا يُجَاوِرُونَكَ فِيهَآ إِلَّا قَلِيلًۭا ﴿٦٠﴾
 مَّلعُونِينَ ۖ أَينَمَا ثُقِفُوٓا۟ أُخِذُوا۟ وَقُتِّلُوا۟ تَقتِيلًۭا ﴿٦١﴾
 سُنَّةَ ٱللَّهِ فِى ٱلَّذِينَ خَلَوا۟ مِن قَبلُ ۖ وَلَن تَجِدَ لِسُنَّةِ ٱللَّهِ تَبدِيلًۭا ﴿٦٢﴾
 يَسـَٔلُكَ ٱلنَّاسُ عَنِ ٱلسَّاعَةِ ۖ قُل إِنَّمَا عِلمُهَا عِندَ ٱللَّهِ ۚ وَمَا يُدرِيكَ لَعَلَّ ٱلسَّاعَةَ تَكُونُ قَرِيبًا ﴿٦٣﴾
 إِنَّ ٱللَّهَ لَعَنَ ٱلكَـٰفِرِينَ وَأَعَدَّ لَهُم سَعِيرًا ﴿٦٤﴾
 خَـٰلِدِينَ فِيهَآ أَبَدًۭا ۖ لَّا يَجِدُونَ وَلِيًّۭا وَلَا نَصِيرًۭا ﴿٦٥﴾
 يَومَ تُقَلَّبُ وُجُوهُهُم فِى ٱلنَّارِ يَقُولُونَ يَـٰلَيتَنَآ أَطَعنَا ٱللَّهَ وَأَطَعنَا ٱلرَّسُولَا۠ ﴿٦٦﴾
 وَقَالُوا۟ رَبَّنَآ إِنَّآ أَطَعنَا سَادَتَنَا وَكُبَرَآءَنَا فَأَضَلُّونَا ٱلسَّبِيلَا۠ ﴿٦٧﴾
 رَبَّنَآ ءَاتِهِم ضِعفَينِ مِنَ ٱلعَذَابِ وَٱلعَنهُم لَعنًۭا كَبِيرًۭا ﴿٦٨﴾
 يَـٰٓأَيُّهَا ٱلَّذِينَ ءَامَنُوا۟ لَا تَكُونُوا۟ كَٱلَّذِينَ ءَاذَوا۟ مُوسَىٰ فَبَرَّأَهُ ٱللَّهُ مِمَّا قَالُوا۟ ۚ وَكَانَ عِندَ ٱللَّهِ وَجِيهًۭا ﴿٦٩﴾
 يَـٰٓأَيُّهَا ٱلَّذِينَ ءَامَنُوا۟ ٱتَّقُوا۟ ٱللَّهَ وَقُولُوا۟ قَولًۭا سَدِيدًۭا ﴿٧٠﴾
 يُصلِح لَكُم أَعمَـٰلَكُم وَيَغفِر لَكُم ذُنُوبَكُم ۗ وَمَن يُطِعِ ٱللَّهَ وَرَسُولَهُۥ فَقَد فَازَ فَوزًا عَظِيمًا ﴿٧١﴾
 إِنَّا عَرَضنَا ٱلأَمَانَةَ عَلَى ٱلسَّمَـٰوَٟتِ وَٱلأَرضِ وَٱلجِبَالِ فَأَبَينَ أَن يَحمِلنَهَا وَأَشفَقنَ مِنهَا وَحَمَلَهَا ٱلإِنسَـٰنُ ۖ إِنَّهُۥ كَانَ ظَلُومًۭا جَهُولًۭا ﴿٧٢﴾
 لِّيُعَذِّبَ ٱللَّهُ ٱلمُنَـٰفِقِينَ وَٱلمُنَـٰفِقَـٰتِ وَٱلمُشرِكِينَ وَٱلمُشرِكَـٰتِ وَيَتُوبَ ٱللَّهُ عَلَى ٱلمُؤمِنِينَ وَٱلمُؤمِنَـٰتِ ۗ وَكَانَ ٱللَّهُ غَفُورًۭا رَّحِيمًۢا ﴿٧٣﴾
 
