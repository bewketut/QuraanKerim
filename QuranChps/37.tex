%% License: BSD style (Berkley) (i.e. Put the Copyright owner's name always)
%% Writer and Copyright (to): Bewketu(Bilal) Tadilo (2016-17)
\shadowbox{\section{\LR{\textamharic{ሱራቱ አስሳፋት -}  \RL{سوره  الصافات}}}}

  
    
  
    
    

\nopagebreak
  بِسمِ ٱللَّهِ ٱلرَّحمَـٰنِ ٱلرَّحِيمِ
  وَٱلصَّـٰٓفَّٰتِ صَفًّۭا ﴿١﴾
 فَٱلزَّٰجِرَٰتِ زَجرًۭا ﴿٢﴾
 فَٱلتَّٰلِيَـٰتِ ذِكرًا ﴿٣﴾
 إِنَّ إِلَـٰهَكُم لَوَٟحِدٌۭ ﴿٤﴾
 رَّبُّ ٱلسَّمَـٰوَٟتِ وَٱلأَرضِ وَمَا بَينَهُمَا وَرَبُّ ٱلمَشَـٰرِقِ ﴿٥﴾
 إِنَّا زَيَّنَّا ٱلسَّمَآءَ ٱلدُّنيَا بِزِينَةٍ ٱلكَوَاكِبِ ﴿٦﴾
 وَحِفظًۭا مِّن كُلِّ شَيطَٰنٍۢ مَّارِدٍۢ ﴿٧﴾
 لَّا يَسَّمَّعُونَ إِلَى ٱلمَلَإِ ٱلأَعلَىٰ وَيُقذَفُونَ مِن كُلِّ جَانِبٍۢ ﴿٨﴾
 دُحُورًۭا ۖ وَلَهُم عَذَابٌۭ وَاصِبٌ ﴿٩﴾
 إِلَّا مَن خَطِفَ ٱلخَطفَةَ فَأَتبَعَهُۥ شِهَابٌۭ ثَاقِبٌۭ ﴿١٠﴾
 فَٱستَفتِهِم أَهُم أَشَدُّ خَلقًا أَم مَّن خَلَقنَآ ۚ إِنَّا خَلَقنَـٰهُم مِّن طِينٍۢ لَّازِبٍۭ ﴿١١﴾
 بَل عَجِبتَ وَيَسخَرُونَ ﴿١٢﴾
 وَإِذَا ذُكِّرُوا۟ لَا يَذكُرُونَ ﴿١٣﴾
 وَإِذَا رَأَوا۟ ءَايَةًۭ يَستَسخِرُونَ ﴿١٤﴾
 وَقَالُوٓا۟ إِن هَـٰذَآ إِلَّا سِحرٌۭ مُّبِينٌ ﴿١٥﴾
 أَءِذَا مِتنَا وَكُنَّا تُرَابًۭا وَعِظَـٰمًا أَءِنَّا لَمَبعُوثُونَ ﴿١٦﴾
 أَوَءَابَآؤُنَا ٱلأَوَّلُونَ ﴿١٧﴾
 قُل نَعَم وَأَنتُم دَٟخِرُونَ ﴿١٨﴾
 فَإِنَّمَا هِىَ زَجرَةٌۭ وَٟحِدَةٌۭ فَإِذَا هُم يَنظُرُونَ ﴿١٩﴾
 وَقَالُوا۟ يَـٰوَيلَنَا هَـٰذَا يَومُ ٱلدِّينِ ﴿٢٠﴾
 هَـٰذَا يَومُ ٱلفَصلِ ٱلَّذِى كُنتُم بِهِۦ تُكَذِّبُونَ ﴿٢١﴾
 ۞ ٱحشُرُوا۟ ٱلَّذِينَ ظَلَمُوا۟ وَأَزوَٟجَهُم وَمَا كَانُوا۟ يَعبُدُونَ ﴿٢٢﴾
 مِن دُونِ ٱللَّهِ فَٱهدُوهُم إِلَىٰ صِرَٰطِ ٱلجَحِيمِ ﴿٢٣﴾
 وَقِفُوهُم ۖ إِنَّهُم مَّسـُٔولُونَ ﴿٢٤﴾
 مَا لَكُم لَا تَنَاصَرُونَ ﴿٢٥﴾
 بَل هُمُ ٱليَومَ مُستَسلِمُونَ ﴿٢٦﴾
 وَأَقبَلَ بَعضُهُم عَلَىٰ بَعضٍۢ يَتَسَآءَلُونَ ﴿٢٧﴾
 قَالُوٓا۟ إِنَّكُم كُنتُم تَأتُونَنَا عَنِ ٱليَمِينِ ﴿٢٨﴾
 قَالُوا۟ بَل لَّم تَكُونُوا۟ مُؤمِنِينَ ﴿٢٩﴾
 وَمَا كَانَ لَنَا عَلَيكُم مِّن سُلطَٰنٍۭ ۖ بَل كُنتُم قَومًۭا طَٰغِينَ ﴿٣٠﴾
 فَحَقَّ عَلَينَا قَولُ رَبِّنَآ ۖ إِنَّا لَذَآئِقُونَ ﴿٣١﴾
 فَأَغوَينَـٰكُم إِنَّا كُنَّا غَٰوِينَ ﴿٣٢﴾
 فَإِنَّهُم يَومَئِذٍۢ فِى ٱلعَذَابِ مُشتَرِكُونَ ﴿٣٣﴾
 إِنَّا كَذَٟلِكَ نَفعَلُ بِٱلمُجرِمِينَ ﴿٣٤﴾
 إِنَّهُم كَانُوٓا۟ إِذَا قِيلَ لَهُم لَآ إِلَـٰهَ إِلَّا ٱللَّهُ يَستَكبِرُونَ ﴿٣٥﴾
 وَيَقُولُونَ أَئِنَّا لَتَارِكُوٓا۟ ءَالِهَتِنَا لِشَاعِرٍۢ مَّجنُونٍۭ ﴿٣٦﴾
 بَل جَآءَ بِٱلحَقِّ وَصَدَّقَ ٱلمُرسَلِينَ ﴿٣٧﴾
 إِنَّكُم لَذَآئِقُوا۟ ٱلعَذَابِ ٱلأَلِيمِ ﴿٣٨﴾
 وَمَا تُجزَونَ إِلَّا مَا كُنتُم تَعمَلُونَ ﴿٣٩﴾
 إِلَّا عِبَادَ ٱللَّهِ ٱلمُخلَصِينَ ﴿٤٠﴾
 أُو۟لَـٰٓئِكَ لَهُم رِزقٌۭ مَّعلُومٌۭ ﴿٤١﴾
 فَوَٟكِهُ ۖ وَهُم مُّكرَمُونَ ﴿٤٢﴾
 فِى جَنَّـٰتِ ٱلنَّعِيمِ ﴿٤٣﴾
 عَلَىٰ سُرُرٍۢ مُّتَقَـٰبِلِينَ ﴿٤٤﴾
 يُطَافُ عَلَيهِم بِكَأسٍۢ مِّن مَّعِينٍۭ ﴿٤٥﴾
 بَيضَآءَ لَذَّةٍۢ لِّلشَّـٰرِبِينَ ﴿٤٦﴾
 لَا فِيهَا غَولٌۭ وَلَا هُم عَنهَا يُنزَفُونَ ﴿٤٧﴾
 وَعِندَهُم قَـٰصِرَٰتُ ٱلطَّرفِ عِينٌۭ ﴿٤٨﴾
 كَأَنَّهُنَّ بَيضٌۭ مَّكنُونٌۭ ﴿٤٩﴾
 فَأَقبَلَ بَعضُهُم عَلَىٰ بَعضٍۢ يَتَسَآءَلُونَ ﴿٥٠﴾
 قَالَ قَآئِلٌۭ مِّنهُم إِنِّى كَانَ لِى قَرِينٌۭ ﴿٥١﴾
 يَقُولُ أَءِنَّكَ لَمِنَ ٱلمُصَدِّقِينَ ﴿٥٢﴾
 أَءِذَا مِتنَا وَكُنَّا تُرَابًۭا وَعِظَـٰمًا أَءِنَّا لَمَدِينُونَ ﴿٥٣﴾
 قَالَ هَل أَنتُم مُّطَّلِعُونَ ﴿٥٤﴾
 فَٱطَّلَعَ فَرَءَاهُ فِى سَوَآءِ ٱلجَحِيمِ ﴿٥٥﴾
 قَالَ تَٱللَّهِ إِن كِدتَّ لَتُردِينِ ﴿٥٦﴾
 وَلَولَا نِعمَةُ رَبِّى لَكُنتُ مِنَ ٱلمُحضَرِينَ ﴿٥٧﴾
 أَفَمَا نَحنُ بِمَيِّتِينَ ﴿٥٨﴾
 إِلَّا مَوتَتَنَا ٱلأُولَىٰ وَمَا نَحنُ بِمُعَذَّبِينَ ﴿٥٩﴾
 إِنَّ هَـٰذَا لَهُوَ ٱلفَوزُ ٱلعَظِيمُ ﴿٦٠﴾
 لِمِثلِ هَـٰذَا فَليَعمَلِ ٱلعَـٰمِلُونَ ﴿٦١﴾
 أَذَٟلِكَ خَيرٌۭ نُّزُلًا أَم شَجَرَةُ ٱلزَّقُّومِ ﴿٦٢﴾
 إِنَّا جَعَلنَـٰهَا فِتنَةًۭ لِّلظَّـٰلِمِينَ ﴿٦٣﴾
 إِنَّهَا شَجَرَةٌۭ تَخرُجُ فِىٓ أَصلِ ٱلجَحِيمِ ﴿٦٤﴾
 طَلعُهَا كَأَنَّهُۥ رُءُوسُ ٱلشَّيَـٰطِينِ ﴿٦٥﴾
 فَإِنَّهُم لَءَاكِلُونَ مِنهَا فَمَالِـُٔونَ مِنهَا ٱلبُطُونَ ﴿٦٦﴾
 ثُمَّ إِنَّ لَهُم عَلَيهَا لَشَوبًۭا مِّن حَمِيمٍۢ ﴿٦٧﴾
 ثُمَّ إِنَّ مَرجِعَهُم لَإِلَى ٱلجَحِيمِ ﴿٦٨﴾
 إِنَّهُم أَلفَوا۟ ءَابَآءَهُم ضَآلِّينَ ﴿٦٩﴾
 فَهُم عَلَىٰٓ ءَاثَـٰرِهِم يُهرَعُونَ ﴿٧٠﴾
 وَلَقَد ضَلَّ قَبلَهُم أَكثَرُ ٱلأَوَّلِينَ ﴿٧١﴾
 وَلَقَد أَرسَلنَا فِيهِم مُّنذِرِينَ ﴿٧٢﴾
 فَٱنظُر كَيفَ كَانَ عَـٰقِبَةُ ٱلمُنذَرِينَ ﴿٧٣﴾
 إِلَّا عِبَادَ ٱللَّهِ ٱلمُخلَصِينَ ﴿٧٤﴾
 وَلَقَد نَادَىٰنَا نُوحٌۭ فَلَنِعمَ ٱلمُجِيبُونَ ﴿٧٥﴾
 وَنَجَّينَـٰهُ وَأَهلَهُۥ مِنَ ٱلكَربِ ٱلعَظِيمِ ﴿٧٦﴾
 وَجَعَلنَا ذُرِّيَّتَهُۥ هُمُ ٱلبَاقِينَ ﴿٧٧﴾
 وَتَرَكنَا عَلَيهِ فِى ٱلءَاخِرِينَ ﴿٧٨﴾
 سَلَـٰمٌ عَلَىٰ نُوحٍۢ فِى ٱلعَـٰلَمِينَ ﴿٧٩﴾
 إِنَّا كَذَٟلِكَ نَجزِى ٱلمُحسِنِينَ ﴿٨٠﴾
 إِنَّهُۥ مِن عِبَادِنَا ٱلمُؤمِنِينَ ﴿٨١﴾
 ثُمَّ أَغرَقنَا ٱلءَاخَرِينَ ﴿٨٢﴾
 ۞ وَإِنَّ مِن شِيعَتِهِۦ لَإِبرَٰهِيمَ ﴿٨٣﴾
 إِذ جَآءَ رَبَّهُۥ بِقَلبٍۢ سَلِيمٍ ﴿٨٤﴾
 إِذ قَالَ لِأَبِيهِ وَقَومِهِۦ مَاذَا تَعبُدُونَ ﴿٨٥﴾
 أَئِفكًا ءَالِهَةًۭ دُونَ ٱللَّهِ تُرِيدُونَ ﴿٨٦﴾
 فَمَا ظَنُّكُم بِرَبِّ ٱلعَـٰلَمِينَ ﴿٨٧﴾
 فَنَظَرَ نَظرَةًۭ فِى ٱلنُّجُومِ ﴿٨٨﴾
 فَقَالَ إِنِّى سَقِيمٌۭ ﴿٨٩﴾
 فَتَوَلَّوا۟ عَنهُ مُدبِرِينَ ﴿٩٠﴾
 فَرَاغَ إِلَىٰٓ ءَالِهَتِهِم فَقَالَ أَلَا تَأكُلُونَ ﴿٩١﴾
 مَا لَكُم لَا تَنطِقُونَ ﴿٩٢﴾
 فَرَاغَ عَلَيهِم ضَربًۢا بِٱليَمِينِ ﴿٩٣﴾
 فَأَقبَلُوٓا۟ إِلَيهِ يَزِفُّونَ ﴿٩٤﴾
 قَالَ أَتَعبُدُونَ مَا تَنحِتُونَ ﴿٩٥﴾
 وَٱللَّهُ خَلَقَكُم وَمَا تَعمَلُونَ ﴿٩٦﴾
 قَالُوا۟ ٱبنُوا۟ لَهُۥ بُنيَـٰنًۭا فَأَلقُوهُ فِى ٱلجَحِيمِ ﴿٩٧﴾
 فَأَرَادُوا۟ بِهِۦ كَيدًۭا فَجَعَلنَـٰهُمُ ٱلأَسفَلِينَ ﴿٩٨﴾
 وَقَالَ إِنِّى ذَاهِبٌ إِلَىٰ رَبِّى سَيَهدِينِ ﴿٩٩﴾
 رَبِّ هَب لِى مِنَ ٱلصَّـٰلِحِينَ ﴿١٠٠﴾
 فَبَشَّرنَـٰهُ بِغُلَـٰمٍ حَلِيمٍۢ ﴿١٠١﴾
 فَلَمَّا بَلَغَ مَعَهُ ٱلسَّعىَ قَالَ يَـٰبُنَىَّ إِنِّىٓ أَرَىٰ فِى ٱلمَنَامِ أَنِّىٓ أَذبَحُكَ فَٱنظُر مَاذَا تَرَىٰ ۚ قَالَ يَـٰٓأَبَتِ ٱفعَل مَا تُؤمَرُ ۖ سَتَجِدُنِىٓ إِن شَآءَ ٱللَّهُ مِنَ ٱلصَّـٰبِرِينَ ﴿١٠٢﴾
 فَلَمَّآ أَسلَمَا وَتَلَّهُۥ لِلجَبِينِ ﴿١٠٣﴾
 وَنَـٰدَينَـٰهُ أَن يَـٰٓإِبرَٰهِيمُ ﴿١٠٤﴾
 قَد صَدَّقتَ ٱلرُّءيَآ ۚ إِنَّا كَذَٟلِكَ نَجزِى ٱلمُحسِنِينَ ﴿١٠٥﴾
 إِنَّ هَـٰذَا لَهُوَ ٱلبَلَـٰٓؤُا۟ ٱلمُبِينُ ﴿١٠٦﴾
 وَفَدَينَـٰهُ بِذِبحٍ عَظِيمٍۢ ﴿١٠٧﴾
 وَتَرَكنَا عَلَيهِ فِى ٱلءَاخِرِينَ ﴿١٠٨﴾
 سَلَـٰمٌ عَلَىٰٓ إِبرَٰهِيمَ ﴿١٠٩﴾
 كَذَٟلِكَ نَجزِى ٱلمُحسِنِينَ ﴿١١٠﴾
 إِنَّهُۥ مِن عِبَادِنَا ٱلمُؤمِنِينَ ﴿١١١﴾
 وَبَشَّرنَـٰهُ بِإِسحَـٰقَ نَبِيًّۭا مِّنَ ٱلصَّـٰلِحِينَ ﴿١١٢﴾
 وَبَٰرَكنَا عَلَيهِ وَعَلَىٰٓ إِسحَـٰقَ ۚ وَمِن ذُرِّيَّتِهِمَا مُحسِنٌۭ وَظَالِمٌۭ لِّنَفسِهِۦ مُبِينٌۭ ﴿١١٣﴾
 وَلَقَد مَنَنَّا عَلَىٰ مُوسَىٰ وَهَـٰرُونَ ﴿١١٤﴾
 وَنَجَّينَـٰهُمَا وَقَومَهُمَا مِنَ ٱلكَربِ ٱلعَظِيمِ ﴿١١٥﴾
 وَنَصَرنَـٰهُم فَكَانُوا۟ هُمُ ٱلغَٰلِبِينَ ﴿١١٦﴾
 وَءَاتَينَـٰهُمَا ٱلكِتَـٰبَ ٱلمُستَبِينَ ﴿١١٧﴾
 وَهَدَينَـٰهُمَا ٱلصِّرَٰطَ ٱلمُستَقِيمَ ﴿١١٨﴾
 وَتَرَكنَا عَلَيهِمَا فِى ٱلءَاخِرِينَ ﴿١١٩﴾
 سَلَـٰمٌ عَلَىٰ مُوسَىٰ وَهَـٰرُونَ ﴿١٢٠﴾
 إِنَّا كَذَٟلِكَ نَجزِى ٱلمُحسِنِينَ ﴿١٢١﴾
 إِنَّهُمَا مِن عِبَادِنَا ٱلمُؤمِنِينَ ﴿١٢٢﴾
 وَإِنَّ إِليَاسَ لَمِنَ ٱلمُرسَلِينَ ﴿١٢٣﴾
 إِذ قَالَ لِقَومِهِۦٓ أَلَا تَتَّقُونَ ﴿١٢٤﴾
 أَتَدعُونَ بَعلًۭا وَتَذَرُونَ أَحسَنَ ٱلخَـٰلِقِينَ ﴿١٢٥﴾
 ٱللَّهَ رَبَّكُم وَرَبَّ ءَابَآئِكُمُ ٱلأَوَّلِينَ ﴿١٢٦﴾
 فَكَذَّبُوهُ فَإِنَّهُم لَمُحضَرُونَ ﴿١٢٧﴾
 إِلَّا عِبَادَ ٱللَّهِ ٱلمُخلَصِينَ ﴿١٢٨﴾
 وَتَرَكنَا عَلَيهِ فِى ٱلءَاخِرِينَ ﴿١٢٩﴾
 سَلَـٰمٌ عَلَىٰٓ إِل يَاسِينَ ﴿١٣٠﴾
 إِنَّا كَذَٟلِكَ نَجزِى ٱلمُحسِنِينَ ﴿١٣١﴾
 إِنَّهُۥ مِن عِبَادِنَا ٱلمُؤمِنِينَ ﴿١٣٢﴾
 وَإِنَّ لُوطًۭا لَّمِنَ ٱلمُرسَلِينَ ﴿١٣٣﴾
 إِذ نَجَّينَـٰهُ وَأَهلَهُۥٓ أَجمَعِينَ ﴿١٣٤﴾
 إِلَّا عَجُوزًۭا فِى ٱلغَٰبِرِينَ ﴿١٣٥﴾
 ثُمَّ دَمَّرنَا ٱلءَاخَرِينَ ﴿١٣٦﴾
 وَإِنَّكُم لَتَمُرُّونَ عَلَيهِم مُّصبِحِينَ ﴿١٣٧﴾
 وَبِٱلَّيلِ ۗ أَفَلَا تَعقِلُونَ ﴿١٣٨﴾
 وَإِنَّ يُونُسَ لَمِنَ ٱلمُرسَلِينَ ﴿١٣٩﴾
 إِذ أَبَقَ إِلَى ٱلفُلكِ ٱلمَشحُونِ ﴿١٤٠﴾
 فَسَاهَمَ فَكَانَ مِنَ ٱلمُدحَضِينَ ﴿١٤١﴾
 فَٱلتَقَمَهُ ٱلحُوتُ وَهُوَ مُلِيمٌۭ ﴿١٤٢﴾
 فَلَولَآ أَنَّهُۥ كَانَ مِنَ ٱلمُسَبِّحِينَ ﴿١٤٣﴾
 لَلَبِثَ فِى بَطنِهِۦٓ إِلَىٰ يَومِ يُبعَثُونَ ﴿١٤٤﴾
 ۞ فَنَبَذنَـٰهُ بِٱلعَرَآءِ وَهُوَ سَقِيمٌۭ ﴿١٤٥﴾
 وَأَنۢبَتنَا عَلَيهِ شَجَرَةًۭ مِّن يَقطِينٍۢ ﴿١٤٦﴾
 وَأَرسَلنَـٰهُ إِلَىٰ مِا۟ئَةِ أَلفٍ أَو يَزِيدُونَ ﴿١٤٧﴾
 فَـَٔامَنُوا۟ فَمَتَّعنَـٰهُم إِلَىٰ حِينٍۢ ﴿١٤٨﴾
 فَٱستَفتِهِم أَلِرَبِّكَ ٱلبَنَاتُ وَلَهُمُ ٱلبَنُونَ ﴿١٤٩﴾
 أَم خَلَقنَا ٱلمَلَـٰٓئِكَةَ إِنَـٰثًۭا وَهُم شَـٰهِدُونَ ﴿١٥٠﴾
 أَلَآ إِنَّهُم مِّن إِفكِهِم لَيَقُولُونَ ﴿١٥١﴾
 وَلَدَ ٱللَّهُ وَإِنَّهُم لَكَـٰذِبُونَ ﴿١٥٢﴾
 أَصطَفَى ٱلبَنَاتِ عَلَى ٱلبَنِينَ ﴿١٥٣﴾
 مَا لَكُم كَيفَ تَحكُمُونَ ﴿١٥٤﴾
 أَفَلَا تَذَكَّرُونَ ﴿١٥٥﴾
 أَم لَكُم سُلطَٰنٌۭ مُّبِينٌۭ ﴿١٥٦﴾
 فَأتُوا۟ بِكِتَـٰبِكُم إِن كُنتُم صَـٰدِقِينَ ﴿١٥٧﴾
 وَجَعَلُوا۟ بَينَهُۥ وَبَينَ ٱلجِنَّةِ نَسَبًۭا ۚ وَلَقَد عَلِمَتِ ٱلجِنَّةُ إِنَّهُم لَمُحضَرُونَ ﴿١٥٨﴾
 سُبحَـٰنَ ٱللَّهِ عَمَّا يَصِفُونَ ﴿١٥٩﴾
 إِلَّا عِبَادَ ٱللَّهِ ٱلمُخلَصِينَ ﴿١٦٠﴾
 فَإِنَّكُم وَمَا تَعبُدُونَ ﴿١٦١﴾
 مَآ أَنتُم عَلَيهِ بِفَـٰتِنِينَ ﴿١٦٢﴾
 إِلَّا مَن هُوَ صَالِ ٱلجَحِيمِ ﴿١٦٣﴾
 وَمَا مِنَّآ إِلَّا لَهُۥ مَقَامٌۭ مَّعلُومٌۭ ﴿١٦٤﴾
 وَإِنَّا لَنَحنُ ٱلصَّآفُّونَ ﴿١٦٥﴾
 وَإِنَّا لَنَحنُ ٱلمُسَبِّحُونَ ﴿١٦٦﴾
 وَإِن كَانُوا۟ لَيَقُولُونَ ﴿١٦٧﴾
 لَو أَنَّ عِندَنَا ذِكرًۭا مِّنَ ٱلأَوَّلِينَ ﴿١٦٨﴾
 لَكُنَّا عِبَادَ ٱللَّهِ ٱلمُخلَصِينَ ﴿١٦٩﴾
 فَكَفَرُوا۟ بِهِۦ ۖ فَسَوفَ يَعلَمُونَ ﴿١٧٠﴾
 وَلَقَد سَبَقَت كَلِمَتُنَا لِعِبَادِنَا ٱلمُرسَلِينَ ﴿١٧١﴾
 إِنَّهُم لَهُمُ ٱلمَنصُورُونَ ﴿١٧٢﴾
 وَإِنَّ جُندَنَا لَهُمُ ٱلغَٰلِبُونَ ﴿١٧٣﴾
 فَتَوَلَّ عَنهُم حَتَّىٰ حِينٍۢ ﴿١٧٤﴾
 وَأَبصِرهُم فَسَوفَ يُبصِرُونَ ﴿١٧٥﴾
 أَفَبِعَذَابِنَا يَستَعجِلُونَ ﴿١٧٦﴾
 فَإِذَا نَزَلَ بِسَاحَتِهِم فَسَآءَ صَبَاحُ ٱلمُنذَرِينَ ﴿١٧٧﴾
 وَتَوَلَّ عَنهُم حَتَّىٰ حِينٍۢ ﴿١٧٨﴾
 وَأَبصِر فَسَوفَ يُبصِرُونَ ﴿١٧٩﴾
 سُبحَـٰنَ رَبِّكَ رَبِّ ٱلعِزَّةِ عَمَّا يَصِفُونَ ﴿١٨٠﴾
 وَسَلَـٰمٌ عَلَى ٱلمُرسَلِينَ ﴿١٨١﴾
 وَٱلحَمدُ لِلَّهِ رَبِّ ٱلعَـٰلَمِينَ ﴿١٨٢﴾
 
