%% License: BSD style (Berkley) (i.e. Put the Copyright owner's name always)
%% Writer and Copyright (to): Bewketu(Bilal) Tadilo (2016-17)
\shadowbox{\section{\LR{\textamharic{ሱራቱ አልዋቂያ -}  \RL{سوره  الواقعة}}}}

  
    
  
    
    

\nopagebreak
  بِسمِ ٱللَّهِ ٱلرَّحمَـٰنِ ٱلرَّحِيمِ
  إِذَا وَقَعَتِ ٱلوَاقِعَةُ ﴿١﴾
 لَيسَ لِوَقعَتِهَا كَاذِبَةٌ ﴿٢﴾
 خَافِضَةٌۭ رَّافِعَةٌ ﴿٣﴾
 إِذَا رُجَّتِ ٱلأَرضُ رَجًّۭا ﴿٤﴾
 وَبُسَّتِ ٱلجِبَالُ بَسًّۭا ﴿٥﴾
 فَكَانَت هَبَآءًۭ مُّنۢبَثًّۭا ﴿٦﴾
 وَكُنتُم أَزوَٟجًۭا ثَلَـٰثَةًۭ ﴿٧﴾
 فَأَصحَـٰبُ ٱلمَيمَنَةِ مَآ أَصحَـٰبُ ٱلمَيمَنَةِ ﴿٨﴾
 وَأَصحَـٰبُ ٱلمَشـَٔمَةِ مَآ أَصحَـٰبُ ٱلمَشـَٔمَةِ ﴿٩﴾
 وَٱلسَّٰبِقُونَ ٱلسَّٰبِقُونَ ﴿١٠﴾
 أُو۟لَـٰٓئِكَ ٱلمُقَرَّبُونَ ﴿١١﴾
 فِى جَنَّـٰتِ ٱلنَّعِيمِ ﴿١٢﴾
 ثُلَّةٌۭ مِّنَ ٱلأَوَّلِينَ ﴿١٣﴾
 وَقَلِيلٌۭ مِّنَ ٱلءَاخِرِينَ ﴿١٤﴾
 عَلَىٰ سُرُرٍۢ مَّوضُونَةٍۢ ﴿١٥﴾
 مُّتَّكِـِٔينَ عَلَيهَا مُتَقَـٰبِلِينَ ﴿١٦﴾
 يَطُوفُ عَلَيهِم وِلدَٟنٌۭ مُّخَلَّدُونَ ﴿١٧﴾
 بِأَكوَابٍۢ وَأَبَارِيقَ وَكَأسٍۢ مِّن مَّعِينٍۢ ﴿١٨﴾
 لَّا يُصَدَّعُونَ عَنهَا وَلَا يُنزِفُونَ ﴿١٩﴾
 وَفَـٰكِهَةٍۢ مِّمَّا يَتَخَيَّرُونَ ﴿٢٠﴾
 وَلَحمِ طَيرٍۢ مِّمَّا يَشتَهُونَ ﴿٢١﴾
 وَحُورٌ عِينٌۭ ﴿٢٢﴾
 كَأَمثَـٰلِ ٱللُّؤلُؤِ ٱلمَكنُونِ ﴿٢٣﴾
 جَزَآءًۢ بِمَا كَانُوا۟ يَعمَلُونَ ﴿٢٤﴾
 لَا يَسمَعُونَ فِيهَا لَغوًۭا وَلَا تَأثِيمًا ﴿٢٥﴾
 إِلَّا قِيلًۭا سَلَـٰمًۭا سَلَـٰمًۭا ﴿٢٦﴾
 وَأَصحَـٰبُ ٱليَمِينِ مَآ أَصحَـٰبُ ٱليَمِينِ ﴿٢٧﴾
 فِى سِدرٍۢ مَّخضُودٍۢ ﴿٢٨﴾
 وَطَلحٍۢ مَّنضُودٍۢ ﴿٢٩﴾
 وَظِلٍّۢ مَّمدُودٍۢ ﴿٣٠﴾
 وَمَآءٍۢ مَّسكُوبٍۢ ﴿٣١﴾
 وَفَـٰكِهَةٍۢ كَثِيرَةٍۢ ﴿٣٢﴾
 لَّا مَقطُوعَةٍۢ وَلَا مَمنُوعَةٍۢ ﴿٣٣﴾
 وَفُرُشٍۢ مَّرفُوعَةٍ ﴿٣٤﴾
 إِنَّآ أَنشَأنَـٰهُنَّ إِنشَآءًۭ ﴿٣٥﴾
 فَجَعَلنَـٰهُنَّ أَبكَارًا ﴿٣٦﴾
 عُرُبًا أَترَابًۭا ﴿٣٧﴾
 لِّأَصحَـٰبِ ٱليَمِينِ ﴿٣٨﴾
 ثُلَّةٌۭ مِّنَ ٱلأَوَّلِينَ ﴿٣٩﴾
 وَثُلَّةٌۭ مِّنَ ٱلءَاخِرِينَ ﴿٤٠﴾
 وَأَصحَـٰبُ ٱلشِّمَالِ مَآ أَصحَـٰبُ ٱلشِّمَالِ ﴿٤١﴾
 فِى سَمُومٍۢ وَحَمِيمٍۢ ﴿٤٢﴾
 وَظِلٍّۢ مِّن يَحمُومٍۢ ﴿٤٣﴾
 لَّا بَارِدٍۢ وَلَا كَرِيمٍ ﴿٤٤﴾
 إِنَّهُم كَانُوا۟ قَبلَ ذَٟلِكَ مُترَفِينَ ﴿٤٥﴾
 وَكَانُوا۟ يُصِرُّونَ عَلَى ٱلحِنثِ ٱلعَظِيمِ ﴿٤٦﴾
 وَكَانُوا۟ يَقُولُونَ أَئِذَا مِتنَا وَكُنَّا تُرَابًۭا وَعِظَـٰمًا أَءِنَّا لَمَبعُوثُونَ ﴿٤٧﴾
 أَوَءَابَآؤُنَا ٱلأَوَّلُونَ ﴿٤٨﴾
 قُل إِنَّ ٱلأَوَّلِينَ وَٱلءَاخِرِينَ ﴿٤٩﴾
 لَمَجمُوعُونَ إِلَىٰ مِيقَـٰتِ يَومٍۢ مَّعلُومٍۢ ﴿٥٠﴾
 ثُمَّ إِنَّكُم أَيُّهَا ٱلضَّآلُّونَ ٱلمُكَذِّبُونَ ﴿٥١﴾
 لَءَاكِلُونَ مِن شَجَرٍۢ مِّن زَقُّومٍۢ ﴿٥٢﴾
 فَمَالِـُٔونَ مِنهَا ٱلبُطُونَ ﴿٥٣﴾
 فَشَـٰرِبُونَ عَلَيهِ مِنَ ٱلحَمِيمِ ﴿٥٤﴾
 فَشَـٰرِبُونَ شُربَ ٱلهِيمِ ﴿٥٥﴾
 هَـٰذَا نُزُلُهُم يَومَ ٱلدِّينِ ﴿٥٦﴾
 نَحنُ خَلَقنَـٰكُم فَلَولَا تُصَدِّقُونَ ﴿٥٧﴾
 أَفَرَءَيتُم مَّا تُمنُونَ ﴿٥٨﴾
 ءَأَنتُم تَخلُقُونَهُۥٓ أَم نَحنُ ٱلخَـٰلِقُونَ ﴿٥٩﴾
 نَحنُ قَدَّرنَا بَينَكُمُ ٱلمَوتَ وَمَا نَحنُ بِمَسبُوقِينَ ﴿٦٠﴾
 عَلَىٰٓ أَن نُّبَدِّلَ أَمثَـٰلَكُم وَنُنشِئَكُم فِى مَا لَا تَعلَمُونَ ﴿٦١﴾
 وَلَقَد عَلِمتُمُ ٱلنَّشأَةَ ٱلأُولَىٰ فَلَولَا تَذَكَّرُونَ ﴿٦٢﴾
 أَفَرَءَيتُم مَّا تَحرُثُونَ ﴿٦٣﴾
 ءَأَنتُم تَزرَعُونَهُۥٓ أَم نَحنُ ٱلزَّٰرِعُونَ ﴿٦٤﴾
 لَو نَشَآءُ لَجَعَلنَـٰهُ حُطَٰمًۭا فَظَلتُم تَفَكَّهُونَ ﴿٦٥﴾
 إِنَّا لَمُغرَمُونَ ﴿٦٦﴾
 بَل نَحنُ مَحرُومُونَ ﴿٦٧﴾
 أَفَرَءَيتُمُ ٱلمَآءَ ٱلَّذِى تَشرَبُونَ ﴿٦٨﴾
 ءَأَنتُم أَنزَلتُمُوهُ مِنَ ٱلمُزنِ أَم نَحنُ ٱلمُنزِلُونَ ﴿٦٩﴾
 لَو نَشَآءُ جَعَلنَـٰهُ أُجَاجًۭا فَلَولَا تَشكُرُونَ ﴿٧٠﴾
 أَفَرَءَيتُمُ ٱلنَّارَ ٱلَّتِى تُورُونَ ﴿٧١﴾
 ءَأَنتُم أَنشَأتُم شَجَرَتَهَآ أَم نَحنُ ٱلمُنشِـُٔونَ ﴿٧٢﴾
 نَحنُ جَعَلنَـٰهَا تَذكِرَةًۭ وَمَتَـٰعًۭا لِّلمُقوِينَ ﴿٧٣﴾
 فَسَبِّح بِٱسمِ رَبِّكَ ٱلعَظِيمِ ﴿٧٤﴾
 ۞ فَلَآ أُقسِمُ بِمَوَٟقِعِ ٱلنُّجُومِ ﴿٧٥﴾
 وَإِنَّهُۥ لَقَسَمٌۭ لَّو تَعلَمُونَ عَظِيمٌ ﴿٧٦﴾
 إِنَّهُۥ لَقُرءَانٌۭ كَرِيمٌۭ ﴿٧٧﴾
 فِى كِتَـٰبٍۢ مَّكنُونٍۢ ﴿٧٨﴾
 لَّا يَمَسُّهُۥٓ إِلَّا ٱلمُطَهَّرُونَ ﴿٧٩﴾
 تَنزِيلٌۭ مِّن رَّبِّ ٱلعَـٰلَمِينَ ﴿٨٠﴾
 أَفَبِهَـٰذَا ٱلحَدِيثِ أَنتُم مُّدهِنُونَ ﴿٨١﴾
 وَتَجعَلُونَ رِزقَكُم أَنَّكُم تُكَذِّبُونَ ﴿٨٢﴾
 فَلَولَآ إِذَا بَلَغَتِ ٱلحُلقُومَ ﴿٨٣﴾
 وَأَنتُم حِينَئِذٍۢ تَنظُرُونَ ﴿٨٤﴾
 وَنَحنُ أَقرَبُ إِلَيهِ مِنكُم وَلَـٰكِن لَّا تُبصِرُونَ ﴿٨٥﴾
 فَلَولَآ إِن كُنتُم غَيرَ مَدِينِينَ ﴿٨٦﴾
 تَرجِعُونَهَآ إِن كُنتُم صَـٰدِقِينَ ﴿٨٧﴾
 فَأَمَّآ إِن كَانَ مِنَ ٱلمُقَرَّبِينَ ﴿٨٨﴾
 فَرَوحٌۭ وَرَيحَانٌۭ وَجَنَّتُ نَعِيمٍۢ ﴿٨٩﴾
 وَأَمَّآ إِن كَانَ مِن أَصحَـٰبِ ٱليَمِينِ ﴿٩٠﴾
 فَسَلَـٰمٌۭ لَّكَ مِن أَصحَـٰبِ ٱليَمِينِ ﴿٩١﴾
 وَأَمَّآ إِن كَانَ مِنَ ٱلمُكَذِّبِينَ ٱلضَّآلِّينَ ﴿٩٢﴾
 فَنُزُلٌۭ مِّن حَمِيمٍۢ ﴿٩٣﴾
 وَتَصلِيَةُ جَحِيمٍ ﴿٩٤﴾
 إِنَّ هَـٰذَا لَهُوَ حَقُّ ٱليَقِينِ ﴿٩٥﴾
 فَسَبِّح بِٱسمِ رَبِّكَ ٱلعَظِيمِ ﴿٩٦﴾
 
