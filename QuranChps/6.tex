%% License: BSD style (Berkley) (i.e. Put the Copyright owner's name always)
%% Writer and Copyright (to): Bewketu(Bilal) Tadilo (2016-17)
\shadowbox{\section{\LR{\textamharic{ሱራቱ አልአነኣም -}  \RL{سوره  الأنعام}}}}

  
    
  
    
    

\nopagebreak
  بِسمِ ٱللَّهِ ٱلرَّحمَـٰنِ ٱلرَّحِيمِ
  ٱلحَمدُ لِلَّهِ ٱلَّذِى خَلَقَ ٱلسَّمَـٰوَٟتِ وَٱلأَرضَ وَجَعَلَ ٱلظُّلُمَـٰتِ وَٱلنُّورَ ۖ ثُمَّ ٱلَّذِينَ كَفَرُوا۟ بِرَبِّهِم يَعدِلُونَ ﴿١﴾
 هُوَ ٱلَّذِى خَلَقَكُم مِّن طِينٍۢ ثُمَّ قَضَىٰٓ أَجَلًۭا ۖ وَأَجَلٌۭ مُّسَمًّى عِندَهُۥ ۖ ثُمَّ أَنتُم تَمتَرُونَ ﴿٢﴾
 وَهُوَ ٱللَّهُ فِى ٱلسَّمَـٰوَٟتِ وَفِى ٱلأَرضِ ۖ يَعلَمُ سِرَّكُم وَجَهرَكُم وَيَعلَمُ مَا تَكسِبُونَ ﴿٣﴾
 وَمَا تَأتِيهِم مِّن ءَايَةٍۢ مِّن ءَايَـٰتِ رَبِّهِم إِلَّا كَانُوا۟ عَنهَا مُعرِضِينَ ﴿٤﴾
 فَقَد كَذَّبُوا۟ بِٱلحَقِّ لَمَّا جَآءَهُم ۖ فَسَوفَ يَأتِيهِم أَنۢبَٰٓؤُا۟ مَا كَانُوا۟ بِهِۦ يَستَهزِءُونَ ﴿٥﴾
 أَلَم يَرَوا۟ كَم أَهلَكنَا مِن قَبلِهِم مِّن قَرنٍۢ مَّكَّنَّـٰهُم فِى ٱلأَرضِ مَا لَم نُمَكِّن لَّكُم وَأَرسَلنَا ٱلسَّمَآءَ عَلَيهِم مِّدرَارًۭا وَجَعَلنَا ٱلأَنهَـٰرَ تَجرِى مِن تَحتِهِم فَأَهلَكنَـٰهُم بِذُنُوبِهِم وَأَنشَأنَا مِنۢ بَعدِهِم قَرنًا ءَاخَرِينَ ﴿٦﴾
 وَلَو نَزَّلنَا عَلَيكَ كِتَـٰبًۭا فِى قِرطَاسٍۢ فَلَمَسُوهُ بِأَيدِيهِم لَقَالَ ٱلَّذِينَ كَفَرُوٓا۟ إِن هَـٰذَآ إِلَّا سِحرٌۭ مُّبِينٌۭ ﴿٧﴾
 وَقَالُوا۟ لَولَآ أُنزِلَ عَلَيهِ مَلَكٌۭ ۖ وَلَو أَنزَلنَا مَلَكًۭا لَّقُضِىَ ٱلأَمرُ ثُمَّ لَا يُنظَرُونَ ﴿٨﴾
 وَلَو جَعَلنَـٰهُ مَلَكًۭا لَّجَعَلنَـٰهُ رَجُلًۭا وَلَلَبَسنَا عَلَيهِم مَّا يَلبِسُونَ ﴿٩﴾
 وَلَقَدِ ٱستُهزِئَ بِرُسُلٍۢ مِّن قَبلِكَ فَحَاقَ بِٱلَّذِينَ سَخِرُوا۟ مِنهُم مَّا كَانُوا۟ بِهِۦ يَستَهزِءُونَ ﴿١٠﴾
 قُل سِيرُوا۟ فِى ٱلأَرضِ ثُمَّ ٱنظُرُوا۟ كَيفَ كَانَ عَـٰقِبَةُ ٱلمُكَذِّبِينَ ﴿١١﴾
 قُل لِّمَن مَّا فِى ٱلسَّمَـٰوَٟتِ وَٱلأَرضِ ۖ قُل لِّلَّهِ ۚ كَتَبَ عَلَىٰ نَفسِهِ ٱلرَّحمَةَ ۚ لَيَجمَعَنَّكُم إِلَىٰ يَومِ ٱلقِيَـٰمَةِ لَا رَيبَ فِيهِ ۚ ٱلَّذِينَ خَسِرُوٓا۟ أَنفُسَهُم فَهُم لَا يُؤمِنُونَ ﴿١٢﴾
 ۞ وَلَهُۥ مَا سَكَنَ فِى ٱلَّيلِ وَٱلنَّهَارِ ۚ وَهُوَ ٱلسَّمِيعُ ٱلعَلِيمُ ﴿١٣﴾
 قُل أَغَيرَ ٱللَّهِ أَتَّخِذُ وَلِيًّۭا فَاطِرِ ٱلسَّمَـٰوَٟتِ وَٱلأَرضِ وَهُوَ يُطعِمُ وَلَا يُطعَمُ ۗ قُل إِنِّىٓ أُمِرتُ أَن أَكُونَ أَوَّلَ مَن أَسلَمَ ۖ وَلَا تَكُونَنَّ مِنَ ٱلمُشرِكِينَ ﴿١٤﴾
 قُل إِنِّىٓ أَخَافُ إِن عَصَيتُ رَبِّى عَذَابَ يَومٍ عَظِيمٍۢ ﴿١٥﴾
 مَّن يُصرَف عَنهُ يَومَئِذٍۢ فَقَد رَحِمَهُۥ ۚ وَذَٟلِكَ ٱلفَوزُ ٱلمُبِينُ ﴿١٦﴾
 وَإِن يَمسَسكَ ٱللَّهُ بِضُرٍّۢ فَلَا كَاشِفَ لَهُۥٓ إِلَّا هُوَ ۖ وَإِن يَمسَسكَ بِخَيرٍۢ فَهُوَ عَلَىٰ كُلِّ شَىءٍۢ قَدِيرٌۭ ﴿١٧﴾
 وَهُوَ ٱلقَاهِرُ فَوقَ عِبَادِهِۦ ۚ وَهُوَ ٱلحَكِيمُ ٱلخَبِيرُ ﴿١٨﴾
 قُل أَىُّ شَىءٍ أَكبَرُ شَهَـٰدَةًۭ ۖ قُلِ ٱللَّهُ ۖ شَهِيدٌۢ بَينِى وَبَينَكُم ۚ وَأُوحِىَ إِلَىَّ هَـٰذَا ٱلقُرءَانُ لِأُنذِرَكُم بِهِۦ وَمَنۢ بَلَغَ ۚ أَئِنَّكُم لَتَشهَدُونَ أَنَّ مَعَ ٱللَّهِ ءَالِهَةً أُخرَىٰ ۚ قُل لَّآ أَشهَدُ ۚ قُل إِنَّمَا هُوَ إِلَـٰهٌۭ وَٟحِدٌۭ وَإِنَّنِى بَرِىٓءٌۭ مِّمَّا تُشرِكُونَ ﴿١٩﴾
 ٱلَّذِينَ ءَاتَينَـٰهُمُ ٱلكِتَـٰبَ يَعرِفُونَهُۥ كَمَا يَعرِفُونَ أَبنَآءَهُمُ ۘ ٱلَّذِينَ خَسِرُوٓا۟ أَنفُسَهُم فَهُم لَا يُؤمِنُونَ ﴿٢٠﴾
 وَمَن أَظلَمُ مِمَّنِ ٱفتَرَىٰ عَلَى ٱللَّهِ كَذِبًا أَو كَذَّبَ بِـَٔايَـٰتِهِۦٓ ۗ إِنَّهُۥ لَا يُفلِحُ ٱلظَّـٰلِمُونَ ﴿٢١﴾
 وَيَومَ نَحشُرُهُم جَمِيعًۭا ثُمَّ نَقُولُ لِلَّذِينَ أَشرَكُوٓا۟ أَينَ شُرَكَآؤُكُمُ ٱلَّذِينَ كُنتُم تَزعُمُونَ ﴿٢٢﴾
 ثُمَّ لَم تَكُن فِتنَتُهُم إِلَّآ أَن قَالُوا۟ وَٱللَّهِ رَبِّنَا مَا كُنَّا مُشرِكِينَ ﴿٢٣﴾
 ٱنظُر كَيفَ كَذَبُوا۟ عَلَىٰٓ أَنفُسِهِم ۚ وَضَلَّ عَنهُم مَّا كَانُوا۟ يَفتَرُونَ ﴿٢٤﴾
 وَمِنهُم مَّن يَستَمِعُ إِلَيكَ ۖ وَجَعَلنَا عَلَىٰ قُلُوبِهِم أَكِنَّةً أَن يَفقَهُوهُ وَفِىٓ ءَاذَانِهِم وَقرًۭا ۚ وَإِن يَرَوا۟ كُلَّ ءَايَةٍۢ لَّا يُؤمِنُوا۟ بِهَا ۚ حَتَّىٰٓ إِذَا جَآءُوكَ يُجَٰدِلُونَكَ يَقُولُ ٱلَّذِينَ كَفَرُوٓا۟ إِن هَـٰذَآ إِلَّآ أَسَـٰطِيرُ ٱلأَوَّلِينَ ﴿٢٥﴾
 وَهُم يَنهَونَ عَنهُ وَيَنـَٔونَ عَنهُ ۖ وَإِن يُهلِكُونَ إِلَّآ أَنفُسَهُم وَمَا يَشعُرُونَ ﴿٢٦﴾
 وَلَو تَرَىٰٓ إِذ وُقِفُوا۟ عَلَى ٱلنَّارِ فَقَالُوا۟ يَـٰلَيتَنَا نُرَدُّ وَلَا نُكَذِّبَ بِـَٔايَـٰتِ رَبِّنَا وَنَكُونَ مِنَ ٱلمُؤمِنِينَ ﴿٢٧﴾
 بَل بَدَا لَهُم مَّا كَانُوا۟ يُخفُونَ مِن قَبلُ ۖ وَلَو رُدُّوا۟ لَعَادُوا۟ لِمَا نُهُوا۟ عَنهُ وَإِنَّهُم لَكَـٰذِبُونَ ﴿٢٨﴾
 وَقَالُوٓا۟ إِن هِىَ إِلَّا حَيَاتُنَا ٱلدُّنيَا وَمَا نَحنُ بِمَبعُوثِينَ ﴿٢٩﴾
 وَلَو تَرَىٰٓ إِذ وُقِفُوا۟ عَلَىٰ رَبِّهِم ۚ قَالَ أَلَيسَ هَـٰذَا بِٱلحَقِّ ۚ قَالُوا۟ بَلَىٰ وَرَبِّنَا ۚ قَالَ فَذُوقُوا۟ ٱلعَذَابَ بِمَا كُنتُم تَكفُرُونَ ﴿٣٠﴾
 قَد خَسِرَ ٱلَّذِينَ كَذَّبُوا۟ بِلِقَآءِ ٱللَّهِ ۖ حَتَّىٰٓ إِذَا جَآءَتهُمُ ٱلسَّاعَةُ بَغتَةًۭ قَالُوا۟ يَـٰحَسرَتَنَا عَلَىٰ مَا فَرَّطنَا فِيهَا وَهُم يَحمِلُونَ أَوزَارَهُم عَلَىٰ ظُهُورِهِم ۚ أَلَا سَآءَ مَا يَزِرُونَ ﴿٣١﴾
 وَمَا ٱلحَيَوٰةُ ٱلدُّنيَآ إِلَّا لَعِبٌۭ وَلَهوٌۭ ۖ وَلَلدَّارُ ٱلءَاخِرَةُ خَيرٌۭ لِّلَّذِينَ يَتَّقُونَ ۗ أَفَلَا تَعقِلُونَ ﴿٣٢﴾
 قَد نَعلَمُ إِنَّهُۥ لَيَحزُنُكَ ٱلَّذِى يَقُولُونَ ۖ فَإِنَّهُم لَا يُكَذِّبُونَكَ وَلَـٰكِنَّ ٱلظَّـٰلِمِينَ بِـَٔايَـٰتِ ٱللَّهِ يَجحَدُونَ ﴿٣٣﴾
 وَلَقَد كُذِّبَت رُسُلٌۭ مِّن قَبلِكَ فَصَبَرُوا۟ عَلَىٰ مَا كُذِّبُوا۟ وَأُوذُوا۟ حَتَّىٰٓ أَتَىٰهُم نَصرُنَا ۚ وَلَا مُبَدِّلَ لِكَلِمَـٰتِ ٱللَّهِ ۚ وَلَقَد جَآءَكَ مِن نَّبَإِى۟ ٱلمُرسَلِينَ ﴿٣٤﴾
 وَإِن كَانَ كَبُرَ عَلَيكَ إِعرَاضُهُم فَإِنِ ٱستَطَعتَ أَن تَبتَغِىَ نَفَقًۭا فِى ٱلأَرضِ أَو سُلَّمًۭا فِى ٱلسَّمَآءِ فَتَأتِيَهُم بِـَٔايَةٍۢ ۚ وَلَو شَآءَ ٱللَّهُ لَجَمَعَهُم عَلَى ٱلهُدَىٰ ۚ فَلَا تَكُونَنَّ مِنَ ٱلجَٰهِلِينَ ﴿٣٥﴾
 ۞ إِنَّمَا يَستَجِيبُ ٱلَّذِينَ يَسمَعُونَ ۘ وَٱلمَوتَىٰ يَبعَثُهُمُ ٱللَّهُ ثُمَّ إِلَيهِ يُرجَعُونَ ﴿٣٦﴾
 وَقَالُوا۟ لَولَا نُزِّلَ عَلَيهِ ءَايَةٌۭ مِّن رَّبِّهِۦ ۚ قُل إِنَّ ٱللَّهَ قَادِرٌ عَلَىٰٓ أَن يُنَزِّلَ ءَايَةًۭ وَلَـٰكِنَّ أَكثَرَهُم لَا يَعلَمُونَ ﴿٣٧﴾
 وَمَا مِن دَآبَّةٍۢ فِى ٱلأَرضِ وَلَا طَٰٓئِرٍۢ يَطِيرُ بِجَنَاحَيهِ إِلَّآ أُمَمٌ أَمثَالُكُم ۚ مَّا فَرَّطنَا فِى ٱلكِتَـٰبِ مِن شَىءٍۢ ۚ ثُمَّ إِلَىٰ رَبِّهِم يُحشَرُونَ ﴿٣٨﴾
 وَٱلَّذِينَ كَذَّبُوا۟ بِـَٔايَـٰتِنَا صُمٌّۭ وَبُكمٌۭ فِى ٱلظُّلُمَـٰتِ ۗ مَن يَشَإِ ٱللَّهُ يُضلِلهُ وَمَن يَشَأ يَجعَلهُ عَلَىٰ صِرَٰطٍۢ مُّستَقِيمٍۢ ﴿٣٩﴾
 قُل أَرَءَيتَكُم إِن أَتَىٰكُم عَذَابُ ٱللَّهِ أَو أَتَتكُمُ ٱلسَّاعَةُ أَغَيرَ ٱللَّهِ تَدعُونَ إِن كُنتُم صَـٰدِقِينَ ﴿٤٠﴾
 بَل إِيَّاهُ تَدعُونَ فَيَكشِفُ مَا تَدعُونَ إِلَيهِ إِن شَآءَ وَتَنسَونَ مَا تُشرِكُونَ ﴿٤١﴾
 وَلَقَد أَرسَلنَآ إِلَىٰٓ أُمَمٍۢ مِّن قَبلِكَ فَأَخَذنَـٰهُم بِٱلبَأسَآءِ وَٱلضَّرَّآءِ لَعَلَّهُم يَتَضَرَّعُونَ ﴿٤٢﴾
 فَلَولَآ إِذ جَآءَهُم بَأسُنَا تَضَرَّعُوا۟ وَلَـٰكِن قَسَت قُلُوبُهُم وَزَيَّنَ لَهُمُ ٱلشَّيطَٰنُ مَا كَانُوا۟ يَعمَلُونَ ﴿٤٣﴾
 فَلَمَّا نَسُوا۟ مَا ذُكِّرُوا۟ بِهِۦ فَتَحنَا عَلَيهِم أَبوَٟبَ كُلِّ شَىءٍ حَتَّىٰٓ إِذَا فَرِحُوا۟ بِمَآ أُوتُوٓا۟ أَخَذنَـٰهُم بَغتَةًۭ فَإِذَا هُم مُّبلِسُونَ ﴿٤٤﴾
 فَقُطِعَ دَابِرُ ٱلقَومِ ٱلَّذِينَ ظَلَمُوا۟ ۚ وَٱلحَمدُ لِلَّهِ رَبِّ ٱلعَـٰلَمِينَ ﴿٤٥﴾
 قُل أَرَءَيتُم إِن أَخَذَ ٱللَّهُ سَمعَكُم وَأَبصَـٰرَكُم وَخَتَمَ عَلَىٰ قُلُوبِكُم مَّن إِلَـٰهٌ غَيرُ ٱللَّهِ يَأتِيكُم بِهِ ۗ ٱنظُر كَيفَ نُصَرِّفُ ٱلءَايَـٰتِ ثُمَّ هُم يَصدِفُونَ ﴿٤٦﴾
 قُل أَرَءَيتَكُم إِن أَتَىٰكُم عَذَابُ ٱللَّهِ بَغتَةً أَو جَهرَةً هَل يُهلَكُ إِلَّا ٱلقَومُ ٱلظَّـٰلِمُونَ ﴿٤٧﴾
 وَمَا نُرسِلُ ٱلمُرسَلِينَ إِلَّا مُبَشِّرِينَ وَمُنذِرِينَ ۖ فَمَن ءَامَنَ وَأَصلَحَ فَلَا خَوفٌ عَلَيهِم وَلَا هُم يَحزَنُونَ ﴿٤٨﴾
 وَٱلَّذِينَ كَذَّبُوا۟ بِـَٔايَـٰتِنَا يَمَسُّهُمُ ٱلعَذَابُ بِمَا كَانُوا۟ يَفسُقُونَ ﴿٤٩﴾
 قُل لَّآ أَقُولُ لَكُم عِندِى خَزَآئِنُ ٱللَّهِ وَلَآ أَعلَمُ ٱلغَيبَ وَلَآ أَقُولُ لَكُم إِنِّى مَلَكٌ ۖ إِن أَتَّبِعُ إِلَّا مَا يُوحَىٰٓ إِلَىَّ ۚ قُل هَل يَستَوِى ٱلأَعمَىٰ وَٱلبَصِيرُ ۚ أَفَلَا تَتَفَكَّرُونَ ﴿٥٠﴾
 وَأَنذِر بِهِ ٱلَّذِينَ يَخَافُونَ أَن يُحشَرُوٓا۟ إِلَىٰ رَبِّهِم ۙ لَيسَ لَهُم مِّن دُونِهِۦ وَلِىٌّۭ وَلَا شَفِيعٌۭ لَّعَلَّهُم يَتَّقُونَ ﴿٥١﴾
 وَلَا تَطرُدِ ٱلَّذِينَ يَدعُونَ رَبَّهُم بِٱلغَدَوٰةِ وَٱلعَشِىِّ يُرِيدُونَ وَجهَهُۥ ۖ مَا عَلَيكَ مِن حِسَابِهِم مِّن شَىءٍۢ وَمَا مِن حِسَابِكَ عَلَيهِم مِّن شَىءٍۢ فَتَطرُدَهُم فَتَكُونَ مِنَ ٱلظَّـٰلِمِينَ ﴿٥٢﴾
 وَكَذَٟلِكَ فَتَنَّا بَعضَهُم بِبَعضٍۢ لِّيَقُولُوٓا۟ أَهَـٰٓؤُلَآءِ مَنَّ ٱللَّهُ عَلَيهِم مِّنۢ بَينِنَآ ۗ أَلَيسَ ٱللَّهُ بِأَعلَمَ بِٱلشَّـٰكِرِينَ ﴿٥٣﴾
 وَإِذَا جَآءَكَ ٱلَّذِينَ يُؤمِنُونَ بِـَٔايَـٰتِنَا فَقُل سَلَـٰمٌ عَلَيكُم ۖ كَتَبَ رَبُّكُم عَلَىٰ نَفسِهِ ٱلرَّحمَةَ ۖ أَنَّهُۥ مَن عَمِلَ مِنكُم سُوٓءًۢا بِجَهَـٰلَةٍۢ ثُمَّ تَابَ مِنۢ بَعدِهِۦ وَأَصلَحَ فَأَنَّهُۥ غَفُورٌۭ رَّحِيمٌۭ ﴿٥٤﴾
 وَكَذَٟلِكَ نُفَصِّلُ ٱلءَايَـٰتِ وَلِتَستَبِينَ سَبِيلُ ٱلمُجرِمِينَ ﴿٥٥﴾
 قُل إِنِّى نُهِيتُ أَن أَعبُدَ ٱلَّذِينَ تَدعُونَ مِن دُونِ ٱللَّهِ ۚ قُل لَّآ أَتَّبِعُ أَهوَآءَكُم ۙ قَد ضَلَلتُ إِذًۭا وَمَآ أَنَا۠ مِنَ ٱلمُهتَدِينَ ﴿٥٦﴾
 قُل إِنِّى عَلَىٰ بَيِّنَةٍۢ مِّن رَّبِّى وَكَذَّبتُم بِهِۦ ۚ مَا عِندِى مَا تَستَعجِلُونَ بِهِۦٓ ۚ إِنِ ٱلحُكمُ إِلَّا لِلَّهِ ۖ يَقُصُّ ٱلحَقَّ ۖ وَهُوَ خَيرُ ٱلفَـٰصِلِينَ ﴿٥٧﴾
 قُل لَّو أَنَّ عِندِى مَا تَستَعجِلُونَ بِهِۦ لَقُضِىَ ٱلأَمرُ بَينِى وَبَينَكُم ۗ وَٱللَّهُ أَعلَمُ بِٱلظَّـٰلِمِينَ ﴿٥٨﴾
 ۞ وَعِندَهُۥ مَفَاتِحُ ٱلغَيبِ لَا يَعلَمُهَآ إِلَّا هُوَ ۚ وَيَعلَمُ مَا فِى ٱلبَرِّ وَٱلبَحرِ ۚ وَمَا تَسقُطُ مِن وَرَقَةٍ إِلَّا يَعلَمُهَا وَلَا حَبَّةٍۢ فِى ظُلُمَـٰتِ ٱلأَرضِ وَلَا رَطبٍۢ وَلَا يَابِسٍ إِلَّا فِى كِتَـٰبٍۢ مُّبِينٍۢ ﴿٥٩﴾
 وَهُوَ ٱلَّذِى يَتَوَفَّىٰكُم بِٱلَّيلِ وَيَعلَمُ مَا جَرَحتُم بِٱلنَّهَارِ ثُمَّ يَبعَثُكُم فِيهِ لِيُقضَىٰٓ أَجَلٌۭ مُّسَمًّۭى ۖ ثُمَّ إِلَيهِ مَرجِعُكُم ثُمَّ يُنَبِّئُكُم بِمَا كُنتُم تَعمَلُونَ ﴿٦٠﴾
 وَهُوَ ٱلقَاهِرُ فَوقَ عِبَادِهِۦ ۖ وَيُرسِلُ عَلَيكُم حَفَظَةً حَتَّىٰٓ إِذَا جَآءَ أَحَدَكُمُ ٱلمَوتُ تَوَفَّتهُ رُسُلُنَا وَهُم لَا يُفَرِّطُونَ ﴿٦١﴾
 ثُمَّ رُدُّوٓا۟ إِلَى ٱللَّهِ مَولَىٰهُمُ ٱلحَقِّ ۚ أَلَا لَهُ ٱلحُكمُ وَهُوَ أَسرَعُ ٱلحَـٰسِبِينَ ﴿٦٢﴾
 قُل مَن يُنَجِّيكُم مِّن ظُلُمَـٰتِ ٱلبَرِّ وَٱلبَحرِ تَدعُونَهُۥ تَضَرُّعًۭا وَخُفيَةًۭ لَّئِن أَنجَىٰنَا مِن هَـٰذِهِۦ لَنَكُونَنَّ مِنَ ٱلشَّـٰكِرِينَ ﴿٦٣﴾
 قُلِ ٱللَّهُ يُنَجِّيكُم مِّنهَا وَمِن كُلِّ كَربٍۢ ثُمَّ أَنتُم تُشرِكُونَ ﴿٦٤﴾
 قُل هُوَ ٱلقَادِرُ عَلَىٰٓ أَن يَبعَثَ عَلَيكُم عَذَابًۭا مِّن فَوقِكُم أَو مِن تَحتِ أَرجُلِكُم أَو يَلبِسَكُم شِيَعًۭا وَيُذِيقَ بَعضَكُم بَأسَ بَعضٍ ۗ ٱنظُر كَيفَ نُصَرِّفُ ٱلءَايَـٰتِ لَعَلَّهُم يَفقَهُونَ ﴿٦٥﴾
 وَكَذَّبَ بِهِۦ قَومُكَ وَهُوَ ٱلحَقُّ ۚ قُل لَّستُ عَلَيكُم بِوَكِيلٍۢ ﴿٦٦﴾
 لِّكُلِّ نَبَإٍۢ مُّستَقَرٌّۭ ۚ وَسَوفَ تَعلَمُونَ ﴿٦٧﴾
 وَإِذَا رَأَيتَ ٱلَّذِينَ يَخُوضُونَ فِىٓ ءَايَـٰتِنَا فَأَعرِض عَنهُم حَتَّىٰ يَخُوضُوا۟ فِى حَدِيثٍ غَيرِهِۦ ۚ وَإِمَّا يُنسِيَنَّكَ ٱلشَّيطَٰنُ فَلَا تَقعُد بَعدَ ٱلذِّكرَىٰ مَعَ ٱلقَومِ ٱلظَّـٰلِمِينَ ﴿٦٨﴾
 وَمَا عَلَى ٱلَّذِينَ يَتَّقُونَ مِن حِسَابِهِم مِّن شَىءٍۢ وَلَـٰكِن ذِكرَىٰ لَعَلَّهُم يَتَّقُونَ ﴿٦٩﴾
 وَذَرِ ٱلَّذِينَ ٱتَّخَذُوا۟ دِينَهُم لَعِبًۭا وَلَهوًۭا وَغَرَّتهُمُ ٱلحَيَوٰةُ ٱلدُّنيَا ۚ وَذَكِّر بِهِۦٓ أَن تُبسَلَ نَفسٌۢ بِمَا كَسَبَت لَيسَ لَهَا مِن دُونِ ٱللَّهِ وَلِىٌّۭ وَلَا شَفِيعٌۭ وَإِن تَعدِل كُلَّ عَدلٍۢ لَّا يُؤخَذ مِنهَآ ۗ أُو۟لَـٰٓئِكَ ٱلَّذِينَ أُبسِلُوا۟ بِمَا كَسَبُوا۟ ۖ لَهُم شَرَابٌۭ مِّن حَمِيمٍۢ وَعَذَابٌ أَلِيمٌۢ بِمَا كَانُوا۟ يَكفُرُونَ ﴿٧٠﴾
 قُل أَنَدعُوا۟ مِن دُونِ ٱللَّهِ مَا لَا يَنفَعُنَا وَلَا يَضُرُّنَا وَنُرَدُّ عَلَىٰٓ أَعقَابِنَا بَعدَ إِذ هَدَىٰنَا ٱللَّهُ كَٱلَّذِى ٱستَهوَتهُ ٱلشَّيَـٰطِينُ فِى ٱلأَرضِ حَيرَانَ لَهُۥٓ أَصحَـٰبٌۭ يَدعُونَهُۥٓ إِلَى ٱلهُدَى ٱئتِنَا ۗ قُل إِنَّ هُدَى ٱللَّهِ هُوَ ٱلهُدَىٰ ۖ وَأُمِرنَا لِنُسلِمَ لِرَبِّ ٱلعَـٰلَمِينَ ﴿٧١﴾
 وَأَن أَقِيمُوا۟ ٱلصَّلَوٰةَ وَٱتَّقُوهُ ۚ وَهُوَ ٱلَّذِىٓ إِلَيهِ تُحشَرُونَ ﴿٧٢﴾
 وَهُوَ ٱلَّذِى خَلَقَ ٱلسَّمَـٰوَٟتِ وَٱلأَرضَ بِٱلحَقِّ ۖ وَيَومَ يَقُولُ كُن فَيَكُونُ ۚ قَولُهُ ٱلحَقُّ ۚ وَلَهُ ٱلمُلكُ يَومَ يُنفَخُ فِى ٱلصُّورِ ۚ عَـٰلِمُ ٱلغَيبِ وَٱلشَّهَـٰدَةِ ۚ وَهُوَ ٱلحَكِيمُ ٱلخَبِيرُ ﴿٧٣﴾
 ۞ وَإِذ قَالَ إِبرَٰهِيمُ لِأَبِيهِ ءَازَرَ أَتَتَّخِذُ أَصنَامًا ءَالِهَةً ۖ إِنِّىٓ أَرَىٰكَ وَقَومَكَ فِى ضَلَـٰلٍۢ مُّبِينٍۢ ﴿٧٤﴾
 وَكَذَٟلِكَ نُرِىٓ إِبرَٰهِيمَ مَلَكُوتَ ٱلسَّمَـٰوَٟتِ وَٱلأَرضِ وَلِيَكُونَ مِنَ ٱلمُوقِنِينَ ﴿٧٥﴾
 فَلَمَّا جَنَّ عَلَيهِ ٱلَّيلُ رَءَا كَوكَبًۭا ۖ قَالَ هَـٰذَا رَبِّى ۖ فَلَمَّآ أَفَلَ قَالَ لَآ أُحِبُّ ٱلءَافِلِينَ ﴿٧٦﴾
 فَلَمَّا رَءَا ٱلقَمَرَ بَازِغًۭا قَالَ هَـٰذَا رَبِّى ۖ فَلَمَّآ أَفَلَ قَالَ لَئِن لَّم يَهدِنِى رَبِّى لَأَكُونَنَّ مِنَ ٱلقَومِ ٱلضَّآلِّينَ ﴿٧٧﴾
 فَلَمَّا رَءَا ٱلشَّمسَ بَازِغَةًۭ قَالَ هَـٰذَا رَبِّى هَـٰذَآ أَكبَرُ ۖ فَلَمَّآ أَفَلَت قَالَ يَـٰقَومِ إِنِّى بَرِىٓءٌۭ مِّمَّا تُشرِكُونَ ﴿٧٨﴾
 إِنِّى وَجَّهتُ وَجهِىَ لِلَّذِى فَطَرَ ٱلسَّمَـٰوَٟتِ وَٱلأَرضَ حَنِيفًۭا ۖ وَمَآ أَنَا۠ مِنَ ٱلمُشرِكِينَ ﴿٧٩﴾
 وَحَآجَّهُۥ قَومُهُۥ ۚ قَالَ أَتُحَـٰٓجُّوٓنِّى فِى ٱللَّهِ وَقَد هَدَىٰنِ ۚ وَلَآ أَخَافُ مَا تُشرِكُونَ بِهِۦٓ إِلَّآ أَن يَشَآءَ رَبِّى شَيـًۭٔا ۗ وَسِعَ رَبِّى كُلَّ شَىءٍ عِلمًا ۗ أَفَلَا تَتَذَكَّرُونَ ﴿٨٠﴾
 وَكَيفَ أَخَافُ مَآ أَشرَكتُم وَلَا تَخَافُونَ أَنَّكُم أَشرَكتُم بِٱللَّهِ مَا لَم يُنَزِّل بِهِۦ عَلَيكُم سُلطَٰنًۭا ۚ فَأَىُّ ٱلفَرِيقَينِ أَحَقُّ بِٱلأَمنِ ۖ إِن كُنتُم تَعلَمُونَ ﴿٨١﴾
 ٱلَّذِينَ ءَامَنُوا۟ وَلَم يَلبِسُوٓا۟ إِيمَـٰنَهُم بِظُلمٍ أُو۟لَـٰٓئِكَ لَهُمُ ٱلأَمنُ وَهُم مُّهتَدُونَ ﴿٨٢﴾
 وَتِلكَ حُجَّتُنَآ ءَاتَينَـٰهَآ إِبرَٰهِيمَ عَلَىٰ قَومِهِۦ ۚ نَرفَعُ دَرَجَٰتٍۢ مَّن نَّشَآءُ ۗ إِنَّ رَبَّكَ حَكِيمٌ عَلِيمٌۭ ﴿٨٣﴾
 وَوَهَبنَا لَهُۥٓ إِسحَـٰقَ وَيَعقُوبَ ۚ كُلًّا هَدَينَا ۚ وَنُوحًا هَدَينَا مِن قَبلُ ۖ وَمِن ذُرِّيَّتِهِۦ دَاوُۥدَ وَسُلَيمَـٰنَ وَأَيُّوبَ وَيُوسُفَ وَمُوسَىٰ وَهَـٰرُونَ ۚ وَكَذَٟلِكَ نَجزِى ٱلمُحسِنِينَ ﴿٨٤﴾
 وَزَكَرِيَّا وَيَحيَىٰ وَعِيسَىٰ وَإِليَاسَ ۖ كُلٌّۭ مِّنَ ٱلصَّـٰلِحِينَ ﴿٨٥﴾
 وَإِسمَـٰعِيلَ وَٱليَسَعَ وَيُونُسَ وَلُوطًۭا ۚ وَكُلًّۭا فَضَّلنَا عَلَى ٱلعَـٰلَمِينَ ﴿٨٦﴾
 وَمِن ءَابَآئِهِم وَذُرِّيَّٰتِهِم وَإِخوَٟنِهِم ۖ وَٱجتَبَينَـٰهُم وَهَدَينَـٰهُم إِلَىٰ صِرَٰطٍۢ مُّستَقِيمٍۢ ﴿٨٧﴾
 ذَٟلِكَ هُدَى ٱللَّهِ يَهدِى بِهِۦ مَن يَشَآءُ مِن عِبَادِهِۦ ۚ وَلَو أَشرَكُوا۟ لَحَبِطَ عَنهُم مَّا كَانُوا۟ يَعمَلُونَ ﴿٨٨﴾
 أُو۟لَـٰٓئِكَ ٱلَّذِينَ ءَاتَينَـٰهُمُ ٱلكِتَـٰبَ وَٱلحُكمَ وَٱلنُّبُوَّةَ ۚ فَإِن يَكفُر بِهَا هَـٰٓؤُلَآءِ فَقَد وَكَّلنَا بِهَا قَومًۭا لَّيسُوا۟ بِهَا بِكَـٰفِرِينَ ﴿٨٩﴾
 أُو۟لَـٰٓئِكَ ٱلَّذِينَ هَدَى ٱللَّهُ ۖ فَبِهُدَىٰهُمُ ٱقتَدِه ۗ قُل لَّآ أَسـَٔلُكُم عَلَيهِ أَجرًا ۖ إِن هُوَ إِلَّا ذِكرَىٰ لِلعَـٰلَمِينَ ﴿٩٠﴾
 وَمَا قَدَرُوا۟ ٱللَّهَ حَقَّ قَدرِهِۦٓ إِذ قَالُوا۟ مَآ أَنزَلَ ٱللَّهُ عَلَىٰ بَشَرٍۢ مِّن شَىءٍۢ ۗ قُل مَن أَنزَلَ ٱلكِتَـٰبَ ٱلَّذِى جَآءَ بِهِۦ مُوسَىٰ نُورًۭا وَهُدًۭى لِّلنَّاسِ ۖ تَجعَلُونَهُۥ قَرَاطِيسَ تُبدُونَهَا وَتُخفُونَ كَثِيرًۭا ۖ وَعُلِّمتُم مَّا لَم تَعلَمُوٓا۟ أَنتُم وَلَآ ءَابَآؤُكُم ۖ قُلِ ٱللَّهُ ۖ ثُمَّ ذَرهُم فِى خَوضِهِم يَلعَبُونَ ﴿٩١﴾
 وَهَـٰذَا كِتَـٰبٌ أَنزَلنَـٰهُ مُبَارَكٌۭ مُّصَدِّقُ ٱلَّذِى بَينَ يَدَيهِ وَلِتُنذِرَ أُمَّ ٱلقُرَىٰ وَمَن حَولَهَا ۚ وَٱلَّذِينَ يُؤمِنُونَ بِٱلءَاخِرَةِ يُؤمِنُونَ بِهِۦ ۖ وَهُم عَلَىٰ صَلَاتِهِم يُحَافِظُونَ ﴿٩٢﴾
 وَمَن أَظلَمُ مِمَّنِ ٱفتَرَىٰ عَلَى ٱللَّهِ كَذِبًا أَو قَالَ أُوحِىَ إِلَىَّ وَلَم يُوحَ إِلَيهِ شَىءٌۭ وَمَن قَالَ سَأُنزِلُ مِثلَ مَآ أَنزَلَ ٱللَّهُ ۗ وَلَو تَرَىٰٓ إِذِ ٱلظَّـٰلِمُونَ فِى غَمَرَٰتِ ٱلمَوتِ وَٱلمَلَـٰٓئِكَةُ بَاسِطُوٓا۟ أَيدِيهِم أَخرِجُوٓا۟ أَنفُسَكُمُ ۖ ٱليَومَ تُجزَونَ عَذَابَ ٱلهُونِ بِمَا كُنتُم تَقُولُونَ عَلَى ٱللَّهِ غَيرَ ٱلحَقِّ وَكُنتُم عَن ءَايَـٰتِهِۦ تَستَكبِرُونَ ﴿٩٣﴾
 وَلَقَد جِئتُمُونَا فُرَٰدَىٰ كَمَا خَلَقنَـٰكُم أَوَّلَ مَرَّةٍۢ وَتَرَكتُم مَّا خَوَّلنَـٰكُم وَرَآءَ ظُهُورِكُم ۖ وَمَا نَرَىٰ مَعَكُم شُفَعَآءَكُمُ ٱلَّذِينَ زَعَمتُم أَنَّهُم فِيكُم شُرَكَـٰٓؤُا۟ ۚ لَقَد تَّقَطَّعَ بَينَكُم وَضَلَّ عَنكُم مَّا كُنتُم تَزعُمُونَ ﴿٩٤﴾
 ۞ إِنَّ ٱللَّهَ فَالِقُ ٱلحَبِّ وَٱلنَّوَىٰ ۖ يُخرِجُ ٱلحَىَّ مِنَ ٱلمَيِّتِ وَمُخرِجُ ٱلمَيِّتِ مِنَ ٱلحَىِّ ۚ ذَٟلِكُمُ ٱللَّهُ ۖ فَأَنَّىٰ تُؤفَكُونَ ﴿٩٥﴾
 فَالِقُ ٱلإِصبَاحِ وَجَعَلَ ٱلَّيلَ سَكَنًۭا وَٱلشَّمسَ وَٱلقَمَرَ حُسبَانًۭا ۚ ذَٟلِكَ تَقدِيرُ ٱلعَزِيزِ ٱلعَلِيمِ ﴿٩٦﴾
 وَهُوَ ٱلَّذِى جَعَلَ لَكُمُ ٱلنُّجُومَ لِتَهتَدُوا۟ بِهَا فِى ظُلُمَـٰتِ ٱلبَرِّ وَٱلبَحرِ ۗ قَد فَصَّلنَا ٱلءَايَـٰتِ لِقَومٍۢ يَعلَمُونَ ﴿٩٧﴾
 وَهُوَ ٱلَّذِىٓ أَنشَأَكُم مِّن نَّفسٍۢ وَٟحِدَةٍۢ فَمُستَقَرٌّۭ وَمُستَودَعٌۭ ۗ قَد فَصَّلنَا ٱلءَايَـٰتِ لِقَومٍۢ يَفقَهُونَ ﴿٩٨﴾
 وَهُوَ ٱلَّذِىٓ أَنزَلَ مِنَ ٱلسَّمَآءِ مَآءًۭ فَأَخرَجنَا بِهِۦ نَبَاتَ كُلِّ شَىءٍۢ فَأَخرَجنَا مِنهُ خَضِرًۭا نُّخرِجُ مِنهُ حَبًّۭا مُّتَرَاكِبًۭا وَمِنَ ٱلنَّخلِ مِن طَلعِهَا قِنوَانٌۭ دَانِيَةٌۭ وَجَنَّـٰتٍۢ مِّن أَعنَابٍۢ وَٱلزَّيتُونَ وَٱلرُّمَّانَ مُشتَبِهًۭا وَغَيرَ مُتَشَـٰبِهٍ ۗ ٱنظُرُوٓا۟ إِلَىٰ ثَمَرِهِۦٓ إِذَآ أَثمَرَ وَيَنعِهِۦٓ ۚ إِنَّ فِى ذَٟلِكُم لَءَايَـٰتٍۢ لِّقَومٍۢ يُؤمِنُونَ ﴿٩٩﴾
 وَجَعَلُوا۟ لِلَّهِ شُرَكَآءَ ٱلجِنَّ وَخَلَقَهُم ۖ وَخَرَقُوا۟ لَهُۥ بَنِينَ وَبَنَـٰتٍۭ بِغَيرِ عِلمٍۢ ۚ سُبحَـٰنَهُۥ وَتَعَـٰلَىٰ عَمَّا يَصِفُونَ ﴿١٠٠﴾
 بَدِيعُ ٱلسَّمَـٰوَٟتِ وَٱلأَرضِ ۖ أَنَّىٰ يَكُونُ لَهُۥ وَلَدٌۭ وَلَم تَكُن لَّهُۥ صَـٰحِبَةٌۭ ۖ وَخَلَقَ كُلَّ شَىءٍۢ ۖ وَهُوَ بِكُلِّ شَىءٍ عَلِيمٌۭ ﴿١٠١﴾
 ذَٟلِكُمُ ٱللَّهُ رَبُّكُم ۖ لَآ إِلَـٰهَ إِلَّا هُوَ ۖ خَـٰلِقُ كُلِّ شَىءٍۢ فَٱعبُدُوهُ ۚ وَهُوَ عَلَىٰ كُلِّ شَىءٍۢ وَكِيلٌۭ ﴿١٠٢﴾
 لَّا تُدرِكُهُ ٱلأَبصَـٰرُ وَهُوَ يُدرِكُ ٱلأَبصَـٰرَ ۖ وَهُوَ ٱللَّطِيفُ ٱلخَبِيرُ ﴿١٠٣﴾
 قَد جَآءَكُم بَصَآئِرُ مِن رَّبِّكُم ۖ فَمَن أَبصَرَ فَلِنَفسِهِۦ ۖ وَمَن عَمِىَ فَعَلَيهَا ۚ وَمَآ أَنَا۠ عَلَيكُم بِحَفِيظٍۢ ﴿١٠٤﴾
 وَكَذَٟلِكَ نُصَرِّفُ ٱلءَايَـٰتِ وَلِيَقُولُوا۟ دَرَستَ وَلِنُبَيِّنَهُۥ لِقَومٍۢ يَعلَمُونَ ﴿١٠٥﴾
 ٱتَّبِع مَآ أُوحِىَ إِلَيكَ مِن رَّبِّكَ ۖ لَآ إِلَـٰهَ إِلَّا هُوَ ۖ وَأَعرِض عَنِ ٱلمُشرِكِينَ ﴿١٠٦﴾
 وَلَو شَآءَ ٱللَّهُ مَآ أَشرَكُوا۟ ۗ وَمَا جَعَلنَـٰكَ عَلَيهِم حَفِيظًۭا ۖ وَمَآ أَنتَ عَلَيهِم بِوَكِيلٍۢ ﴿١٠٧﴾
 وَلَا تَسُبُّوا۟ ٱلَّذِينَ يَدعُونَ مِن دُونِ ٱللَّهِ فَيَسُبُّوا۟ ٱللَّهَ عَدوًۢا بِغَيرِ عِلمٍۢ ۗ كَذَٟلِكَ زَيَّنَّا لِكُلِّ أُمَّةٍ عَمَلَهُم ثُمَّ إِلَىٰ رَبِّهِم مَّرجِعُهُم فَيُنَبِّئُهُم بِمَا كَانُوا۟ يَعمَلُونَ ﴿١٠٨﴾
 وَأَقسَمُوا۟ بِٱللَّهِ جَهدَ أَيمَـٰنِهِم لَئِن جَآءَتهُم ءَايَةٌۭ لَّيُؤمِنُنَّ بِهَا ۚ قُل إِنَّمَا ٱلءَايَـٰتُ عِندَ ٱللَّهِ ۖ وَمَا يُشعِرُكُم أَنَّهَآ إِذَا جَآءَت لَا يُؤمِنُونَ ﴿١٠٩﴾
 وَنُقَلِّبُ أَفـِٔدَتَهُم وَأَبصَـٰرَهُم كَمَا لَم يُؤمِنُوا۟ بِهِۦٓ أَوَّلَ مَرَّةٍۢ وَنَذَرُهُم فِى طُغيَـٰنِهِم يَعمَهُونَ ﴿١١٠﴾
 ۞ وَلَو أَنَّنَا نَزَّلنَآ إِلَيهِمُ ٱلمَلَـٰٓئِكَةَ وَكَلَّمَهُمُ ٱلمَوتَىٰ وَحَشَرنَا عَلَيهِم كُلَّ شَىءٍۢ قُبُلًۭا مَّا كَانُوا۟ لِيُؤمِنُوٓا۟ إِلَّآ أَن يَشَآءَ ٱللَّهُ وَلَـٰكِنَّ أَكثَرَهُم يَجهَلُونَ ﴿١١١﴾
 وَكَذَٟلِكَ جَعَلنَا لِكُلِّ نَبِىٍّ عَدُوًّۭا شَيَـٰطِينَ ٱلإِنسِ وَٱلجِنِّ يُوحِى بَعضُهُم إِلَىٰ بَعضٍۢ زُخرُفَ ٱلقَولِ غُرُورًۭا ۚ وَلَو شَآءَ رَبُّكَ مَا فَعَلُوهُ ۖ فَذَرهُم وَمَا يَفتَرُونَ ﴿١١٢﴾
 وَلِتَصغَىٰٓ إِلَيهِ أَفـِٔدَةُ ٱلَّذِينَ لَا يُؤمِنُونَ بِٱلءَاخِرَةِ وَلِيَرضَوهُ وَلِيَقتَرِفُوا۟ مَا هُم مُّقتَرِفُونَ ﴿١١٣﴾
 أَفَغَيرَ ٱللَّهِ أَبتَغِى حَكَمًۭا وَهُوَ ٱلَّذِىٓ أَنزَلَ إِلَيكُمُ ٱلكِتَـٰبَ مُفَصَّلًۭا ۚ وَٱلَّذِينَ ءَاتَينَـٰهُمُ ٱلكِتَـٰبَ يَعلَمُونَ أَنَّهُۥ مُنَزَّلٌۭ مِّن رَّبِّكَ بِٱلحَقِّ ۖ فَلَا تَكُونَنَّ مِنَ ٱلمُمتَرِينَ ﴿١١٤﴾
 وَتَمَّت كَلِمَتُ رَبِّكَ صِدقًۭا وَعَدلًۭا ۚ لَّا مُبَدِّلَ لِكَلِمَـٰتِهِۦ ۚ وَهُوَ ٱلسَّمِيعُ ٱلعَلِيمُ ﴿١١٥﴾
 وَإِن تُطِع أَكثَرَ مَن فِى ٱلأَرضِ يُضِلُّوكَ عَن سَبِيلِ ٱللَّهِ ۚ إِن يَتَّبِعُونَ إِلَّا ٱلظَّنَّ وَإِن هُم إِلَّا يَخرُصُونَ ﴿١١٦﴾
 إِنَّ رَبَّكَ هُوَ أَعلَمُ مَن يَضِلُّ عَن سَبِيلِهِۦ ۖ وَهُوَ أَعلَمُ بِٱلمُهتَدِينَ ﴿١١٧﴾
 فَكُلُوا۟ مِمَّا ذُكِرَ ٱسمُ ٱللَّهِ عَلَيهِ إِن كُنتُم بِـَٔايَـٰتِهِۦ مُؤمِنِينَ ﴿١١٨﴾
 وَمَا لَكُم أَلَّا تَأكُلُوا۟ مِمَّا ذُكِرَ ٱسمُ ٱللَّهِ عَلَيهِ وَقَد فَصَّلَ لَكُم مَّا حَرَّمَ عَلَيكُم إِلَّا مَا ٱضطُرِرتُم إِلَيهِ ۗ وَإِنَّ كَثِيرًۭا لَّيُضِلُّونَ بِأَهوَآئِهِم بِغَيرِ عِلمٍ ۗ إِنَّ رَبَّكَ هُوَ أَعلَمُ بِٱلمُعتَدِينَ ﴿١١٩﴾
 وَذَرُوا۟ ظَـٰهِرَ ٱلإِثمِ وَبَاطِنَهُۥٓ ۚ إِنَّ ٱلَّذِينَ يَكسِبُونَ ٱلإِثمَ سَيُجزَونَ بِمَا كَانُوا۟ يَقتَرِفُونَ ﴿١٢٠﴾
 وَلَا تَأكُلُوا۟ مِمَّا لَم يُذكَرِ ٱسمُ ٱللَّهِ عَلَيهِ وَإِنَّهُۥ لَفِسقٌۭ ۗ وَإِنَّ ٱلشَّيَـٰطِينَ لَيُوحُونَ إِلَىٰٓ أَولِيَآئِهِم لِيُجَٰدِلُوكُم ۖ وَإِن أَطَعتُمُوهُم إِنَّكُم لَمُشرِكُونَ ﴿١٢١﴾
 أَوَمَن كَانَ مَيتًۭا فَأَحيَينَـٰهُ وَجَعَلنَا لَهُۥ نُورًۭا يَمشِى بِهِۦ فِى ٱلنَّاسِ كَمَن مَّثَلُهُۥ فِى ٱلظُّلُمَـٰتِ لَيسَ بِخَارِجٍۢ مِّنهَا ۚ كَذَٟلِكَ زُيِّنَ لِلكَـٰفِرِينَ مَا كَانُوا۟ يَعمَلُونَ ﴿١٢٢﴾
 وَكَذَٟلِكَ جَعَلنَا فِى كُلِّ قَريَةٍ أَكَـٰبِرَ مُجرِمِيهَا لِيَمكُرُوا۟ فِيهَا ۖ وَمَا يَمكُرُونَ إِلَّا بِأَنفُسِهِم وَمَا يَشعُرُونَ ﴿١٢٣﴾
 وَإِذَا جَآءَتهُم ءَايَةٌۭ قَالُوا۟ لَن نُّؤمِنَ حَتَّىٰ نُؤتَىٰ مِثلَ مَآ أُوتِىَ رُسُلُ ٱللَّهِ ۘ ٱللَّهُ أَعلَمُ حَيثُ يَجعَلُ رِسَالَتَهُۥ ۗ سَيُصِيبُ ٱلَّذِينَ أَجرَمُوا۟ صَغَارٌ عِندَ ٱللَّهِ وَعَذَابٌۭ شَدِيدٌۢ بِمَا كَانُوا۟ يَمكُرُونَ ﴿١٢٤﴾
 فَمَن يُرِدِ ٱللَّهُ أَن يَهدِيَهُۥ يَشرَح صَدرَهُۥ لِلإِسلَـٰمِ ۖ وَمَن يُرِد أَن يُضِلَّهُۥ يَجعَل صَدرَهُۥ ضَيِّقًا حَرَجًۭا كَأَنَّمَا يَصَّعَّدُ فِى ٱلسَّمَآءِ ۚ كَذَٟلِكَ يَجعَلُ ٱللَّهُ ٱلرِّجسَ عَلَى ٱلَّذِينَ لَا يُؤمِنُونَ ﴿١٢٥﴾
 وَهَـٰذَا صِرَٰطُ رَبِّكَ مُستَقِيمًۭا ۗ قَد فَصَّلنَا ٱلءَايَـٰتِ لِقَومٍۢ يَذَّكَّرُونَ ﴿١٢٦﴾
 ۞ لَهُم دَارُ ٱلسَّلَـٰمِ عِندَ رَبِّهِم ۖ وَهُوَ وَلِيُّهُم بِمَا كَانُوا۟ يَعمَلُونَ ﴿١٢٧﴾
 وَيَومَ يَحشُرُهُم جَمِيعًۭا يَـٰمَعشَرَ ٱلجِنِّ قَدِ ٱستَكثَرتُم مِّنَ ٱلإِنسِ ۖ وَقَالَ أَولِيَآؤُهُم مِّنَ ٱلإِنسِ رَبَّنَا ٱستَمتَعَ بَعضُنَا بِبَعضٍۢ وَبَلَغنَآ أَجَلَنَا ٱلَّذِىٓ أَجَّلتَ لَنَا ۚ قَالَ ٱلنَّارُ مَثوَىٰكُم خَـٰلِدِينَ فِيهَآ إِلَّا مَا شَآءَ ٱللَّهُ ۗ إِنَّ رَبَّكَ حَكِيمٌ عَلِيمٌۭ ﴿١٢٨﴾
 وَكَذَٟلِكَ نُوَلِّى بَعضَ ٱلظَّـٰلِمِينَ بَعضًۢا بِمَا كَانُوا۟ يَكسِبُونَ ﴿١٢٩﴾
 يَـٰمَعشَرَ ٱلجِنِّ وَٱلإِنسِ أَلَم يَأتِكُم رُسُلٌۭ مِّنكُم يَقُصُّونَ عَلَيكُم ءَايَـٰتِى وَيُنذِرُونَكُم لِقَآءَ يَومِكُم هَـٰذَا ۚ قَالُوا۟ شَهِدنَا عَلَىٰٓ أَنفُسِنَا ۖ وَغَرَّتهُمُ ٱلحَيَوٰةُ ٱلدُّنيَا وَشَهِدُوا۟ عَلَىٰٓ أَنفُسِهِم أَنَّهُم كَانُوا۟ كَـٰفِرِينَ ﴿١٣٠﴾
 ذَٟلِكَ أَن لَّم يَكُن رَّبُّكَ مُهلِكَ ٱلقُرَىٰ بِظُلمٍۢ وَأَهلُهَا غَٰفِلُونَ ﴿١٣١﴾
 وَلِكُلٍّۢ دَرَجَٰتٌۭ مِّمَّا عَمِلُوا۟ ۚ وَمَا رَبُّكَ بِغَٰفِلٍ عَمَّا يَعمَلُونَ ﴿١٣٢﴾
 وَرَبُّكَ ٱلغَنِىُّ ذُو ٱلرَّحمَةِ ۚ إِن يَشَأ يُذهِبكُم وَيَستَخلِف مِنۢ بَعدِكُم مَّا يَشَآءُ كَمَآ أَنشَأَكُم مِّن ذُرِّيَّةِ قَومٍ ءَاخَرِينَ ﴿١٣٣﴾
 إِنَّ مَا تُوعَدُونَ لَءَاتٍۢ ۖ وَمَآ أَنتُم بِمُعجِزِينَ ﴿١٣٤﴾
 قُل يَـٰقَومِ ٱعمَلُوا۟ عَلَىٰ مَكَانَتِكُم إِنِّى عَامِلٌۭ ۖ فَسَوفَ تَعلَمُونَ مَن تَكُونُ لَهُۥ عَـٰقِبَةُ ٱلدَّارِ ۗ إِنَّهُۥ لَا يُفلِحُ ٱلظَّـٰلِمُونَ ﴿١٣٥﴾
 وَجَعَلُوا۟ لِلَّهِ مِمَّا ذَرَأَ مِنَ ٱلحَرثِ وَٱلأَنعَـٰمِ نَصِيبًۭا فَقَالُوا۟ هَـٰذَا لِلَّهِ بِزَعمِهِم وَهَـٰذَا لِشُرَكَآئِنَا ۖ فَمَا كَانَ لِشُرَكَآئِهِم فَلَا يَصِلُ إِلَى ٱللَّهِ ۖ وَمَا كَانَ لِلَّهِ فَهُوَ يَصِلُ إِلَىٰ شُرَكَآئِهِم ۗ سَآءَ مَا يَحكُمُونَ ﴿١٣٦﴾
 وَكَذَٟلِكَ زَيَّنَ لِكَثِيرٍۢ مِّنَ ٱلمُشرِكِينَ قَتلَ أَولَـٰدِهِم شُرَكَآؤُهُم لِيُردُوهُم وَلِيَلبِسُوا۟ عَلَيهِم دِينَهُم ۖ وَلَو شَآءَ ٱللَّهُ مَا فَعَلُوهُ ۖ فَذَرهُم وَمَا يَفتَرُونَ ﴿١٣٧﴾
 وَقَالُوا۟ هَـٰذِهِۦٓ أَنعَـٰمٌۭ وَحَرثٌ حِجرٌۭ لَّا يَطعَمُهَآ إِلَّا مَن نَّشَآءُ بِزَعمِهِم وَأَنعَـٰمٌ حُرِّمَت ظُهُورُهَا وَأَنعَـٰمٌۭ لَّا يَذكُرُونَ ٱسمَ ٱللَّهِ عَلَيهَا ٱفتِرَآءً عَلَيهِ ۚ سَيَجزِيهِم بِمَا كَانُوا۟ يَفتَرُونَ ﴿١٣٨﴾
 وَقَالُوا۟ مَا فِى بُطُونِ هَـٰذِهِ ٱلأَنعَـٰمِ خَالِصَةٌۭ لِّذُكُورِنَا وَمُحَرَّمٌ عَلَىٰٓ أَزوَٟجِنَا ۖ وَإِن يَكُن مَّيتَةًۭ فَهُم فِيهِ شُرَكَآءُ ۚ سَيَجزِيهِم وَصفَهُم ۚ إِنَّهُۥ حَكِيمٌ عَلِيمٌۭ ﴿١٣٩﴾
 قَد خَسِرَ ٱلَّذِينَ قَتَلُوٓا۟ أَولَـٰدَهُم سَفَهًۢا بِغَيرِ عِلمٍۢ وَحَرَّمُوا۟ مَا رَزَقَهُمُ ٱللَّهُ ٱفتِرَآءً عَلَى ٱللَّهِ ۚ قَد ضَلُّوا۟ وَمَا كَانُوا۟ مُهتَدِينَ ﴿١٤٠﴾
 ۞ وَهُوَ ٱلَّذِىٓ أَنشَأَ جَنَّـٰتٍۢ مَّعرُوشَـٰتٍۢ وَغَيرَ مَعرُوشَـٰتٍۢ وَٱلنَّخلَ وَٱلزَّرعَ مُختَلِفًا أُكُلُهُۥ وَٱلزَّيتُونَ وَٱلرُّمَّانَ مُتَشَـٰبِهًۭا وَغَيرَ مُتَشَـٰبِهٍۢ ۚ كُلُوا۟ مِن ثَمَرِهِۦٓ إِذَآ أَثمَرَ وَءَاتُوا۟ حَقَّهُۥ يَومَ حَصَادِهِۦ ۖ وَلَا تُسرِفُوٓا۟ ۚ إِنَّهُۥ لَا يُحِبُّ ٱلمُسرِفِينَ ﴿١٤١﴾
 وَمِنَ ٱلأَنعَـٰمِ حَمُولَةًۭ وَفَرشًۭا ۚ كُلُوا۟ مِمَّا رَزَقَكُمُ ٱللَّهُ وَلَا تَتَّبِعُوا۟ خُطُوَٟتِ ٱلشَّيطَٰنِ ۚ إِنَّهُۥ لَكُم عَدُوٌّۭ مُّبِينٌۭ ﴿١٤٢﴾
 ثَمَـٰنِيَةَ أَزوَٟجٍۢ ۖ مِّنَ ٱلضَّأنِ ٱثنَينِ وَمِنَ ٱلمَعزِ ٱثنَينِ ۗ قُل ءَآلذَّكَرَينِ حَرَّمَ أَمِ ٱلأُنثَيَينِ أَمَّا ٱشتَمَلَت عَلَيهِ أَرحَامُ ٱلأُنثَيَينِ ۖ نَبِّـُٔونِى بِعِلمٍ إِن كُنتُم صَـٰدِقِينَ ﴿١٤٣﴾
 وَمِنَ ٱلإِبِلِ ٱثنَينِ وَمِنَ ٱلبَقَرِ ٱثنَينِ ۗ قُل ءَآلذَّكَرَينِ حَرَّمَ أَمِ ٱلأُنثَيَينِ أَمَّا ٱشتَمَلَت عَلَيهِ أَرحَامُ ٱلأُنثَيَينِ ۖ أَم كُنتُم شُهَدَآءَ إِذ وَصَّىٰكُمُ ٱللَّهُ بِهَـٰذَا ۚ فَمَن أَظلَمُ مِمَّنِ ٱفتَرَىٰ عَلَى ٱللَّهِ كَذِبًۭا لِّيُضِلَّ ٱلنَّاسَ بِغَيرِ عِلمٍ ۗ إِنَّ ٱللَّهَ لَا يَهدِى ٱلقَومَ ٱلظَّـٰلِمِينَ ﴿١٤٤﴾
 قُل لَّآ أَجِدُ فِى مَآ أُوحِىَ إِلَىَّ مُحَرَّمًا عَلَىٰ طَاعِمٍۢ يَطعَمُهُۥٓ إِلَّآ أَن يَكُونَ مَيتَةً أَو دَمًۭا مَّسفُوحًا أَو لَحمَ خِنزِيرٍۢ فَإِنَّهُۥ رِجسٌ أَو فِسقًا أُهِلَّ لِغَيرِ ٱللَّهِ بِهِۦ ۚ فَمَنِ ٱضطُرَّ غَيرَ بَاغٍۢ وَلَا عَادٍۢ فَإِنَّ رَبَّكَ غَفُورٌۭ رَّحِيمٌۭ ﴿١٤٥﴾
 وَعَلَى ٱلَّذِينَ هَادُوا۟ حَرَّمنَا كُلَّ ذِى ظُفُرٍۢ ۖ وَمِنَ ٱلبَقَرِ وَٱلغَنَمِ حَرَّمنَا عَلَيهِم شُحُومَهُمَآ إِلَّا مَا حَمَلَت ظُهُورُهُمَآ أَوِ ٱلحَوَايَآ أَو مَا ٱختَلَطَ بِعَظمٍۢ ۚ ذَٟلِكَ جَزَينَـٰهُم بِبَغيِهِم ۖ وَإِنَّا لَصَـٰدِقُونَ ﴿١٤٦﴾
 فَإِن كَذَّبُوكَ فَقُل رَّبُّكُم ذُو رَحمَةٍۢ وَٟسِعَةٍۢ وَلَا يُرَدُّ بَأسُهُۥ عَنِ ٱلقَومِ ٱلمُجرِمِينَ ﴿١٤٧﴾
 سَيَقُولُ ٱلَّذِينَ أَشرَكُوا۟ لَو شَآءَ ٱللَّهُ مَآ أَشرَكنَا وَلَآ ءَابَآؤُنَا وَلَا حَرَّمنَا مِن شَىءٍۢ ۚ كَذَٟلِكَ كَذَّبَ ٱلَّذِينَ مِن قَبلِهِم حَتَّىٰ ذَاقُوا۟ بَأسَنَا ۗ قُل هَل عِندَكُم مِّن عِلمٍۢ فَتُخرِجُوهُ لَنَآ ۖ إِن تَتَّبِعُونَ إِلَّا ٱلظَّنَّ وَإِن أَنتُم إِلَّا تَخرُصُونَ ﴿١٤٨﴾
 قُل فَلِلَّهِ ٱلحُجَّةُ ٱلبَٰلِغَةُ ۖ فَلَو شَآءَ لَهَدَىٰكُم أَجمَعِينَ ﴿١٤٩﴾
 قُل هَلُمَّ شُهَدَآءَكُمُ ٱلَّذِينَ يَشهَدُونَ أَنَّ ٱللَّهَ حَرَّمَ هَـٰذَا ۖ فَإِن شَهِدُوا۟ فَلَا تَشهَد مَعَهُم ۚ وَلَا تَتَّبِع أَهوَآءَ ٱلَّذِينَ كَذَّبُوا۟ بِـَٔايَـٰتِنَا وَٱلَّذِينَ لَا يُؤمِنُونَ بِٱلءَاخِرَةِ وَهُم بِرَبِّهِم يَعدِلُونَ ﴿١٥٠﴾
 ۞ قُل تَعَالَوا۟ أَتلُ مَا حَرَّمَ رَبُّكُم عَلَيكُم ۖ أَلَّا تُشرِكُوا۟ بِهِۦ شَيـًۭٔا ۖ وَبِٱلوَٟلِدَينِ إِحسَـٰنًۭا ۖ وَلَا تَقتُلُوٓا۟ أَولَـٰدَكُم مِّن إِملَـٰقٍۢ ۖ نَّحنُ نَرزُقُكُم وَإِيَّاهُم ۖ وَلَا تَقرَبُوا۟ ٱلفَوَٟحِشَ مَا ظَهَرَ مِنهَا وَمَا بَطَنَ ۖ وَلَا تَقتُلُوا۟ ٱلنَّفسَ ٱلَّتِى حَرَّمَ ٱللَّهُ إِلَّا بِٱلحَقِّ ۚ ذَٟلِكُم وَصَّىٰكُم بِهِۦ لَعَلَّكُم تَعقِلُونَ ﴿١٥١﴾
 وَلَا تَقرَبُوا۟ مَالَ ٱليَتِيمِ إِلَّا بِٱلَّتِى هِىَ أَحسَنُ حَتَّىٰ يَبلُغَ أَشُدَّهُۥ ۖ وَأَوفُوا۟ ٱلكَيلَ وَٱلمِيزَانَ بِٱلقِسطِ ۖ لَا نُكَلِّفُ نَفسًا إِلَّا وُسعَهَا ۖ وَإِذَا قُلتُم فَٱعدِلُوا۟ وَلَو كَانَ ذَا قُربَىٰ ۖ وَبِعَهدِ ٱللَّهِ أَوفُوا۟ ۚ ذَٟلِكُم وَصَّىٰكُم بِهِۦ لَعَلَّكُم تَذَكَّرُونَ ﴿١٥٢﴾
 وَأَنَّ هَـٰذَا صِرَٰطِى مُستَقِيمًۭا فَٱتَّبِعُوهُ ۖ وَلَا تَتَّبِعُوا۟ ٱلسُّبُلَ فَتَفَرَّقَ بِكُم عَن سَبِيلِهِۦ ۚ ذَٟلِكُم وَصَّىٰكُم بِهِۦ لَعَلَّكُم تَتَّقُونَ ﴿١٥٣﴾
 ثُمَّ ءَاتَينَا مُوسَى ٱلكِتَـٰبَ تَمَامًا عَلَى ٱلَّذِىٓ أَحسَنَ وَتَفصِيلًۭا لِّكُلِّ شَىءٍۢ وَهُدًۭى وَرَحمَةًۭ لَّعَلَّهُم بِلِقَآءِ رَبِّهِم يُؤمِنُونَ ﴿١٥٤﴾
 وَهَـٰذَا كِتَـٰبٌ أَنزَلنَـٰهُ مُبَارَكٌۭ فَٱتَّبِعُوهُ وَٱتَّقُوا۟ لَعَلَّكُم تُرحَمُونَ ﴿١٥٥﴾
 أَن تَقُولُوٓا۟ إِنَّمَآ أُنزِلَ ٱلكِتَـٰبُ عَلَىٰ طَآئِفَتَينِ مِن قَبلِنَا وَإِن كُنَّا عَن دِرَاسَتِهِم لَغَٰفِلِينَ ﴿١٥٦﴾
 أَو تَقُولُوا۟ لَو أَنَّآ أُنزِلَ عَلَينَا ٱلكِتَـٰبُ لَكُنَّآ أَهدَىٰ مِنهُم ۚ فَقَد جَآءَكُم بَيِّنَةٌۭ مِّن رَّبِّكُم وَهُدًۭى وَرَحمَةٌۭ ۚ فَمَن أَظلَمُ مِمَّن كَذَّبَ بِـَٔايَـٰتِ ٱللَّهِ وَصَدَفَ عَنهَا ۗ سَنَجزِى ٱلَّذِينَ يَصدِفُونَ عَن ءَايَـٰتِنَا سُوٓءَ ٱلعَذَابِ بِمَا كَانُوا۟ يَصدِفُونَ ﴿١٥٧﴾
 هَل يَنظُرُونَ إِلَّآ أَن تَأتِيَهُمُ ٱلمَلَـٰٓئِكَةُ أَو يَأتِىَ رَبُّكَ أَو يَأتِىَ بَعضُ ءَايَـٰتِ رَبِّكَ ۗ يَومَ يَأتِى بَعضُ ءَايَـٰتِ رَبِّكَ لَا يَنفَعُ نَفسًا إِيمَـٰنُهَا لَم تَكُن ءَامَنَت مِن قَبلُ أَو كَسَبَت فِىٓ إِيمَـٰنِهَا خَيرًۭا ۗ قُلِ ٱنتَظِرُوٓا۟ إِنَّا مُنتَظِرُونَ ﴿١٥٨﴾
 إِنَّ ٱلَّذِينَ فَرَّقُوا۟ دِينَهُم وَكَانُوا۟ شِيَعًۭا لَّستَ مِنهُم فِى شَىءٍ ۚ إِنَّمَآ أَمرُهُم إِلَى ٱللَّهِ ثُمَّ يُنَبِّئُهُم بِمَا كَانُوا۟ يَفعَلُونَ ﴿١٥٩﴾
 مَن جَآءَ بِٱلحَسَنَةِ فَلَهُۥ عَشرُ أَمثَالِهَا ۖ وَمَن جَآءَ بِٱلسَّيِّئَةِ فَلَا يُجزَىٰٓ إِلَّا مِثلَهَا وَهُم لَا يُظلَمُونَ ﴿١٦٠﴾
 قُل إِنَّنِى هَدَىٰنِى رَبِّىٓ إِلَىٰ صِرَٰطٍۢ مُّستَقِيمٍۢ دِينًۭا قِيَمًۭا مِّلَّةَ إِبرَٰهِيمَ حَنِيفًۭا ۚ وَمَا كَانَ مِنَ ٱلمُشرِكِينَ ﴿١٦١﴾
 قُل إِنَّ صَلَاتِى وَنُسُكِى وَمَحيَاىَ وَمَمَاتِى لِلَّهِ رَبِّ ٱلعَـٰلَمِينَ ﴿١٦٢﴾
 لَا شَرِيكَ لَهُۥ ۖ وَبِذَٟلِكَ أُمِرتُ وَأَنَا۠ أَوَّلُ ٱلمُسلِمِينَ ﴿١٦٣﴾
 قُل أَغَيرَ ٱللَّهِ أَبغِى رَبًّۭا وَهُوَ رَبُّ كُلِّ شَىءٍۢ ۚ وَلَا تَكسِبُ كُلُّ نَفسٍ إِلَّا عَلَيهَا ۚ وَلَا تَزِرُ وَازِرَةٌۭ وِزرَ أُخرَىٰ ۚ ثُمَّ إِلَىٰ رَبِّكُم مَّرجِعُكُم فَيُنَبِّئُكُم بِمَا كُنتُم فِيهِ تَختَلِفُونَ ﴿١٦٤﴾
 وَهُوَ ٱلَّذِى جَعَلَكُم خَلَـٰٓئِفَ ٱلأَرضِ وَرَفَعَ بَعضَكُم فَوقَ بَعضٍۢ دَرَجَٰتٍۢ لِّيَبلُوَكُم فِى مَآ ءَاتَىٰكُم ۗ إِنَّ رَبَّكَ سَرِيعُ ٱلعِقَابِ وَإِنَّهُۥ لَغَفُورٌۭ رَّحِيمٌۢ ﴿١٦٥﴾
 
