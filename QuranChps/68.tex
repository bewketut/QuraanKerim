%% License: BSD style (Berkley) (i.e. Put the Copyright owner's name always)
%% Writer and Copyright (to): Bewketu(Bilal) Tadilo (2016-17)
\shadowbox{\section{\LR{\textamharic{ሱራቱ አልቀለም -}  \RL{سوره  القلم}}}}

  
    
  
    
    

\nopagebreak
  بِسمِ ٱللَّهِ ٱلرَّحمَـٰنِ ٱلرَّحِيمِ
  نٓ ۚ وَٱلقَلَمِ وَمَا يَسطُرُونَ ﴿١﴾
 مَآ أَنتَ بِنِعمَةِ رَبِّكَ بِمَجنُونٍۢ ﴿٢﴾
 وَإِنَّ لَكَ لَأَجرًا غَيرَ مَمنُونٍۢ ﴿٣﴾
 وَإِنَّكَ لَعَلَىٰ خُلُقٍ عَظِيمٍۢ ﴿٤﴾
 فَسَتُبصِرُ وَيُبصِرُونَ ﴿٥﴾
 بِأَييِّكُمُ ٱلمَفتُونُ ﴿٦﴾
 إِنَّ رَبَّكَ هُوَ أَعلَمُ بِمَن ضَلَّ عَن سَبِيلِهِۦ وَهُوَ أَعلَمُ بِٱلمُهتَدِينَ ﴿٧﴾
 فَلَا تُطِعِ ٱلمُكَذِّبِينَ ﴿٨﴾
 وَدُّوا۟ لَو تُدهِنُ فَيُدهِنُونَ ﴿٩﴾
 وَلَا تُطِع كُلَّ حَلَّافٍۢ مَّهِينٍ ﴿١٠﴾
 هَمَّازٍۢ مَّشَّآءٍۭ بِنَمِيمٍۢ ﴿١١﴾
 مَّنَّاعٍۢ لِّلخَيرِ مُعتَدٍ أَثِيمٍ ﴿١٢﴾
 عُتُلٍّۭ بَعدَ ذَٟلِكَ زَنِيمٍ ﴿١٣﴾
 أَن كَانَ ذَا مَالٍۢ وَبَنِينَ ﴿١٤﴾
 إِذَا تُتلَىٰ عَلَيهِ ءَايَـٰتُنَا قَالَ أَسَـٰطِيرُ ٱلأَوَّلِينَ ﴿١٥﴾
 سَنَسِمُهُۥ عَلَى ٱلخُرطُومِ ﴿١٦﴾
 إِنَّا بَلَونَـٰهُم كَمَا بَلَونَآ أَصحَـٰبَ ٱلجَنَّةِ إِذ أَقسَمُوا۟ لَيَصرِمُنَّهَا مُصبِحِينَ ﴿١٧﴾
 وَلَا يَستَثنُونَ ﴿١٨﴾
 فَطَافَ عَلَيهَا طَآئِفٌۭ مِّن رَّبِّكَ وَهُم نَآئِمُونَ ﴿١٩﴾
 فَأَصبَحَت كَٱلصَّرِيمِ ﴿٢٠﴾
 فَتَنَادَوا۟ مُصبِحِينَ ﴿٢١﴾
 أَنِ ٱغدُوا۟ عَلَىٰ حَرثِكُم إِن كُنتُم صَـٰرِمِينَ ﴿٢٢﴾
 فَٱنطَلَقُوا۟ وَهُم يَتَخَـٰفَتُونَ ﴿٢٣﴾
 أَن لَّا يَدخُلَنَّهَا ٱليَومَ عَلَيكُم مِّسكِينٌۭ ﴿٢٤﴾
 وَغَدَوا۟ عَلَىٰ حَردٍۢ قَـٰدِرِينَ ﴿٢٥﴾
 فَلَمَّا رَأَوهَا قَالُوٓا۟ إِنَّا لَضَآلُّونَ ﴿٢٦﴾
 بَل نَحنُ مَحرُومُونَ ﴿٢٧﴾
 قَالَ أَوسَطُهُم أَلَم أَقُل لَّكُم لَولَا تُسَبِّحُونَ ﴿٢٨﴾
 قَالُوا۟ سُبحَـٰنَ رَبِّنَآ إِنَّا كُنَّا ظَـٰلِمِينَ ﴿٢٩﴾
 فَأَقبَلَ بَعضُهُم عَلَىٰ بَعضٍۢ يَتَلَـٰوَمُونَ ﴿٣٠﴾
 قَالُوا۟ يَـٰوَيلَنَآ إِنَّا كُنَّا طَٰغِينَ ﴿٣١﴾
 عَسَىٰ رَبُّنَآ أَن يُبدِلَنَا خَيرًۭا مِّنهَآ إِنَّآ إِلَىٰ رَبِّنَا رَٰغِبُونَ ﴿٣٢﴾
 كَذَٟلِكَ ٱلعَذَابُ ۖ وَلَعَذَابُ ٱلءَاخِرَةِ أَكبَرُ ۚ لَو كَانُوا۟ يَعلَمُونَ ﴿٣٣﴾
 إِنَّ لِلمُتَّقِينَ عِندَ رَبِّهِم جَنَّـٰتِ ٱلنَّعِيمِ ﴿٣٤﴾
 أَفَنَجعَلُ ٱلمُسلِمِينَ كَٱلمُجرِمِينَ ﴿٣٥﴾
 مَا لَكُم كَيفَ تَحكُمُونَ ﴿٣٦﴾
 أَم لَكُم كِتَـٰبٌۭ فِيهِ تَدرُسُونَ ﴿٣٧﴾
 إِنَّ لَكُم فِيهِ لَمَا تَخَيَّرُونَ ﴿٣٨﴾
 أَم لَكُم أَيمَـٰنٌ عَلَينَا بَٰلِغَةٌ إِلَىٰ يَومِ ٱلقِيَـٰمَةِ ۙ إِنَّ لَكُم لَمَا تَحكُمُونَ ﴿٣٩﴾
 سَلهُم أَيُّهُم بِذَٟلِكَ زَعِيمٌ ﴿٤٠﴾
 أَم لَهُم شُرَكَآءُ فَليَأتُوا۟ بِشُرَكَآئِهِم إِن كَانُوا۟ صَـٰدِقِينَ ﴿٤١﴾
 يَومَ يُكشَفُ عَن سَاقٍۢ وَيُدعَونَ إِلَى ٱلسُّجُودِ فَلَا يَستَطِيعُونَ ﴿٤٢﴾
 خَـٰشِعَةً أَبصَـٰرُهُم تَرهَقُهُم ذِلَّةٌۭ ۖ وَقَد كَانُوا۟ يُدعَونَ إِلَى ٱلسُّجُودِ وَهُم سَـٰلِمُونَ ﴿٤٣﴾
 فَذَرنِى وَمَن يُكَذِّبُ بِهَـٰذَا ٱلحَدِيثِ ۖ سَنَستَدرِجُهُم مِّن حَيثُ لَا يَعلَمُونَ ﴿٤٤﴾
 وَأُملِى لَهُم ۚ إِنَّ كَيدِى مَتِينٌ ﴿٤٥﴾
 أَم تَسـَٔلُهُم أَجرًۭا فَهُم مِّن مَّغرَمٍۢ مُّثقَلُونَ ﴿٤٦﴾
 أَم عِندَهُمُ ٱلغَيبُ فَهُم يَكتُبُونَ ﴿٤٧﴾
 فَٱصبِر لِحُكمِ رَبِّكَ وَلَا تَكُن كَصَاحِبِ ٱلحُوتِ إِذ نَادَىٰ وَهُوَ مَكظُومٌۭ ﴿٤٨﴾
 لَّولَآ أَن تَدَٟرَكَهُۥ نِعمَةٌۭ مِّن رَّبِّهِۦ لَنُبِذَ بِٱلعَرَآءِ وَهُوَ مَذمُومٌۭ ﴿٤٩﴾
 فَٱجتَبَٰهُ رَبُّهُۥ فَجَعَلَهُۥ مِنَ ٱلصَّـٰلِحِينَ ﴿٥٠﴾
 وَإِن يَكَادُ ٱلَّذِينَ كَفَرُوا۟ لَيُزلِقُونَكَ بِأَبصَـٰرِهِم لَمَّا سَمِعُوا۟ ٱلذِّكرَ وَيَقُولُونَ إِنَّهُۥ لَمَجنُونٌۭ ﴿٥١﴾
 وَمَا هُوَ إِلَّا ذِكرٌۭ لِّلعَـٰلَمِينَ ﴿٥٢﴾
 
