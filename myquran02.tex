%% License: BSD style (Berkley) (i.e. Put the Copyright owner's name always)
%% Writer and Copyright (to): Bewketu(Bilal) Tadilo (2016-17)
\documentclass[11pt,a4paper,oneside,openleft]{l3doc}%, ,fleqn oneside]
\usepackage[top=2cm,bottom=2cm,left=3cm,right=3cm,headsep=10pt,a4paper]{geometry}
\usepackage[no-math]{fontspec}
\usepackage{graphicx}
%usepackage{arabxetex}
%usepackage{titletoc}
\usepackage{booktabs}
\usepackage{longtable}
\usepackage{polyglossia}
\usepackage{ifxetex}
\usepackage{xcolor} % Required for specifying colors by name
\usepackage{expl3}
\usepackage{hyperref}
%usepackage{fancyhdr}
%citecolor={dark-blue},urlcolor={dark-blue}
\hypersetup{colorlinks,linkcolor={black}, pdfauthor= {Bilal Al Gonder/x-bewketu}, pdftitle={ቁርኣን ከሪም-ኢሥላም}, pdfsubject={ቁርኣን,ኢሥላም}}

%\titlecontents{section}[1 25cm] % Indentation
%{\addvspace{5pt}\bfseries} % Spacing and font options for sections
%{\contentslabel[\thecontentslabel]{1
%25cm}} % Section number
%{}
%{\hfill\color{black}\thecontentspage} % Page number
%[]
%usepackage{underscore}
\newfontfamily\arabicfont[Script=Arabic, Scale = 1.5]{mequran}%KFGQPC Uthmanic Script HAFS}% Taha Naskh}
%\newfontfamily\arabicfonttt[Script = Arabic, Scale = 1.5]{Amiri}%KFGQPC Uthmanic Script HAFS}% Taha Naskh}
\newfontfamily\amharicfont[Script=Ethiopic, Scale = 1.1]{Abyssinica SIL}
\setmainlanguage[locale=mashriq]{arabic}
\setotherlanguage{amharic}
\newcommand{\mytextarabic}[1]{\RL{ #1 \flushright}}
%usepackage{array}[numerals=mashriq]
\newcommand{\enq}[1]{\rq\rq{}#1\rq\rq{}}
%newcommand{\textLANGUAGENAME}[1]{\begin{langname}#1\end{langname}}
 %put here your language name
\newcommand{\textamh}[1]{\LR{\begin{amharic}#1\flushleft\end{amharic}}}
\newcommand{\textamhsec}[1]{\begin{amharic} #1 \end{amharic}}
%\selectbackgroundlanguage{arabic}
%title{}
%\author{}
\begin{document}
%\maketitle
%\Hijritoday[0] 
%\textamharic{\today-} 
\pagenumbering{}
\tableofcontents{}
%\%%%%%%%%%%%%%%%%%%%%%%%%
%%ــٰ
%TODO: please substitute alif which is broken by      ـٰ
%%%%%%%%%%%%%%%%%%%%%%
\cleardoublepage
\pagenumbering{arabic}
%pagestyle{fancy}
\centering\section{\LR{\textamharic{ሱራቱ አልፈቲሃ - } \RL{سوره  الفاتحة}}}
\begin{longtable}{%
  @{}
    p{.4\textwidth}
  @{~~~~~~~~~~~~~}
    p{.4\textwidth}
    @{}
}
\textamh{1.\ ቢስሚላሂ አራህመኒ ራሂይም  } &  بِسْمِ ٱللَّهِ الرَّحْمَـٰنِ الرَّحِيمِ﴿١﴾     \\
\textamh{2.\ (ኣልሃምዱሊላሂ) ምስጋና ሁሉ ለኣላህ የአለሚን (የሰዎች፥ ጅኖች፥ ያለ ነገር ሁሉ) ጌታ   } & ٱلْحَمْدُ لِلَّهِ رَبِّ ٱلْعَـٰلَمِينَ﴿٢﴾ \\
\textamh{3.\ ከሁሉም በላይ ሰጪ፥ ከሁሉም በላይ ምህረተኛው   } & ٱلرَّحْمَـٰنِ ٱلرَّحِيمِ﴿٣﴾   \\
\textamh{4.\ የዛች ቀን (የፍርድ ቀን) ብቸኛ ባለቤት   } &   مَـٰلِكِ يَوْمِ ٱلدِّينِ ﴿٤﴾   \\
\textamh{5.\ አንተን ብቻ እናመልካለን፤ አንተን ብቻ እርዳታ እንጠይቃለን   } &  إِيَّاكَ نَعْبُدُ وَإِيَّاكَ نَسْتَعِينُ ﴿٥﴾   \\
\textamh{6.\ ምራነ በቀጥተኛው (በትክክለኛው)  መንገድ   } &  ٱهْدِنَا ٱلصِّرَٟطَ ٱلْمُسْتَقِيمَ ﴿٦﴾  \\
\textamh{7.\ የአንተን ፀጋ ያደረግክላቸውን (ሰዎች)  መንገድ፥ የአንተን ቁጣ እንዳተርፉት (እንደይሁዶች) ሳይሆን ፥እንደሳቱትም (እንደክርስቲያኖች) ሳይሆን } &   صِرَٟطَ ٱلَّذِينَ أَنْعَمْتَ عَلَيْهِمْ غَيْرِ ٱلْمَغْضُوبِ عَلَيْهِمْ وَلَا ٱلضَّآلِّينَ ﴿٧﴾ 
\end{longtable} 
\clearpage
\centering\section{\LR{\textamharic{ሱራቱ አልበቀራ -}  \RL{سوره  البقرة}}}
\begin{longtable}{%
  @{}
    p{.5\textwidth}
  @{~~~~~~~~~~~~~}
    p{.5\textwidth}
    @{}
}
\textamh{\ \ \ \ \ \ ቢስሚላሂ አራህመኒ ራሂይም } &  بِسْمِ ٱللَّهِ ٱلرَّحْمَـٰنِ ٱلرَّحِيمِ\\
\textamh{1.\ አሊፍ ላም ሚም (የፊደላቱን ትርጉም ኣላህ ብቻ ያዉቃል)  } &  الٓمٓ ﴿١﴾  \\
\textamh{2.\ ይሄ ነው መጽሃፉ፥ ያለ ምንም ጥርጥር አምላክን ለሚፈሩ መሪ የሆነ   } &  ذَٟلِكَ ٱلْكِتَـٰبُ لَا رَيْبَ ۛ فِيهِ ۛ هُدًۭى لِّلْمُتَّقِينَ ﴿٢﴾  \\
\textamh{3.\ በማይታየው(ጋይብ) የሚያምኑ፥ሳላት የሚቆሙ የሰጠናቸዉን (የረዘቅናቸውን) የሚሰጡ   } & ٱلَّذِينَ يُؤْمِنُونَ بِٱلْغَيْبِ وَيُقِيمُونَ ٱلصَّلَوٰةَ وَمِمَّا رَزَقْنَـٰهُمْ يُنفِقُونَ ﴿٣﴾  \\
\textamh{4.\ ለአንተ በወርደው (ኦ! ሙሐመድ(ሠአወሰ)) (በዚህ ቁርአን) የሚያምኑ፤ ደግሞ ከአንተ በፊት በወረደላቸው (ተውራት፥ ወንጌል) እና በሰማያዊ ህይወት (አኪራ) ያለ ምንም ጥርጥር የሚያምኑ   } &  وَٱلَّذِينَ يُؤْمِنُونَ بِمَآ أُنزِلَ إِلَيْكَ وَمَآ أُنزِلَ مِن قَبْلِكَ وَبِٱلْءَاخِرَةِ هُمْ يُوقِنُونَ ﴿٤﴾  \\
\textamh{5.\ እነሱ ናቸው ከአምላካቸው ምሬት ያገኙ (የተመሩ) እነሱም ናቸው (በስኬት) አላፊዎች   } &  أُو۟لَـٰٓئِكَ عَلَىٰ هُدًۭى مِّن رَّبِّهِمْ ۖ وَأُو۟لَـٰٓئِكَ هُمُ ٱلْمُفْلِحُونَ ﴿٥﴾   \\
\textamh{6.\ በእውነት ለእነዚያ ለማይምኑት (ካፊሮች) (ኦ! ሙሐመድ(ሠአወሰ))ብታስጠነቅቃቸውም ባታስጠነቅቃቸውም አንድ ነው፤ አያምኑም።   } &  إِنَّ ٱلَّذِينَ كَفَرُوا۟ سَوَآءٌ عَلَيْهِمْ ءَأَنذَرْتَهُمْ أَمْ لَمْ تُنذِرْهُمْ لَا يُؤْمِنُونَ ﴿٦﴾  \\
\textamh{7.\ ኣላህ ልባቸዉን አትሞታል መስሚያቸውንም እንደዚያው፥ ማያቸው ላይ ግርዶሽ አለ፤ ለነሱ ታላቅ ቅጣት ይጠብቃቸዋል።   } &  خَتَمَ ٱللَّهُ عَلَىٰ قُلُوبِهِمْ وَعَلَىٰ سَمْعِهِمْ ۖ وَعَلَىٰٓ أَبْصَـٰرِهِمْ غِشَـٰوَةٌۭ ۖ وَلَهُمْ عَذَابٌ عَظِيمٌۭ ﴿٧﴾  \\
\textamh{8.\ ከሰዎች መካከል ደግሞ በኣላህ እና በፍርድ ቀን (የውሚ አኪራ) እናምናለን  የሚሉ አሉ፤ ግን አማኞች አይደሉም።   } &  وَمِنَ ٱلنَّاسِ مَن يَقُولُ ءَامَنَّا بِٱللَّهِ وَبِٱلْيَوْمِ ٱلْءَاخِرِ وَمَا هُم بِمُؤْمِنِينَ ﴿٨﴾  \\
\textamh{9.\ ኣላህንና አማኞችን ሊያጭበረብሩ (ያስባሉ)፤  ከራሳቸው በቀር ማንንም አያጭበረብሩም፤ ግን አያውቁትም።   } &  يُخَـٰدِعُونَ ٱللَّهَ وَٱلَّذِينَ ءَامَنُوا۟ وَمَا يَخْدَعُونَ إِلَّآ أَنفُسَهُمْ وَمَا يَشْعُرُونَ ﴿٩﴾  \\
\textamh{10.\ ልባቸው ዉስጥ በሽታ አለ (የጥርጣሬና የንፍቀት) ኣላህም በሽታቸውን ጨምሮበታል፤ አሰቃቂ ስቃይ ለነሱ ይሆናል ሃሰት ሲናገሩ ስለቆዩ } & 
\  فِى قُلُوبِهِم مَّرَضٌۭ فَزَادَهُمُ ٱللَّهُ مَرَضًۭا ۖ وَلَهُمْ عَذَابٌ أَلِيمٌۢ بِمَا كَانُوا۟ يَكْذِبُونَ ﴿١٠﴾  \\
\textamh{11.\ \rq\rq{}ምድር (መሬት) ላይ አትበጥብጡ\rq\rq{} ሲባሉ፥ \rq\rq{}እኛ እኮ ሰላም ፈጣሪዎች ነን\rq\rq{} ይላሉ   } &  وَإِذَا قِيلَ لَهُمْ لَا تُفْسِدُوا۟ فِى ٱلْأَرْضِ قَالُوٓا۟ إِنَّمَا نَحْنُ مُصْلِحُونَ ﴿١١﴾  \\
\textamh{12.\ በእዉነት! ራሳቸው ናቸው በጥባጮቹ ግን አያዉቁትም።   } &  أَلَآ إِنَّهُمْ هُمُ ٱلْمُفْسِدُونَ وَلَٟكِن لَّا يَشْعُرُونَ ﴿١٢﴾ \\
\textamh{13.\ \rq\rq{}እመኑ ልክ እንደአማኞቹ ሰዎች\rq\rq{} ሲባሉ፥ \rq\rq{}ሞኞቹ እንዳመኑት እንመን እንዴ?\rq\rq{} አሉ። በእዉነት! እነሱው ናቸው ሞኞቹ ግን አያውቁትም።   } &  وَإِذَا قِيلَ لَهُمْ ءَامِنُوا۟ كَمَآ ءَامَنَ ٱلنَّاسُ قَالُوٓا۟ أَنُؤْمِنُ كَمَآ ءَامَنَ ٱلسُّفَهَآءُ ۗ أَلَآ إِنَّهُمْ هُمُ ٱلسُّفَهَآءُ وَلَٟكِن لَّا يَعْلَمُونَ ﴿١٣﴾   \\
\textamh{14.\ አማኞችን ሲያገኙ \rq\rq{}እናምናለን\rq\rq{} ይላሉ፤ ነገር ግን ከሰይጣኖቻቸው (ሌሎች መናፍቃን) ጋር ብቻቸውን ሲሆኑ \rq\rq{}በእውነት ከናንት ጋር ነን፤ ስናሾፍ ነው የነበር\rq\rq{} ይላሉ።   } &  وَإِذَا لَقُوا۟ ٱلَّذِينَ ءَامَنُوا۟ قَالُوٓا۟ ءَامَنَّا وَإِذَا خَلَوْا۟ إِلَىٰ شَيَـٰطِينِهِمْ قَالُوٓا۟ إِنَّا مَعَكُمْ إِنَّمَا نَحْنُ مُسْتَهْزِءُونَ ﴿١٤﴾ \\
\textamh{15.\ ኣላህ ራሱ ያላግጥባቸዋል፥ እንዲቅበዘበዙ መጥፎ ስራቸዉን ያበዘላቸዋል።  } &  ٱللَّهُ يَسْتَهْزِئُ بِهِمْ وَيَمُدُّهُمْ فِى طُغْيَـٰنِهِمْ يَعْمَهُونَ ﴿١٥﴾\\ 
\textamh{16.\ እነዚህ ናቸው ምሬት (መመራትን) ባለመመራት የገዙት፤ ንግዳቸውም ትርፍ አልባ ሁኖ ቀረ። ሳይመሩ ቀሩ።   } &  أُو۟لَٟٓئِكَ ٱلَّذِينَ ٱشْتَرَوُا۟ ٱلضَّلَٟلَةَ بِٱلْهُدَىٰ فَمَا رَبِحَت تِّجَٟرَتُهُمْ وَمَا كَانُوا۟ مُهْتَدِينَ ﴿١﴾ \\
\textamh{17.\ ምሳሌቸው ልክ እሳት እንዳቃጠለ ሰው ነው፤ ነዶ ብረሃን ሲሆንለት ኣላህ ብረሃናቸዉን ወስዶ ጨለማ ዉስጥ ከተታቸው። ማየት  አይችሉም።
\ } &   مَثَلُهُمْ كَمَثَلِ ٱلَّذِى ٱسْتَوْقَدَ نَارًۭا فَلَمَّآ أَضَآءَتْ مَا حَوْلَهُۥ ذَهَبَ ٱللَّهُ بِنُورِهِمْ وَتَرَكَهُمْ فِى ظُلُمَـٰتٍۢ لَّا يُبْصِرُونَ ﴿١٧﴾\\ 
\textamh{18.\ ደንቆሮ፥ ዲዳ፥ እና እዉር ናቸው፤  አይመለሱም።    } &   صُمٌّۢ بُكْمٌ عُمْىٌۭ فَهُمْ لَا يَرْجِعُونَ ﴿١٨﴾\\
\textamh{19.\ ወይም ደግሞ ልክ እንደ ድቅድቅ ደምና  ዉስጡ ጨለማ፥ ነጎድጓድ (ረአድ)፥በርቅ  (ብልጭታ)ጣታቸዉን ጆሯቸው ዉስጥ  ይከታሉ ከበርቁ ድምጽ የሞት ፍርሃት የተነሳ።  ኣላህ ግን የማይምኑትን አጥሮ ይይዛል።   } &   أَوْ كَصَيِّبٍۢ مِّنَ ٱلسَّمَآءِ فِيهِ ظُلُمَـٰتٌۭ وَرَعْدٌۭ وَبَرْقٌۭ يَجْعَلُونَ أَصَٟبِعَهُمْ فِىٓ ءَاذَانِهِم مِّنَ ٱلصَّوَٟعِقِ حَذَرَ ٱلْمَوْتِ ۚ وَٱللَّهُ مُحِيطٌۢ بِٱلْكَٟفِرِينَ ﴿١٩﴾\\
\textamh{20.\ ብልጭታዉ ማያቸዉን ይወስዳል፥ ሲበራ በዚያ ይሄዳሉ፥ ጨለማ ሲሆን ደግሞ ይቆማሉ፤ ኣላህ ቢፈቅድ ኑሮ መስሚያቸዉንና ማያቸውን ይወስድ ነበር። በእርግጠኛንት ኣላህ ሁሉን ማድረግ ይችላል።   } &  يَكَادُ ٱلْبَرْقُ يَخْطَفُ أَبْصَٟرَهُمْ ۖ كُلَّمَآ أَضَآءَ لَهُم مَّشَوْا۟ فِيهِ وَإِذَآ أَظْلَمَ عَلَيْهِمْ قَامُوا۟ ۚ وَلَوْ شَآءَ ٱللَّهُ لَذَهَبَ بِسَمْعِهِمْ وَأَبْصَٟرِهِمْ ۚ إِنَّ ٱللَّهَ عَلَىٰ كُلِّ شَىْءٍۢ قَدِيرٌۭ ﴿٢٠﴾\\
\textamh{21.\ ኦ! ሰዎች ሆይ፥ አምላካችሁን አምልኩ እናንተንም ሆነ ከናንተ በፊት የነበሩትን የፈጠረ እናንተም ሙታቁን (አምለክ ተገዥ/ፈሪ) እንድትሆኑ።   } &  يَـٰٓأَيُّهَا ٱلنَّاسُ ٱعْبُدُوا۟ رَبَّكُمُ ٱلَّذِى خَلَقَكُمْ وَٱلَّذِينَ مِن قَبْلِكُمْ لَعَلَّكُمْ تَتَّقُونَ ﴿٢١﴾\\
\textamh{22.\ መሬትን (ምድርን) (እንደፍራሽ) ማረፊያ ሰማይን መከለያ ያደረገላችሁ እናም  ከሰማይ ዉሃ አወረደ፥ በዚያም አዝእርትና ፍራፍሬ አበቀለላችሁ ለናንተ ሪዝቅ የሚሆን። ስለዚህ ለኣላህ ሌላ እኩያ አታድርጉ፤ እያወቃችሁ ሳል (እሱ ብቻ መመለክ እንዳለበት)   } &  ٱلَّذِى جَعَلَ لَكُمُ ٱلْأَرْضَ فِرَٟشًۭا وَٱلسَّمَآءَ بِنَآءًۭ وَأَنزَلَ مِنَ ٱلسَّمَآءِ مَآءًۭ فَأَخْرَجَ بِهِۦ مِنَ ٱلثَّمَرَٟتِ رِزْقًۭا لَّكُمْ ۖ فَلَا تَجْعَلُوا۟ لِلَّهِ أَندَادًۭا وَأَنتُمْ تَعْلَمُونَ ﴿٢٢﴾\\
\textamh{23.\ ለባሪያችን (ሙሐመድ(ሠአወሰ)) ባወርደነው (ቁረአን) ጥርጣሬ ካላችሁ (እናነተ ፓጋን አረቦችና አይሁዶች) እስኪ በሉ አንድ እንዲህ ያለ ምእራፍ (ሱራ) አምጡ (ፍጠሩ) እና ከኣላህ በቀር ምስክሮቻችሁን (ረዳቶቻችሁን) ጥሩ ፤እዉነተኛ ከሆናችሁ   } &  وَإِن كُنتُمْ فِى رَيْبٍۢ مِّمَّا نَزَّلْنَا عَلَىٰ عَبْدِنَا فَأْتُوا۟ بِسُورَةٍۢ مِّن مِّثْلِهِۦ وَٱدْعُوا۟ شُهَدَآءَكُم مِّن دُونِ ٱللَّهِ إِن كُنتُمْ صَٟدِقِينَ ﴿٢٣﴾\\
\textamh{24.\ ካላደረጋችሁ ግን ደግሞም አታደርጉትም  ማቀጣጠያውና ነዳጁ ሰዉና ድንጋይ የሆኑበትን እሳት ፍሩ (ጀሃነም)፤ ለከሃዲዎች (ለማያምኑት) የተዘጋጀ።   } &  فَإِن لَّمْ تَفْعَلُوا۟ وَلَن تَفْعَلُوا۟ فَٱتَّقُوا۟ ٱلنَّارَ ٱلَّتِى وَقُودُهَا ٱلنَّاسُ وَٱلْحِجَارَةُ ۖ أُعِدَّتْ لِلْكَٟفِرِينَ ﴿٢٤﴾\\
\textamh{25.\ አማኞች ሁነው ጥሩ ስራ ለሚሰሩ አብስር(ሩ) ለነሱ ገነት (ጀነት)፥ በስራቸዉ ወንዞች የሚፈሱበት፥ ሁሌ ከዚያ ፍራፍሬ ሲሰጡ \rq\rq{}እንደዚህ አይነት  በፊት ተሰጥቶናል\rq\rq{} ይላሉ (ያስታዉሳሉ)እናም  በአምሳያ ይሰጣቸዋል (አንድ አይነት ግን ጣእሙ  የተለያየ)፤ እዚያም ጠሃራ (ንጹህ) የሆኑ ሚስቶች ይኖሯቸዋል፤ ለዘላለሙ ይቀመጣሉ።   } &  وَبَشِّرِ ٱلَّذِينَ ءَامَنُوا۟ وَعَمِلُوا۟ ٱلصَّٟلِحَٟتِ أَنَّ لَهُمْ جَنَّٟتٍۢ تَجْرِى مِن تَحْتِهَا ٱلْأَنْهَـٰرُ ۖ كُلَّمَا رُزِقُوا۟ مِنْهَا مِن ثَمَرَةٍۢ رِّزْقًۭا ۙ قَالُوا۟ هَـٰذَا ٱلَّذِى رُزِقْنَا مِن قَبْلُ ۖ وَأُتُوا۟ بِهِۦ مُتَشَٟبِهًۭا ۖ وَلَهُمْ فِيهَآ أَزْوَٟجٌۭ مُّطَهَّرَةٌۭ ۖ وَهُمْ فِيهَا خَـٰلِدُونَ ﴿٢٥﴾ ۞\\
\textamh{26.\ በእዉነት ኣላህ ምሳሌ (በትንሿም) በትንኝ ወይም ከሷም ባነሰ ወይም በተለቀ ለማቅረብ አያፍርም፤ ለሚያምኑት እዉነቱ (ሀቁ) ከአምላካቸው  እንደሆነ ያዉቃሉ፤ የማየምኑት ግን \rq\rq{}ኣላህ በዚህ ምሳሌ ምን አስቦ (ማለቱ) ነው?\rq\rq{} ይላሉ።  በዚያ ግን ብዙዎችን ያስታል፥ ብዙዎችንም ይመራል የሚያስተው ፋሲቁን (የማይገዙለትን፥ የሚያምጹትን) ነው።   } &   إِنَّ ٱللَّهَ لَا يَسْتَحْىِۦٓ أَن يَضْرِبَ مَثَلًۭا مَّا بَعُوضَةًۭ فَمَا فَوْقَهَا ۚ فَأَمَّا ٱلَّذِينَ ءَامَنُوا۟ فَيَعْلَمُونَ أَنَّهُ ٱلْحَقُّ مِن رَّبِّهِمْ ۖ وَأَمَّا ٱلَّذِينَ كَفَرُوا۟ فَيَقُولُونَ مَاذَآ أَرَادَ ٱللَّهُ بِهَـٰذَا مَثَلًۭا ۘ يُضِلُّ بِهِۦ كَثِيرًۭا وَيَهْدِى بِهِۦ كَثِيرًۭا ۚ وَمَا يُضِلُّ بِهِۦٓ إِلَّا ٱلْفَٟسِقِينَ ﴿٢٦﴾\\
\textamh{27.\ የኣላህን ዉል ስምምነት ከገቡ በኋላ የሚበጥሱ፥ እዲደረግ ያዘዘዉን የሚያጣሙ (የሚያፈርሱ) እና ምድር (መሬት) ላይ የሚበጠብጡ፥እነሱ ናቸው ካሲሩን (የሚከስሩ)   } &  ٱلَّذِينَ يَنقُضُونَ عَهْدَ ٱللَّهِ مِنۢ بَعْدِ مِيثَٟقِهِۦ وَيَقْطَعُونَ مَآ أَمَرَ ٱللَّهُ بِهِۦٓ أَن يُوصَلَ وَيُفْسِدُونَ فِى ٱلْأَرْضِ ۚ أُو۟لَٟٓئِكَ هُمُ ٱلْخَـٰسِرُونَ ﴿٢٧﴾\\
\textamh{28.\ እንዴት በኣላህ አታምኑም? ሙት እንደነበራችሁ  እያያችሁ ህይወት ሰጣችሁ። ከዚያም ሞትን ይሰጣችኋል፥ ከዚያም ደግሞ ህይወት ይስጣችኋል (ያስነሳችኋል የትንሳኤ ቀን፥ የፍርድ ቀን) ከዚያም ወደሱ ትመለሳላችሁ።   } &  كَيْفَ تَكْفُرُونَ بِٱللَّهِ وَكُنتُمْ أَمْوَٟتًۭا فَأَحْيَـٰكُمْ ۖ ثُمَّ يُمِيتُكُمْ ثُمَّ يُحْيِيكُمْ ثُمَّ إِلَيْهِ تُرْجَعُونَ ﴿٢٨﴾\\
\textamh{29.\ እሱ እኮ ነው ምድር ላይ ያለዉን ሁሉ ለእናንተ የፈጠረው። ከዚያም ከፍ ብሎ (ኢስትወ) ወደ ሰማይ ሰባት ሰማያት አደረጋቸው እና የሁሉ  ነገር አዋቂ ነው   } &  هُوَ ٱلَّذِى خَلَقَ لَكُم مَّا فِى ٱلْأَرْضِ جَمِيعًۭا ثُمَّ ٱسْتَوَىٰٓ إِلَى ٱلسَّمَآءِ فَسَوَّىٰهُنَّ سَبْعَ سَمَـٰوَٟتٍۢ ۚ وَهُوَ بِكُلِّ شَىْءٍ عَلِيمٌۭ ﴿٢٩﴾\\
\textamh{30.\ አምላክህ ለመላኢክት (እንዲህ) አላቸው: \rq\rq{}በእዉነት (ሰዉን) ትውልድ በትውልድ ምድር  ላይ ላስቀምጥ ነው\rq\rq{}።  (እንዲህ) አሉ: \rq\rq{}የሚበጠብጥና ደምን የሚያፈስ ታስቀምጣለህን? እኛ ስባሃትክንና  ምስጋናህንን (እያደርግን) እና እየቀደስነህ\rq\rq{} (ኣላህ) አለ: \rq\rq{}እናንተ የማተውቁትን አዉቃለሁ\rq\rq{}   } &  وَإِذْ قَالَ رَبُّكَ لِلْمَلَٟٓئِكَةِ إِنِّى جَاعِلٌۭ فِى ٱلْأَرْضِ خَلِيفَةًۭ ۖ قَالُوٓا۟ أَتَجْعَلُ فِيهَا مَن يُفْسِدُ فِيهَا وَيَسْفِكُ ٱلدِّمَآءَ وَنَحْنُ نُسَبِّحُ بِحَمْدِكَ وَنُقَدِّسُ لَكَ ۖ قَالَ إِنِّىٓ أَعْلَمُ مَا لَا تَعْلَمُونَ ﴿٣٠﴾\\
\textamh{31.\ እና አደምን (አዳም) ሁሉን ስም (የሁሉን ነገር) አስተማረው፤ ከዚያም ለመላኢክት (ሁሉን) አሳየና \rq\rq{}በሉ የነዚህን ስም ካወቃችሁ ንገሩኝ እዉነተኛ ከሆናችሁ\rq\rq{} አላቸው።   } &  وَعَلَّمَ ءَادَمَ ٱلْأَسْمَآءَ كُلَّهَا ثُمَّ عَرَضَهُمْ عَلَى ٱلْمَلَٟٓئِكَةِ فَقَالَ أَنۢبِـُٔونِى بِأَسْمَآءِ هَـٰٓؤُلَآءِ إِن كُنتُمْ صَٟدِقِينَ ﴿٣١﴾\\

\textamh{32.\ (እነሱም) አሉ: \rq\rq{}ስብሃት ለአንተ ይሁን፥ አንተ ካስተመርከነ ዉጭ ሌላ እዉቀት የለነም፥ አንተ ነህ ሁሉን አወቂ፥ ሁሉን መርማሪ (ጥበበኛ) ነህ   } &  قَالُوا۟ سُبْحَٟنَكَ لَا عِلْمَ لَنَآ إِلَّا مَا عَلَّمْتَنَآ ۖ إِنَّكَ أَنتَ ٱلْعَلِيمُ ٱلْحَكِيمُ ﴿٣٢﴾\\
\textamh{33.\ (ኣላህ) አለ: \rq\rq{}ያኣ አደም (አዳም)! ስማቸዉን ንገራቸው\rq\rq{}፥ (አደምም) ስማቸዉን ከነገራቸው  በኋላ (ኣላህ) አለ: \rq\rq{}የማይታየዉን በሰማይና በምድር ዉስጥ ያለዉን አውቀዋለሁ፥ የደብቀችሁትንም  የምትገልጹትንም አዉቀዋለሁ አላልኳችሁንም?\rq\rq{}   } &  قَالَ يَـٰٓـَٔادَمُ أَنۢبِئْهُم بِأَسْمَآئِهِمْ ۖ فَلَمَّآ أَنۢبَأَهُم بِأَسْمَآئِهِمْ قَالَ أَلَمْ أَقُل لَّكُمْ إِنِّىٓ أَعْلَمُ غَيْبَ ٱلسَّمَـٰوَٟتِ وَٱلْأَرْضِ وَأَعْلَمُ مَا تُبْدُونَ وَمَا كُنتُمْ تَكْتُمُونَ ﴿٣٣﴾\\
\textamh{34.\ ለመላኢክት \rq\rq{}ለአዳም ስገዱ\rq\rq{} አልናቸው እነሱም ሰገዱ ከኢብሊስ (ሰይጣን) በስተቀር እሱ ተቃወመ እና ራሱን ከፍ አደረገ (ኮራ) እናም ከካህዲዎች (ካፊሮች) ሆነ (ኣላህን የማይታዘዝ)።   } &  وَإِذْ قُلْنَا لِلْمَلَٟٓئِكَةِ ٱسْجُدُوا۟ لِءَادَمَ فَسَجَدُوٓا۟ إِلَّآ إِبْلِيسَ أَبَىٰ وَٱسْتَكْبَرَ وَكَانَ مِنَ ٱلْكَٟفِرِينَ ﴿٣٤﴾\\
\textamh{35.\ እና አልነ: \rq\rq{}ያኣ አደሙ! (አንተ አዳም) አንተና ሚስትህ ገነት (ጀነት) ዉስጥ  ተቀመጡ፤ ብሉ በነጻነት የፈለገችሁትንና ያማረችሁን ነገር በሙሉ፤ ነገር ግን ከዚች ዛፍ አትቅረቡ ከመጥፎ ሰሪዎች (ዛሊሙን) መካከል ትሆናላችሁ።   } &  وَقُلْنَا يَـٰٓـَٔادَمُ ٱسْكُنْ أَنتَ وَزَوْجُكَ ٱلْجَنَّةَ وَكُلَا مِنْهَا رَغَدًا حَيْثُ شِئْتُمَا وَلَا تَقْرَبَا هَـٰذِهِ ٱلشَّجَرَةَ فَتَكُونَا مِنَ ٱلظَّٟلِمِينَ ﴿٣٥﴾\\
\textamh{36.\ ከዚያም ሸይጣን (ሰይጣን) ሸተት አደረጋቸው (አሳሳታቸው) ከነበሩበት አስወጣቸው። አልናቸው (  ኣላህ): \rq\rq{}ዉረዱ (ዉጡ)፥ ሁላችሁ፥ እርስበራሰችሁ ጠላት ሁናችሁ። ምድር መኖሪያችሁ ይሆናል ለጊዜዉም መደሰቻ\rq\rq{}   } &  فَأَزَلَّهُمَا ٱلشَّيْطَٟنُ عَنْهَا فَأَخْرَجَهُمَا مِمَّا كَانَا فِيهِ ۖ وَقُلْنَا ٱهْبِطُوا۟ بَعْضُكُمْ لِبَعْضٍ عَدُوٌّۭ ۖ وَلَكُمْ فِى ٱلْأَرْضِ مُسْتَقَرٌّۭ وَمَتَـٰعٌ إِلَىٰ حِينٍۢ ﴿٣٦﴾\\
\textamh{37.\ ከዚያም አዳም (አደም) ከአምላኩ ድምጽ ሰማ፤ ይቅርም አለው። በእዉነት እሱ ብቻ ነው ይቅር ባይ፤ ከሁሉም በላይ ምህረተኛው።   } &  فَتَلَقَّىٰٓ ءَادَمُ مِن رَّبِّهِۦ كَلِمَـٰتٍۢ فَتَابَ عَلَيْهِ ۚ إِنَّهُۥ هُوَ ٱلتَّوَّابُ ٱلرَّحِيمُ ﴿٣٧﴾\\
\textamh{38.\ አልነ ( ኣላህ): \rq\rq{}ሁላችሁም ከዚህ ቦታ ዉረዱ (ዉጡ)፥ ከዚያም ከ እኔ ምሬት (መመሪያ) ሲመጣለችሁ፥ የእኔን መመሪያ የሚከተል፥ ከነሱ ላይ ፍራሀት አይኖርም አያዙኑምም   } &   قُلْنَا ٱهْبِطُوا۟ مِنْهَا جَمِيعًۭا ۖ فَإِمَّا يَأْتِيَنَّكُم مِّنِّى هُدًۭى فَمَن تَبِعَ هُدَاىَ فَلَا خَوْفٌ عَلَيْهِمْ وَلَا هُمْ يَحْزَنُونَ ﴿٣٨﴾\\
\textamh{39.\ ነገር ግን የሚክዱት (የማይምኑት) እና  አያትችን (ምልክታችን፥ ጥቅሳችን፥ ማስረጃችን) የማይቀበሉ፥ እነሱ የእሳቱ ነዋሪዎች ናቸው፥ ለዘላለም ይኖሩበታል።   } &  وَٱلَّذِينَ كَفَرُوا۟ وَكَذَّبُوا۟ بِـَٔايَـٰتِنَآ أُو۟لَٟٓئِكَ أَصْحَٟبُ ٱلنَّارِ ۖ هُمْ فِيهَا خَـٰلِدُونَ ﴿٣٩﴾ \\
\textamh{40.\ እናንት የእስራኤል ልጆች! ለእናንተ የደረግኩትን አስታዉሱ፥ እናንተም ቃል ኪዳኔን አክብሩ እኔም ኪዳናችሁን እንዳሟላላችሁ (እንዳከብርላችሁ) ከኔ በቀር ማንንም አትፍሩ።  } &  يَـٰبَنِىٓ إِسْرَٟٓءِيلَ ٱذْكُرُوا۟ نِعْمَتِىَ ٱلَّتِىٓ أَنْعَمْتُ عَلَيْكُمْ وَأَوْفُوا۟ بِعَهْدِىٓ أُوفِ بِعَهْدِكُمْ وَإِيَّٟىَ فَٱرْهَبُونِ ﴿٤٠﴾\\ 
\textamh{41.\ ባወርደኩት (በዚህ ቁርአን) እመኑ፥ እናንተ ያለዉን (ተውራት፥ ወንጌል) የሚያረጋግጥላችሁ፤  ከካሀዲዎች የመጀመሪያ አትሁኑ፤ አያቴን  (ተውራት፥ ወንጌልን፥ ምልክቴን፥ ጥቅሶቼን) በትንሽ  ዋጋ አትቸርችሩ፤ ፍሩኝ እኔን ብቻ ፍሩ   } &  وَءَامِنُوا۟ بِمَآ أَنزَلْتُ مُصَدِّقًۭا لِّمَا مَعَكُمْ وَلَا تَكُونُوٓا۟ أَوَّلَ كَافِرٍۭ بِهِۦ ۖ وَلَا تَشْتَرُوا۟ بِـَٔايَـٰتِى ثَمَنًۭا قَلِيلًۭا وَإِيَّٟىَ فَٱتَّقُونِ ﴿٤١﴾\\
\textamh{42.\ ሀቁን (እዉነቱን) በሐሰት አታልብሱ እዉነቱንም አትደብቁ እናንተ እያወቃችሁ (ሙሐመድ(ሠአወሰ) የኣላህ መልክተኛ መሆኑን)   } &  وَلَا تَلْبِسُوا۟ ٱلْحَقَّ بِٱلْبَٟطِلِ وَتَكْتُمُوا۟ ٱلْحَقَّ وَأَنتُمْ تَعْلَمُونَ ﴿٤٢﴾\\
\textamh{43.\ ሳላት ቁሙ፥ ዘካት ክፈሉ፥ ኢርከ (ጎንበስ ብላችሁ ለኣላህ) አር-ራኪኡን (ስገዱ)   } &  وَأَقِيمُوا۟ ٱلصَّلَوٰةَ وَءَاتُوا۟ ٱلزَّكَوٰةَ وَٱرْكَعُوا۟ مَعَ ٱلرَّٟكِعِينَ ﴿٤٣﴾ ۞\\
\textamh{44.\ ሰዉን የጽድቅ ስራ እንዲሰሩ (ለኣላህ እንዲገዙ) ታዛላችሁ ራሳችሁ ማድረጉን ረስታችሁ፥ መጽሃፉን እያነበባችሁ? አቅል የላችሁም (አታስቡም) ወይ?   } &   أَتَأْمُرُونَ ٱلنَّاسَ بِٱلْبِرِّ وَتَنسَوْنَ أَنفُسَكُمْ وَأَنتُمْ تَتْلُونَ ٱلْكِتَـٰبَ ۚ أَفَلَا تَعْقِلُونَ ﴿٤٤﴾\\
\textamh{45.\ በትእግስትና በሳለት (ጸሎት) እርዳታ ፈልጉ፤ በእዉነት ከባድ (ፈተና -ከቢር) ነው ለአል-ኻሺሁኡን (እዉነተኛ የኣላህ  አማኞች) በስተቀር   } &   وَٱسْتَعِينُوا۟ بِٱلصَّبْرِ وَٱلصَّلَوٰةِ ۚ وَإِنَّهَا لَكَبِيرَةٌ إِلَّا عَلَى ٱلْخَـٰشِعِينَ ﴿٤٥﴾\\
\textamh{46.\ እነዚህ ናቸው አምላካቸዉን በእርግ- ጠኝነት እንደሚጋናኙ የሚያውቁ፤ ወደሱም ይመለሳሉ።   } &  ٱلَّذِينَ يَظُنُّونَ أَنَّهُم مُّلَٟقُوا۟ رَبِّهِمْ وَأَنَّهُمْ إِلَيْهِ رَٟجِعُونَ ﴿٤٦﴾\\
\textamh{47.\ እናንት የእስራኤል ልጆች! ለእናንተ የደረግኩትን አስታዉሱ፥ ከአላሚን አስበልጬ እንደመረጥኳችሁ   } &  يَـٰبَنِىٓ إِسْرَٟٓءِيلَ ٱذْكُرُوا۟ نِعْمَتِىَ ٱلَّتِىٓ أَنْعَمْتُ عَلَيْكُمْ وَأَنِّى فَضَّلْتُكُمْ عَلَى ٱلْعَٟلَمِينَ ﴿٤٧﴾\\
\textamh{48.\ አንድ ቀን ግን ፍሩ (የፍርድ ቀን) አንዱ ሌላው የማያወጣበት፥ ወይንም ምልድጃ የማይቀበልበት ወይንም ካሳ ክፍያ የማይቀበሉበት ወይንም የማይረዱበት   } &  وَٱتَّقُوا۟ يَوْمًۭا لَّا تَجْزِى نَفْسٌ عَن نَّفْسٍۢ شَيْـًۭٔا وَلَا يُقْبَلُ مِنْهَا شَفَٟعَةٌۭ وَلَا يُؤْخَذُ مِنْهَا عَدْلٌۭ وَلَا هُمْ يُنصَرُونَ ﴿٤٨﴾\\
\textamh{49.\ ከፈርኦን ሰዎች አወጣናችሁ፥ በከባድ  ቅጣት ሲቀጧችሁ፥ ልጆቻችሁን እየገደሉ ሴቶቻችሁን እያቆዩ፥ እዚያ ከአምላካችሁ  ከባድ ፈተና ነበር   } &  وَإِذْ نَجَّيْنَـٰكُم مِّنْ ءَالِ فِرْعَوْنَ يَسُومُونَكُمْ سُوٓءَ ٱلْعَذَابِ يُذَبِّحُونَ أَبْنَآءَكُمْ وَيَسْتَحْيُونَ نِسَآءَكُمْ ۚ وَفِى ذَٟلِكُم بَلَآءٌۭ مِّن رَّبِّكُمْ عَظِيمٌۭ ﴿٤٩﴾\\
\textamh{50.\ ባህሩን ከፍለን እናንተን አድነን የፊራኡን (የፈርኦንን) ሰዎች አይናችሁ እያየ  አሰመጥናቸው    } &  وَإِذْ فَرَقْنَا بِكُمُ ٱلْبَحْرَ فَأَنجَيْنَـٰكُمْ وَأَغْرَقْنَآ ءَالَ فِرْعَوْنَ وَأَنتُمْ تَنظُرُونَ ﴿٥٠﴾\\
\textamh{51.\ ለአረባ ለሊት ሙሳን (ሙሴን) ስናደርግለት  (ለብቻው)፥ (በሌለበት) ጥጃዉን  (እንደአምላክ) ለራሳችሁ አደረጋችሁ እናንተም ዛሊሙን(ጣኦት አምላኪ፥ ጥፋተኞች) ሆናችሁ።   } &  وَإِذْ وَٟعَدْنَا مُوسَىٰٓ أَرْبَعِينَ لَيْلَةًۭ ثُمَّ ٱتَّخَذْتُمُ ٱلْعِجْلَ مِنۢ بَعْدِهِۦ وَأَنتُمْ ظَٟلِمُونَ ﴿٥١﴾\\
\textamh{52.\ ከዚያም በኋላ ይቅር አለናችሁ እንድታመሰግኑ   } &  ثُمَّ عَفَوْنَا عَنكُم مِّنۢ بَعْدِ ذَٟلِكَ لَعَلَّكُمْ تَشْكُرُونَ ﴿٥٢﴾\\
\textamh{53.\ ለሙሳም መጽሃፍና መፍረጃ (እዉነቱን ከሐሰት) ሰጠነው በዚያ በትክክል መመራት እንድትችሉ።   } &  وَإِذْ ءَاتَيْنَا مُوسَى ٱلْكِتَـٰبَ وَٱلْفُرْقَانَ لَعَلَّكُمْ تَهْتَدُونَ ﴿٥٣﴾\\
\textamh{54.\ ሙሳም ወደ ሰዎቹ አለ: \rq\rq{}ሰዎቼ ሆይ!፥ በእዉነት ራሳችሁን በድላችኋል ጥጃዉን  በማምለክ። ወደ አምላክችሁ ንስሃ ግቡ፥ ራሳች- ሁን (ያጠፉትን) ግደሉ፥ ያ በአምላካችሁ ዘነድ ጥሩ ይሆንላችኋል\rq\rq{} (ኣላህም) ንስሀችሁን ተቀበለ። በእዉነት እሱ ብቻ ነው ንስሀ ተቀበይ፥ ከሁሉም በላይ  ምህረተኛው   } &  وَإِذْ قَالَ مُوسَىٰ لِقَوْمِهِۦ يَـٰقَوْمِ إِنَّكُمْ ظَلَمْتُمْ أَنفُسَكُم بِٱتِّخَاذِكُمُ ٱلْعِجْلَ فَتُوبُوٓا۟ إِلَىٰ بَارِئِكُمْ فَٱقْتُلُوٓا۟ أَنفُسَكُمْ ذَٟلِكُمْ خَيْرٌۭ لَّكُمْ عِندَ بَارِئِكُمْ فَتَابَ عَلَيْكُمْ ۚ إِنَّهُۥ هُوَ ٱلتَّوَّابُ ٱلرَّحِيمُ ﴿٥٤﴾\\
\textamh{55.\ እናንተም ሙሳን: \rq\rq{}ያኣ ሙሳ (ኦ ሙሳ)! ኣላህን ካላየን ምንም አናምንህም\rq\rq{} አላችሁ። ወዲያዉም መብረቅ መጥቶ አይናችሁ እያየ ያዛችሁ።   } &  وَإِذْ قُلْتُمْ يَـٰمُوسَىٰ لَن نُّؤْمِنَ لَكَ حَتَّىٰ نَرَى ٱللَّهَ جَهْرَةًۭ فَأَخَذَتْكُمُ ٱلصَّٟعِقَةُ وَأَنتُمْ تَنظُرُونَ ﴿٥٥﴾\\
\textamh{56.\ ከዚያም አስነሳናችሁ (ህይወት ሰጠናችሁ) ከሞታችሁ በኋላ፥ አመስጋኝ እንድትሆኑ   } &  ثُمَّ بَعَثْنَـٰكُم مِّنۢ بَعْدِ مَوْتِكُمْ لَعَلَّكُمْ تَشْكُرُونَ ﴿٥٦﴾\\
\textamh{57.\ በደመና ጋረድናችሁ፥ ከሰማይም መናና  ሰልዋ አወርድንላችሁ፤ \rq\rq{}ብሉ የሰጠናችሁን (ያወርደነዉን) ጥሩና የተፈቀደ (ሃላል) ምግብ\rq\rq{} (ግን አማጹ)። እኛን አልበደሉነም ነገር ግን ራሳቸዉን ነው የበደሉ።   } &  وَظَلَّلْنَا عَلَيْكُمُ ٱلْغَمَامَ وَأَنزَلْنَا عَلَيْكُمُ ٱلْمَنَّ وَٱلسَّلْوَىٰ ۖ كُلُوا۟ مِن طَيِّبَٟتِ مَا رَزَقْنَـٰكُمْ ۖ وَمَا ظَلَمُونَا وَلَٟكِن كَانُوٓا۟ أَنفُسَهُمْ يَظْلِمُونَ ﴿٥٧﴾\\
\textamh{58.\ አልን (ኣላህ) : \rq\rq{}እዚህ ከተማ ግቡ  (እየሩሳሌም) እና ብሉ እንደፈለጋችሁ በደስታ (ያማራችሁን)ከፈልገችሁበት ቦታ ግቡ በአክብሮት (በሱጀደ፥ በአክብሮት ጎንበስ ብላችሁ) እናም (እንዲህ) በሉ: \rq\rq{}ይቅር በለነ\rq\rq{} ሀጢያታችሁን ይቅር እንላችኋለን ጥሩ የሚ- ሰሩትን እንጨምርላቸዋለን።   } &  وَإِذْ قُلْنَا ٱدْخُلُوا۟ هَـٰذِهِ ٱلْقَرْيَةَ فَكُلُوا۟ مِنْهَا حَيْثُ شِئْتُمْ رَغَدًۭا وَٱدْخُلُوا۟ ٱلْبَابَ سُجَّدًۭا وَقُولُوا۟ حِطَّةٌۭ نَّغْفِرْ لَكُمْ خَطَٟيَـٰكُمْ ۚ وَسَنَزِيدُ ٱلْمُحْسِنِينَ ﴿٥٨﴾\\
\textamh{59.\ ነገር ግን መጥፎ ሰሪዎቹ የተነገራቸውን ቃል በሌላ ቀየሩት፤ ከነዚህ ዛሊሞች (መጥፎ ሰሪዎች) ላይ ሪጅዘን (ቅጣት) ከሰማይ አወርድንባቸው በኣላህ ትእዛዝ ላይ ስላመጹ   } &  فَبَدَّلَ ٱلَّذِينَ ظَلَمُوا۟ قَوْلًا غَيْرَ ٱلَّذِى قِيلَ لَهُمْ فَأَنزَلْنَا عَلَى ٱلَّذِينَ ظَلَمُوا۟ رِجْزًۭا مِّنَ ٱلسَّمَآءِ بِمَا كَانُوا۟ يَفْسُقُونَ ﴿٥٩﴾ ۞\\
\textamh{60.\ ሙሳ ለሰዎቹ ውሃ ሲጠይቅ፤ አልን (ኣላህ): \rq\rq{}አለቱን በበትርህ ምታው\rq\rq{}። ከዚያም አስራ ሁለት ምንጮች ፈሰሱ። ሁሉም (ነገድ) የየራሱን ውሃ መጠጫ ቦታ የዉቁ ነበር። \rq\rq{}ብሉ ጠጡ ኣላህ የሰጣችሁን፤ በደል አትስሩ መሬት (ምድር) ላይ እየበጠበጣችሁ።   } &  وَإِذِ ٱسْتَسْقَىٰ مُوسَىٰ لِقَوْمِهِۦ فَقُلْنَا ٱضْرِب بِّعَصَاكَ ٱلْحَجَرَ ۖ فَٱنفَجَرَتْ مِنْهُ ٱثْنَتَا عَشْرَةَ عَيْنًۭا ۖ قَدْ عَلِمَ كُلُّ أُنَاسٍۢ مَّشْرَبَهُمْ ۖ كُلُوا۟ وَٱشْرَبُوا۟ مِن رِّزْقِ ٱللَّهِ وَلَا تَعْثَوْا۟ فِى ٱلْأَرْضِ مُفْسِدِينَ ﴿٦٠﴾\\
\textamh{61.\ እናንተም (እንዲህ) አላችሁ: \rq\rq{}ያኣ ሙሳ! (ኦ! ሙሴ) አንድ አይነት  ምግብ ብቻ አልተቻለንም። ስለዚህ አምላክህን ምድር የሚያበቅለዉን ስጠን ብለህ ጠይቅልን፥ ባቄላዉን ኮከምበር(?)፥ ፉም (ነጭ ሽንኩርት ወይም ስንዴ)፥ ምስሩን ቀይ ሽንኩርቱን\rq\rq{}። አለ \rq\rq{}ጥሩ የሆነዉን  ከዚያ ባነሰ ትለዉጣላችሁ? ሂዱ ዉረዱ (ዉጡ ከዚህ) ወደ አንዱ ከተማና የፈለጋችሁትን ታገኛላችሁ\rq\rq{} ሀፍረሀትና ስቃይ ተከናነቡ፥ ራሳቸው ላይ የኣላህን ቁጣ አመጡ። ያም የሆነው የኣላህን አያት (ጥቅሶችን፥ ማስረጃዎችን፥ ምልክቶችን ተአምራቱን) እየካዱ ስለነበርና ነቢያቱን በሃሰት ሲገሉ ስለኖሩ ነው። ያም የሆነው ስለማይገዙና (ትእዛዝ ስለማያከብሩ) ከማይገባ በላይ ተላላፊዎች ስለነበሩ ነው።   } &  وَإِذْ قُلْتُمْ يَـٰمُوسَىٰ لَن نَّصْبِرَ عَلَىٰ طَعَامٍۢ وَٟحِدٍۢ فَٱدْعُ لَنَا رَبَّكَ يُخْرِجْ لَنَا مِمَّا تُنۢبِتُ ٱلْأَرْضُ مِنۢ بَقْلِهَا وَقِثَّآئِهَا وَفُومِهَا وَعَدَسِهَا وَبَصَلِهَا ۖ قَالَ أَتَسْتَبْدِلُونَ ٱلَّذِى هُوَ أَدْنَىٰ بِٱلَّذِى هُوَ خَيْرٌ ۚ ٱهْبِطُوا۟ مِصْرًۭا فَإِنَّ لَكُم مَّا سَأَلْتُمْ ۗ وَضُرِبَتْ عَلَيْهِمُ ٱلذِّلَّةُ وَٱلْمَسْكَنَةُ وَبَآءُو بِغَضَبٍۢ مِّنَ ٱللَّهِ ۗ ذَٟلِكَ بِأَنَّهُمْ كَانُوا۟ يَكْفُرُونَ بِـَٔايَـٰتِ ٱللَّهِ وَيَقْتُلُونَ ٱلنَّبِيِّۦنَ بِغَيْرِ ٱلْحَقِّ ۗ ذَٟلِكَ بِمَا عَصَوا۟ وَّكَانُوا۟ يَعْتَدُونَ ﴿٦١﴾\\
\textamh{62.\ በእውነት የሚያምኑትና አይሁዶች፥ ናሳራዎች  (ክርስቲያኖች)፥ ሳቢያኖች ማንኛዉም በኣላህና በመጨረሻዉ ቀን የሚያምን እና ጥሩ ስራ  የሚሰራ፥ እነሱ ክፍያቸዉን ከአምላካቸው ያገኛሉ፤ እነሱም ላይ ፍርሃት አይኖርም  አያዝኑምም።   } &   إِنَّ ٱلَّذِينَ ءَامَنُوا۟ وَٱلَّذِينَ هَادُوا۟ وَٱلنَّصَٟرَىٰ وَٱلصَّٟبِـِٔينَ مَنْ ءَامَنَ بِٱللَّهِ وَٱلْيَوْمِ ٱلْءَاخِرِ وَعَمِلَ صَٟلِحًۭا فَلَهُمْ أَجْرُهُمْ عِندَ رَبِّهِمْ وَلَا خَوْفٌ عَلَيْهِمْ وَلَا هُمْ يَحْزَنُونَ ﴿٦٢﴾\\
\textamh{63.\ (ኦ! የእስራእል ልጆች) ቃል ኪዳናችሁን ገብተን ተራራዉን ከናንተ በላይ አድርገን \rq\rq{}ይህን የሰጠናችሁን ጠበቅ አድርጋችሁያዙ፥ ዉስጡ ያለዉን አስታውሱ  በዚያም አል-ሙታቁን (ፈሪሃ-ኣላህ ያለው)  ትሆናላችሁ   } &  وَإِذْ أَخَذْنَا مِيثَٟقَكُمْ وَرَفَعْنَا فَوْقَكُمُ ٱلطُّورَ خُذُوا۟ مَآ ءَاتَيْنَـٰكُم بِقُوَّةٍۢ وَٱذْكُرُوا۟ مَا فِيهِ لَعَلَّكُمْ تَتَّقُونَ ﴿٦٣﴾\\
\textamh{64.\ ከዚያም (ራሳችሁ) ዘወር አላችሁ። የኣላህ ጸጋና ምህረት እናንተ ላይ ባይሆን ኑሮ ከከሳሪዎች መካከል ትሆኑ ነበር   } &  ثُمَّ تَوَلَّيْتُم مِّنۢ بَعْدِ ذَٟلِكَ ۖ فَلَوْلَا فَضْلُ ٱللَّهِ عَلَيْكُمْ وَرَحْمَتُهُۥ لَكُنتُم مِّنَ ٱلْخَـٰسِرِينَ ﴿٦٤﴾\\
\textamh{65.\ እናም ታዉቃላችሁ ከናንተ መካከል ሰንበትን የተላለፉትን፤ እኛም አልናችዉ \rq\rq{}ሁኑ ዝንጆሮዎች፥ የረከሰና የተጣለ\rq\rq{}   } &  وَلَقَدْ عَلِمْتُمُ ٱلَّذِينَ ٱعْتَدَوْا۟ مِنكُمْ فِى ٱلسَّبْتِ فَقُلْنَا لَهُمْ كُونُوا۟ قِرَدَةً خَـٰسِـِٔينَ ﴿٦٥﴾\\
\textamh{66.\ ይህንም ቅጣት ምሳሌ አደርገነው ለነሱም ከነሱም በኋላ ለመጡት ትዉልዶች እና  ለአል-ሙታቁን (ፈሪሃ-ኣላህ ላላቸው) ትምህርት።   } &  فَجَعَلْنَـٰهَا نَكَٟلًۭا لِّمَا بَيْنَ يَدَيْهَا وَمَا خَلْفَهَا وَمَوْعِظَةًۭ لِّلْمُتَّقِينَ ﴿٦٦﴾\\
\textamh{67.\ ሙሳም (አላቸው): \rq\rq{}በእዉነት፥ ኣላህ አንድ ላም ታርዱለት ዘንድ ያዛችኋል\rq\rq{}። እነሱም አሉ: \rq\rq{}ታላግጥብናለህ እንዴ?\rq\rq{}። እሱም አለ: \rq\rq{}በኣላህ እከለላለሁ ከጅሎች መካከል እንዳልሆን\rq\rq{}   } &  وَإِذْ قَالَ مُوسَىٰ لِقَوْمِهِۦٓ إِنَّ ٱللَّهَ يَأْمُرُكُمْ أَن تَذْبَحُوا۟ بَقَرَةًۭ ۖ قَالُوٓا۟ أَتَتَّخِذُنَا هُزُوًۭا ۖ قَالَ أَعُوذُ بِٱللَّهِ أَنْ أَكُونَ مِنَ ٱلْجَٟهِلِينَ ﴿٦٧﴾\\
\textamh{68.\ እነሱም አሉ: \rq\rq{}አምላክህን ጠይቅልን ምን እንደሆነ በትክክል እንዲገልጽልን\rq\rq{} እሱም አለ: \rq\rq{}(አምላክ) እንዲህ ይላል፥  በእዉነት፥ ያላረጀች ወይም  ትንሽም ያልሆነች፥ ነገር ግን በሁለቱ መካከል የሆነች። በሉ የታዘዛችሁትን አድርጉ።   } &  قَالُوا۟ ٱدْعُ لَنَا رَبَّكَ يُبَيِّن لَّنَا مَا هِىَ ۚ قَالَ إِنَّهُۥ يَقُولُ إِنَّهَا بَقَرَةٌۭ لَّا فَارِضٌۭ وَلَا بِكْرٌ عَوَانٌۢ بَيْنَ ذَٟلِكَ ۖ فَٱفْعَلُوا۟ مَا تُؤْمَرُونَ ﴿٦٨﴾\\
\textamh{69.\ እነሱም አሉ: \rq\rq{}አምላክህን ጠይቅልን ቀለሟ ምን እንደሆነ እንዲገልጽልን\rq\rq{} እሱም አለ: \rq\rq{}(አምላክ) እንዲህ ይላል፥ ቢጫ ላም፥ ቀለሟ ቦግ ያለ፥ ለሚያያት የሚስደስት\rq\rq{}   } &  قَالُوا۟ ٱدْعُ لَنَا رَبَّكَ يُبَيِّن لَّنَا مَا لَوْنُهَا ۚ قَالَ إِنَّهُۥ يَقُولُ إِنَّهَا بَقَرَةٌۭ صَفْرَآءُ فَاقِعٌۭ لَّوْنُهَا تَسُرُّ ٱلنَّٟظِرِينَ ﴿٦٩﴾\\
\textamh{70.\ እነሱም አሉ: \rq\rq{}አምላክህን ጠይቅልን ምን እንደሆነ በትክክል እንዲገልጽልን። ለእኛ ሁሉም ላሞች አንድ አይነት ናቸው፤ እናም በእርግጠኝነት፥ ኣላህ ከፈቀደ፥ እኛ እንመራለን\rq\rq{}    } &  قَالُوا۟ ٱدْعُ لَنَا رَبَّكَ يُبَيِّن لَّنَا مَا هِىَ إِنَّ ٱلْبَقَرَ تَشَٟبَهَ عَلَيْنَا وَإِنَّآ إِن شَآءَ ٱللَّهُ لَمُهْتَدُونَ ﴿٧٠﴾\\
\textamh{71.\ እሱም (ሙሳ) አለ: \rq\rq{}(አምላክ) እንዲህ ይላል፥ መሬት ለማረስ ወይንም ሜዳ ዉሃ ለማጠጣት ያልሰለጠነ፥ ጤነኛ ቀለሙም ቦግ ካለ (ደማቅ?) ከቢጫ ሌላ ያልሆነ። እነሱም አሉ: \rq\rq{}አሁን እዉነቱን አመጣህልን\rq\rq{}። እናም አረዱ ግን ላለማድረግ ተቃርበው ነበር።   } &  قَالَ إِنَّهُۥ يَقُولُ إِنَّهَا بَقَرَةٌۭ لَّا ذَلُولٌۭ تُثِيرُ ٱلْأَرْضَ وَلَا تَسْقِى ٱلْحَرْثَ مُسَلَّمَةٌۭ لَّا شِيَةَ فِيهَا ۚ قَالُوا۟ ٱلْـَٟٔنَ جِئْتَ بِٱلْحَقِّ ۚ فَذَبَحُوهَا وَمَا كَادُوا۟ يَفْعَلُونَ ﴿٧١﴾\\
\textamh{72.\ እናም ሰው ገደላችሁ እርስበራሰችሁ ማን እንዳደረገው ስትወነጃጀሉ፤ ነገር ግን ኣላህ አወጣው ስትደበቁት የነበረዉን   } &  وَإِذْ قَتَلْتُمْ نَفْسًۭا فَٱدَّٟرَْٟٔتُمْ فِيهَا ۖ وَٱللَّهُ مُخْرِجٌۭ مَّا كُنتُمْ تَكْتُمُونَ ﴿٧٢﴾\\
\textamh{73.\ እናም አልነ: \rq\rq{}(የሞተዉን ሰው በላሟ) በቁራጭ ምቱት\rq\rq{}። ስለዚህ ኣላህ የሞተዉን ያስነሳል እና አያቱን (ማረጋገጫ፥ ጥቅሶች) ያሳያል በዚያ እንዲገባችሁ።   } &   فَقُلْنَا ٱضْرِبُوهُ بِبَعْضِهَا ۚ كَذَٟلِكَ يُحْىِ ٱللَّهُ ٱلْمَوْتَىٰ وَيُرِيكُمْ ءَايَـٰتِهِۦ لَعَلَّكُمْ تَعْقِلُونَ ﴿٧٣﴾\\
\textamh{74.\ ከዚያም በኋላ ልባችሁ ደንደነ፥ አለት ሆኑ ከዚያም የከፋ ድንዳኔ። ከአለቶች እንኳን ውሃ ያሚወጣባቸው አሉ፥ አንዳዶችም ሲሰነጠቁ ዉሃ ይፈሳል፥ ከነሱም መካከል በኣላህ ፍርሃት የሚወድቁ አሉ። እና ኣላህ የምታደርጉትን የማያዉቅ አይደለም።   } &   ثُمَّ قَسَتْ قُلُوبُكُم مِّنۢ بَعْدِ ذَٟلِكَ فَهِىَ كَٱلْحِجَارَةِ أَوْ أَشَدُّ قَسْوَةًۭ ۚ وَإِنَّ مِنَ ٱلْحِجَارَةِ لَمَا يَتَفَجَّرُ مِنْهُ ٱلْأَنْهَـٰرُ ۚ وَإِنَّ مِنْهَا لَمَا يَشَّقَّقُ فَيَخْرُجُ مِنْهُ ٱلْمَآءُ ۚ وَإِنَّ مِنْهَا لَمَا يَهْبِطُ مِنْ خَشْيَةِ ٱللَّهِ ۗ وَمَا ٱللَّهُ بِغَٟفِلٍ عَمَّا تَعْمَلُونَ ﴿٧٤﴾ ۞\\
\textamh{75.\ እናንተ (አማኞች) በሃይማኖታችሁ ያምናሉ (ይሁዶችን) ብላችሁ ታስባላችሁ፥ የኣላህን ቃል (ተውራት(ቶራህ)) ሲሰሙ ኑረው ነገር ግን በራሳቸው እያወቁ ከገባቸው በኋላ እየቀይሩት አልነበር።    } &   أَفَتَطْمَعُونَ أَن يُؤْمِنُوا۟ لَكُمْ وَقَدْ كَانَ فَرِيقٌۭ مِّنْهُمْ يَسْمَعُونَ كَلَٟمَ ٱللَّهِ ثُمَّ يُحَرِّفُونَهُۥ مِنۢ بَعْدِ مَا عَقَلُوهُ وَهُمْ يَعْلَمُونَ ﴿٧٥﴾\\
\textamh{76.\ አማኞችን ሲያገኙ (ይሁዶች) \rq\rq{}እናምናለን\rq\rq{} ይላሉ ብቻቸውን እርስበርስ ሲገናኙ \rq\rq{}እናንተ (ይሁዶች) ለነሱ (ለሙስሊሞች) ኣላህ የገለጸላችሁን (ይሁዶችን፥ ስለነብዩ ሙሐመድ (ሠአወሰ) ባህሪይ ተውራት (ቶራህ) ዉስጥ የተጻፈ ገለጻ) ትነገሯቸዋላችሁን\rq\rq{} እናነተ (ይሁዶች) አእምሮ የላችሁም ወይ?   } &  وَإِذَا لَقُوا۟ ٱلَّذِينَ ءَامَنُوا۟ قَالُوٓا۟ ءَامَنَّا وَإِذَا خَلَا بَعْضُهُمْ إِلَىٰ بَعْضٍۢ قَالُوٓا۟ أَتُحَدِّثُونَهُم بِمَا فَتَحَ ٱللَّهُ عَلَيْكُمْ لِيُحَآجُّوكُم بِهِۦ عِندَ رَبِّكُمْ ۚ أَفَلَا تَعْقِلُونَ ﴿٧٦﴾\\
\textamh{77.\ ኣላህ የሚገልጹትንና የሚደብቁትን እንደሚያውቅ አያውቁምን?   } &   أَوَلَا يَعْلَمُونَ أَنَّ ٱللَّهَ يَعْلَمُ مَا يُسِرُّونَ وَمَا يُعْلِنُونَ ﴿٧٧﴾\\
\textamh{78.\ ከነሱ መካከል ደግሞ ያልተማሩ (ፊደል ያልቆጠሩ) አሉ፥ መጽሐፉን የማይውቁ፥ ሀሰት የሆነ ምኞትን ያምናሉ፤ ሌላ ሳይሆን የሚያደርጉት መገመት ብቻ።   } &  وَمِنْهُمْ أُمِّيُّونَ لَا يَعْلَمُونَ ٱلْكِتَـٰبَ إِلَّآ أَمَانِىَّ وَإِنْ هُمْ إِلَّا يَظُنُّونَ ﴿٧٨﴾\\
\textamh{79.\ ወዮለቸው በራሳቸው እጅ መጽሐፉን ጽፈው ከዚያም \rq\rq{}ይሄ ከኣላህ ነው\rq\rq{} የሚሉ በትንሽ ዋጋ ለመቸርቸር! ወዮ እጃቸው ለጻፈው ነገር፥ ወዮ በዚያም ለሚያገኙት፤    } &  فَوَيْلٌۭ لِّلَّذِينَ يَكْتُبُونَ ٱلْكِتَـٰبَ بِأَيْدِيهِمْ ثُمَّ يَقُولُونَ هَـٰذَا مِنْ عِندِ ٱللَّهِ لِيَشْتَرُوا۟ بِهِۦ ثَمَنًۭا قَلِيلًۭا ۖ فَوَيْلٌۭ لَّهُم مِّمَّا كَتَبَتْ أَيْدِيهِمْ وَوَيْلٌۭ لَّهُم مِّمَّا يَكْسِبُونَ ﴿٧٩﴾\\
\textamh{80.\ እናም ይላሉ (ይሁዶች): \rq\rq{}እሳቱ (ጀሀነም) ከተወሰኑ ቀናት በቀር አይነካንም\rq\rq{}። (እንዲህ) በል (ኦ ሙሐመድ (ሠአወሰ): \rq\rq{}ከኣላህ ዉል አላችሁ ወይ፥ ኣላህ ዉሉን እንዳይሰብር? ወይስ ስለኣላህ የማታዉቁትን ትላላችሁ?\rq\rq{}   } &  وَقَالُوا۟ لَن تَمَسَّنَا ٱلنَّارُ إِلَّآ أَيَّامًۭا مَّعْدُودَةًۭ ۚ قُلْ أَتَّخَذْتُمْ عِندَ ٱللَّهِ عَهْدًۭا فَلَن يُخْلِفَ ٱللَّهُ عَهْدَهُۥٓ ۖ أَمْ تَقُولُونَ عَلَى ٱللَّهِ مَا لَا تَعْلَمُونَ ﴿٨٠﴾\\
\textamh{81.\ አዎ! ማንም መጥፎ ስራውን ያገኘና ሀጢያቱ የከበበዉ፥ እነሱ የእሳቱ ነዋሪዎች ናቸው፤ እዛም ለዘላለም ይኖራሉ   } &  بَلَىٰ مَن كَسَبَ سَيِّئَةًۭ وَأَحَٟطَتْ بِهِۦ خَطِيٓـَٔتُهُۥ فَأُو۟لَٟٓئِكَ أَصْحَٟبُ ٱلنَّارِ ۖ هُمْ فِيهَا خَـٰلِدُونَ ﴿٨١﴾\\
\textamh{82.\ የሚያምኑና ጥሩ ስራ የሚሰሩ፥ እነሱ የገነት ነዋሪዎች ናቸው፥ እዛም ለዘላልም ይኖሩበታል   } &  وَٱلَّذِينَ ءَامَنُوا۟ وَعَمِلُوا۟ ٱلصَّٟلِحَٟتِ أُو۟لَٟٓئِكَ أَصْحَٟبُ ٱلْجَنَّةِ ۖ هُمْ فِيهَا خَـٰلِدُونَ ﴿٨٢﴾\\
\textamh{83.\ ከእስራእል ልጆች ጋር ቃል ኪዳን ስንገባ: ከኣላህ በቀር ማንንም አታምልኩ፥ ለወላጆቻችሁ ታዛዥና (አሳቢ) ጥሩ ሰሪ ሁኑ፥ ለዘመዶች፥ ለወላጅ አልባዎች ለማሳኪን (ለድሆች) እና ጥሩ የሆነ ለሰዎች ተናገሩ፥ ሳላት ቁሞ፥ ዘካት ክፈሉ። ከዚያም ወደኋለ ሸተት አላችሁ ትንሽ ቁጥር ካላቸው በቀር፥ እናንተም ወደ ኋለ ባዮች (ዘወር ባዮች) ናችሁ።   } &  وَإِذْ أَخَذْنَا مِيثَٟقَ بَنِىٓ إِسْرَٟٓءِيلَ لَا تَعْبُدُونَ إِلَّا ٱللَّهَ وَبِٱلْوَٟلِدَيْنِ إِحْسَانًۭا وَذِى ٱلْقُرْبَىٰ وَٱلْيَتَـٰمَىٰ وَٱلْمَسَٟكِينِ وَقُولُوا۟ لِلنَّاسِ حُسْنًۭا وَأَقِيمُوا۟ ٱلصَّلَوٰةَ وَءَاتُوا۟ ٱلزَّكَوٰةَ ثُمَّ تَوَلَّيْتُمْ إِلَّا قَلِيلًۭا مِّنكُمْ وَأَنتُم مُّعْرِضُونَ ﴿٨٣﴾\\
\textamh{84.\ ከናንተ ጋር ቃል ኪዳን ስንገባ: የራሳችሁን ሰዎች ደም አታፍስሱ፥ ደግሞም ከመኖሪያቸው አታስወጧቸው። ከዚያም ዉሉን ወሰዳችሁ (ተቀበላችሁ) ራሳችሁ እየመሰከራችሁ።   } &  وَإِذْ أَخَذْنَا مِيثَٟقَكُمْ لَا تَسْفِكُونَ دِمَآءَكُمْ وَلَا تُخْرِجُونَ أَنفُسَكُم مِّن دِيَـٰرِكُمْ ثُمَّ أَقْرَرْتُمْ وَأَنتُمْ تَشْهَدُونَ ﴿٨٤﴾\\
\textamh{85.\ ከዚያም በኋላ እናንተው ናችሁ እርስበርስ የተገዳደላችሁ፥ ከናንተ መካከል ያሉትንም ከቤታቸው ያስወጣችሁ፥ (ጠላቶቻቸዉን) የረዳችሁ፥ በሀጢያትና በመተላለፍ። ወደ እናንተ ምርኮኞች ሁነው ሲመጡ፥ ዋጋ (የማስፈቻ) ትከፍላላችሁ፥ ነገር ግን እነሱን ማስወጣት ክልክል ነበር። ስለዚህ አንዱን የመጽሃፍ ክፍል እያመናችሁ ሌላኛዉን ትክዳላችሁ? ምንድነው ታዲያ እንዲህ ለሚያደርግ ክፍያው በዚህ አለም ዉርዴት፥ የትንሳኤ ቀን ደግሞ ክፉ የሆነ ስቃይ ካለበት መመደብ። እና ኣላህ የምታደርጉትን የማያዉቅ አይደለም።   } &   ثُمَّ أَنتُمْ هَـٰٓؤُلَآءِ تَقْتُلُونَ أَنفُسَكُمْ وَتُخْرِجُونَ فَرِيقًۭا مِّنكُم مِّن دِيَـٰرِهِمْ تَظَٟهَرُونَ عَلَيْهِم بِٱلْإِثْمِ وَٱلْعُدْوَٟنِ وَإِن يَأْتُوكُمْ أُسَٟرَىٰ تُفَٟدُوهُمْ وَهُوَ مُحَرَّمٌ عَلَيْكُمْ إِخْرَاجُهُمْ ۚ أَفَتُؤْمِنُونَ بِبَعْضِ ٱلْكِتَـٰبِ وَتَكْفُرُونَ بِبَعْضٍۢ ۚ فَمَا جَزَآءُ مَن يَفْعَلُ ذَٟلِكَ مِنكُمْ إِلَّا خِزْىٌۭ فِى ٱلْحَيَوٰةِ ٱلدُّنْيَا ۖ وَيَوْمَ ٱلْقِيَـٰمَةِ يُرَدُّونَ إِلَىٰٓ أَشَدِّ ٱلْعَذَابِ ۗ وَمَا ٱللَّهُ بِغَٟفِلٍ عَمَّا تَعْمَلُونَ ﴿٨٥﴾\\
\textamh{86.\ እነዚህ ናቸው የዚህን አለም በሰማያዊ (በሚቀጥለው አለም) (በአኪራ) የነገዱ። ቅጣቸው አይቀለልም ደግሞም እርዳታ አይኖራቸውም፤   } &  أُو۟لَٟٓئِكَ ٱلَّذِينَ ٱشْتَرَوُا۟ ٱلْحَيَوٰةَ ٱلدُّنْيَا بِٱلْءَاخِرَةِ ۖ فَلَا يُخَفَّفُ عَنْهُمُ ٱلْعَذَابُ وَلَا هُمْ يُنصَرُونَ ﴿٨٦﴾\\
\textamh{87.\ ለሙሳ (ሙሴ) መጽሃፍ ሰጠነው እናም ተከታታይ መልእክተኞች አስከተልነ። ለኢሳ (የሱስ)፥ የማሪያም ልጅ፥ ግልጽ ምልክት ሰጠነው፥ በመንፈስ ቅዱስ (ጂብሪል (ገብርኤል)) ረዳነው። እናንተ የማትፈልጉት መልእክተኛ ሲመጣላችሁ ኮራችሁ? አንዳንዶችን ካዳችሁ፥ አንዳዶችንም ገደላችሁ።   } &  وَلَقَدْ ءَاتَيْنَا مُوسَى ٱلْكِتَـٰبَ وَقَفَّيْنَا مِنۢ بَعْدِهِۦ بِٱلرُّسُلِ ۖ وَءَاتَيْنَا عِيسَى ٱبْنَ مَرْيَمَ ٱلْبَيِّنَـٰتِ وَأَيَّدْنَـٰهُ بِرُوحِ ٱلْقُدُسِ ۗ أَفَكُلَّمَا جَآءَكُمْ رَسُولٌۢ بِمَا لَا تَهْوَىٰٓ أَنفُسُكُمُ ٱسْتَكْبَرْتُمْ فَفَرِيقًۭا كَذَّبْتُمْ وَفَرِيقًۭا تَقْتُلُونَ ﴿٨٧﴾\\
\textamh{88.\ እነሱም አሉ (ሰዎች) \rq\rq{}ልባችን የታሸገ (የኣላህን ቀል ከማወቅ) ነው።\rq\rq{} አይደለም፥ ኣላህ ስለክህደታቸው ረግሞኣቸዋል፥ ከትንሽ በታች ነው የሚያምኑ፤   } &  وَقَالُوا۟ قُلُوبُنَا غُلْفٌۢ ۚ بَل لَّعَنَهُمُ ٱللَّهُ بِكُفْرِهِمْ فَقَلِيلًۭا مَّا يُؤْمِنُونَ ﴿٨٨﴾\\
\textamh{89.\ ከኣላህ መጽሐፍ (ይሄ ቁርአን) ሲመጣላቸው ከነሱ ያለዉን የሚያረጋግጥ (ተውራት፥ ወንጌል)፥ ምንም እንኳ በፊት ኣላህን ቢጠይቁም (ሙሐመድ (ሠአወሰ) እንዲመጣ) ከሃዲዎችን (የማያምኑትን) ለማሸነፍ፥ ከዚያ የሚያዉቁት ነገር ወደነሱ ሲመጣ፥ ካዱ። ስለዚህ የኣላህ እርግማን ከከሀዲዎች ላይ ይሁን።   } &  وَلَمَّا جَآءَهُمْ كِتَـٰبٌۭ مِّنْ عِندِ ٱللَّهِ مُصَدِّقٌۭ لِّمَا مَعَهُمْ وَكَانُوا۟ مِن قَبْلُ يَسْتَفْتِحُونَ عَلَى ٱلَّذِينَ كَفَرُوا۟ فَلَمَّا جَآءَهُم مَّا عَرَفُوا۟ كَفَرُوا۟ بِهِۦ ۚ فَلَعْنَةُ ٱللَّهِ عَلَى ٱلْكَٟفِرِينَ ﴿٨٩﴾\\
\textamh{90.\ እንዴት ለከፋ ነገር ነው ራሳቸዉን የሸጡ፥ ኣላህ በገለጸው (በዚህ ቁርአን) የማያምኑ፥ ኣላህ በፈለገው ባሪያው ጸጋዉን መገልጹ እየቆጫቸው። ስለዚህ ራሳቸው ላይ ከማአት ላይ ማአት አምጥተዋል። ለማየምኑት የዉርዴት ቅጣት (ስቃይ) ይጠብቃቸዋል።   } &   بِئْسَمَا ٱشْتَرَوْا۟ بِهِۦٓ أَنفُسَهُمْ أَن يَكْفُرُوا۟ بِمَآ أَنزَلَ ٱللَّهُ بَغْيًا أَن يُنَزِّلَ ٱللَّهُ مِن فَضْلِهِۦ عَلَىٰ مَن يَشَآءُ مِنْ عِبَادِهِۦ ۖ فَبَآءُو بِغَضَبٍ عَلَىٰ غَضَبٍۢ ۚ وَلِلْكَٟفِرِينَ عَذَابٌۭ مُّهِينٌۭ ﴿٩٠﴾\\
\textamh{91.\ \rq\rq{}ኣላህ በአወረደው እመኑ\rq\rq{} ሲበሉ (ለይሁዶች)፥ (እንዲህ) ይላሉ: \rq\rq{}ለኛ በወረደው ነው የምናምን\rq\rq{}። ከዚያ በኋላ በመጣው አያምኑም፤ እነሱ ጋር ያለዉን የሚያረጋግጥ። (እንዲህ) በል (ኦ! ሙሐመድ (ሠአወሰ): \rq\rq{}ለምን ታዲያ  የኣላህን (በፊት የመጡ) ነቢያት   ገደላችሁ፥ እንዴው በእዉነት አማኞች ከሆናችሁ?\rq\rq{}    } &   وَإِذَا قِيلَ لَهُمْ ءَامِنُوا۟ بِمَآ أَنزَلَ ٱللَّهُ قَالُوا۟ نُؤْمِنُ بِمَآ أُنزِلَ عَلَيْنَا وَيَكْفُرُونَ بِمَا وَرَآءَهُۥ وَهُوَ ٱلْحَقُّ مُصَدِّقًۭا لِّمَا مَعَهُمْ ۗ قُلْ فَلِمَ تَقْتُلُونَ أَنۢبِيَآءَ ٱللَّهِ مِن قَبْلُ إِن كُنتُم مُّؤْمِنِينَ ﴿٩١﴾ ۞\\
\textamh{92.\ በእዉነት ሙሳ (ሙሴ) ግልጽ የሆነ መስረጃ ይዞ መጥቷል፥ ነገር ግን እሱ ሲሄድ እናንተ ጥጃዉን አመለካችሁ እናንተም ዛሊሙን(ጣኦት አምላኪ፥ ጥፋተኞች) ሆናችሁ።   } &  وَلَقَدْ جَآءَكُم مُّوسَىٰ بِٱلْبَيِّنَـٰتِ ثُمَّ ٱتَّخَذْتُمُ ٱلْعِجْلَ مِنۢ بَعْدِهِۦ وَأَنتُمْ ظَٟلِمُونَ ﴿٩٢﴾\\
\textamh{93.\ ቃል ኪዳናችሁን ገብተን ተራራዉን ከናንተ በላይ አድርገን \rq\rq{}ይህን የሰጠናችሁን ጠበቅ አድርጋችሁ ያዙ፥እና ስሙ (ቃላችን)። እነሱም አሉ: \rq\rq{}ሰምተናል እና አንተገብርም\rq\rq{}። ልባቸዉም ወደጥጃዉ (ማምለክ) ተመሰጠ ስለክህደታቸው። (እንዲህ) በል: \rq\rq{}የከፋ ነው በእዉነት እምነታችሁ የሚያዝ አማኞች ከሆናችሁ\rq\rq{}።   } &  وَإِذْ أَخَذْنَا مِيثَٟقَكُمْ وَرَفَعْنَا فَوْقَكُمُ ٱلطُّورَ خُذُوا۟ مَآ ءَاتَيْنَـٰكُم بِقُوَّةٍۢ وَٱسْمَعُوا۟ ۖ قَالُوا۟ سَمِعْنَا وَعَصَيْنَا وَأُشْرِبُوا۟ فِى قُلُوبِهِمُ ٱلْعِجْلَ بِكُفْرِهِمْ ۚ قُلْ بِئْسَمَا يَأْمُرُكُم بِهِۦٓ إِيمَـٰنُكُمْ إِن كُنتُم مُّؤْمِنِينَ ﴿٩٣﴾\\
\textamh{94.\ (እንዲህ) በላቸው: \rq\rq{}የሰማይዊ ቤት ከኣላህ ጋር ለእናንተ ብቻ ከሆነና ለሌሎች ሰዎችም ካልሆነ፤ ሞት ተመኙ እዉነተኛ ከሆናችሁ\rq\rq{}   } &   قُلْ إِن كَانَتْ لَكُمُ ٱلدَّارُ ٱلْءَاخِرَةُ عِندَ ٱللَّهِ خَالِصَةًۭ مِّن دُونِ ٱلنَّاسِ فَتَمَنَّوُا۟ ٱلْمَوْتَ إِن كُنتُمْ صَٟدِقِينَ ﴿٩٤﴾ \\
\textamh{95.\ ነገር ግን አይመኙም እጃቸው ከፊታቸው በአደረገው (ስራቸው)። ኣላህ ሁሉን-ተገንዛቢ ነው የዛሊሙን (ጣኦት አምላኪ፥ ጥፋተኞች)   } &  وَلَن يَتَمَنَّوْهُ أَبَدًۢا بِمَا قَدَّمَتْ أَيْدِيهِمْ ۗ وَٱللَّهُ عَلِيمٌۢ بِٱلظَّٟلِمِينَ ﴿٩٥﴾\\
\textamh{96.\ በእዉነት ደግሞ፥ ለህይወት (ይሁዶች) ጓጊዎች (ስስታሞች) ናቸው እንዲያውም ከሙሽሪኮች(ብዝሃት አማልክት አምላኪዎች) የበለጠ። ሁላቸዉም ቢሆን አንድ ሺ አመት ቢኖሩ ይመኛሉ። ያ ህይወት ቢሰጠው ከትንሿም ቅጣት አያድነዉም። ኣላህ የሚሰሩትን ሁሉ ያያል   } &  وَلَتَجِدَنَّهُمْ أَحْرَصَ ٱلنَّاسِ عَلَىٰ حَيَوٰةٍۢ وَمِنَ ٱلَّذِينَ أَشْرَكُوا۟ ۚ يَوَدُّ أَحَدُهُمْ لَوْ يُعَمَّرُ أَلْفَ سَنَةٍۢ وَمَا هُوَ بِمُزَحْزِحِهِۦ مِنَ ٱلْعَذَابِ أَن يُعَمَّرَ ۗ وَٱللَّهُ بَصِيرٌۢ بِمَا يَعْمَلُونَ ﴿٩٦﴾\\
\textamh{97.\ (እንዲህ) በል (ኦ ሙሐመድ(ሠአወሰ)): \rq\rq{}ማንም የጂብሪል (ገብርኤል) ጠላት ቢሆን (በንዴት ይሙት)፥ በእዉነት ከልብህ ላይ (ይሄን ቁርአን) በኣላህ ፈቃድ አድሮጎታል ከሱ በፊት የነበረዉን (ተውራት፥ ወንጌል) የሚያረጋግጥ እና ምሬት (መመሪያ)ና ብስሪያ (ደስታ) ለአማኞች    } &  قُلْ مَن كَانَ عَدُوًّۭا لِّجِبْرِيلَ فَإِنَّهُۥ نَزَّلَهُۥ عَلَىٰ قَلْبِكَ بِإِذْنِ ٱللَّهِ مُصَدِّقًۭا لِّمَا بَيْنَ يَدَيْهِ وَهُدًۭى وَبُشْرَىٰ لِلْمُؤْمِنِينَ ﴿٩٧﴾\\
\textamh{98.\ \rq\rq{}ማንም የኣላህ ጠላት፥ የመላኢክት፥ የመልክእክተኞቹ፥ የጅብሪል፥ የሚካእል ጠላት ቢሆን፥ ኣላህ የካሀዲዎች ጠላት ነው\rq\rq{}   } &  مَن كَانَ عَدُوًّۭا لِّلَّهِ وَمَلَٟٓئِكَتِهِۦ وَرُسُلِهِۦ وَجِبْرِيلَ وَمِيكَىٰلَ فَإِنَّ ٱللَّهَ عَدُوٌّۭ لِّلْكَٟفِرِينَ ﴿٩٨﴾\\
\textamh{99.\ እንዲህ በጣም ግልጽ የሆነ አያት አዉርደንልሀል እና ማንም አይክድም ከፈሲቁን (በኣላህ ትእዛዝ ከሚያምጹ) በቀር    } &  وَلَقَدْ أَنزَلْنَآ إِلَيْكَ ءَايَـٰتٍۭ بَيِّنَـٰتٍۢ ۖ وَمَا يَكْفُرُ بِهَآ إِلَّا ٱلْفَٟسِقُونَ ﴿٩٩﴾\\
\textamh{100.\ እንዲህ አይደለም ሁሌ ቃል ኪዳን ሲገቡ፥ ግማሾቹ (ኪዳኑን) በጎን አይወረዉሩትም? የለም! እዉነቱ ብዙዎቹ አያምኑም።    } &  أَوَكُلَّمَا عَٟهَدُوا۟ عَهْدًۭا نَّبَذَهُۥ فَرِيقٌۭ مِّنْهُم ۚ بَلْ أَكْثَرُهُمْ لَا يُؤْمِنُونَ ﴿١٠٠﴾\\
\textamh{101.\ መልእክተኛ (ሙሐመድ(ሠአወሰ)) ከኣላህ ሲመጣላቸው ከነሱ ያለዉን የሚረጋግጥ፥ መጽሐፍ ከተሰጣቸው ዉስጥ የኣላህን መጽሃፍ በጀርባቸው ይወረውሩታል ልክ እንደማያውቁ    } &   وَلَمَّا جَآءَهُمْ رَسُولٌۭ مِّنْ عِندِ ٱللَّهِ مُصَدِّقٌۭ لِّمَا مَعَهُمْ نَبَذَ فَرِيقٌۭ مِّنَ ٱلَّذِينَ أُوتُوا۟ ٱلْكِتَـٰبَ كِتَـٰبَ ٱللَّهِ وَرَآءَ ظُهُورِهِمْ كَأَنَّهُمْ لَا يَعْلَمُونَ ﴿١٠١﴾\\
\textamh{102.\ እናም ሻያጢን (ሰይጣኖች) (በሃሰት) በሱሌይማን (ሰለሞን) ጊዜ ያወጡትን ይከተላሉ። ሱሌይማን አልካደም፥ ነገር የካዱት ሰይጣኖች ነበሩ፥ ሰዉን አስማትና (ድግምት) እንዲያ አይነት ነገሮችን ያስተማሩ (በሁለቱ) መላኢክት፥ ሀሩትና ማሩት፥ በባቢይሎን የወረደዉን ነገር፤ ነገር ግን ሁለቱ (መላኢክት) እዲህ ሳይሉ ለማንም አላስተማሩም: \rq\rq{}እኛ ለፈተና ብቻ ነን፥ ስለዚህ አትካዱ (አስማት ከኛ በመማር)\rq\rq{}። ከነዚህ (መላኢክት) ሰዎች ወንድና ሚስቱን የሚያፋቱበትን (የሚያጠሉበትን) መንገድ ተማሩ፥ ነገር ግን ማንንም ከኣላህ ፈቃድ ዉጭ መጉዳት አይችሉም። የሚጎዳቸዉን እንጂ የሚያተርፍ ነገር አልተማሩም። ቢያዉቁ ኑሩ፥ ይህንን የገዛ (አስማት)፥ በሰማያዊ ህይወት ድርሻ የለዉም። እንዴት ለከፋ ነገር ራሳቸዉን የሸጡት፥ ቢያዉቁ።   } &   وَٱتَّبَعُوا۟ مَا تَتْلُوا۟ ٱلشَّيَـٰطِينُ عَلَىٰ مُلْكِ سُلَيْمَـٰنَ ۖ وَمَا كَفَرَ سُلَيْمَـٰنُ وَلَٟكِنَّ ٱلشَّيَـٰطِينَ كَفَرُوا۟ يُعَلِّمُونَ ٱلنَّاسَ ٱلسِّحْرَ وَمَآ أُنزِلَ عَلَى ٱلْمَلَكَيْنِ بِبَابِلَ هَـٰرُوتَ وَمَـٰرُوتَ ۚ وَمَا يُعَلِّمَانِ مِنْ أَحَدٍ حَتَّىٰ يَقُولَآ إِنَّمَا نَحْنُ فِتْنَةٌۭ فَلَا تَكْفُرْ ۖ فَيَتَعَلَّمُونَ مِنْهُمَا مَا يُفَرِّقُونَ بِهِۦ بَيْنَ ٱلْمَرْءِ وَزَوْجِهِۦ ۚ وَمَا هُم بِضَآرِّينَ بِهِۦ مِنْ أَحَدٍ إِلَّا بِإِذْنِ ٱللَّهِ ۚ وَيَتَعَلَّمُونَ مَا يَضُرُّهُمْ وَلَا يَنفَعُهُمْ ۚ وَلَقَدْ عَلِمُوا۟ لَمَنِ ٱشْتَرَىٰهُ مَا لَهُۥ فِى ٱلْءَاخِرَةِ مِنْ خَلَٟقٍۢ ۚ وَلَبِئْسَ مَا شَرَوْا۟ بِهِۦٓ أَنفُسَهُمْ ۚ لَوْ كَانُوا۟ يَعْلَمُونَ ﴿١٠٢﴾\\
\textamh{103.\ ቢያይምኑ፥ ራሳቸዉን ከመጥፎ ነገር ቢጠብቁና ለኣላህ ሃላፊነተቸዉን ቢያክብሩ፥ ብዙ እጥፍ ይሆን ነበር የአምላካቸው ክፍያ፥ ቢያዉቁት!   } &  وَلَوْ أَنَّهُمْ ءَامَنُوا۟ وَٱتَّقَوْا۟ لَمَثُوبَةٌۭ مِّنْ عِندِ ٱللَّهِ خَيْرٌۭ ۖ لَّوْ كَانُوا۟ يَعْلَمُونَ ﴿١٠٣﴾\\
\textamh{104.\ ኦ እናንት አማኞች፥ (ለመልክእክተኛው (ሠአወሰ)) ራይነ አትበሉ ነገር ግን ኡንዙርነ (እንዲገባን አድርግ) በሉ እና ስሙ። ለማያምኑት (ለከሀዲዎች) ታላቅ ቅጣት አለ።   } &  يَـٰٓأَيُّهَا ٱلَّذِينَ ءَامَنُوا۟ لَا تَقُولُوا۟ رَٟعِنَا وَقُولُوا۟ ٱنظُرْنَا وَٱسْمَعُوا۟ ۗ وَلِلْكَٟفِرِينَ عَذَابٌ أَلِيمٌۭ ﴿١٠٤﴾\\
\textamh{105.\ ከመጽሐፉ ባለቤቶች (ይሁዶችና ክርስቲያኖች) ወይም ከሙሽሪኮች(ኣላህ አንድ መሆኑን የሚክዱ፥ ጠኦት አምላኪዎች፥ ፓጋኖች፥...) አንድ ጥሩ ነገር ከአምላካችሁ እንዲወርድላችሁ አይፈልጉም። ነገር ግን ኣላህ የፈለገዉን ለምህረቱ ይመርጣል። ኣላህ የታላቅ ጸጋ ባለቤት ነው።   } &   مَّا يَوَدُّ ٱلَّذِينَ كَفَرُوا۟ مِنْ أَهْلِ ٱلْكِتَـٰبِ وَلَا ٱلْمُشْرِكِينَ أَن يُنَزَّلَ عَلَيْكُم مِّنْ خَيْرٍۢ مِّن رَّبِّكُمْ ۗ وَٱللَّهُ يَخْتَصُّ بِرَحْمَتِهِۦ مَن يَشَآءُ ۚ وَٱللَّهُ ذُو ٱلْفَضْلِ ٱلْعَظِيمِ ﴿١٠٥﴾ ۞\\
\textamh{106.\ አንድ ጥቅስ ብንተው (አላፊ ብናደርገው) ወይንም ብናሰረሳው፥ አዲስ ከሱ የበለጠ ወይም ተመሳሳይ እናመጣለን ። ኣላህ ሁሉን ማድረግ እንደሚይችል አታዉቁም ወይ?   } &  مَا نَنسَخْ مِنْ ءَايَةٍ أَوْ نُنسِهَا نَأْتِ بِخَيْرٍۢ مِّنْهَآ أَوْ مِثْلِهَآ ۗ أَلَمْ تَعْلَمْ أَنَّ ٱللَّهَ عَلَىٰ كُلِّ شَىْءٍۢ قَدِيرٌ ﴿١٠٦﴾\\
\textamh{107.\ የመሬትና(የምድርና) የሰማይ ግዛት (ስልጣን) የኣላህ እንደሆነ አታውቁም? ከኣላህ በስተቀር ወሊ (ተክላካይ፥ተንከባካቢ፥ ጠባቂ) ወይም ረዳት የላችሁም   } &   أَلَمْ تَعْلَمْ أَنَّ ٱللَّهَ لَهُۥ مُلْكُ ٱلسَّمَـٰوَٟتِ وَٱلْأَرْضِ ۗ وَمَا لَكُم مِّن دُونِ ٱللَّهِ مِن وَلِىٍّۢ وَلَا نَصِيرٍ ﴿١٠٧﴾\\
\textamh{108.\ ወይስ መልእክተኛቹህን (ሙሐመድ(ሠአወሰ)) ሙሳን እንደጠየቁት ትጠይቁታላችሁ (ኣምላክህን አሳየን)? ማን ነው እምነትን በክህደት የሚቀይር፥ በእዉነት፥ ከትክክለኛው መንገድ ስቷል።   } &   أَمْ تُرِيدُونَ أَن تَسْـَٔلُوا۟ رَسُولَكُمْ كَمَا سُئِلَ مُوسَىٰ مِن قَبْلُ ۗ وَمَن يَتَبَدَّلِ ٱلْكُفْرَ بِٱلْإِيمَـٰنِ فَقَدْ ضَلَّ سَوَآءَ ٱلسَّبِيلِ ﴿١٠٨﴾\\
\textamh{109.\ ብዙዎቹ የመጽሐፉ ባለቤት (ይሁዶችና ክርስቲያኖች) ከሀዲዎች አድርገው ቢመልሷቹህ ይመኛሉ፥ ከራሳቸው የሚፈልቅ ምቀኝነታቸው የተነሳ፥ እዉነቱ (ሙሐመድ(ሰ አወሰ) የኣላህ መልእክተኛ መሆኑ) ግልጽ ከሆነላቸው በኋላም። ግን ይቅር በሉና እለፉት፥ ኣላህ ትእዛዙን እስኪያመጣ። በእዉነት፥ኣላህ ሁሉን ማድረግ ይችላል    } &  وَدَّ كَثِيرٌۭ مِّنْ أَهْلِ ٱلْكِتَـٰبِ لَوْ يَرُدُّونَكُم مِّنۢ بَعْدِ إِيمَـٰنِكُمْ كُفَّارًا حَسَدًۭا مِّنْ عِندِ أَنفُسِهِم مِّنۢ بَعْدِ مَا تَبَيَّنَ لَهُمُ ٱلْحَقُّ ۖ فَٱعْفُوا۟ وَٱصْفَحُوا۟ حَتَّىٰ يَأْتِىَ ٱللَّهُ بِأَمْرِهِۦٓ ۗ إِنَّ ٱللَّهَ عَلَىٰ كُلِّ شَىْءٍۢ قَدِيرٌۭ ﴿١٠٩﴾\\
\textamh{110.\ እና ሳለት ቁሙ፥ ዘካት ስጡ ማንኛዉም ጥሩ ነገር በፊታችሁ ብታደርጉ፥ ከኣላህ ታገኙታላችሁ። በእርግጠኛነት ኣላህ የምትሰሩትን ሁሉ ያያል።   } &   وَأَقِيمُوا۟ ٱلصَّلَوٰةَ وَءَاتُوا۟ ٱلزَّكَوٰةَ ۚ وَمَا تُقَدِّمُوا۟ لِأَنفُسِكُم مِّنْ خَيْرٍۢ تَجِدُوهُ عِندَ ٱللَّهِ ۗ إِنَّ ٱللَّهَ بِمَا تَعْمَلُونَ بَصِيرٌۭ ﴿١١٠﴾\\
\textamh{111.\ እናም ይላሉ \rq\rq{}ማንም ይሁዲየ ወይንም ክርስቲያን ካልሆነ ገነት ዉስጥ አይገባም\rq\rq{}። ይሄ የራሳቸው ምኞት ነው። (እንዲህ) በል (ኦ ሙሐመድ(ሠአወሰ): \rq\rq{}መረጋገጫችሁን አምጡ እዉነተኛ ከሆናችሁ\rq\rq{}    } &  وَقَالُوا۟ لَن يَدْخُلَ ٱلْجَنَّةَ إِلَّا مَن كَانَ هُودًا أَوْ نَصَٟرَىٰ ۗ تِلْكَ أَمَانِيُّهُمْ ۗ قُلْ هَاتُوا۟ بُرْهَـٰنَكُمْ إِن كُنتُمْ صَٟدِقِينَ ﴿١١١﴾\\
\textamh{112.\ አዎ፥ ነገር ግን ማንም ወደኣላህ ፊቱን ቢያዞር (በመገዛት) እና ጥሩ ሰሪ ከሆነ ክፍያው ከአምላኩ አለ፥ ከነዚህ ላይ ፍርሃት አይኖርም፥ አያዝኑምም    } &  بَلَىٰ مَنْ أَسْلَمَ وَجْهَهُۥ لِلَّهِ وَهُوَ مُحْسِنٌۭ فَلَهُۥٓ أَجْرُهُۥ عِندَ رَبِّهِۦ وَلَا خَوْفٌ عَلَيْهِمْ وَلَا هُمْ يَحْزَنُونَ ﴿١١٢﴾\\
\textamh{113.\ ይሁዶች ክርስቲያኖች ምንም ነገር አይከተሉም አሉ፥ ክርስቲያኖች ይሁዶች ምንም አይከተሉም አሉ፤ ምንም እንኳ ሁለቱም (ከአንድ) መጽሐፍ ቢያነቡም። እንደነሱ ቃል፥ (ፓገኖችም)የማያዉቁት ተመሳሳይ ነገር አሉ። ኣላህ የትንሳኤ ቀን ይፈርድላቸዋል የሚለያዩበት ነገር ላይ    } &  وَقَالَتِ ٱلْيَهُودُ لَيْسَتِ ٱلنَّصَٟرَىٰ عَلَىٰ شَىْءٍۢ وَقَالَتِ ٱلنَّصَٟرَىٰ لَيْسَتِ ٱلْيَهُودُ عَلَىٰ شَىْءٍۢ وَهُمْ يَتْلُونَ ٱلْكِتَـٰبَ ۗ كَذَٟلِكَ قَالَ ٱلَّذِينَ لَا يَعْلَمُونَ مِثْلَ قَوْلِهِمْ ۚ فَٱللَّهُ يَحْكُمُ بَيْنَهُمْ يَوْمَ ٱلْقِيَـٰمَةِ فِيمَا كَانُوا۟ فِيهِ يَخْتَلِفُونَ ﴿١١٣﴾\\
\textamh{114.\ ከዚህ በላይ ማነው ጠማማ የኣላህ ስም በኣላህ መስጂድ ዉስጥ ብዙ እንዳይጠራና እንዳይከበር የሚከለክል እና እንዲጠፉ የሚታገል? እነዚህ ራሳቸው(መስጂድ) ይገቡ ዘንድ አይገባም በፍራህት በስተቀር። ለነዚህ እዚህ አለም ዉስጥ ዉርዴት፥ በሰማይዊ ህይወት ደግሞ ታላቅ ቅጣት ይኖራቸዋል    } &   وَمَنْ أَظْلَمُ مِمَّن مَّنَعَ مَسَٟجِدَ ٱللَّهِ أَن يُذْكَرَ فِيهَا ٱسْمُهُۥ وَسَعَىٰ فِى خَرَابِهَآ ۚ أُو۟لَٟٓئِكَ مَا كَانَ لَهُمْ أَن يَدْخُلُوهَآ إِلَّا خَآئِفِينَ ۚ لَهُمْ فِى ٱلدُّنْيَا خِزْىٌۭ وَلَهُمْ فِى ٱلْءَاخِرَةِ عَذَابٌ عَظِيمٌۭ ﴿١١٤﴾\\
\textamh{115.\ ምስራቁም ምእራቡም የኣላህ ነው፥ ስለዚህ ፊታችሁን ባዞራችሁበት ሁሉ፥ የኣላህ ፊት አለ (እሱም ከፍ ብሎ፥ ከዙፋኑ ላይ)። በእርግጠኝነት! ኣላህ ለፍጥረቶቹ ፍላጎት ከሁሉ በላይ በቂ ነው። ሁሉን አወቂ   } &  وَلِلَّهِ ٱلْمَشْرِقُ وَٱلْمَغْرِبُ ۚ فَأَيْنَمَا تُوَلُّوا۟ فَثَمَّ وَجْهُ ٱللَّهِ ۚ إِنَّ ٱللَّهَ وَٟسِعٌ عَلِيمٌۭ ﴿١١٥﴾\\
\textamh{116.\ እናም አሉ (ይሁዶች፥ ክርስቲያኖች እና ፓጋኖች): ኣላህ ልጅ ወልዷል። ስብሃት ለሱ ይሁን (እነሱ ከሚያሻርኩት በላይ ክብር ለሱ ይሁን)። የለም፥ ሰማይና መሬት የሱ ናቸው፥ ሁሉም ለሱ በመገዛት ይሰግዳሉ።    } &  وَقَالُوا۟ ٱتَّخَذَ ٱللَّهُ وَلَدًۭا ۗ سُبْحَٟنَهُۥ ۖ بَل لَّهُۥ مَا فِى ٱلسَّمَـٰوَٟتِ وَٱلْأَرْضِ ۖ كُلٌّۭ لَّهُۥ قَٟنِتُونَ ﴿١١٦﴾\\
\textamh{117.\ የሰማይና የመሬት (ምድር) ጀመሪ። አንድ ነገር ሲያዝ፥ (እንዲህ) ብቻ ነው የሚለው: \rq\rq{}ኩን!\rq\rq{} (ሁን)-እናም ይሆ[ኮ]ናል    } &   بَدِيعُ ٱلسَّمَـٰوَٟتِ وَٱلْأَرْضِ ۖ وَإِذَا قَضَىٰٓ أَمْرًۭا فَإِنَّمَا يَقُولُ لَهُۥ كُن فَيَكُونُ ﴿١١٧﴾\\
\textamh{118.\ እዉቀት የሌላቸው (እንዲህ) አሉ: \rq\rq{}ለምድነው ኣላህ (ፊት ለፊት) እኛን የማያናገረው ወይም ምልክት ወደኛ ለምን አይመጣም?\rq\rq{} ከነሱም በፊት የነበሩት ሰዎች እንዲሁ መሳይ ቃል ተናገረዋል። ልባቸው አንድ አይነት ነው፥ በእዉነት እኛ ምልክቱን ግልጽ አድረገናል በእርግጠኝነት ለሚያምኑ ሰዎች።    } &  وَقَالَ ٱلَّذِينَ لَا يَعْلَمُونَ لَوْلَا يُكَلِّمُنَا ٱللَّهُ أَوْ تَأْتِينَآ ءَايَةٌۭ ۗ كَذَٟلِكَ قَالَ ٱلَّذِينَ مِن قَبْلِهِم مِّثْلَ قَوْلِهِمْ ۘ تَشَٟبَهَتْ قُلُوبُهُمْ ۗ قَدْ بَيَّنَّا ٱلْءَايَـٰتِ لِقَوْمٍۢ يُوقِنُونَ ﴿١١٨﴾\\
\textamh{119.\ በእዉነት፥ አንተን (ኦ ሙሐመድ (ሠአወሰ)) በሃቅ (ኢስለም) ልከነሀል፥ አብሳሪና አስጠንቃቂ። ስለሚነደው እሳት ነዋሪዎች አትጠየቅም።   } &   إِنَّآ أَرْسَلْنَـٰكَ بِٱلْحَقِّ بَشِيرًۭا وَنَذِيرًۭا ۖ وَلَا تُسْـَٔلُ عَنْ أَصْحَٟبِ ٱلْجَحِيمِ ﴿١١٩﴾\\
\textamh{120.\ በፍጹም ይሁዶች ወይንም ነሳራዎች (ክርስቲያኖች) አይደሰቱም፥ ሃይማኖታቸውን እስክትከተል ድረስ። (እንዲህ) በል: \rq\rq{}በእዉነት የኣላህ መመሪያ (ኢስላም) ያ ነው ትክክለኛ መመሪያ። እናም አንተ (ኦ ሙሐመድ(ሠአወሰ)) የነሱን (ይሁዶችና ክርስቲያኖች) ምኞት ብትከተል እዉቀት ከመጣልህ በኋላ፥ ከዚያም ከኣላህ ወሊ(ተከላካይ ወይም ጠባቂ) ወይንም ረዳት አይኖርህም    } &   وَلَن تَرْضَىٰ عَنكَ ٱلْيَهُودُ وَلَا ٱلنَّصَٟرَىٰ حَتَّىٰ تَتَّبِعَ مِلَّتَهُمْ ۗ قُلْ إِنَّ هُدَى ٱللَّهِ هُوَ ٱلْهُدَىٰ ۗ وَلَىِٕنِ ٱتَّبَعْتَ أَهْوَآءَهُم بَعْدَ ٱلَّذِى جَآءَكَ مِنَ ٱلْعِلْمِ ۙ مَا لَكَ مِنَ ٱللَّهِ مِن وَلِىٍّۢ وَلَا نَصِيرٍ ﴿١٢٠﴾\\
\textamh{121.\ እነዚያ (ከእስራኤል ልጆች ወደኢስላም የገቡ) መጽሃፍ (ተውራት) የሰጠናቸው እና ይሄን መጽሃፍ (ቁርአን) የሰጠናቸው እንዲያነቡት መነበብ እንዳለበት፥ እነሱ ናቸው እዚህ ዉስጥ ባለው የሚያምኑ። እና ማንም (በዚህ ቁርአን) የማያምን፥ እነሱ ናቸው ከሳሪዎቹ።   } &   ٱلَّذِينَ ءَاتَيْنَـٰهُمُ ٱلْكِتَـٰبَ يَتْلُونَهُۥ حَقَّ تِلَاوَتِهِۦٓ أُو۟لَٟٓئِكَ يُؤْمِنُونَ بِهِۦ ۗ وَمَن يَكْفُرْ بِهِۦ فَأُو۟لَٟٓئِكَ هُمُ ٱلْخَـٰسِرُونَ ﴿١٢١﴾\\
\textamh{122.\ ኦ! እናንት የእስራኤል ልጆች! ለእናንተ ያደረግኩትን አስታዉሱ፥ ከአላሚን አስበልጬ እንደመረጥኳችሁ   } &  يَـٰبَنِىٓ إِسْرَٟٓءِيلَ ٱذْكُرُوا۟ نِعْمَتِىَ ٱلَّتِىٓ أَنْعَمْتُ عَلَيْكُمْ وَأَنِّى فَضَّلْتُكُمْ عَلَى ٱلْعَٟلَمِينَ ﴿١٢٢﴾\\
\textamh{123.\ አንድ ቀን ግን ፍሩ (የፍርድ ቀን) አንዱ ሌላው የማያወጣበት፥ ወይንም ካሳ ክፍያ የማይቀበሉበት  ወይንም ምልድጃ ምንም ጥቅም የማይኖረው ወይንም የማይረዱበት   } &  وَٱتَّقُوا۟ يَوْمًۭا لَّا تَجْزِى نَفْسٌ عَن نَّفْسٍۢ شَيْـًۭٔا وَلَا يُقْبَلُ مِنْهَا عَدْلٌۭ وَلَا تَنفَعُهَا شَفَٟعَةٌۭ وَلَا هُمْ يُنصَرُونَ ﴿١٢٣﴾ ۞\\
\textamh{124.\ የኢብራሂም (አብርሃም) አምላክ በትእዛዝ ሲፈትነው (ኢብራሂምን)፥ (ትእዛዙን) ፈጸመ። እሱም (ኣላህ) አለ(ው): \rq\rq{}በእዉነት፥ የሰዎች መሪ (ኢማም) አደርግሀለሁ\rq\rq{} (ኢብራሂምም) አለ፥\rq\rq{}የኔን ዘር ደግሞስ (መሪ አድረጋቸው)\rq\rq{}። (ኣላህ) አለ፥ \rq\rq{}ቃል ኪዳኔ ዛሊሙን (አጥፊዎችና አማልክት አምላኪዎችን) አይጨምርም\rq\rq{}።   } &  وَإِذِ ٱبْتَلَىٰٓ إِبْرَٟهِۦمَ رَبُّهُۥ بِكَلِمَـٰتٍۢ فَأَتَمَّهُنَّ ۖ قَالَ إِنِّى جَاعِلُكَ لِلنَّاسِ إِمَامًۭا ۖ قَالَ وَمِن ذُرِّيَّتِى ۖ قَالَ لَا يَنَالُ عَهْدِى ٱلظَّٟلِمِينَ ﴿١٢٤﴾\\
\textamh{125.\ እናም  ቤቱን (መካ ያለውን ካባ) የሰዎች መናገሻና ሰላም ማግኛ አድርገነዋል። እናም እናንተ(ሰዎች) የኢብራሂምን መቆሚያ መጸለያ አድሩጉት እና እኛ ኢብራሂምንና(አብርሃምን) ኢስማኢል (ኢስማኤል) ቤቴን እንዲያነጹ አዘናቸዋል፥ ለሚዞሩትና፥ ለሚቀመጡ (ኢቲካፍ)፥ ወይም ጎንበስ ለሚሉት ወይም ለሚሰግዱት።    } &   وَإِذْ جَعَلْنَا ٱلْبَيْتَ مَثَابَةًۭ لِّلنَّاسِ وَأَمْنًۭا وَٱتَّخِذُوا۟ مِن مَّقَامِ إِبْرَٟهِۦمَ مُصَلًّۭى ۖ وَعَهِدْنَآ إِلَىٰٓ إِبْرَٟهِۦمَ وَإِسْمَـٰعِيلَ أَن طَهِّرَا بَيْتِىَ لِلطَّآئِفِينَ وَٱلْعَٟكِفِينَ وَٱلرُّكَّعِ ٱلسُّجُودِ ﴿١٢٥﴾\\
\textamh{126.\ እናም ኢብራሂም አለ፥\rq\rq{}አምላኬ!፥ ይህችን ከተማ (መካ) የሰላም ማግኛ አድርጋት እና ለስዎቿ ፍራፍሬ ስጣቸው፥ በኣላህና በመጨረሻው ቀን ለሚያምኑ።\rq\rq{} እሱም (ኣላህ) መለሰለት: \rq\rq{}ለማይምኑት፥ ለጊዜው ፍለጎቱን አሟላለታለሁ ከዚያ ወደ እሳቱ እንዲገባ አስገድደዋለሁ፥ ከሁሉም የከፋ መሄጃ (ከሱ ሌላ የከፋ መሄጃ የለም)   } &  وَإِذْ قَالَ إِبْرَٟهِۦمُ رَبِّ ٱجْعَلْ هَـٰذَا بَلَدًا ءَامِنًۭا وَٱرْزُقْ أَهْلَهُۥ مِنَ ٱلثَّمَرَٟتِ مَنْ ءَامَنَ مِنْهُم بِٱللَّهِ وَٱلْيَوْمِ ٱلْءَاخِرِ ۖ قَالَ وَمَن كَفَرَ فَأُمَتِّعُهُۥ قَلِيلًۭا ثُمَّ أَضْطَرُّهُۥٓ إِلَىٰ عَذَابِ ٱلنَّارِ ۖ وَبِئْسَ ٱلْمَصِيرُ ﴿١٢٦﴾\\
\textamh{127.\ ኢብራሂምና (አብርሃም) ኢስማኢል የቤቱን (የካባ) መሰረት ሲጥሉ: \rq\rq{}አምላክችን! ይህንን ከኛ ተቀበል፤ በእዉነት! አንተ ሁሉን-ሰሚ፥ ሁሉን-አዋቂ ነህ\rq\rq{}   } &   وَإِذْ يَرْفَعُ إِبْرَٟهِۦمُ ٱلْقَوَاعِدَ مِنَ ٱلْبَيْتِ وَإِسْمَـٰعِيلُ رَبَّنَا تَقَبَّلْ مِنَّآ ۖ إِنَّكَ أَنتَ ٱلسَّمِيعُ ٱلْعَلِيمُ ﴿١٢٧﴾\\
\textamh{128.\ \rq\rq{}አምላካችን! ለአንተ ተገዢ አድረገን (ሙስሊም) እና ዘሮቻችን ለአንተ ተገዢ ብሄር አድርጋቸው፥ ማናሲክ (ባህላቶችን፥ ሃጅ፥ ኡምራህ..) አሳየን እና ንሳሃችን ተቀበል። በእዉነት አንተ (ብቻ) ነህ ንስሃ ተቀባይ፥ ከሁሉም ባላይ ምህርተኛው።   } &  رَبَّنَا وَٱجْعَلْنَا مُسْلِمَيْنِ لَكَ وَمِن ذُرِّيَّتِنَآ أُمَّةًۭ مُّسْلِمَةًۭ لَّكَ وَأَرِنَا مَنَاسِكَنَا وَتُبْ عَلَيْنَآ ۖ إِنَّكَ أَنتَ ٱلتَّوَّابُ ٱلرَّحِيمُ ﴿١٢٨﴾\\
\textamh{129.\ \rq\rq{}አምላካችን! ከነሱ መካከል መልእክተኛ ላክላቸው (በእዉነት ኣላህም ሙሐመድን(ሠአወሰ) በመላክ ዱዋቸዉን መልሶላቸዋል)፥ ጥቅሶችህን የሚያነበላቸው እና በመጽሐፍ (ቁርአን) የሚያዛቸው እና አል-ሂክማህ( ሙሉ የእስልምና መንገዶችን እዉቀት) እና አንጻቸው። በእዉነት! አንተ ከሁሉም በላይ ሀያል፥ ከሁሉም በላይ መርማሪ-አወቂ ነህ።    } &   رَبَّنَا وَٱبْعَثْ فِيهِمْ رَسُولًۭا مِّنْهُمْ يَتْلُوا۟ عَلَيْهِمْ ءَايَـٰتِكَ وَيُعَلِّمُهُمُ ٱلْكِتَـٰبَ وَٱلْحِكْمَةَ وَيُزَكِّيهِمْ ۚ إِنَّكَ أَنتَ ٱلْعَزِيزُ ٱلْحَكِيمُ ﴿١٢٩﴾\\
\textamh{130.\ ማን ነው ከኢብራሂም ሃይማኖት (ኢስላም) ዘወር የሚል ራሱን ከማታለል በቀር? በእዉነት፥ በዚህ አለም መረጥነው እና በእውነት፥ በሚቀጥለው አለም ከጸድቃን መካከል ነው የሚሆን   } &  وَمَن يَرْغَبُ عَن مِّلَّةِ إِبْرَٟهِۦمَ إِلَّا مَن سَفِهَ نَفْسَهُۥ ۚ وَلَقَدِ ٱصْطَفَيْنَـٰهُ فِى ٱلدُّنْيَا ۖ وَإِنَّهُۥ فِى ٱلْءَاخِرَةِ لَمِنَ ٱلصَّٟلِحِينَ ﴿١٣٠﴾\\
\textamh{131.\ አምላኩ (እንዲህ) ሲለው: \rq\rq{}ተገዛ (ስለም)!\rq\rq{}፥ እሱም አለ: \rq\rq{}እገዛለሁ (እሰልማለሁ) ለአላሚን (ሰዎች፥ ጅኖች እና ያለ ነገር በሙሉ) ጌታ\rq\rq{}   } &  إِذْ قَالَ لَهُۥ رَبُّهُۥٓ أَسْلِمْ ۖ قَالَ أَسْلَمْتُ لِرَبِّ ٱلْعَٟلَمِينَ ﴿١٣١﴾\\
\textamh{132.\ ይሄም በልጆቹ ላይ (እንዲገዙ) ኢብረሂም ትእዛዝ አስተላለፈ (ጋበዛቸው)፥ እና ያቁብ (ያቆብ)፥ \rq\rq{}ኦ ልጆቼ! ኣላህ (ሀቁን) ሃይመኖት መርጦላችኋል፥ ስለዚህ ሙስሊም ሳትሆኑ አትሙቱ።    } &  وَوَصَّىٰ بِهَآ إِبْرَٟهِۦمُ بَنِيهِ وَيَعْقُوبُ يَـٰبَنِىَّ إِنَّ ٱللَّهَ ٱصْطَفَىٰ لَكُمُ ٱلدِّينَ فَلَا تَمُوتُنَّ إِلَّا وَأَنتُم مُّسْلِمُونَ ﴿١٣٢﴾\\
\textamh{133.\ ወይስ ምስክሮች ነበራችሁ ያቁብን (ያቆብን) ሞት ሲቀርበው? ለልጆቹ እንዲህ ሲል፥\rq\rq{}ከኔ በኋላ ምን ታምልካላችሁ?\rq\rq{} እነሱም አሉ፥\rq\rq{}እኛ የአንተን አምላክ ፥ የአባቶችህን የኢብራሂም፥ የኢስማኢል፥ የኢስሃቅን አምላክ እናመልካላን፥ አንድ አምላክ፥ ለሱ ተገዝተናል (ሰልመናል)   } &   أَمْ كُنتُمْ شُهَدَآءَ إِذْ حَضَرَ يَعْقُوبَ ٱلْمَوْتُ إِذْ قَالَ لِبَنِيهِ مَا تَعْبُدُونَ مِنۢ بَعْدِى قَالُوا۟ نَعْبُدُ إِلَٟهَكَ وَإِلَٟهَ ءَابَآئِكَ إِبْرَٟهِۦمَ وَإِسْمَـٰعِيلَ وَإِسْحَٟقَ إِلَٟهًۭا وَٟحِدًۭا وَنَحْنُ لَهُۥ مُسْلِمُونَ ﴿١٣٣﴾\\
\textamh{134.\ እነዚህ ያለፉ ብሄሮች ናቸው። የሰሩትን ክፍያ ለራሳቸው ይቀበላሉ እናንተም የሰራችሁትን። እነሱ ምን እይስሩ እንደነበር አትጠየቁም    } &   تِلْكَ أُمَّةٌۭ قَدْ خَلَتْ ۖ لَهَا مَا كَسَبَتْ وَلَكُم مَّا كَسَبْتُمْ ۖ وَلَا تُسْـَٔلُونَ عَمَّا كَانُوا۟ يَعْمَلُونَ ﴿١٣٤﴾\\
\textamh{135.\ እናም ይላሉ: \rq\rq{}ይሁዲያ ወይም ክርስቲያን ሁኑ፥ እንድትመሩ\rq\rq{} (እንዲህ) በል (ለነሱ፥ ኦ ሙሐመድ(ሠአወሰ)) \rq\rq{}የለም፥ የኢብራሂምን (የአብርሃምን) ሃኒፋ (ከሽርክ የጸዳ) ሃይማኖት (እንከተላለን)፥ ከሙሽሪኮች (ከኣላህ ጋር ሌላ አምላክ(አማልክት)ን የሚደርቡ) አልነበረም።    } &  وَقَالُوا۟ كُونُوا۟ هُودًا أَوْ نَصَٟرَىٰ تَهْتَدُوا۟ ۗ قُلْ بَلْ مِلَّةَ إِبْرَٟهِۦمَ حَنِيفًۭا ۖ وَمَا كَانَ مِنَ ٱلْمُشْرِكِينَ ﴿١٣٥﴾\\
\textamh{136.\ (እንዲህ) በሉ (ኦ ሙስሊሞች) \rq\rq{}በኣላህ እናምናለን ለእኛ በወረደው (በዚህ ቁርአን) እና ለኢብራሂም (አብርሃም)፥ ለኢስማኢል፥ ለኢስሃቅ፥ ለያቁብ (ያቆብ) እና ለአል-አስባጥ (የያቁብ(ያቆብ) አስራሁለት ልጆች)፥ እና ለሙሳ(ሙሴ) እና ኢሳ (የሱስ) በተሰጠው እና ለነቢያት ከአምላካቸው በተሰጠው። ምንም አንለያያቸዉም፥ ለሱ ተገዝተናል (ሰልመናል)\rq\rq{}   } &   قُولُوٓا۟ ءَامَنَّا بِٱللَّهِ وَمَآ أُنزِلَ إِلَيْنَا وَمَآ أُنزِلَ إِلَىٰٓ إِبْرَٟهِۦمَ وَإِسْمَـٰعِيلَ وَإِسْحَٟقَ وَيَعْقُوبَ وَٱلْأَسْبَاطِ وَمَآ أُوتِىَ مُوسَىٰ وَعِيسَىٰ وَمَآ أُوتِىَ ٱلنَّبِيُّونَ مِن رَّبِّهِمْ لَا نُفَرِّقُ بَيْنَ أَحَدٍۢ مِّنْهُمْ وَنَحْنُ لَهُۥ مُسْلِمُونَ ﴿١٣٦﴾\\
\textamh{137.\ ስለዚህ እናንተ እንዳመናችሁት ቢያምኑ፥ በትክክለኛዉ መንገድ ተመርተዋል፥ ነገር ግን ቢዞሩ፥ ተቃራኒ ናቸው። (ስለነሱ) ኣላህ ለእናንተ በቂ ነው። ደግሞም ሁሉን-ሰሚ፥ ሁሉን-አወቂ ነው።   } &   فَإِنْ ءَامَنُوا۟ بِمِثْلِ مَآ ءَامَنتُم بِهِۦ فَقَدِ ٱهْتَدَوا۟ ۖ وَّإِن تَوَلَّوْا۟ فَإِنَّمَا هُمْ فِى شِقَاقٍۢ ۖ فَسَيَكْفِيكَهُمُ ٱللَّهُ ۚ وَهُوَ ٱلسَّمِيعُ ٱلْعَلِيمُ ﴿١٣٧﴾\\
\textamh{138.\ (የኛ ሲብጋህ (ሃይማኖት))የኣላህ ሲብጋህ (ሃይማኖት)፥ የትኛው ሲብጋህ (ሃይማኖት) ከኣላህ ሲብጋህ (ሃይማኖት) ነው የሚሻል? እኛ አምላኪዎቹ ነን።   } &  صِبْغَةَ ٱللَّهِ ۖ وَمَنْ أَحْسَنُ مِنَ ٱللَّهِ صِبْغَةًۭ ۖ وَنَحْنُ لَهُۥ عَٟبِدُونَ ﴿١٣٨﴾\\
\textamh{139.\ (እንዲህ) በል (ኦ ሙሐመድ(ሠአወሰ)) (ለይሁዶችና ክርስቲያኖች):\rq\rq{}ከኛ ጋር ስለ ኣላህ ትከራከራላችሁ፥ እሱ የኛም የናንተም አምላክ ሁኖ ሳል? እኛም የስራችን ይክፈለናል እናንተም የስራችሁ። እኛ ልባችን ለሱ እንሰጣለን በአምልኮ ሆነ በመገዛት   } &  قُلْ أَتُحَآجُّونَنَا فِى ٱللَّهِ وَهُوَ رَبُّنَا وَرَبُّكُمْ وَلَنَآ أَعْمَـٰلُنَا وَلَكُمْ أَعْمَـٰلُكُمْ وَنَحْنُ لَهُۥ مُخْلِصُونَ ﴿١٣٩﴾\\
\textamh{140.\ ወይስ ትላላችሁ እናንተ ኢብራሂም (አብርሃም)፥ ኢስማኢል፥ ኢስሃቅ፥ ያቁብ(ያቆብ) እና አል-አስባጥ (የያቁብ አስራሁለት ልጆች) ይሁዶች ወይም ክርስቲያኖች ነበሩ? (እንዲህ) በሉ: \rq\rq{}እናንተ የተሻለ ታውቃላችሁ ወይስ ኣላህ? ከዚህ የበለጠ ጠማማ ማነው እዉነተኛ ምስክርነት ከኣላህ ያለዉን የሚደብቅ? (ከመጸሀፉ እንደተጻፈው ሙሐመድ(ሠአወሰ) የሚመጣ መሆኑን የሚደብቅ) ኣላህ የምታደርጉትን የማያዉቅ አይደለም።    } &  أَمْ تَقُولُونَ إِنَّ إِبْرَٟهِۦمَ وَإِسْمَـٰعِيلَ وَإِسْحَٟقَ وَيَعْقُوبَ وَٱلْأَسْبَاطَ كَانُوا۟ هُودًا أَوْ نَصَٟرَىٰ ۗ قُلْ ءَأَنتُمْ أَعْلَمُ أَمِ ٱللَّهُ ۗ وَمَنْ أَظْلَمُ مِمَّن كَتَمَ شَهَـٰدَةً عِندَهُۥ مِنَ ٱللَّهِ ۗ وَمَا ٱللَّهُ بِغَٟفِلٍ عَمَّا تَعْمَلُونَ ﴿١٤٠﴾\\
\textamh{141.\ እነዚህ ያለፉ ብሄሮች ናቸው። የሰሩትን ክፍያ ለራሳቸው ይቀበላሉ እናንተም የሰራችሁትን። እነሱ ምን እይስሩ እንደነበር አትጠየቁም   } &   تِلْكَ أُمَّةٌۭ قَدْ خَلَتْ ۖ لَهَا مَا كَسَبَتْ وَلَكُم مَّا كَسَبْتُمْ ۖ وَلَا تُسْـَٔلُونَ عَمَّا كَانُوا۟ يَعْمَلُونَ ﴿١٤١﴾ ۞ \\
\textamh{142.\ ከሰዎች መካከል ጅሎች (ፓጋኖች፥ መናፍቃን፥ እነ ይሁዶች) (እንዲህ) ይላሉ: \rq\rq{}ምንድነው እነዚህ ሙስሊሞች ያዞራቸው (ከመጸለያቸው አቅጣጫ-ቂብለህ) ሲጸልዩ ይዞሩበት ከነበረው (ከየሩሳሌም)?\rq\rq{} (እንዲህ) በል (ኦ ሙሐመድ(ሠአወሰ)):\rq\rq{}ምስራቁም ምእራቡም የኣላህ ነው። የፈለገዉን ወደ ትክክለኛው መንገድ ይመራል\rq\rq{}    } &  سَيَقُولُ ٱلسُّفَهَآءُ مِنَ ٱلنَّاسِ مَا وَلَّىٰهُمْ عَن قِبْلَتِهِمُ ٱلَّتِى كَانُوا۟ عَلَيْهَا ۚ قُل لِّلَّهِ ٱلْمَشْرِقُ وَٱلْمَغْرِبُ ۚ يَهْدِى مَن يَشَآءُ إِلَىٰ صِرَٟطٍۢ مُّسْتَقِيمٍۢ ﴿١٤٢﴾\\
\textamh{143.\ እናም አደረግናችሁ (ትክክለኛ ሙስሊሞች)፥ ቅን (ከሁሉም የተሻለ) ብሄር፥ የሰው ልጆች ላይ ምስክር ትሆኑ ዘንድ እና መልእክተኛው (ሙሐመድ(ሠአወሰ))እናንተ ላይ ምስክር። ቂብላዉን ስትዞሩበት ወደነበረ ያደረግነው መልእክተኛዉን (ሙሐመድ(ሠአወሰ) የሚከተሉትን ለመፈተን (ለማወቅ) ነበር ከነዚያ እግራቸው ከሚያዞሩት (የማይከተሉህን) ለመለየት። በእውነት ከባድ ነበር ኣላህ ከመራቸው በስተቀር። ኣላህ ደግሞ እምነታችሁን እንድታጡ አያደርግም። በእዉነት፥ ኣላህ ብዙ ርህራሄ አለው፥ ከሁሉ የበለጠ ምህረተኛው ለሰው ልጆች   } &  وَكَذَٟلِكَ جَعَلْنَـٰكُمْ أُمَّةًۭ وَسَطًۭا لِّتَكُونُوا۟ شُهَدَآءَ عَلَى ٱلنَّاسِ وَيَكُونَ ٱلرَّسُولُ عَلَيْكُمْ شَهِيدًۭا ۗ وَمَا جَعَلْنَا ٱلْقِبْلَةَ ٱلَّتِى كُنتَ عَلَيْهَآ إِلَّا لِنَعْلَمَ مَن يَتَّبِعُ ٱلرَّسُولَ مِمَّن يَنقَلِبُ عَلَىٰ عَقِبَيْهِ ۚ وَإِن كَانَتْ لَكَبِيرَةً إِلَّا عَلَى ٱلَّذِينَ هَدَى ٱللَّهُ ۗ وَمَا كَانَ ٱللَّهُ لِيُضِيعَ إِيمَـٰنَكُمْ ۚ إِنَّ ٱللَّهَ بِٱلنَّاسِ لَرَءُوفٌۭ رَّحِيمٌۭ ﴿١٤٣﴾\\
\textamh{144.\ በእዉነት! ፊትክን (ኦ ሙሐመድ(ሠአወሰ)) ወደሰማይ ስታደረግ ተመልክተናል። በእርግጠኝነት፥ ወደ ሚያስደስትህ አቅጣጫ ቂብላህን (የመጸለያ አቅጣጫ) እናዞርልሀለን፥ ስለዚህ ፊትክን ወደ አል-መስጂድ-አል-ሀራም (መካ) አዙር። የትም ብትኖሩ (የተቀመጣችሁ) እናንት ሰዎች፥ ፊታችሁን ወደዚያ አቅጣጫ አዙሩ። በእርግጠኝነት እንዚያ መጽሐፉ የተሰጣቸው (ይሁዶችና ክርስቲያኖች) ከአምላክችሁ እውነቱ (ሀቁ) እንደሆነ ያዉቃሉ።ኣላህ የሚያደርጉትን የማያዉቅ አይደለም።    } &  قَدْ نَرَىٰ تَقَلُّبَ وَجْهِكَ فِى ٱلسَّمَآءِ ۖ فَلَنُوَلِّيَنَّكَ قِبْلَةًۭ تَرْضَىٰهَا ۚ فَوَلِّ وَجْهَكَ شَطْرَ ٱلْمَسْجِدِ ٱلْحَرَامِ ۚ وَحَيْثُ مَا كُنتُمْ فَوَلُّوا۟ وُجُوهَكُمْ شَطْرَهُۥ ۗ وَإِنَّ ٱلَّذِينَ أُوتُوا۟ ٱلْكِتَـٰبَ لَيَعْلَمُونَ أَنَّهُ ٱلْحَقُّ مِن رَّبِّهِمْ ۗ وَمَا ٱللَّهُ بِغَٟفِلٍ عَمَّا يَعْمَلُونَ ﴿١٤٤﴾\\
\textamh{145.\ የፈለግከው አይነት አያት (ምልክት፥ ተአምር) ለመጽሐፉ ባለቤቶች (ለይሁዶችና ክርስቲያኖች) ብታመጣላቸው፥ የአንተን ቂብለ (የጽለያ አቅጣጫ) አይከተሉም፥ አንተም የነሱን ቂብለ አትከተልም። የየረሳቸውን ቂብለ አይከተሉም። በእዉነት፥ የነሱን ምኞት ብትከተል እዉቀት ከመጣልህ በኋላ (ከኣላህ)፥ ከዚያ በእዉነት አንተ ከዛሊሙን (ከአጥፊዎች፥ ከአማልክት አምላኪዎች) መካከል ትሆናለህ።   } &  وَلَىِٕنْ أَتَيْتَ ٱلَّذِينَ أُوتُوا۟ ٱلْكِتَـٰبَ بِكُلِّ ءَايَةٍۢ مَّا تَبِعُوا۟ قِبْلَتَكَ ۚ وَمَآ أَنتَ بِتَابِعٍۢ قِبْلَتَهُمْ ۚ وَمَا بَعْضُهُم بِتَابِعٍۢ قِبْلَةَ بَعْضٍۢ ۚ وَلَىِٕنِ ٱتَّبَعْتَ أَهْوَآءَهُم مِّنۢ بَعْدِ مَا جَآءَكَ مِنَ ٱلْعِلْمِ ۙ إِنَّكَ إِذًۭا لَّمِنَ ٱلظَّٟلِمِينَ ﴿١٤٥﴾\\
\textamh{146.\ ለነዚያ መጽሐፍ የሰጠናቸው (ይሁዶችና ክርስቲያኖች) ልጆቻቸዉን እንደሚያዉቁ አርገው (ሙሐመድን(ሠአወሰ)) ያውቁታል (ከመጸሀፍቸው እንደተጻፈው)። ነገር ግን በእዉነት፥ ከነሱ መካከል እዉነቱን እያወቁ የሚደብቁ ናቸው።   } &  ٱلَّذِينَ ءَاتَيْنَـٰهُمُ ٱلْكِتَـٰبَ يَعْرِفُونَهُۥ كَمَا يَعْرِفُونَ أَبْنَآءَهُمْ ۖ وَإِنَّ فَرِيقًۭا مِّنْهُمْ لَيَكْتُمُونَ ٱلْحَقَّ وَهُمْ يَعْلَمُونَ ﴿١٤٦﴾\\
\textamh{147.\ (ይሄ) እዉነቱ (ሀቁ) ነው ከአምላካችሁ። ስለዚህ ከሚጠራጠሩት መካከል አትሁኑ።   } &   ٱلْحَقُّ مِن رَّبِّكَ ۖ فَلَا تَكُونَنَّ مِنَ ٱلْمُمْتَرِينَ ﴿١٤٧﴾\\
\textamh{148.\ ለሁሉም ብሄር የሚዞርበት አቅጣጫ አለ (ለመጸለይ)። ስለዚህ ጥሩ ወደሆነው ነገር ሁሉ ተጣደፉ። የትም ቦታ ብትሆኑ፥ ኣላህ ይሰበስባችኋል (የትንሳኤ ቀን)። በእዉነት፥ ኣላህ ሁሉን ነገር ማድረግ ይችላል።   } &   وَلِكُلٍّۢ وِجْهَةٌ هُوَ مُوَلِّيهَا ۖ فَٱسْتَبِقُوا۟ ٱلْخَيْرَٟتِ ۚ أَيْنَ مَا تَكُونُوا۟ يَأْتِ بِكُمُ ٱللَّهُ جَمِيعًا ۚ إِنَّ ٱللَّهَ عَلَىٰ كُلِّ شَىْءٍۢ قَدِيرٌۭ ﴿١٤٨﴾\\
\textamh{149.\ የትም ቦታ ሁናችሁ ብተጀምሩ (ጸሎት)፥ ፊታችሁን ወደ አል-መስጂድ-አል-ሀራም አቅጣጫ አዙሩ፥ ይሄ እዉነት ከአምላካችሁ ነው። ኣላህ የምታደርጉትን የማያዉቅ አይደለም።   } &   وَمِنْ حَيْثُ خَرَجْتَ فَوَلِّ وَجْهَكَ شَطْرَ ٱلْمَسْجِدِ ٱلْحَرَامِ ۖ وَإِنَّهُۥ لَلْحَقُّ مِن رَّبِّكَ ۗ وَمَا ٱللَّهُ بِغَٟفِلٍ عَمَّا تَعْمَلُونَ ﴿١٤٩﴾\\
\textamh{150.\ የትም ቦታ ሁናችሁ ብተጀምሩ (ጸሎት)፥ ፊታችሁን ወደ አል-መስጂድ-አል-ሀራም አቅጣጫ አዙሩ፥ እናም የትም ብትሆኑ፥ ፊታችሁን ወደዚያ አዙሩ፥ ሰዎች ክርክር ከእናንተ ጋር እንዳይኖራቸው ከመጥፎ ሰሪዎች በቀር፥ ስለዚህ አትፍሯቸው፥ ነገር ግን እኔን ፍሩ!- በረከቴን እናንተ ላይ እንደፈጽምላችሁ እናም የተመራችሁ እንድትሆኑ።    } &  وَمِنْ حَيْثُ خَرَجْتَ فَوَلِّ وَجْهَكَ شَطْرَ ٱلْمَسْجِدِ ٱلْحَرَامِ ۚ وَحَيْثُ مَا كُنتُمْ فَوَلُّوا۟ وُجُوهَكُمْ شَطْرَهُۥ لِئَلَّا يَكُونَ لِلنَّاسِ عَلَيْكُمْ حُجَّةٌ إِلَّا ٱلَّذِينَ ظَلَمُوا۟ مِنْهُمْ فَلَا تَخْشَوْهُمْ وَٱخْشَوْنِى وَلِأُتِمَّ نِعْمَتِى عَلَيْكُمْ وَلَعَلَّكُمْ تَهْتَدُونَ ﴿١٥٠﴾\\
\textamh{151.\ በተመሳሳይ፥ የራሳቹህ የሆነ መልእክተኛ (ሙሐመድ(ሠአወሰ)) ልከንላችኋል፥ ጥቅሶቻችን (ቁርአን) እያነበበላችሁ፥ እና እያጸዳችሁ እና መጽሐፉን እና ሂክማ (ሱና፥ ህግጋት፥ ፊቅ) እያስተማራችሁ፥ እና የማታቁትን እያስተማራችሁ   } &  كَمَآ أَرْسَلْنَا فِيكُمْ رَسُولًۭا مِّنكُمْ يَتْلُوا۟ عَلَيْكُمْ ءَايَـٰتِنَا وَيُزَكِّيكُمْ وَيُعَلِّمُكُمُ ٱلْكِتَـٰبَ وَٱلْحِكْمَةَ وَيُعَلِّمُكُم مَّا لَمْ تَكُونُوا۟ تَعْلَمُونَ ﴿١٥١﴾\\
\textamh{152.\ ስለዚህ አስታውሱኝ፥ አስታውሳችኋለሁ እና አመስግኑኝ እና አትካዱኝ   } &  فَٱذْكُرُونِىٓ أَذْكُرْكُمْ وَٱشْكُرُوا۟ لِى وَلَا تَكْفُرُونِ ﴿١٥٢﴾\\
\textamh{153.\ ኦ እናንት አማኞች! በትእግስትና በሳላት (ጸሎት) እርዳታ ፈልጉ። በእዉነት! ኣላህ ከትእግስተኞች ጋር ነው።   } &  يَـٰٓأَيُّهَا ٱلَّذِينَ ءَامَنُوا۟ ٱسْتَعِينُوا۟ بِٱلصَّبْرِ وَٱلصَّلَوٰةِ ۚ إِنَّ ٱللَّهَ مَعَ ٱلصَّٟبِرِينَ ﴿١٥٣﴾\\
\textamh{154.\ በኣላህ መንገድ የተገደሉትን: \rq\rq{}ሞተዋል\rq\rq{} አትበሉ። የለም! ህያዋን ናቸው እናንተ ግን አይታወቃችሁም   } &  وَلَا تَقُولُوا۟ لِمَن يُقْتَلُ فِى سَبِيلِ ٱللَّهِ أَمْوَٟتٌۢ ۚ بَلْ أَحْيَآءٌۭ وَلَٟكِن لَّا تَشْعُرُونَ ﴿١٥٤﴾\\
\textamh{155.\ በእርግጠኝነት በፍርሃት፥ ረሀብ፥ የሀብት (ማጣት)፥ ህይወት እና ፍራፍሬ ማጣት የመሰሉ ነገሮች እንፈትናችኋለን ነገር ግን ለትእግስተኞች አብስሩ   } &  وَلَنَبْلُوَنَّكُم بِشَىْءٍۢ مِّنَ ٱلْخَوْفِ وَٱلْجُوعِ وَنَقْصٍۢ مِّنَ ٱلْأَمْوَٟلِ وَٱلْأَنفُسِ وَٱلثَّمَرَٟتِ ۗ وَبَشِّرِ ٱلصَّٟبِرِينَ ﴿١٥٥﴾\\
\textamh{156.\ መከራ ሲገጥመዉ (እንዲህ) የሚል \rq\rq{}በእዉነት፥ የኣላህ ነን እና በእዉነት ወደ እሱ እንመለሳለን\rq\rq{}   } &  ٱلَّذِينَ إِذَآ أَصَٟبَتْهُم مُّصِيبَةٌۭ قَالُوٓا۟ إِنَّا لِلَّهِ وَإِنَّآ إِلَيْهِ رَٟجِعُونَ ﴿١٥٦﴾\\
\textamh{157.\ እነሱ ናቸው ከአምላካቸው ሰላዋት (የተባረኩ) እና ምህረቱን የሚቀበሉ፥ እነዚህ ናቸው የተመሩ።   } &   أُو۟لَٟٓئِكَ عَلَيْهِمْ صَلَوَٟتٌۭ مِّن رَّبِّهِمْ وَرَحْمَةٌۭ ۖ وَأُو۟لَٟٓئِكَ هُمُ ٱلْمُهْتَدُونَ ﴿١٥٧﴾ ۞\\
\textamh{158.\ በእዉነት አስ-ሳፋ እና አል-ማርዋ (መካ ያሉ ሁላት ተራሮች) የኣላህ ምልክቶች ናቸው። ስለዚህ ሃጅና ኡምራ በነሱ መካከል የሚሄድ (ጠዋፍ) ሀጢያት የለበትም። በራሱ ፈቃድ ጥሩ የሚያደርግ፥ በእዉነት ኣላህ ሁሉን አስተዋይና ሁሉን-አዋቂ ነው።    } &  إِنَّ ٱلصَّفَا وَٱلْمَرْوَةَ مِن شَعَآئِرِ ٱللَّهِ ۖ فَمَنْ حَجَّ ٱلْبَيْتَ أَوِ ٱعْتَمَرَ فَلَا جُنَاحَ عَلَيْهِ أَن يَطَّوَّفَ بِهِمَا ۚ وَمَن تَطَوَّعَ خَيْرًۭا فَإِنَّ ٱللَّهَ شَاكِرٌ عَلِيمٌ ﴿١٥٨﴾\\
\textamh{159.\ በእውነት፥ ግልጽ መረጋገጫ፥ መስረጃ፥ እና መመሪያን የሚደብቁ፥ ያወረድነዉን፥ ለመጽሐፉ ባለቤቶች ግልጽ ካደርገን በኋላ፥ እነሱ ናቸው በኣላህ የተረገሙ እና በረጋሚዎች የተረገሙ    } &  إِنَّ ٱلَّذِينَ يَكْتُمُونَ مَآ أَنزَلْنَا مِنَ ٱلْبَيِّنَـٰتِ وَٱلْهُدَىٰ مِنۢ بَعْدِ مَا بَيَّنَّٟهُ لِلنَّاسِ فِى ٱلْكِتَـٰبِ ۙ أُو۟لَٟٓئِكَ يَلْعَنُهُمُ ٱللَّهُ وَيَلْعَنُهُمُ ٱللَّٟعِنُونَ ﴿١٥٩﴾\\
\textamh{160.\ ንስሃ ከሚገቡና ጥሩ ስራ የሚሰሩ እና (እዉነቱን) በግልጽ የሚያዉጁ በቀር። እነዚህን ንስሃቸዉን እቀበላለሁ። እኔ ነኝ ንስሀ ተቀበይ፥ ከሁሉም በላይ ምህርተኛ   } &   إِلَّا ٱلَّذِينَ تَابُوا۟ وَأَصْلَحُوا۟ وَبَيَّنُوا۟ فَأُو۟لَٟٓئِكَ أَتُوبُ عَلَيْهِمْ ۚ وَأَنَا ٱلتَّوَّابُ ٱلرَّحِيمُ ﴿١٦٠﴾\\
\textamh{161.\ በእዉነት ለማይምኑት፥ በክህደታቸው ለሚሞቱት፥ እነሱ ናቸው የኣላህ፥ የመላኢክት እና የሰው ልጆች አንድ ላይ እርግማን ያለባቸው።   } &   إِنَّ ٱلَّذِينَ كَفَرُوا۟ وَمَاتُوا۟ وَهُمْ كُفَّارٌ أُو۟لَٟٓئِكَ عَلَيْهِمْ لَعْنَةُ ٱللَّهِ وَٱلْمَلَٟٓئِكَةِ وَٱلنَّاسِ أَجْمَعِينَ ﴿١٦١﴾\\
\textamh{162.\ እዚያ ዉስጥ (በእርግማኑ ጀሀነም ዉስጥ) ይኖራሉ፥ ቅጣቸው አይቃለልም፥ ወይንም አፍታ አይስጣቸዉም   } &  خَـٰلِدِينَ فِيهَا ۖ لَا يُخَفَّفُ عَنْهُمُ ٱلْعَذَابُ وَلَا هُمْ يُنظَرُونَ ﴿١٦٢﴾\\
\textamh{163.\ አምላካችሁ አንድ አምላክ ነው፥ ላ ኢለሀ ኢለ ሁዋ (ከሱ ሌላ መመለክ የሚገባው ሌላ አምላክ የለም)፥ ከሁሉም በላይ ሰጪዉ፥ ከሁሉም በላይ ምህረተኛው   } &  وَإِلَٟهُكُمْ إِلَٟهٌۭ وَٟحِدٌۭ ۖ لَّآ إِلَٟهَ إِلَّا هُوَ ٱلرَّحْمَـٰنُ ٱلرَّحِيمُ ﴿١٦٣﴾\\
\textamh{164.\ በእዉነት በሰማይና ምድር አፈጣጠር፥ በቀንና ለሊት መፈራረቅ፥ እና በመርከቦች ባህር አቋርጠው በሚጓዙት ለሰዎች ጥቅም፥ እና ከሰማይ ኣላህ በሚያወርደው ዝናብ እና መሬቱን (ምድሩን) ከሞተበት እንደገና ወደ ህይወት በሚሰጠው፥ እና በሚንቀሳቀሱ (በየቦታው) የተዘሩ (ያሉ) ሁሉም አይነት ፍጥረታት፥ በንፋስና በደመና በሰማይና መሬት የተያዘ እንቅስቃሴ በእውነት ለሚያስቡ (ሰዎች) አያት (ምልክት...) ናቸው ።   } &  إِنَّ فِى خَلْقِ ٱلسَّمَـٰوَٟتِ وَٱلْأَرْضِ وَٱخْتِلَٟفِ ٱلَّيْلِ وَٱلنَّهَارِ وَٱلْفُلْكِ ٱلَّتِى تَجْرِى فِى ٱلْبَحْرِ بِمَا يَنفَعُ ٱلنَّاسَ وَمَآ أَنزَلَ ٱللَّهُ مِنَ ٱلسَّمَآءِ مِن مَّآءٍۢ فَأَحْيَا بِهِ ٱلْأَرْضَ بَعْدَ مَوْتِهَا وَبَثَّ فِيهَا مِن كُلِّ دَآبَّةٍۢ وَتَصْرِيفِ ٱلرِّيَـٰحِ وَٱلسَّحَابِ ٱلْمُسَخَّرِ بَيْنَ ٱلسَّمَآءِ وَٱلْأَرْضِ لَءَايَـٰتٍۢ لِّقَوْمٍۢ يَعْقِلُونَ ﴿١٦٤﴾\\
\textamh{165.\ ከሰው ልጆች መካከል ከኣላህ ሌላ (የኣላህ) ተወዳዳሪ አርገው የሚወስዱ አሉ። ኣላህን እንደሚወዱት ይወዷቸዋል ነገር ግን አማኞች፥ ኣላህን (ከማንም) የበለጠ ይወዱታል። ስህተት ሰሪዎች ቢያዩ ኑሮ፥ ቅጣቱን ሲዩ፥ ሁሉም ሀይል የኣላህ እንደሆነ እና ኣላህ በቅጣቱ ከባድ ነው።   } &  وَمِنَ ٱلنَّاسِ مَن يَتَّخِذُ مِن دُونِ ٱللَّهِ أَندَادًۭا يُحِبُّونَهُمْ كَحُبِّ ٱللَّهِ ۖ وَٱلَّذِينَ ءَامَنُوٓا۟ أَشَدُّ حُبًّۭا لِّلَّهِ ۗ وَلَوْ يَرَى ٱلَّذِينَ ظَلَمُوٓا۟ إِذْ يَرَوْنَ ٱلْعَذَابَ أَنَّ ٱلْقُوَّةَ لِلَّهِ جَمِيعًۭا وَأَنَّ ٱللَّهَ شَدِيدُ ٱلْعَذَابِ ﴿١٦٥﴾\\
\textamh{166.\ ያስከተሉት የተከተሏቸዉን ሲክዱ፥ ቅጣቱን (ስቃዩን) ሲያ፥ ሁሉም ግንኙነታቸው ይቆረጥባቸዋል   } &  إِذْ تَبَرَّأَ ٱلَّذِينَ ٱتُّبِعُوا۟ مِنَ ٱلَّذِينَ ٱتَّبَعُوا۟ وَرَأَوُا۟ ٱلْعَذَابَ وَتَقَطَّعَتْ بِهِمُ ٱلْأَسْبَابُ ﴿١٦٦﴾\\
\textamh{167.\ ተከታዮቹ (እንዲህ) ይላሉ: \rq\rq{}አንድ እድል ብቻ ቢኖረን ለመመለስ (ወደአለም)፥ እነሱን እንክዳቸዋል፥ እኛን እንደካዱን።\rq\rq{} ስለዚህ ኣላህ ስራቸዉን ቁጭት አድርጎ ያሳያቸዋል። ከእሳቱ በፍጹም አይወጡም።   } &   وَقَالَ ٱلَّذِينَ ٱتَّبَعُوا۟ لَوْ أَنَّ لَنَا كَرَّةًۭ فَنَتَبَرَّأَ مِنْهُمْ كَمَا تَبَرَّءُوا۟ مِنَّا ۗ كَذَٟلِكَ يُرِيهِمُ ٱللَّهُ أَعْمَـٰلَهُمْ حَسَرَٟتٍ عَلَيْهِمْ ۖ وَمَا هُم بِخَـٰرِجِينَ مِنَ ٱلنَّارِ ﴿١٦٧﴾\\
\textamh{168.\ ኦ የሰው ልጆች፥ ህጋዊ (ሃላል) እና ጥሩ የሆነዉን ብሉ፥ የሰይጣንን ኮቴ አትከተሉ። በእዉነት፥ እሱ ለእናንተ ግልጽ የሆነ ጠላታቹህ ነው   } &  يَـٰٓأَيُّهَا ٱلنَّاسُ كُلُوا۟ مِمَّا فِى ٱلْأَرْضِ حَلَٟلًۭا طَيِّبًۭا وَلَا تَتَّبِعُوا۟ خُطُوَٟتِ ٱلشَّيْطَٟنِ ۚ إِنَّهُۥ لَكُمْ عَدُوٌّۭ مُّبِينٌ ﴿١٦٨﴾\\
\textamh{169.\ (ሸይጣን) ክፋትና ፋህሻ (ሐጢያት) የሆነ ነገር ያዛችኋል፥ እና ስለኣላህ የማታውቁትን እንድትሉ   } &   إِنَّمَا يَأْمُرُكُم بِٱلسُّوٓءِ وَٱلْفَحْشَآءِ وَأَن تَقُولُوا۟ عَلَى ٱللَّهِ مَا لَا تَعْلَمُونَ ﴿١٦٩﴾\\
\textamh{170.\ (እንዲህ) ሲባሉ: \rq\rq{}ኣላህ ያወረደዉን ተከተሉ\rq\rq{}፤ አሉ: \rq\rq{}የለም! አባቶቻችን ሲከተሉት ያገኘናቸዉን ነው የምንከተል።\rq\rq{} ምንም እንኳ አባቶቻቸው ምንም ነገር ሳይገባቸው እና ሳይመሩ የቀሩ ሆነው ሳል?   } &  وَإِذَا قِيلَ لَهُمُ ٱتَّبِعُوا۟ مَآ أَنزَلَ ٱللَّهُ قَالُوا۟ بَلْ نَتَّبِعُ مَآ أَلْفَيْنَا عَلَيْهِ ءَابَآءَنَآ ۗ أَوَلَوْ كَانَ ءَابَآؤُهُمْ لَا يَعْقِلُونَ شَيْـًۭٔا وَلَا يَهْتَدُونَ ﴿١٧٠﴾\\
\textamh{171.\ የማያምኑት ምሳሌ አንድ ሰው (ወደበጎች) እንደሚጮህ አይነት ነገር ነው ምንም የማይሰሙ ከጩሀትና ከዋይታ (ለቅሶ) በስተቀር። ደንቆሮ፥ ዲዳ፥ እና እዉር ናቸው። ስለዚህ አይገባቸዉም።   } &   وَمَثَلُ ٱلَّذِينَ كَفَرُوا۟ كَمَثَلِ ٱلَّذِى يَنْعِقُ بِمَا لَا يَسْمَعُ إِلَّا دُعَآءًۭ وَنِدَآءًۭ ۚ صُمٌّۢ بُكْمٌ عُمْىٌۭ فَهُمْ لَا يَعْقِلُونَ ﴿١٧١﴾\\
\textamh{172.\ ኦ እናንት አማኞች፥ ህጋዊ የሆኑትን (ሀላል) የሰጠናችሁን ነገሮች ብሉ፥ እና ኣላህን አመስግኑ፥ በእዉነት እሱን ከሆነ የምታመልኩት   } &   يَـٰٓأَيُّهَا ٱلَّذِينَ ءَامَنُوا۟ كُلُوا۟ مِن طَيِّبَٟتِ مَا رَزَقْنَـٰكُمْ وَٱشْكُرُوا۟ لِلَّهِ إِن كُنتُمْ إِيَّاهُ تَعْبُدُونَ ﴿١٧٢﴾\\
\textamh{173.\ የሞተ ነገር፥ ደም፥ የአስማ ስጋ፥ ከኣላህ ለሌሎች የታረደ (ለጣኦት፥ በሌላ ስም) ከልክሏችኋል። ነገር ግን በችግር ምክንያት ቢገደድ ያላ ፈቀደዊ አለመታዘዝ ወይንም ሳይተላለፍ፥ እዚያ ላይ ሐጢያት የለበትም። በእዉነት ኣላህ ሁሌ-ይቅር ባይ፥ ከሁሉም በላይ ምህረተኛ ነው።    } &   إِنَّمَا حَرَّمَ عَلَيْكُمُ ٱلْمَيْتَةَ وَٱلدَّمَ وَلَحْمَ ٱلْخِنزِيرِ وَمَآ أُهِلَّ بِهِۦ لِغَيْرِ ٱللَّهِ ۖ فَمَنِ ٱضْطُرَّ غَيْرَ بَاغٍۢ وَلَا عَادٍۢ فَلَآ إِثْمَ عَلَيْهِ ۚ إِنَّ ٱللَّهَ غَفُورٌۭ رَّحِيمٌ ﴿١٧٣﴾\\
\textamh{174.\ በእዉነት፥ እዉነቱን ኣላህ ያወረደዉን መጽሐፍ የሚደብቁ እና የማይረባ ነገር ለሚሸምቱ (አለማዊ)፥ ወደሆዳቸው ዉስጥ ሌላ ሳይሆን እሳት ነው የሚበሉት። ኣላህ የትንሰኤ ቀን አያናግራቸዉም፥ ወይንም አያጸዳቸዉም፥ እና ለነሱ አሰቀቂ ስቃይ የተሞላበት ቅጣት ይሆናል።   } &  إِنَّ ٱلَّذِينَ يَكْتُمُونَ مَآ أَنزَلَ ٱللَّهُ مِنَ ٱلْكِتَـٰبِ وَيَشْتَرُونَ بِهِۦ ثَمَنًۭا قَلِيلًا ۙ أُو۟لَٟٓئِكَ مَا يَأْكُلُونَ فِى بُطُونِهِمْ إِلَّا ٱلنَّارَ وَلَا يُكَلِّمُهُمُ ٱللَّهُ يَوْمَ ٱلْقِيَـٰمَةِ وَلَا يُزَكِّيهِمْ وَلَهُمْ عَذَابٌ أَلِيمٌ ﴿١٧٤﴾\\
\textamh{175.\ እነዚህ ናቸው ስህተትን በመመራት የገዙ፥ ቅጣትን በይቅር መባል ወጋ። ምን ያህል ቢሆን ነው ድፍረታቸው ወደ እሳቱ (ለመገባት)።   } &   أُو۟لَٟٓئِكَ ٱلَّذِينَ ٱشْتَرَوُا۟ ٱلضَّلَٟلَةَ بِٱلْهُدَىٰ وَٱلْعَذَابَ بِٱلْمَغْفِرَةِ ۚ فَمَآ أَصْبَرَهُمْ عَلَى ٱلنَّارِ ﴿١٧٥﴾\\
\textamh{176.\ ይሄም ኣላህ መጽሐፉን በሀቅ (በእዉነት) ስለአወረደው ነው። እና በእዉነት ስለመጽሐፉ የሚከራከሩ በመቃረን ሩቅ ሄደዋል።   } &  ذَٟلِكَ بِأَنَّ ٱللَّهَ نَزَّلَ ٱلْكِتَـٰبَ بِٱلْحَقِّ ۗ وَإِنَّ ٱلَّذِينَ ٱخْتَلَفُوا۟ فِى ٱلْكِتَـٰبِ لَفِى شِقَاقٍۭ بَعِيدٍۢ ﴿١٧٦﴾ ۞\\
\textamh{177.\ ወደ ምስራቅ ወይም ወደ ምእራብ መዞር (ለመጸለይ) ጽድቅ ስራ አይደለም፤ ነገር ጽድቅ ስራ በኣላህ፥ በመጨረሻው ቀን፥ በመላኢክት፥ በመጸህፉ፥ በነቢያቱ ማመን እና ሃብትን፥ ምንም እንኳ (ሀብትን) ቢወዱ፥ ለዘመድ፥ ለወላጅ አልባው፥ ለድሆች፥ ለመንገደኛው፥ እና ለሚጠይቁት መስጠት፥ እና ባሪያዎችን ነፃ መልቀቅ፥ ሳላት መቆም፥ ዘካት መስጠት፥ እና ዉልን (ቃል ኪዳንን) መጠበቅ፥ እና በታላቅ ረሃብ እና በሽታ እና በዉጊያ (ጦርነት) ጊዜ ታጋሾች መሆን። እነዚህ ናቸው ለእዉነት የቆሙ ሰዎች እና ሙታቁን የሆኑ (ፈሪሃአላህ ያላቸው)   } &    لَّيْسَ ٱلْبِرَّ أَن تُوَلُّوا۟ وُجُوهَكُمْ قِبَلَ ٱلْمَشْرِقِ وَٱلْمَغْرِبِ وَلَٟكِنَّ ٱلْبِرَّ مَنْ ءَامَنَ بِٱللَّهِ وَٱلْيَوْمِ ٱلْءَاخِرِ وَٱلْمَلَٟٓئِكَةِ وَٱلْكِتَـٰبِ وَٱلنَّبِيِّۦنَ وَءَاتَى ٱلْمَالَ عَلَىٰ حُبِّهِۦ ذَوِى ٱلْقُرْبَىٰ وَٱلْيَتَـٰمَىٰ وَٱلْمَسَٟكِينَ وَٱبْنَ ٱلسَّبِيلِ وَٱلسَّآئِلِينَ وَفِى ٱلرِّقَابِ وَأَقَامَ ٱلصَّلَوٰةَ وَءَاتَى ٱلزَّكَوٰةَ وَٱلْمُوفُونَ بِعَهْدِهِمْ إِذَا عَٟهَدُوا۟ ۖ وَٱلصَّٟبِرِينَ فِى ٱلْبَأْسَآءِ وَٱلضَّرَّآءِ وَحِينَ ٱلْبَأْسِ ۗ أُو۟لَٟٓئِكَ ٱلَّذِينَ صَدَقُوا۟ ۖ وَأُو۟لَٟٓئِكَ هُمُ ٱلْمُتَّقُونَ ﴿١٧٧﴾\\
\textamh{178.\ ኦ እናንት አማኞች! አል-ቂሳስ (እኩል የካሳ ግድያ) በነፍስ ግድያ ጊዜ ታዝዞላችኋል: ነጻው ሰው በነጻው ሰው፥ ባሪያው በባሪያ፥ እና ሴቷ በሴት። ነገር ግን ገዳዩ በተገደለው ወንድም በደም ካሳ ገንዘብ ይቅር ከተባለ፥ ከዚያ አስከትሎ (ጥሩ ስራና)አግባብ ባላው መልኩና (በገንዘቡ ክፍያ)፥ (ለይቅር ባዩ) አግባብ ያለው ነገር መደረግ አለበት። ይሄ ከአምላካችሁ ለእናንተ እፎይታና ምህረት ነው። ከዚህ በኋላ ልኩን የሚያልፍ፥ ለሱ ታላቅ ቅጣት አለው።   } &  يَـٰٓأَيُّهَا ٱلَّذِينَ ءَامَنُوا۟ كُتِبَ عَلَيْكُمُ ٱلْقِصَاصُ فِى ٱلْقَتْلَى ۖ ٱلْحُرُّ بِٱلْحُرِّ وَٱلْعَبْدُ بِٱلْعَبْدِ وَٱلْأُنثَىٰ بِٱلْأُنثَىٰ ۚ فَمَنْ عُفِىَ لَهُۥ مِنْ أَخِيهِ شَىْءٌۭ فَٱتِّبَاعٌۢ بِٱلْمَعْرُوفِ وَأَدَآءٌ إِلَيْهِ بِإِحْسَٟنٍۢ ۗ ذَٟلِكَ تَخْفِيفٌۭ مِّن رَّبِّكُمْ وَرَحْمَةٌۭ ۗ فَمَنِ ٱعْتَدَىٰ بَعْدَ ذَٟلِكَ فَلَهُۥ عَذَابٌ أَلِيمٌۭ ﴿١٧٨﴾\\
\textamh{179.\ በአል-ቂሳስ (ካሳ ቅጣት) ህይወት ለናንተ አለ፥ ኦ አቅል ያላችሁ ሰዎች (የምታስቡ)፥ በዚያም ሙታቁን (ፈሪሃ-ኣላህ ያላችሁ) ትሆናላችሁ   } &  وَلَكُمْ فِى ٱلْقِصَاصِ حَيَوٰةٌۭ يَـٰٓأُو۟لِى ٱلْأَلْبَٟبِ لَعَلَّكُمْ تَتَّقُونَ ﴿١٧٩﴾\\
\textamh{180.\ ተዝዞላችኋል፥ ማናችሁን ሞት ቢቀርባችሁ፥ ሀብቱን ቢተው፥ ለወላጆቹና ቤተሰቦቹ ኑዛዜ አግባብ ባለው መልኩ ይተው። ይሄ ሙታቁን ላይ ሀላፊነት ነው።   } &  كُتِبَ عَلَيْكُمْ إِذَا حَضَرَ أَحَدَكُمُ ٱلْمَوْتُ إِن تَرَكَ خَيْرًا ٱلْوَصِيَّةُ لِلْوَٟلِدَيْنِ وَٱلْأَقْرَبِينَ بِٱلْمَعْرُوفِ ۖ حَقًّا عَلَى ٱلْمُتَّقِينَ ﴿١٨٠﴾\\
\textamh{181.\ ከዚያም ማንም ኑዛዜዉን ከሰማ በኋላ ቢቀይር፥ ሀጢያቱ ከሚቀይሩት ላይ ይሆናል። በእዉነት፥ ኣላህ ሁሉን-ሰሚ ሁሉን-አወቂ ነው።   } &  فَمَنۢ بَدَّلَهُۥ بَعْدَمَا سَمِعَهُۥ فَإِنَّمَآ إِثْمُهُۥ عَلَى ٱلَّذِينَ يُبَدِّلُونَهُۥٓ ۚ إِنَّ ٱللَّهَ سَمِيعٌ عَلِيمٌۭ ﴿١٨١﴾\\
\textamh{182.\ ነገር ግን አንድ ሰው ጠማማ ወይም መጥፎ ነገር ከተናዛዡ ቢፈራ፥ እናም በዚያ (በመካከላቸው) ሰላም አምጥቶ ቢያስታርቅ፥ ሀጢያት አይኖርበትም። በእርግጠኛነት፥ ኣላህ ሁሌ-ይቅር ባይ፥ ከሁሉ በላይ ምህርተኛ ነው።   } &  فَمَنْ خَافَ مِن مُّوصٍۢ جَنَفًا أَوْ إِثْمًۭا فَأَصْلَحَ بَيْنَهُمْ فَلَآ إِثْمَ عَلَيْهِ ۚ إِنَّ ٱللَّهَ غَفُورٌۭ رَّحِيمٌۭ ﴿١٨٢﴾\\
\textamh{183.\ ኦ እናንት አማኞች፥ መጾም ተዝዞላችኋል ከናንተ በፊት እንደታዘዘላቸው፥ ሙታቁን እንድትሆኑ።   } &   يَـٰٓأَيُّهَا ٱلَّذِينَ ءَامَنُوا۟ كُتِبَ عَلَيْكُمُ ٱلصِّيَامُ كَمَا كُتِبَ عَلَى ٱلَّذِينَ مِن قَبْلِكُمْ لَعَلَّكُمْ تَتَّقُونَ ﴿١٨٣﴾\\
\textamh{184.\ (በጊዜ) ለተወሰነ(ኑ) ቀናት (አንድ ወር)፥ ነገር ግን ማናችሁም የታመመ ቢሆን ወይንም መንገድ ላይ ቢሆን፥ በቁጥር እኩል ቀናት (መጾም) በሌላ ጊዜ። ጾም እየጾሙ ለሚከብድብቸው (ምሳሌ: ሽማግሌ..)፥ ድሆችን የማብላት (አማራጭ) አላቸው። ነገር ግን ማንም ከራሱ ፈቃድ ጥሩ ቢሰራ፥ ለሱ ይሻለዋል። እናም ብትጾሙ፥ ለእናንተ ይሻላል፥ ብታውቁት።    } &  أَيَّامًۭا مَّعْدُودَٟتٍۢ ۚ فَمَن كَانَ مِنكُم مَّرِيضًا أَوْ عَلَىٰ سَفَرٍۢ فَعِدَّةٌۭ مِّنْ أَيَّامٍ أُخَرَ ۚ وَعَلَى ٱلَّذِينَ يُطِيقُونَهُۥ فِدْيَةٌۭ طَعَامُ مِسْكِينٍۢ ۖ فَمَن تَطَوَّعَ خَيْرًۭا فَهُوَ خَيْرٌۭ لَّهُۥ ۚ وَأَن تَصُومُوا۟ خَيْرٌۭ لَّكُمْ ۖ إِن كُنتُمْ تَعْلَمُونَ ﴿١٨٤﴾\\
\textamh{185.\ የረመዳን ወር ቁርአን የተገለጸበት፥ ለሰው ልጆች መመሪያ እና ግልጽ መረጋገጫ ለመመሪያና መፍረጃ (ትክክሉን ከ ስህተት)። ስለዚህ ማንም (ጨረቃ) በዚያ ወር (በመጀመሪያው ቀን) ካየ፥ ጾሙን መጠበቅ (መጀመር) በዚያ ወር አለበት፥ እና ማንም ቢታመም ወይንም መንገድ ጉዞ ላይ ካለ፥ ተመሳሳይ ቀናት በሌላ ጊዜ መጾም አለበት። ኣላህ እንዲቀልላቸሁ ያሰባል፥ እንዲከብድባችሁ አይፈልግም፥ እና ኣላህን እንድታከብሩት (አላሁ-አክበር ጨረቃ ባያችሁ ጊዜ) ስለመራችሁ እንድታመሰግኑት።    } &  شَهْرُ رَمَضَانَ ٱلَّذِىٓ أُنزِلَ فِيهِ ٱلْقُرْءَانُ هُدًۭى لِّلنَّاسِ وَبَيِّنَـٰتٍۢ مِّنَ ٱلْهُدَىٰ وَٱلْفُرْقَانِ ۚ فَمَن شَهِدَ مِنكُمُ ٱلشَّهْرَ فَلْيَصُمْهُ ۖ وَمَن كَانَ مَرِيضًا أَوْ عَلَىٰ سَفَرٍۢ فَعِدَّةٌۭ مِّنْ أَيَّامٍ أُخَرَ ۗ يُرِيدُ ٱللَّهُ بِكُمُ ٱلْيُسْرَ وَلَا يُرِيدُ بِكُمُ ٱلْعُسْرَ وَلِتُكْمِلُوا۟ ٱلْعِدَّةَ وَلِتُكَبِّرُوا۟ ٱللَّهَ عَلَىٰ مَا هَدَىٰكُمْ وَلَعَلَّكُمْ تَشْكُرُونَ ﴿١٨٥﴾\\
\textamh{186.\ ባሪያዎቼ ስለኔ ሲጠይቁህ (ኦ! ሙሐመድ(ሠአወሰ))፥ እኔ (ለነሱ) በጣም ቅርብ ነኝ። ድዋውዉን ለሚያደረገው እኔን ሲጠራ (ያለምንም አማካይ ወይም አማላጅ) እመልስልታለሁ። ስለዚህ ለእኔ ይገዙ እና ይመኑ፥ በትክክል (ወደቀኝ) እንዲመሩ።   } &  وَإِذَا سَأَلَكَ عِبَادِى عَنِّى فَإِنِّى قَرِيبٌ ۖ أُجِيبُ دَعْوَةَ ٱلدَّاعِ إِذَا دَعَانِ ۖ فَلْيَسْتَجِيبُوا۟ لِى وَلْيُؤْمِنُوا۟ بِى لَعَلَّهُمْ يَرْشُدُونَ ﴿١٨٦﴾\\
\textamh{187.\ ከሚስቶቻችሁ ጋር በጾሙ ለሊት ግንኙነት ተፈቅዶላችኋል። እነሱ የእናንተ ልባስ ናቻው፥ እናንተም የነሱ ልባስ ናችሁ። ኣላህ ራሳችሁን ታታሉ እንደነበር ያውቃል፥ ስለዚህ ወደእናንተ ፊቱን አዞረና ይቅር አላችሁ። ስለዚህ ከነሱ ጋር ግንኙነት አድርጉ እና ኣላህ ያዘዘላችሁን ነገር ፈልጉ (ልጆች)፥ እና ብሉ፥ጠጡ የማለዳ ወገግታ ከጨለማው እስኪጀምር ድረስ፥ ከዚያም ጾማችሁን እስከምሽት ድረስ ጨርሱ። ኢቲካፍ ላይ መስጂድ ዉስጥ ሁናችሁ ግን ከነሱ ጋር ግንኙነት አታድርጉ። ይሄ የኣላህ ድንበር ነው፥ ስለዚህ አትቅረቧቸው። ለዚህም ኣላህ አያቱን (ጥቅሶቹን፥ ምልክቶቹን) ግልጽ ለሰው ልጆች ያደርጋል በዚያ ሙታቁን እንዲሆኑ።   } &  أُحِلَّ لَكُمْ لَيْلَةَ ٱلصِّيَامِ ٱلرَّفَثُ إِلَىٰ نِسَآئِكُمْ ۚ هُنَّ لِبَاسٌۭ لَّكُمْ وَأَنتُمْ لِبَاسٌۭ لَّهُنَّ ۗ عَلِمَ ٱللَّهُ أَنَّكُمْ كُنتُمْ تَخْتَانُونَ أَنفُسَكُمْ فَتَابَ عَلَيْكُمْ وَعَفَا عَنكُمْ ۖ فَٱلْـَٟٔنَ بَٟشِرُوهُنَّ وَٱبْتَغُوا۟ مَا كَتَبَ ٱللَّهُ لَكُمْ ۚ وَكُلُوا۟ وَٱشْرَبُوا۟ حَتَّىٰ يَتَبَيَّنَ لَكُمُ ٱلْخَيْطُ ٱلْأَبْيَضُ مِنَ ٱلْخَيْطِ ٱلْأَسْوَدِ مِنَ ٱلْفَجْرِ ۖ ثُمَّ أَتِمُّوا۟ ٱلصِّيَامَ إِلَى ٱلَّيْلِ ۚ وَلَا تُبَٟشِرُوهُنَّ وَأَنتُمْ عَٟكِفُونَ فِى ٱلْمَسَٟجِدِ ۗ تِلْكَ حُدُودُ ٱللَّهِ فَلَا تَقْرَبُوهَا ۗ كَذَٟلِكَ يُبَيِّنُ ٱللَّهُ ءَايَـٰتِهِۦ لِلنَّاسِ لَعَلَّهُمْ يَتَّقُونَ ﴿١٨٧﴾\\
\textamh{188.\ ንብረታችሁን በሀሰት (በማታለል፥ በስርቆት) አትብሉ (አታክስሩ)፥ ወይንም ግቦ ለገዢዎች አትስጡ የሌሎችን ንብረት በሀጢያት እያወቃችሁ ለመብላት ስትሉ።   } &  وَلَا تَأْكُلُوٓا۟ أَمْوَٟلَكُم بَيْنَكُم بِٱلْبَٟطِلِ وَتُدْلُوا۟ بِهَآ إِلَى ٱلْحُكَّامِ لِتَأْكُلُوا۟ فَرِيقًۭا مِّنْ أَمْوَٟلِ ٱلنَّاسِ بِٱلْإِثْمِ وَأَنتُمْ تَعْلَمُونَ ﴿١٨٨﴾ ۞\\
\textamh{189.\ ስለጨረቃ ዉልደት ይጠይቁሀል (ኦ ሙሐመድ(ሠአወሰ)) (እንዲህ) በል: \rq\rq{}እነዚህ ለሰዎችና ለመንፈሳዊ ተጓዦች ወሰን ያለዉን ጊዜ ማመላከቻ ምልክቶች ናቸው።\rq\rq{} ቤቶችን በጀርባቸው (በኋላቸው) መግባት ፅድቅ አይደለም ነገር ግን ፅድቅ ኣላህን የሚፈራ ነው። ስለዚህ ቤቶችን በትክክለኛ በሮቻቸው ግቡ፥ እና ኣላህን ፍሩ (በስኬት) አላፊ እንድትሆኑ።   } &   يَسْـَٔلُونَكَ عَنِ ٱلْأَهِلَّةِ ۖ قُلْ هِىَ مَوَٟقِيتُ لِلنَّاسِ وَٱلْحَجِّ ۗ وَلَيْسَ ٱلْبِرُّ بِأَن تَأْتُوا۟ ٱلْبُيُوتَ مِن ظُهُورِهَا وَلَٟكِنَّ ٱلْبِرَّ مَنِ ٱتَّقَىٰ ۗ وَأْتُوا۟ ٱلْبُيُوتَ مِنْ أَبْوَٟبِهَا ۚ وَٱتَّقُوا۟ ٱللَّهَ لَعَلَّكُمْ تُفْلِحُونَ ﴿١٨٩﴾\\
\textamh{190.\ በኣላህ መንገድ የሚወጓችሁን ተዋጓቸው ነገር ግን ልክ አትለፉ። በእዉነት ኣላህ ልክ የሚያልፉትን አይወድም።    } &  وَقَٟتِلُوا۟ فِى سَبِيلِ ٱللَّهِ ٱلَّذِينَ يُقَٟتِلُونَكُمْ وَلَا تَعْتَدُوٓا۟ ۚ إِنَّ ٱللَّهَ لَا يُحِبُّ ٱلْمُعْتَدِينَ ﴿١٩٠﴾\\
\textamh{191.\ እናም ካገኛቹኋቸው ቦታ ሁሉ ግደሏቸው፥ ከስወጧችሁ ቦታ አስወጧቸው፤ አል-ፊትና (ፈተና ማምጣት) ከግድያ ይከብዳል። ከአል-መስጂድ-አል-ሀራም ላይ አትዋጓቸው፥ እናንተን (መጀመሪያ) ካልተዋጓቹህ። ነገር ግን እዛ ቢዋጓችሁ፥ ግደሏቸው። ይሄ ነው የከሀዲዎች ክፍያ።   } &  وَٱقْتُلُوهُمْ حَيْثُ ثَقِفْتُمُوهُمْ وَأَخْرِجُوهُم مِّنْ حَيْثُ أَخْرَجُوكُمْ ۚ وَٱلْفِتْنَةُ أَشَدُّ مِنَ ٱلْقَتْلِ ۚ وَلَا تُقَٟتِلُوهُمْ عِندَ ٱلْمَسْجِدِ ٱلْحَرَامِ حَتَّىٰ يُقَٟتِلُوكُمْ فِيهِ ۖ فَإِن قَٟتَلُوكُمْ فَٱقْتُلُوهُمْ ۗ كَذَٟلِكَ جَزَآءُ ٱلْكَٟفِرِينَ ﴿١٩١﴾\\
\textamh{192.\ ነገር ግን ቢያቆሙ፥ ኣላህ ብዙ-ጊዜ ይቅር ባይ፥ ከሁሉም በላይ ምህረተኛ ነው   } &   فَإِنِ ٱنتَهَوْا۟ فَإِنَّ ٱللَّهَ غَفُورٌۭ رَّحِيمٌۭ ﴿١٩٢﴾\\
\textamh{193.\ ፊትና (ፈትና ማምጣት) እስካይኖር ድረስ ተዋጓቸው እና ለኣላህ ብቻ ሁሉም አምልኮ እስኪሆን ድረስ። ነገር ግን ቢያቆሙ፥ ልክ መተላለፍ አይኑር ከዛሊሞች ላይ በስተቀር   } &  وَقَٟتِلُوهُمْ حَتَّىٰ لَا تَكُونَ فِتْنَةٌۭ وَيَكُونَ ٱلدِّينُ لِلَّهِ ۖ فَإِنِ ٱنتَهَوْا۟ فَلَا عُدْوَٟنَ إِلَّا عَلَى ٱلظَّٟلِمِينَ ﴿١٩٣﴾\\
\textamh{194.\ የተከበረው ወር ለተከበረው ወር ነው፥ እና ለተከለከሉ ነገሮች፥ የቂሳስ (የካሳ) ህግ አለ። ከዚያ ማንም ከእናንተ ላይ ከልክ ቢያልፍ፥ እናንተም እንደዚያው ልክ እንዳደረጋችሁ አድሩጉበት። እና ኣላህን ፍሩ፥ እና ኣላህ ከሙታቁን ጋር እንደሆነ እወቁ።   } &  ٱلشَّهْرُ ٱلْحَرَامُ بِٱلشَّهْرِ ٱلْحَرَامِ وَٱلْحُرُمَـٰتُ قِصَاصٌۭ ۚ فَمَنِ ٱعْتَدَىٰ عَلَيْكُمْ فَٱعْتَدُوا۟ عَلَيْهِ بِمِثْلِ مَا ٱعْتَدَىٰ عَلَيْكُمْ ۚ وَٱتَّقُوا۟ ٱللَّهَ وَٱعْلَمُوٓا۟ أَنَّ ٱللَّهَ مَعَ ٱلْمُتَّقِينَ ﴿١٩٤﴾\\
\textamh{195.\ በኣላህ መንገድ አውጡ እና ራሳችሁን ወደ መፍረስ አትወርውሩ እና ጥሩ ስሩ። በእዉነት፥ ኣላህ ጥሩ ሰሪዎችን (ሙህሲኑን) ይወዳል።   } &  وَأَنفِقُوا۟ فِى سَبِيلِ ٱللَّهِ وَلَا تُلْقُوا۟ بِأَيْدِيكُمْ إِلَى ٱلتَّهْلُكَةِ ۛ وَأَحْسِنُوٓا۟ ۛ إِنَّ ٱللَّهَ يُحِبُّ ٱلْمُحْسِنِينَ ﴿١٩٥﴾\\
\textamh{196.\ እና በትክክል ሀጅና ኡምራን ለኣላህ አድርጉ። ነገር ግን መድረግ ካልቻላችሁ፥ ሀድይ (እንስሳ: በግ፥ ከብት፥ ግመል)(መስዋት) ሠዉ ፥ እንደአቅማችሁ፥ እና ራሳችሁን ሀድይው መሰዊያው ቦታ እስኪደርስ ድረስ አትላጩ። እና ማናችሁም ቢታመም ወይንም ራሱ ላይ ቁስል ነገር ቢኖር (ለመላጨት ቢያስፈልገው)፥ፊድያ(ቤዛ) ይክፈል: (ሶስት ቀን) በመጾም ወይም ሰደቃ (ለስድስት ሰዎች በማብላት) ወይም የሚሰዋ ነገር (አንድ በግ) ያቅርብ። ከዚያም በሰላም ከሆናችሁ እና ማንም በሀጅ ወር ኡምራ ቢያደርግ፥ ሀጁን ከማድረጉ በፊት፥ ሀድይ መሰዋት (የአቅሙን ያህል) አለበት፥ ነገር ግን አቅሙ የማይፈቅድ ከሆነ፥ ሶስት ቀን በሀጅ ጊዜ መጾም ከተመለሰ በኋላ ደግሞ ሰባት ቀናት መጾም (ቤቱ)፥ ጠቅላላ አስር ቀናት። ይሄ ቤተሰቡ አል-መስጂድ-አል-ሀራም የለሌሉ ከሆነ ነው (የመካ ነዋሪ ካልሆኑ)። እና ኣላህን በጣም ፍሩ እናም እወቁ ኣላህ በቅጣቱ ከባድ መሆኑን።   } &  وَأَتِمُّوا۟ ٱلْحَجَّ وَٱلْعُمْرَةَ لِلَّهِ ۚ فَإِنْ أُحْصِرْتُمْ فَمَا ٱسْتَيْسَرَ مِنَ ٱلْهَدْىِ ۖ وَلَا تَحْلِقُوا۟ رُءُوسَكُمْ حَتَّىٰ يَبْلُغَ ٱلْهَدْىُ مَحِلَّهُۥ ۚ فَمَن كَانَ مِنكُم مَّرِيضًا أَوْ بِهِۦٓ أَذًۭى مِّن رَّأْسِهِۦ فَفِدْيَةٌۭ مِّن صِيَامٍ أَوْ صَدَقَةٍ أَوْ نُسُكٍۢ ۚ فَإِذَآ أَمِنتُمْ فَمَن تَمَتَّعَ بِٱلْعُمْرَةِ إِلَى ٱلْحَجِّ فَمَا ٱسْتَيْسَرَ مِنَ ٱلْهَدْىِ ۚ فَمَن لَّمْ يَجِدْ فَصِيَامُ ثَلَٟثَةِ أَيَّامٍۢ فِى ٱلْحَجِّ وَسَبْعَةٍ إِذَا رَجَعْتُمْ ۗ تِلْكَ عَشَرَةٌۭ كَامِلَةٌۭ ۗ ذَٟلِكَ لِمَن لَّمْ يَكُنْ أَهْلُهُۥ حَاضِرِى ٱلْمَسْجِدِ ٱلْحَرَامِ ۚ وَٱتَّقُوا۟ ٱللَّهَ وَٱعْلَمُوٓا۟ أَنَّ ٱللَّهَ شَدِيدُ ٱلْعِقَابِ ﴿١٩٦﴾\\
\textamh{197.\ ሀጅ በታወቁ ወራት ዉስጥ ነው (በእስልምና ዘመን አቆጣጠር 10ኛ ወር፥ 11ኛ ወር እና 12ኛው ወር በመጀመሪያዎቹ አስር ቀናት) ማንም ሀጅ ማድረግ ቢፈልግ በኢህራም ሁኖ፥ ግንኙነት ማድረግ የለበትም፥ ወይንም ሀጢያት መስራት፥ ወይንም መጨቃጨቅ በሀጅ ጊዜ የለበትም። እና ማናቸውም ጥሩ ነገር ብታደርጉ፥ ኣላህ ያዉቀዋል። ለመንገዳችሁ ስንቅ ያዙ፥ ነገር ግን ታላቁ ስንቅ ታቅዋ (ጽድቅ፥ጥሩ መስራት) ነው። ስለዚህ እኔን ፍሩኝ፥ ኦ አቅል ያላችሁ (የምታስቡ) ሰዎች!   } &  ٱلْحَجُّ أَشْهُرٌۭ مَّعْلُومَـٰتٌۭ ۚ فَمَن فَرَضَ فِيهِنَّ ٱلْحَجَّ فَلَا رَفَثَ وَلَا فُسُوقَ وَلَا جِدَالَ فِى ٱلْحَجِّ ۗ وَمَا تَفْعَلُوا۟ مِنْ خَيْرٍۢ يَعْلَمْهُ ٱللَّهُ ۗ وَتَزَوَّدُوا۟ فَإِنَّ خَيْرَ ٱلزَّادِ ٱلتَّقْوَىٰ ۚ وَٱتَّقُونِ يَـٰٓأُو۟لِى ٱلْأَلْبَٟبِ ﴿١٩٧﴾\\
\textamh{198.\ ከአምላካችሁ በረከት መፈለግ (በመንፈሳዊው ጉዞ ላይ) ሀጢያት የለባችሁም። ከዚያ አረፋት ስትለቁ፥ ኣላህን ከመሻር-ኢል-ሀራም አስታውሱ። እና አስተዉሱት ስለመራችሁ፥ እና በእዉነት፥ በፊት፥ ከሳቱት መካከል ነበራችሁ።    } &   لَيْسَ عَلَيْكُمْ جُنَاحٌ أَن تَبْتَغُوا۟ فَضْلًۭا مِّن رَّبِّكُمْ ۚ فَإِذَآ أَفَضْتُم مِّنْ عَرَفَٟتٍۢ فَٱذْكُرُوا۟ ٱللَّهَ عِندَ ٱلْمَشْعَرِ ٱلْحَرَامِ ۖ وَٱذْكُرُوهُ كَمَا هَدَىٰكُمْ وَإِن كُنتُم مِّن قَبْلِهِۦ لَمِنَ ٱلضَّآلِّينَ ﴿١٩٨﴾\\
\textamh{199.\ ከዚያም ሰዎች ሲሄዱ ከቦታው (አብራችሁ) ተነሱ እና አላህን ይቅርታዉን ጠይቁ። በእዉነት ኣላህ ሁሌ-ይቅር ባይ፥ ከሁሉም በላይ ምህርተኛ ነው።   } &   ثُمَّ أَفِيضُوا۟ مِنْ حَيْثُ أَفَاضَ ٱلنَّاسُ وَٱسْتَغْفِرُوا۟ ٱللَّهَ ۚ إِنَّ ٱللَّهَ غَفُورٌۭ رَّحِيمٌۭ ﴿١٩٩﴾\\
\textamh{200.\ ማናሲኩን እንደጨረሳችሁ፥(አረፋት ላይ ሁኑ፥ ሙዝዳሊፋ እና ሚና፥ የጀማራት ራምይ ሀድይዉን እየስዋችሁ።) ኣላህን አስታውሱ ልክ አያቶቻችሁን (ቅደመ አያቶቻችሁን) እንደምታስታውሱት ከዚያም የበለጠ ማስታወስ። ከሰው ልጆች መካከል እንዲህ የሚሉ አሉ: \rq\rq{}አምላካችን! ከዚህ አለም ስጠን!\rq\rq{} እና ለነዚህ ከሚመጣው አለም ድርሻ የላቸዉም።   } &  فَإِذَا قَضَيْتُم مَّنَـٰسِكَكُمْ فَٱذْكُرُوا۟ ٱللَّهَ كَذِكْرِكُمْ ءَابَآءَكُمْ أَوْ أَشَدَّ ذِكْرًۭا ۗ فَمِنَ ٱلنَّاسِ مَن يَقُولُ رَبَّنَآ ءَاتِنَا فِى ٱلدُّنْيَا وَمَا لَهُۥ فِى ٱلْءَاخِرَةِ مِنْ خَلَٟقٍۢ ﴿٢٠٠﴾\\
\textamh{201.\ እናም ከነሱ ዉስጥ እንዲህ የሚሉ አሉ: \rq\rq{}አምላካችን! ጥሩ የሆነ ነገር እዚህ አለም ዉስጥ ስጠን እና ከሚመጣው አለም (አኪራ) ጥሩ የሆነ ነገር፥ እና ከእሳቱ ስቃይ አድነን!\rq\rq{}   } &  وَمِنْهُم مَّن يَقُولُ رَبَّنَآ ءَاتِنَا فِى ٱلدُّنْيَا حَسَنَةًۭ وَفِى ٱلْءَاخِرَةِ حَسَنَةًۭ وَقِنَا عَذَابَ ٱلنَّارِ ﴿٢٠١﴾\\
\textamh{202.\ ለነዚህ ለአገኙት ተከፍሎ ድርሻ ይሰጣቸዋል። ኣላህ ሂሳብ በመስጠት ፈጣን ነው (በፍርዱ ፈጣን ነው)   } &  أُو۟لَٟٓئِكَ لَهُمْ نَصِيبٌۭ مِّمَّا كَسَبُوا۟ ۚ وَٱللَّهُ سَرِيعُ ٱلْحِسَابِ ﴿٢٠٢﴾ ۞\\
\textamh{203.\ እና በተወሰኑት ቀናት ኣላህን አስታውሱ። ነገር ግን ማንም በሁለት ቀን ለመሄድ ከፈለገ፥ ሀጢያት የለበትም እና ማንም ቢቆይ፥ እሱም ላይ ሀጢያት የለበትም፥ ሀሳቡ ጥሩ ለመስራትና ኣላህን ለመታዘዝ ከሆነ፥ እናም እወቁ በእርግጠኝነት ወደእሱ ትሰበሰባላችሁ።   } &   وَٱذْكُرُوا۟ ٱللَّهَ فِىٓ أَيَّامٍۢ مَّعْدُودَٟتٍۢ ۚ فَمَن تَعَجَّلَ فِى يَوْمَيْنِ فَلَآ إِثْمَ عَلَيْهِ وَمَن تَأَخَّرَ فَلَآ إِثْمَ عَلَيْهِ ۚ لِمَنِ ٱتَّقَىٰ ۗ وَٱتَّقُوا۟ ٱللَّهَ وَٱعْلَمُوٓا۟ أَنَّكُمْ إِلَيْهِ تُحْشَرُونَ ﴿٢٠٣﴾\\
\textamh{204.\ ከሰው ልጆች መካከል ንግግሩ የሚያስደስትህ አለ (ኦ! ሙሐመድ(ሠአወሰ))፥ በዚህ አለም ኑሮ፥ እናም ከልቡ ላለው ኣላህን ምስክሩ አድርጎ ይጠራል፥ ነገር ግን ከተቃሪኒዎች ተጨቃጫቂ መካከል ነው።   } &  وَمِنَ ٱلنَّاسِ مَن يُعْجِبُكَ قَوْلُهُۥ فِى ٱلْحَيَوٰةِ ٱلدُّنْيَا وَيُشْهِدُ ٱللَّهَ عَلَىٰ مَا فِى قَلْبِهِۦ وَهُوَ أَلَدُّ ٱلْخِصَامِ ﴿٢٠٤﴾\\
\textamh{205.\ እና (ከአንተ-ኦ ሙሐመድ(ሠአወሰ)) ሲዞር፥ ጥረቱ ምድር ላይ ብጥብጥ መፍጠር ነው እና አዝእርትንና ከብቶችን ማጥፋት፥ እና ኣላህ ብጥብጥን አይወድም።   } &  وَإِذَا تَوَلَّىٰ سَعَىٰ فِى ٱلْأَرْضِ لِيُفْسِدَ فِيهَا وَيُهْلِكَ ٱلْحَرْثَ وَٱلنَّسْلَ ۗ وَٱللَّهُ لَا يُحِبُّ ٱلْفَسَادَ ﴿٢٠٥﴾\\
\textamh{206.\ \rq\rq{}ኣላህን ፍራ\rq\rq{} ሲባል፥ በኩራት (ክብር) የበለጠ ወንጀል ለመስራት ይመራል። ስለዚህ ለሱ ጀሀነም በቂው ነው፥ በእዉነት ከመጥፎች ቦታ በላይ ነው ለመረፊያ።    } &  وَإِذَا قِيلَ لَهُ ٱتَّقِ ٱللَّهَ أَخَذَتْهُ ٱلْعِزَّةُ بِٱلْإِثْمِ ۚ فَحَسْبُهُۥ جَهَنَّمُ ۚ وَلَبِئْسَ ٱلْمِهَادُ ﴿٢٠٦﴾\\
\textamh{207.\ ከሰዎች መካከል እራሱን የሚሸጥ አለ፥ የኣላህን ደስታ በመፈለግ። ኣላህ ለባሪያዎች ሙሉ የሆነ ርህራሄ አለው።    } &  وَمِنَ ٱلنَّاسِ مَن يَشْرِى نَفْسَهُ ٱبْتِغَآءَ مَرْضَاتِ ٱللَّهِ ۗ وَٱللَّهُ رَءُوفٌۢ بِٱلْعِبَادِ ﴿٢٠٧﴾\\
\textamh{208.\ ኦ እናንት አማኞች! በትክክል ወደ ኢስላም ግቡ እና የሸይጣንን (ሰይጣን) ኮቴ አትከተሉ። በእዉነት፥ እሱ ለእናንተ ግልጽ የሆነ ጠላታችሁ ነው።    } &  يَـٰٓأَيُّهَا ٱلَّذِينَ ءَامَنُوا۟ ٱدْخُلُوا۟ فِى ٱلسِّلْمِ كَآفَّةًۭ وَلَا تَتَّبِعُوا۟ خُطُوَٟتِ ٱلشَّيْطَٟنِ ۚ إِنَّهُۥ لَكُمْ عَدُوٌّۭ مُّبِينٌۭ ﴿٢٠٨﴾\\
\textamh{209.\ ከዚያ ግልጽ የሆነ ምልክት ከመጣላችሁ በኋላ ሸተት ብትሉ፥ እወቁ ኣላህ ከሁሉ በላይ ሀያል ከሁሉ በላይ መርማሪ-ጥበበኛ መሆኑን።   } &   فَإِن زَلَلْتُم مِّنۢ بَعْدِ مَا جَآءَتْكُمُ ٱلْبَيِّنَـٰتُ فَٱعْلَمُوٓا۟ أَنَّ ٱللَّهَ عَزِيزٌ حَكِيمٌ ﴿٢٠٩﴾\\
\textamh{210.\ ኣላህ በደመና ጥላ ከመላኢክቶቹ ጋር እስኪመጣ ይጠብቃሉ? (ያኔ) ነገሩ ፍርዱን አግኝቷል። የሁሉም ነገር ዉሳኔ(ፍርድ) ወደኣላህ ይመለሳል   } &  هَلْ يَنظُرُونَ إِلَّآ أَن يَأْتِيَهُمُ ٱللَّهُ فِى ظُلَلٍۢ مِّنَ ٱلْغَمَامِ وَٱلْمَلَٟٓئِكَةُ وَقُضِىَ ٱلْأَمْرُ ۚ وَإِلَى ٱللَّهِ تُرْجَعُ ٱلْأُمُورُ ﴿٢١٠﴾\\
\textamh{211.\ የእስራኤል ልጆችን ጠይቁ ምን ያህል አያት (ማስራጃ፥ ምልክት) እንደሰጠናቸው። የኣላህን ስጦታ ከመጣለት በኋላ የሚቀይር፥ ከዚያ በእርግጠኝነት፥ ኣላህ በቅጣት ከባድ ነው።    } &  سَلْ بَنِىٓ إِسْرَٟٓءِيلَ كَمْ ءَاتَيْنَـٰهُم مِّنْ ءَايَةٍۭ بَيِّنَةٍۢ ۗ وَمَن يُبَدِّلْ نِعْمَةَ ٱللَّهِ مِنۢ بَعْدِ مَا جَآءَتْهُ فَإِنَّ ٱللَّهَ شَدِيدُ ٱلْعِقَابِ ﴿٢١١﴾\\
\textamh{212.\ ለማያምኑት የዚህ አለም ነገር ያማረ ይመስላል፥ እናም ከአማኞች ላይ ይዘብታሉ። ነገር ግን የኣላህን ትእዛዝ የሚጠብቁና ራሳቸው ከተከልከለ ነገር የሚጠብቁት የትንሳኤ ቀን ከነዚያ በላይ ይሆናሉ። እና ኣላህ ለፈለገው ያለምንም ገደብ ይስጠዋል።   } &  زُيِّنَ لِلَّذِينَ كَفَرُوا۟ ٱلْحَيَوٰةُ ٱلدُّنْيَا وَيَسْخَرُونَ مِنَ ٱلَّذِينَ ءَامَنُوا۟ ۘ وَٱلَّذِينَ ٱتَّقَوْا۟ فَوْقَهُمْ يَوْمَ ٱلْقِيَـٰمَةِ ۗ وَٱللَّهُ يَرْزُقُ مَن يَشَآءُ بِغَيْرِ حِسَابٍۢ ﴿٢١٢﴾\\
\textamh{213.\ የሰው ልጆች አንድ ህብረተሰብ ነበሩ እና ኣላህ ነቢያትን ሊያበስሩና ሊያስጠነቅቁ ላከ፥ ከነሱም ጋር አብሮ መጽሐፍ በሀቅ ላከ ሰዎች የተለያዩበት ነገር ላይ እንዲፈረድ። እና (መጽሐፉ) የተሰጣቸው፥ ከእርስ በርስ ጥላቻ የተነሳ፥ ግልጽ የሆነ ማረጋገጫ ከመጣላቸው በኋላ (ስለመጽሐፉ) ተለያዩ። ከዚያ ኣላህ በፍቃዱ ያመኑትን ከተለያዩበት ላይ ወደእዉነቱ መራ። ኣላህ ያሻዉን ወደ ቀጥኛው መንገድ (ትክክለኛ መንገድ) ይመራል።   } &   كَانَ ٱلنَّاسُ أُمَّةًۭ وَٟحِدَةًۭ فَبَعَثَ ٱللَّهُ ٱلنَّبِيِّۦنَ مُبَشِّرِينَ وَمُنذِرِينَ وَأَنزَلَ مَعَهُمُ ٱلْكِتَـٰبَ بِٱلْحَقِّ لِيَحْكُمَ بَيْنَ ٱلنَّاسِ فِيمَا ٱخْتَلَفُوا۟ فِيهِ ۚ وَمَا ٱخْتَلَفَ فِيهِ إِلَّا ٱلَّذِينَ أُوتُوهُ مِنۢ بَعْدِ مَا جَآءَتْهُمُ ٱلْبَيِّنَـٰتُ بَغْيًۢا بَيْنَهُمْ ۖ فَهَدَى ٱللَّهُ ٱلَّذِينَ ءَامَنُوا۟ لِمَا ٱخْتَلَفُوا۟ فِيهِ مِنَ ٱلْحَقِّ بِإِذْنِهِۦ ۗ وَٱللَّهُ يَهْدِى مَن يَشَآءُ إِلَىٰ صِرَٟطٍۢ مُّسْتَقِيمٍ ﴿٢١٣﴾\\
\textamh{214.\ ወይንስ ከእናንተ በፊት ካለፉት በታች (ያለፈተና) ገነት እንገባለን ብላችሁ ታስባላችሁ? በከባድ ረሃብና በሽታ ነበር የተመቱት እና እነሱም ነ ሆኑ አብረው የነበሩት መልእክተኞችና አማኞች ከመንቀጥቀጣቸው የተነሳ:\rq\rq{}መቼ ነው የኣላህ እርዳታ የሚመጣ?\rq\rq{} አሉ፤ አዎ፥ የኣላህ እርዳታ ቅርብ ነው።   } &  أَمْ حَسِبْتُمْ أَن تَدْخُلُوا۟ ٱلْجَنَّةَ وَلَمَّا يَأْتِكُم مَّثَلُ ٱلَّذِينَ خَلَوْا۟ مِن قَبْلِكُم ۖ مَّسَّتْهُمُ ٱلْبَأْسَآءُ وَٱلضَّرَّآءُ وَزُلْزِلُوا۟ حَتَّىٰ يَقُولَ ٱلرَّسُولُ وَٱلَّذِينَ ءَامَنُوا۟ مَعَهُۥ مَتَىٰ نَصْرُ ٱللَّهِ ۗ أَلَآ إِنَّ نَصْرَ ٱللَّهِ قَرِيبٌۭ ﴿٢١٤﴾\\
\textamh{215.\ ምን ማውጣት እንዳለባቸው ይጠይቁሀል (ኦ ሙሐመድ(ሠአወሰ))። (እንዲህ) በል: \rq\rq{}ምንም አይነት ጥሩ ነገር የምታወጡት ለወላጆቻችሁ፥ ለዘመዶቸችሁ፥ ለወላጅ አልባዎች፥ ለድሆች፥ ለመንገድኞች መሆን አለበት እና ማናቸዉም ጥሩ ነገር ብትሰሩ፥ በእዉነት፥ ኣላህ በደንብ ያዉቀዋል   } &   يَسْـَٔلُونَكَ مَاذَا يُنفِقُونَ ۖ قُلْ مَآ أَنفَقْتُم مِّنْ خَيْرٍۢ فَلِلْوَٟلِدَيْنِ وَٱلْأَقْرَبِينَ وَٱلْيَتَـٰمَىٰ وَٱلْمَسَٟكِينِ وَٱبْنِ ٱلسَّبِيلِ ۗ وَمَا تَفْعَلُوا۟ مِنْ خَيْرٍۢ فَإِنَّ ٱللَّهَ بِهِۦ عَلِيمٌۭ ﴿٢١٥﴾\\
\textamh{216.\ ጅሀድ ተዞላችኋል ምንም እንኳ ብትጠሉት፥ የምትጠሉት ነገር ለናንት ጥሩ ሊሆን ይችላል፥ ደግሞ የምትወዱት ነገር ለናንት መጥፎ ይሆናል። ኣላህ ያውቃል እናንተ አታውቁም።   } &  كُتِبَ عَلَيْكُمُ ٱلْقِتَالُ وَهُوَ كُرْهٌۭ لَّكُمْ ۖ وَعَسَىٰٓ أَن تَكْرَهُوا۟ شَيْـًۭٔا وَهُوَ خَيْرٌۭ لَّكُمْ ۖ وَعَسَىٰٓ أَن تُحِبُّوا۟ شَيْـًۭٔا وَهُوَ شَرٌّۭ لَّكُمْ ۗ وَٱللَّهُ يَعْلَمُ وَأَنتُمْ لَا تَعْلَمُونَ ﴿٢١٦﴾\\
\textamh{217.\ በተከበሩት ወራት (በእስልምና ዘመን አቆጣጠር 1ኛው፥ 7ኛው፥ 11ኛው እና 12ኛው ወሮች) ጦርነት ስለማድረግ ይጠይቁሀል። (እንዲህ) በል: \rq\rq{}በእነዚያ (ወራት) ጦርነት ትልቅ (መተላለፍ) ነው ነገር ግን ከዚያ የተለቀ (መተላለፍ) ሰዎችን በኣላህ መንገድ እንዳይሄዱ መከልከል፥ በሱ መካድ፥ ወደ አል-መስጂድ-አል-ሀራም እንዳይሄዱ መከልከል፥ ነዋሪዎችን መስወጣት፥ አል-ፊትና (ፈትና መምጣት) ከግድያ ይልቃል። እና ከሀይማኖታችሁ እስክትወጡ ድረስ መዋጋታቸዉን አያቆሙም፥ ቢችሉ። እና ማንም ከሀይማኖቱ ቢወጣና ከሀዲ ሁኖ ቢሞት፥ ከዚያ ስራው በዚህ አለምና በሚመጣው ይጠፋል፥ እና የእሳቱ ነዋሪዎች ይሆናሉ። እዚያ ዉስጥ ለዘላለም ይቀመጣሉ።\rq\rq{}   } &  يَسْـَٔلُونَكَ عَنِ ٱلشَّهْرِ ٱلْحَرَامِ قِتَالٍۢ فِيهِ ۖ قُلْ قِتَالٌۭ فِيهِ كَبِيرٌۭ ۖ وَصَدٌّ عَن سَبِيلِ ٱللَّهِ وَكُفْرٌۢ بِهِۦ وَٱلْمَسْجِدِ ٱلْحَرَامِ وَإِخْرَاجُ أَهْلِهِۦ مِنْهُ أَكْبَرُ عِندَ ٱللَّهِ ۚ وَٱلْفِتْنَةُ أَكْبَرُ مِنَ ٱلْقَتْلِ ۗ وَلَا يَزَالُونَ يُقَٟتِلُونَكُمْ حَتَّىٰ يَرُدُّوكُمْ عَن دِينِكُمْ إِنِ ٱسْتَطَٟعُوا۟ ۚ وَمَن يَرْتَدِدْ مِنكُمْ عَن دِينِهِۦ فَيَمُتْ وَهُوَ كَافِرٌۭ فَأُو۟لَٟٓئِكَ حَبِطَتْ أَعْمَـٰلُهُمْ فِى ٱلدُّنْيَا وَٱلْءَاخِرَةِ ۖ وَأُو۟لَٟٓئِكَ أَصْحَٟبُ ٱلنَّارِ ۖ هُمْ فِيهَا خَـٰلِدُونَ ﴿٢١٧﴾\\
\textamh{218.\ በእዉነት፥ ያመኑ፥ እና የተሰደዱ (በኣላህ ሃይማኖት) እና በኣላህ መንገድ የለፉ፥ እኒህ የኣላህን ምህረት ተስፋ ያደርጋሉ። እና ኣላህ ሁሌ-ይቅር ባይ፥ ከሁሉም በላይ ምህረተኛ ነው።   } &    إِنَّ ٱلَّذِينَ ءَامَنُوا۟ وَٱلَّذِينَ هَاجَرُوا۟ وَجَٟهَدُوا۟ فِى سَبِيلِ ٱللَّهِ أُو۟لَٟٓئِكَ يَرْجُونَ رَحْمَتَ ٱللَّهِ ۚ وَٱللَّهُ غَفُورٌۭ رَّحِيمٌۭ ﴿٢١٨﴾ ۞ \\
\textamh{219.\ ስለአልኮሆል (የሚያሰክር) መጠጥና ቁማር ይጠይቁሀል። (እንዲህ) በል: \rq\rq{}በነዚህ ትልቅ ሀጢያት አለ፥ እና ትንሽ ጥቅም ለሰዎች፥ ነገር ግን ሀጢያታቸው ከጥቅማቸው ይልቃል\rq\rq{}። ምን መዉጣት እንዳለባቸው ይጠይቁሀል። (እንዲህ) በል: \rq\rq{}ከሚያስፈልጋችሁ በላይ ያለዉን\rq\rq{}። እናም ኣላህ ህጉን ግልጽ ያደርግላችኋል እንድታስቡበት።   } &   يَسْـَٔلُونَكَ عَنِ ٱلْخَمْرِ وَٱلْمَيْسِرِ ۖ قُلْ فِيهِمَآ إِثْمٌۭ كَبِيرٌۭ وَمَنَـٰفِعُ لِلنَّاسِ وَإِثْمُهُمَآ أَكْبَرُ مِن نَّفْعِهِمَا ۗ وَيَسْـَٔلُونَكَ مَاذَا يُنفِقُونَ قُلِ ٱلْعَفْوَ ۗ كَذَٟلِكَ يُبَيِّنُ ٱللَّهُ لَكُمُ ٱلْءَايَـٰتِ لَعَلَّكُمْ تَتَفَكَّرُونَ ﴿٢١٩﴾\\
\textamh{220.\ በዚህ አለምና በሚመጣው አለም። ስለወላጅ አልባዎቹ ይጠይቁሀል። (እንዲህ) በል: \rq\rq{}ከሁሉም የተሻለው ነገር በንብረታቸው ላይ በእዉነት መስራት ነው፥ ከነሱ ጋር ነገራችሁን ከአደባለቃችሁ፥ ከዚያ ወንድሞቻችሁ ናቸው። ኣላህ ያዉቃል ማን ብጥብጥ እንደፈለገ (የነሱን ንብረት ለመብላት) ማን ደግሞ ጥሩ እንደፈለገ። ኣላህ ቢፈልግ፥ እናንተን ችግር ዉስጥ መክተት ይችላል። በእዉነት ኣላህ ከሁሉም በላይ ሀያል፥ ከሁሉ በላይ መርማሪ-ጥበበኛ ነው።\rq\rq{}   } &  فِى ٱلدُّنْيَا وَٱلْءَاخِرَةِ ۗ وَيَسْـَٔلُونَكَ عَنِ ٱلْيَتَـٰمَىٰ ۖ قُلْ إِصْلَاحٌۭ لَّهُمْ خَيْرٌۭ ۖ وَإِن تُخَالِطُوهُمْ فَإِخْوَٟنُكُمْ ۚ وَٱللَّهُ يَعْلَمُ ٱلْمُفْسِدَ مِنَ ٱلْمُصْلِحِ ۚ وَلَوْ شَآءَ ٱللَّهُ لَأَعْنَتَكُمْ ۚ إِنَّ ٱللَّهَ عَزِيزٌ حَكِيمٌۭ ﴿٢٢٠﴾\\
\textamh{221.\ ሙሽሪካትን (ከኣላህ ጋር ሌሎችን አማልክት የምታመልክ/ኣላህ ሸሪክ አለው የሚሉ) አታግቡ እስኪያምኑ (ኣላህን ብቻ እስኪያመልኩ) ድረስ። እናም በእውነት ሴት የምታምን ባሪያ ከሙሽሪካ ትሻላለች ምንም እንኳ እኒያ ቢያስደስቱ። እና (ሴት ልጆቻችሁን) ለሙሽሪኩን ለጋብቻ አትስጡ እስኪያምኑ ድረስ (በኣላህ ብቻ) እና በእዉነት፥ አማኝ ባሪያ ከሙሽሪክ ይሻለል፥ ምንም እንኳ ያ ቢያስደስትህ። እነሱ (ሙሽሪኮች) ወደ እሳት ይጋብዟችኋል፥ ነገር ግን ኣላህ ወደ ገነት እና ወደ ይቅር መባል ይጋብዛችኋል በፈቃዱ፥ እና አያዉን (ጥቅሱን፥ ምልክቱን...) ለሰው ልጆች ግልጽ ያደርጋል እንዲያስታውሱ።   } &  وَلَا تَنكِحُوا۟ ٱلْمُشْرِكَٟتِ حَتَّىٰ يُؤْمِنَّ ۚ وَلَأَمَةٌۭ مُّؤْمِنَةٌ خَيْرٌۭ مِّن مُّشْرِكَةٍۢ وَلَوْ أَعْجَبَتْكُمْ ۗ وَلَا تُنكِحُوا۟ ٱلْمُشْرِكِينَ حَتَّىٰ يُؤْمِنُوا۟ ۚ وَلَعَبْدٌۭ مُّؤْمِنٌ خَيْرٌۭ مِّن مُّشْرِكٍۢ وَلَوْ أَعْجَبَكُمْ ۗ أُو۟لَٟٓئِكَ يَدْعُونَ إِلَى ٱلنَّارِ ۖ وَٱللَّهُ يَدْعُوٓا۟ إِلَى ٱلْجَنَّةِ وَٱلْمَغْفِرَةِ بِإِذْنِهِۦ ۖ وَيُبَيِّنُ ءَايَـٰتِهِۦ لِلنَّاسِ لَعَلَّهُمْ يَتَذَكَّرُونَ ﴿٢٢١﴾\\
\textamh{222.\ ስለወርአበባ ይጠይቁሀል። (እንዲህ) በል: \rq\rq{}ያ አድሀ (ወንድን የሚጎዳ ነው በዚህ ጊዜ ግንኙነት ቢያደርግ) ነው፥ ስለዚህ በሴቶች የወርአበባ ጊዜ አትቅረቡ እና እስኪነጹ ድረስ አትሂዱ (ለመገናኘት)። እና ራሳቸዉን ከነጹ፥ ያኔ (ለመገናኘት) ኣላህ በፈቀደዉ (ባዘዘው) ግቡ። በእዉነት ኣላህ ወደሱ በንስሃ የሚመለሱትን ይወዳል እና ራሳቸዉን የሚያነጹትን ይወደል   } &  وَيَسْـَٔلُونَكَ عَنِ ٱلْمَحِيضِ ۖ قُلْ هُوَ أَذًۭى فَٱعْتَزِلُوا۟ ٱلنِّسَآءَ فِى ٱلْمَحِيضِ ۖ وَلَا تَقْرَبُوهُنَّ حَتَّىٰ يَطْهُرْنَ ۖ فَإِذَا تَطَهَّرْنَ فَأْتُوهُنَّ مِنْ حَيْثُ أَمَرَكُمُ ٱللَّهُ ۚ إِنَّ ٱللَّهَ يُحِبُّ ٱلتَّوَّٟبِينَ وَيُحِبُّ ٱلْمُتَطَهِّرِينَ ﴿٢٢٢﴾\\
\textamh{223.\ ሚስቶቻችሁ እንደእርሻ መሬት ናቸው፥ ስለዚህ ሂዱ ወደ እርሻችሁ (ተገናኙቸው)፥ መቼም እንዴትም እንደፈለጋችሁ እና (ጥሩ ነገር) በፊታችሁ አድርጉ። እና ኣላህን ፍሩ፥ እና እንደምትገናኙት እወቁ።   } &  نِسَآؤُكُمْ حَرْثٌۭ لَّكُمْ فَأْتُوا۟ حَرْثَكُمْ أَنَّىٰ شِئْتُمْ ۖ وَقَدِّمُوا۟ لِأَنفُسِكُمْ ۚ وَٱتَّقُوا۟ ٱللَّهَ وَٱعْلَمُوٓا۟ أَنَّكُم مُّلَٟقُوهُ ۗ وَبَشِّرِ ٱلْمُؤْمِنِينَ ﴿٢٢٣﴾\\
\textamh{224.\ የኣላህን (ስም) እንደምክንያት በመሃላ ጥሩ ላለመስራት እና ጻዲቅ ላለመሆን፥ እና ሰላም በሰዎች መካከል ላለመድረግ አታድርጉት። እና ኣላህ ሁሉን-ሰሚ ሁሉን-አዋቂ ነው።   } &  وَلَا تَجْعَلُوا۟ ٱللَّهَ عُرْضَةًۭ لِأَيْمَـٰنِكُمْ أَن تَبَرُّوا۟ وَتَتَّقُوا۟ وَتُصْلِحُوا۟ بَيْنَ ٱلنَّاسِ ۗ وَٱللَّهُ سَمِيعٌ عَلِيمٌۭ ﴿٢٢٤﴾\\
\textamh{225.\ ኣላህ ሳታስቡት በማላችሁት ምክንያት ሀላፊነት እንድትወስዱ አያደረግም፥ ነገር ግን ልባችሁ ባገኘው ሀላፊነት ያስወስዳችኋል። እና ኣላህ ሁሌ-ይቅር ባይ ከሁሉም በላይ ምህርተኛ ነው።    } &  لَّا يُؤَاخِذُكُمُ ٱللَّهُ بِٱللَّغْوِ فِىٓ أَيْمَـٰنِكُمْ وَلَٟكِن يُؤَاخِذُكُم بِمَا كَسَبَتْ قُلُوبُكُمْ ۗ وَٱللَّهُ غَفُورٌ حَلِيمٌۭ ﴿٢٢٥﴾\\
\textamh{226.\ ከሚስቶቻቸው ጋር ላለመገናኘት የሚምሉ አራት ወር መጠበቅ አለባቸው፥ ከዚያ ቢመለሱ፥ በእዉነት፥ ኣላህ ሁሌ-ይቅር ባይ ከሁሉም በላይ ምህርተኛ ነው።   } &  لِّلَّذِينَ يُؤْلُونَ مِن نِّسَآئِهِمْ تَرَبُّصُ أَرْبَعَةِ أَشْهُرٍۢ ۖ فَإِن فَآءُو فَإِنَّ ٱللَّهَ غَفُورٌۭ رَّحِيمٌۭ ﴿٢٢٦﴾\\
\textamh{227.\ እናም ለመፋታት ቢወስኑ፥ ኣላህ ሁሉን-ሰሚ፥ ሁሉን-አዋቂ ነው።   } &  وَإِنْ عَزَمُوا۟ ٱلطَّلَٟقَ فَإِنَّ ٱللَّهَ سَمِيعٌ عَلِيمٌۭ ﴿٢٢٧﴾\\
\textamh{228.\ የተፋቱት ሴቶች ሶስት የወርአበባ ጊዜ መጠበቅ አለባቸው፥ እና ለነሱ ማህጸናቸዉ ዉስጥ ኣላህ የፈጠረዉን መደበቅ ህጋዊ አይደለም፥ በኣላህና በመጨረሻው ቀን የሚያምኑ ከሆነ። እና ባሎቻቸው በዚያ ጊዜ እነሱን መልሶ የመዉሰድ የተሻለ መብት አላቸው፥ ለመታረቅ ቢፈልጉ። እና እነሱም (ሴቶቹ) ተመሳሳይ መብት አላቸው አግባብ ባለው መልኩ ነገር ግን ወንዶች አንድ ደረጃ (ሀላፊነት) እነሱ ላይ አለባቸው። እና ኣላህ ከሁሉ በላይ ሀያል ሁሉን መርማሪ-ጥበበኛ ነው።   } &   وَٱلْمُطَلَّقَٟتُ يَتَرَبَّصْنَ بِأَنفُسِهِنَّ ثَلَٟثَةَ قُرُوٓءٍۢ ۚ وَلَا يَحِلُّ لَهُنَّ أَن يَكْتُمْنَ مَا خَلَقَ ٱللَّهُ فِىٓ أَرْحَامِهِنَّ إِن كُنَّ يُؤْمِنَّ بِٱللَّهِ وَٱلْيَوْمِ ٱلْءَاخِرِ ۚ وَبُعُولَتُهُنَّ أَحَقُّ بِرَدِّهِنَّ فِى ذَٟلِكَ إِنْ أَرَادُوٓا۟ إِصْلَٟحًۭا ۚ وَلَهُنَّ مِثْلُ ٱلَّذِى عَلَيْهِنَّ بِٱلْمَعْرُوفِ ۚ وَلِلرِّجَالِ عَلَيْهِنَّ دَرَجَةٌۭ ۗ وَٱللَّهُ عَزِيزٌ حَكِيمٌ ﴿٢٢٨﴾\\
\textamh{229.\ መፋታት ሁለት ጊዜ ነው፥ ከዚያ በኋላ፥ አግባብ ባለው መልኩ ትይዟቸዋላችሁ ወይን በርህራሄ ተዉአቸው። (ወንዶች) የሰጣችሁትን መህር (በመጋቢያ ጊዜ የሰጡን ገንዘብ) መውሰድ (ማስመለስ) ህጋዊ አይደለም፥ ሁለቱም ወገኖች በኣላህ የተደነገገዉን ድንበር (ልክ) መድረግ የሚሳናቸው መሆኑን ከፈሩ ብቻ (ማስመለስ ይችላል) በቀር። ከዚያም የኣላህን ድንጋጌ የተወሰነላቸዉን ማድረግ የማይችሉ ሁኖው ከሰጉ፥ ያኔ ለመፈታት (አል-ኹል) ብትመልስለት ሀጢያት የለበት። እነዚህ ናቸው በኣላህ ትእዛዝ የተደርጉ ልኮች፥ ስለዚህ አትተላለፏቸው። እና ማንም ኣላህ ያዘዘዉን ልክ ቢያልፍ፥ እነዚህ ዛሊሙን (ስህተት (መጥፎ) ሰሪዎች) ናቸው።    } &  ٱلطَّلَٟقُ مَرَّتَانِ ۖ فَإِمْسَاكٌۢ بِمَعْرُوفٍ أَوْ تَسْرِيحٌۢ بِإِحْسَٟنٍۢ ۗ وَلَا يَحِلُّ لَكُمْ أَن تَأْخُذُوا۟ مِمَّآ ءَاتَيْتُمُوهُنَّ شَيْـًٔا إِلَّآ أَن يَخَافَآ أَلَّا يُقِيمَا حُدُودَ ٱللَّهِ ۖ فَإِنْ خِفْتُمْ أَلَّا يُقِيمَا حُدُودَ ٱللَّهِ فَلَا جُنَاحَ عَلَيْهِمَا فِيمَا ٱفْتَدَتْ بِهِۦ ۗ تِلْكَ حُدُودُ ٱللَّهِ فَلَا تَعْتَدُوهَا ۚ وَمَن يَتَعَدَّ حُدُودَ ٱللَّهِ فَأُو۟لَٟٓئِكَ هُمُ ٱلظَّٟلِمُونَ ﴿٢٢٩﴾\\
\textamh{230.\ እና ከፈታት (ለሶስተኛ ጊዜ)፥ ከዚያ በኋለ ሌላ ባል ካላገባች ለሱ ህጋዊ አይደለችም። ከዚያ፥ ሌላኛው ባል ከፈታት፥ ሁለቱ ላይ ሀጢያት የለም ተመልሰው ቢሆኑ፥ የኣላህን ድንበር (ልክ፥ ህግ) የሚጠብቁ ከመሰላቸው። እኒህ የኣላህ ገደብ ናቸው፥ እዉቀት ለአላቸው ግልጽ የሚያደርገው።   } &  فَإِن طَلَّقَهَا فَلَا تَحِلُّ لَهُۥ مِنۢ بَعْدُ حَتَّىٰ تَنكِحَ زَوْجًا غَيْرَهُۥ ۗ فَإِن طَلَّقَهَا فَلَا جُنَاحَ عَلَيْهِمَآ أَن يَتَرَاجَعَآ إِن ظَنَّآ أَن يُقِيمَا حُدُودَ ٱللَّهِ ۗ وَتِلْكَ حُدُودُ ٱللَّهِ يُبَيِّنُهَا لِقَوْمٍۢ يَعْلَمُونَ ﴿٢٣٠﴾\\
\textamh{231.\ እና ሴቶችን ከፈታችሁ በኋላና የተወሰነላቸዉን ጊዜ ከጨረሱ፥ አግባብ ባለው መልኩ መልሳችሁ ዉስዷቸው ወይንም አግባብ ባለው መልኩ ነጻ አድርጓቸው። ነገር ግን ለመጉዳት አትዉሰዷቸው፥ እና ማንም ያን ቢያደርግ፥ ራሱን ጎድቷል። እና የኣላህን ጥቅሶች እንደቀልድ አትዉስዱ፥ ነገር ግን የኣላህን ስጦታ አስታውሱ (ኢስላምን)፥ እናም ያወርደላችሁን መጽሐፍ እና አል-ሂክማ በዚያ የሚያዛችሁ። እና ኣላህን ፍሩ፥ እና እወቁ ኣላህ ከሁሉ በላይ የሁሉን ነገሮች ተረጂ መሆኑን።   } &   وَإِذَا طَلَّقْتُمُ ٱلنِّسَآءَ فَبَلَغْنَ أَجَلَهُنَّ فَأَمْسِكُوهُنَّ بِمَعْرُوفٍ أَوْ سَرِّحُوهُنَّ بِمَعْرُوفٍۢ ۚ وَلَا تُمْسِكُوهُنَّ ضِرَارًۭا لِّتَعْتَدُوا۟ ۚ وَمَن يَفْعَلْ ذَٟلِكَ فَقَدْ ظَلَمَ نَفْسَهُۥ ۚ وَلَا تَتَّخِذُوٓا۟ ءَايَـٰتِ ٱللَّهِ هُزُوًۭا ۚ وَٱذْكُرُوا۟ نِعْمَتَ ٱللَّهِ عَلَيْكُمْ وَمَآ أَنزَلَ عَلَيْكُم مِّنَ ٱلْكِتَـٰبِ وَٱلْحِكْمَةِ يَعِظُكُم بِهِۦ ۚ وَٱتَّقُوا۟ ٱللَّهَ وَٱعْلَمُوٓا۟ أَنَّ ٱللَّهَ بِكُلِّ شَىْءٍ عَلِيمٌۭ ﴿٢٣١﴾\\
\textamh{232.\ እና ሴቶችን ከፈታችሁ በኋላና የተወሰነላቸዉን ጊዜ ከጨረሱ፥ (የቀድሞ) ባሎቻቸዉን እንዳያገቡ አትከልክሏቸው፥ ሁለቱም አግባብ ባለው መልኩ ከተስማሙ። ይሄ (ትእዛዝ) በኣላህና በመጨረሻው ቀን ለሚያምኑ ማስታወሻ (ማስገንዘቢያ) ነው። ያ የተሻለና የነፃ (የፀዳ) ነው። ኣላህ ያዉቃል እናንተ አታውቁም።   } &   وَإِذَا طَلَّقْتُمُ ٱلنِّسَآءَ فَبَلَغْنَ أَجَلَهُنَّ فَلَا تَعْضُلُوهُنَّ أَن يَنكِحْنَ أَزْوَٟجَهُنَّ إِذَا تَرَٟضَوْا۟ بَيْنَهُم بِٱلْمَعْرُوفِ ۗ ذَٟلِكَ يُوعَظُ بِهِۦ مَن كَانَ مِنكُمْ يُؤْمِنُ بِٱللَّهِ وَٱلْيَوْمِ ٱلْءَاخِرِ ۗ ذَٟلِكُمْ أَزْكَىٰ لَكُمْ وَأَطْهَرُ ۗ وَٱللَّهُ يَعْلَمُ وَأَنتُمْ لَا تَعْلَمُونَ ﴿٢٣٢﴾ ۞ \\
\textamh{233.\ እናቶች ለልጆች ለሁለት ሙሉ አመታት ማጥባት አለባቸው፥ (ያ) የማጥቢያ ጊዜን ለመጨረስ የፈልጉ ከሆነ፥ ነገር ግን አባቱ የእናቶችን ምግብና ልብስ ወጪ መሸፈን አለበት፥ አግባብ ባለው መልኩ። ማንም ሰው አቅሙ ከሚፈቅደው በላይ ጫና አይኖርበትም። የትኛዋም እናት በልጇ ምክንያት ያለአግባብ መጎዳት የለባትም ወይንም አባት መጎዳት የለበትም። ለአሳዳጊም አንድ አይነት አግባብ ነው። መለያየት ቢፈልጉ፥ በስምምነት፥ ከመመካር በኋለ፥ ሁለቱም ላይ ሀጢያት አይኖርም። አሳዳጊ አጥቢ እናት ቢቀጥሩ፥ ሀጢያት የለዉም፥ አግባብ ባለው መልኩ (ተቀጣሪዋን)የተስማሙትን መክፈል ከቻሉ። እና ኣላህን ፍሩ እና እወቁ ኣላህ የምትሰሩትን ሁሉን-የሚያይ ነው።   } &  وَٱلْوَٟلِدَٟتُ يُرْضِعْنَ أَوْلَٟدَهُنَّ حَوْلَيْنِ كَامِلَيْنِ ۖ لِمَنْ أَرَادَ أَن يُتِمَّ ٱلرَّضَاعَةَ ۚ وَعَلَى ٱلْمَوْلُودِ لَهُۥ رِزْقُهُنَّ وَكِسْوَتُهُنَّ بِٱلْمَعْرُوفِ ۚ لَا تُكَلَّفُ نَفْسٌ إِلَّا وُسْعَهَا ۚ لَا تُضَآرَّ وَٟلِدَةٌۢ بِوَلَدِهَا وَلَا مَوْلُودٌۭ لَّهُۥ بِوَلَدِهِۦ ۚ وَعَلَى ٱلْوَارِثِ مِثْلُ ذَٟلِكَ ۗ فَإِنْ أَرَادَا فِصَالًا عَن تَرَاضٍۢ مِّنْهُمَا وَتَشَاوُرٍۢ فَلَا جُنَاحَ عَلَيْهِمَا ۗ وَإِنْ أَرَدتُّمْ أَن تَسْتَرْضِعُوٓا۟ أَوْلَٟدَكُمْ فَلَا جُنَاحَ عَلَيْكُمْ إِذَا سَلَّمْتُم مَّآ ءَاتَيْتُم بِٱلْمَعْرُوفِ ۗ وَٱتَّقُوا۟ ٱللَّهَ وَٱعْلَمُوٓا۟ أَنَّ ٱللَّهَ بِمَا تَعْمَلُونَ بَصِيرٌۭ ﴿٢٣٣﴾\\
\textamh{234.\ እና ከናንተ የሚሞቱትና ሚስት ትተው የሚያልፉ፥ እነሱ (ሚስቶቹ) አራት ወር ከአስር ቀን መጠበቅ አለባቸው፥ ከዚያ የተወሰነላቸዉን ጊዜ ከጨረሱ፥ እነሱ ላይ ሀጢያት የለም ራሳቸዉን ፍትሃዊና በተከብረ ሁኔታ (ከሞተው ሰው ጋብቻ) መውጣት ይችላሉ። እና ኣላህ የምትሰሩትን በደንብ ያዉቀዋል።   } &   وَٱلَّذِينَ يُتَوَفَّوْنَ مِنكُمْ وَيَذَرُونَ أَزْوَٟجًۭا يَتَرَبَّصْنَ بِأَنفُسِهِنَّ أَرْبَعَةَ أَشْهُرٍۢ وَعَشْرًۭا ۖ فَإِذَا بَلَغْنَ أَجَلَهُنَّ فَلَا جُنَاحَ عَلَيْكُمْ فِيمَا فَعَلْنَ فِىٓ أَنفُسِهِنَّ بِٱلْمَعْرُوفِ ۗ وَٱللَّهُ بِمَا تَعْمَلُونَ خَبِيرٌۭ ﴿٢٣٤﴾\\
\textamh{235.\ እናንተ ላይ ሀጢያት የለም (ለነዚህ ሴቶች) በግልጽ ለጋብቻ ብትጠይቋቸው ወይንም (ሁለታችሁ) በሚስጥር ብትይዙት። እንደምታስታዉሷቸው ኣላህ ያውቃል። ነገር ግን በሚስጥር (የጋብቻ) ኮንትራት ቃል አትግቡ ጥሩ ነገር ከማለት ዉጪ (እንደ ኢስላም ህግ)። ከነሱ ጋር ጋብቻ አትፈጽሙ የተወሰነላቸው ጊዜ እስኪፈጸም። እና እወቁ ኣላህ በአምሮችሁ (በልባችሁ፥ ሀሳባችሁን) ያለዉን ያዉቃል፥ ስለዚህ ፍሩት። እና እወቁ ኣላህ ሁል-ጊዜ ይቅር ባይ፥ ከሁሉም በላይ ቻይ ነው   } &  وَلَا جُنَاحَ عَلَيْكُمْ فِيمَا عَرَّضْتُم بِهِۦ مِنْ خِطْبَةِ ٱلنِّسَآءِ أَوْ أَكْنَنتُمْ فِىٓ أَنفُسِكُمْ ۚ عَلِمَ ٱللَّهُ أَنَّكُمْ سَتَذْكُرُونَهُنَّ وَلَٟكِن لَّا تُوَاعِدُوهُنَّ سِرًّا إِلَّآ أَن تَقُولُوا۟ قَوْلًۭا مَّعْرُوفًۭا ۚ وَلَا تَعْزِمُوا۟ عُقْدَةَ ٱلنِّكَاحِ حَتَّىٰ يَبْلُغَ ٱلْكِتَـٰبُ أَجَلَهُۥ ۚ وَٱعْلَمُوٓا۟ أَنَّ ٱللَّهَ يَعْلَمُ مَا فِىٓ أَنفُسِكُمْ فَٱحْذَرُوهُ ۚ وَٱعْلَمُوٓا۟ أَنَّ ٱللَّهَ غَفُورٌ حَلِيمٌۭ ﴿٢٣٥﴾\\
\textamh{236.\ ሀጢያት የለባችሁም ሴቶችን ሳትነኩ ብትፈቷቸው (ሳትገናኟቸው) ወይንም መህር ባትከፍሉ። ነገር ግን ሀብታሙ(ስጦታ) እንደሚችለው ይስጣት፥ ድሀዉም እንደሚችለው፥ አግባብ ያለው ስጦታ መስጠት የጥሩ ሰሪዎች ሀላፊነት ነው።   } &  لَّا جُنَاحَ عَلَيْكُمْ إِن طَلَّقْتُمُ ٱلنِّسَآءَ مَا لَمْ تَمَسُّوهُنَّ أَوْ تَفْرِضُوا۟ لَهُنَّ فَرِيضَةًۭ ۚ وَمَتِّعُوهُنَّ عَلَى ٱلْمُوسِعِ قَدَرُهُۥ وَعَلَى ٱلْمُقْتِرِ قَدَرُهُۥ مَتَـٰعًۢا بِٱلْمَعْرُوفِ ۖ حَقًّا عَلَى ٱلْمُحْسِنِينَ ﴿٢٣٦﴾\\
\textamh{237.\ እናም ሳትነኳቸው ብትፈቱ፥ እና መህር ለነሱ አዘጋጅታችሁ ከሆነ፥ ከዚያ ግማሹን ክፍሉ፥ እነሱ (ሴቶቹ) በስምምነት ከተዉሏችሁ በቀር ወይንም እሱ፥ ጋብቻው እጁ ያለው (ሰዉየ) በስምምነት ከተወና ሙሉውን መህር ከሰጣት በስተቀር። እና መተዉን እና መስጠት ለአል-ታቅዋ (ጽድቅ መስራት) ቅርብ ነው። እና ነጻነትን በመካከላችሁ አትርሱ። በእዉነት ኣላህ የምትሰሩትን ሁሉን-የሚያይ ነው።   } &  وَإِن طَلَّقْتُمُوهُنَّ مِن قَبْلِ أَن تَمَسُّوهُنَّ وَقَدْ فَرَضْتُمْ لَهُنَّ فَرِيضَةًۭ فَنِصْفُ مَا فَرَضْتُمْ إِلَّآ أَن يَعْفُونَ أَوْ يَعْفُوَا۟ ٱلَّذِى بِيَدِهِۦ عُقْدَةُ ٱلنِّكَاحِ ۚ وَأَن تَعْفُوٓا۟ أَقْرَبُ لِلتَّقْوَىٰ ۚ وَلَا تَنسَوُا۟ ٱلْفَضْلَ بَيْنَكُمْ ۚ إِنَّ ٱللَّهَ بِمَا تَعْمَلُونَ بَصِيرٌ ﴿٢٣٧﴾\\
\textamh{238.\ ሳላት በጥንቃቄ ያዙ (አትርሱ) በተለይ የመካከለኛዉን ሳለት (አሶር)። እና ከኣላህ ፊት በመታዘዝ ቁሙ።   } &  حَٟفِظُوا۟ عَلَى ٱلصَّلَوَٟتِ وَٱلصَّلَوٰةِ ٱلْوُسْطَىٰ وَقُومُوا۟ لِلَّهِ قَٟنِتِينَ ﴿٢٣٨﴾\\
\textamh{239.\ እና ብትፈሩ (ጠላት)፥ ሳላት በእግር (እየሄዳችሁ) ወይንም እየጋለባችሁ አድርጉ። እና በሰላም ስትሆኑ ሳላቱን አቅርቡ እሱ (ኣላህ) እንዳስተማራችሁ፥ ድሮ የማታውቁት።   } &  فَإِنْ خِفْتُمْ فَرِجَالًا أَوْ رُكْبَانًۭا ۖ فَإِذَآ أَمِنتُمْ فَٱذْكُرُوا۟ ٱللَّهَ كَمَا عَلَّمَكُم مَّا لَمْ تَكُونُوا۟ تَعْلَمُونَ ﴿٢٣٩﴾\\
\textamh{240.\ እና ከናንተ የሚሞቱትና ሚስት ትተው የሚያልፉ፥ ለአንድ አመት ሳይወጡ የሚያቆያቸው ኑዛዜ ተዉሏቸው። (በራሳቸው ፍላጎት) ቢለቁ፥ ከዚያ እናንተ ላይ ራሳቸው ባደረጉት ነገር ሀጢያት የለዉም፤ አግባብ ባለው መልኩ ከሆነ። እና ኣላህ ከሁሉ በላይ ሀያል ከሁሉ በላይ መርማሪ-ጥበበኛ ነው። (የዚህ ጥቅስ ትእዛዝ በ4:12 ተተክቷል)    } &  وَٱلَّذِينَ يُتَوَفَّوْنَ مِنكُمْ وَيَذَرُونَ أَزْوَٟجًۭا وَصِيَّةًۭ لِأَزْوَٟجِهِم مَّتَـٰعًا إِلَى ٱلْحَوْلِ غَيْرَ إِخْرَاجٍۢ ۚ فَإِنْ خَرَجْنَ فَلَا جُنَاحَ عَلَيْكُمْ فِى مَا فَعَلْنَ فِىٓ أَنفُسِهِنَّ مِن مَّعْرُوفٍۢ ۗ وَٱللَّهُ عَزِيزٌ حَكِيمٌۭ ﴿٢٤٠﴾\\
\textamh{241.\ ለተፈቱት ሴቶች አግባብ ባለው መልኩ መጠበቅ (ማቆያ መስጠት) አለባቸው። ይሄ የሙታቁን ግዴታ ነው   } &  وَلِلْمُطَلَّقَٟتِ مَتَـٰعٌۢ بِٱلْمَعْرُوفِ ۖ حَقًّا عَلَى ٱلْمُتَّقِينَ ﴿٢٤١﴾\\
\textamh{242.\ ስለዚህ ኣላህ አያቱን (ምልክቶቹን፥ ህጎቹን) ግልጽ ያደረግላችኋል፥ እንዲገባቸሁ።   } &  كَذَٟلِكَ يُبَيِّنُ ٱللَّهُ لَكُمْ ءَايَـٰتِهِۦ لَعَلَّكُمْ تَعْقِلُونَ ﴿٢٤٢﴾ ۞\\
\textamh{243.\ አንተ (ኦ ሙሐመድ(ሠአወሰ)) አላሰበክም ሺዎች ሁነው ከቤታቸው ስለሄዱት፥ ሞትን እየፈሩ? ኣላህ (እንዲህ) አላቸው፥ \rq\rq{}ሙቱ\rq\rq{}። እና ከዚያ ወደ ህይወት መለሳቸው። በእዉነት ኣላህ ብዙ በረከት ለሰው ልጆች አለው፥ ነገር ግን ብዙዎች ሰዎች አያመሰግኑም።    } &   أَلَمْ تَرَ إِلَى ٱلَّذِينَ خَرَجُوا۟ مِن دِيَـٰرِهِمْ وَهُمْ أُلُوفٌ حَذَرَ ٱلْمَوْتِ فَقَالَ لَهُمُ ٱللَّهُ مُوتُوا۟ ثُمَّ أَحْيَـٰهُمْ ۚ إِنَّ ٱللَّهَ لَذُو فَضْلٍ عَلَى ٱلنَّاسِ وَلَٟكِنَّ أَكْثَرَ ٱلنَّاسِ لَا يَشْكُرُونَ ﴿٢٤٣﴾\\
\textamh{244.\ እና በኣላህ መንገድ ተጋደሉ እና እወቁ ኣላህ ሁሉን-ሰሚ ሁሉን-አወቂ መሆኑን።   } &  وَقَٟتِلُوا۟ فِى سَبِيلِ ٱللَّهِ وَٱعْلَمُوٓا۟ أَنَّ ٱللَّهَ سَمِيعٌ عَلِيمٌۭ ﴿٢٤٤﴾\\
\textamh{245.\ ማን ነው እሱ ለኣላህ ጥሩ ብድር የሚያበድር ብዙ ጊዜ እንዲያበዛለት? እና ኣላህ ነው የሚቀንስ ወይም የሚጨምር። ወደእሱ ትመለሳላችሁ።   } &  مَّن ذَا ٱلَّذِى يُقْرِضُ ٱللَّهَ قَرْضًا حَسَنًۭا فَيُضَٟعِفَهُۥ لَهُۥٓ أَضْعَافًۭا كَثِيرَةًۭ ۚ وَٱللَّهُ يَقْبِضُ وَيَبْصُۜطُ وَإِلَيْهِ تُرْجَعُونَ ﴿٢٤٥﴾\\
\textamh{246.\ ስለተወሰኑ ከሙሳ በኋላ ስለነበሩ የእስራእል ልጆች አላሰባችሁም? ለነቢያቸው (እንዲህ) ሲሉ: \rq\rq{}ንጉስ አድርግልነ እና በኣላህ መንገድ እንታገላለን\rq\rq{} እሱም አለ: \rq\rq{}ከመታገል (ከመዋጋት) ትቆማላችሁ፥ መዋጋት ከታዘዘላችሁ?\rq\rq{} እነሱም አሉ \rq\rq{} ለምን በኣላህ መንገድ አንዋጋም ከቤታችን ወጥተን ሳለ እና ልጆቻችን ጭምር?\rq\rq{} ነገር ግን ጦርነት በታዘዘላቸው ጊዜ፥ ዘወር አሉ፥ ሁሉም ከጥቂቶቻቸው በስተቀር። እና ኣላህ የዛሊሙን(አጥፊዎች) ተገንዛቢ ነው።   } &   أَلَمْ تَرَ إِلَى ٱلْمَلَإِ مِنۢ بَنِىٓ إِسْرَٟٓءِيلَ مِنۢ بَعْدِ مُوسَىٰٓ إِذْ قَالُوا۟ لِنَبِىٍّۢ لَّهُمُ ٱبْعَثْ لَنَا مَلِكًۭا نُّقَٟتِلْ فِى سَبِيلِ ٱللَّهِ ۖ قَالَ هَلْ عَسَيْتُمْ إِن كُتِبَ عَلَيْكُمُ ٱلْقِتَالُ أَلَّا تُقَٟتِلُوا۟ ۖ قَالُوا۟ وَمَا لَنَآ أَلَّا نُقَٟتِلَ فِى سَبِيلِ ٱللَّهِ وَقَدْ أُخْرِجْنَا مِن دِيَـٰرِنَا وَأَبْنَآئِنَا ۖ فَلَمَّا كُتِبَ عَلَيْهِمُ ٱلْقِتَالُ تَوَلَّوْا۟ إِلَّا قَلِيلًۭا مِّنْهُمْ ۗ وَٱللَّهُ عَلِيمٌۢ بِٱلظَّٟلِمِينَ ﴿٢٤٦﴾\\
\textamh{247.\ እና ነቢያቸው (ሳሙኤል) እንዲህ አላቸው: \rq\rq{}በእዉነት ኣላህ ታሉትን (ሳኦልን) ንጉስ አድረጎ እናንተ ላይ ሹሞአል\rq\rq{} እነሱም አሉ:\rq\rq{} እንዴት እሱን ከኛ ላይ ይሾመዋል እኛ ከሱ የተሻለ ለመንግስቱ ሆነን ሳል፥ እና ለሱ በቂ የሆነ ሀብት አልተሰጠዉም\rq\rq{} እሱም አለ: \rq\rq{}በእዉነት፥ ኣላህ ከእናንተ አስበልጦ መርጦታል እና በእዉቀትና በቁመና በደንብ ጨምሮታል። እና ኣላህ መንግስቱን ለፈልገው (ላሻው) ይስጣል። እና ኣላህ ለፍጥረቶቹ ፍላጎት ከሁሉ በላይ በቂ ነው፥ ከሁሉ በላይ ሁሉን አዋቂ\rq\rq{}   } &  وَقَالَ لَهُمْ نَبِيُّهُمْ إِنَّ ٱللَّهَ قَدْ بَعَثَ لَكُمْ طَالُوتَ مَلِكًۭا ۚ قَالُوٓا۟ أَنَّىٰ يَكُونُ لَهُ ٱلْمُلْكُ عَلَيْنَا وَنَحْنُ أَحَقُّ بِٱلْمُلْكِ مِنْهُ وَلَمْ يُؤْتَ سَعَةًۭ مِّنَ ٱلْمَالِ ۚ قَالَ إِنَّ ٱللَّهَ ٱصْطَفَىٰهُ عَلَيْكُمْ وَزَادَهُۥ بَسْطَةًۭ فِى ٱلْعِلْمِ وَٱلْجِسْمِ ۖ وَٱللَّهُ يُؤْتِى مُلْكَهُۥ مَن يَشَآءُ ۚ وَٱللَّهُ وَٟسِعٌ عَلِيمٌۭ ﴿٢٤٧﴾\\
\textamh{248.\ እና ነቢያቸው (ሳሙኤል) (እንዲህ) አላቸው: \rq\rq{}በእዉነት! የመንግስቱ ምልክት አት-ታቡት (ታቦት? የእጨት ሳጥን)፥ ዉስጥ ሰኪና (ሰላም) ከአምላካችሁ ያለበት እና ሙሳና ሀሩን የተዉት ቅሬት፥ መላኢክት የተሸከሙት ይመጣላችኋል። በእውነት፥ በዚህ ምልክት ለእናንተ አለ፥ በእዉነት አማኞች ከሆናችሁ።   } &  وَقَالَ لَهُمْ نَبِيُّهُمْ إِنَّ ءَايَةَ مُلْكِهِۦٓ أَن يَأْتِيَكُمُ ٱلتَّابُوتُ فِيهِ سَكِينَةٌۭ مِّن رَّبِّكُمْ وَبَقِيَّةٌۭ مِّمَّا تَرَكَ ءَالُ مُوسَىٰ وَءَالُ هَـٰرُونَ تَحْمِلُهُ ٱلْمَلَٟٓئِكَةُ ۚ إِنَّ فِى ذَٟلِكَ لَءَايَةًۭ لَّكُمْ إِن كُنتُم مُّؤْمِنِينَ ﴿٢٤٨﴾\\
\textamh{249.\ ከዚያም ታሉት (ሳኦል) ከሰራዊቱ ጋር ሲወጣ (እንዲህ) አለ: \rq\rq{}በእዉነት! ኣላህ በወንዝ ይፈትናችኋል። ማንም ከዚያ ቢጠጣ፥ ከኔ አይደለም፥ እና የማይቀምሰው፥ ከኔ ጋር ነው በእጁ መደፍ ከሚወስደው በቀር\rq\rq{} ነገር ግን፥ ሁሉም ከዚያ ጠጡ ከጥቂቶች በስቀር። እናም አቋረጠው (ወንዙን)፥ እሱና እሱን ያመኑት፥ (እንዲህ) አሉ: \rq\rq{}ዛሬ ከጃሉትና (ጎሊያድ) ሰራዊቶቹ ጋር አቅም የለንም\rq\rq{} ነገር ግን አምላካቸዉን በእርግጠኝነት እንደሚገናኙት የሚያውቁት (እንዲህ) አሉ: \rq\rq{}ስንቴ ነው ትንሽ ሰራዊት በኣላህ ፈቃድ ሀያል ሰራዊት የሚያሸንፉት?\rq\rq{} እናም ኣላህ ከትእግስተኞች (አስ-ሳቢሪን) ጋር ነው።    } &  فَلَمَّا فَصَلَ طَالُوتُ بِٱلْجُنُودِ قَالَ إِنَّ ٱللَّهَ مُبْتَلِيكُم بِنَهَرٍۢ فَمَن شَرِبَ مِنْهُ فَلَيْسَ مِنِّى وَمَن لَّمْ يَطْعَمْهُ فَإِنَّهُۥ مِنِّىٓ إِلَّا مَنِ ٱغْتَرَفَ غُرْفَةًۢ بِيَدِهِۦ ۚ فَشَرِبُوا۟ مِنْهُ إِلَّا قَلِيلًۭا مِّنْهُمْ ۚ فَلَمَّا جَاوَزَهُۥ هُوَ وَٱلَّذِينَ ءَامَنُوا۟ مَعَهُۥ قَالُوا۟ لَا طَاقَةَ لَنَا ٱلْيَوْمَ بِجَالُوتَ وَجُنُودِهِۦ ۚ قَالَ ٱلَّذِينَ يَظُنُّونَ أَنَّهُم مُّلَٟقُوا۟ ٱللَّهِ كَم مِّن فِئَةٍۢ قَلِيلَةٍ غَلَبَتْ فِئَةًۭ كَثِيرَةًۢ بِإِذْنِ ٱللَّهِ ۗ وَٱللَّهُ مَعَ ٱلصَّٟبِرِينَ ﴿٢٤٩﴾\\
\textamh{250.\ እና ጃሉትንና (ጎሊያድን) ሰራዊቱን ለመገናኘት ሲገሰግሱ (እንዲህ ብለው) ድዋ አደረጉ: \rq\rq{}አምላካችን! ትእግስትን አውርድብን እና ከካሃዲ ሰዎች ላይ ድልን ስጠን\rq\rq{}   } &  وَلَمَّا بَرَزُوا۟ لِجَالُوتَ وَجُنُودِهِۦ قَالُوا۟ رَبَّنَآ أَفْرِغْ عَلَيْنَا صَبْرًۭا وَثَبِّتْ أَقْدَامَنَا وَٱنصُرْنَا عَلَى ٱلْقَوْمِ ٱلْكَٟفِرِينَ ﴿٢٥٠﴾\\
\textamh{251.\ በኣላህ ፈቀድ እነዚያን ወጓቸው እና ዳዉድ (ዳዊት) ጃሉትን (ጎሊያድን) ገደለው። እና ኣላህ መንግስቱን (ለዳዉድ (ለዳዊት)) ሰጠው እና አል-ሂክማ እና የፈለገዉን ነገር አስተማረው። እና ኣላህ አንድን ሕብረተሰብ በሌላ ካልያዘው፥ በእዉነት ምድር ሙሉ ብጥብጥ ይሆን ነበር። ነገር ግን ኣላህ ሙሉ በረከት ለአላሚን (ሰዎች፥ ጅኖች እና ያለነገር በሙሉ) አለው።   } &  فَهَزَمُوهُم بِإِذْنِ ٱللَّهِ وَقَتَلَ دَاوُۥدُ جَالُوتَ وَءَاتَىٰهُ ٱللَّهُ ٱلْمُلْكَ وَٱلْحِكْمَةَ وَعَلَّمَهُۥ مِمَّا يَشَآءُ ۗ وَلَوْلَا دَفْعُ ٱللَّهِ ٱلنَّاسَ بَعْضَهُم بِبَعْضٍۢ لَّفَسَدَتِ ٱلْأَرْضُ وَلَٟكِنَّ ٱللَّهَ ذُو فَضْلٍ عَلَى ٱلْعَٟلَمِينَ ﴿٢٥١﴾\\
\textamh{252.\ እኒህ የኣላህ ጥቅሶች ናቸው፥ እኛ በሀቅ እናነብልሀለን (ኦ! ሙሐመድ(ሠአወሰ)) እና በእርግጠኝነት አንተ ከመልእክተኞቹ (የኣላህ) አንዱ ነህ   } &   تِلْكَ ءَايَـٰتُ ٱللَّهِ نَتْلُوهَا عَلَيْكَ بِٱلْحَقِّ ۚ وَإِنَّكَ لَمِنَ ٱلْمُرْسَلِينَ ﴿٢٥٢﴾ ۞\\
\textamh{253.\ እነዚያ መልእክተኞች! አንዳንዶችን ከሌሎች አስበለጥናቸው፤ ለአንዳንዶች ኣላህ ተናገረ (በቀጥታ)፤ ሌሎችን በደረጃ ከፍ አደረገ፤ እና ለኢሳ(የሱስ)፥ የማሪያም ልጅ፥ ግልጽ የሆነ ማረጋገጫና ማስረጃ ሰጠነው፥ እና በመንፈስ ቅዱስ (ጂብሪል(ገብርኤል)) ረዳነው። ኣላህ ቢፈቅድ ኑሮ፥ (ከዚያ በኋላ) የተከተሉት ትውልዶች እርስበርስ ባልተፋጁ ነበር፥ ግልጽ የሆነ ጥቅስ ከኣላህ ከመጣላቸው በኋላ፥ ነገር ግን ተለያዩ- አንዳንዶቹ አመኑ እና ሌሎችም ካዱ። ኣላህ፥ ቢፈቅድ ኑሮ፥ እርስበርስ ባልተጋጩ ነበር ነገር ግን ኣላህ የፈለገዉን ያደርጋል።   } &    تِلْكَ ٱلرُّسُلُ فَضَّلْنَا بَعْضَهُمْ عَلَىٰ بَعْضٍۢ ۘ مِّنْهُم مَّن كَلَّمَ ٱللَّهُ ۖ وَرَفَعَ بَعْضَهُمْ دَرَجَٟتٍۢ ۚ وَءَاتَيْنَا عِيسَى ٱبْنَ مَرْيَمَ ٱلْبَيِّنَـٰتِ وَأَيَّدْنَـٰهُ بِرُوحِ ٱلْقُدُسِ ۗ وَلَوْ شَآءَ ٱللَّهُ مَا ٱقْتَتَلَ ٱلَّذِينَ مِنۢ بَعْدِهِم مِّنۢ بَعْدِ مَا جَآءَتْهُمُ ٱلْبَيِّنَـٰتُ وَلَٟكِنِ ٱخْتَلَفُوا۟ فَمِنْهُم مَّنْ ءَامَنَ وَمِنْهُم مَّن كَفَرَ ۚ وَلَوْ شَآءَ ٱللَّهُ مَا ٱقْتَتَلُوا۟ وَلَٟكِنَّ ٱللَّهَ يَفْعَلُ مَا يُرِيدُ ﴿٢٥٣﴾\\
\textamh{254.\ ኦ እናንት አማኞች! የሰጠናችሁን አውጡና ስጡ፥ ያ ቀን ከመምጣቱ በፊት ክርክር ፥ ወይንም ጓደኛ ወይንም ምልጃ የለለበት። እና ከሀዲዎቹ ናቸው ዛሊሙን (ስህተት ሰሪ)።   } &  يَـٰٓأَيُّهَا ٱلَّذِينَ ءَامَنُوٓا۟ أَنفِقُوا۟ مِمَّا رَزَقْنَـٰكُم مِّن قَبْلِ أَن يَأْتِىَ يَوْمٌۭ لَّا بَيْعٌۭ فِيهِ وَلَا خُلَّةٌۭ وَلَا شَفَٟعَةٌۭ ۗ وَٱلْكَٟفِرُونَ هُمُ ٱلظَّٟلِمُونَ ﴿٢٥٤﴾\\
\textamh{255.\ ኣላህ! ላ ኢላሀ ኢለ ሁዋ (ማንም አምልኮ የሚገባው የለም ከሱ (ከኣላህ) በቀር)፥ ሁሌም ኗሪይዉ፥ የሚያኖረው እና ሁሉን ጠባቂዉ። ማንጎላቸት ወይን እንቅልፍ አይዘዉም። ማናቸዉን ነገር በሰማይ እና ማናቸዉም ነገር በምድር የሱ ናቸው። ማን ነው ከሱ ፈቃድ ዉጭ የሚያማልደው? እነሱ (ፍጥረቶቹ) ላይ ምን እንደሚሆን በዚህ አለም ያዉቃል፥ በሚመጣዉም አለም ምን እንደሚሆን (ያዉቃል)። እና የሱን እውቀት ምንም አይጨብጡም ከፈቀደው በቀር። ኩርሲው ሰማያትን እና ምድርን ያካልላል፥ እና እነሱን ከመጠበቅና ከማቆየት ድካም አይሰማዉም። እና እሱ ነው ከሁሉም በላይ ከፍ ያለ፥ ከሁሉ በላይ ታላቅ።   } &  ٱللَّهُ لَآ إِلَٟهَ إِلَّا هُوَ ٱلْحَىُّ ٱلْقَيُّومُ ۚ لَا تَأْخُذُهُۥ سِنَةٌۭ وَلَا نَوْمٌۭ ۚ لَّهُۥ مَا فِى ٱلسَّمَـٰوَٟتِ وَمَا فِى ٱلْأَرْضِ ۗ مَن ذَا ٱلَّذِى يَشْفَعُ عِندَهُۥٓ إِلَّا بِإِذْنِهِۦ ۚ يَعْلَمُ مَا بَيْنَ أَيْدِيهِمْ وَمَا خَلْفَهُمْ ۖ وَلَا يُحِيطُونَ بِشَىْءٍۢ مِّنْ عِلْمِهِۦٓ إِلَّا بِمَا شَآءَ ۚ وَسِعَ كُرْسِيُّهُ ٱلسَّمَـٰوَٟتِ وَٱلْأَرْضَ ۖ وَلَا يَـُٔودُهُۥ حِفْظُهُمَا ۚ وَهُوَ ٱلْعَلِىُّ ٱلْعَظِيمُ ﴿٢٥٥﴾\\
\textamh{256.\ በሀይማኖት ግዴታ የለም። በእዉነት እውነተኛው መንገድ ከስህተቱ መንገድ ተለይቶአል። ማንም ጣኹት ክዶ እና በኣላህ ካመነ፥ ጥብቅ ታማኝ የሆነ የማይሰበር እጀታ ተጨብጧል። እና ኣላህ ሁሉን-ሰሚ፥ ሁሉን-አዋቂ ነው።   } &  لَآ إِكْرَاهَ فِى ٱلدِّينِ ۖ قَد تَّبَيَّنَ ٱلرُّشْدُ مِنَ ٱلْغَىِّ ۚ فَمَن يَكْفُرْ بِٱلطَّٟغُوتِ وَيُؤْمِنۢ بِٱللَّهِ فَقَدِ ٱسْتَمْسَكَ بِٱلْعُرْوَةِ ٱلْوُثْقَىٰ لَا ٱنفِصَامَ لَهَا ۗ وَٱللَّهُ سَمِيعٌ عَلِيمٌ ﴿٢٥٦﴾\\
\textamh{257.\ ኣላህ የአማኞች ወሊ (ጠባቂ) ነው። ከጨለማ አውጥቶ ወደብርሃን ያስገባቸዋል። ነገር ግን ለሚክዱት፥ የነሱ አውሊያ (አጋዦች) ጣኹት(ጠኦታት) ናቸው፤ ከብርሀን አውጠተው ጨለማ ዉስጥ ይከቷቸዋል። እንዚያ ናቸው የእሳቱ ነዋሪዎች፤ እዚያ ዉስጥ ለዘላለም ይቀመጣሉ።   } &  ٱللَّهُ وَلِىُّ ٱلَّذِينَ ءَامَنُوا۟ يُخْرِجُهُم مِّنَ ٱلظُّلُمَـٰتِ إِلَى ٱلنُّورِ ۖ وَٱلَّذِينَ كَفَرُوٓا۟ أَوْلِيَآؤُهُمُ ٱلطَّٟغُوتُ يُخْرِجُونَهُم مِّنَ ٱلنُّورِ إِلَى ٱلظُّلُمَـٰتِ ۗ أُو۟لَٟٓئِكَ أَصْحَٟبُ ٱلنَّارِ ۖ هُمْ فِيهَا خَـٰلِدُونَ ﴿٢٥٧﴾\\
\textamh{258.\ ከኢብራሂም (አብርሃም) ጋር ስለአምላኩ የተከራከረዉን አላየህም(ችሁም) ኣላህ መንግስቱን ስለሰጠው? እና ኢብራሂም (አብርሃም) (እንዲህ) ሲለው:\rq\rq{} አምላኬ ነው ህይወንትንም ሞትንም የሚሰጥ\rq\rq{} (እንዲህ) አለው: \rq\rq{}እኔ ነኝ ህይወትንም ሞትን የምሰጥ\rq\rq{} ኢብራሂምም (እንዲህ) አለ:\rq\rq{}በእውነት! ኣላህ ነው ፀሀይ በምስራቅ እንደትወጣ የሚያደርግ፥ እስኪ በምእራብ እንድትወጣ አድርግ\rq\rq{} ከዚያም ከሀዲው ያለጥርጣሬ ተሸነፈ። ኣላህ አይመራም ዛሊሙን (ጠማማ) የሆኑ ሰዎችን።   } &   أَلَمْ تَرَ إِلَى ٱلَّذِى حَآجَّ إِبْرَٟهِۦمَ فِى رَبِّهِۦٓ أَنْ ءَاتَىٰهُ ٱللَّهُ ٱلْمُلْكَ إِذْ قَالَ إِبْرَٟهِۦمُ رَبِّىَ ٱلَّذِى يُحْىِۦ وَيُمِيتُ قَالَ أَنَا۠ أُحْىِۦ وَأُمِيتُ ۖ قَالَ إِبْرَٟهِۦمُ فَإِنَّ ٱللَّهَ يَأْتِى بِٱلشَّمْسِ مِنَ ٱلْمَشْرِقِ فَأْتِ بِهَا مِنَ ٱلْمَغْرِبِ فَبُهِتَ ٱلَّذِى كَفَرَ ۗ وَٱللَّهُ لَا يَهْدِى ٱلْقَوْمَ ٱلظَّٟلِمِينَ ﴿٢٥٨﴾\\
\textamh{259.\ ወይስ ልክ እንደአንዱ በከተማ እንዳለፈውና ከተማዉ ተገልብጦ (ሰዎቹ በሙሉ ሞተዋል) እንዳየው። (እሱም) አለ:\rq\rq{}ኦ! እንዴት ኣላህ (ከተማዉን) ከሞተበት ወደ ህይወት ይመልሰዋል?\rq\rq{} ኣላህም ለመቶ አመት እንዲሞት አደርገው፥ ከዚያም (እንደገና) አስነሳው። አለው: \rq\rq{}ምን ያህል ጊዜ (ሞተህ) ቆየህ?\rq\rq{} እሱም አለ: \rq\rq{} (ምንአልባት) አንድ ቀን ወይም የቀኑ ክፋይ ቢሆን ነው\rq\rq{}። አለው: \rq\rq{}የለም፥ ለመቶ አመት ነው (ሞተህ) የነበረ፥ ምግብህንና መጠጥህን ተመልከት፥ አልተቀየሩም፥ እና አህያህን ተመልከት! እና ለሰዎች ምልክት አድርገንሀል። አጥንቶችን ተመልከት፥ እንዴት አንድ ላይ እንደምናደርጋቸዉና በስጋ እንደምናለብሳቸው\rq\rq{}። ይሄ ግልጽ ሲደረግለት፥ እሱም አለ: \rq\rq{}(አሁን) አውቃለሁ ኣላህ ሁሉን ማድረግ እንደሚችል\rq\rq{}   } &  أَوْ كَٱلَّذِى مَرَّ عَلَىٰ قَرْيَةٍۢ وَهِىَ خَاوِيَةٌ عَلَىٰ عُرُوشِهَا قَالَ أَنَّىٰ يُحْىِۦ هَـٰذِهِ ٱللَّهُ بَعْدَ مَوْتِهَا ۖ فَأَمَاتَهُ ٱللَّهُ مِا۟ئَةَ عَامٍۢ ثُمَّ بَعَثَهُۥ ۖ قَالَ كَمْ لَبِثْتَ ۖ قَالَ لَبِثْتُ يَوْمًا أَوْ بَعْضَ يَوْمٍۢ ۖ قَالَ بَل لَّبِثْتَ مِا۟ئَةَ عَامٍۢ فَٱنظُرْ إِلَىٰ طَعَامِكَ وَشَرَابِكَ لَمْ يَتَسَنَّهْ ۖ وَٱنظُرْ إِلَىٰ حِمَارِكَ وَلِنَجْعَلَكَ ءَايَةًۭ لِّلنَّاسِ ۖ وَٱنظُرْ إِلَى ٱلْعِظَامِ كَيْفَ نُنشِزُهَا ثُمَّ نَكْسُوهَا لَحْمًۭا ۚ فَلَمَّا تَبَيَّنَ لَهُۥ قَالَ أَعْلَمُ أَنَّ ٱللَّهَ عَلَىٰ كُلِّ شَىْءٍۢ قَدِيرٌۭ ﴿٢٥٩﴾\\
\textamh{260.\ እና ኢብራሂም (አብርሃም) (እዲህ) ሲል: \rq\rq{}አምላኬ! ለሞቱት ህይወት እንዴት እንደምትሰጥ አሳየኝ\rq\rq{} እሱም (ኣላህ) አለው: \rq\rq{}አታምንም?\rq\rq{} እሱም (ኢብራሂም) አለ:\rq\rq{}አዎን (አምናለሁ)፥ ነገር በእምነቴ ጠንካራ እንድሆን\rq\rq{}። እሱም አለ: \rq\rq{}አራት ወፎች ዉሰድ፥ ከዚያም ወደአንተ ገደም እንዲሉ አድርጋቸው (እናም እረዳቸው፥ ቁረጣቸው)፥ እና ከዚያም ክፋያቸዉን ሁሉም ኮረብታዎች ላይ አድርግ፤ እና ጥራቸው፤ ወደአንተ እየፈጠኑ ይመጣሉ። እና እወቅ ኣላህ ከሁሉም በላይ ሀያል፥ ከሁሉም በላይ ሁሉን መርማሪ-አዋቂ መሆኑን\rq\rq{}   } &  وَإِذْ قَالَ إِبْرَٟهِۦمُ رَبِّ أَرِنِى كَيْفَ تُحْىِ ٱلْمَوْتَىٰ ۖ قَالَ أَوَلَمْ تُؤْمِن ۖ قَالَ بَلَىٰ وَلَٟكِن لِّيَطْمَئِنَّ قَلْبِى ۖ قَالَ فَخُذْ أَرْبَعَةًۭ مِّنَ ٱلطَّيْرِ فَصُرْهُنَّ إِلَيْكَ ثُمَّ ٱجْعَلْ عَلَىٰ كُلِّ جَبَلٍۢ مِّنْهُنَّ جُزْءًۭا ثُمَّ ٱدْعُهُنَّ يَأْتِينَكَ سَعْيًۭا ۚ وَٱعْلَمْ أَنَّ ٱللَّهَ عَزِيزٌ حَكِيمٌۭ ﴿٢٦٠﴾\\
\textamh{261.\ በኣላህ መንገድ ሀብቱን የሚያወጣ ምሳሌው ልክ እንደ(በቆሎ) ፍሬ ነው፤ ሰባት ጆሮ ያወጣል፤ እና እያንዳንዱ ጆሮ መቶ ፍሬ አለው። ኣላህ አባዝቶ ላስደሰተው ይስጣል። እና ኣላህ ለፍጥረቶቹ ፍላጎት ከሁሉ በላይ በቂ ነው፤ ሁሉን-አዋቂ።   } &   مَّثَلُ ٱلَّذِينَ يُنفِقُونَ أَمْوَٟلَهُمْ فِى سَبِيلِ ٱللَّهِ كَمَثَلِ حَبَّةٍ أَنۢبَتَتْ سَبْعَ سَنَابِلَ فِى كُلِّ سُنۢبُلَةٍۢ مِّا۟ئَةُ حَبَّةٍۢ ۗ وَٱللَّهُ يُضَٟعِفُ لِمَن يَشَآءُ ۗ وَٱللَّهُ وَٟسِعٌ عَلِيمٌ ﴿٢٦١﴾\\
\textamh{262.\ እንዚያ ሀብታቸዉን በኣላህ መንገድ የሚሰጡ፥ እና ለጋስነታቸውን በስጦታቸው ማስታወስ ያማይሹ ወይንም በመጉዳት የማያስከትሉ፥ ክፍያቸው ከአምላካቸው አለ። እነሱ ላይ ሀዘን አይኖርም፥ አያዝኑምም።   } &  ٱلَّذِينَ يُنفِقُونَ أَمْوَٟلَهُمْ فِى سَبِيلِ ٱللَّهِ ثُمَّ لَا يُتْبِعُونَ مَآ أَنفَقُوا۟ مَنًّۭا وَلَآ أَذًۭى ۙ لَّهُمْ أَجْرُهُمْ عِندَ رَبِّهِمْ وَلَا خَوْفٌ عَلَيْهِمْ وَلَا هُمْ يَحْزَنُونَ ﴿٢٦٢﴾ ۞\\
\textamh{263.\ ጥሩ ቃላት እና ስህተትን ይቅር ማለት በመጉዳት ከሚከተል ሰደቃ ይበልጣል። እና ኣላህ ሀብታም ነው (ከሁሉ ነገር ነጻ) እና ከሁሉም በላይ ቻይ ነው።   } &    قَوْلٌۭ مَّعْرُوفٌۭ وَمَغْفِرَةٌ خَيْرٌۭ مِّن صَدَقَةٍۢ يَتْبَعُهَآ أَذًۭى ۗ وَٱللَّهُ غَنِىٌّ حَلِيمٌۭ ﴿٢٦٣﴾\\
\textamh{264.\ ኦ እናንት አማኞች! ሰደቃችሁን ባዶ አታድርጉት ለጋስነታችሁን በማስታወስና በመጉዳት፥ ልክ እንደዚያ ሀብቱን በሰዎች ለመታየት እንደሚያወጣው፥ እና በኣላህ አያምንም ወይንም በመጨረሻው ቀን። የሱ ምሳሌ ልክ እንደ ለስላሳ አለት ነው ከላዩ ላይ ትንሽ ትቢያ ያለበት ዝናብ ሲዘንብበት ባዶዉን ይቀራል። በአገኙት ነገር ላይ ምንም ማድረግ አይችሉም። ኣላህ አይመራም የማያምኑ ሰዎችን።   } &  يَـٰٓأَيُّهَا ٱلَّذِينَ ءَامَنُوا۟ لَا تُبْطِلُوا۟ صَدَقَٟتِكُم بِٱلْمَنِّ وَٱلْأَذَىٰ كَٱلَّذِى يُنفِقُ مَالَهُۥ رِئَآءَ ٱلنَّاسِ وَلَا يُؤْمِنُ بِٱللَّهِ وَٱلْيَوْمِ ٱلْءَاخِرِ ۖ فَمَثَلُهُۥ كَمَثَلِ صَفْوَانٍ عَلَيْهِ تُرَابٌۭ فَأَصَابَهُۥ وَابِلٌۭ فَتَرَكَهُۥ صَلْدًۭا ۖ لَّا يَقْدِرُونَ عَلَىٰ شَىْءٍۢ مِّمَّا كَسَبُوا۟ ۗ وَٱللَّهُ لَا يَهْدِى ٱلْقَوْمَ ٱلْكَٟفِرِينَ ﴿٢٦٤﴾\\
\textamh{265.\ የኣላህን ሪድዋን (ደስታ) በመፍለግ ሀብታቸው የሚያወጡ ምሳሌ፥ እናም ራሳቸው ኣላህ እንደሚገናኙት እርግጠኛ የሆኑ ልክ ከፍታ ላይ እንዳለ የትክልት ቦታ ናቸው፥ ከባድ ዝናብ ይዘንብበታል እና ሁለት እጥፍ ያፈራል። እና ከባድ ዝናብ ባይዘንብበት ቀላል ዝናብ ይበቃዋል። እና ኣላህ የምትሰሩት ሁሉን-የሚያይ ነው።   } &  وَمَثَلُ ٱلَّذِينَ يُنفِقُونَ أَمْوَٟلَهُمُ ٱبْتِغَآءَ مَرْضَاتِ ٱللَّهِ وَتَثْبِيتًۭا مِّنْ أَنفُسِهِمْ كَمَثَلِ جَنَّةٍۭ بِرَبْوَةٍ أَصَابَهَا وَابِلٌۭ فَـَٔاتَتْ أُكُلَهَا ضِعْفَيْنِ فَإِن لَّمْ يُصِبْهَا وَابِلٌۭ فَطَلٌّۭ ۗ وَٱللَّهُ بِمَا تَعْمَلُونَ بَصِيرٌ ﴿٢٦٥﴾\\
\textamh{266.\ ከእናንተ ዉስጥ የአትክልት ቦታ ሊኖረው የሚፈልግ አለ፥ ዘንባባ እና ጽዶች፥ ወንዝ በስሩ የሚፈስ፥ እና ሁሉም አይነት ፍራፍሬ ለሱ እዚያ ዉስጥ፥ እናም በእድሜ መግፋት ቢያዝ፥ እና ልጆቹ ደካማ ቢሆኑ፥ ከዚያም በአውሎ ነፈስ ተመታ፥ ተቃጠለበት? ስለዚህ ኣላህ አያቱን (ምልክቶቹን፥ መረጋገጫዉን) ግልጽ ያደርግላችኋል እንድታስቡበት።   } &  أَيَوَدُّ أَحَدُكُمْ أَن تَكُونَ لَهُۥ جَنَّةٌۭ مِّن نَّخِيلٍۢ وَأَعْنَابٍۢ تَجْرِى مِن تَحْتِهَا ٱلْأَنْهَـٰرُ لَهُۥ فِيهَا مِن كُلِّ ٱلثَّمَرَٟتِ وَأَصَابَهُ ٱلْكِبَرُ وَلَهُۥ ذُرِّيَّةٌۭ ضُعَفَآءُ فَأَصَابَهَآ إِعْصَارٌۭ فِيهِ نَارٌۭ فَٱحْتَرَقَتْ ۗ كَذَٟلِكَ يُبَيِّنُ ٱللَّهُ لَكُمُ ٱلْءَايَـٰتِ لَعَلَّكُمْ تَتَفَكَّرُونَ ﴿٢٦٦﴾\\
\textamh{267.\ ኦ! እናንት አማኞች! ጥሩዉን ነገር አውጡ (በህጋዊ) ያገኛችሁትን፥ እና እኛ ከመሬት ያፈራንላችሁን፥ እና መጥፎ የሆነዉን ለማዉጣት አላማ አታድርጉ፤ እናንተ የማትቀበሉትን አይናችሁን ከድናችሁ ካልተቋቋማችሁ በስተቀር። እና እወቁ ኣላህ ሀብታም (ከፍላጎት ሁሉ ነጻ ነው) ነው እና ሁሉ-አይነት ክብር ይገባዋል።   } &   يَـٰٓأَيُّهَا ٱلَّذِينَ ءَامَنُوٓا۟ أَنفِقُوا۟ مِن طَيِّبَٟتِ مَا كَسَبْتُمْ وَمِمَّآ أَخْرَجْنَا لَكُم مِّنَ ٱلْأَرْضِ ۖ وَلَا تَيَمَّمُوا۟ ٱلْخَبِيثَ مِنْهُ تُنفِقُونَ وَلَسْتُم بِـَٔاخِذِيهِ إِلَّآ أَن تُغْمِضُوا۟ فِيهِ ۚ وَٱعْلَمُوٓا۟ أَنَّ ٱللَّهَ غَنِىٌّ حَمِيدٌ ﴿٢٦٧﴾\\
\textamh{268.\ ሸይጣን (ሰይጣን) በረሃብ (ሀብት ማጣት) ያስፈራራችኋል እና ፋህሻ (መጥፎ ነገር) እንድታደርጉ ያዛችኋል፤ ኣላህ ደግሞ ይቅር መባልን ከራሱ እና ለበረከት ቃል ይገባላችኋል፤ እና ኣላህ ለፍጥረቶቹ ፍላጎት ከሁሉም በላይ በቂ ነው፤ ሁሉን-አዋቂው።   } &  ٱلشَّيْطَٟنُ يَعِدُكُمُ ٱلْفَقْرَ وَيَأْمُرُكُم بِٱلْفَحْشَآءِ ۖ وَٱللَّهُ يَعِدُكُم مَّغْفِرَةًۭ مِّنْهُ وَفَضْلًۭا ۗ وَٱللَّهُ وَٟسِعٌ عَلِيمٌۭ ﴿٢٦٨﴾\\
\textamh{269.\ (ኣላህ) ለፈለገው (ላስደስተው) ሂክማ ይስጣል፤ እና እሱ፥ ሂክማ የተሰጠው፥ በእዉነት ብዙ ጥሩ ነገር ተስጦታል። ነገር ግን ማንም አያስታዉስም አቅል ካላቸው ሰዎች (ነገር ከሚገባቸው) በስተቀር   } &  يُؤْتِى ٱلْحِكْمَةَ مَن يَشَآءُ ۚ وَمَن يُؤْتَ ٱلْحِكْمَةَ فَقَدْ أُوتِىَ خَيْرًۭا كَثِيرًۭا ۗ وَمَا يَذَّكَّرُ إِلَّآ أُو۟لُوا۟ ٱلْأَلْبَٟبِ ﴿٢٦٩﴾\\
\textamh{270.\ እናን ማናቸዉም ነገር የምታወጡት ወጪ ወይንም ለማድረግ ቃል የምትገቡት፥ ኣላህ እንደሚያዉቀው እርግጠኛ ሁኑ። እና ለዛሊሙን ረዳት የለም።   } &  وَمَآ أَنفَقْتُم مِّن نَّفَقَةٍ أَوْ نَذَرْتُم مِّن نَّذْرٍۢ فَإِنَّ ٱللَّهَ يَعْلَمُهُۥ ۗ وَمَا لِلظَّٟلِمِينَ مِنْ أَنصَارٍ ﴿٢٧٠﴾\\
\textamh{271.\ ሰደቃችሁን ብትገልጹ ጥሩ ነው፥ ነገር ግን ብትደብቁና ለድሆች ብትሰጡ ለእናንተ የተሻለ ነው። (ኣላህ) የተወሰነ ሀጢያታችሁን ይቅር ይላችኋል። ኣላህ የምትሰሩትን በደንብ ነው የሚያዉቅ   } &  إِن تُبْدُوا۟ ٱلصَّدَقَٟتِ فَنِعِمَّا هِىَ ۖ وَإِن تُخْفُوهَا وَتُؤْتُوهَا ٱلْفُقَرَآءَ فَهُوَ خَيْرٌۭ لَّكُمْ ۚ وَيُكَفِّرُ عَنكُم مِّن سَيِّـَٔاتِكُمْ ۗ وَٱللَّهُ بِمَا تَعْمَلُونَ خَبِيرٌۭ ﴿٢٧١﴾ ۞\\
\textamh{272.\ አንተ ላይ አይደለም (ኦ! ሙሐመድ(ሠአወሰ)) የነሱ መመራት፤ ነገር ግን ኣላህ የፈለገዉን ይመራል። እናም ማናቸውም ነገር ጥሩ የምታወጡት ለራሳችሁ ነው፤ ያለኣላህ መኖርን (በምትሰጡት) በመፈለግ በቀር አታውጡ። እና ማናቸው ጥሩ የምታወጡት ነገር፥ በሙሉ ይከፈላችኋል እና ስህተት አይሰራባችሁም።   } &   لَّيْسَ عَلَيْكَ هُدَىٰهُمْ وَلَٟكِنَّ ٱللَّهَ يَهْدِى مَن يَشَآءُ ۗ وَمَا تُنفِقُوا۟ مِنْ خَيْرٍۢ فَلِأَنفُسِكُمْ ۚ وَمَا تُنفِقُونَ إِلَّا ٱبْتِغَآءَ وَجْهِ ٱللَّهِ ۚ وَمَا تُنفِقُوا۟ مِنْ خَيْرٍۢ يُوَفَّ إِلَيْكُمْ وَأَنتُمْ لَا تُظْلَمُونَ ﴿٢٧٢﴾\\
\textamh{273.\ (ሰደቃ) ለፉቀራ (ለድሆች) ለኣላህ ምክንያት ችግር የያዛቸው እና መልቀቅ (መሰደድ) የማይችሉ። የማያዉቃቸው በጸባያቸው ጥሩነት ሀብታም ይመስሉታል። እነሱን በምልክታቸው ታውቋቸዋለችሁ፤ ሰዉን በፍጹም አይለምኑም። እና ማናቸዉም ለጥሩ (ነገር) ብታወጡ፥ በእርግጠኝነት ኣላህ በደንብ ያውቀዋል።    } &   لِلْفُقَرَآءِ ٱلَّذِينَ أُحْصِرُوا۟ فِى سَبِيلِ ٱللَّهِ لَا يَسْتَطِيعُونَ ضَرْبًۭا فِى ٱلْأَرْضِ يَحْسَبُهُمُ ٱلْجَاهِلُ أَغْنِيَآءَ مِنَ ٱلتَّعَفُّفِ تَعْرِفُهُم بِسِيمَـٰهُمْ لَا يَسْـَٔلُونَ ٱلنَّاسَ إِلْحَافًۭا ۗ وَمَا تُنفِقُوا۟ مِنْ خَيْرٍۢ فَإِنَّ ٱللَّهَ بِهِۦ عَلِيمٌ ﴿٢٧٣﴾\\
\textamh{274.\ እነዚያ በኣላህ (መንገድ) ሀብታቸዉን በቀንና ለሊት የሚያወጡ፥ በድብቅ ወይም በግልጽ፥ ክፍያቸዉን ከአምላካቸው ያገኛሉ። እነሱ ላይ ፍርሃት አይኖርም አያዝኑምም።   } &  ٱلَّذِينَ يُنفِقُونَ أَمْوَٟلَهُم بِٱلَّيْلِ وَٱلنَّهَارِ سِرًّۭا وَعَلَانِيَةًۭ فَلَهُمْ أَجْرُهُمْ عِندَ رَبِّهِمْ وَلَا خَوْفٌ عَلَيْهِمْ وَلَا هُمْ يَحْزَنُونَ ﴿٢٧٤﴾\\
\textamh{275.\ እነዚያ ሪባ (አራጣ) የሚበሉ አይቆሙም (የትንሳኤ ቀን) ሰይጣን እንደመታውና ወደ እብደት እንደመራው ሰው አይነት ከልሆነ በስተቀር። ያም የሆነ እንዲ ስለሚሉ ነው:\rq\rq{}ንግድ ልክ እንደሪባ (አራጣ) ነው\rq\rq{} ነገር ኣላህ ንግድን ፈቅዷል እና ሪባን (አራጣን) ከልክሏል። ስለዚህ ማንም ከአምላኩ ማስታወሻ የሚቀበል እና አራጣን መብላት የሚያቆም ስለአለፈው ህይወቱ አይቀጣም፥ የሱ ፍርድ ለኣላህ ነው፥ ነገር ግን ማንም (ወደሪባ መብላት) የሚመለስ፥ እነዚህ ናቸው የእሳቱ ነዋሪዎች- እዚያ ይኖሩበታል።   } &  ٱلَّذِينَ يَأْكُلُونَ ٱلرِّبَوٰا۟ لَا يَقُومُونَ إِلَّا كَمَا يَقُومُ ٱلَّذِى يَتَخَبَّطُهُ ٱلشَّيْطَٟنُ مِنَ ٱلْمَسِّ ۚ ذَٟلِكَ بِأَنَّهُمْ قَالُوٓا۟ إِنَّمَا ٱلْبَيْعُ مِثْلُ ٱلرِّبَوٰا۟ ۗ وَأَحَلَّ ٱللَّهُ ٱلْبَيْعَ وَحَرَّمَ ٱلرِّبَوٰا۟ ۚ فَمَن جَآءَهُۥ مَوْعِظَةٌۭ مِّن رَّبِّهِۦ فَٱنتَهَىٰ فَلَهُۥ مَا سَلَفَ وَأَمْرُهُۥٓ إِلَى ٱللَّهِ ۖ وَمَنْ عَادَ فَأُو۟لَٟٓئِكَ أَصْحَٟبُ ٱلنَّارِ ۖ هُمْ فِيهَا خَـٰلِدُونَ ﴿٢٧٥﴾\\
\textamh{276.\ ኣላህ ሪባን (አራጣን) ያጠፋል እና ለሰደቃ ይጨምራል። እና ኣላህ የማያምኑትን አይወድም፥ ሀጢያተኞች።   } &  يَمْحَقُ ٱللَّهُ ٱلرِّبَوٰا۟ وَيُرْبِى ٱلصَّدَقَٟتِ ۗ وَٱللَّهُ لَا يُحِبُّ كُلَّ كَفَّارٍ أَثِيمٍ ﴿٢٧٦﴾\\
\textamh{277.\ በእዉነት የሚያምኑ፥ እና ጥሩ (የጽድቅ) ስራ የሚሰሩ፥ እና ሳላት የሚቆሙ፥ እና ዘካት የሚሰጡ፥ እነሱ ከአምላካቸው ክፍያቸው ይስጣቸዋል። እነሱ ላይ ፍርሀት አይኖርም፥ አያዝኑምም።   } &  إِنَّ ٱلَّذِينَ ءَامَنُوا۟ وَعَمِلُوا۟ ٱلصَّٟلِحَٟتِ وَأَقَامُوا۟ ٱلصَّلَوٰةَ وَءَاتَوُا۟ ٱلزَّكَوٰةَ لَهُمْ أَجْرُهُمْ عِندَ رَبِّهِمْ وَلَا خَوْفٌ عَلَيْهِمْ وَلَا هُمْ يَحْزَنُونَ ﴿٢٧٧﴾\\
\textamh{278.\ ኦ! እናንት አማኞች! ኣላህን ፍሩ እና ከአራጣ የቀረዉን ስጡ በእዉነት አማኞች ከሆናችሁ   } &  يَـٰٓأَيُّهَا ٱلَّذِينَ ءَامَنُوا۟ ٱتَّقُوا۟ ٱللَّهَ وَذَرُوا۟ مَا بَقِىَ مِنَ ٱلرِّبَوٰٓا۟ إِن كُنتُم مُّؤْمِنِينَ ﴿٢٧٨﴾\\
\textamh{279.\ ካላደረጋችሁት፥ ከኣላህና ከመልእክተኛው የጦርነት ማስታወቂያ ዉስዱ፤ ነገር ግን ንስሀ ብትገቡ፥ ትክክለኛ ገንዘባችሁን ታገኛላችሁ። ያለፍትህ አትደራደሩ (አራጣ በመፈለግ) እና ያለፍትህ አትጎዱም (የራሳችሁን በመቀበል)   } &  فَإِن لَّمْ تَفْعَلُوا۟ فَأْذَنُوا۟ بِحَرْبٍۢ مِّنَ ٱللَّهِ وَرَسُولِهِۦ ۖ وَإِن تُبْتُمْ فَلَكُمْ رُءُوسُ أَمْوَٟلِكُمْ لَا تَظْلِمُونَ وَلَا تُظْلَمُونَ ﴿٢٧٩﴾\\
\textamh{280.\ ያበደራችሁት ሰው ችግር ዉስጥ ካለ (ገንዘብ የለለው ከሆነ)፥ ከዚያ ጊዜ ስጡት መክፈል መቻል እስኪቀልለት ድረስ፥ ነገር ብትተዉት እንደሰደቃ አድርጋችሁ፥ ያ ለናንተ የተሻለ ነው፥ ብታውቁት   } &  وَإِن كَانَ ذُو عُسْرَةٍۢ فَنَظِرَةٌ إِلَىٰ مَيْسَرَةٍۢ ۚ وَأَن تَصَدَّقُوا۟ خَيْرٌۭ لَّكُمْ ۖ إِن كُنتُمْ تَعْلَمُونَ ﴿٢٨٠﴾\\
\textamh{281.\ እና ወደ ኣላህ የምትመለሱበት ቀን ፍሩ። ያኔ ሁሉም ሰው ያገኘዉን ይከፈላል፥ እናም ያለፍትህ አይፈርድባቸዉም   } &  وَٱتَّقُوا۟ يَوْمًۭا تُرْجَعُونَ فِيهِ إِلَى ٱللَّهِ ۖ ثُمَّ تُوَفَّىٰ كُلُّ نَفْسٍۢ مَّا كَسَبَتْ وَهُمْ لَا يُظْلَمُونَ ﴿٢٨١﴾\\
\textamh{282.\ ኦ! እናንት አማኞች! ብድር ኮንትራት ለተወሰነ ጊዜ ስትገቡ፥ ጻፉት። ጸሀፊ በእውነት በመካከላችሁ ይጻፈው። ጸሀፊው እምቢይ አይበል ኣላህ እንደአስተማረው (መጻፍን)፥ ስለዚህ ይጻፈው። (አበዳሪው) ምን እንደሚጻፍ ይናገር፥ እና ኣላህን መፍራት አለበት፥ አምላኩን፥ እናም የሚያበደረዉን አሳንሶ አይጥራ (አይጻፍ)። ነገር ግን፥ አበዳሪዉ ብዙ የማይገባው ከሆነ፥ ወይም ደካማ፥ ወይንም ማጻፍ የማይችል ከሆነ፥ የሱ ጠበቂ በእዉነት ያጽፍለት። እና ሁለት ወንድ ምስክሮች አድርጉ። ሁለት ወንዶች ከለሉ፥ አንድ ወንድ እና ሁለት ሴት፥ የምትግባቡበት ምስክሮች፥ አንዷ ስህተት ብትሰራ፥ ሌላኛዋ ታስታዉሳታለች። እና ምስክሮች ለማስረጃ ቢጠሩ እምቢይ አይበሉ። ለመጸፍ አትሰላቹ፥ ትንሽም ሆነ ትልቅ፥ ለተወሰነ ጊዜ፥ ያ በኣላህ ዘንድ ተቀባይ ነው፤ የበለጠ ጥሩ ማስረጃ፥ እና በመካከላችሁ ጥርጥሬ እንዳይኖር የበለጠ የተሻለ ነው፤ እዚያው ቦታ ላይ ከምታደርጉት ንግር በስተቀር፤ ያኔ ባትጽፉት ሀጢያት አይሆንባችሁም። ነገር ግን አንድ የንግድ ኮንትራት በምታደርጉበት ጊዜ ሁለት ምስክሮች አድርጉ። ጸሀፊዉም ሆነ ምስክሮቹ እንዳይጎዱ፤ ነገር ግን ብታደርጉ (ብቶግዷቸው)፥ የራሳችሁ ብልሹነት ነው። ስለዚህ ኣላህን ፍሩ፤ ኣላህ ያስተምራችኋል። እና ኣላህ የእያንዳንዷን ነገርና የሁሉ ነገር ሁሉን-አወቂ ነው።    } &   يَـٰٓأَيُّهَا ٱلَّذِينَ ءَامَنُوٓا۟ إِذَا تَدَايَنتُم بِدَيْنٍ إِلَىٰٓ أَجَلٍۢ مُّسَمًّۭى فَٱكْتُبُوهُ ۚ وَلْيَكْتُب بَّيْنَكُمْ كَاتِبٌۢ بِٱلْعَدْلِ ۚ وَلَا يَأْبَ كَاتِبٌ أَن يَكْتُبَ كَمَا عَلَّمَهُ ٱللَّهُ ۚ فَلْيَكْتُبْ وَلْيُمْلِلِ ٱلَّذِى عَلَيْهِ ٱلْحَقُّ وَلْيَتَّقِ ٱللَّهَ رَبَّهُۥ وَلَا يَبْخَسْ مِنْهُ شَيْـًۭٔا ۚ فَإِن كَانَ ٱلَّذِى عَلَيْهِ ٱلْحَقُّ سَفِيهًا أَوْ ضَعِيفًا أَوْ لَا يَسْتَطِيعُ أَن يُمِلَّ هُوَ فَلْيُمْلِلْ وَلِيُّهُۥ بِٱلْعَدْلِ ۚ وَٱسْتَشْهِدُوا۟ شَهِيدَيْنِ مِن رِّجَالِكُمْ ۖ فَإِن لَّمْ يَكُونَا رَجُلَيْنِ فَرَجُلٌۭ وَٱمْرَأَتَانِ مِمَّن تَرْضَوْنَ مِنَ ٱلشُّهَدَآءِ أَن تَضِلَّ إِحْدَىٰهُمَا فَتُذَكِّرَ إِحْدَىٰهُمَا ٱلْأُخْرَىٰ ۚ وَلَا يَأْبَ ٱلشُّهَدَآءُ إِذَا مَا دُعُوا۟ ۚ وَلَا تَسْـَٔمُوٓا۟ أَن تَكْتُبُوهُ صَغِيرًا أَوْ كَبِيرًا إِلَىٰٓ أَجَلِهِۦ ۚ ذَٟلِكُمْ أَقْسَطُ عِندَ ٱللَّهِ وَأَقْوَمُ لِلشَّهَـٰدَةِ وَأَدْنَىٰٓ أَلَّا تَرْتَابُوٓا۟ ۖ إِلَّآ أَن تَكُونَ تِجَٟرَةً حَاضِرَةًۭ تُدِيرُونَهَا بَيْنَكُمْ فَلَيْسَ عَلَيْكُمْ جُنَاحٌ أَلَّا تَكْتُبُوهَا ۗ وَأَشْهِدُوٓا۟ إِذَا تَبَايَعْتُمْ ۚ وَلَا يُضَآرَّ كَاتِبٌۭ وَلَا شَهِيدٌۭ ۚ وَإِن تَفْعَلُوا۟ فَإِنَّهُۥ فُسُوقٌۢ بِكُمْ ۗ وَٱتَّقُوا۟ ٱللَّهَ ۖ وَيُعَلِّمُكُمُ ٱللَّهُ ۗ وَٱللَّهُ بِكُلِّ شَىْءٍ عَلِيمٌۭ ﴿٢٨٢﴾ ۞\\
\textamh{283.\ በመንገድ ላይ ብትሆኑና ጸሀፊ ባታገኙ፥ ከዚያ እምነት (ዉል) ይወሰድ፤ ከዚያም አንዳችሁ ከሌላው ላይ ዉል ካደረጋችሁ፥ ዉል የተወስደበት ሰው ዉሉን ይወጣ፤ እና ኣላህን ይፍራ፥ አምላኩን። እና ማስረጃውን አይደበቅ፥ ያ የሚደብቀው (ሰው) በእዉነት ልቡ ሀጢያተኛ ነው። እና ኣላህ የምትሰሩትን ሁሉን-አዋቂ ነው።   } &   وَإِن كُنتُمْ عَلَىٰ سَفَرٍۢ وَلَمْ تَجِدُوا۟ كَاتِبًۭا فَرِهَـٰنٌۭ مَّقْبُوضَةٌۭ ۖ فَإِنْ أَمِنَ بَعْضُكُم بَعْضًۭا فَلْيُؤَدِّ ٱلَّذِى ٱؤْتُمِنَ أَمَـٰنَتَهُۥ وَلْيَتَّقِ ٱللَّهَ رَبَّهُۥ ۗ وَلَا تَكْتُمُوا۟ ٱلشَّهَـٰدَةَ ۚ وَمَن يَكْتُمْهَا فَإِنَّهُۥٓ ءَاثِمٌۭ قَلْبُهُۥ ۗ وَٱللَّهُ بِمَا تَعْمَلُونَ عَلِيمٌۭ ﴿٢٨٣﴾\\
\textamh{284.\ በሰማያትና በምድር ያለ ሁሉ የኣላህ ነው፤ ዉስጣችሁ ያለዉን ብታወጡት ወይንም ብትድብቁ፥ ኣላህ ሀላፊነት ያስወስዳችኋል። ከዚያም የፈለገዉን ይቅር ይላል እና የፈለገዉን ይቅጣል። እና ኣላህ ሁሉን ማድረግ ይችላል።   } &  لِّلَّهِ مَا فِى ٱلسَّمَـٰوَٟتِ وَمَا فِى ٱلْأَرْضِ ۗ وَإِن تُبْدُوا۟ مَا فِىٓ أَنفُسِكُمْ أَوْ تُخْفُوهُ يُحَاسِبْكُم بِهِ ٱللَّهُ ۖ فَيَغْفِرُ لِمَن يَشَآءُ وَيُعَذِّبُ مَن يَشَآءُ ۗ وَٱللَّهُ عَلَىٰ كُلِّ شَىْءٍۢ قَدِيرٌ ﴿٢٨٤﴾\\
\textamh{285.\ መልእከተኛው (ሙሐመድ(ሠአወሰ)) ከአምላኩ በወረደው ያምናል እናም አማኞቹ። እያንአንዳንዱ (ሁሉም) በኣላህ፥ በመላኢክት፥ በመጽሀፉ፥ እና በመልእክተኞቹ ያምናሉ። (እንዲህ) ይላሉ:\rq\rq{}በመልእክተኞቹ መካከል ልዩነት አናደርግም\rq\rq{} እናም ይላሉ: \rq\rq{}ሰማነ፥ እና ተዘዝነ (አደርግነ)። ይቅርታህን ስጠን አምላካችን፤ እና ወደአንተ እንመለሳለን\rq\rq{}    } &  ءَامَنَ ٱلرَّسُولُ بِمَآ أُنزِلَ إِلَيْهِ مِن رَّبِّهِۦ وَٱلْمُؤْمِنُونَ ۚ كُلٌّ ءَامَنَ بِٱللَّهِ وَمَلَٟٓئِكَتِهِۦ وَكُتُبِهِۦ وَرُسُلِهِۦ لَا نُفَرِّقُ بَيْنَ أَحَدٍۢ مِّن رُّسُلِهِۦ ۚ وَقَالُوا۟ سَمِعْنَا وَأَطَعْنَا ۖ غُفْرَانَكَ رَبَّنَا وَإِلَيْكَ ٱلْمَصِيرُ ﴿٢٨٥﴾\\
\textamh{286.\ ኣላህ አንድ ሰው ከአቅሙ በላይ አይጭንም። (ጥሩ) ለሰራው ይከፈላል፥ (መጥፎ) ለሰራው (ደግሞ) ይቀጣል። \rq\rq{}አምላካችን! ብንረሳ ወይም ስህተት ብንገባ አትቅጣነ። አምላካችን! ከኛ በፊት ለነበሩት (ይሁዶችና ክርስቲያኖች) እንደጫንከው አትጫንብን። አምላካችን! አቅማችን ከሚችለው በላይ አትጫነን፥ እለፈን፥ ይቅር በለን፥ ምህረት አድርግልን። አንተ መውላችን (አጋዣችን) ነህ እና ከማያምኑ (ከካሀዲዎች) ላይ ድልን ስጠን\rq\rq{} } &  لَا يُكَلِّفُ ٱللَّهُ نَفْسًا إِلَّا وُسْعَهَا ۚ لَهَا مَا كَسَبَتْ وَعَلَيْهَا مَا ٱكْتَسَبَتْ ۗ رَبَّنَا لَا تُؤَاخِذْنَآ إِن نَّسِينَآ أَوْ أَخْطَأْنَا ۚ رَبَّنَا وَلَا تَحْمِلْ عَلَيْنَآ إِصْرًۭا كَمَا حَمَلْتَهُۥ عَلَى ٱلَّذِينَ مِن قَبْلِنَا ۚ رَبَّنَا وَلَا تُحَمِّلْنَا مَا لَا طَاقَةَ لَنَا بِهِۦ ۖ وَٱعْفُ عَنَّا وَٱغْفِرْ لَنَا وَٱرْحَمْنَآ ۚ أَنتَ مَوْلَىٰنَا فَٱنصُرْنَا عَلَى ٱلْقَوْمِ ٱلْكَٟفِرِينَ ﴿٢٦﴾
\end{longtable} \newpage

%% License: BSD style (Berkley) (i.e. Put the Copyright owner's name always)
%% Writer and Copyright (to): Bewketu(Bilal) Tadilo (2016-17)
\centering\section{\LR{\textamharic{ሱራቱ አልኢምራን -}  \RL{سوره  عمران}}}
\begin{longtable}{%
  @{}
    p{.5\textwidth}
  @{~~~~~~~~~~~~~}
    p{.5\textwidth}
    @{}
}
\nopagebreak
\textamh{ቢስሚላሂ አራህመኒ ራሂይም } &  بِسْمِ ٱللَّهِ ٱلرَّحْمَـٰنِ ٱلرَّحِيمِ\\
\textamh{1.\  } &  الٓمٓ ﴿١﴾\\
\textamh{2.\  } & ٱللَّهُ لَآ إِلَـٰهَ إِلَّا هُوَ ٱلْحَىُّ ٱلْقَيُّومُ ﴿٢﴾\\
\textamh{3.\  } & نَزَّلَ عَلَيْكَ ٱلْكِتَـٰبَ بِٱلْحَقِّ مُصَدِّقًۭا لِّمَا بَيْنَ يَدَيْهِ وَأَنزَلَ ٱلتَّوْرَىٰةَ وَٱلْإِنجِيلَ ﴿٣﴾\\
\textamh{4.\  } & مِن قَبْلُ هُدًۭى لِّلنَّاسِ وَأَنزَلَ ٱلْفُرْقَانَ ۗ إِنَّ ٱلَّذِينَ كَفَرُوا۟ بِـَٔايَـٰتِ ٱللَّهِ لَهُمْ عَذَابٌۭ شَدِيدٌۭ ۗ وَٱللَّهُ عَزِيزٌۭ ذُو ٱنتِقَامٍ ﴿٤﴾\\
\textamh{5.\  } & إِنَّ ٱللَّهَ لَا يَخْفَىٰ عَلَيْهِ شَىْءٌۭ فِى ٱلْأَرْضِ وَلَا فِى ٱلسَّمَآءِ ﴿٥﴾\\
\textamh{6.\  } & هُوَ ٱلَّذِى يُصَوِّرُكُمْ فِى ٱلْأَرْحَامِ كَيْفَ يَشَآءُ ۚ لَآ إِلَـٰهَ إِلَّا هُوَ ٱلْعَزِيزُ ٱلْحَكِيمُ ﴿٦﴾\\
\textamh{7.\  } & هُوَ ٱلَّذِىٓ أَنزَلَ عَلَيْكَ ٱلْكِتَـٰبَ مِنْهُ ءَايَـٰتٌۭ مُّحْكَمَـٰتٌ هُنَّ أُمُّ ٱلْكِتَـٰبِ وَأُخَرُ مُتَشَـٰبِهَـٰتٌۭ ۖ فَأَمَّا ٱلَّذِينَ فِى قُلُوبِهِمْ زَيْغٌۭ فَيَتَّبِعُونَ مَا تَشَـٰبَهَ مِنْهُ ٱبْتِغَآءَ ٱلْفِتْنَةِ وَٱبْتِغَآءَ تَأْوِيلِهِۦ ۗ وَمَا يَعْلَمُ تَأْوِيلَهُۥٓ إِلَّا ٱللَّهُ ۗ وَٱلرَّٟسِخُونَ فِى ٱلْعِلْمِ يَقُولُونَ ءَامَنَّا بِهِۦ كُلٌّۭ مِّنْ عِندِ رَبِّنَا ۗ وَمَا يَذَّكَّرُ إِلَّآ أُو۟لُوا۟ ٱلْأَلْبَٰبِ ﴿٧﴾\\
\textamh{8.\  } & رَبَّنَا لَا تُزِغْ قُلُوبَنَا بَعْدَ إِذْ هَدَيْتَنَا وَهَبْ لَنَا مِن لَّدُنكَ رَحْمَةً ۚ إِنَّكَ أَنتَ ٱلْوَهَّابُ ﴿٨﴾\\
\textamh{9.\  } & رَبَّنَآ إِنَّكَ جَامِعُ ٱلنَّاسِ لِيَوْمٍۢ لَّا رَيْبَ فِيهِ ۚ إِنَّ ٱللَّهَ لَا يُخْلِفُ ٱلْمِيعَادَ ﴿٩﴾\\
\textamh{10.\  } & إِنَّ ٱلَّذِينَ كَفَرُوا۟ لَن تُغْنِىَ عَنْهُمْ أَمْوَٟلُهُمْ وَلَآ أَوْلَـٰدُهُم مِّنَ ٱللَّهِ شَيْـًۭٔا ۖ وَأُو۟لَـٰٓئِكَ هُمْ وَقُودُ ٱلنَّارِ ﴿١٠﴾\\
\textamh{11.\  } & كَدَأْبِ ءَالِ فِرْعَوْنَ وَٱلَّذِينَ مِن قَبْلِهِمْ ۚ كَذَّبُوا۟ بِـَٔايَـٰتِنَا فَأَخَذَهُمُ ٱللَّهُ بِذُنُوبِهِمْ ۗ وَٱللَّهُ شَدِيدُ ٱلْعِقَابِ ﴿١١﴾\\
\textamh{12.\  } & قُل لِّلَّذِينَ كَفَرُوا۟ سَتُغْلَبُونَ وَتُحْشَرُونَ إِلَىٰ جَهَنَّمَ ۚ وَبِئْسَ ٱلْمِهَادُ ﴿١٢﴾\\
\textamh{13.\  } & قَدْ كَانَ لَكُمْ ءَايَةٌۭ فِى فِئَتَيْنِ ٱلْتَقَتَا ۖ فِئَةٌۭ تُقَـٰتِلُ فِى سَبِيلِ ٱللَّهِ وَأُخْرَىٰ كَافِرَةٌۭ يَرَوْنَهُم مِّثْلَيْهِمْ رَأْىَ ٱلْعَيْنِ ۚ وَٱللَّهُ يُؤَيِّدُ بِنَصْرِهِۦ مَن يَشَآءُ ۗ إِنَّ فِى ذَٟلِكَ لَعِبْرَةًۭ لِّأُو۟لِى ٱلْأَبْصَـٰرِ ﴿١٣﴾\\
\textamh{14.\  } & زُيِّنَ لِلنَّاسِ حُبُّ ٱلشَّهَوَٟتِ مِنَ ٱلنِّسَآءِ وَٱلْبَنِينَ وَٱلْقَنَـٰطِيرِ ٱلْمُقَنطَرَةِ مِنَ ٱلذَّهَبِ وَٱلْفِضَّةِ وَٱلْخَيْلِ ٱلْمُسَوَّمَةِ وَٱلْأَنْعَـٰمِ وَٱلْحَرْثِ ۗ ذَٟلِكَ مَتَـٰعُ ٱلْحَيَوٰةِ ٱلدُّنْيَا ۖ وَٱللَّهُ عِندَهُۥ حُسْنُ ٱلْمَـَٔابِ ﴿١٤﴾\\
\textamh{15.\  } & ۞ قُلْ أَؤُنَبِّئُكُم بِخَيْرٍۢ مِّن ذَٟلِكُمْ ۚ لِلَّذِينَ ٱتَّقَوْا۟ عِندَ رَبِّهِمْ جَنَّـٰتٌۭ تَجْرِى مِن تَحْتِهَا ٱلْأَنْهَـٰرُ خَـٰلِدِينَ فِيهَا وَأَزْوَٟجٌۭ مُّطَهَّرَةٌۭ وَرِضْوَٟنٌۭ مِّنَ ٱللَّهِ ۗ وَٱللَّهُ بَصِيرٌۢ بِٱلْعِبَادِ ﴿١٥﴾\\
\textamh{16.\  } & ٱلَّذِينَ يَقُولُونَ رَبَّنَآ إِنَّنَآ ءَامَنَّا فَٱغْفِرْ لَنَا ذُنُوبَنَا وَقِنَا عَذَابَ ٱلنَّارِ ﴿١٦﴾\\
\textamh{17.\  } & ٱلصَّـٰبِرِينَ وَٱلصَّـٰدِقِينَ وَٱلْقَـٰنِتِينَ وَٱلْمُنفِقِينَ وَٱلْمُسْتَغْفِرِينَ بِٱلْأَسْحَارِ ﴿١٧﴾\\
\textamh{18.\  } & شَهِدَ ٱللَّهُ أَنَّهُۥ لَآ إِلَـٰهَ إِلَّا هُوَ وَٱلْمَلَـٰٓئِكَةُ وَأُو۟لُوا۟ ٱلْعِلْمِ قَآئِمًۢا بِٱلْقِسْطِ ۚ لَآ إِلَـٰهَ إِلَّا هُوَ ٱلْعَزِيزُ ٱلْحَكِيمُ ﴿١٨﴾\\
\textamh{19.\  } & إِنَّ ٱلدِّينَ عِندَ ٱللَّهِ ٱلْإِسْلَـٰمُ ۗ وَمَا ٱخْتَلَفَ ٱلَّذِينَ أُوتُوا۟ ٱلْكِتَـٰبَ إِلَّا مِنۢ بَعْدِ مَا جَآءَهُمُ ٱلْعِلْمُ بَغْيًۢا بَيْنَهُمْ ۗ وَمَن يَكْفُرْ بِـَٔايَـٰتِ ٱللَّهِ فَإِنَّ ٱللَّهَ سَرِيعُ ٱلْحِسَابِ ﴿١٩﴾\\
\textamh{20.\  } & فَإِنْ حَآجُّوكَ فَقُلْ أَسْلَمْتُ وَجْهِىَ لِلَّهِ وَمَنِ ٱتَّبَعَنِ ۗ وَقُل لِّلَّذِينَ أُوتُوا۟ ٱلْكِتَـٰبَ وَٱلْأُمِّيِّۦنَ ءَأَسْلَمْتُمْ ۚ فَإِنْ أَسْلَمُوا۟ فَقَدِ ٱهْتَدَوا۟ ۖ وَّإِن تَوَلَّوْا۟ فَإِنَّمَا عَلَيْكَ ٱلْبَلَـٰغُ ۗ وَٱللَّهُ بَصِيرٌۢ بِٱلْعِبَادِ ﴿٢٠﴾\\
\textamh{21.\  } & إِنَّ ٱلَّذِينَ يَكْفُرُونَ بِـَٔايَـٰتِ ٱللَّهِ وَيَقْتُلُونَ ٱلنَّبِيِّۦنَ بِغَيْرِ حَقٍّۢ وَيَقْتُلُونَ ٱلَّذِينَ يَأْمُرُونَ بِٱلْقِسْطِ مِنَ ٱلنَّاسِ فَبَشِّرْهُم بِعَذَابٍ أَلِيمٍ ﴿٢١﴾\\
\textamh{22.\  } & أُو۟لَـٰٓئِكَ ٱلَّذِينَ حَبِطَتْ أَعْمَـٰلُهُمْ فِى ٱلدُّنْيَا وَٱلْءَاخِرَةِ وَمَا لَهُم مِّن نَّـٰصِرِينَ ﴿٢٢﴾\\
\textamh{23.\  } & أَلَمْ تَرَ إِلَى ٱلَّذِينَ أُوتُوا۟ نَصِيبًۭا مِّنَ ٱلْكِتَـٰبِ يُدْعَوْنَ إِلَىٰ كِتَـٰبِ ٱللَّهِ لِيَحْكُمَ بَيْنَهُمْ ثُمَّ يَتَوَلَّىٰ فَرِيقٌۭ مِّنْهُمْ وَهُم مُّعْرِضُونَ ﴿٢٣﴾\\
\textamh{24.\  } & ذَٟلِكَ بِأَنَّهُمْ قَالُوا۟ لَن تَمَسَّنَا ٱلنَّارُ إِلَّآ أَيَّامًۭا مَّعْدُودَٟتٍۢ ۖ وَغَرَّهُمْ فِى دِينِهِم مَّا كَانُوا۟ يَفْتَرُونَ ﴿٢٤﴾\\
\textamh{25.\  } & فَكَيْفَ إِذَا جَمَعْنَـٰهُمْ لِيَوْمٍۢ لَّا رَيْبَ فِيهِ وَوُفِّيَتْ كُلُّ نَفْسٍۢ مَّا كَسَبَتْ وَهُمْ لَا يُظْلَمُونَ ﴿٢٥﴾\\
\textamh{26.\  } & قُلِ ٱللَّهُمَّ مَـٰلِكَ ٱلْمُلْكِ تُؤْتِى ٱلْمُلْكَ مَن تَشَآءُ وَتَنزِعُ ٱلْمُلْكَ مِمَّن تَشَآءُ وَتُعِزُّ مَن تَشَآءُ وَتُذِلُّ مَن تَشَآءُ ۖ بِيَدِكَ ٱلْخَيْرُ ۖ إِنَّكَ عَلَىٰ كُلِّ شَىْءٍۢ قَدِيرٌۭ ﴿٢٦﴾\\
\textamh{27.\  } & تُولِجُ ٱلَّيْلَ فِى ٱلنَّهَارِ وَتُولِجُ ٱلنَّهَارَ فِى ٱلَّيْلِ ۖ وَتُخْرِجُ ٱلْحَىَّ مِنَ ٱلْمَيِّتِ وَتُخْرِجُ ٱلْمَيِّتَ مِنَ ٱلْحَىِّ ۖ وَتَرْزُقُ مَن تَشَآءُ بِغَيْرِ حِسَابٍۢ ﴿٢٧﴾\\
\textamh{28.\  } & لَّا يَتَّخِذِ ٱلْمُؤْمِنُونَ ٱلْكَـٰفِرِينَ أَوْلِيَآءَ مِن دُونِ ٱلْمُؤْمِنِينَ ۖ وَمَن يَفْعَلْ ذَٟلِكَ فَلَيْسَ مِنَ ٱللَّهِ فِى شَىْءٍ إِلَّآ أَن تَتَّقُوا۟ مِنْهُمْ تُقَىٰةًۭ ۗ وَيُحَذِّرُكُمُ ٱللَّهُ نَفْسَهُۥ ۗ وَإِلَى ٱللَّهِ ٱلْمَصِيرُ ﴿٢٨﴾\\
\textamh{29.\  } & قُلْ إِن تُخْفُوا۟ مَا فِى صُدُورِكُمْ أَوْ تُبْدُوهُ يَعْلَمْهُ ٱللَّهُ ۗ وَيَعْلَمُ مَا فِى ٱلسَّمَـٰوَٟتِ وَمَا فِى ٱلْأَرْضِ ۗ وَٱللَّهُ عَلَىٰ كُلِّ شَىْءٍۢ قَدِيرٌۭ ﴿٢٩﴾\\
\textamh{30.\  } & يَوْمَ تَجِدُ كُلُّ نَفْسٍۢ مَّا عَمِلَتْ مِنْ خَيْرٍۢ مُّحْضَرًۭا وَمَا عَمِلَتْ مِن سُوٓءٍۢ تَوَدُّ لَوْ أَنَّ بَيْنَهَا وَبَيْنَهُۥٓ أَمَدًۢا بَعِيدًۭا ۗ وَيُحَذِّرُكُمُ ٱللَّهُ نَفْسَهُۥ ۗ وَٱللَّهُ رَءُوفٌۢ بِٱلْعِبَادِ ﴿٣٠﴾\\
\textamh{31.\  } & قُلْ إِن كُنتُمْ تُحِبُّونَ ٱللَّهَ فَٱتَّبِعُونِى يُحْبِبْكُمُ ٱللَّهُ وَيَغْفِرْ لَكُمْ ذُنُوبَكُمْ ۗ وَٱللَّهُ غَفُورٌۭ رَّحِيمٌۭ ﴿٣١﴾\\
\textamh{32.\  } & قُلْ أَطِيعُوا۟ ٱللَّهَ وَٱلرَّسُولَ ۖ فَإِن تَوَلَّوْا۟ فَإِنَّ ٱللَّهَ لَا يُحِبُّ ٱلْكَـٰفِرِينَ ﴿٣٢﴾\\
\textamh{33.\  } & ۞ إِنَّ ٱللَّهَ ٱصْطَفَىٰٓ ءَادَمَ وَنُوحًۭا وَءَالَ إِبْرَٰهِيمَ وَءَالَ عِمْرَٰنَ عَلَى ٱلْعَـٰلَمِينَ ﴿٣٣﴾\\
\textamh{34.\  } & ذُرِّيَّةًۢ بَعْضُهَا مِنۢ بَعْضٍۢ ۗ وَٱللَّهُ سَمِيعٌ عَلِيمٌ ﴿٣٤﴾\\
\textamh{35.\  } & إِذْ قَالَتِ ٱمْرَأَتُ عِمْرَٰنَ رَبِّ إِنِّى نَذَرْتُ لَكَ مَا فِى بَطْنِى مُحَرَّرًۭا فَتَقَبَّلْ مِنِّىٓ ۖ إِنَّكَ أَنتَ ٱلسَّمِيعُ ٱلْعَلِيمُ ﴿٣٥﴾\\
\textamh{36.\  } & فَلَمَّا وَضَعَتْهَا قَالَتْ رَبِّ إِنِّى وَضَعْتُهَآ أُنثَىٰ وَٱللَّهُ أَعْلَمُ بِمَا وَضَعَتْ وَلَيْسَ ٱلذَّكَرُ كَٱلْأُنثَىٰ ۖ وَإِنِّى سَمَّيْتُهَا مَرْيَمَ وَإِنِّىٓ أُعِيذُهَا بِكَ وَذُرِّيَّتَهَا مِنَ ٱلشَّيْطَٰنِ ٱلرَّجِيمِ ﴿٣٦﴾\\
\textamh{37.\  } & فَتَقَبَّلَهَا رَبُّهَا بِقَبُولٍ حَسَنٍۢ وَأَنۢبَتَهَا نَبَاتًا حَسَنًۭا وَكَفَّلَهَا زَكَرِيَّا ۖ كُلَّمَا دَخَلَ عَلَيْهَا زَكَرِيَّا ٱلْمِحْرَابَ وَجَدَ عِندَهَا رِزْقًۭا ۖ قَالَ يَـٰمَرْيَمُ أَنَّىٰ لَكِ هَـٰذَا ۖ قَالَتْ هُوَ مِنْ عِندِ ٱللَّهِ ۖ إِنَّ ٱللَّهَ يَرْزُقُ مَن يَشَآءُ بِغَيْرِ حِسَابٍ ﴿٣٧﴾\\
\textamh{38.\  } & هُنَالِكَ دَعَا زَكَرِيَّا رَبَّهُۥ ۖ قَالَ رَبِّ هَبْ لِى مِن لَّدُنكَ ذُرِّيَّةًۭ طَيِّبَةً ۖ إِنَّكَ سَمِيعُ ٱلدُّعَآءِ ﴿٣٨﴾\\
\textamh{39.\  } & فَنَادَتْهُ ٱلْمَلَـٰٓئِكَةُ وَهُوَ قَآئِمٌۭ يُصَلِّى فِى ٱلْمِحْرَابِ أَنَّ ٱللَّهَ يُبَشِّرُكَ بِيَحْيَىٰ مُصَدِّقًۢا بِكَلِمَةٍۢ مِّنَ ٱللَّهِ وَسَيِّدًۭا وَحَصُورًۭا وَنَبِيًّۭا مِّنَ ٱلصَّـٰلِحِينَ ﴿٣٩﴾\\
\textamh{40.\  } & قَالَ رَبِّ أَنَّىٰ يَكُونُ لِى غُلَـٰمٌۭ وَقَدْ بَلَغَنِىَ ٱلْكِبَرُ وَٱمْرَأَتِى عَاقِرٌۭ ۖ قَالَ كَذَٟلِكَ ٱللَّهُ يَفْعَلُ مَا يَشَآءُ ﴿٤٠﴾\\
\textamh{41.\  } & قَالَ رَبِّ ٱجْعَل لِّىٓ ءَايَةًۭ ۖ قَالَ ءَايَتُكَ أَلَّا تُكَلِّمَ ٱلنَّاسَ ثَلَـٰثَةَ أَيَّامٍ إِلَّا رَمْزًۭا ۗ وَٱذْكُر رَّبَّكَ كَثِيرًۭا وَسَبِّحْ بِٱلْعَشِىِّ وَٱلْإِبْكَـٰرِ ﴿٤١﴾\\
\textamh{42.\  } & وَإِذْ قَالَتِ ٱلْمَلَـٰٓئِكَةُ يَـٰمَرْيَمُ إِنَّ ٱللَّهَ ٱصْطَفَىٰكِ وَطَهَّرَكِ وَٱصْطَفَىٰكِ عَلَىٰ نِسَآءِ ٱلْعَـٰلَمِينَ ﴿٤٢﴾\\
\textamh{43.\  } & يَـٰمَرْيَمُ ٱقْنُتِى لِرَبِّكِ وَٱسْجُدِى وَٱرْكَعِى مَعَ ٱلرَّٟكِعِينَ ﴿٤٣﴾\\
\textamh{44.\  } & ذَٟلِكَ مِنْ أَنۢبَآءِ ٱلْغَيْبِ نُوحِيهِ إِلَيْكَ ۚ وَمَا كُنتَ لَدَيْهِمْ إِذْ يُلْقُونَ أَقْلَـٰمَهُمْ أَيُّهُمْ يَكْفُلُ مَرْيَمَ وَمَا كُنتَ لَدَيْهِمْ إِذْ يَخْتَصِمُونَ ﴿٤٤﴾\\
\textamh{45.\  } & إِذْ قَالَتِ ٱلْمَلَـٰٓئِكَةُ يَـٰمَرْيَمُ إِنَّ ٱللَّهَ يُبَشِّرُكِ بِكَلِمَةٍۢ مِّنْهُ ٱسْمُهُ ٱلْمَسِيحُ عِيسَى ٱبْنُ مَرْيَمَ وَجِيهًۭا فِى ٱلدُّنْيَا وَٱلْءَاخِرَةِ وَمِنَ ٱلْمُقَرَّبِينَ ﴿٤٥﴾\\
\textamh{46.\  } & وَيُكَلِّمُ ٱلنَّاسَ فِى ٱلْمَهْدِ وَكَهْلًۭا وَمِنَ ٱلصَّـٰلِحِينَ ﴿٤٦﴾\\
\textamh{47.\  } & قَالَتْ رَبِّ أَنَّىٰ يَكُونُ لِى وَلَدٌۭ وَلَمْ يَمْسَسْنِى بَشَرٌۭ ۖ قَالَ كَذَٟلِكِ ٱللَّهُ يَخْلُقُ مَا يَشَآءُ ۚ إِذَا قَضَىٰٓ أَمْرًۭا فَإِنَّمَا يَقُولُ لَهُۥ كُن فَيَكُونُ ﴿٤٧﴾\\
\textamh{48.\  } & وَيُعَلِّمُهُ ٱلْكِتَـٰبَ وَٱلْحِكْمَةَ وَٱلتَّوْرَىٰةَ وَٱلْإِنجِيلَ ﴿٤٨﴾\\
\textamh{49.\  } & وَرَسُولًا إِلَىٰ بَنِىٓ إِسْرَٰٓءِيلَ أَنِّى قَدْ جِئْتُكُم بِـَٔايَةٍۢ مِّن رَّبِّكُمْ ۖ أَنِّىٓ أَخْلُقُ لَكُم مِّنَ ٱلطِّينِ كَهَيْـَٔةِ ٱلطَّيْرِ فَأَنفُخُ فِيهِ فَيَكُونُ طَيْرًۢا بِإِذْنِ ٱللَّهِ ۖ وَأُبْرِئُ ٱلْأَكْمَهَ وَٱلْأَبْرَصَ وَأُحْىِ ٱلْمَوْتَىٰ بِإِذْنِ ٱللَّهِ ۖ وَأُنَبِّئُكُم بِمَا تَأْكُلُونَ وَمَا تَدَّخِرُونَ فِى بُيُوتِكُمْ ۚ إِنَّ فِى ذَٟلِكَ لَءَايَةًۭ لَّكُمْ إِن كُنتُم مُّؤْمِنِينَ ﴿٤٩﴾\\
\textamh{50.\  } & وَمُصَدِّقًۭا لِّمَا بَيْنَ يَدَىَّ مِنَ ٱلتَّوْرَىٰةِ وَلِأُحِلَّ لَكُم بَعْضَ ٱلَّذِى حُرِّمَ عَلَيْكُمْ ۚ وَجِئْتُكُم بِـَٔايَةٍۢ مِّن رَّبِّكُمْ فَٱتَّقُوا۟ ٱللَّهَ وَأَطِيعُونِ ﴿٥٠﴾\\
\textamh{51.\  } & إِنَّ ٱللَّهَ رَبِّى وَرَبُّكُمْ فَٱعْبُدُوهُ ۗ هَـٰذَا صِرَٰطٌۭ مُّسْتَقِيمٌۭ ﴿٥١﴾\\
\textamh{52.\  } & ۞ فَلَمَّآ أَحَسَّ عِيسَىٰ مِنْهُمُ ٱلْكُفْرَ قَالَ مَنْ أَنصَارِىٓ إِلَى ٱللَّهِ ۖ قَالَ ٱلْحَوَارِيُّونَ نَحْنُ أَنصَارُ ٱللَّهِ ءَامَنَّا بِٱللَّهِ وَٱشْهَدْ بِأَنَّا مُسْلِمُونَ ﴿٥٢﴾\\
\textamh{53.\  } & رَبَّنَآ ءَامَنَّا بِمَآ أَنزَلْتَ وَٱتَّبَعْنَا ٱلرَّسُولَ فَٱكْتُبْنَا مَعَ ٱلشَّـٰهِدِينَ ﴿٥٣﴾\\
\textamh{54.\  } & وَمَكَرُوا۟ وَمَكَرَ ٱللَّهُ ۖ وَٱللَّهُ خَيْرُ ٱلْمَـٰكِرِينَ ﴿٥٤﴾\\
\textamh{55.\  } & إِذْ قَالَ ٱللَّهُ يَـٰعِيسَىٰٓ إِنِّى مُتَوَفِّيكَ وَرَافِعُكَ إِلَىَّ وَمُطَهِّرُكَ مِنَ ٱلَّذِينَ كَفَرُوا۟ وَجَاعِلُ ٱلَّذِينَ ٱتَّبَعُوكَ فَوْقَ ٱلَّذِينَ كَفَرُوٓا۟ إِلَىٰ يَوْمِ ٱلْقِيَـٰمَةِ ۖ ثُمَّ إِلَىَّ مَرْجِعُكُمْ فَأَحْكُمُ بَيْنَكُمْ فِيمَا كُنتُمْ فِيهِ تَخْتَلِفُونَ ﴿٥٥﴾\\
\textamh{56.\  } & فَأَمَّا ٱلَّذِينَ كَفَرُوا۟ فَأُعَذِّبُهُمْ عَذَابًۭا شَدِيدًۭا فِى ٱلدُّنْيَا وَٱلْءَاخِرَةِ وَمَا لَهُم مِّن نَّـٰصِرِينَ ﴿٥٦﴾\\
\textamh{57.\  } & وَأَمَّا ٱلَّذِينَ ءَامَنُوا۟ وَعَمِلُوا۟ ٱلصَّـٰلِحَـٰتِ فَيُوَفِّيهِمْ أُجُورَهُمْ ۗ وَٱللَّهُ لَا يُحِبُّ ٱلظَّـٰلِمِينَ ﴿٥٧﴾\\
\textamh{58.\  } & ذَٟلِكَ نَتْلُوهُ عَلَيْكَ مِنَ ٱلْءَايَـٰتِ وَٱلذِّكْرِ ٱلْحَكِيمِ ﴿٥٨﴾\\
\textamh{59.\  } & إِنَّ مَثَلَ عِيسَىٰ عِندَ ٱللَّهِ كَمَثَلِ ءَادَمَ ۖ خَلَقَهُۥ مِن تُرَابٍۢ ثُمَّ قَالَ لَهُۥ كُن فَيَكُونُ ﴿٥٩﴾\\
\textamh{60.\  } & ٱلْحَقُّ مِن رَّبِّكَ فَلَا تَكُن مِّنَ ٱلْمُمْتَرِينَ ﴿٦٠﴾\\
\textamh{61.\  } & فَمَنْ حَآجَّكَ فِيهِ مِنۢ بَعْدِ مَا جَآءَكَ مِنَ ٱلْعِلْمِ فَقُلْ تَعَالَوْا۟ نَدْعُ أَبْنَآءَنَا وَأَبْنَآءَكُمْ وَنِسَآءَنَا وَنِسَآءَكُمْ وَأَنفُسَنَا وَأَنفُسَكُمْ ثُمَّ نَبْتَهِلْ فَنَجْعَل لَّعْنَتَ ٱللَّهِ عَلَى ٱلْكَـٰذِبِينَ ﴿٦١﴾\\
\textamh{62.\  } & إِنَّ هَـٰذَا لَهُوَ ٱلْقَصَصُ ٱلْحَقُّ ۚ وَمَا مِنْ إِلَـٰهٍ إِلَّا ٱللَّهُ ۚ وَإِنَّ ٱللَّهَ لَهُوَ ٱلْعَزِيزُ ٱلْحَكِيمُ ﴿٦٢﴾\\
\textamh{63.\  } & فَإِن تَوَلَّوْا۟ فَإِنَّ ٱللَّهَ عَلِيمٌۢ بِٱلْمُفْسِدِينَ ﴿٦٣﴾\\
\textamh{64.\  } & قُلْ يَـٰٓأَهْلَ ٱلْكِتَـٰبِ تَعَالَوْا۟ إِلَىٰ كَلِمَةٍۢ سَوَآءٍۭ بَيْنَنَا وَبَيْنَكُمْ أَلَّا نَعْبُدَ إِلَّا ٱللَّهَ وَلَا نُشْرِكَ بِهِۦ شَيْـًۭٔا وَلَا يَتَّخِذَ بَعْضُنَا بَعْضًا أَرْبَابًۭا مِّن دُونِ ٱللَّهِ ۚ فَإِن تَوَلَّوْا۟ فَقُولُوا۟ ٱشْهَدُوا۟ بِأَنَّا مُسْلِمُونَ ﴿٦٤﴾\\
\textamh{65.\  } & يَـٰٓأَهْلَ ٱلْكِتَـٰبِ لِمَ تُحَآجُّونَ فِىٓ إِبْرَٰهِيمَ وَمَآ أُنزِلَتِ ٱلتَّوْرَىٰةُ وَٱلْإِنجِيلُ إِلَّا مِنۢ بَعْدِهِۦٓ ۚ أَفَلَا تَعْقِلُونَ ﴿٦٥﴾\\
\textamh{66.\  } & هَـٰٓأَنتُمْ هَـٰٓؤُلَآءِ حَـٰجَجْتُمْ فِيمَا لَكُم بِهِۦ عِلْمٌۭ فَلِمَ تُحَآجُّونَ فِيمَا لَيْسَ لَكُم بِهِۦ عِلْمٌۭ ۚ وَٱللَّهُ يَعْلَمُ وَأَنتُمْ لَا تَعْلَمُونَ ﴿٦٦﴾\\
\textamh{67.\  } & مَا كَانَ إِبْرَٰهِيمُ يَهُودِيًّۭا وَلَا نَصْرَانِيًّۭا وَلَـٰكِن كَانَ حَنِيفًۭا مُّسْلِمًۭا وَمَا كَانَ مِنَ ٱلْمُشْرِكِينَ ﴿٦٧﴾\\
\textamh{68.\  } & إِنَّ أَوْلَى ٱلنَّاسِ بِإِبْرَٰهِيمَ لَلَّذِينَ ٱتَّبَعُوهُ وَهَـٰذَا ٱلنَّبِىُّ وَٱلَّذِينَ ءَامَنُوا۟ ۗ وَٱللَّهُ وَلِىُّ ٱلْمُؤْمِنِينَ ﴿٦٨﴾\\
\textamh{69.\  } & وَدَّت طَّآئِفَةٌۭ مِّنْ أَهْلِ ٱلْكِتَـٰبِ لَوْ يُضِلُّونَكُمْ وَمَا يُضِلُّونَ إِلَّآ أَنفُسَهُمْ وَمَا يَشْعُرُونَ ﴿٦٩﴾\\
\textamh{70.\  } & يَـٰٓأَهْلَ ٱلْكِتَـٰبِ لِمَ تَكْفُرُونَ بِـَٔايَـٰتِ ٱللَّهِ وَأَنتُمْ تَشْهَدُونَ ﴿٧٠﴾\\
\textamh{71.\  } & يَـٰٓأَهْلَ ٱلْكِتَـٰبِ لِمَ تَلْبِسُونَ ٱلْحَقَّ بِٱلْبَٰطِلِ وَتَكْتُمُونَ ٱلْحَقَّ وَأَنتُمْ تَعْلَمُونَ ﴿٧١﴾\\
\textamh{72.\  } & وَقَالَت طَّآئِفَةٌۭ مِّنْ أَهْلِ ٱلْكِتَـٰبِ ءَامِنُوا۟ بِٱلَّذِىٓ أُنزِلَ عَلَى ٱلَّذِينَ ءَامَنُوا۟ وَجْهَ ٱلنَّهَارِ وَٱكْفُرُوٓا۟ ءَاخِرَهُۥ لَعَلَّهُمْ يَرْجِعُونَ ﴿٧٢﴾\\
\textamh{73.\  } & وَلَا تُؤْمِنُوٓا۟ إِلَّا لِمَن تَبِعَ دِينَكُمْ قُلْ إِنَّ ٱلْهُدَىٰ هُدَى ٱللَّهِ أَن يُؤْتَىٰٓ أَحَدٌۭ مِّثْلَ مَآ أُوتِيتُمْ أَوْ يُحَآجُّوكُمْ عِندَ رَبِّكُمْ ۗ قُلْ إِنَّ ٱلْفَضْلَ بِيَدِ ٱللَّهِ يُؤْتِيهِ مَن يَشَآءُ ۗ وَٱللَّهُ وَٟسِعٌ عَلِيمٌۭ ﴿٧٣﴾\\
\textamh{74.\  } & يَخْتَصُّ بِرَحْمَتِهِۦ مَن يَشَآءُ ۗ وَٱللَّهُ ذُو ٱلْفَضْلِ ٱلْعَظِيمِ ﴿٧٤﴾\\
\textamh{75.\  } & ۞ وَمِنْ أَهْلِ ٱلْكِتَـٰبِ مَنْ إِن تَأْمَنْهُ بِقِنطَارٍۢ يُؤَدِّهِۦٓ إِلَيْكَ وَمِنْهُم مَّنْ إِن تَأْمَنْهُ بِدِينَارٍۢ لَّا يُؤَدِّهِۦٓ إِلَيْكَ إِلَّا مَا دُمْتَ عَلَيْهِ قَآئِمًۭا ۗ ذَٟلِكَ بِأَنَّهُمْ قَالُوا۟ لَيْسَ عَلَيْنَا فِى ٱلْأُمِّيِّۦنَ سَبِيلٌۭ وَيَقُولُونَ عَلَى ٱللَّهِ ٱلْكَذِبَ وَهُمْ يَعْلَمُونَ ﴿٧٥﴾\\
\textamh{76.\  } & بَلَىٰ مَنْ أَوْفَىٰ بِعَهْدِهِۦ وَٱتَّقَىٰ فَإِنَّ ٱللَّهَ يُحِبُّ ٱلْمُتَّقِينَ ﴿٧٦﴾\\
\textamh{77.\  } & إِنَّ ٱلَّذِينَ يَشْتَرُونَ بِعَهْدِ ٱللَّهِ وَأَيْمَـٰنِهِمْ ثَمَنًۭا قَلِيلًا أُو۟لَـٰٓئِكَ لَا خَلَـٰقَ لَهُمْ فِى ٱلْءَاخِرَةِ وَلَا يُكَلِّمُهُمُ ٱللَّهُ وَلَا يَنظُرُ إِلَيْهِمْ يَوْمَ ٱلْقِيَـٰمَةِ وَلَا يُزَكِّيهِمْ وَلَهُمْ عَذَابٌ أَلِيمٌۭ ﴿٧٧﴾\\
\textamh{78.\  } & وَإِنَّ مِنْهُمْ لَفَرِيقًۭا يَلْوُۥنَ أَلْسِنَتَهُم بِٱلْكِتَـٰبِ لِتَحْسَبُوهُ مِنَ ٱلْكِتَـٰبِ وَمَا هُوَ مِنَ ٱلْكِتَـٰبِ وَيَقُولُونَ هُوَ مِنْ عِندِ ٱللَّهِ وَمَا هُوَ مِنْ عِندِ ٱللَّهِ وَيَقُولُونَ عَلَى ٱللَّهِ ٱلْكَذِبَ وَهُمْ يَعْلَمُونَ ﴿٧٨﴾\\
\textamh{79.\  } & مَا كَانَ لِبَشَرٍ أَن يُؤْتِيَهُ ٱللَّهُ ٱلْكِتَـٰبَ وَٱلْحُكْمَ وَٱلنُّبُوَّةَ ثُمَّ يَقُولَ لِلنَّاسِ كُونُوا۟ عِبَادًۭا لِّى مِن دُونِ ٱللَّهِ وَلَـٰكِن كُونُوا۟ رَبَّـٰنِيِّۦنَ بِمَا كُنتُمْ تُعَلِّمُونَ ٱلْكِتَـٰبَ وَبِمَا كُنتُمْ تَدْرُسُونَ ﴿٧٩﴾\\
\textamh{80.\  } & وَلَا يَأْمُرَكُمْ أَن تَتَّخِذُوا۟ ٱلْمَلَـٰٓئِكَةَ وَٱلنَّبِيِّۦنَ أَرْبَابًا ۗ أَيَأْمُرُكُم بِٱلْكُفْرِ بَعْدَ إِذْ أَنتُم مُّسْلِمُونَ ﴿٨٠﴾\\
\textamh{81.\  } & وَإِذْ أَخَذَ ٱللَّهُ مِيثَـٰقَ ٱلنَّبِيِّۦنَ لَمَآ ءَاتَيْتُكُم مِّن كِتَـٰبٍۢ وَحِكْمَةٍۢ ثُمَّ جَآءَكُمْ رَسُولٌۭ مُّصَدِّقٌۭ لِّمَا مَعَكُمْ لَتُؤْمِنُنَّ بِهِۦ وَلَتَنصُرُنَّهُۥ ۚ قَالَ ءَأَقْرَرْتُمْ وَأَخَذْتُمْ عَلَىٰ ذَٟلِكُمْ إِصْرِى ۖ قَالُوٓا۟ أَقْرَرْنَا ۚ قَالَ فَٱشْهَدُوا۟ وَأَنَا۠ مَعَكُم مِّنَ ٱلشَّـٰهِدِينَ ﴿٨١﴾\\
\textamh{82.\  } & فَمَن تَوَلَّىٰ بَعْدَ ذَٟلِكَ فَأُو۟لَـٰٓئِكَ هُمُ ٱلْفَـٰسِقُونَ ﴿٨٢﴾\\
\textamh{83.\  } & أَفَغَيْرَ دِينِ ٱللَّهِ يَبْغُونَ وَلَهُۥٓ أَسْلَمَ مَن فِى ٱلسَّمَـٰوَٟتِ وَٱلْأَرْضِ طَوْعًۭا وَكَرْهًۭا وَإِلَيْهِ يُرْجَعُونَ ﴿٨٣﴾\\
\textamh{84.\  } & قُلْ ءَامَنَّا بِٱللَّهِ وَمَآ أُنزِلَ عَلَيْنَا وَمَآ أُنزِلَ عَلَىٰٓ إِبْرَٰهِيمَ وَإِسْمَـٰعِيلَ وَإِسْحَـٰقَ وَيَعْقُوبَ وَٱلْأَسْبَاطِ وَمَآ أُوتِىَ مُوسَىٰ وَعِيسَىٰ وَٱلنَّبِيُّونَ مِن رَّبِّهِمْ لَا نُفَرِّقُ بَيْنَ أَحَدٍۢ مِّنْهُمْ وَنَحْنُ لَهُۥ مُسْلِمُونَ ﴿٨٤﴾\\
\textamh{85.\  } & وَمَن يَبْتَغِ غَيْرَ ٱلْإِسْلَـٰمِ دِينًۭا فَلَن يُقْبَلَ مِنْهُ وَهُوَ فِى ٱلْءَاخِرَةِ مِنَ ٱلْخَـٰسِرِينَ ﴿٨٥﴾\\
\textamh{86.\  } & كَيْفَ يَهْدِى ٱللَّهُ قَوْمًۭا كَفَرُوا۟ بَعْدَ إِيمَـٰنِهِمْ وَشَهِدُوٓا۟ أَنَّ ٱلرَّسُولَ حَقٌّۭ وَجَآءَهُمُ ٱلْبَيِّنَـٰتُ ۚ وَٱللَّهُ لَا يَهْدِى ٱلْقَوْمَ ٱلظَّـٰلِمِينَ ﴿٨٦﴾\\
\textamh{87.\  } & أُو۟لَـٰٓئِكَ جَزَآؤُهُمْ أَنَّ عَلَيْهِمْ لَعْنَةَ ٱللَّهِ وَٱلْمَلَـٰٓئِكَةِ وَٱلنَّاسِ أَجْمَعِينَ ﴿٨٧﴾\\
\textamh{88.\  } & خَـٰلِدِينَ فِيهَا لَا يُخَفَّفُ عَنْهُمُ ٱلْعَذَابُ وَلَا هُمْ يُنظَرُونَ ﴿٨٨﴾\\
\textamh{89.\  } & إِلَّا ٱلَّذِينَ تَابُوا۟ مِنۢ بَعْدِ ذَٟلِكَ وَأَصْلَحُوا۟ فَإِنَّ ٱللَّهَ غَفُورٌۭ رَّحِيمٌ ﴿٨٩﴾\\
\textamh{90.\  } & إِنَّ ٱلَّذِينَ كَفَرُوا۟ بَعْدَ إِيمَـٰنِهِمْ ثُمَّ ٱزْدَادُوا۟ كُفْرًۭا لَّن تُقْبَلَ تَوْبَتُهُمْ وَأُو۟لَـٰٓئِكَ هُمُ ٱلضَّآلُّونَ ﴿٩٠﴾\\
\textamh{91.\  } & إِنَّ ٱلَّذِينَ كَفَرُوا۟ وَمَاتُوا۟ وَهُمْ كُفَّارٌۭ فَلَن يُقْبَلَ مِنْ أَحَدِهِم مِّلْءُ ٱلْأَرْضِ ذَهَبًۭا وَلَوِ ٱفْتَدَىٰ بِهِۦٓ ۗ أُو۟لَـٰٓئِكَ لَهُمْ عَذَابٌ أَلِيمٌۭ وَمَا لَهُم مِّن نَّـٰصِرِينَ ﴿٩١﴾\\
\textamh{92.\  } & لَن تَنَالُوا۟ ٱلْبِرَّ حَتَّىٰ تُنفِقُوا۟ مِمَّا تُحِبُّونَ ۚ وَمَا تُنفِقُوا۟ مِن شَىْءٍۢ فَإِنَّ ٱللَّهَ بِهِۦ عَلِيمٌۭ ﴿٩٢﴾\\
\textamh{93.\  } & ۞ كُلُّ ٱلطَّعَامِ كَانَ حِلًّۭا لِّبَنِىٓ إِسْرَٰٓءِيلَ إِلَّا مَا حَرَّمَ إِسْرَٰٓءِيلُ عَلَىٰ نَفْسِهِۦ مِن قَبْلِ أَن تُنَزَّلَ ٱلتَّوْرَىٰةُ ۗ قُلْ فَأْتُوا۟ بِٱلتَّوْرَىٰةِ فَٱتْلُوهَآ إِن كُنتُمْ صَـٰدِقِينَ ﴿٩٣﴾\\
\textamh{94.\  } & فَمَنِ ٱفْتَرَىٰ عَلَى ٱللَّهِ ٱلْكَذِبَ مِنۢ بَعْدِ ذَٟلِكَ فَأُو۟لَـٰٓئِكَ هُمُ ٱلظَّـٰلِمُونَ ﴿٩٤﴾\\
\textamh{95.\  } & قُلْ صَدَقَ ٱللَّهُ ۗ فَٱتَّبِعُوا۟ مِلَّةَ إِبْرَٰهِيمَ حَنِيفًۭا وَمَا كَانَ مِنَ ٱلْمُشْرِكِينَ ﴿٩٥﴾\\
\textamh{96.\  } & إِنَّ أَوَّلَ بَيْتٍۢ وُضِعَ لِلنَّاسِ لَلَّذِى بِبَكَّةَ مُبَارَكًۭا وَهُدًۭى لِّلْعَـٰلَمِينَ ﴿٩٦﴾\\
\textamh{97.\  } & فِيهِ ءَايَـٰتٌۢ بَيِّنَـٰتٌۭ مَّقَامُ إِبْرَٰهِيمَ ۖ وَمَن دَخَلَهُۥ كَانَ ءَامِنًۭا ۗ وَلِلَّهِ عَلَى ٱلنَّاسِ حِجُّ ٱلْبَيْتِ مَنِ ٱسْتَطَاعَ إِلَيْهِ سَبِيلًۭا ۚ وَمَن كَفَرَ فَإِنَّ ٱللَّهَ غَنِىٌّ عَنِ ٱلْعَـٰلَمِينَ ﴿٩٧﴾\\
\textamh{98.\  } & قُلْ يَـٰٓأَهْلَ ٱلْكِتَـٰبِ لِمَ تَكْفُرُونَ بِـَٔايَـٰتِ ٱللَّهِ وَٱللَّهُ شَهِيدٌ عَلَىٰ مَا تَعْمَلُونَ ﴿٩٨﴾\\
\textamh{99.\  } & قُلْ يَـٰٓأَهْلَ ٱلْكِتَـٰبِ لِمَ تَصُدُّونَ عَن سَبِيلِ ٱللَّهِ مَنْ ءَامَنَ تَبْغُونَهَا عِوَجًۭا وَأَنتُمْ شُهَدَآءُ ۗ وَمَا ٱللَّهُ بِغَٰفِلٍ عَمَّا تَعْمَلُونَ ﴿٩٩﴾\\
\textamh{100.\  } & يَـٰٓأَيُّهَا ٱلَّذِينَ ءَامَنُوٓا۟ إِن تُطِيعُوا۟ فَرِيقًۭا مِّنَ ٱلَّذِينَ أُوتُوا۟ ٱلْكِتَـٰبَ يَرُدُّوكُم بَعْدَ إِيمَـٰنِكُمْ كَـٰفِرِينَ ﴿١٠٠﴾\\
\textamh{101.\  } & وَكَيْفَ تَكْفُرُونَ وَأَنتُمْ تُتْلَىٰ عَلَيْكُمْ ءَايَـٰتُ ٱللَّهِ وَفِيكُمْ رَسُولُهُۥ ۗ وَمَن يَعْتَصِم بِٱللَّهِ فَقَدْ هُدِىَ إِلَىٰ صِرَٰطٍۢ مُّسْتَقِيمٍۢ ﴿١٠١﴾\\
\textamh{102.\  } & يَـٰٓأَيُّهَا ٱلَّذِينَ ءَامَنُوا۟ ٱتَّقُوا۟ ٱللَّهَ حَقَّ تُقَاتِهِۦ وَلَا تَمُوتُنَّ إِلَّا وَأَنتُم مُّسْلِمُونَ ﴿١٠٢﴾\\
\textamh{103.\  } & وَٱعْتَصِمُوا۟ بِحَبْلِ ٱللَّهِ جَمِيعًۭا وَلَا تَفَرَّقُوا۟ ۚ وَٱذْكُرُوا۟ نِعْمَتَ ٱللَّهِ عَلَيْكُمْ إِذْ كُنتُمْ أَعْدَآءًۭ فَأَلَّفَ بَيْنَ قُلُوبِكُمْ فَأَصْبَحْتُم بِنِعْمَتِهِۦٓ إِخْوَٟنًۭا وَكُنتُمْ عَلَىٰ شَفَا حُفْرَةٍۢ مِّنَ ٱلنَّارِ فَأَنقَذَكُم مِّنْهَا ۗ كَذَٟلِكَ يُبَيِّنُ ٱللَّهُ لَكُمْ ءَايَـٰتِهِۦ لَعَلَّكُمْ تَهْتَدُونَ ﴿١٠٣﴾\\
\textamh{104.\  } & وَلْتَكُن مِّنكُمْ أُمَّةٌۭ يَدْعُونَ إِلَى ٱلْخَيْرِ وَيَأْمُرُونَ بِٱلْمَعْرُوفِ وَيَنْهَوْنَ عَنِ ٱلْمُنكَرِ ۚ وَأُو۟لَـٰٓئِكَ هُمُ ٱلْمُفْلِحُونَ ﴿١٠٤﴾\\
\textamh{105.\  } & وَلَا تَكُونُوا۟ كَٱلَّذِينَ تَفَرَّقُوا۟ وَٱخْتَلَفُوا۟ مِنۢ بَعْدِ مَا جَآءَهُمُ ٱلْبَيِّنَـٰتُ ۚ وَأُو۟لَـٰٓئِكَ لَهُمْ عَذَابٌ عَظِيمٌۭ ﴿١٠٥﴾\\
\textamh{106.\  } & يَوْمَ تَبْيَضُّ وُجُوهٌۭ وَتَسْوَدُّ وُجُوهٌۭ ۚ فَأَمَّا ٱلَّذِينَ ٱسْوَدَّتْ وُجُوهُهُمْ أَكَفَرْتُم بَعْدَ إِيمَـٰنِكُمْ فَذُوقُوا۟ ٱلْعَذَابَ بِمَا كُنتُمْ تَكْفُرُونَ ﴿١٠٦﴾\\
\textamh{107.\  } & وَأَمَّا ٱلَّذِينَ ٱبْيَضَّتْ وُجُوهُهُمْ فَفِى رَحْمَةِ ٱللَّهِ هُمْ فِيهَا خَـٰلِدُونَ ﴿١٠٧﴾\\
\textamh{108.\  } & تِلْكَ ءَايَـٰتُ ٱللَّهِ نَتْلُوهَا عَلَيْكَ بِٱلْحَقِّ ۗ وَمَا ٱللَّهُ يُرِيدُ ظُلْمًۭا لِّلْعَـٰلَمِينَ ﴿١٠٨﴾\\
\textamh{109.\  } & وَلِلَّهِ مَا فِى ٱلسَّمَـٰوَٟتِ وَمَا فِى ٱلْأَرْضِ ۚ وَإِلَى ٱللَّهِ تُرْجَعُ ٱلْأُمُورُ ﴿١٠٩﴾\\
\textamh{110.\  } & كُنتُمْ خَيْرَ أُمَّةٍ أُخْرِجَتْ لِلنَّاسِ تَأْمُرُونَ بِٱلْمَعْرُوفِ وَتَنْهَوْنَ عَنِ ٱلْمُنكَرِ وَتُؤْمِنُونَ بِٱللَّهِ ۗ وَلَوْ ءَامَنَ أَهْلُ ٱلْكِتَـٰبِ لَكَانَ خَيْرًۭا لَّهُم ۚ مِّنْهُمُ ٱلْمُؤْمِنُونَ وَأَكْثَرُهُمُ ٱلْفَـٰسِقُونَ ﴿١١٠﴾\\
\textamh{111.\  } & لَن يَضُرُّوكُمْ إِلَّآ أَذًۭى ۖ وَإِن يُقَـٰتِلُوكُمْ يُوَلُّوكُمُ ٱلْأَدْبَارَ ثُمَّ لَا يُنصَرُونَ ﴿١١١﴾\\
\textamh{112.\  } & ضُرِبَتْ عَلَيْهِمُ ٱلذِّلَّةُ أَيْنَ مَا ثُقِفُوٓا۟ إِلَّا بِحَبْلٍۢ مِّنَ ٱللَّهِ وَحَبْلٍۢ مِّنَ ٱلنَّاسِ وَبَآءُو بِغَضَبٍۢ مِّنَ ٱللَّهِ وَضُرِبَتْ عَلَيْهِمُ ٱلْمَسْكَنَةُ ۚ ذَٟلِكَ بِأَنَّهُمْ كَانُوا۟ يَكْفُرُونَ بِـَٔايَـٰتِ ٱللَّهِ وَيَقْتُلُونَ ٱلْأَنۢبِيَآءَ بِغَيْرِ حَقٍّۢ ۚ ذَٟلِكَ بِمَا عَصَوا۟ وَّكَانُوا۟ يَعْتَدُونَ ﴿١١٢﴾\\
\textamh{113.\  } & ۞ لَيْسُوا۟ سَوَآءًۭ ۗ مِّنْ أَهْلِ ٱلْكِتَـٰبِ أُمَّةٌۭ قَآئِمَةٌۭ يَتْلُونَ ءَايَـٰتِ ٱللَّهِ ءَانَآءَ ٱلَّيْلِ وَهُمْ يَسْجُدُونَ ﴿١١٣﴾\\
\textamh{114.\  } & يُؤْمِنُونَ بِٱللَّهِ وَٱلْيَوْمِ ٱلْءَاخِرِ وَيَأْمُرُونَ بِٱلْمَعْرُوفِ وَيَنْهَوْنَ عَنِ ٱلْمُنكَرِ وَيُسَـٰرِعُونَ فِى ٱلْخَيْرَٰتِ وَأُو۟لَـٰٓئِكَ مِنَ ٱلصَّـٰلِحِينَ ﴿١١٤﴾\\
\textamh{115.\  } & وَمَا يَفْعَلُوا۟ مِنْ خَيْرٍۢ فَلَن يُكْفَرُوهُ ۗ وَٱللَّهُ عَلِيمٌۢ بِٱلْمُتَّقِينَ ﴿١١٥﴾\\
\textamh{116.\  } & إِنَّ ٱلَّذِينَ كَفَرُوا۟ لَن تُغْنِىَ عَنْهُمْ أَمْوَٟلُهُمْ وَلَآ أَوْلَـٰدُهُم مِّنَ ٱللَّهِ شَيْـًۭٔا ۖ وَأُو۟لَـٰٓئِكَ أَصْحَـٰبُ ٱلنَّارِ ۚ هُمْ فِيهَا خَـٰلِدُونَ ﴿١١٦﴾\\
\textamh{117.\  } & مَثَلُ مَا يُنفِقُونَ فِى هَـٰذِهِ ٱلْحَيَوٰةِ ٱلدُّنْيَا كَمَثَلِ رِيحٍۢ فِيهَا صِرٌّ أَصَابَتْ حَرْثَ قَوْمٍۢ ظَلَمُوٓا۟ أَنفُسَهُمْ فَأَهْلَكَتْهُ ۚ وَمَا ظَلَمَهُمُ ٱللَّهُ وَلَـٰكِنْ أَنفُسَهُمْ يَظْلِمُونَ ﴿١١٧﴾\\
\textamh{118.\  } & يَـٰٓأَيُّهَا ٱلَّذِينَ ءَامَنُوا۟ لَا تَتَّخِذُوا۟ بِطَانَةًۭ مِّن دُونِكُمْ لَا يَأْلُونَكُمْ خَبَالًۭا وَدُّوا۟ مَا عَنِتُّمْ قَدْ بَدَتِ ٱلْبَغْضَآءُ مِنْ أَفْوَٟهِهِمْ وَمَا تُخْفِى صُدُورُهُمْ أَكْبَرُ ۚ قَدْ بَيَّنَّا لَكُمُ ٱلْءَايَـٰتِ ۖ إِن كُنتُمْ تَعْقِلُونَ ﴿١١٨﴾\\
\textamh{119.\  } & هَـٰٓأَنتُمْ أُو۟لَآءِ تُحِبُّونَهُمْ وَلَا يُحِبُّونَكُمْ وَتُؤْمِنُونَ بِٱلْكِتَـٰبِ كُلِّهِۦ وَإِذَا لَقُوكُمْ قَالُوٓا۟ ءَامَنَّا وَإِذَا خَلَوْا۟ عَضُّوا۟ عَلَيْكُمُ ٱلْأَنَامِلَ مِنَ ٱلْغَيْظِ ۚ قُلْ مُوتُوا۟ بِغَيْظِكُمْ ۗ إِنَّ ٱللَّهَ عَلِيمٌۢ بِذَاتِ ٱلصُّدُورِ ﴿١١٩﴾\\
\textamh{120.\  } & إِن تَمْسَسْكُمْ حَسَنَةٌۭ تَسُؤْهُمْ وَإِن تُصِبْكُمْ سَيِّئَةٌۭ يَفْرَحُوا۟ بِهَا ۖ وَإِن تَصْبِرُوا۟ وَتَتَّقُوا۟ لَا يَضُرُّكُمْ كَيْدُهُمْ شَيْـًٔا ۗ إِنَّ ٱللَّهَ بِمَا يَعْمَلُونَ مُحِيطٌۭ ﴿١٢٠﴾\\
\textamh{121.\  } & وَإِذْ غَدَوْتَ مِنْ أَهْلِكَ تُبَوِّئُ ٱلْمُؤْمِنِينَ مَقَـٰعِدَ لِلْقِتَالِ ۗ وَٱللَّهُ سَمِيعٌ عَلِيمٌ ﴿١٢١﴾\\
\textamh{122.\  } & إِذْ هَمَّت طَّآئِفَتَانِ مِنكُمْ أَن تَفْشَلَا وَٱللَّهُ وَلِيُّهُمَا ۗ وَعَلَى ٱللَّهِ فَلْيَتَوَكَّلِ ٱلْمُؤْمِنُونَ ﴿١٢٢﴾\\
\textamh{123.\  } & وَلَقَدْ نَصَرَكُمُ ٱللَّهُ بِبَدْرٍۢ وَأَنتُمْ أَذِلَّةٌۭ ۖ فَٱتَّقُوا۟ ٱللَّهَ لَعَلَّكُمْ تَشْكُرُونَ ﴿١٢٣﴾\\
\textamh{124.\  } & إِذْ تَقُولُ لِلْمُؤْمِنِينَ أَلَن يَكْفِيَكُمْ أَن يُمِدَّكُمْ رَبُّكُم بِثَلَـٰثَةِ ءَالَـٰفٍۢ مِّنَ ٱلْمَلَـٰٓئِكَةِ مُنزَلِينَ ﴿١٢٤﴾\\
\textamh{125.\  } & بَلَىٰٓ ۚ إِن تَصْبِرُوا۟ وَتَتَّقُوا۟ وَيَأْتُوكُم مِّن فَوْرِهِمْ هَـٰذَا يُمْدِدْكُمْ رَبُّكُم بِخَمْسَةِ ءَالَـٰفٍۢ مِّنَ ٱلْمَلَـٰٓئِكَةِ مُسَوِّمِينَ ﴿١٢٥﴾\\
\textamh{126.\  } & وَمَا جَعَلَهُ ٱللَّهُ إِلَّا بُشْرَىٰ لَكُمْ وَلِتَطْمَئِنَّ قُلُوبُكُم بِهِۦ ۗ وَمَا ٱلنَّصْرُ إِلَّا مِنْ عِندِ ٱللَّهِ ٱلْعَزِيزِ ٱلْحَكِيمِ ﴿١٢٦﴾\\
\textamh{127.\  } & لِيَقْطَعَ طَرَفًۭا مِّنَ ٱلَّذِينَ كَفَرُوٓا۟ أَوْ يَكْبِتَهُمْ فَيَنقَلِبُوا۟ خَآئِبِينَ ﴿١٢٧﴾\\
\textamh{128.\  } & لَيْسَ لَكَ مِنَ ٱلْأَمْرِ شَىْءٌ أَوْ يَتُوبَ عَلَيْهِمْ أَوْ يُعَذِّبَهُمْ فَإِنَّهُمْ ظَـٰلِمُونَ ﴿١٢٨﴾\\
\textamh{129.\  } & وَلِلَّهِ مَا فِى ٱلسَّمَـٰوَٟتِ وَمَا فِى ٱلْأَرْضِ ۚ يَغْفِرُ لِمَن يَشَآءُ وَيُعَذِّبُ مَن يَشَآءُ ۚ وَٱللَّهُ غَفُورٌۭ رَّحِيمٌۭ ﴿١٢٩﴾\\
\textamh{130.\  } & يَـٰٓأَيُّهَا ٱلَّذِينَ ءَامَنُوا۟ لَا تَأْكُلُوا۟ ٱلرِّبَوٰٓا۟ أَضْعَـٰفًۭا مُّضَٰعَفَةًۭ ۖ وَٱتَّقُوا۟ ٱللَّهَ لَعَلَّكُمْ تُفْلِحُونَ ﴿١٣٠﴾\\
\textamh{131.\  } & وَٱتَّقُوا۟ ٱلنَّارَ ٱلَّتِىٓ أُعِدَّتْ لِلْكَـٰفِرِينَ ﴿١٣١﴾\\
\textamh{132.\  } & وَأَطِيعُوا۟ ٱللَّهَ وَٱلرَّسُولَ لَعَلَّكُمْ تُرْحَمُونَ ﴿١٣٢﴾\\
\textamh{133.\  } & ۞ وَسَارِعُوٓا۟ إِلَىٰ مَغْفِرَةٍۢ مِّن رَّبِّكُمْ وَجَنَّةٍ عَرْضُهَا ٱلسَّمَـٰوَٟتُ وَٱلْأَرْضُ أُعِدَّتْ لِلْمُتَّقِينَ ﴿١٣٣﴾\\
\textamh{134.\  } & ٱلَّذِينَ يُنفِقُونَ فِى ٱلسَّرَّآءِ وَٱلضَّرَّآءِ وَٱلْكَـٰظِمِينَ ٱلْغَيْظَ وَٱلْعَافِينَ عَنِ ٱلنَّاسِ ۗ وَٱللَّهُ يُحِبُّ ٱلْمُحْسِنِينَ ﴿١٣٤﴾\\
\textamh{135.\  } & وَٱلَّذِينَ إِذَا فَعَلُوا۟ فَـٰحِشَةً أَوْ ظَلَمُوٓا۟ أَنفُسَهُمْ ذَكَرُوا۟ ٱللَّهَ فَٱسْتَغْفَرُوا۟ لِذُنُوبِهِمْ وَمَن يَغْفِرُ ٱلذُّنُوبَ إِلَّا ٱللَّهُ وَلَمْ يُصِرُّوا۟ عَلَىٰ مَا فَعَلُوا۟ وَهُمْ يَعْلَمُونَ ﴿١٣٥﴾\\
\textamh{136.\  } & أُو۟لَـٰٓئِكَ جَزَآؤُهُم مَّغْفِرَةٌۭ مِّن رَّبِّهِمْ وَجَنَّـٰتٌۭ تَجْرِى مِن تَحْتِهَا ٱلْأَنْهَـٰرُ خَـٰلِدِينَ فِيهَا ۚ وَنِعْمَ أَجْرُ ٱلْعَـٰمِلِينَ ﴿١٣٦﴾\\
\textamh{137.\  } & قَدْ خَلَتْ مِن قَبْلِكُمْ سُنَنٌۭ فَسِيرُوا۟ فِى ٱلْأَرْضِ فَٱنظُرُوا۟ كَيْفَ كَانَ عَـٰقِبَةُ ٱلْمُكَذِّبِينَ ﴿١٣٧﴾\\
\textamh{138.\  } & هَـٰذَا بَيَانٌۭ لِّلنَّاسِ وَهُدًۭى وَمَوْعِظَةٌۭ لِّلْمُتَّقِينَ ﴿١٣٨﴾\\
\textamh{139.\  } & وَلَا تَهِنُوا۟ وَلَا تَحْزَنُوا۟ وَأَنتُمُ ٱلْأَعْلَوْنَ إِن كُنتُم مُّؤْمِنِينَ ﴿١٣٩﴾\\
\textamh{140.\  } & إِن يَمْسَسْكُمْ قَرْحٌۭ فَقَدْ مَسَّ ٱلْقَوْمَ قَرْحٌۭ مِّثْلُهُۥ ۚ وَتِلْكَ ٱلْأَيَّامُ نُدَاوِلُهَا بَيْنَ ٱلنَّاسِ وَلِيَعْلَمَ ٱللَّهُ ٱلَّذِينَ ءَامَنُوا۟ وَيَتَّخِذَ مِنكُمْ شُهَدَآءَ ۗ وَٱللَّهُ لَا يُحِبُّ ٱلظَّـٰلِمِينَ ﴿١٤٠﴾\\
\textamh{141.\  } & وَلِيُمَحِّصَ ٱللَّهُ ٱلَّذِينَ ءَامَنُوا۟ وَيَمْحَقَ ٱلْكَـٰفِرِينَ ﴿١٤١﴾\\
\textamh{142.\  } & أَمْ حَسِبْتُمْ أَن تَدْخُلُوا۟ ٱلْجَنَّةَ وَلَمَّا يَعْلَمِ ٱللَّهُ ٱلَّذِينَ جَٰهَدُوا۟ مِنكُمْ وَيَعْلَمَ ٱلصَّـٰبِرِينَ ﴿١٤٢﴾\\
\textamh{143.\  } & وَلَقَدْ كُنتُمْ تَمَنَّوْنَ ٱلْمَوْتَ مِن قَبْلِ أَن تَلْقَوْهُ فَقَدْ رَأَيْتُمُوهُ وَأَنتُمْ تَنظُرُونَ ﴿١٤٣﴾\\
\textamh{144.\  } & وَمَا مُحَمَّدٌ إِلَّا رَسُولٌۭ قَدْ خَلَتْ مِن قَبْلِهِ ٱلرُّسُلُ ۚ أَفَإِي۟ن مَّاتَ أَوْ قُتِلَ ٱنقَلَبْتُمْ عَلَىٰٓ أَعْقَـٰبِكُمْ ۚ وَمَن يَنقَلِبْ عَلَىٰ عَقِبَيْهِ فَلَن يَضُرَّ ٱللَّهَ شَيْـًۭٔا ۗ وَسَيَجْزِى ٱللَّهُ ٱلشَّـٰكِرِينَ ﴿١٤٤﴾\\
\textamh{145.\  } & وَمَا كَانَ لِنَفْسٍ أَن تَمُوتَ إِلَّا بِإِذْنِ ٱللَّهِ كِتَـٰبًۭا مُّؤَجَّلًۭا ۗ وَمَن يُرِدْ ثَوَابَ ٱلدُّنْيَا نُؤْتِهِۦ مِنْهَا وَمَن يُرِدْ ثَوَابَ ٱلْءَاخِرَةِ نُؤْتِهِۦ مِنْهَا ۚ وَسَنَجْزِى ٱلشَّـٰكِرِينَ ﴿١٤٥﴾\\
\textamh{146.\  } & وَكَأَيِّن مِّن نَّبِىٍّۢ قَـٰتَلَ مَعَهُۥ رِبِّيُّونَ كَثِيرٌۭ فَمَا وَهَنُوا۟ لِمَآ أَصَابَهُمْ فِى سَبِيلِ ٱللَّهِ وَمَا ضَعُفُوا۟ وَمَا ٱسْتَكَانُوا۟ ۗ وَٱللَّهُ يُحِبُّ ٱلصَّـٰبِرِينَ ﴿١٤٦﴾\\
\textamh{147.\  } & وَمَا كَانَ قَوْلَهُمْ إِلَّآ أَن قَالُوا۟ رَبَّنَا ٱغْفِرْ لَنَا ذُنُوبَنَا وَإِسْرَافَنَا فِىٓ أَمْرِنَا وَثَبِّتْ أَقْدَامَنَا وَٱنصُرْنَا عَلَى ٱلْقَوْمِ ٱلْكَـٰفِرِينَ ﴿١٤٧﴾\\
\textamh{148.\  } & فَـَٔاتَىٰهُمُ ٱللَّهُ ثَوَابَ ٱلدُّنْيَا وَحُسْنَ ثَوَابِ ٱلْءَاخِرَةِ ۗ وَٱللَّهُ يُحِبُّ ٱلْمُحْسِنِينَ ﴿١٤٨﴾\\
\textamh{149.\  } & يَـٰٓأَيُّهَا ٱلَّذِينَ ءَامَنُوٓا۟ إِن تُطِيعُوا۟ ٱلَّذِينَ كَفَرُوا۟ يَرُدُّوكُمْ عَلَىٰٓ أَعْقَـٰبِكُمْ فَتَنقَلِبُوا۟ خَـٰسِرِينَ ﴿١٤٩﴾\\
\textamh{150.\  } & بَلِ ٱللَّهُ مَوْلَىٰكُمْ ۖ وَهُوَ خَيْرُ ٱلنَّـٰصِرِينَ ﴿١٥٠﴾\\
\textamh{151.\  } & سَنُلْقِى فِى قُلُوبِ ٱلَّذِينَ كَفَرُوا۟ ٱلرُّعْبَ بِمَآ أَشْرَكُوا۟ بِٱللَّهِ مَا لَمْ يُنَزِّلْ بِهِۦ سُلْطَٰنًۭا ۖ وَمَأْوَىٰهُمُ ٱلنَّارُ ۚ وَبِئْسَ مَثْوَى ٱلظَّـٰلِمِينَ ﴿١٥١﴾\\
\textamh{152.\  } & وَلَقَدْ صَدَقَكُمُ ٱللَّهُ وَعْدَهُۥٓ إِذْ تَحُسُّونَهُم بِإِذْنِهِۦ ۖ حَتَّىٰٓ إِذَا فَشِلْتُمْ وَتَنَـٰزَعْتُمْ فِى ٱلْأَمْرِ وَعَصَيْتُم مِّنۢ بَعْدِ مَآ أَرَىٰكُم مَّا تُحِبُّونَ ۚ مِنكُم مَّن يُرِيدُ ٱلدُّنْيَا وَمِنكُم مَّن يُرِيدُ ٱلْءَاخِرَةَ ۚ ثُمَّ صَرَفَكُمْ عَنْهُمْ لِيَبْتَلِيَكُمْ ۖ وَلَقَدْ عَفَا عَنكُمْ ۗ وَٱللَّهُ ذُو فَضْلٍ عَلَى ٱلْمُؤْمِنِينَ ﴿١٥٢﴾\\
\textamh{153.\  } & ۞ إِذْ تُصْعِدُونَ وَلَا تَلْوُۥنَ عَلَىٰٓ أَحَدٍۢ وَٱلرَّسُولُ يَدْعُوكُمْ فِىٓ أُخْرَىٰكُمْ فَأَثَـٰبَكُمْ غَمًّۢا بِغَمٍّۢ لِّكَيْلَا تَحْزَنُوا۟ عَلَىٰ مَا فَاتَكُمْ وَلَا مَآ أَصَـٰبَكُمْ ۗ وَٱللَّهُ خَبِيرٌۢ بِمَا تَعْمَلُونَ ﴿١٥٣﴾\\
\textamh{154.\  } & ثُمَّ أَنزَلَ عَلَيْكُم مِّنۢ بَعْدِ ٱلْغَمِّ أَمَنَةًۭ نُّعَاسًۭا يَغْشَىٰ طَآئِفَةًۭ مِّنكُمْ ۖ وَطَآئِفَةٌۭ قَدْ أَهَمَّتْهُمْ أَنفُسُهُمْ يَظُنُّونَ بِٱللَّهِ غَيْرَ ٱلْحَقِّ ظَنَّ ٱلْجَٰهِلِيَّةِ ۖ يَقُولُونَ هَل لَّنَا مِنَ ٱلْأَمْرِ مِن شَىْءٍۢ ۗ قُلْ إِنَّ ٱلْأَمْرَ كُلَّهُۥ لِلَّهِ ۗ يُخْفُونَ فِىٓ أَنفُسِهِم مَّا لَا يُبْدُونَ لَكَ ۖ يَقُولُونَ لَوْ كَانَ لَنَا مِنَ ٱلْأَمْرِ شَىْءٌۭ مَّا قُتِلْنَا هَـٰهُنَا ۗ قُل لَّوْ كُنتُمْ فِى بُيُوتِكُمْ لَبَرَزَ ٱلَّذِينَ كُتِبَ عَلَيْهِمُ ٱلْقَتْلُ إِلَىٰ مَضَاجِعِهِمْ ۖ وَلِيَبْتَلِىَ ٱللَّهُ مَا فِى صُدُورِكُمْ وَلِيُمَحِّصَ مَا فِى قُلُوبِكُمْ ۗ وَٱللَّهُ عَلِيمٌۢ بِذَاتِ ٱلصُّدُورِ ﴿١٥٤﴾\\
\textamh{155.\  } & إِنَّ ٱلَّذِينَ تَوَلَّوْا۟ مِنكُمْ يَوْمَ ٱلْتَقَى ٱلْجَمْعَانِ إِنَّمَا ٱسْتَزَلَّهُمُ ٱلشَّيْطَٰنُ بِبَعْضِ مَا كَسَبُوا۟ ۖ وَلَقَدْ عَفَا ٱللَّهُ عَنْهُمْ ۗ إِنَّ ٱللَّهَ غَفُورٌ حَلِيمٌۭ ﴿١٥٥﴾\\
\textamh{156.\  } & يَـٰٓأَيُّهَا ٱلَّذِينَ ءَامَنُوا۟ لَا تَكُونُوا۟ كَٱلَّذِينَ كَفَرُوا۟ وَقَالُوا۟ لِإِخْوَٟنِهِمْ إِذَا ضَرَبُوا۟ فِى ٱلْأَرْضِ أَوْ كَانُوا۟ غُزًّۭى لَّوْ كَانُوا۟ عِندَنَا مَا مَاتُوا۟ وَمَا قُتِلُوا۟ لِيَجْعَلَ ٱللَّهُ ذَٟلِكَ حَسْرَةًۭ فِى قُلُوبِهِمْ ۗ وَٱللَّهُ يُحْىِۦ وَيُمِيتُ ۗ وَٱللَّهُ بِمَا تَعْمَلُونَ بَصِيرٌۭ ﴿١٥٦﴾\\
\textamh{157.\  } & وَلَئِن قُتِلْتُمْ فِى سَبِيلِ ٱللَّهِ أَوْ مُتُّمْ لَمَغْفِرَةٌۭ مِّنَ ٱللَّهِ وَرَحْمَةٌ خَيْرٌۭ مِّمَّا يَجْمَعُونَ ﴿١٥٧﴾\\
\textamh{158.\  } & وَلَئِن مُّتُّمْ أَوْ قُتِلْتُمْ لَإِلَى ٱللَّهِ تُحْشَرُونَ ﴿١٥٨﴾\\
\textamh{159.\  } & فَبِمَا رَحْمَةٍۢ مِّنَ ٱللَّهِ لِنتَ لَهُمْ ۖ وَلَوْ كُنتَ فَظًّا غَلِيظَ ٱلْقَلْبِ لَٱنفَضُّوا۟ مِنْ حَوْلِكَ ۖ فَٱعْفُ عَنْهُمْ وَٱسْتَغْفِرْ لَهُمْ وَشَاوِرْهُمْ فِى ٱلْأَمْرِ ۖ فَإِذَا عَزَمْتَ فَتَوَكَّلْ عَلَى ٱللَّهِ ۚ إِنَّ ٱللَّهَ يُحِبُّ ٱلْمُتَوَكِّلِينَ ﴿١٥٩﴾\\
\textamh{160.\  } & إِن يَنصُرْكُمُ ٱللَّهُ فَلَا غَالِبَ لَكُمْ ۖ وَإِن يَخْذُلْكُمْ فَمَن ذَا ٱلَّذِى يَنصُرُكُم مِّنۢ بَعْدِهِۦ ۗ وَعَلَى ٱللَّهِ فَلْيَتَوَكَّلِ ٱلْمُؤْمِنُونَ ﴿١٦٠﴾\\
\textamh{161.\  } & وَمَا كَانَ لِنَبِىٍّ أَن يَغُلَّ ۚ وَمَن يَغْلُلْ يَأْتِ بِمَا غَلَّ يَوْمَ ٱلْقِيَـٰمَةِ ۚ ثُمَّ تُوَفَّىٰ كُلُّ نَفْسٍۢ مَّا كَسَبَتْ وَهُمْ لَا يُظْلَمُونَ ﴿١٦١﴾\\
\textamh{162.\  } & أَفَمَنِ ٱتَّبَعَ رِضْوَٟنَ ٱللَّهِ كَمَنۢ بَآءَ بِسَخَطٍۢ مِّنَ ٱللَّهِ وَمَأْوَىٰهُ جَهَنَّمُ ۚ وَبِئْسَ ٱلْمَصِيرُ ﴿١٦٢﴾\\
\textamh{163.\  } & هُمْ دَرَجَٰتٌ عِندَ ٱللَّهِ ۗ وَٱللَّهُ بَصِيرٌۢ بِمَا يَعْمَلُونَ ﴿١٦٣﴾\\
\textamh{164.\  } & لَقَدْ مَنَّ ٱللَّهُ عَلَى ٱلْمُؤْمِنِينَ إِذْ بَعَثَ فِيهِمْ رَسُولًۭا مِّنْ أَنفُسِهِمْ يَتْلُوا۟ عَلَيْهِمْ ءَايَـٰتِهِۦ وَيُزَكِّيهِمْ وَيُعَلِّمُهُمُ ٱلْكِتَـٰبَ وَٱلْحِكْمَةَ وَإِن كَانُوا۟ مِن قَبْلُ لَفِى ضَلَـٰلٍۢ مُّبِينٍ ﴿١٦٤﴾\\
\textamh{165.\  } & أَوَلَمَّآ أَصَـٰبَتْكُم مُّصِيبَةٌۭ قَدْ أَصَبْتُم مِّثْلَيْهَا قُلْتُمْ أَنَّىٰ هَـٰذَا ۖ قُلْ هُوَ مِنْ عِندِ أَنفُسِكُمْ ۗ إِنَّ ٱللَّهَ عَلَىٰ كُلِّ شَىْءٍۢ قَدِيرٌۭ ﴿١٦٥﴾\\
\textamh{166.\  } & وَمَآ أَصَـٰبَكُمْ يَوْمَ ٱلْتَقَى ٱلْجَمْعَانِ فَبِإِذْنِ ٱللَّهِ وَلِيَعْلَمَ ٱلْمُؤْمِنِينَ ﴿١٦٦﴾\\
\textamh{167.\  } & وَلِيَعْلَمَ ٱلَّذِينَ نَافَقُوا۟ ۚ وَقِيلَ لَهُمْ تَعَالَوْا۟ قَـٰتِلُوا۟ فِى سَبِيلِ ٱللَّهِ أَوِ ٱدْفَعُوا۟ ۖ قَالُوا۟ لَوْ نَعْلَمُ قِتَالًۭا لَّٱتَّبَعْنَـٰكُمْ ۗ هُمْ لِلْكُفْرِ يَوْمَئِذٍ أَقْرَبُ مِنْهُمْ لِلْإِيمَـٰنِ ۚ يَقُولُونَ بِأَفْوَٟهِهِم مَّا لَيْسَ فِى قُلُوبِهِمْ ۗ وَٱللَّهُ أَعْلَمُ بِمَا يَكْتُمُونَ ﴿١٦٧﴾\\
\textamh{168.\  } & ٱلَّذِينَ قَالُوا۟ لِإِخْوَٟنِهِمْ وَقَعَدُوا۟ لَوْ أَطَاعُونَا مَا قُتِلُوا۟ ۗ قُلْ فَٱدْرَءُوا۟ عَنْ أَنفُسِكُمُ ٱلْمَوْتَ إِن كُنتُمْ صَـٰدِقِينَ ﴿١٦٨﴾\\
\textamh{169.\  } & وَلَا تَحْسَبَنَّ ٱلَّذِينَ قُتِلُوا۟ فِى سَبِيلِ ٱللَّهِ أَمْوَٟتًۢا ۚ بَلْ أَحْيَآءٌ عِندَ رَبِّهِمْ يُرْزَقُونَ ﴿١٦٩﴾\\
\textamh{170.\  } & فَرِحِينَ بِمَآ ءَاتَىٰهُمُ ٱللَّهُ مِن فَضْلِهِۦ وَيَسْتَبْشِرُونَ بِٱلَّذِينَ لَمْ يَلْحَقُوا۟ بِهِم مِّنْ خَلْفِهِمْ أَلَّا خَوْفٌ عَلَيْهِمْ وَلَا هُمْ يَحْزَنُونَ ﴿١٧٠﴾\\
\textamh{171.\  } & ۞ يَسْتَبْشِرُونَ بِنِعْمَةٍۢ مِّنَ ٱللَّهِ وَفَضْلٍۢ وَأَنَّ ٱللَّهَ لَا يُضِيعُ أَجْرَ ٱلْمُؤْمِنِينَ ﴿١٧١﴾\\
\textamh{172.\  } & ٱلَّذِينَ ٱسْتَجَابُوا۟ لِلَّهِ وَٱلرَّسُولِ مِنۢ بَعْدِ مَآ أَصَابَهُمُ ٱلْقَرْحُ ۚ لِلَّذِينَ أَحْسَنُوا۟ مِنْهُمْ وَٱتَّقَوْا۟ أَجْرٌ عَظِيمٌ ﴿١٧٢﴾\\
\textamh{173.\  } & ٱلَّذِينَ قَالَ لَهُمُ ٱلنَّاسُ إِنَّ ٱلنَّاسَ قَدْ جَمَعُوا۟ لَكُمْ فَٱخْشَوْهُمْ فَزَادَهُمْ إِيمَـٰنًۭا وَقَالُوا۟ حَسْبُنَا ٱللَّهُ وَنِعْمَ ٱلْوَكِيلُ ﴿١٧٣﴾\\
\textamh{174.\  } & فَٱنقَلَبُوا۟ بِنِعْمَةٍۢ مِّنَ ٱللَّهِ وَفَضْلٍۢ لَّمْ يَمْسَسْهُمْ سُوٓءٌۭ وَٱتَّبَعُوا۟ رِضْوَٟنَ ٱللَّهِ ۗ وَٱللَّهُ ذُو فَضْلٍ عَظِيمٍ ﴿١٧٤﴾\\
\textamh{175.\  } & إِنَّمَا ذَٟلِكُمُ ٱلشَّيْطَٰنُ يُخَوِّفُ أَوْلِيَآءَهُۥ فَلَا تَخَافُوهُمْ وَخَافُونِ إِن كُنتُم مُّؤْمِنِينَ ﴿١٧٥﴾\\
\textamh{176.\  } & وَلَا يَحْزُنكَ ٱلَّذِينَ يُسَـٰرِعُونَ فِى ٱلْكُفْرِ ۚ إِنَّهُمْ لَن يَضُرُّوا۟ ٱللَّهَ شَيْـًۭٔا ۗ يُرِيدُ ٱللَّهُ أَلَّا يَجْعَلَ لَهُمْ حَظًّۭا فِى ٱلْءَاخِرَةِ ۖ وَلَهُمْ عَذَابٌ عَظِيمٌ ﴿١٧٦﴾\\
\textamh{177.\  } & إِنَّ ٱلَّذِينَ ٱشْتَرَوُا۟ ٱلْكُفْرَ بِٱلْإِيمَـٰنِ لَن يَضُرُّوا۟ ٱللَّهَ شَيْـًۭٔا وَلَهُمْ عَذَابٌ أَلِيمٌۭ ﴿١٧٧﴾\\
\textamh{178.\  } & وَلَا يَحْسَبَنَّ ٱلَّذِينَ كَفَرُوٓا۟ أَنَّمَا نُمْلِى لَهُمْ خَيْرٌۭ لِّأَنفُسِهِمْ ۚ إِنَّمَا نُمْلِى لَهُمْ لِيَزْدَادُوٓا۟ إِثْمًۭا ۚ وَلَهُمْ عَذَابٌۭ مُّهِينٌۭ ﴿١٧٨﴾\\
\textamh{179.\  } & مَّا كَانَ ٱللَّهُ لِيَذَرَ ٱلْمُؤْمِنِينَ عَلَىٰ مَآ أَنتُمْ عَلَيْهِ حَتَّىٰ يَمِيزَ ٱلْخَبِيثَ مِنَ ٱلطَّيِّبِ ۗ وَمَا كَانَ ٱللَّهُ لِيُطْلِعَكُمْ عَلَى ٱلْغَيْبِ وَلَـٰكِنَّ ٱللَّهَ يَجْتَبِى مِن رُّسُلِهِۦ مَن يَشَآءُ ۖ فَـَٔامِنُوا۟ بِٱللَّهِ وَرُسُلِهِۦ ۚ وَإِن تُؤْمِنُوا۟ وَتَتَّقُوا۟ فَلَكُمْ أَجْرٌ عَظِيمٌۭ ﴿١٧٩﴾\\
\textamh{180.\  } & وَلَا يَحْسَبَنَّ ٱلَّذِينَ يَبْخَلُونَ بِمَآ ءَاتَىٰهُمُ ٱللَّهُ مِن فَضْلِهِۦ هُوَ خَيْرًۭا لَّهُم ۖ بَلْ هُوَ شَرٌّۭ لَّهُمْ ۖ سَيُطَوَّقُونَ مَا بَخِلُوا۟ بِهِۦ يَوْمَ ٱلْقِيَـٰمَةِ ۗ وَلِلَّهِ مِيرَٰثُ ٱلسَّمَـٰوَٟتِ وَٱلْأَرْضِ ۗ وَٱللَّهُ بِمَا تَعْمَلُونَ خَبِيرٌۭ ﴿١٨٠﴾\\
\textamh{181.\  } & لَّقَدْ سَمِعَ ٱللَّهُ قَوْلَ ٱلَّذِينَ قَالُوٓا۟ إِنَّ ٱللَّهَ فَقِيرٌۭ وَنَحْنُ أَغْنِيَآءُ ۘ سَنَكْتُبُ مَا قَالُوا۟ وَقَتْلَهُمُ ٱلْأَنۢبِيَآءَ بِغَيْرِ حَقٍّۢ وَنَقُولُ ذُوقُوا۟ عَذَابَ ٱلْحَرِيقِ ﴿١٨١﴾\\
\textamh{182.\  } & ذَٟلِكَ بِمَا قَدَّمَتْ أَيْدِيكُمْ وَأَنَّ ٱللَّهَ لَيْسَ بِظَلَّامٍۢ لِّلْعَبِيدِ ﴿١٨٢﴾\\
\textamh{183.\  } & ٱلَّذِينَ قَالُوٓا۟ إِنَّ ٱللَّهَ عَهِدَ إِلَيْنَآ أَلَّا نُؤْمِنَ لِرَسُولٍ حَتَّىٰ يَأْتِيَنَا بِقُرْبَانٍۢ تَأْكُلُهُ ٱلنَّارُ ۗ قُلْ قَدْ جَآءَكُمْ رُسُلٌۭ مِّن قَبْلِى بِٱلْبَيِّنَـٰتِ وَبِٱلَّذِى قُلْتُمْ فَلِمَ قَتَلْتُمُوهُمْ إِن كُنتُمْ صَـٰدِقِينَ ﴿١٨٣﴾\\
\textamh{184.\  } & فَإِن كَذَّبُوكَ فَقَدْ كُذِّبَ رُسُلٌۭ مِّن قَبْلِكَ جَآءُو بِٱلْبَيِّنَـٰتِ وَٱلزُّبُرِ وَٱلْكِتَـٰبِ ٱلْمُنِيرِ ﴿١٨٤﴾\\
\textamh{185.\  } & كُلُّ نَفْسٍۢ ذَآئِقَةُ ٱلْمَوْتِ ۗ وَإِنَّمَا تُوَفَّوْنَ أُجُورَكُمْ يَوْمَ ٱلْقِيَـٰمَةِ ۖ فَمَن زُحْزِحَ عَنِ ٱلنَّارِ وَأُدْخِلَ ٱلْجَنَّةَ فَقَدْ فَازَ ۗ وَمَا ٱلْحَيَوٰةُ ٱلدُّنْيَآ إِلَّا مَتَـٰعُ ٱلْغُرُورِ ﴿١٨٥﴾\\
\textamh{186.\  } & ۞ لَتُبْلَوُنَّ فِىٓ أَمْوَٟلِكُمْ وَأَنفُسِكُمْ وَلَتَسْمَعُنَّ مِنَ ٱلَّذِينَ أُوتُوا۟ ٱلْكِتَـٰبَ مِن قَبْلِكُمْ وَمِنَ ٱلَّذِينَ أَشْرَكُوٓا۟ أَذًۭى كَثِيرًۭا ۚ وَإِن تَصْبِرُوا۟ وَتَتَّقُوا۟ فَإِنَّ ذَٟلِكَ مِنْ عَزْمِ ٱلْأُمُورِ ﴿١٨٦﴾\\
\textamh{187.\  } & وَإِذْ أَخَذَ ٱللَّهُ مِيثَـٰقَ ٱلَّذِينَ أُوتُوا۟ ٱلْكِتَـٰبَ لَتُبَيِّنُنَّهُۥ لِلنَّاسِ وَلَا تَكْتُمُونَهُۥ فَنَبَذُوهُ وَرَآءَ ظُهُورِهِمْ وَٱشْتَرَوْا۟ بِهِۦ ثَمَنًۭا قَلِيلًۭا ۖ فَبِئْسَ مَا يَشْتَرُونَ ﴿١٨٧﴾\\
\textamh{188.\  } & لَا تَحْسَبَنَّ ٱلَّذِينَ يَفْرَحُونَ بِمَآ أَتَوا۟ وَّيُحِبُّونَ أَن يُحْمَدُوا۟ بِمَا لَمْ يَفْعَلُوا۟ فَلَا تَحْسَبَنَّهُم بِمَفَازَةٍۢ مِّنَ ٱلْعَذَابِ ۖ وَلَهُمْ عَذَابٌ أَلِيمٌۭ ﴿١٨٨﴾\\
\textamh{189.\  } & وَلِلَّهِ مُلْكُ ٱلسَّمَـٰوَٟتِ وَٱلْأَرْضِ ۗ وَٱللَّهُ عَلَىٰ كُلِّ شَىْءٍۢ قَدِيرٌ ﴿١٨٩﴾\\
\textamh{190.\  } & إِنَّ فِى خَلْقِ ٱلسَّمَـٰوَٟتِ وَٱلْأَرْضِ وَٱخْتِلَـٰفِ ٱلَّيْلِ وَٱلنَّهَارِ لَءَايَـٰتٍۢ لِّأُو۟لِى ٱلْأَلْبَٰبِ ﴿١٩٠﴾\\
\textamh{191.\  } & ٱلَّذِينَ يَذْكُرُونَ ٱللَّهَ قِيَـٰمًۭا وَقُعُودًۭا وَعَلَىٰ جُنُوبِهِمْ وَيَتَفَكَّرُونَ فِى خَلْقِ ٱلسَّمَـٰوَٟتِ وَٱلْأَرْضِ رَبَّنَا مَا خَلَقْتَ هَـٰذَا بَٰطِلًۭا سُبْحَـٰنَكَ فَقِنَا عَذَابَ ٱلنَّارِ ﴿١٩١﴾\\
\textamh{192.\  } & رَبَّنَآ إِنَّكَ مَن تُدْخِلِ ٱلنَّارَ فَقَدْ أَخْزَيْتَهُۥ ۖ وَمَا لِلظَّـٰلِمِينَ مِنْ أَنصَارٍۢ ﴿١٩٢﴾\\
\textamh{193.\  } & رَّبَّنَآ إِنَّنَا سَمِعْنَا مُنَادِيًۭا يُنَادِى لِلْإِيمَـٰنِ أَنْ ءَامِنُوا۟ بِرَبِّكُمْ فَـَٔامَنَّا ۚ رَبَّنَا فَٱغْفِرْ لَنَا ذُنُوبَنَا وَكَفِّرْ عَنَّا سَيِّـَٔاتِنَا وَتَوَفَّنَا مَعَ ٱلْأَبْرَارِ ﴿١٩٣﴾\\
\textamh{194.\  } & رَبَّنَا وَءَاتِنَا مَا وَعَدتَّنَا عَلَىٰ رُسُلِكَ وَلَا تُخْزِنَا يَوْمَ ٱلْقِيَـٰمَةِ ۗ إِنَّكَ لَا تُخْلِفُ ٱلْمِيعَادَ ﴿١٩٤﴾\\
\textamh{195.\  } & فَٱسْتَجَابَ لَهُمْ رَبُّهُمْ أَنِّى لَآ أُضِيعُ عَمَلَ عَـٰمِلٍۢ مِّنكُم مِّن ذَكَرٍ أَوْ أُنثَىٰ ۖ بَعْضُكُم مِّنۢ بَعْضٍۢ ۖ فَٱلَّذِينَ هَاجَرُوا۟ وَأُخْرِجُوا۟ مِن دِيَـٰرِهِمْ وَأُوذُوا۟ فِى سَبِيلِى وَقَـٰتَلُوا۟ وَقُتِلُوا۟ لَأُكَفِّرَنَّ عَنْهُمْ سَيِّـَٔاتِهِمْ وَلَأُدْخِلَنَّهُمْ جَنَّـٰتٍۢ تَجْرِى مِن تَحْتِهَا ٱلْأَنْهَـٰرُ ثَوَابًۭا مِّنْ عِندِ ٱللَّهِ ۗ وَٱللَّهُ عِندَهُۥ حُسْنُ ٱلثَّوَابِ ﴿١٩٥﴾\\
\textamh{196.\  } & لَا يَغُرَّنَّكَ تَقَلُّبُ ٱلَّذِينَ كَفَرُوا۟ فِى ٱلْبِلَـٰدِ ﴿١٩٦﴾\\
\textamh{197.\  } & مَتَـٰعٌۭ قَلِيلٌۭ ثُمَّ مَأْوَىٰهُمْ جَهَنَّمُ ۚ وَبِئْسَ ٱلْمِهَادُ ﴿١٩٧﴾\\
\textamh{198.\  } & لَـٰكِنِ ٱلَّذِينَ ٱتَّقَوْا۟ رَبَّهُمْ لَهُمْ جَنَّـٰتٌۭ تَجْرِى مِن تَحْتِهَا ٱلْأَنْهَـٰرُ خَـٰلِدِينَ فِيهَا نُزُلًۭا مِّنْ عِندِ ٱللَّهِ ۗ وَمَا عِندَ ٱللَّهِ خَيْرٌۭ لِّلْأَبْرَارِ ﴿١٩٨﴾\\
\textamh{199.\  } & وَإِنَّ مِنْ أَهْلِ ٱلْكِتَـٰبِ لَمَن يُؤْمِنُ بِٱللَّهِ وَمَآ أُنزِلَ إِلَيْكُمْ وَمَآ أُنزِلَ إِلَيْهِمْ خَـٰشِعِينَ لِلَّهِ لَا يَشْتَرُونَ بِـَٔايَـٰتِ ٱللَّهِ ثَمَنًۭا قَلِيلًا ۗ أُو۟لَـٰٓئِكَ لَهُمْ أَجْرُهُمْ عِندَ رَبِّهِمْ ۗ إِنَّ ٱللَّهَ سَرِيعُ ٱلْحِسَابِ ﴿١٩٩﴾\\
\textamh{200.\  } & يَـٰٓأَيُّهَا ٱلَّذِينَ ءَامَنُوا۟ ٱصْبِرُوا۟ وَصَابِرُوا۟ وَرَابِطُوا۟ وَٱتَّقُوا۟ ٱللَّهَ لَعَلَّكُمْ تُفْلِحُونَ ﴿٢٠٠﴾\\
\end{longtable}
\clearpage

%% License: BSD style (Berkley) (i.e. Put the Copyright owner's name always)
%% Writer and Copyright (to): Bewketu(Bilal) Tadilo (2016-17)
\centering\section{\LR{\textamharic{ሱራቱ አንኒሳ -}  \RL{سوره  النساء}}}
\begin{longtable}{%
  @{}
    p{.5\textwidth}
  @{~~~~~~~~~~~~~}
    p{.5\textwidth}
    @{}
}
\nopagebreak
\textamh{ቢስሚላሂ አራህመኒ ራሂይም } &  بِسْمِ ٱللَّهِ ٱلرَّحْمَـٰنِ ٱلرَّحِيمِ\\
\textamh{1.\  } &  يَـٰٓأَيُّهَا ٱلنَّاسُ ٱتَّقُوا۟ رَبَّكُمُ ٱلَّذِى خَلَقَكُم مِّن نَّفْسٍۢ وَٟحِدَةٍۢ وَخَلَقَ مِنْهَا زَوْجَهَا وَبَثَّ مِنْهُمَا رِجَالًۭا كَثِيرًۭا وَنِسَآءًۭ ۚ وَٱتَّقُوا۟ ٱللَّهَ ٱلَّذِى تَسَآءَلُونَ بِهِۦ وَٱلْأَرْحَامَ ۚ إِنَّ ٱللَّهَ كَانَ عَلَيْكُمْ رَقِيبًۭا ﴿١﴾\\
\textamh{2.\  } & وَءَاتُوا۟ ٱلْيَتَـٰمَىٰٓ أَمْوَٟلَهُمْ ۖ وَلَا تَتَبَدَّلُوا۟ ٱلْخَبِيثَ بِٱلطَّيِّبِ ۖ وَلَا تَأْكُلُوٓا۟ أَمْوَٟلَهُمْ إِلَىٰٓ أَمْوَٟلِكُمْ ۚ إِنَّهُۥ كَانَ حُوبًۭا كَبِيرًۭا ﴿٢﴾\\
\textamh{3.\  } & وَإِنْ خِفْتُمْ أَلَّا تُقْسِطُوا۟ فِى ٱلْيَتَـٰمَىٰ فَٱنكِحُوا۟ مَا طَابَ لَكُم مِّنَ ٱلنِّسَآءِ مَثْنَىٰ وَثُلَـٰثَ وَرُبَٰعَ ۖ فَإِنْ خِفْتُمْ أَلَّا تَعْدِلُوا۟ فَوَٟحِدَةً أَوْ مَا مَلَكَتْ أَيْمَـٰنُكُمْ ۚ ذَٟلِكَ أَدْنَىٰٓ أَلَّا تَعُولُوا۟ ﴿٣﴾\\
\textamh{4.\  } & وَءَاتُوا۟ ٱلنِّسَآءَ صَدُقَـٰتِهِنَّ نِحْلَةًۭ ۚ فَإِن طِبْنَ لَكُمْ عَن شَىْءٍۢ مِّنْهُ نَفْسًۭا فَكُلُوهُ هَنِيٓـًۭٔا مَّرِيٓـًۭٔا ﴿٤﴾\\
\textamh{5.\  } & وَلَا تُؤْتُوا۟ ٱلسُّفَهَآءَ أَمْوَٟلَكُمُ ٱلَّتِى جَعَلَ ٱللَّهُ لَكُمْ قِيَـٰمًۭا وَٱرْزُقُوهُمْ فِيهَا وَٱكْسُوهُمْ وَقُولُوا۟ لَهُمْ قَوْلًۭا مَّعْرُوفًۭا ﴿٥﴾\\
\textamh{6.\  } & وَٱبْتَلُوا۟ ٱلْيَتَـٰمَىٰ حَتَّىٰٓ إِذَا بَلَغُوا۟ ٱلنِّكَاحَ فَإِنْ ءَانَسْتُم مِّنْهُمْ رُشْدًۭا فَٱدْفَعُوٓا۟ إِلَيْهِمْ أَمْوَٟلَهُمْ ۖ وَلَا تَأْكُلُوهَآ إِسْرَافًۭا وَبِدَارًا أَن يَكْبَرُوا۟ ۚ وَمَن كَانَ غَنِيًّۭا فَلْيَسْتَعْفِفْ ۖ وَمَن كَانَ فَقِيرًۭا فَلْيَأْكُلْ بِٱلْمَعْرُوفِ ۚ فَإِذَا دَفَعْتُمْ إِلَيْهِمْ أَمْوَٟلَهُمْ فَأَشْهِدُوا۟ عَلَيْهِمْ ۚ وَكَفَىٰ بِٱللَّهِ حَسِيبًۭا ﴿٦﴾\\
\textamh{7.\  } & لِّلرِّجَالِ نَصِيبٌۭ مِّمَّا تَرَكَ ٱلْوَٟلِدَانِ وَٱلْأَقْرَبُونَ وَلِلنِّسَآءِ نَصِيبٌۭ مِّمَّا تَرَكَ ٱلْوَٟلِدَانِ وَٱلْأَقْرَبُونَ مِمَّا قَلَّ مِنْهُ أَوْ كَثُرَ ۚ نَصِيبًۭا مَّفْرُوضًۭا ﴿٧﴾\\
\textamh{8.\  } & وَإِذَا حَضَرَ ٱلْقِسْمَةَ أُو۟لُوا۟ ٱلْقُرْبَىٰ وَٱلْيَتَـٰمَىٰ وَٱلْمَسَـٰكِينُ فَٱرْزُقُوهُم مِّنْهُ وَقُولُوا۟ لَهُمْ قَوْلًۭا مَّعْرُوفًۭا ﴿٨﴾\\
\textamh{9.\  } & وَلْيَخْشَ ٱلَّذِينَ لَوْ تَرَكُوا۟ مِنْ خَلْفِهِمْ ذُرِّيَّةًۭ ضِعَـٰفًا خَافُوا۟ عَلَيْهِمْ فَلْيَتَّقُوا۟ ٱللَّهَ وَلْيَقُولُوا۟ قَوْلًۭا سَدِيدًا ﴿٩﴾\\
\textamh{10.\  } & إِنَّ ٱلَّذِينَ يَأْكُلُونَ أَمْوَٟلَ ٱلْيَتَـٰمَىٰ ظُلْمًا إِنَّمَا يَأْكُلُونَ فِى بُطُونِهِمْ نَارًۭا ۖ وَسَيَصْلَوْنَ سَعِيرًۭا ﴿١٠﴾\\
\textamh{11.\  } & يُوصِيكُمُ ٱللَّهُ فِىٓ أَوْلَـٰدِكُمْ ۖ لِلذَّكَرِ مِثْلُ حَظِّ ٱلْأُنثَيَيْنِ ۚ فَإِن كُنَّ نِسَآءًۭ فَوْقَ ٱثْنَتَيْنِ فَلَهُنَّ ثُلُثَا مَا تَرَكَ ۖ وَإِن كَانَتْ وَٟحِدَةًۭ فَلَهَا ٱلنِّصْفُ ۚ وَلِأَبَوَيْهِ لِكُلِّ وَٟحِدٍۢ مِّنْهُمَا ٱلسُّدُسُ مِمَّا تَرَكَ إِن كَانَ لَهُۥ وَلَدٌۭ ۚ فَإِن لَّمْ يَكُن لَّهُۥ وَلَدٌۭ وَوَرِثَهُۥٓ أَبَوَاهُ فَلِأُمِّهِ ٱلثُّلُثُ ۚ فَإِن كَانَ لَهُۥٓ إِخْوَةٌۭ فَلِأُمِّهِ ٱلسُّدُسُ ۚ مِنۢ بَعْدِ وَصِيَّةٍۢ يُوصِى بِهَآ أَوْ دَيْنٍ ۗ ءَابَآؤُكُمْ وَأَبْنَآؤُكُمْ لَا تَدْرُونَ أَيُّهُمْ أَقْرَبُ لَكُمْ نَفْعًۭا ۚ فَرِيضَةًۭ مِّنَ ٱللَّهِ ۗ إِنَّ ٱللَّهَ كَانَ عَلِيمًا حَكِيمًۭا ﴿١١﴾\\
\textamh{12.\  } & ۞ وَلَكُمْ نِصْفُ مَا تَرَكَ أَزْوَٟجُكُمْ إِن لَّمْ يَكُن لَّهُنَّ وَلَدٌۭ ۚ فَإِن كَانَ لَهُنَّ وَلَدٌۭ فَلَكُمُ ٱلرُّبُعُ مِمَّا تَرَكْنَ ۚ مِنۢ بَعْدِ وَصِيَّةٍۢ يُوصِينَ بِهَآ أَوْ دَيْنٍۢ ۚ وَلَهُنَّ ٱلرُّبُعُ مِمَّا تَرَكْتُمْ إِن لَّمْ يَكُن لَّكُمْ وَلَدٌۭ ۚ فَإِن كَانَ لَكُمْ وَلَدٌۭ فَلَهُنَّ ٱلثُّمُنُ مِمَّا تَرَكْتُم ۚ مِّنۢ بَعْدِ وَصِيَّةٍۢ تُوصُونَ بِهَآ أَوْ دَيْنٍۢ ۗ وَإِن كَانَ رَجُلٌۭ يُورَثُ كَلَـٰلَةً أَوِ ٱمْرَأَةٌۭ وَلَهُۥٓ أَخٌ أَوْ أُخْتٌۭ فَلِكُلِّ وَٟحِدٍۢ مِّنْهُمَا ٱلسُّدُسُ ۚ فَإِن كَانُوٓا۟ أَكْثَرَ مِن ذَٟلِكَ فَهُمْ شُرَكَآءُ فِى ٱلثُّلُثِ ۚ مِنۢ بَعْدِ وَصِيَّةٍۢ يُوصَىٰ بِهَآ أَوْ دَيْنٍ غَيْرَ مُضَآرٍّۢ ۚ وَصِيَّةًۭ مِّنَ ٱللَّهِ ۗ وَٱللَّهُ عَلِيمٌ حَلِيمٌۭ ﴿١٢﴾\\
\textamh{13.\  } & تِلْكَ حُدُودُ ٱللَّهِ ۚ وَمَن يُطِعِ ٱللَّهَ وَرَسُولَهُۥ يُدْخِلْهُ جَنَّـٰتٍۢ تَجْرِى مِن تَحْتِهَا ٱلْأَنْهَـٰرُ خَـٰلِدِينَ فِيهَا ۚ وَذَٟلِكَ ٱلْفَوْزُ ٱلْعَظِيمُ ﴿١٣﴾\\
\textamh{14.\  } & وَمَن يَعْصِ ٱللَّهَ وَرَسُولَهُۥ وَيَتَعَدَّ حُدُودَهُۥ يُدْخِلْهُ نَارًا خَـٰلِدًۭا فِيهَا وَلَهُۥ عَذَابٌۭ مُّهِينٌۭ ﴿١٤﴾\\
\textamh{15.\  } & وَٱلَّٰتِى يَأْتِينَ ٱلْفَـٰحِشَةَ مِن نِّسَآئِكُمْ فَٱسْتَشْهِدُوا۟ عَلَيْهِنَّ أَرْبَعَةًۭ مِّنكُمْ ۖ فَإِن شَهِدُوا۟ فَأَمْسِكُوهُنَّ فِى ٱلْبُيُوتِ حَتَّىٰ يَتَوَفَّىٰهُنَّ ٱلْمَوْتُ أَوْ يَجْعَلَ ٱللَّهُ لَهُنَّ سَبِيلًۭا ﴿١٥﴾\\
\textamh{16.\  } & وَٱلَّذَانِ يَأْتِيَـٰنِهَا مِنكُمْ فَـَٔاذُوهُمَا ۖ فَإِن تَابَا وَأَصْلَحَا فَأَعْرِضُوا۟ عَنْهُمَآ ۗ إِنَّ ٱللَّهَ كَانَ تَوَّابًۭا رَّحِيمًا ﴿١٦﴾\\
\textamh{17.\  } & إِنَّمَا ٱلتَّوْبَةُ عَلَى ٱللَّهِ لِلَّذِينَ يَعْمَلُونَ ٱلسُّوٓءَ بِجَهَـٰلَةٍۢ ثُمَّ يَتُوبُونَ مِن قَرِيبٍۢ فَأُو۟لَـٰٓئِكَ يَتُوبُ ٱللَّهُ عَلَيْهِمْ ۗ وَكَانَ ٱللَّهُ عَلِيمًا حَكِيمًۭا ﴿١٧﴾\\
\textamh{18.\  } & وَلَيْسَتِ ٱلتَّوْبَةُ لِلَّذِينَ يَعْمَلُونَ ٱلسَّيِّـَٔاتِ حَتَّىٰٓ إِذَا حَضَرَ أَحَدَهُمُ ٱلْمَوْتُ قَالَ إِنِّى تُبْتُ ٱلْـَٰٔنَ وَلَا ٱلَّذِينَ يَمُوتُونَ وَهُمْ كُفَّارٌ ۚ أُو۟لَـٰٓئِكَ أَعْتَدْنَا لَهُمْ عَذَابًا أَلِيمًۭا ﴿١٨﴾\\
\textamh{19.\  } & يَـٰٓأَيُّهَا ٱلَّذِينَ ءَامَنُوا۟ لَا يَحِلُّ لَكُمْ أَن تَرِثُوا۟ ٱلنِّسَآءَ كَرْهًۭا ۖ وَلَا تَعْضُلُوهُنَّ لِتَذْهَبُوا۟ بِبَعْضِ مَآ ءَاتَيْتُمُوهُنَّ إِلَّآ أَن يَأْتِينَ بِفَـٰحِشَةٍۢ مُّبَيِّنَةٍۢ ۚ وَعَاشِرُوهُنَّ بِٱلْمَعْرُوفِ ۚ فَإِن كَرِهْتُمُوهُنَّ فَعَسَىٰٓ أَن تَكْرَهُوا۟ شَيْـًۭٔا وَيَجْعَلَ ٱللَّهُ فِيهِ خَيْرًۭا كَثِيرًۭا ﴿١٩﴾\\
\textamh{20.\  } & وَإِنْ أَرَدتُّمُ ٱسْتِبْدَالَ زَوْجٍۢ مَّكَانَ زَوْجٍۢ وَءَاتَيْتُمْ إِحْدَىٰهُنَّ قِنطَارًۭا فَلَا تَأْخُذُوا۟ مِنْهُ شَيْـًٔا ۚ أَتَأْخُذُونَهُۥ بُهْتَـٰنًۭا وَإِثْمًۭا مُّبِينًۭا ﴿٢٠﴾\\
\textamh{21.\  } & وَكَيْفَ تَأْخُذُونَهُۥ وَقَدْ أَفْضَىٰ بَعْضُكُمْ إِلَىٰ بَعْضٍۢ وَأَخَذْنَ مِنكُم مِّيثَـٰقًا غَلِيظًۭا ﴿٢١﴾\\
\textamh{22.\  } & وَلَا تَنكِحُوا۟ مَا نَكَحَ ءَابَآؤُكُم مِّنَ ٱلنِّسَآءِ إِلَّا مَا قَدْ سَلَفَ ۚ إِنَّهُۥ كَانَ فَـٰحِشَةًۭ وَمَقْتًۭا وَسَآءَ سَبِيلًا ﴿٢٢﴾\\
\textamh{23.\  } & حُرِّمَتْ عَلَيْكُمْ أُمَّهَـٰتُكُمْ وَبَنَاتُكُمْ وَأَخَوَٟتُكُمْ وَعَمَّٰتُكُمْ وَخَـٰلَـٰتُكُمْ وَبَنَاتُ ٱلْأَخِ وَبَنَاتُ ٱلْأُخْتِ وَأُمَّهَـٰتُكُمُ ٱلَّٰتِىٓ أَرْضَعْنَكُمْ وَأَخَوَٟتُكُم مِّنَ ٱلرَّضَٰعَةِ وَأُمَّهَـٰتُ نِسَآئِكُمْ وَرَبَٰٓئِبُكُمُ ٱلَّٰتِى فِى حُجُورِكُم مِّن نِّسَآئِكُمُ ٱلَّٰتِى دَخَلْتُم بِهِنَّ فَإِن لَّمْ تَكُونُوا۟ دَخَلْتُم بِهِنَّ فَلَا جُنَاحَ عَلَيْكُمْ وَحَلَـٰٓئِلُ أَبْنَآئِكُمُ ٱلَّذِينَ مِنْ أَصْلَـٰبِكُمْ وَأَن تَجْمَعُوا۟ بَيْنَ ٱلْأُخْتَيْنِ إِلَّا مَا قَدْ سَلَفَ ۗ إِنَّ ٱللَّهَ كَانَ غَفُورًۭا رَّحِيمًۭا ﴿٢٣﴾\\
\textamh{24.\  } & ۞ وَٱلْمُحْصَنَـٰتُ مِنَ ٱلنِّسَآءِ إِلَّا مَا مَلَكَتْ أَيْمَـٰنُكُمْ ۖ كِتَـٰبَ ٱللَّهِ عَلَيْكُمْ ۚ وَأُحِلَّ لَكُم مَّا وَرَآءَ ذَٟلِكُمْ أَن تَبْتَغُوا۟ بِأَمْوَٟلِكُم مُّحْصِنِينَ غَيْرَ مُسَـٰفِحِينَ ۚ فَمَا ٱسْتَمْتَعْتُم بِهِۦ مِنْهُنَّ فَـَٔاتُوهُنَّ أُجُورَهُنَّ فَرِيضَةًۭ ۚ وَلَا جُنَاحَ عَلَيْكُمْ فِيمَا تَرَٰضَيْتُم بِهِۦ مِنۢ بَعْدِ ٱلْفَرِيضَةِ ۚ إِنَّ ٱللَّهَ كَانَ عَلِيمًا حَكِيمًۭا ﴿٢٤﴾\\
\textamh{25.\  } & وَمَن لَّمْ يَسْتَطِعْ مِنكُمْ طَوْلًا أَن يَنكِحَ ٱلْمُحْصَنَـٰتِ ٱلْمُؤْمِنَـٰتِ فَمِن مَّا مَلَكَتْ أَيْمَـٰنُكُم مِّن فَتَيَـٰتِكُمُ ٱلْمُؤْمِنَـٰتِ ۚ وَٱللَّهُ أَعْلَمُ بِإِيمَـٰنِكُم ۚ بَعْضُكُم مِّنۢ بَعْضٍۢ ۚ فَٱنكِحُوهُنَّ بِإِذْنِ أَهْلِهِنَّ وَءَاتُوهُنَّ أُجُورَهُنَّ بِٱلْمَعْرُوفِ مُحْصَنَـٰتٍ غَيْرَ مُسَـٰفِحَـٰتٍۢ وَلَا مُتَّخِذَٟتِ أَخْدَانٍۢ ۚ فَإِذَآ أُحْصِنَّ فَإِنْ أَتَيْنَ بِفَـٰحِشَةٍۢ فَعَلَيْهِنَّ نِصْفُ مَا عَلَى ٱلْمُحْصَنَـٰتِ مِنَ ٱلْعَذَابِ ۚ ذَٟلِكَ لِمَنْ خَشِىَ ٱلْعَنَتَ مِنكُمْ ۚ وَأَن تَصْبِرُوا۟ خَيْرٌۭ لَّكُمْ ۗ وَٱللَّهُ غَفُورٌۭ رَّحِيمٌۭ ﴿٢٥﴾\\
\textamh{26.\  } & يُرِيدُ ٱللَّهُ لِيُبَيِّنَ لَكُمْ وَيَهْدِيَكُمْ سُنَنَ ٱلَّذِينَ مِن قَبْلِكُمْ وَيَتُوبَ عَلَيْكُمْ ۗ وَٱللَّهُ عَلِيمٌ حَكِيمٌۭ ﴿٢٦﴾\\
\textamh{27.\  } & وَٱللَّهُ يُرِيدُ أَن يَتُوبَ عَلَيْكُمْ وَيُرِيدُ ٱلَّذِينَ يَتَّبِعُونَ ٱلشَّهَوَٟتِ أَن تَمِيلُوا۟ مَيْلًا عَظِيمًۭا ﴿٢٧﴾\\
\textamh{28.\  } & يُرِيدُ ٱللَّهُ أَن يُخَفِّفَ عَنكُمْ ۚ وَخُلِقَ ٱلْإِنسَـٰنُ ضَعِيفًۭا ﴿٢٨﴾\\
\textamh{29.\  } & يَـٰٓأَيُّهَا ٱلَّذِينَ ءَامَنُوا۟ لَا تَأْكُلُوٓا۟ أَمْوَٟلَكُم بَيْنَكُم بِٱلْبَٰطِلِ إِلَّآ أَن تَكُونَ تِجَٰرَةً عَن تَرَاضٍۢ مِّنكُمْ ۚ وَلَا تَقْتُلُوٓا۟ أَنفُسَكُمْ ۚ إِنَّ ٱللَّهَ كَانَ بِكُمْ رَحِيمًۭا ﴿٢٩﴾\\
\textamh{30.\  } & وَمَن يَفْعَلْ ذَٟلِكَ عُدْوَٟنًۭا وَظُلْمًۭا فَسَوْفَ نُصْلِيهِ نَارًۭا ۚ وَكَانَ ذَٟلِكَ عَلَى ٱللَّهِ يَسِيرًا ﴿٣٠﴾\\
\textamh{31.\  } & إِن تَجْتَنِبُوا۟ كَبَآئِرَ مَا تُنْهَوْنَ عَنْهُ نُكَفِّرْ عَنكُمْ سَيِّـَٔاتِكُمْ وَنُدْخِلْكُم مُّدْخَلًۭا كَرِيمًۭا ﴿٣١﴾\\
\textamh{32.\  } & وَلَا تَتَمَنَّوْا۟ مَا فَضَّلَ ٱللَّهُ بِهِۦ بَعْضَكُمْ عَلَىٰ بَعْضٍۢ ۚ لِّلرِّجَالِ نَصِيبٌۭ مِّمَّا ٱكْتَسَبُوا۟ ۖ وَلِلنِّسَآءِ نَصِيبٌۭ مِّمَّا ٱكْتَسَبْنَ ۚ وَسْـَٔلُوا۟ ٱللَّهَ مِن فَضْلِهِۦٓ ۗ إِنَّ ٱللَّهَ كَانَ بِكُلِّ شَىْءٍ عَلِيمًۭا ﴿٣٢﴾\\
\textamh{33.\  } & وَلِكُلٍّۢ جَعَلْنَا مَوَٟلِىَ مِمَّا تَرَكَ ٱلْوَٟلِدَانِ وَٱلْأَقْرَبُونَ ۚ وَٱلَّذِينَ عَقَدَتْ أَيْمَـٰنُكُمْ فَـَٔاتُوهُمْ نَصِيبَهُمْ ۚ إِنَّ ٱللَّهَ كَانَ عَلَىٰ كُلِّ شَىْءٍۢ شَهِيدًا ﴿٣٣﴾\\
\textamh{34.\  } & ٱلرِّجَالُ قَوَّٰمُونَ عَلَى ٱلنِّسَآءِ بِمَا فَضَّلَ ٱللَّهُ بَعْضَهُمْ عَلَىٰ بَعْضٍۢ وَبِمَآ أَنفَقُوا۟ مِنْ أَمْوَٟلِهِمْ ۚ فَٱلصَّـٰلِحَـٰتُ قَـٰنِتَـٰتٌ حَـٰفِظَـٰتٌۭ لِّلْغَيْبِ بِمَا حَفِظَ ٱللَّهُ ۚ وَٱلَّٰتِى تَخَافُونَ نُشُوزَهُنَّ فَعِظُوهُنَّ وَٱهْجُرُوهُنَّ فِى ٱلْمَضَاجِعِ وَٱضْرِبُوهُنَّ ۖ فَإِنْ أَطَعْنَكُمْ فَلَا تَبْغُوا۟ عَلَيْهِنَّ سَبِيلًا ۗ إِنَّ ٱللَّهَ كَانَ عَلِيًّۭا كَبِيرًۭا ﴿٣٤﴾\\
\textamh{35.\  } & وَإِنْ خِفْتُمْ شِقَاقَ بَيْنِهِمَا فَٱبْعَثُوا۟ حَكَمًۭا مِّنْ أَهْلِهِۦ وَحَكَمًۭا مِّنْ أَهْلِهَآ إِن يُرِيدَآ إِصْلَـٰحًۭا يُوَفِّقِ ٱللَّهُ بَيْنَهُمَآ ۗ إِنَّ ٱللَّهَ كَانَ عَلِيمًا خَبِيرًۭا ﴿٣٥﴾\\
\textamh{36.\  } & ۞ وَٱعْبُدُوا۟ ٱللَّهَ وَلَا تُشْرِكُوا۟ بِهِۦ شَيْـًۭٔا ۖ وَبِٱلْوَٟلِدَيْنِ إِحْسَـٰنًۭا وَبِذِى ٱلْقُرْبَىٰ وَٱلْيَتَـٰمَىٰ وَٱلْمَسَـٰكِينِ وَٱلْجَارِ ذِى ٱلْقُرْبَىٰ وَٱلْجَارِ ٱلْجُنُبِ وَٱلصَّاحِبِ بِٱلْجَنۢبِ وَٱبْنِ ٱلسَّبِيلِ وَمَا مَلَكَتْ أَيْمَـٰنُكُمْ ۗ إِنَّ ٱللَّهَ لَا يُحِبُّ مَن كَانَ مُخْتَالًۭا فَخُورًا ﴿٣٦﴾\\
\textamh{37.\  } & ٱلَّذِينَ يَبْخَلُونَ وَيَأْمُرُونَ ٱلنَّاسَ بِٱلْبُخْلِ وَيَكْتُمُونَ مَآ ءَاتَىٰهُمُ ٱللَّهُ مِن فَضْلِهِۦ ۗ وَأَعْتَدْنَا لِلْكَـٰفِرِينَ عَذَابًۭا مُّهِينًۭا ﴿٣٧﴾\\
\textamh{38.\  } & وَٱلَّذِينَ يُنفِقُونَ أَمْوَٟلَهُمْ رِئَآءَ ٱلنَّاسِ وَلَا يُؤْمِنُونَ بِٱللَّهِ وَلَا بِٱلْيَوْمِ ٱلْءَاخِرِ ۗ وَمَن يَكُنِ ٱلشَّيْطَٰنُ لَهُۥ قَرِينًۭا فَسَآءَ قَرِينًۭا ﴿٣٨﴾\\
\textamh{39.\  } & وَمَاذَا عَلَيْهِمْ لَوْ ءَامَنُوا۟ بِٱللَّهِ وَٱلْيَوْمِ ٱلْءَاخِرِ وَأَنفَقُوا۟ مِمَّا رَزَقَهُمُ ٱللَّهُ ۚ وَكَانَ ٱللَّهُ بِهِمْ عَلِيمًا ﴿٣٩﴾\\
\textamh{40.\  } & إِنَّ ٱللَّهَ لَا يَظْلِمُ مِثْقَالَ ذَرَّةٍۢ ۖ وَإِن تَكُ حَسَنَةًۭ يُضَٰعِفْهَا وَيُؤْتِ مِن لَّدُنْهُ أَجْرًا عَظِيمًۭا ﴿٤٠﴾\\
\textamh{41.\  } & فَكَيْفَ إِذَا جِئْنَا مِن كُلِّ أُمَّةٍۭ بِشَهِيدٍۢ وَجِئْنَا بِكَ عَلَىٰ هَـٰٓؤُلَآءِ شَهِيدًۭا ﴿٤١﴾\\
\textamh{42.\  } & يَوْمَئِذٍۢ يَوَدُّ ٱلَّذِينَ كَفَرُوا۟ وَعَصَوُا۟ ٱلرَّسُولَ لَوْ تُسَوَّىٰ بِهِمُ ٱلْأَرْضُ وَلَا يَكْتُمُونَ ٱللَّهَ حَدِيثًۭا ﴿٤٢﴾\\
\textamh{43.\  } & يَـٰٓأَيُّهَا ٱلَّذِينَ ءَامَنُوا۟ لَا تَقْرَبُوا۟ ٱلصَّلَوٰةَ وَأَنتُمْ سُكَـٰرَىٰ حَتَّىٰ تَعْلَمُوا۟ مَا تَقُولُونَ وَلَا جُنُبًا إِلَّا عَابِرِى سَبِيلٍ حَتَّىٰ تَغْتَسِلُوا۟ ۚ وَإِن كُنتُم مَّرْضَىٰٓ أَوْ عَلَىٰ سَفَرٍ أَوْ جَآءَ أَحَدٌۭ مِّنكُم مِّنَ ٱلْغَآئِطِ أَوْ لَـٰمَسْتُمُ ٱلنِّسَآءَ فَلَمْ تَجِدُوا۟ مَآءًۭ فَتَيَمَّمُوا۟ صَعِيدًۭا طَيِّبًۭا فَٱمْسَحُوا۟ بِوُجُوهِكُمْ وَأَيْدِيكُمْ ۗ إِنَّ ٱللَّهَ كَانَ عَفُوًّا غَفُورًا ﴿٤٣﴾\\
\textamh{44.\  } & أَلَمْ تَرَ إِلَى ٱلَّذِينَ أُوتُوا۟ نَصِيبًۭا مِّنَ ٱلْكِتَـٰبِ يَشْتَرُونَ ٱلضَّلَـٰلَةَ وَيُرِيدُونَ أَن تَضِلُّوا۟ ٱلسَّبِيلَ ﴿٤٤﴾\\
\textamh{45.\  } & وَٱللَّهُ أَعْلَمُ بِأَعْدَآئِكُمْ ۚ وَكَفَىٰ بِٱللَّهِ وَلِيًّۭا وَكَفَىٰ بِٱللَّهِ نَصِيرًۭا ﴿٤٥﴾\\
\textamh{46.\  } & مِّنَ ٱلَّذِينَ هَادُوا۟ يُحَرِّفُونَ ٱلْكَلِمَ عَن مَّوَاضِعِهِۦ وَيَقُولُونَ سَمِعْنَا وَعَصَيْنَا وَٱسْمَعْ غَيْرَ مُسْمَعٍۢ وَرَٰعِنَا لَيًّۢا بِأَلْسِنَتِهِمْ وَطَعْنًۭا فِى ٱلدِّينِ ۚ وَلَوْ أَنَّهُمْ قَالُوا۟ سَمِعْنَا وَأَطَعْنَا وَٱسْمَعْ وَٱنظُرْنَا لَكَانَ خَيْرًۭا لَّهُمْ وَأَقْوَمَ وَلَـٰكِن لَّعَنَهُمُ ٱللَّهُ بِكُفْرِهِمْ فَلَا يُؤْمِنُونَ إِلَّا قَلِيلًۭا ﴿٤٦﴾\\
\textamh{47.\  } & يَـٰٓأَيُّهَا ٱلَّذِينَ أُوتُوا۟ ٱلْكِتَـٰبَ ءَامِنُوا۟ بِمَا نَزَّلْنَا مُصَدِّقًۭا لِّمَا مَعَكُم مِّن قَبْلِ أَن نَّطْمِسَ وُجُوهًۭا فَنَرُدَّهَا عَلَىٰٓ أَدْبَارِهَآ أَوْ نَلْعَنَهُمْ كَمَا لَعَنَّآ أَصْحَـٰبَ ٱلسَّبْتِ ۚ وَكَانَ أَمْرُ ٱللَّهِ مَفْعُولًا ﴿٤٧﴾\\
\textamh{48.\  } & إِنَّ ٱللَّهَ لَا يَغْفِرُ أَن يُشْرَكَ بِهِۦ وَيَغْفِرُ مَا دُونَ ذَٟلِكَ لِمَن يَشَآءُ ۚ وَمَن يُشْرِكْ بِٱللَّهِ فَقَدِ ٱفْتَرَىٰٓ إِثْمًا عَظِيمًا ﴿٤٨﴾\\
\textamh{49.\  } & أَلَمْ تَرَ إِلَى ٱلَّذِينَ يُزَكُّونَ أَنفُسَهُم ۚ بَلِ ٱللَّهُ يُزَكِّى مَن يَشَآءُ وَلَا يُظْلَمُونَ فَتِيلًا ﴿٤٩﴾\\
\textamh{50.\  } & ٱنظُرْ كَيْفَ يَفْتَرُونَ عَلَى ٱللَّهِ ٱلْكَذِبَ ۖ وَكَفَىٰ بِهِۦٓ إِثْمًۭا مُّبِينًا ﴿٥٠﴾\\
\textamh{51.\  } & أَلَمْ تَرَ إِلَى ٱلَّذِينَ أُوتُوا۟ نَصِيبًۭا مِّنَ ٱلْكِتَـٰبِ يُؤْمِنُونَ بِٱلْجِبْتِ وَٱلطَّٰغُوتِ وَيَقُولُونَ لِلَّذِينَ كَفَرُوا۟ هَـٰٓؤُلَآءِ أَهْدَىٰ مِنَ ٱلَّذِينَ ءَامَنُوا۟ سَبِيلًا ﴿٥١﴾\\
\textamh{52.\  } & أُو۟لَـٰٓئِكَ ٱلَّذِينَ لَعَنَهُمُ ٱللَّهُ ۖ وَمَن يَلْعَنِ ٱللَّهُ فَلَن تَجِدَ لَهُۥ نَصِيرًا ﴿٥٢﴾\\
\textamh{53.\  } & أَمْ لَهُمْ نَصِيبٌۭ مِّنَ ٱلْمُلْكِ فَإِذًۭا لَّا يُؤْتُونَ ٱلنَّاسَ نَقِيرًا ﴿٥٣﴾\\
\textamh{54.\  } & أَمْ يَحْسُدُونَ ٱلنَّاسَ عَلَىٰ مَآ ءَاتَىٰهُمُ ٱللَّهُ مِن فَضْلِهِۦ ۖ فَقَدْ ءَاتَيْنَآ ءَالَ إِبْرَٰهِيمَ ٱلْكِتَـٰبَ وَٱلْحِكْمَةَ وَءَاتَيْنَـٰهُم مُّلْكًا عَظِيمًۭا ﴿٥٤﴾\\
\textamh{55.\  } & فَمِنْهُم مَّنْ ءَامَنَ بِهِۦ وَمِنْهُم مَّن صَدَّ عَنْهُ ۚ وَكَفَىٰ بِجَهَنَّمَ سَعِيرًا ﴿٥٥﴾\\
\textamh{56.\  } & إِنَّ ٱلَّذِينَ كَفَرُوا۟ بِـَٔايَـٰتِنَا سَوْفَ نُصْلِيهِمْ نَارًۭا كُلَّمَا نَضِجَتْ جُلُودُهُم بَدَّلْنَـٰهُمْ جُلُودًا غَيْرَهَا لِيَذُوقُوا۟ ٱلْعَذَابَ ۗ إِنَّ ٱللَّهَ كَانَ عَزِيزًا حَكِيمًۭا ﴿٥٦﴾\\
\textamh{57.\  } & وَٱلَّذِينَ ءَامَنُوا۟ وَعَمِلُوا۟ ٱلصَّـٰلِحَـٰتِ سَنُدْخِلُهُمْ جَنَّـٰتٍۢ تَجْرِى مِن تَحْتِهَا ٱلْأَنْهَـٰرُ خَـٰلِدِينَ فِيهَآ أَبَدًۭا ۖ لَّهُمْ فِيهَآ أَزْوَٟجٌۭ مُّطَهَّرَةٌۭ ۖ وَنُدْخِلُهُمْ ظِلًّۭا ظَلِيلًا ﴿٥٧﴾\\
\textamh{58.\  } & ۞ إِنَّ ٱللَّهَ يَأْمُرُكُمْ أَن تُؤَدُّوا۟ ٱلْأَمَـٰنَـٰتِ إِلَىٰٓ أَهْلِهَا وَإِذَا حَكَمْتُم بَيْنَ ٱلنَّاسِ أَن تَحْكُمُوا۟ بِٱلْعَدْلِ ۚ إِنَّ ٱللَّهَ نِعِمَّا يَعِظُكُم بِهِۦٓ ۗ إِنَّ ٱللَّهَ كَانَ سَمِيعًۢا بَصِيرًۭا ﴿٥٨﴾\\
\textamh{59.\  } & يَـٰٓأَيُّهَا ٱلَّذِينَ ءَامَنُوٓا۟ أَطِيعُوا۟ ٱللَّهَ وَأَطِيعُوا۟ ٱلرَّسُولَ وَأُو۟لِى ٱلْأَمْرِ مِنكُمْ ۖ فَإِن تَنَـٰزَعْتُمْ فِى شَىْءٍۢ فَرُدُّوهُ إِلَى ٱللَّهِ وَٱلرَّسُولِ إِن كُنتُمْ تُؤْمِنُونَ بِٱللَّهِ وَٱلْيَوْمِ ٱلْءَاخِرِ ۚ ذَٟلِكَ خَيْرٌۭ وَأَحْسَنُ تَأْوِيلًا ﴿٥٩﴾\\
\textamh{60.\  } & أَلَمْ تَرَ إِلَى ٱلَّذِينَ يَزْعُمُونَ أَنَّهُمْ ءَامَنُوا۟ بِمَآ أُنزِلَ إِلَيْكَ وَمَآ أُنزِلَ مِن قَبْلِكَ يُرِيدُونَ أَن يَتَحَاكَمُوٓا۟ إِلَى ٱلطَّٰغُوتِ وَقَدْ أُمِرُوٓا۟ أَن يَكْفُرُوا۟ بِهِۦ وَيُرِيدُ ٱلشَّيْطَٰنُ أَن يُضِلَّهُمْ ضَلَـٰلًۢا بَعِيدًۭا ﴿٦٠﴾\\
\textamh{61.\  } & وَإِذَا قِيلَ لَهُمْ تَعَالَوْا۟ إِلَىٰ مَآ أَنزَلَ ٱللَّهُ وَإِلَى ٱلرَّسُولِ رَأَيْتَ ٱلْمُنَـٰفِقِينَ يَصُدُّونَ عَنكَ صُدُودًۭا ﴿٦١﴾\\
\textamh{62.\  } & فَكَيْفَ إِذَآ أَصَـٰبَتْهُم مُّصِيبَةٌۢ بِمَا قَدَّمَتْ أَيْدِيهِمْ ثُمَّ جَآءُوكَ يَحْلِفُونَ بِٱللَّهِ إِنْ أَرَدْنَآ إِلَّآ إِحْسَـٰنًۭا وَتَوْفِيقًا ﴿٦٢﴾\\
\textamh{63.\  } & أُو۟لَـٰٓئِكَ ٱلَّذِينَ يَعْلَمُ ٱللَّهُ مَا فِى قُلُوبِهِمْ فَأَعْرِضْ عَنْهُمْ وَعِظْهُمْ وَقُل لَّهُمْ فِىٓ أَنفُسِهِمْ قَوْلًۢا بَلِيغًۭا ﴿٦٣﴾\\
\textamh{64.\  } & وَمَآ أَرْسَلْنَا مِن رَّسُولٍ إِلَّا لِيُطَاعَ بِإِذْنِ ٱللَّهِ ۚ وَلَوْ أَنَّهُمْ إِذ ظَّلَمُوٓا۟ أَنفُسَهُمْ جَآءُوكَ فَٱسْتَغْفَرُوا۟ ٱللَّهَ وَٱسْتَغْفَرَ لَهُمُ ٱلرَّسُولُ لَوَجَدُوا۟ ٱللَّهَ تَوَّابًۭا رَّحِيمًۭا ﴿٦٤﴾\\
\textamh{65.\  } & فَلَا وَرَبِّكَ لَا يُؤْمِنُونَ حَتَّىٰ يُحَكِّمُوكَ فِيمَا شَجَرَ بَيْنَهُمْ ثُمَّ لَا يَجِدُوا۟ فِىٓ أَنفُسِهِمْ حَرَجًۭا مِّمَّا قَضَيْتَ وَيُسَلِّمُوا۟ تَسْلِيمًۭا ﴿٦٥﴾\\
\textamh{66.\  } & وَلَوْ أَنَّا كَتَبْنَا عَلَيْهِمْ أَنِ ٱقْتُلُوٓا۟ أَنفُسَكُمْ أَوِ ٱخْرُجُوا۟ مِن دِيَـٰرِكُم مَّا فَعَلُوهُ إِلَّا قَلِيلٌۭ مِّنْهُمْ ۖ وَلَوْ أَنَّهُمْ فَعَلُوا۟ مَا يُوعَظُونَ بِهِۦ لَكَانَ خَيْرًۭا لَّهُمْ وَأَشَدَّ تَثْبِيتًۭا ﴿٦٦﴾\\
\textamh{67.\  } & وَإِذًۭا لَّءَاتَيْنَـٰهُم مِّن لَّدُنَّآ أَجْرًا عَظِيمًۭا ﴿٦٧﴾\\
\textamh{68.\  } & وَلَهَدَيْنَـٰهُمْ صِرَٰطًۭا مُّسْتَقِيمًۭا ﴿٦٨﴾\\
\textamh{69.\  } & وَمَن يُطِعِ ٱللَّهَ وَٱلرَّسُولَ فَأُو۟لَـٰٓئِكَ مَعَ ٱلَّذِينَ أَنْعَمَ ٱللَّهُ عَلَيْهِم مِّنَ ٱلنَّبِيِّۦنَ وَٱلصِّدِّيقِينَ وَٱلشُّهَدَآءِ وَٱلصَّـٰلِحِينَ ۚ وَحَسُنَ أُو۟لَـٰٓئِكَ رَفِيقًۭا ﴿٦٩﴾\\
\textamh{70.\  } & ذَٟلِكَ ٱلْفَضْلُ مِنَ ٱللَّهِ ۚ وَكَفَىٰ بِٱللَّهِ عَلِيمًۭا ﴿٧٠﴾\\
\textamh{71.\  } & يَـٰٓأَيُّهَا ٱلَّذِينَ ءَامَنُوا۟ خُذُوا۟ حِذْرَكُمْ فَٱنفِرُوا۟ ثُبَاتٍ أَوِ ٱنفِرُوا۟ جَمِيعًۭا ﴿٧١﴾\\
\textamh{72.\  } & وَإِنَّ مِنكُمْ لَمَن لَّيُبَطِّئَنَّ فَإِنْ أَصَـٰبَتْكُم مُّصِيبَةٌۭ قَالَ قَدْ أَنْعَمَ ٱللَّهُ عَلَىَّ إِذْ لَمْ أَكُن مَّعَهُمْ شَهِيدًۭا ﴿٧٢﴾\\
\textamh{73.\  } & وَلَئِنْ أَصَـٰبَكُمْ فَضْلٌۭ مِّنَ ٱللَّهِ لَيَقُولَنَّ كَأَن لَّمْ تَكُنۢ بَيْنَكُمْ وَبَيْنَهُۥ مَوَدَّةٌۭ يَـٰلَيْتَنِى كُنتُ مَعَهُمْ فَأَفُوزَ فَوْزًا عَظِيمًۭا ﴿٧٣﴾\\
\textamh{74.\  } & ۞ فَلْيُقَـٰتِلْ فِى سَبِيلِ ٱللَّهِ ٱلَّذِينَ يَشْرُونَ ٱلْحَيَوٰةَ ٱلدُّنْيَا بِٱلْءَاخِرَةِ ۚ وَمَن يُقَـٰتِلْ فِى سَبِيلِ ٱللَّهِ فَيُقْتَلْ أَوْ يَغْلِبْ فَسَوْفَ نُؤْتِيهِ أَجْرًا عَظِيمًۭا ﴿٧٤﴾\\
\textamh{75.\  } & وَمَا لَكُمْ لَا تُقَـٰتِلُونَ فِى سَبِيلِ ٱللَّهِ وَٱلْمُسْتَضْعَفِينَ مِنَ ٱلرِّجَالِ وَٱلنِّسَآءِ وَٱلْوِلْدَٟنِ ٱلَّذِينَ يَقُولُونَ رَبَّنَآ أَخْرِجْنَا مِنْ هَـٰذِهِ ٱلْقَرْيَةِ ٱلظَّالِمِ أَهْلُهَا وَٱجْعَل لَّنَا مِن لَّدُنكَ وَلِيًّۭا وَٱجْعَل لَّنَا مِن لَّدُنكَ نَصِيرًا ﴿٧٥﴾\\
\textamh{76.\  } & ٱلَّذِينَ ءَامَنُوا۟ يُقَـٰتِلُونَ فِى سَبِيلِ ٱللَّهِ ۖ وَٱلَّذِينَ كَفَرُوا۟ يُقَـٰتِلُونَ فِى سَبِيلِ ٱلطَّٰغُوتِ فَقَـٰتِلُوٓا۟ أَوْلِيَآءَ ٱلشَّيْطَٰنِ ۖ إِنَّ كَيْدَ ٱلشَّيْطَٰنِ كَانَ ضَعِيفًا ﴿٧٦﴾\\
\textamh{77.\  } & أَلَمْ تَرَ إِلَى ٱلَّذِينَ قِيلَ لَهُمْ كُفُّوٓا۟ أَيْدِيَكُمْ وَأَقِيمُوا۟ ٱلصَّلَوٰةَ وَءَاتُوا۟ ٱلزَّكَوٰةَ فَلَمَّا كُتِبَ عَلَيْهِمُ ٱلْقِتَالُ إِذَا فَرِيقٌۭ مِّنْهُمْ يَخْشَوْنَ ٱلنَّاسَ كَخَشْيَةِ ٱللَّهِ أَوْ أَشَدَّ خَشْيَةًۭ ۚ وَقَالُوا۟ رَبَّنَا لِمَ كَتَبْتَ عَلَيْنَا ٱلْقِتَالَ لَوْلَآ أَخَّرْتَنَآ إِلَىٰٓ أَجَلٍۢ قَرِيبٍۢ ۗ قُلْ مَتَـٰعُ ٱلدُّنْيَا قَلِيلٌۭ وَٱلْءَاخِرَةُ خَيْرٌۭ لِّمَنِ ٱتَّقَىٰ وَلَا تُظْلَمُونَ فَتِيلًا ﴿٧٧﴾\\
\textamh{78.\  } & أَيْنَمَا تَكُونُوا۟ يُدْرِككُّمُ ٱلْمَوْتُ وَلَوْ كُنتُمْ فِى بُرُوجٍۢ مُّشَيَّدَةٍۢ ۗ وَإِن تُصِبْهُمْ حَسَنَةٌۭ يَقُولُوا۟ هَـٰذِهِۦ مِنْ عِندِ ٱللَّهِ ۖ وَإِن تُصِبْهُمْ سَيِّئَةٌۭ يَقُولُوا۟ هَـٰذِهِۦ مِنْ عِندِكَ ۚ قُلْ كُلٌّۭ مِّنْ عِندِ ٱللَّهِ ۖ فَمَالِ هَـٰٓؤُلَآءِ ٱلْقَوْمِ لَا يَكَادُونَ يَفْقَهُونَ حَدِيثًۭا ﴿٧٨﴾\\
\textamh{79.\  } & مَّآ أَصَابَكَ مِنْ حَسَنَةٍۢ فَمِنَ ٱللَّهِ ۖ وَمَآ أَصَابَكَ مِن سَيِّئَةٍۢ فَمِن نَّفْسِكَ ۚ وَأَرْسَلْنَـٰكَ لِلنَّاسِ رَسُولًۭا ۚ وَكَفَىٰ بِٱللَّهِ شَهِيدًۭا ﴿٧٩﴾\\
\textamh{80.\  } & مَّن يُطِعِ ٱلرَّسُولَ فَقَدْ أَطَاعَ ٱللَّهَ ۖ وَمَن تَوَلَّىٰ فَمَآ أَرْسَلْنَـٰكَ عَلَيْهِمْ حَفِيظًۭا ﴿٨٠﴾\\
\textamh{81.\  } & وَيَقُولُونَ طَاعَةٌۭ فَإِذَا بَرَزُوا۟ مِنْ عِندِكَ بَيَّتَ طَآئِفَةٌۭ مِّنْهُمْ غَيْرَ ٱلَّذِى تَقُولُ ۖ وَٱللَّهُ يَكْتُبُ مَا يُبَيِّتُونَ ۖ فَأَعْرِضْ عَنْهُمْ وَتَوَكَّلْ عَلَى ٱللَّهِ ۚ وَكَفَىٰ بِٱللَّهِ وَكِيلًا ﴿٨١﴾\\
\textamh{82.\  } & أَفَلَا يَتَدَبَّرُونَ ٱلْقُرْءَانَ ۚ وَلَوْ كَانَ مِنْ عِندِ غَيْرِ ٱللَّهِ لَوَجَدُوا۟ فِيهِ ٱخْتِلَـٰفًۭا كَثِيرًۭا ﴿٨٢﴾\\
\textamh{83.\  } & وَإِذَا جَآءَهُمْ أَمْرٌۭ مِّنَ ٱلْأَمْنِ أَوِ ٱلْخَوْفِ أَذَاعُوا۟ بِهِۦ ۖ وَلَوْ رَدُّوهُ إِلَى ٱلرَّسُولِ وَإِلَىٰٓ أُو۟لِى ٱلْأَمْرِ مِنْهُمْ لَعَلِمَهُ ٱلَّذِينَ يَسْتَنۢبِطُونَهُۥ مِنْهُمْ ۗ وَلَوْلَا فَضْلُ ٱللَّهِ عَلَيْكُمْ وَرَحْمَتُهُۥ لَٱتَّبَعْتُمُ ٱلشَّيْطَٰنَ إِلَّا قَلِيلًۭا ﴿٨٣﴾\\
\textamh{84.\  } & فَقَـٰتِلْ فِى سَبِيلِ ٱللَّهِ لَا تُكَلَّفُ إِلَّا نَفْسَكَ ۚ وَحَرِّضِ ٱلْمُؤْمِنِينَ ۖ عَسَى ٱللَّهُ أَن يَكُفَّ بَأْسَ ٱلَّذِينَ كَفَرُوا۟ ۚ وَٱللَّهُ أَشَدُّ بَأْسًۭا وَأَشَدُّ تَنكِيلًۭا ﴿٨٤﴾\\
\textamh{85.\  } & مَّن يَشْفَعْ شَفَـٰعَةً حَسَنَةًۭ يَكُن لَّهُۥ نَصِيبٌۭ مِّنْهَا ۖ وَمَن يَشْفَعْ شَفَـٰعَةًۭ سَيِّئَةًۭ يَكُن لَّهُۥ كِفْلٌۭ مِّنْهَا ۗ وَكَانَ ٱللَّهُ عَلَىٰ كُلِّ شَىْءٍۢ مُّقِيتًۭا ﴿٨٥﴾\\
\textamh{86.\  } & وَإِذَا حُيِّيتُم بِتَحِيَّةٍۢ فَحَيُّوا۟ بِأَحْسَنَ مِنْهَآ أَوْ رُدُّوهَآ ۗ إِنَّ ٱللَّهَ كَانَ عَلَىٰ كُلِّ شَىْءٍ حَسِيبًا ﴿٨٦﴾\\
\textamh{87.\  } & ٱللَّهُ لَآ إِلَـٰهَ إِلَّا هُوَ ۚ لَيَجْمَعَنَّكُمْ إِلَىٰ يَوْمِ ٱلْقِيَـٰمَةِ لَا رَيْبَ فِيهِ ۗ وَمَنْ أَصْدَقُ مِنَ ٱللَّهِ حَدِيثًۭا ﴿٨٧﴾\\
\textamh{88.\  } & ۞ فَمَا لَكُمْ فِى ٱلْمُنَـٰفِقِينَ فِئَتَيْنِ وَٱللَّهُ أَرْكَسَهُم بِمَا كَسَبُوٓا۟ ۚ أَتُرِيدُونَ أَن تَهْدُوا۟ مَنْ أَضَلَّ ٱللَّهُ ۖ وَمَن يُضْلِلِ ٱللَّهُ فَلَن تَجِدَ لَهُۥ سَبِيلًۭا ﴿٨٨﴾\\
\textamh{89.\  } & وَدُّوا۟ لَوْ تَكْفُرُونَ كَمَا كَفَرُوا۟ فَتَكُونُونَ سَوَآءًۭ ۖ فَلَا تَتَّخِذُوا۟ مِنْهُمْ أَوْلِيَآءَ حَتَّىٰ يُهَاجِرُوا۟ فِى سَبِيلِ ٱللَّهِ ۚ فَإِن تَوَلَّوْا۟ فَخُذُوهُمْ وَٱقْتُلُوهُمْ حَيْثُ وَجَدتُّمُوهُمْ ۖ وَلَا تَتَّخِذُوا۟ مِنْهُمْ وَلِيًّۭا وَلَا نَصِيرًا ﴿٨٩﴾\\
\textamh{90.\  } & إِلَّا ٱلَّذِينَ يَصِلُونَ إِلَىٰ قَوْمٍۭ بَيْنَكُمْ وَبَيْنَهُم مِّيثَـٰقٌ أَوْ جَآءُوكُمْ حَصِرَتْ صُدُورُهُمْ أَن يُقَـٰتِلُوكُمْ أَوْ يُقَـٰتِلُوا۟ قَوْمَهُمْ ۚ وَلَوْ شَآءَ ٱللَّهُ لَسَلَّطَهُمْ عَلَيْكُمْ فَلَقَـٰتَلُوكُمْ ۚ فَإِنِ ٱعْتَزَلُوكُمْ فَلَمْ يُقَـٰتِلُوكُمْ وَأَلْقَوْا۟ إِلَيْكُمُ ٱلسَّلَمَ فَمَا جَعَلَ ٱللَّهُ لَكُمْ عَلَيْهِمْ سَبِيلًۭا ﴿٩٠﴾\\
\textamh{91.\  } & سَتَجِدُونَ ءَاخَرِينَ يُرِيدُونَ أَن يَأْمَنُوكُمْ وَيَأْمَنُوا۟ قَوْمَهُمْ كُلَّ مَا رُدُّوٓا۟ إِلَى ٱلْفِتْنَةِ أُرْكِسُوا۟ فِيهَا ۚ فَإِن لَّمْ يَعْتَزِلُوكُمْ وَيُلْقُوٓا۟ إِلَيْكُمُ ٱلسَّلَمَ وَيَكُفُّوٓا۟ أَيْدِيَهُمْ فَخُذُوهُمْ وَٱقْتُلُوهُمْ حَيْثُ ثَقِفْتُمُوهُمْ ۚ وَأُو۟لَـٰٓئِكُمْ جَعَلْنَا لَكُمْ عَلَيْهِمْ سُلْطَٰنًۭا مُّبِينًۭا ﴿٩١﴾\\
\textamh{92.\  } & وَمَا كَانَ لِمُؤْمِنٍ أَن يَقْتُلَ مُؤْمِنًا إِلَّا خَطَـًۭٔا ۚ وَمَن قَتَلَ مُؤْمِنًا خَطَـًۭٔا فَتَحْرِيرُ رَقَبَةٍۢ مُّؤْمِنَةٍۢ وَدِيَةٌۭ مُّسَلَّمَةٌ إِلَىٰٓ أَهْلِهِۦٓ إِلَّآ أَن يَصَّدَّقُوا۟ ۚ فَإِن كَانَ مِن قَوْمٍ عَدُوٍّۢ لَّكُمْ وَهُوَ مُؤْمِنٌۭ فَتَحْرِيرُ رَقَبَةٍۢ مُّؤْمِنَةٍۢ ۖ وَإِن كَانَ مِن قَوْمٍۭ بَيْنَكُمْ وَبَيْنَهُم مِّيثَـٰقٌۭ فَدِيَةٌۭ مُّسَلَّمَةٌ إِلَىٰٓ أَهْلِهِۦ وَتَحْرِيرُ رَقَبَةٍۢ مُّؤْمِنَةٍۢ ۖ فَمَن لَّمْ يَجِدْ فَصِيَامُ شَهْرَيْنِ مُتَتَابِعَيْنِ تَوْبَةًۭ مِّنَ ٱللَّهِ ۗ وَكَانَ ٱللَّهُ عَلِيمًا حَكِيمًۭا ﴿٩٢﴾\\
\textamh{93.\  } & وَمَن يَقْتُلْ مُؤْمِنًۭا مُّتَعَمِّدًۭا فَجَزَآؤُهُۥ جَهَنَّمُ خَـٰلِدًۭا فِيهَا وَغَضِبَ ٱللَّهُ عَلَيْهِ وَلَعَنَهُۥ وَأَعَدَّ لَهُۥ عَذَابًا عَظِيمًۭا ﴿٩٣﴾\\
\textamh{94.\  } & يَـٰٓأَيُّهَا ٱلَّذِينَ ءَامَنُوٓا۟ إِذَا ضَرَبْتُمْ فِى سَبِيلِ ٱللَّهِ فَتَبَيَّنُوا۟ وَلَا تَقُولُوا۟ لِمَنْ أَلْقَىٰٓ إِلَيْكُمُ ٱلسَّلَـٰمَ لَسْتَ مُؤْمِنًۭا تَبْتَغُونَ عَرَضَ ٱلْحَيَوٰةِ ٱلدُّنْيَا فَعِندَ ٱللَّهِ مَغَانِمُ كَثِيرَةٌۭ ۚ كَذَٟلِكَ كُنتُم مِّن قَبْلُ فَمَنَّ ٱللَّهُ عَلَيْكُمْ فَتَبَيَّنُوٓا۟ ۚ إِنَّ ٱللَّهَ كَانَ بِمَا تَعْمَلُونَ خَبِيرًۭا ﴿٩٤﴾\\
\textamh{95.\  } & لَّا يَسْتَوِى ٱلْقَـٰعِدُونَ مِنَ ٱلْمُؤْمِنِينَ غَيْرُ أُو۟لِى ٱلضَّرَرِ وَٱلْمُجَٰهِدُونَ فِى سَبِيلِ ٱللَّهِ بِأَمْوَٟلِهِمْ وَأَنفُسِهِمْ ۚ فَضَّلَ ٱللَّهُ ٱلْمُجَٰهِدِينَ بِأَمْوَٟلِهِمْ وَأَنفُسِهِمْ عَلَى ٱلْقَـٰعِدِينَ دَرَجَةًۭ ۚ وَكُلًّۭا وَعَدَ ٱللَّهُ ٱلْحُسْنَىٰ ۚ وَفَضَّلَ ٱللَّهُ ٱلْمُجَٰهِدِينَ عَلَى ٱلْقَـٰعِدِينَ أَجْرًا عَظِيمًۭا ﴿٩٥﴾\\
\textamh{96.\  } & دَرَجَٰتٍۢ مِّنْهُ وَمَغْفِرَةًۭ وَرَحْمَةًۭ ۚ وَكَانَ ٱللَّهُ غَفُورًۭا رَّحِيمًا ﴿٩٦﴾\\
\textamh{97.\  } & إِنَّ ٱلَّذِينَ تَوَفَّىٰهُمُ ٱلْمَلَـٰٓئِكَةُ ظَالِمِىٓ أَنفُسِهِمْ قَالُوا۟ فِيمَ كُنتُمْ ۖ قَالُوا۟ كُنَّا مُسْتَضْعَفِينَ فِى ٱلْأَرْضِ ۚ قَالُوٓا۟ أَلَمْ تَكُنْ أَرْضُ ٱللَّهِ وَٟسِعَةًۭ فَتُهَاجِرُوا۟ فِيهَا ۚ فَأُو۟لَـٰٓئِكَ مَأْوَىٰهُمْ جَهَنَّمُ ۖ وَسَآءَتْ مَصِيرًا ﴿٩٧﴾\\
\textamh{98.\  } & إِلَّا ٱلْمُسْتَضْعَفِينَ مِنَ ٱلرِّجَالِ وَٱلنِّسَآءِ وَٱلْوِلْدَٟنِ لَا يَسْتَطِيعُونَ حِيلَةًۭ وَلَا يَهْتَدُونَ سَبِيلًۭا ﴿٩٨﴾\\
\textamh{99.\  } & فَأُو۟لَـٰٓئِكَ عَسَى ٱللَّهُ أَن يَعْفُوَ عَنْهُمْ ۚ وَكَانَ ٱللَّهُ عَفُوًّا غَفُورًۭا ﴿٩٩﴾\\
\textamh{100.\  } & ۞ وَمَن يُهَاجِرْ فِى سَبِيلِ ٱللَّهِ يَجِدْ فِى ٱلْأَرْضِ مُرَٰغَمًۭا كَثِيرًۭا وَسَعَةًۭ ۚ وَمَن يَخْرُجْ مِنۢ بَيْتِهِۦ مُهَاجِرًا إِلَى ٱللَّهِ وَرَسُولِهِۦ ثُمَّ يُدْرِكْهُ ٱلْمَوْتُ فَقَدْ وَقَعَ أَجْرُهُۥ عَلَى ٱللَّهِ ۗ وَكَانَ ٱللَّهُ غَفُورًۭا رَّحِيمًۭا ﴿١٠٠﴾\\
\textamh{101.\  } & وَإِذَا ضَرَبْتُمْ فِى ٱلْأَرْضِ فَلَيْسَ عَلَيْكُمْ جُنَاحٌ أَن تَقْصُرُوا۟ مِنَ ٱلصَّلَوٰةِ إِنْ خِفْتُمْ أَن يَفْتِنَكُمُ ٱلَّذِينَ كَفَرُوٓا۟ ۚ إِنَّ ٱلْكَـٰفِرِينَ كَانُوا۟ لَكُمْ عَدُوًّۭا مُّبِينًۭا ﴿١٠١﴾\\
\textamh{102.\  } & وَإِذَا كُنتَ فِيهِمْ فَأَقَمْتَ لَهُمُ ٱلصَّلَوٰةَ فَلْتَقُمْ طَآئِفَةٌۭ مِّنْهُم مَّعَكَ وَلْيَأْخُذُوٓا۟ أَسْلِحَتَهُمْ فَإِذَا سَجَدُوا۟ فَلْيَكُونُوا۟ مِن وَرَآئِكُمْ وَلْتَأْتِ طَآئِفَةٌ أُخْرَىٰ لَمْ يُصَلُّوا۟ فَلْيُصَلُّوا۟ مَعَكَ وَلْيَأْخُذُوا۟ حِذْرَهُمْ وَأَسْلِحَتَهُمْ ۗ وَدَّ ٱلَّذِينَ كَفَرُوا۟ لَوْ تَغْفُلُونَ عَنْ أَسْلِحَتِكُمْ وَأَمْتِعَتِكُمْ فَيَمِيلُونَ عَلَيْكُم مَّيْلَةًۭ وَٟحِدَةًۭ ۚ وَلَا جُنَاحَ عَلَيْكُمْ إِن كَانَ بِكُمْ أَذًۭى مِّن مَّطَرٍ أَوْ كُنتُم مَّرْضَىٰٓ أَن تَضَعُوٓا۟ أَسْلِحَتَكُمْ ۖ وَخُذُوا۟ حِذْرَكُمْ ۗ إِنَّ ٱللَّهَ أَعَدَّ لِلْكَـٰفِرِينَ عَذَابًۭا مُّهِينًۭا ﴿١٠٢﴾\\
\textamh{103.\  } & فَإِذَا قَضَيْتُمُ ٱلصَّلَوٰةَ فَٱذْكُرُوا۟ ٱللَّهَ قِيَـٰمًۭا وَقُعُودًۭا وَعَلَىٰ جُنُوبِكُمْ ۚ فَإِذَا ٱطْمَأْنَنتُمْ فَأَقِيمُوا۟ ٱلصَّلَوٰةَ ۚ إِنَّ ٱلصَّلَوٰةَ كَانَتْ عَلَى ٱلْمُؤْمِنِينَ كِتَـٰبًۭا مَّوْقُوتًۭا ﴿١٠٣﴾\\
\textamh{104.\  } & وَلَا تَهِنُوا۟ فِى ٱبْتِغَآءِ ٱلْقَوْمِ ۖ إِن تَكُونُوا۟ تَأْلَمُونَ فَإِنَّهُمْ يَأْلَمُونَ كَمَا تَأْلَمُونَ ۖ وَتَرْجُونَ مِنَ ٱللَّهِ مَا لَا يَرْجُونَ ۗ وَكَانَ ٱللَّهُ عَلِيمًا حَكِيمًا ﴿١٠٤﴾\\
\textamh{105.\  } & إِنَّآ أَنزَلْنَآ إِلَيْكَ ٱلْكِتَـٰبَ بِٱلْحَقِّ لِتَحْكُمَ بَيْنَ ٱلنَّاسِ بِمَآ أَرَىٰكَ ٱللَّهُ ۚ وَلَا تَكُن لِّلْخَآئِنِينَ خَصِيمًۭا ﴿١٠٥﴾\\
\textamh{106.\  } & وَٱسْتَغْفِرِ ٱللَّهَ ۖ إِنَّ ٱللَّهَ كَانَ غَفُورًۭا رَّحِيمًۭا ﴿١٠٦﴾\\
\textamh{107.\  } & وَلَا تُجَٰدِلْ عَنِ ٱلَّذِينَ يَخْتَانُونَ أَنفُسَهُمْ ۚ إِنَّ ٱللَّهَ لَا يُحِبُّ مَن كَانَ خَوَّانًا أَثِيمًۭا ﴿١٠٧﴾\\
\textamh{108.\  } & يَسْتَخْفُونَ مِنَ ٱلنَّاسِ وَلَا يَسْتَخْفُونَ مِنَ ٱللَّهِ وَهُوَ مَعَهُمْ إِذْ يُبَيِّتُونَ مَا لَا يَرْضَىٰ مِنَ ٱلْقَوْلِ ۚ وَكَانَ ٱللَّهُ بِمَا يَعْمَلُونَ مُحِيطًا ﴿١٠٨﴾\\
\textamh{109.\  } & هَـٰٓأَنتُمْ هَـٰٓؤُلَآءِ جَٰدَلْتُمْ عَنْهُمْ فِى ٱلْحَيَوٰةِ ٱلدُّنْيَا فَمَن يُجَٰدِلُ ٱللَّهَ عَنْهُمْ يَوْمَ ٱلْقِيَـٰمَةِ أَم مَّن يَكُونُ عَلَيْهِمْ وَكِيلًۭا ﴿١٠٩﴾\\
\textamh{110.\  } & وَمَن يَعْمَلْ سُوٓءًا أَوْ يَظْلِمْ نَفْسَهُۥ ثُمَّ يَسْتَغْفِرِ ٱللَّهَ يَجِدِ ٱللَّهَ غَفُورًۭا رَّحِيمًۭا ﴿١١٠﴾\\
\textamh{111.\  } & وَمَن يَكْسِبْ إِثْمًۭا فَإِنَّمَا يَكْسِبُهُۥ عَلَىٰ نَفْسِهِۦ ۚ وَكَانَ ٱللَّهُ عَلِيمًا حَكِيمًۭا ﴿١١١﴾\\
\textamh{112.\  } & وَمَن يَكْسِبْ خَطِيٓـَٔةً أَوْ إِثْمًۭا ثُمَّ يَرْمِ بِهِۦ بَرِيٓـًۭٔا فَقَدِ ٱحْتَمَلَ بُهْتَـٰنًۭا وَإِثْمًۭا مُّبِينًۭا ﴿١١٢﴾\\
\textamh{113.\  } & وَلَوْلَا فَضْلُ ٱللَّهِ عَلَيْكَ وَرَحْمَتُهُۥ لَهَمَّت طَّآئِفَةٌۭ مِّنْهُمْ أَن يُضِلُّوكَ وَمَا يُضِلُّونَ إِلَّآ أَنفُسَهُمْ ۖ وَمَا يَضُرُّونَكَ مِن شَىْءٍۢ ۚ وَأَنزَلَ ٱللَّهُ عَلَيْكَ ٱلْكِتَـٰبَ وَٱلْحِكْمَةَ وَعَلَّمَكَ مَا لَمْ تَكُن تَعْلَمُ ۚ وَكَانَ فَضْلُ ٱللَّهِ عَلَيْكَ عَظِيمًۭا ﴿١١٣﴾\\
\textamh{114.\  } & ۞ لَّا خَيْرَ فِى كَثِيرٍۢ مِّن نَّجْوَىٰهُمْ إِلَّا مَنْ أَمَرَ بِصَدَقَةٍ أَوْ مَعْرُوفٍ أَوْ إِصْلَـٰحٍۭ بَيْنَ ٱلنَّاسِ ۚ وَمَن يَفْعَلْ ذَٟلِكَ ٱبْتِغَآءَ مَرْضَاتِ ٱللَّهِ فَسَوْفَ نُؤْتِيهِ أَجْرًا عَظِيمًۭا ﴿١١٤﴾\\
\textamh{115.\  } & وَمَن يُشَاقِقِ ٱلرَّسُولَ مِنۢ بَعْدِ مَا تَبَيَّنَ لَهُ ٱلْهُدَىٰ وَيَتَّبِعْ غَيْرَ سَبِيلِ ٱلْمُؤْمِنِينَ نُوَلِّهِۦ مَا تَوَلَّىٰ وَنُصْلِهِۦ جَهَنَّمَ ۖ وَسَآءَتْ مَصِيرًا ﴿١١٥﴾\\
\textamh{116.\  } & إِنَّ ٱللَّهَ لَا يَغْفِرُ أَن يُشْرَكَ بِهِۦ وَيَغْفِرُ مَا دُونَ ذَٟلِكَ لِمَن يَشَآءُ ۚ وَمَن يُشْرِكْ بِٱللَّهِ فَقَدْ ضَلَّ ضَلَـٰلًۢا بَعِيدًا ﴿١١٦﴾\\
\textamh{117.\  } & إِن يَدْعُونَ مِن دُونِهِۦٓ إِلَّآ إِنَـٰثًۭا وَإِن يَدْعُونَ إِلَّا شَيْطَٰنًۭا مَّرِيدًۭا ﴿١١٧﴾\\
\textamh{118.\  } & لَّعَنَهُ ٱللَّهُ ۘ وَقَالَ لَأَتَّخِذَنَّ مِنْ عِبَادِكَ نَصِيبًۭا مَّفْرُوضًۭا ﴿١١٨﴾\\
\textamh{119.\  } & وَلَأُضِلَّنَّهُمْ وَلَأُمَنِّيَنَّهُمْ وَلَءَامُرَنَّهُمْ فَلَيُبَتِّكُنَّ ءَاذَانَ ٱلْأَنْعَـٰمِ وَلَءَامُرَنَّهُمْ فَلَيُغَيِّرُنَّ خَلْقَ ٱللَّهِ ۚ وَمَن يَتَّخِذِ ٱلشَّيْطَٰنَ وَلِيًّۭا مِّن دُونِ ٱللَّهِ فَقَدْ خَسِرَ خُسْرَانًۭا مُّبِينًۭا ﴿١١٩﴾\\
\textamh{120.\  } & يَعِدُهُمْ وَيُمَنِّيهِمْ ۖ وَمَا يَعِدُهُمُ ٱلشَّيْطَٰنُ إِلَّا غُرُورًا ﴿١٢٠﴾\\
\textamh{121.\  } & أُو۟لَـٰٓئِكَ مَأْوَىٰهُمْ جَهَنَّمُ وَلَا يَجِدُونَ عَنْهَا مَحِيصًۭا ﴿١٢١﴾\\
\textamh{122.\  } & وَٱلَّذِينَ ءَامَنُوا۟ وَعَمِلُوا۟ ٱلصَّـٰلِحَـٰتِ سَنُدْخِلُهُمْ جَنَّـٰتٍۢ تَجْرِى مِن تَحْتِهَا ٱلْأَنْهَـٰرُ خَـٰلِدِينَ فِيهَآ أَبَدًۭا ۖ وَعْدَ ٱللَّهِ حَقًّۭا ۚ وَمَنْ أَصْدَقُ مِنَ ٱللَّهِ قِيلًۭا ﴿١٢٢﴾\\
\textamh{123.\  } & لَّيْسَ بِأَمَانِيِّكُمْ وَلَآ أَمَانِىِّ أَهْلِ ٱلْكِتَـٰبِ ۗ مَن يَعْمَلْ سُوٓءًۭا يُجْزَ بِهِۦ وَلَا يَجِدْ لَهُۥ مِن دُونِ ٱللَّهِ وَلِيًّۭا وَلَا نَصِيرًۭا ﴿١٢٣﴾\\
\textamh{124.\  } & وَمَن يَعْمَلْ مِنَ ٱلصَّـٰلِحَـٰتِ مِن ذَكَرٍ أَوْ أُنثَىٰ وَهُوَ مُؤْمِنٌۭ فَأُو۟لَـٰٓئِكَ يَدْخُلُونَ ٱلْجَنَّةَ وَلَا يُظْلَمُونَ نَقِيرًۭا ﴿١٢٤﴾\\
\textamh{125.\  } & وَمَنْ أَحْسَنُ دِينًۭا مِّمَّنْ أَسْلَمَ وَجْهَهُۥ لِلَّهِ وَهُوَ مُحْسِنٌۭ وَٱتَّبَعَ مِلَّةَ إِبْرَٰهِيمَ حَنِيفًۭا ۗ وَٱتَّخَذَ ٱللَّهُ إِبْرَٰهِيمَ خَلِيلًۭا ﴿١٢٥﴾\\
\textamh{126.\  } & وَلِلَّهِ مَا فِى ٱلسَّمَـٰوَٟتِ وَمَا فِى ٱلْأَرْضِ ۚ وَكَانَ ٱللَّهُ بِكُلِّ شَىْءٍۢ مُّحِيطًۭا ﴿١٢٦﴾\\
\textamh{127.\  } & وَيَسْتَفْتُونَكَ فِى ٱلنِّسَآءِ ۖ قُلِ ٱللَّهُ يُفْتِيكُمْ فِيهِنَّ وَمَا يُتْلَىٰ عَلَيْكُمْ فِى ٱلْكِتَـٰبِ فِى يَتَـٰمَى ٱلنِّسَآءِ ٱلَّٰتِى لَا تُؤْتُونَهُنَّ مَا كُتِبَ لَهُنَّ وَتَرْغَبُونَ أَن تَنكِحُوهُنَّ وَٱلْمُسْتَضْعَفِينَ مِنَ ٱلْوِلْدَٟنِ وَأَن تَقُومُوا۟ لِلْيَتَـٰمَىٰ بِٱلْقِسْطِ ۚ وَمَا تَفْعَلُوا۟ مِنْ خَيْرٍۢ فَإِنَّ ٱللَّهَ كَانَ بِهِۦ عَلِيمًۭا ﴿١٢٧﴾\\
\textamh{128.\  } & وَإِنِ ٱمْرَأَةٌ خَافَتْ مِنۢ بَعْلِهَا نُشُوزًا أَوْ إِعْرَاضًۭا فَلَا جُنَاحَ عَلَيْهِمَآ أَن يُصْلِحَا بَيْنَهُمَا صُلْحًۭا ۚ وَٱلصُّلْحُ خَيْرٌۭ ۗ وَأُحْضِرَتِ ٱلْأَنفُسُ ٱلشُّحَّ ۚ وَإِن تُحْسِنُوا۟ وَتَتَّقُوا۟ فَإِنَّ ٱللَّهَ كَانَ بِمَا تَعْمَلُونَ خَبِيرًۭا ﴿١٢٨﴾\\
\textamh{129.\  } & وَلَن تَسْتَطِيعُوٓا۟ أَن تَعْدِلُوا۟ بَيْنَ ٱلنِّسَآءِ وَلَوْ حَرَصْتُمْ ۖ فَلَا تَمِيلُوا۟ كُلَّ ٱلْمَيْلِ فَتَذَرُوهَا كَٱلْمُعَلَّقَةِ ۚ وَإِن تُصْلِحُوا۟ وَتَتَّقُوا۟ فَإِنَّ ٱللَّهَ كَانَ غَفُورًۭا رَّحِيمًۭا ﴿١٢٩﴾\\
\textamh{130.\  } & وَإِن يَتَفَرَّقَا يُغْنِ ٱللَّهُ كُلًّۭا مِّن سَعَتِهِۦ ۚ وَكَانَ ٱللَّهُ وَٟسِعًا حَكِيمًۭا ﴿١٣٠﴾\\
\textamh{131.\  } & وَلِلَّهِ مَا فِى ٱلسَّمَـٰوَٟتِ وَمَا فِى ٱلْأَرْضِ ۗ وَلَقَدْ وَصَّيْنَا ٱلَّذِينَ أُوتُوا۟ ٱلْكِتَـٰبَ مِن قَبْلِكُمْ وَإِيَّاكُمْ أَنِ ٱتَّقُوا۟ ٱللَّهَ ۚ وَإِن تَكْفُرُوا۟ فَإِنَّ لِلَّهِ مَا فِى ٱلسَّمَـٰوَٟتِ وَمَا فِى ٱلْأَرْضِ ۚ وَكَانَ ٱللَّهُ غَنِيًّا حَمِيدًۭا ﴿١٣١﴾\\
\textamh{132.\  } & وَلِلَّهِ مَا فِى ٱلسَّمَـٰوَٟتِ وَمَا فِى ٱلْأَرْضِ ۚ وَكَفَىٰ بِٱللَّهِ وَكِيلًا ﴿١٣٢﴾\\
\textamh{133.\  } & إِن يَشَأْ يُذْهِبْكُمْ أَيُّهَا ٱلنَّاسُ وَيَأْتِ بِـَٔاخَرِينَ ۚ وَكَانَ ٱللَّهُ عَلَىٰ ذَٟلِكَ قَدِيرًۭا ﴿١٣٣﴾\\
\textamh{134.\  } & مَّن كَانَ يُرِيدُ ثَوَابَ ٱلدُّنْيَا فَعِندَ ٱللَّهِ ثَوَابُ ٱلدُّنْيَا وَٱلْءَاخِرَةِ ۚ وَكَانَ ٱللَّهُ سَمِيعًۢا بَصِيرًۭا ﴿١٣٤﴾\\
\textamh{135.\  } & ۞ يَـٰٓأَيُّهَا ٱلَّذِينَ ءَامَنُوا۟ كُونُوا۟ قَوَّٰمِينَ بِٱلْقِسْطِ شُهَدَآءَ لِلَّهِ وَلَوْ عَلَىٰٓ أَنفُسِكُمْ أَوِ ٱلْوَٟلِدَيْنِ وَٱلْأَقْرَبِينَ ۚ إِن يَكُنْ غَنِيًّا أَوْ فَقِيرًۭا فَٱللَّهُ أَوْلَىٰ بِهِمَا ۖ فَلَا تَتَّبِعُوا۟ ٱلْهَوَىٰٓ أَن تَعْدِلُوا۟ ۚ وَإِن تَلْوُۥٓا۟ أَوْ تُعْرِضُوا۟ فَإِنَّ ٱللَّهَ كَانَ بِمَا تَعْمَلُونَ خَبِيرًۭا ﴿١٣٥﴾\\
\textamh{136.\  } & يَـٰٓأَيُّهَا ٱلَّذِينَ ءَامَنُوٓا۟ ءَامِنُوا۟ بِٱللَّهِ وَرَسُولِهِۦ وَٱلْكِتَـٰبِ ٱلَّذِى نَزَّلَ عَلَىٰ رَسُولِهِۦ وَٱلْكِتَـٰبِ ٱلَّذِىٓ أَنزَلَ مِن قَبْلُ ۚ وَمَن يَكْفُرْ بِٱللَّهِ وَمَلَـٰٓئِكَتِهِۦ وَكُتُبِهِۦ وَرُسُلِهِۦ وَٱلْيَوْمِ ٱلْءَاخِرِ فَقَدْ ضَلَّ ضَلَـٰلًۢا بَعِيدًا ﴿١٣٦﴾\\
\textamh{137.\  } & إِنَّ ٱلَّذِينَ ءَامَنُوا۟ ثُمَّ كَفَرُوا۟ ثُمَّ ءَامَنُوا۟ ثُمَّ كَفَرُوا۟ ثُمَّ ٱزْدَادُوا۟ كُفْرًۭا لَّمْ يَكُنِ ٱللَّهُ لِيَغْفِرَ لَهُمْ وَلَا لِيَهْدِيَهُمْ سَبِيلًۢا ﴿١٣٧﴾\\
\textamh{138.\  } & بَشِّرِ ٱلْمُنَـٰفِقِينَ بِأَنَّ لَهُمْ عَذَابًا أَلِيمًا ﴿١٣٨﴾\\
\textamh{139.\  } & ٱلَّذِينَ يَتَّخِذُونَ ٱلْكَـٰفِرِينَ أَوْلِيَآءَ مِن دُونِ ٱلْمُؤْمِنِينَ ۚ أَيَبْتَغُونَ عِندَهُمُ ٱلْعِزَّةَ فَإِنَّ ٱلْعِزَّةَ لِلَّهِ جَمِيعًۭا ﴿١٣٩﴾\\
\textamh{140.\  } & وَقَدْ نَزَّلَ عَلَيْكُمْ فِى ٱلْكِتَـٰبِ أَنْ إِذَا سَمِعْتُمْ ءَايَـٰتِ ٱللَّهِ يُكْفَرُ بِهَا وَيُسْتَهْزَأُ بِهَا فَلَا تَقْعُدُوا۟ مَعَهُمْ حَتَّىٰ يَخُوضُوا۟ فِى حَدِيثٍ غَيْرِهِۦٓ ۚ إِنَّكُمْ إِذًۭا مِّثْلُهُمْ ۗ إِنَّ ٱللَّهَ جَامِعُ ٱلْمُنَـٰفِقِينَ وَٱلْكَـٰفِرِينَ فِى جَهَنَّمَ جَمِيعًا ﴿١٤٠﴾\\
\textamh{141.\  } & ٱلَّذِينَ يَتَرَبَّصُونَ بِكُمْ فَإِن كَانَ لَكُمْ فَتْحٌۭ مِّنَ ٱللَّهِ قَالُوٓا۟ أَلَمْ نَكُن مَّعَكُمْ وَإِن كَانَ لِلْكَـٰفِرِينَ نَصِيبٌۭ قَالُوٓا۟ أَلَمْ نَسْتَحْوِذْ عَلَيْكُمْ وَنَمْنَعْكُم مِّنَ ٱلْمُؤْمِنِينَ ۚ فَٱللَّهُ يَحْكُمُ بَيْنَكُمْ يَوْمَ ٱلْقِيَـٰمَةِ ۗ وَلَن يَجْعَلَ ٱللَّهُ لِلْكَـٰفِرِينَ عَلَى ٱلْمُؤْمِنِينَ سَبِيلًا ﴿١٤١﴾\\
\textamh{142.\  } & إِنَّ ٱلْمُنَـٰفِقِينَ يُخَـٰدِعُونَ ٱللَّهَ وَهُوَ خَـٰدِعُهُمْ وَإِذَا قَامُوٓا۟ إِلَى ٱلصَّلَوٰةِ قَامُوا۟ كُسَالَىٰ يُرَآءُونَ ٱلنَّاسَ وَلَا يَذْكُرُونَ ٱللَّهَ إِلَّا قَلِيلًۭا ﴿١٤٢﴾\\
\textamh{143.\  } & مُّذَبْذَبِينَ بَيْنَ ذَٟلِكَ لَآ إِلَىٰ هَـٰٓؤُلَآءِ وَلَآ إِلَىٰ هَـٰٓؤُلَآءِ ۚ وَمَن يُضْلِلِ ٱللَّهُ فَلَن تَجِدَ لَهُۥ سَبِيلًۭا ﴿١٤٣﴾\\
\textamh{144.\  } & يَـٰٓأَيُّهَا ٱلَّذِينَ ءَامَنُوا۟ لَا تَتَّخِذُوا۟ ٱلْكَـٰفِرِينَ أَوْلِيَآءَ مِن دُونِ ٱلْمُؤْمِنِينَ ۚ أَتُرِيدُونَ أَن تَجْعَلُوا۟ لِلَّهِ عَلَيْكُمْ سُلْطَٰنًۭا مُّبِينًا ﴿١٤٤﴾\\
\textamh{145.\  } & إِنَّ ٱلْمُنَـٰفِقِينَ فِى ٱلدَّرْكِ ٱلْأَسْفَلِ مِنَ ٱلنَّارِ وَلَن تَجِدَ لَهُمْ نَصِيرًا ﴿١٤٥﴾\\
\textamh{146.\  } & إِلَّا ٱلَّذِينَ تَابُوا۟ وَأَصْلَحُوا۟ وَٱعْتَصَمُوا۟ بِٱللَّهِ وَأَخْلَصُوا۟ دِينَهُمْ لِلَّهِ فَأُو۟لَـٰٓئِكَ مَعَ ٱلْمُؤْمِنِينَ ۖ وَسَوْفَ يُؤْتِ ٱللَّهُ ٱلْمُؤْمِنِينَ أَجْرًا عَظِيمًۭا ﴿١٤٦﴾\\
\textamh{147.\  } & مَّا يَفْعَلُ ٱللَّهُ بِعَذَابِكُمْ إِن شَكَرْتُمْ وَءَامَنتُمْ ۚ وَكَانَ ٱللَّهُ شَاكِرًا عَلِيمًۭا ﴿١٤٧﴾\\
\textamh{148.\  } & ۞ لَّا يُحِبُّ ٱللَّهُ ٱلْجَهْرَ بِٱلسُّوٓءِ مِنَ ٱلْقَوْلِ إِلَّا مَن ظُلِمَ ۚ وَكَانَ ٱللَّهُ سَمِيعًا عَلِيمًا ﴿١٤٨﴾\\
\textamh{149.\  } & إِن تُبْدُوا۟ خَيْرًا أَوْ تُخْفُوهُ أَوْ تَعْفُوا۟ عَن سُوٓءٍۢ فَإِنَّ ٱللَّهَ كَانَ عَفُوًّۭا قَدِيرًا ﴿١٤٩﴾\\
\textamh{150.\  } & إِنَّ ٱلَّذِينَ يَكْفُرُونَ بِٱللَّهِ وَرُسُلِهِۦ وَيُرِيدُونَ أَن يُفَرِّقُوا۟ بَيْنَ ٱللَّهِ وَرُسُلِهِۦ وَيَقُولُونَ نُؤْمِنُ بِبَعْضٍۢ وَنَكْفُرُ بِبَعْضٍۢ وَيُرِيدُونَ أَن يَتَّخِذُوا۟ بَيْنَ ذَٟلِكَ سَبِيلًا ﴿١٥٠﴾\\
\textamh{151.\  } & أُو۟لَـٰٓئِكَ هُمُ ٱلْكَـٰفِرُونَ حَقًّۭا ۚ وَأَعْتَدْنَا لِلْكَـٰفِرِينَ عَذَابًۭا مُّهِينًۭا ﴿١٥١﴾\\
\textamh{152.\  } & وَٱلَّذِينَ ءَامَنُوا۟ بِٱللَّهِ وَرُسُلِهِۦ وَلَمْ يُفَرِّقُوا۟ بَيْنَ أَحَدٍۢ مِّنْهُمْ أُو۟لَـٰٓئِكَ سَوْفَ يُؤْتِيهِمْ أُجُورَهُمْ ۗ وَكَانَ ٱللَّهُ غَفُورًۭا رَّحِيمًۭا ﴿١٥٢﴾\\
\textamh{153.\  } & يَسْـَٔلُكَ أَهْلُ ٱلْكِتَـٰبِ أَن تُنَزِّلَ عَلَيْهِمْ كِتَـٰبًۭا مِّنَ ٱلسَّمَآءِ ۚ فَقَدْ سَأَلُوا۟ مُوسَىٰٓ أَكْبَرَ مِن ذَٟلِكَ فَقَالُوٓا۟ أَرِنَا ٱللَّهَ جَهْرَةًۭ فَأَخَذَتْهُمُ ٱلصَّـٰعِقَةُ بِظُلْمِهِمْ ۚ ثُمَّ ٱتَّخَذُوا۟ ٱلْعِجْلَ مِنۢ بَعْدِ مَا جَآءَتْهُمُ ٱلْبَيِّنَـٰتُ فَعَفَوْنَا عَن ذَٟلِكَ ۚ وَءَاتَيْنَا مُوسَىٰ سُلْطَٰنًۭا مُّبِينًۭا ﴿١٥٣﴾\\
\textamh{154.\  } & وَرَفَعْنَا فَوْقَهُمُ ٱلطُّورَ بِمِيثَـٰقِهِمْ وَقُلْنَا لَهُمُ ٱدْخُلُوا۟ ٱلْبَابَ سُجَّدًۭا وَقُلْنَا لَهُمْ لَا تَعْدُوا۟ فِى ٱلسَّبْتِ وَأَخَذْنَا مِنْهُم مِّيثَـٰقًا غَلِيظًۭا ﴿١٥٤﴾\\
\textamh{155.\  } & فَبِمَا نَقْضِهِم مِّيثَـٰقَهُمْ وَكُفْرِهِم بِـَٔايَـٰتِ ٱللَّهِ وَقَتْلِهِمُ ٱلْأَنۢبِيَآءَ بِغَيْرِ حَقٍّۢ وَقَوْلِهِمْ قُلُوبُنَا غُلْفٌۢ ۚ بَلْ طَبَعَ ٱللَّهُ عَلَيْهَا بِكُفْرِهِمْ فَلَا يُؤْمِنُونَ إِلَّا قَلِيلًۭا ﴿١٥٥﴾\\
\textamh{156.\  } & وَبِكُفْرِهِمْ وَقَوْلِهِمْ عَلَىٰ مَرْيَمَ بُهْتَـٰنًا عَظِيمًۭا ﴿١٥٦﴾\\
\textamh{157.\  } & وَقَوْلِهِمْ إِنَّا قَتَلْنَا ٱلْمَسِيحَ عِيسَى ٱبْنَ مَرْيَمَ رَسُولَ ٱللَّهِ وَمَا قَتَلُوهُ وَمَا صَلَبُوهُ وَلَـٰكِن شُبِّهَ لَهُمْ ۚ وَإِنَّ ٱلَّذِينَ ٱخْتَلَفُوا۟ فِيهِ لَفِى شَكٍّۢ مِّنْهُ ۚ مَا لَهُم بِهِۦ مِنْ عِلْمٍ إِلَّا ٱتِّبَاعَ ٱلظَّنِّ ۚ وَمَا قَتَلُوهُ يَقِينًۢا ﴿١٥٧﴾\\
\textamh{158.\  } & بَل رَّفَعَهُ ٱللَّهُ إِلَيْهِ ۚ وَكَانَ ٱللَّهُ عَزِيزًا حَكِيمًۭا ﴿١٥٨﴾\\
\textamh{159.\  } & وَإِن مِّنْ أَهْلِ ٱلْكِتَـٰبِ إِلَّا لَيُؤْمِنَنَّ بِهِۦ قَبْلَ مَوْتِهِۦ ۖ وَيَوْمَ ٱلْقِيَـٰمَةِ يَكُونُ عَلَيْهِمْ شَهِيدًۭا ﴿١٥٩﴾\\
\textamh{160.\  } & فَبِظُلْمٍۢ مِّنَ ٱلَّذِينَ هَادُوا۟ حَرَّمْنَا عَلَيْهِمْ طَيِّبَٰتٍ أُحِلَّتْ لَهُمْ وَبِصَدِّهِمْ عَن سَبِيلِ ٱللَّهِ كَثِيرًۭا ﴿١٦٠﴾\\
\textamh{161.\  } & وَأَخْذِهِمُ ٱلرِّبَوٰا۟ وَقَدْ نُهُوا۟ عَنْهُ وَأَكْلِهِمْ أَمْوَٟلَ ٱلنَّاسِ بِٱلْبَٰطِلِ ۚ وَأَعْتَدْنَا لِلْكَـٰفِرِينَ مِنْهُمْ عَذَابًا أَلِيمًۭا ﴿١٦١﴾\\
\textamh{162.\  } & لَّٰكِنِ ٱلرَّٟسِخُونَ فِى ٱلْعِلْمِ مِنْهُمْ وَٱلْمُؤْمِنُونَ يُؤْمِنُونَ بِمَآ أُنزِلَ إِلَيْكَ وَمَآ أُنزِلَ مِن قَبْلِكَ ۚ وَٱلْمُقِيمِينَ ٱلصَّلَوٰةَ ۚ وَٱلْمُؤْتُونَ ٱلزَّكَوٰةَ وَٱلْمُؤْمِنُونَ بِٱللَّهِ وَٱلْيَوْمِ ٱلْءَاخِرِ أُو۟لَـٰٓئِكَ سَنُؤْتِيهِمْ أَجْرًا عَظِيمًا ﴿١٦٢﴾\\
\textamh{163.\  } & ۞ إِنَّآ أَوْحَيْنَآ إِلَيْكَ كَمَآ أَوْحَيْنَآ إِلَىٰ نُوحٍۢ وَٱلنَّبِيِّۦنَ مِنۢ بَعْدِهِۦ ۚ وَأَوْحَيْنَآ إِلَىٰٓ إِبْرَٰهِيمَ وَإِسْمَـٰعِيلَ وَإِسْحَـٰقَ وَيَعْقُوبَ وَٱلْأَسْبَاطِ وَعِيسَىٰ وَأَيُّوبَ وَيُونُسَ وَهَـٰرُونَ وَسُلَيْمَـٰنَ ۚ وَءَاتَيْنَا دَاوُۥدَ زَبُورًۭا ﴿١٦٣﴾\\
\textamh{164.\  } & وَرُسُلًۭا قَدْ قَصَصْنَـٰهُمْ عَلَيْكَ مِن قَبْلُ وَرُسُلًۭا لَّمْ نَقْصُصْهُمْ عَلَيْكَ ۚ وَكَلَّمَ ٱللَّهُ مُوسَىٰ تَكْلِيمًۭا ﴿١٦٤﴾\\
\textamh{165.\  } & رُّسُلًۭا مُّبَشِّرِينَ وَمُنذِرِينَ لِئَلَّا يَكُونَ لِلنَّاسِ عَلَى ٱللَّهِ حُجَّةٌۢ بَعْدَ ٱلرُّسُلِ ۚ وَكَانَ ٱللَّهُ عَزِيزًا حَكِيمًۭا ﴿١٦٥﴾\\
\textamh{166.\  } & لَّٰكِنِ ٱللَّهُ يَشْهَدُ بِمَآ أَنزَلَ إِلَيْكَ ۖ أَنزَلَهُۥ بِعِلْمِهِۦ ۖ وَٱلْمَلَـٰٓئِكَةُ يَشْهَدُونَ ۚ وَكَفَىٰ بِٱللَّهِ شَهِيدًا ﴿١٦٦﴾\\
\textamh{167.\  } & إِنَّ ٱلَّذِينَ كَفَرُوا۟ وَصَدُّوا۟ عَن سَبِيلِ ٱللَّهِ قَدْ ضَلُّوا۟ ضَلَـٰلًۢا بَعِيدًا ﴿١٦٧﴾\\
\textamh{168.\  } & إِنَّ ٱلَّذِينَ كَفَرُوا۟ وَظَلَمُوا۟ لَمْ يَكُنِ ٱللَّهُ لِيَغْفِرَ لَهُمْ وَلَا لِيَهْدِيَهُمْ طَرِيقًا ﴿١٦٨﴾\\
\textamh{169.\  } & إِلَّا طَرِيقَ جَهَنَّمَ خَـٰلِدِينَ فِيهَآ أَبَدًۭا ۚ وَكَانَ ذَٟلِكَ عَلَى ٱللَّهِ يَسِيرًۭا ﴿١٦٩﴾\\
\textamh{170.\  } & يَـٰٓأَيُّهَا ٱلنَّاسُ قَدْ جَآءَكُمُ ٱلرَّسُولُ بِٱلْحَقِّ مِن رَّبِّكُمْ فَـَٔامِنُوا۟ خَيْرًۭا لَّكُمْ ۚ وَإِن تَكْفُرُوا۟ فَإِنَّ لِلَّهِ مَا فِى ٱلسَّمَـٰوَٟتِ وَٱلْأَرْضِ ۚ وَكَانَ ٱللَّهُ عَلِيمًا حَكِيمًۭا ﴿١٧٠﴾\\
\textamh{171.\  } & يَـٰٓأَهْلَ ٱلْكِتَـٰبِ لَا تَغْلُوا۟ فِى دِينِكُمْ وَلَا تَقُولُوا۟ عَلَى ٱللَّهِ إِلَّا ٱلْحَقَّ ۚ إِنَّمَا ٱلْمَسِيحُ عِيسَى ٱبْنُ مَرْيَمَ رَسُولُ ٱللَّهِ وَكَلِمَتُهُۥٓ أَلْقَىٰهَآ إِلَىٰ مَرْيَمَ وَرُوحٌۭ مِّنْهُ ۖ فَـَٔامِنُوا۟ بِٱللَّهِ وَرُسُلِهِۦ ۖ وَلَا تَقُولُوا۟ ثَلَـٰثَةٌ ۚ ٱنتَهُوا۟ خَيْرًۭا لَّكُمْ ۚ إِنَّمَا ٱللَّهُ إِلَـٰهٌۭ وَٟحِدٌۭ ۖ سُبْحَـٰنَهُۥٓ أَن يَكُونَ لَهُۥ وَلَدٌۭ ۘ لَّهُۥ مَا فِى ٱلسَّمَـٰوَٟتِ وَمَا فِى ٱلْأَرْضِ ۗ وَكَفَىٰ بِٱللَّهِ وَكِيلًۭا ﴿١٧١﴾\\
\textamh{172.\  } & لَّن يَسْتَنكِفَ ٱلْمَسِيحُ أَن يَكُونَ عَبْدًۭا لِّلَّهِ وَلَا ٱلْمَلَـٰٓئِكَةُ ٱلْمُقَرَّبُونَ ۚ وَمَن يَسْتَنكِفْ عَنْ عِبَادَتِهِۦ وَيَسْتَكْبِرْ فَسَيَحْشُرُهُمْ إِلَيْهِ جَمِيعًۭا ﴿١٧٢﴾\\
\textamh{173.\  } & فَأَمَّا ٱلَّذِينَ ءَامَنُوا۟ وَعَمِلُوا۟ ٱلصَّـٰلِحَـٰتِ فَيُوَفِّيهِمْ أُجُورَهُمْ وَيَزِيدُهُم مِّن فَضْلِهِۦ ۖ وَأَمَّا ٱلَّذِينَ ٱسْتَنكَفُوا۟ وَٱسْتَكْبَرُوا۟ فَيُعَذِّبُهُمْ عَذَابًا أَلِيمًۭا وَلَا يَجِدُونَ لَهُم مِّن دُونِ ٱللَّهِ وَلِيًّۭا وَلَا نَصِيرًۭا ﴿١٧٣﴾\\
\textamh{174.\  } & يَـٰٓأَيُّهَا ٱلنَّاسُ قَدْ جَآءَكُم بُرْهَـٰنٌۭ مِّن رَّبِّكُمْ وَأَنزَلْنَآ إِلَيْكُمْ نُورًۭا مُّبِينًۭا ﴿١٧٤﴾\\
\textamh{175.\  } & فَأَمَّا ٱلَّذِينَ ءَامَنُوا۟ بِٱللَّهِ وَٱعْتَصَمُوا۟ بِهِۦ فَسَيُدْخِلُهُمْ فِى رَحْمَةٍۢ مِّنْهُ وَفَضْلٍۢ وَيَهْدِيهِمْ إِلَيْهِ صِرَٰطًۭا مُّسْتَقِيمًۭا ﴿١٧٥﴾\\
\textamh{176.\  } & يَسْتَفْتُونَكَ قُلِ ٱللَّهُ يُفْتِيكُمْ فِى ٱلْكَلَـٰلَةِ ۚ إِنِ ٱمْرُؤٌا۟ هَلَكَ لَيْسَ لَهُۥ وَلَدٌۭ وَلَهُۥٓ أُخْتٌۭ فَلَهَا نِصْفُ مَا تَرَكَ ۚ وَهُوَ يَرِثُهَآ إِن لَّمْ يَكُن لَّهَا وَلَدٌۭ ۚ فَإِن كَانَتَا ٱثْنَتَيْنِ فَلَهُمَا ٱلثُّلُثَانِ مِمَّا تَرَكَ ۚ وَإِن كَانُوٓا۟ إِخْوَةًۭ رِّجَالًۭا وَنِسَآءًۭ فَلِلذَّكَرِ مِثْلُ حَظِّ ٱلْأُنثَيَيْنِ ۗ يُبَيِّنُ ٱللَّهُ لَكُمْ أَن تَضِلُّوا۟ ۗ وَٱللَّهُ بِكُلِّ شَىْءٍ عَلِيمٌۢ ﴿١٧٦﴾\\
\end{longtable}
\clearpage

%% License: BSD style (Berkley) (i.e. Put the Copyright owner's name always)
%% Writer and Copyright (to): Bewketu(Bilal) Tadilo (2016-17)
\centering\section{\LR{\textamharic{ሱራቱ አልመአዳ -}  \RL{سوره  المائدة}}}
\begin{longtable}{%
  @{}
    p{.5\textwidth}
  @{~~~~~~~~~~~~~}
    p{.5\textwidth}
    @{}
}
\nopagebreak
\textamh{ቢስሚላሂ አራህመኒ ራሂይም } &  بِسْمِ ٱللَّهِ ٱلرَّحْمَـٰنِ ٱلرَّحِيمِ\\
\textamh{1.\  } &  يَـٰٓأَيُّهَا ٱلَّذِينَ ءَامَنُوٓا۟ أَوْفُوا۟ بِٱلْعُقُودِ ۚ أُحِلَّتْ لَكُم بَهِيمَةُ ٱلْأَنْعَـٰمِ إِلَّا مَا يُتْلَىٰ عَلَيْكُمْ غَيْرَ مُحِلِّى ٱلصَّيْدِ وَأَنتُمْ حُرُمٌ ۗ إِنَّ ٱللَّهَ يَحْكُمُ مَا يُرِيدُ ﴿١﴾\\
\textamh{2.\  } & يَـٰٓأَيُّهَا ٱلَّذِينَ ءَامَنُوا۟ لَا تُحِلُّوا۟ شَعَـٰٓئِرَ ٱللَّهِ وَلَا ٱلشَّهْرَ ٱلْحَرَامَ وَلَا ٱلْهَدْىَ وَلَا ٱلْقَلَـٰٓئِدَ وَلَآ ءَآمِّينَ ٱلْبَيْتَ ٱلْحَرَامَ يَبْتَغُونَ فَضْلًۭا مِّن رَّبِّهِمْ وَرِضْوَٟنًۭا ۚ وَإِذَا حَلَلْتُمْ فَٱصْطَادُوا۟ ۚ وَلَا يَجْرِمَنَّكُمْ شَنَـَٔانُ قَوْمٍ أَن صَدُّوكُمْ عَنِ ٱلْمَسْجِدِ ٱلْحَرَامِ أَن تَعْتَدُوا۟ ۘ وَتَعَاوَنُوا۟ عَلَى ٱلْبِرِّ وَٱلتَّقْوَىٰ ۖ وَلَا تَعَاوَنُوا۟ عَلَى ٱلْإِثْمِ وَٱلْعُدْوَٟنِ ۚ وَٱتَّقُوا۟ ٱللَّهَ ۖ إِنَّ ٱللَّهَ شَدِيدُ ٱلْعِقَابِ ﴿٢﴾\\
\textamh{3.\  } & حُرِّمَتْ عَلَيْكُمُ ٱلْمَيْتَةُ وَٱلدَّمُ وَلَحْمُ ٱلْخِنزِيرِ وَمَآ أُهِلَّ لِغَيْرِ ٱللَّهِ بِهِۦ وَٱلْمُنْخَنِقَةُ وَٱلْمَوْقُوذَةُ وَٱلْمُتَرَدِّيَةُ وَٱلنَّطِيحَةُ وَمَآ أَكَلَ ٱلسَّبُعُ إِلَّا مَا ذَكَّيْتُمْ وَمَا ذُبِحَ عَلَى ٱلنُّصُبِ وَأَن تَسْتَقْسِمُوا۟ بِٱلْأَزْلَـٰمِ ۚ ذَٟلِكُمْ فِسْقٌ ۗ ٱلْيَوْمَ يَئِسَ ٱلَّذِينَ كَفَرُوا۟ مِن دِينِكُمْ فَلَا تَخْشَوْهُمْ وَٱخْشَوْنِ ۚ ٱلْيَوْمَ أَكْمَلْتُ لَكُمْ دِينَكُمْ وَأَتْمَمْتُ عَلَيْكُمْ نِعْمَتِى وَرَضِيتُ لَكُمُ ٱلْإِسْلَـٰمَ دِينًۭا ۚ فَمَنِ ٱضْطُرَّ فِى مَخْمَصَةٍ غَيْرَ مُتَجَانِفٍۢ لِّإِثْمٍۢ ۙ فَإِنَّ ٱللَّهَ غَفُورٌۭ رَّحِيمٌۭ ﴿٣﴾\\
\textamh{4.\  } & يَسْـَٔلُونَكَ مَاذَآ أُحِلَّ لَهُمْ ۖ قُلْ أُحِلَّ لَكُمُ ٱلطَّيِّبَٰتُ ۙ وَمَا عَلَّمْتُم مِّنَ ٱلْجَوَارِحِ مُكَلِّبِينَ تُعَلِّمُونَهُنَّ مِمَّا عَلَّمَكُمُ ٱللَّهُ ۖ فَكُلُوا۟ مِمَّآ أَمْسَكْنَ عَلَيْكُمْ وَٱذْكُرُوا۟ ٱسْمَ ٱللَّهِ عَلَيْهِ ۖ وَٱتَّقُوا۟ ٱللَّهَ ۚ إِنَّ ٱللَّهَ سَرِيعُ ٱلْحِسَابِ ﴿٤﴾\\
\textamh{5.\  } & ٱلْيَوْمَ أُحِلَّ لَكُمُ ٱلطَّيِّبَٰتُ ۖ وَطَعَامُ ٱلَّذِينَ أُوتُوا۟ ٱلْكِتَـٰبَ حِلٌّۭ لَّكُمْ وَطَعَامُكُمْ حِلٌّۭ لَّهُمْ ۖ وَٱلْمُحْصَنَـٰتُ مِنَ ٱلْمُؤْمِنَـٰتِ وَٱلْمُحْصَنَـٰتُ مِنَ ٱلَّذِينَ أُوتُوا۟ ٱلْكِتَـٰبَ مِن قَبْلِكُمْ إِذَآ ءَاتَيْتُمُوهُنَّ أُجُورَهُنَّ مُحْصِنِينَ غَيْرَ مُسَـٰفِحِينَ وَلَا مُتَّخِذِىٓ أَخْدَانٍۢ ۗ وَمَن يَكْفُرْ بِٱلْإِيمَـٰنِ فَقَدْ حَبِطَ عَمَلُهُۥ وَهُوَ فِى ٱلْءَاخِرَةِ مِنَ ٱلْخَـٰسِرِينَ ﴿٥﴾\\
\textamh{6.\  } & يَـٰٓأَيُّهَا ٱلَّذِينَ ءَامَنُوٓا۟ إِذَا قُمْتُمْ إِلَى ٱلصَّلَوٰةِ فَٱغْسِلُوا۟ وُجُوهَكُمْ وَأَيْدِيَكُمْ إِلَى ٱلْمَرَافِقِ وَٱمْسَحُوا۟ بِرُءُوسِكُمْ وَأَرْجُلَكُمْ إِلَى ٱلْكَعْبَيْنِ ۚ وَإِن كُنتُمْ جُنُبًۭا فَٱطَّهَّرُوا۟ ۚ وَإِن كُنتُم مَّرْضَىٰٓ أَوْ عَلَىٰ سَفَرٍ أَوْ جَآءَ أَحَدٌۭ مِّنكُم مِّنَ ٱلْغَآئِطِ أَوْ لَـٰمَسْتُمُ ٱلنِّسَآءَ فَلَمْ تَجِدُوا۟ مَآءًۭ فَتَيَمَّمُوا۟ صَعِيدًۭا طَيِّبًۭا فَٱمْسَحُوا۟ بِوُجُوهِكُمْ وَأَيْدِيكُم مِّنْهُ ۚ مَا يُرِيدُ ٱللَّهُ لِيَجْعَلَ عَلَيْكُم مِّنْ حَرَجٍۢ وَلَـٰكِن يُرِيدُ لِيُطَهِّرَكُمْ وَلِيُتِمَّ نِعْمَتَهُۥ عَلَيْكُمْ لَعَلَّكُمْ تَشْكُرُونَ ﴿٦﴾\\
\textamh{7.\  } & وَٱذْكُرُوا۟ نِعْمَةَ ٱللَّهِ عَلَيْكُمْ وَمِيثَـٰقَهُ ٱلَّذِى وَاثَقَكُم بِهِۦٓ إِذْ قُلْتُمْ سَمِعْنَا وَأَطَعْنَا ۖ وَٱتَّقُوا۟ ٱللَّهَ ۚ إِنَّ ٱللَّهَ عَلِيمٌۢ بِذَاتِ ٱلصُّدُورِ ﴿٧﴾\\
\textamh{8.\  } & يَـٰٓأَيُّهَا ٱلَّذِينَ ءَامَنُوا۟ كُونُوا۟ قَوَّٰمِينَ لِلَّهِ شُهَدَآءَ بِٱلْقِسْطِ ۖ وَلَا يَجْرِمَنَّكُمْ شَنَـَٔانُ قَوْمٍ عَلَىٰٓ أَلَّا تَعْدِلُوا۟ ۚ ٱعْدِلُوا۟ هُوَ أَقْرَبُ لِلتَّقْوَىٰ ۖ وَٱتَّقُوا۟ ٱللَّهَ ۚ إِنَّ ٱللَّهَ خَبِيرٌۢ بِمَا تَعْمَلُونَ ﴿٨﴾\\
\textamh{9.\  } & وَعَدَ ٱللَّهُ ٱلَّذِينَ ءَامَنُوا۟ وَعَمِلُوا۟ ٱلصَّـٰلِحَـٰتِ ۙ لَهُم مَّغْفِرَةٌۭ وَأَجْرٌ عَظِيمٌۭ ﴿٩﴾\\
\textamh{10.\  } & وَٱلَّذِينَ كَفَرُوا۟ وَكَذَّبُوا۟ بِـَٔايَـٰتِنَآ أُو۟لَـٰٓئِكَ أَصْحَـٰبُ ٱلْجَحِيمِ ﴿١٠﴾\\
\textamh{11.\  } & يَـٰٓأَيُّهَا ٱلَّذِينَ ءَامَنُوا۟ ٱذْكُرُوا۟ نِعْمَتَ ٱللَّهِ عَلَيْكُمْ إِذْ هَمَّ قَوْمٌ أَن يَبْسُطُوٓا۟ إِلَيْكُمْ أَيْدِيَهُمْ فَكَفَّ أَيْدِيَهُمْ عَنكُمْ ۖ وَٱتَّقُوا۟ ٱللَّهَ ۚ وَعَلَى ٱللَّهِ فَلْيَتَوَكَّلِ ٱلْمُؤْمِنُونَ ﴿١١﴾\\
\textamh{12.\  } & ۞ وَلَقَدْ أَخَذَ ٱللَّهُ مِيثَـٰقَ بَنِىٓ إِسْرَٰٓءِيلَ وَبَعَثْنَا مِنْهُمُ ٱثْنَىْ عَشَرَ نَقِيبًۭا ۖ وَقَالَ ٱللَّهُ إِنِّى مَعَكُمْ ۖ لَئِنْ أَقَمْتُمُ ٱلصَّلَوٰةَ وَءَاتَيْتُمُ ٱلزَّكَوٰةَ وَءَامَنتُم بِرُسُلِى وَعَزَّرْتُمُوهُمْ وَأَقْرَضْتُمُ ٱللَّهَ قَرْضًا حَسَنًۭا لَّأُكَفِّرَنَّ عَنكُمْ سَيِّـَٔاتِكُمْ وَلَأُدْخِلَنَّكُمْ جَنَّـٰتٍۢ تَجْرِى مِن تَحْتِهَا ٱلْأَنْهَـٰرُ ۚ فَمَن كَفَرَ بَعْدَ ذَٟلِكَ مِنكُمْ فَقَدْ ضَلَّ سَوَآءَ ٱلسَّبِيلِ ﴿١٢﴾\\
\textamh{13.\  } & فَبِمَا نَقْضِهِم مِّيثَـٰقَهُمْ لَعَنَّـٰهُمْ وَجَعَلْنَا قُلُوبَهُمْ قَـٰسِيَةًۭ ۖ يُحَرِّفُونَ ٱلْكَلِمَ عَن مَّوَاضِعِهِۦ ۙ وَنَسُوا۟ حَظًّۭا مِّمَّا ذُكِّرُوا۟ بِهِۦ ۚ وَلَا تَزَالُ تَطَّلِعُ عَلَىٰ خَآئِنَةٍۢ مِّنْهُمْ إِلَّا قَلِيلًۭا مِّنْهُمْ ۖ فَٱعْفُ عَنْهُمْ وَٱصْفَحْ ۚ إِنَّ ٱللَّهَ يُحِبُّ ٱلْمُحْسِنِينَ ﴿١٣﴾\\
\textamh{14.\  } & وَمِنَ ٱلَّذِينَ قَالُوٓا۟ إِنَّا نَصَـٰرَىٰٓ أَخَذْنَا مِيثَـٰقَهُمْ فَنَسُوا۟ حَظًّۭا مِّمَّا ذُكِّرُوا۟ بِهِۦ فَأَغْرَيْنَا بَيْنَهُمُ ٱلْعَدَاوَةَ وَٱلْبَغْضَآءَ إِلَىٰ يَوْمِ ٱلْقِيَـٰمَةِ ۚ وَسَوْفَ يُنَبِّئُهُمُ ٱللَّهُ بِمَا كَانُوا۟ يَصْنَعُونَ ﴿١٤﴾\\
\textamh{15.\  } & يَـٰٓأَهْلَ ٱلْكِتَـٰبِ قَدْ جَآءَكُمْ رَسُولُنَا يُبَيِّنُ لَكُمْ كَثِيرًۭا مِّمَّا كُنتُمْ تُخْفُونَ مِنَ ٱلْكِتَـٰبِ وَيَعْفُوا۟ عَن كَثِيرٍۢ ۚ قَدْ جَآءَكُم مِّنَ ٱللَّهِ نُورٌۭ وَكِتَـٰبٌۭ مُّبِينٌۭ ﴿١٥﴾\\
\textamh{16.\  } & يَهْدِى بِهِ ٱللَّهُ مَنِ ٱتَّبَعَ رِضْوَٟنَهُۥ سُبُلَ ٱلسَّلَـٰمِ وَيُخْرِجُهُم مِّنَ ٱلظُّلُمَـٰتِ إِلَى ٱلنُّورِ بِإِذْنِهِۦ وَيَهْدِيهِمْ إِلَىٰ صِرَٰطٍۢ مُّسْتَقِيمٍۢ ﴿١٦﴾\\
\textamh{17.\  } & لَّقَدْ كَفَرَ ٱلَّذِينَ قَالُوٓا۟ إِنَّ ٱللَّهَ هُوَ ٱلْمَسِيحُ ٱبْنُ مَرْيَمَ ۚ قُلْ فَمَن يَمْلِكُ مِنَ ٱللَّهِ شَيْـًٔا إِنْ أَرَادَ أَن يُهْلِكَ ٱلْمَسِيحَ ٱبْنَ مَرْيَمَ وَأُمَّهُۥ وَمَن فِى ٱلْأَرْضِ جَمِيعًۭا ۗ وَلِلَّهِ مُلْكُ ٱلسَّمَـٰوَٟتِ وَٱلْأَرْضِ وَمَا بَيْنَهُمَا ۚ يَخْلُقُ مَا يَشَآءُ ۚ وَٱللَّهُ عَلَىٰ كُلِّ شَىْءٍۢ قَدِيرٌۭ ﴿١٧﴾\\
\textamh{18.\  } & وَقَالَتِ ٱلْيَهُودُ وَٱلنَّصَـٰرَىٰ نَحْنُ أَبْنَـٰٓؤُا۟ ٱللَّهِ وَأَحِبَّـٰٓؤُهُۥ ۚ قُلْ فَلِمَ يُعَذِّبُكُم بِذُنُوبِكُم ۖ بَلْ أَنتُم بَشَرٌۭ مِّمَّنْ خَلَقَ ۚ يَغْفِرُ لِمَن يَشَآءُ وَيُعَذِّبُ مَن يَشَآءُ ۚ وَلِلَّهِ مُلْكُ ٱلسَّمَـٰوَٟتِ وَٱلْأَرْضِ وَمَا بَيْنَهُمَا ۖ وَإِلَيْهِ ٱلْمَصِيرُ ﴿١٨﴾\\
\textamh{19.\  } & يَـٰٓأَهْلَ ٱلْكِتَـٰبِ قَدْ جَآءَكُمْ رَسُولُنَا يُبَيِّنُ لَكُمْ عَلَىٰ فَتْرَةٍۢ مِّنَ ٱلرُّسُلِ أَن تَقُولُوا۟ مَا جَآءَنَا مِنۢ بَشِيرٍۢ وَلَا نَذِيرٍۢ ۖ فَقَدْ جَآءَكُم بَشِيرٌۭ وَنَذِيرٌۭ ۗ وَٱللَّهُ عَلَىٰ كُلِّ شَىْءٍۢ قَدِيرٌۭ ﴿١٩﴾\\
\textamh{20.\  } & وَإِذْ قَالَ مُوسَىٰ لِقَوْمِهِۦ يَـٰقَوْمِ ٱذْكُرُوا۟ نِعْمَةَ ٱللَّهِ عَلَيْكُمْ إِذْ جَعَلَ فِيكُمْ أَنۢبِيَآءَ وَجَعَلَكُم مُّلُوكًۭا وَءَاتَىٰكُم مَّا لَمْ يُؤْتِ أَحَدًۭا مِّنَ ٱلْعَـٰلَمِينَ ﴿٢٠﴾\\
\textamh{21.\  } & يَـٰقَوْمِ ٱدْخُلُوا۟ ٱلْأَرْضَ ٱلْمُقَدَّسَةَ ٱلَّتِى كَتَبَ ٱللَّهُ لَكُمْ وَلَا تَرْتَدُّوا۟ عَلَىٰٓ أَدْبَارِكُمْ فَتَنقَلِبُوا۟ خَـٰسِرِينَ ﴿٢١﴾\\
\textamh{22.\  } & قَالُوا۟ يَـٰمُوسَىٰٓ إِنَّ فِيهَا قَوْمًۭا جَبَّارِينَ وَإِنَّا لَن نَّدْخُلَهَا حَتَّىٰ يَخْرُجُوا۟ مِنْهَا فَإِن يَخْرُجُوا۟ مِنْهَا فَإِنَّا دَٟخِلُونَ ﴿٢٢﴾\\
\textamh{23.\  } & قَالَ رَجُلَانِ مِنَ ٱلَّذِينَ يَخَافُونَ أَنْعَمَ ٱللَّهُ عَلَيْهِمَا ٱدْخُلُوا۟ عَلَيْهِمُ ٱلْبَابَ فَإِذَا دَخَلْتُمُوهُ فَإِنَّكُمْ غَٰلِبُونَ ۚ وَعَلَى ٱللَّهِ فَتَوَكَّلُوٓا۟ إِن كُنتُم مُّؤْمِنِينَ ﴿٢٣﴾\\
\textamh{24.\  } & قَالُوا۟ يَـٰمُوسَىٰٓ إِنَّا لَن نَّدْخُلَهَآ أَبَدًۭا مَّا دَامُوا۟ فِيهَا ۖ فَٱذْهَبْ أَنتَ وَرَبُّكَ فَقَـٰتِلَآ إِنَّا هَـٰهُنَا قَـٰعِدُونَ ﴿٢٤﴾\\
\textamh{25.\  } & قَالَ رَبِّ إِنِّى لَآ أَمْلِكُ إِلَّا نَفْسِى وَأَخِى ۖ فَٱفْرُقْ بَيْنَنَا وَبَيْنَ ٱلْقَوْمِ ٱلْفَـٰسِقِينَ ﴿٢٥﴾\\
\textamh{26.\  } & قَالَ فَإِنَّهَا مُحَرَّمَةٌ عَلَيْهِمْ ۛ أَرْبَعِينَ سَنَةًۭ ۛ يَتِيهُونَ فِى ٱلْأَرْضِ ۚ فَلَا تَأْسَ عَلَى ٱلْقَوْمِ ٱلْفَـٰسِقِينَ ﴿٢٦﴾\\
\textamh{27.\  } & ۞ وَٱتْلُ عَلَيْهِمْ نَبَأَ ٱبْنَىْ ءَادَمَ بِٱلْحَقِّ إِذْ قَرَّبَا قُرْبَانًۭا فَتُقُبِّلَ مِنْ أَحَدِهِمَا وَلَمْ يُتَقَبَّلْ مِنَ ٱلْءَاخَرِ قَالَ لَأَقْتُلَنَّكَ ۖ قَالَ إِنَّمَا يَتَقَبَّلُ ٱللَّهُ مِنَ ٱلْمُتَّقِينَ ﴿٢٧﴾\\
\textamh{28.\  } & لَئِنۢ بَسَطتَ إِلَىَّ يَدَكَ لِتَقْتُلَنِى مَآ أَنَا۠ بِبَاسِطٍۢ يَدِىَ إِلَيْكَ لِأَقْتُلَكَ ۖ إِنِّىٓ أَخَافُ ٱللَّهَ رَبَّ ٱلْعَـٰلَمِينَ ﴿٢٨﴾\\
\textamh{29.\  } & إِنِّىٓ أُرِيدُ أَن تَبُوٓأَ بِإِثْمِى وَإِثْمِكَ فَتَكُونَ مِنْ أَصْحَـٰبِ ٱلنَّارِ ۚ وَذَٟلِكَ جَزَٰٓؤُا۟ ٱلظَّـٰلِمِينَ ﴿٢٩﴾\\
\textamh{30.\  } & فَطَوَّعَتْ لَهُۥ نَفْسُهُۥ قَتْلَ أَخِيهِ فَقَتَلَهُۥ فَأَصْبَحَ مِنَ ٱلْخَـٰسِرِينَ ﴿٣٠﴾\\
\textamh{31.\  } & فَبَعَثَ ٱللَّهُ غُرَابًۭا يَبْحَثُ فِى ٱلْأَرْضِ لِيُرِيَهُۥ كَيْفَ يُوَٟرِى سَوْءَةَ أَخِيهِ ۚ قَالَ يَـٰوَيْلَتَىٰٓ أَعَجَزْتُ أَنْ أَكُونَ مِثْلَ هَـٰذَا ٱلْغُرَابِ فَأُوَٟرِىَ سَوْءَةَ أَخِى ۖ فَأَصْبَحَ مِنَ ٱلنَّـٰدِمِينَ ﴿٣١﴾\\
\textamh{32.\  } & مِنْ أَجْلِ ذَٟلِكَ كَتَبْنَا عَلَىٰ بَنِىٓ إِسْرَٰٓءِيلَ أَنَّهُۥ مَن قَتَلَ نَفْسًۢا بِغَيْرِ نَفْسٍ أَوْ فَسَادٍۢ فِى ٱلْأَرْضِ فَكَأَنَّمَا قَتَلَ ٱلنَّاسَ جَمِيعًۭا وَمَنْ أَحْيَاهَا فَكَأَنَّمَآ أَحْيَا ٱلنَّاسَ جَمِيعًۭا ۚ وَلَقَدْ جَآءَتْهُمْ رُسُلُنَا بِٱلْبَيِّنَـٰتِ ثُمَّ إِنَّ كَثِيرًۭا مِّنْهُم بَعْدَ ذَٟلِكَ فِى ٱلْأَرْضِ لَمُسْرِفُونَ ﴿٣٢﴾\\
\textamh{33.\  } & إِنَّمَا جَزَٰٓؤُا۟ ٱلَّذِينَ يُحَارِبُونَ ٱللَّهَ وَرَسُولَهُۥ وَيَسْعَوْنَ فِى ٱلْأَرْضِ فَسَادًا أَن يُقَتَّلُوٓا۟ أَوْ يُصَلَّبُوٓا۟ أَوْ تُقَطَّعَ أَيْدِيهِمْ وَأَرْجُلُهُم مِّنْ خِلَـٰفٍ أَوْ يُنفَوْا۟ مِنَ ٱلْأَرْضِ ۚ ذَٟلِكَ لَهُمْ خِزْىٌۭ فِى ٱلدُّنْيَا ۖ وَلَهُمْ فِى ٱلْءَاخِرَةِ عَذَابٌ عَظِيمٌ ﴿٣٣﴾\\
\textamh{34.\  } & إِلَّا ٱلَّذِينَ تَابُوا۟ مِن قَبْلِ أَن تَقْدِرُوا۟ عَلَيْهِمْ ۖ فَٱعْلَمُوٓا۟ أَنَّ ٱللَّهَ غَفُورٌۭ رَّحِيمٌۭ ﴿٣٤﴾\\
\textamh{35.\  } & يَـٰٓأَيُّهَا ٱلَّذِينَ ءَامَنُوا۟ ٱتَّقُوا۟ ٱللَّهَ وَٱبْتَغُوٓا۟ إِلَيْهِ ٱلْوَسِيلَةَ وَجَٰهِدُوا۟ فِى سَبِيلِهِۦ لَعَلَّكُمْ تُفْلِحُونَ ﴿٣٥﴾\\
\textamh{36.\  } & إِنَّ ٱلَّذِينَ كَفَرُوا۟ لَوْ أَنَّ لَهُم مَّا فِى ٱلْأَرْضِ جَمِيعًۭا وَمِثْلَهُۥ مَعَهُۥ لِيَفْتَدُوا۟ بِهِۦ مِنْ عَذَابِ يَوْمِ ٱلْقِيَـٰمَةِ مَا تُقُبِّلَ مِنْهُمْ ۖ وَلَهُمْ عَذَابٌ أَلِيمٌۭ ﴿٣٦﴾\\
\textamh{37.\  } & يُرِيدُونَ أَن يَخْرُجُوا۟ مِنَ ٱلنَّارِ وَمَا هُم بِخَـٰرِجِينَ مِنْهَا ۖ وَلَهُمْ عَذَابٌۭ مُّقِيمٌۭ ﴿٣٧﴾\\
\textamh{38.\  } & وَٱلسَّارِقُ وَٱلسَّارِقَةُ فَٱقْطَعُوٓا۟ أَيْدِيَهُمَا جَزَآءًۢ بِمَا كَسَبَا نَكَـٰلًۭا مِّنَ ٱللَّهِ ۗ وَٱللَّهُ عَزِيزٌ حَكِيمٌۭ ﴿٣٨﴾\\
\textamh{39.\  } & فَمَن تَابَ مِنۢ بَعْدِ ظُلْمِهِۦ وَأَصْلَحَ فَإِنَّ ٱللَّهَ يَتُوبُ عَلَيْهِ ۗ إِنَّ ٱللَّهَ غَفُورٌۭ رَّحِيمٌ ﴿٣٩﴾\\
\textamh{40.\  } & أَلَمْ تَعْلَمْ أَنَّ ٱللَّهَ لَهُۥ مُلْكُ ٱلسَّمَـٰوَٟتِ وَٱلْأَرْضِ يُعَذِّبُ مَن يَشَآءُ وَيَغْفِرُ لِمَن يَشَآءُ ۗ وَٱللَّهُ عَلَىٰ كُلِّ شَىْءٍۢ قَدِيرٌۭ ﴿٤٠﴾\\
\textamh{41.\  } & ۞ يَـٰٓأَيُّهَا ٱلرَّسُولُ لَا يَحْزُنكَ ٱلَّذِينَ يُسَـٰرِعُونَ فِى ٱلْكُفْرِ مِنَ ٱلَّذِينَ قَالُوٓا۟ ءَامَنَّا بِأَفْوَٟهِهِمْ وَلَمْ تُؤْمِن قُلُوبُهُمْ ۛ وَمِنَ ٱلَّذِينَ هَادُوا۟ ۛ سَمَّٰعُونَ لِلْكَذِبِ سَمَّٰعُونَ لِقَوْمٍ ءَاخَرِينَ لَمْ يَأْتُوكَ ۖ يُحَرِّفُونَ ٱلْكَلِمَ مِنۢ بَعْدِ مَوَاضِعِهِۦ ۖ يَقُولُونَ إِنْ أُوتِيتُمْ هَـٰذَا فَخُذُوهُ وَإِن لَّمْ تُؤْتَوْهُ فَٱحْذَرُوا۟ ۚ وَمَن يُرِدِ ٱللَّهُ فِتْنَتَهُۥ فَلَن تَمْلِكَ لَهُۥ مِنَ ٱللَّهِ شَيْـًٔا ۚ أُو۟لَـٰٓئِكَ ٱلَّذِينَ لَمْ يُرِدِ ٱللَّهُ أَن يُطَهِّرَ قُلُوبَهُمْ ۚ لَهُمْ فِى ٱلدُّنْيَا خِزْىٌۭ ۖ وَلَهُمْ فِى ٱلْءَاخِرَةِ عَذَابٌ عَظِيمٌۭ ﴿٤١﴾\\
\textamh{42.\  } & سَمَّٰعُونَ لِلْكَذِبِ أَكَّٰلُونَ لِلسُّحْتِ ۚ فَإِن جَآءُوكَ فَٱحْكُم بَيْنَهُمْ أَوْ أَعْرِضْ عَنْهُمْ ۖ وَإِن تُعْرِضْ عَنْهُمْ فَلَن يَضُرُّوكَ شَيْـًۭٔا ۖ وَإِنْ حَكَمْتَ فَٱحْكُم بَيْنَهُم بِٱلْقِسْطِ ۚ إِنَّ ٱللَّهَ يُحِبُّ ٱلْمُقْسِطِينَ ﴿٤٢﴾\\
\textamh{43.\  } & وَكَيْفَ يُحَكِّمُونَكَ وَعِندَهُمُ ٱلتَّوْرَىٰةُ فِيهَا حُكْمُ ٱللَّهِ ثُمَّ يَتَوَلَّوْنَ مِنۢ بَعْدِ ذَٟلِكَ ۚ وَمَآ أُو۟لَـٰٓئِكَ بِٱلْمُؤْمِنِينَ ﴿٤٣﴾\\
\textamh{44.\  } & إِنَّآ أَنزَلْنَا ٱلتَّوْرَىٰةَ فِيهَا هُدًۭى وَنُورٌۭ ۚ يَحْكُمُ بِهَا ٱلنَّبِيُّونَ ٱلَّذِينَ أَسْلَمُوا۟ لِلَّذِينَ هَادُوا۟ وَٱلرَّبَّـٰنِيُّونَ وَٱلْأَحْبَارُ بِمَا ٱسْتُحْفِظُوا۟ مِن كِتَـٰبِ ٱللَّهِ وَكَانُوا۟ عَلَيْهِ شُهَدَآءَ ۚ فَلَا تَخْشَوُا۟ ٱلنَّاسَ وَٱخْشَوْنِ وَلَا تَشْتَرُوا۟ بِـَٔايَـٰتِى ثَمَنًۭا قَلِيلًۭا ۚ وَمَن لَّمْ يَحْكُم بِمَآ أَنزَلَ ٱللَّهُ فَأُو۟لَـٰٓئِكَ هُمُ ٱلْكَـٰفِرُونَ ﴿٤٤﴾\\
\textamh{45.\  } & وَكَتَبْنَا عَلَيْهِمْ فِيهَآ أَنَّ ٱلنَّفْسَ بِٱلنَّفْسِ وَٱلْعَيْنَ بِٱلْعَيْنِ وَٱلْأَنفَ بِٱلْأَنفِ وَٱلْأُذُنَ بِٱلْأُذُنِ وَٱلسِّنَّ بِٱلسِّنِّ وَٱلْجُرُوحَ قِصَاصٌۭ ۚ فَمَن تَصَدَّقَ بِهِۦ فَهُوَ كَفَّارَةٌۭ لَّهُۥ ۚ وَمَن لَّمْ يَحْكُم بِمَآ أَنزَلَ ٱللَّهُ فَأُو۟لَـٰٓئِكَ هُمُ ٱلظَّـٰلِمُونَ ﴿٤٥﴾\\
\textamh{46.\  } & وَقَفَّيْنَا عَلَىٰٓ ءَاثَـٰرِهِم بِعِيسَى ٱبْنِ مَرْيَمَ مُصَدِّقًۭا لِّمَا بَيْنَ يَدَيْهِ مِنَ ٱلتَّوْرَىٰةِ ۖ وَءَاتَيْنَـٰهُ ٱلْإِنجِيلَ فِيهِ هُدًۭى وَنُورٌۭ وَمُصَدِّقًۭا لِّمَا بَيْنَ يَدَيْهِ مِنَ ٱلتَّوْرَىٰةِ وَهُدًۭى وَمَوْعِظَةًۭ لِّلْمُتَّقِينَ ﴿٤٦﴾\\
\textamh{47.\  } & وَلْيَحْكُمْ أَهْلُ ٱلْإِنجِيلِ بِمَآ أَنزَلَ ٱللَّهُ فِيهِ ۚ وَمَن لَّمْ يَحْكُم بِمَآ أَنزَلَ ٱللَّهُ فَأُو۟لَـٰٓئِكَ هُمُ ٱلْفَـٰسِقُونَ ﴿٤٧﴾\\
\textamh{48.\  } & وَأَنزَلْنَآ إِلَيْكَ ٱلْكِتَـٰبَ بِٱلْحَقِّ مُصَدِّقًۭا لِّمَا بَيْنَ يَدَيْهِ مِنَ ٱلْكِتَـٰبِ وَمُهَيْمِنًا عَلَيْهِ ۖ فَٱحْكُم بَيْنَهُم بِمَآ أَنزَلَ ٱللَّهُ ۖ وَلَا تَتَّبِعْ أَهْوَآءَهُمْ عَمَّا جَآءَكَ مِنَ ٱلْحَقِّ ۚ لِكُلٍّۢ جَعَلْنَا مِنكُمْ شِرْعَةًۭ وَمِنْهَاجًۭا ۚ وَلَوْ شَآءَ ٱللَّهُ لَجَعَلَكُمْ أُمَّةًۭ وَٟحِدَةًۭ وَلَـٰكِن لِّيَبْلُوَكُمْ فِى مَآ ءَاتَىٰكُمْ ۖ فَٱسْتَبِقُوا۟ ٱلْخَيْرَٰتِ ۚ إِلَى ٱللَّهِ مَرْجِعُكُمْ جَمِيعًۭا فَيُنَبِّئُكُم بِمَا كُنتُمْ فِيهِ تَخْتَلِفُونَ ﴿٤٨﴾\\
\textamh{49.\  } & وَأَنِ ٱحْكُم بَيْنَهُم بِمَآ أَنزَلَ ٱللَّهُ وَلَا تَتَّبِعْ أَهْوَآءَهُمْ وَٱحْذَرْهُمْ أَن يَفْتِنُوكَ عَنۢ بَعْضِ مَآ أَنزَلَ ٱللَّهُ إِلَيْكَ ۖ فَإِن تَوَلَّوْا۟ فَٱعْلَمْ أَنَّمَا يُرِيدُ ٱللَّهُ أَن يُصِيبَهُم بِبَعْضِ ذُنُوبِهِمْ ۗ وَإِنَّ كَثِيرًۭا مِّنَ ٱلنَّاسِ لَفَـٰسِقُونَ ﴿٤٩﴾\\
\textamh{50.\  } & أَفَحُكْمَ ٱلْجَٰهِلِيَّةِ يَبْغُونَ ۚ وَمَنْ أَحْسَنُ مِنَ ٱللَّهِ حُكْمًۭا لِّقَوْمٍۢ يُوقِنُونَ ﴿٥٠﴾\\
\textamh{51.\  } & ۞ يَـٰٓأَيُّهَا ٱلَّذِينَ ءَامَنُوا۟ لَا تَتَّخِذُوا۟ ٱلْيَهُودَ وَٱلنَّصَـٰرَىٰٓ أَوْلِيَآءَ ۘ بَعْضُهُمْ أَوْلِيَآءُ بَعْضٍۢ ۚ وَمَن يَتَوَلَّهُم مِّنكُمْ فَإِنَّهُۥ مِنْهُمْ ۗ إِنَّ ٱللَّهَ لَا يَهْدِى ٱلْقَوْمَ ٱلظَّـٰلِمِينَ ﴿٥١﴾\\
\textamh{52.\  } & فَتَرَى ٱلَّذِينَ فِى قُلُوبِهِم مَّرَضٌۭ يُسَـٰرِعُونَ فِيهِمْ يَقُولُونَ نَخْشَىٰٓ أَن تُصِيبَنَا دَآئِرَةٌۭ ۚ فَعَسَى ٱللَّهُ أَن يَأْتِىَ بِٱلْفَتْحِ أَوْ أَمْرٍۢ مِّنْ عِندِهِۦ فَيُصْبِحُوا۟ عَلَىٰ مَآ أَسَرُّوا۟ فِىٓ أَنفُسِهِمْ نَـٰدِمِينَ ﴿٥٢﴾\\
\textamh{53.\  } & وَيَقُولُ ٱلَّذِينَ ءَامَنُوٓا۟ أَهَـٰٓؤُلَآءِ ٱلَّذِينَ أَقْسَمُوا۟ بِٱللَّهِ جَهْدَ أَيْمَـٰنِهِمْ ۙ إِنَّهُمْ لَمَعَكُمْ ۚ حَبِطَتْ أَعْمَـٰلُهُمْ فَأَصْبَحُوا۟ خَـٰسِرِينَ ﴿٥٣﴾\\
\textamh{54.\  } & يَـٰٓأَيُّهَا ٱلَّذِينَ ءَامَنُوا۟ مَن يَرْتَدَّ مِنكُمْ عَن دِينِهِۦ فَسَوْفَ يَأْتِى ٱللَّهُ بِقَوْمٍۢ يُحِبُّهُمْ وَيُحِبُّونَهُۥٓ أَذِلَّةٍ عَلَى ٱلْمُؤْمِنِينَ أَعِزَّةٍ عَلَى ٱلْكَـٰفِرِينَ يُجَٰهِدُونَ فِى سَبِيلِ ٱللَّهِ وَلَا يَخَافُونَ لَوْمَةَ لَآئِمٍۢ ۚ ذَٟلِكَ فَضْلُ ٱللَّهِ يُؤْتِيهِ مَن يَشَآءُ ۚ وَٱللَّهُ وَٟسِعٌ عَلِيمٌ ﴿٥٤﴾\\
\textamh{55.\  } & إِنَّمَا وَلِيُّكُمُ ٱللَّهُ وَرَسُولُهُۥ وَٱلَّذِينَ ءَامَنُوا۟ ٱلَّذِينَ يُقِيمُونَ ٱلصَّلَوٰةَ وَيُؤْتُونَ ٱلزَّكَوٰةَ وَهُمْ رَٰكِعُونَ ﴿٥٥﴾\\
\textamh{56.\  } & وَمَن يَتَوَلَّ ٱللَّهَ وَرَسُولَهُۥ وَٱلَّذِينَ ءَامَنُوا۟ فَإِنَّ حِزْبَ ٱللَّهِ هُمُ ٱلْغَٰلِبُونَ ﴿٥٦﴾\\
\textamh{57.\  } & يَـٰٓأَيُّهَا ٱلَّذِينَ ءَامَنُوا۟ لَا تَتَّخِذُوا۟ ٱلَّذِينَ ٱتَّخَذُوا۟ دِينَكُمْ هُزُوًۭا وَلَعِبًۭا مِّنَ ٱلَّذِينَ أُوتُوا۟ ٱلْكِتَـٰبَ مِن قَبْلِكُمْ وَٱلْكُفَّارَ أَوْلِيَآءَ ۚ وَٱتَّقُوا۟ ٱللَّهَ إِن كُنتُم مُّؤْمِنِينَ ﴿٥٧﴾\\
\textamh{58.\  } & وَإِذَا نَادَيْتُمْ إِلَى ٱلصَّلَوٰةِ ٱتَّخَذُوهَا هُزُوًۭا وَلَعِبًۭا ۚ ذَٟلِكَ بِأَنَّهُمْ قَوْمٌۭ لَّا يَعْقِلُونَ ﴿٥٨﴾\\
\textamh{59.\  } & قُلْ يَـٰٓأَهْلَ ٱلْكِتَـٰبِ هَلْ تَنقِمُونَ مِنَّآ إِلَّآ أَنْ ءَامَنَّا بِٱللَّهِ وَمَآ أُنزِلَ إِلَيْنَا وَمَآ أُنزِلَ مِن قَبْلُ وَأَنَّ أَكْثَرَكُمْ فَـٰسِقُونَ ﴿٥٩﴾\\
\textamh{60.\  } & قُلْ هَلْ أُنَبِّئُكُم بِشَرٍّۢ مِّن ذَٟلِكَ مَثُوبَةً عِندَ ٱللَّهِ ۚ مَن لَّعَنَهُ ٱللَّهُ وَغَضِبَ عَلَيْهِ وَجَعَلَ مِنْهُمُ ٱلْقِرَدَةَ وَٱلْخَنَازِيرَ وَعَبَدَ ٱلطَّٰغُوتَ ۚ أُو۟لَـٰٓئِكَ شَرٌّۭ مَّكَانًۭا وَأَضَلُّ عَن سَوَآءِ ٱلسَّبِيلِ ﴿٦٠﴾\\
\textamh{61.\  } & وَإِذَا جَآءُوكُمْ قَالُوٓا۟ ءَامَنَّا وَقَد دَّخَلُوا۟ بِٱلْكُفْرِ وَهُمْ قَدْ خَرَجُوا۟ بِهِۦ ۚ وَٱللَّهُ أَعْلَمُ بِمَا كَانُوا۟ يَكْتُمُونَ ﴿٦١﴾\\
\textamh{62.\  } & وَتَرَىٰ كَثِيرًۭا مِّنْهُمْ يُسَـٰرِعُونَ فِى ٱلْإِثْمِ وَٱلْعُدْوَٟنِ وَأَكْلِهِمُ ٱلسُّحْتَ ۚ لَبِئْسَ مَا كَانُوا۟ يَعْمَلُونَ ﴿٦٢﴾\\
\textamh{63.\  } & لَوْلَا يَنْهَىٰهُمُ ٱلرَّبَّـٰنِيُّونَ وَٱلْأَحْبَارُ عَن قَوْلِهِمُ ٱلْإِثْمَ وَأَكْلِهِمُ ٱلسُّحْتَ ۚ لَبِئْسَ مَا كَانُوا۟ يَصْنَعُونَ ﴿٦٣﴾\\
\textamh{64.\  } & وَقَالَتِ ٱلْيَهُودُ يَدُ ٱللَّهِ مَغْلُولَةٌ ۚ غُلَّتْ أَيْدِيهِمْ وَلُعِنُوا۟ بِمَا قَالُوا۟ ۘ بَلْ يَدَاهُ مَبْسُوطَتَانِ يُنفِقُ كَيْفَ يَشَآءُ ۚ وَلَيَزِيدَنَّ كَثِيرًۭا مِّنْهُم مَّآ أُنزِلَ إِلَيْكَ مِن رَّبِّكَ طُغْيَـٰنًۭا وَكُفْرًۭا ۚ وَأَلْقَيْنَا بَيْنَهُمُ ٱلْعَدَٟوَةَ وَٱلْبَغْضَآءَ إِلَىٰ يَوْمِ ٱلْقِيَـٰمَةِ ۚ كُلَّمَآ أَوْقَدُوا۟ نَارًۭا لِّلْحَرْبِ أَطْفَأَهَا ٱللَّهُ ۚ وَيَسْعَوْنَ فِى ٱلْأَرْضِ فَسَادًۭا ۚ وَٱللَّهُ لَا يُحِبُّ ٱلْمُفْسِدِينَ ﴿٦٤﴾\\
\textamh{65.\  } & وَلَوْ أَنَّ أَهْلَ ٱلْكِتَـٰبِ ءَامَنُوا۟ وَٱتَّقَوْا۟ لَكَفَّرْنَا عَنْهُمْ سَيِّـَٔاتِهِمْ وَلَأَدْخَلْنَـٰهُمْ جَنَّـٰتِ ٱلنَّعِيمِ ﴿٦٥﴾\\
\textamh{66.\  } & وَلَوْ أَنَّهُمْ أَقَامُوا۟ ٱلتَّوْرَىٰةَ وَٱلْإِنجِيلَ وَمَآ أُنزِلَ إِلَيْهِم مِّن رَّبِّهِمْ لَأَكَلُوا۟ مِن فَوْقِهِمْ وَمِن تَحْتِ أَرْجُلِهِم ۚ مِّنْهُمْ أُمَّةٌۭ مُّقْتَصِدَةٌۭ ۖ وَكَثِيرٌۭ مِّنْهُمْ سَآءَ مَا يَعْمَلُونَ ﴿٦٦﴾\\
\textamh{67.\  } & ۞ يَـٰٓأَيُّهَا ٱلرَّسُولُ بَلِّغْ مَآ أُنزِلَ إِلَيْكَ مِن رَّبِّكَ ۖ وَإِن لَّمْ تَفْعَلْ فَمَا بَلَّغْتَ رِسَالَتَهُۥ ۚ وَٱللَّهُ يَعْصِمُكَ مِنَ ٱلنَّاسِ ۗ إِنَّ ٱللَّهَ لَا يَهْدِى ٱلْقَوْمَ ٱلْكَـٰفِرِينَ ﴿٦٧﴾\\
\textamh{68.\  } & قُلْ يَـٰٓأَهْلَ ٱلْكِتَـٰبِ لَسْتُمْ عَلَىٰ شَىْءٍ حَتَّىٰ تُقِيمُوا۟ ٱلتَّوْرَىٰةَ وَٱلْإِنجِيلَ وَمَآ أُنزِلَ إِلَيْكُم مِّن رَّبِّكُمْ ۗ وَلَيَزِيدَنَّ كَثِيرًۭا مِّنْهُم مَّآ أُنزِلَ إِلَيْكَ مِن رَّبِّكَ طُغْيَـٰنًۭا وَكُفْرًۭا ۖ فَلَا تَأْسَ عَلَى ٱلْقَوْمِ ٱلْكَـٰفِرِينَ ﴿٦٨﴾\\
\textamh{69.\  } & إِنَّ ٱلَّذِينَ ءَامَنُوا۟ وَٱلَّذِينَ هَادُوا۟ وَٱلصَّـٰبِـُٔونَ وَٱلنَّصَـٰرَىٰ مَنْ ءَامَنَ بِٱللَّهِ وَٱلْيَوْمِ ٱلْءَاخِرِ وَعَمِلَ صَـٰلِحًۭا فَلَا خَوْفٌ عَلَيْهِمْ وَلَا هُمْ يَحْزَنُونَ ﴿٦٩﴾\\
\textamh{70.\  } & لَقَدْ أَخَذْنَا مِيثَـٰقَ بَنِىٓ إِسْرَٰٓءِيلَ وَأَرْسَلْنَآ إِلَيْهِمْ رُسُلًۭا ۖ كُلَّمَا جَآءَهُمْ رَسُولٌۢ بِمَا لَا تَهْوَىٰٓ أَنفُسُهُمْ فَرِيقًۭا كَذَّبُوا۟ وَفَرِيقًۭا يَقْتُلُونَ ﴿٧٠﴾\\
\textamh{71.\  } & وَحَسِبُوٓا۟ أَلَّا تَكُونَ فِتْنَةٌۭ فَعَمُوا۟ وَصَمُّوا۟ ثُمَّ تَابَ ٱللَّهُ عَلَيْهِمْ ثُمَّ عَمُوا۟ وَصَمُّوا۟ كَثِيرٌۭ مِّنْهُمْ ۚ وَٱللَّهُ بَصِيرٌۢ بِمَا يَعْمَلُونَ ﴿٧١﴾\\
\textamh{72.\  } & لَقَدْ كَفَرَ ٱلَّذِينَ قَالُوٓا۟ إِنَّ ٱللَّهَ هُوَ ٱلْمَسِيحُ ٱبْنُ مَرْيَمَ ۖ وَقَالَ ٱلْمَسِيحُ يَـٰبَنِىٓ إِسْرَٰٓءِيلَ ٱعْبُدُوا۟ ٱللَّهَ رَبِّى وَرَبَّكُمْ ۖ إِنَّهُۥ مَن يُشْرِكْ بِٱللَّهِ فَقَدْ حَرَّمَ ٱللَّهُ عَلَيْهِ ٱلْجَنَّةَ وَمَأْوَىٰهُ ٱلنَّارُ ۖ وَمَا لِلظَّـٰلِمِينَ مِنْ أَنصَارٍۢ ﴿٧٢﴾\\
\textamh{73.\  } & لَّقَدْ كَفَرَ ٱلَّذِينَ قَالُوٓا۟ إِنَّ ٱللَّهَ ثَالِثُ ثَلَـٰثَةٍۢ ۘ وَمَا مِنْ إِلَـٰهٍ إِلَّآ إِلَـٰهٌۭ وَٟحِدٌۭ ۚ وَإِن لَّمْ يَنتَهُوا۟ عَمَّا يَقُولُونَ لَيَمَسَّنَّ ٱلَّذِينَ كَفَرُوا۟ مِنْهُمْ عَذَابٌ أَلِيمٌ ﴿٧٣﴾\\
\textamh{74.\  } & أَفَلَا يَتُوبُونَ إِلَى ٱللَّهِ وَيَسْتَغْفِرُونَهُۥ ۚ وَٱللَّهُ غَفُورٌۭ رَّحِيمٌۭ ﴿٧٤﴾\\
\textamh{75.\  } & مَّا ٱلْمَسِيحُ ٱبْنُ مَرْيَمَ إِلَّا رَسُولٌۭ قَدْ خَلَتْ مِن قَبْلِهِ ٱلرُّسُلُ وَأُمُّهُۥ صِدِّيقَةٌۭ ۖ كَانَا يَأْكُلَانِ ٱلطَّعَامَ ۗ ٱنظُرْ كَيْفَ نُبَيِّنُ لَهُمُ ٱلْءَايَـٰتِ ثُمَّ ٱنظُرْ أَنَّىٰ يُؤْفَكُونَ ﴿٧٥﴾\\
\textamh{76.\  } & قُلْ أَتَعْبُدُونَ مِن دُونِ ٱللَّهِ مَا لَا يَمْلِكُ لَكُمْ ضَرًّۭا وَلَا نَفْعًۭا ۚ وَٱللَّهُ هُوَ ٱلسَّمِيعُ ٱلْعَلِيمُ ﴿٧٦﴾\\
\textamh{77.\  } & قُلْ يَـٰٓأَهْلَ ٱلْكِتَـٰبِ لَا تَغْلُوا۟ فِى دِينِكُمْ غَيْرَ ٱلْحَقِّ وَلَا تَتَّبِعُوٓا۟ أَهْوَآءَ قَوْمٍۢ قَدْ ضَلُّوا۟ مِن قَبْلُ وَأَضَلُّوا۟ كَثِيرًۭا وَضَلُّوا۟ عَن سَوَآءِ ٱلسَّبِيلِ ﴿٧٧﴾\\
\textamh{78.\  } & لُعِنَ ٱلَّذِينَ كَفَرُوا۟ مِنۢ بَنِىٓ إِسْرَٰٓءِيلَ عَلَىٰ لِسَانِ دَاوُۥدَ وَعِيسَى ٱبْنِ مَرْيَمَ ۚ ذَٟلِكَ بِمَا عَصَوا۟ وَّكَانُوا۟ يَعْتَدُونَ ﴿٧٨﴾\\
\textamh{79.\  } & كَانُوا۟ لَا يَتَنَاهَوْنَ عَن مُّنكَرٍۢ فَعَلُوهُ ۚ لَبِئْسَ مَا كَانُوا۟ يَفْعَلُونَ ﴿٧٩﴾\\
\textamh{80.\  } & تَرَىٰ كَثِيرًۭا مِّنْهُمْ يَتَوَلَّوْنَ ٱلَّذِينَ كَفَرُوا۟ ۚ لَبِئْسَ مَا قَدَّمَتْ لَهُمْ أَنفُسُهُمْ أَن سَخِطَ ٱللَّهُ عَلَيْهِمْ وَفِى ٱلْعَذَابِ هُمْ خَـٰلِدُونَ ﴿٨٠﴾\\
\textamh{81.\  } & وَلَوْ كَانُوا۟ يُؤْمِنُونَ بِٱللَّهِ وَٱلنَّبِىِّ وَمَآ أُنزِلَ إِلَيْهِ مَا ٱتَّخَذُوهُمْ أَوْلِيَآءَ وَلَـٰكِنَّ كَثِيرًۭا مِّنْهُمْ فَـٰسِقُونَ ﴿٨١﴾\\
\textamh{82.\  } & ۞ لَتَجِدَنَّ أَشَدَّ ٱلنَّاسِ عَدَٟوَةًۭ لِّلَّذِينَ ءَامَنُوا۟ ٱلْيَهُودَ وَٱلَّذِينَ أَشْرَكُوا۟ ۖ وَلَتَجِدَنَّ أَقْرَبَهُم مَّوَدَّةًۭ لِّلَّذِينَ ءَامَنُوا۟ ٱلَّذِينَ قَالُوٓا۟ إِنَّا نَصَـٰرَىٰ ۚ ذَٟلِكَ بِأَنَّ مِنْهُمْ قِسِّيسِينَ وَرُهْبَانًۭا وَأَنَّهُمْ لَا يَسْتَكْبِرُونَ ﴿٨٢﴾\\
\textamh{83.\  } & وَإِذَا سَمِعُوا۟ مَآ أُنزِلَ إِلَى ٱلرَّسُولِ تَرَىٰٓ أَعْيُنَهُمْ تَفِيضُ مِنَ ٱلدَّمْعِ مِمَّا عَرَفُوا۟ مِنَ ٱلْحَقِّ ۖ يَقُولُونَ رَبَّنَآ ءَامَنَّا فَٱكْتُبْنَا مَعَ ٱلشَّـٰهِدِينَ ﴿٨٣﴾\\
\textamh{84.\  } & وَمَا لَنَا لَا نُؤْمِنُ بِٱللَّهِ وَمَا جَآءَنَا مِنَ ٱلْحَقِّ وَنَطْمَعُ أَن يُدْخِلَنَا رَبُّنَا مَعَ ٱلْقَوْمِ ٱلصَّـٰلِحِينَ ﴿٨٤﴾\\
\textamh{85.\  } & فَأَثَـٰبَهُمُ ٱللَّهُ بِمَا قَالُوا۟ جَنَّـٰتٍۢ تَجْرِى مِن تَحْتِهَا ٱلْأَنْهَـٰرُ خَـٰلِدِينَ فِيهَا ۚ وَذَٟلِكَ جَزَآءُ ٱلْمُحْسِنِينَ ﴿٨٥﴾\\
\textamh{86.\  } & وَٱلَّذِينَ كَفَرُوا۟ وَكَذَّبُوا۟ بِـَٔايَـٰتِنَآ أُو۟لَـٰٓئِكَ أَصْحَـٰبُ ٱلْجَحِيمِ ﴿٨٦﴾\\
\textamh{87.\  } & يَـٰٓأَيُّهَا ٱلَّذِينَ ءَامَنُوا۟ لَا تُحَرِّمُوا۟ طَيِّبَٰتِ مَآ أَحَلَّ ٱللَّهُ لَكُمْ وَلَا تَعْتَدُوٓا۟ ۚ إِنَّ ٱللَّهَ لَا يُحِبُّ ٱلْمُعْتَدِينَ ﴿٨٧﴾\\
\textamh{88.\  } & وَكُلُوا۟ مِمَّا رَزَقَكُمُ ٱللَّهُ حَلَـٰلًۭا طَيِّبًۭا ۚ وَٱتَّقُوا۟ ٱللَّهَ ٱلَّذِىٓ أَنتُم بِهِۦ مُؤْمِنُونَ ﴿٨٨﴾\\
\textamh{89.\  } & لَا يُؤَاخِذُكُمُ ٱللَّهُ بِٱللَّغْوِ فِىٓ أَيْمَـٰنِكُمْ وَلَـٰكِن يُؤَاخِذُكُم بِمَا عَقَّدتُّمُ ٱلْأَيْمَـٰنَ ۖ فَكَفَّٰرَتُهُۥٓ إِطْعَامُ عَشَرَةِ مَسَـٰكِينَ مِنْ أَوْسَطِ مَا تُطْعِمُونَ أَهْلِيكُمْ أَوْ كِسْوَتُهُمْ أَوْ تَحْرِيرُ رَقَبَةٍۢ ۖ فَمَن لَّمْ يَجِدْ فَصِيَامُ ثَلَـٰثَةِ أَيَّامٍۢ ۚ ذَٟلِكَ كَفَّٰرَةُ أَيْمَـٰنِكُمْ إِذَا حَلَفْتُمْ ۚ وَٱحْفَظُوٓا۟ أَيْمَـٰنَكُمْ ۚ كَذَٟلِكَ يُبَيِّنُ ٱللَّهُ لَكُمْ ءَايَـٰتِهِۦ لَعَلَّكُمْ تَشْكُرُونَ ﴿٨٩﴾\\
\textamh{90.\  } & يَـٰٓأَيُّهَا ٱلَّذِينَ ءَامَنُوٓا۟ إِنَّمَا ٱلْخَمْرُ وَٱلْمَيْسِرُ وَٱلْأَنصَابُ وَٱلْأَزْلَـٰمُ رِجْسٌۭ مِّنْ عَمَلِ ٱلشَّيْطَٰنِ فَٱجْتَنِبُوهُ لَعَلَّكُمْ تُفْلِحُونَ ﴿٩٠﴾\\
\textamh{91.\  } & إِنَّمَا يُرِيدُ ٱلشَّيْطَٰنُ أَن يُوقِعَ بَيْنَكُمُ ٱلْعَدَٟوَةَ وَٱلْبَغْضَآءَ فِى ٱلْخَمْرِ وَٱلْمَيْسِرِ وَيَصُدَّكُمْ عَن ذِكْرِ ٱللَّهِ وَعَنِ ٱلصَّلَوٰةِ ۖ فَهَلْ أَنتُم مُّنتَهُونَ ﴿٩١﴾\\
\textamh{92.\  } & وَأَطِيعُوا۟ ٱللَّهَ وَأَطِيعُوا۟ ٱلرَّسُولَ وَٱحْذَرُوا۟ ۚ فَإِن تَوَلَّيْتُمْ فَٱعْلَمُوٓا۟ أَنَّمَا عَلَىٰ رَسُولِنَا ٱلْبَلَـٰغُ ٱلْمُبِينُ ﴿٩٢﴾\\
\textamh{93.\  } & لَيْسَ عَلَى ٱلَّذِينَ ءَامَنُوا۟ وَعَمِلُوا۟ ٱلصَّـٰلِحَـٰتِ جُنَاحٌۭ فِيمَا طَعِمُوٓا۟ إِذَا مَا ٱتَّقَوا۟ وَّءَامَنُوا۟ وَعَمِلُوا۟ ٱلصَّـٰلِحَـٰتِ ثُمَّ ٱتَّقَوا۟ وَّءَامَنُوا۟ ثُمَّ ٱتَّقَوا۟ وَّأَحْسَنُوا۟ ۗ وَٱللَّهُ يُحِبُّ ٱلْمُحْسِنِينَ ﴿٩٣﴾\\
\textamh{94.\  } & يَـٰٓأَيُّهَا ٱلَّذِينَ ءَامَنُوا۟ لَيَبْلُوَنَّكُمُ ٱللَّهُ بِشَىْءٍۢ مِّنَ ٱلصَّيْدِ تَنَالُهُۥٓ أَيْدِيكُمْ وَرِمَاحُكُمْ لِيَعْلَمَ ٱللَّهُ مَن يَخَافُهُۥ بِٱلْغَيْبِ ۚ فَمَنِ ٱعْتَدَىٰ بَعْدَ ذَٟلِكَ فَلَهُۥ عَذَابٌ أَلِيمٌۭ ﴿٩٤﴾\\
\textamh{95.\  } & يَـٰٓأَيُّهَا ٱلَّذِينَ ءَامَنُوا۟ لَا تَقْتُلُوا۟ ٱلصَّيْدَ وَأَنتُمْ حُرُمٌۭ ۚ وَمَن قَتَلَهُۥ مِنكُم مُّتَعَمِّدًۭا فَجَزَآءٌۭ مِّثْلُ مَا قَتَلَ مِنَ ٱلنَّعَمِ يَحْكُمُ بِهِۦ ذَوَا عَدْلٍۢ مِّنكُمْ هَدْيًۢا بَٰلِغَ ٱلْكَعْبَةِ أَوْ كَفَّٰرَةٌۭ طَعَامُ مَسَـٰكِينَ أَوْ عَدْلُ ذَٟلِكَ صِيَامًۭا لِّيَذُوقَ وَبَالَ أَمْرِهِۦ ۗ عَفَا ٱللَّهُ عَمَّا سَلَفَ ۚ وَمَنْ عَادَ فَيَنتَقِمُ ٱللَّهُ مِنْهُ ۗ وَٱللَّهُ عَزِيزٌۭ ذُو ٱنتِقَامٍ ﴿٩٥﴾\\
\textamh{96.\  } & أُحِلَّ لَكُمْ صَيْدُ ٱلْبَحْرِ وَطَعَامُهُۥ مَتَـٰعًۭا لَّكُمْ وَلِلسَّيَّارَةِ ۖ وَحُرِّمَ عَلَيْكُمْ صَيْدُ ٱلْبَرِّ مَا دُمْتُمْ حُرُمًۭا ۗ وَٱتَّقُوا۟ ٱللَّهَ ٱلَّذِىٓ إِلَيْهِ تُحْشَرُونَ ﴿٩٦﴾\\
\textamh{97.\  } & ۞ جَعَلَ ٱللَّهُ ٱلْكَعْبَةَ ٱلْبَيْتَ ٱلْحَرَامَ قِيَـٰمًۭا لِّلنَّاسِ وَٱلشَّهْرَ ٱلْحَرَامَ وَٱلْهَدْىَ وَٱلْقَلَـٰٓئِدَ ۚ ذَٟلِكَ لِتَعْلَمُوٓا۟ أَنَّ ٱللَّهَ يَعْلَمُ مَا فِى ٱلسَّمَـٰوَٟتِ وَمَا فِى ٱلْأَرْضِ وَأَنَّ ٱللَّهَ بِكُلِّ شَىْءٍ عَلِيمٌ ﴿٩٧﴾\\
\textamh{98.\  } & ٱعْلَمُوٓا۟ أَنَّ ٱللَّهَ شَدِيدُ ٱلْعِقَابِ وَأَنَّ ٱللَّهَ غَفُورٌۭ رَّحِيمٌۭ ﴿٩٨﴾\\
\textamh{99.\  } & مَّا عَلَى ٱلرَّسُولِ إِلَّا ٱلْبَلَـٰغُ ۗ وَٱللَّهُ يَعْلَمُ مَا تُبْدُونَ وَمَا تَكْتُمُونَ ﴿٩٩﴾\\
\textamh{100.\  } & قُل لَّا يَسْتَوِى ٱلْخَبِيثُ وَٱلطَّيِّبُ وَلَوْ أَعْجَبَكَ كَثْرَةُ ٱلْخَبِيثِ ۚ فَٱتَّقُوا۟ ٱللَّهَ يَـٰٓأُو۟لِى ٱلْأَلْبَٰبِ لَعَلَّكُمْ تُفْلِحُونَ ﴿١٠٠﴾\\
\textamh{101.\  } & يَـٰٓأَيُّهَا ٱلَّذِينَ ءَامَنُوا۟ لَا تَسْـَٔلُوا۟ عَنْ أَشْيَآءَ إِن تُبْدَ لَكُمْ تَسُؤْكُمْ وَإِن تَسْـَٔلُوا۟ عَنْهَا حِينَ يُنَزَّلُ ٱلْقُرْءَانُ تُبْدَ لَكُمْ عَفَا ٱللَّهُ عَنْهَا ۗ وَٱللَّهُ غَفُورٌ حَلِيمٌۭ ﴿١٠١﴾\\
\textamh{102.\  } & قَدْ سَأَلَهَا قَوْمٌۭ مِّن قَبْلِكُمْ ثُمَّ أَصْبَحُوا۟ بِهَا كَـٰفِرِينَ ﴿١٠٢﴾\\
\textamh{103.\  } & مَا جَعَلَ ٱللَّهُ مِنۢ بَحِيرَةٍۢ وَلَا سَآئِبَةٍۢ وَلَا وَصِيلَةٍۢ وَلَا حَامٍۢ ۙ وَلَـٰكِنَّ ٱلَّذِينَ كَفَرُوا۟ يَفْتَرُونَ عَلَى ٱللَّهِ ٱلْكَذِبَ ۖ وَأَكْثَرُهُمْ لَا يَعْقِلُونَ ﴿١٠٣﴾\\
\textamh{104.\  } & وَإِذَا قِيلَ لَهُمْ تَعَالَوْا۟ إِلَىٰ مَآ أَنزَلَ ٱللَّهُ وَإِلَى ٱلرَّسُولِ قَالُوا۟ حَسْبُنَا مَا وَجَدْنَا عَلَيْهِ ءَابَآءَنَآ ۚ أَوَلَوْ كَانَ ءَابَآؤُهُمْ لَا يَعْلَمُونَ شَيْـًۭٔا وَلَا يَهْتَدُونَ ﴿١٠٤﴾\\
\textamh{105.\  } & يَـٰٓأَيُّهَا ٱلَّذِينَ ءَامَنُوا۟ عَلَيْكُمْ أَنفُسَكُمْ ۖ لَا يَضُرُّكُم مَّن ضَلَّ إِذَا ٱهْتَدَيْتُمْ ۚ إِلَى ٱللَّهِ مَرْجِعُكُمْ جَمِيعًۭا فَيُنَبِّئُكُم بِمَا كُنتُمْ تَعْمَلُونَ ﴿١٠٥﴾\\
\textamh{106.\  } & يَـٰٓأَيُّهَا ٱلَّذِينَ ءَامَنُوا۟ شَهَـٰدَةُ بَيْنِكُمْ إِذَا حَضَرَ أَحَدَكُمُ ٱلْمَوْتُ حِينَ ٱلْوَصِيَّةِ ٱثْنَانِ ذَوَا عَدْلٍۢ مِّنكُمْ أَوْ ءَاخَرَانِ مِنْ غَيْرِكُمْ إِنْ أَنتُمْ ضَرَبْتُمْ فِى ٱلْأَرْضِ فَأَصَـٰبَتْكُم مُّصِيبَةُ ٱلْمَوْتِ ۚ تَحْبِسُونَهُمَا مِنۢ بَعْدِ ٱلصَّلَوٰةِ فَيُقْسِمَانِ بِٱللَّهِ إِنِ ٱرْتَبْتُمْ لَا نَشْتَرِى بِهِۦ ثَمَنًۭا وَلَوْ كَانَ ذَا قُرْبَىٰ ۙ وَلَا نَكْتُمُ شَهَـٰدَةَ ٱللَّهِ إِنَّآ إِذًۭا لَّمِنَ ٱلْءَاثِمِينَ ﴿١٠٦﴾\\
\textamh{107.\  } & فَإِنْ عُثِرَ عَلَىٰٓ أَنَّهُمَا ٱسْتَحَقَّآ إِثْمًۭا فَـَٔاخَرَانِ يَقُومَانِ مَقَامَهُمَا مِنَ ٱلَّذِينَ ٱسْتَحَقَّ عَلَيْهِمُ ٱلْأَوْلَيَـٰنِ فَيُقْسِمَانِ بِٱللَّهِ لَشَهَـٰدَتُنَآ أَحَقُّ مِن شَهَـٰدَتِهِمَا وَمَا ٱعْتَدَيْنَآ إِنَّآ إِذًۭا لَّمِنَ ٱلظَّـٰلِمِينَ ﴿١٠٧﴾\\
\textamh{108.\  } & ذَٟلِكَ أَدْنَىٰٓ أَن يَأْتُوا۟ بِٱلشَّهَـٰدَةِ عَلَىٰ وَجْهِهَآ أَوْ يَخَافُوٓا۟ أَن تُرَدَّ أَيْمَـٰنٌۢ بَعْدَ أَيْمَـٰنِهِمْ ۗ وَٱتَّقُوا۟ ٱللَّهَ وَٱسْمَعُوا۟ ۗ وَٱللَّهُ لَا يَهْدِى ٱلْقَوْمَ ٱلْفَـٰسِقِينَ ﴿١٠٨﴾\\
\textamh{109.\  } & ۞ يَوْمَ يَجْمَعُ ٱللَّهُ ٱلرُّسُلَ فَيَقُولُ مَاذَآ أُجِبْتُمْ ۖ قَالُوا۟ لَا عِلْمَ لَنَآ ۖ إِنَّكَ أَنتَ عَلَّٰمُ ٱلْغُيُوبِ ﴿١٠٩﴾\\
\textamh{110.\  } & إِذْ قَالَ ٱللَّهُ يَـٰعِيسَى ٱبْنَ مَرْيَمَ ٱذْكُرْ نِعْمَتِى عَلَيْكَ وَعَلَىٰ وَٟلِدَتِكَ إِذْ أَيَّدتُّكَ بِرُوحِ ٱلْقُدُسِ تُكَلِّمُ ٱلنَّاسَ فِى ٱلْمَهْدِ وَكَهْلًۭا ۖ وَإِذْ عَلَّمْتُكَ ٱلْكِتَـٰبَ وَٱلْحِكْمَةَ وَٱلتَّوْرَىٰةَ وَٱلْإِنجِيلَ ۖ وَإِذْ تَخْلُقُ مِنَ ٱلطِّينِ كَهَيْـَٔةِ ٱلطَّيْرِ بِإِذْنِى فَتَنفُخُ فِيهَا فَتَكُونُ طَيْرًۢا بِإِذْنِى ۖ وَتُبْرِئُ ٱلْأَكْمَهَ وَٱلْأَبْرَصَ بِإِذْنِى ۖ وَإِذْ تُخْرِجُ ٱلْمَوْتَىٰ بِإِذْنِى ۖ وَإِذْ كَفَفْتُ بَنِىٓ إِسْرَٰٓءِيلَ عَنكَ إِذْ جِئْتَهُم بِٱلْبَيِّنَـٰتِ فَقَالَ ٱلَّذِينَ كَفَرُوا۟ مِنْهُمْ إِنْ هَـٰذَآ إِلَّا سِحْرٌۭ مُّبِينٌۭ ﴿١١٠﴾\\
\textamh{111.\  } & وَإِذْ أَوْحَيْتُ إِلَى ٱلْحَوَارِيِّۦنَ أَنْ ءَامِنُوا۟ بِى وَبِرَسُولِى قَالُوٓا۟ ءَامَنَّا وَٱشْهَدْ بِأَنَّنَا مُسْلِمُونَ ﴿١١١﴾\\
\textamh{112.\  } & إِذْ قَالَ ٱلْحَوَارِيُّونَ يَـٰعِيسَى ٱبْنَ مَرْيَمَ هَلْ يَسْتَطِيعُ رَبُّكَ أَن يُنَزِّلَ عَلَيْنَا مَآئِدَةًۭ مِّنَ ٱلسَّمَآءِ ۖ قَالَ ٱتَّقُوا۟ ٱللَّهَ إِن كُنتُم مُّؤْمِنِينَ ﴿١١٢﴾\\
\textamh{113.\  } & قَالُوا۟ نُرِيدُ أَن نَّأْكُلَ مِنْهَا وَتَطْمَئِنَّ قُلُوبُنَا وَنَعْلَمَ أَن قَدْ صَدَقْتَنَا وَنَكُونَ عَلَيْهَا مِنَ ٱلشَّـٰهِدِينَ ﴿١١٣﴾\\
\textamh{114.\  } & قَالَ عِيسَى ٱبْنُ مَرْيَمَ ٱللَّهُمَّ رَبَّنَآ أَنزِلْ عَلَيْنَا مَآئِدَةًۭ مِّنَ ٱلسَّمَآءِ تَكُونُ لَنَا عِيدًۭا لِّأَوَّلِنَا وَءَاخِرِنَا وَءَايَةًۭ مِّنكَ ۖ وَٱرْزُقْنَا وَأَنتَ خَيْرُ ٱلرَّٟزِقِينَ ﴿١١٤﴾\\
\textamh{115.\  } & قَالَ ٱللَّهُ إِنِّى مُنَزِّلُهَا عَلَيْكُمْ ۖ فَمَن يَكْفُرْ بَعْدُ مِنكُمْ فَإِنِّىٓ أُعَذِّبُهُۥ عَذَابًۭا لَّآ أُعَذِّبُهُۥٓ أَحَدًۭا مِّنَ ٱلْعَـٰلَمِينَ ﴿١١٥﴾\\
\textamh{116.\  } & وَإِذْ قَالَ ٱللَّهُ يَـٰعِيسَى ٱبْنَ مَرْيَمَ ءَأَنتَ قُلْتَ لِلنَّاسِ ٱتَّخِذُونِى وَأُمِّىَ إِلَـٰهَيْنِ مِن دُونِ ٱللَّهِ ۖ قَالَ سُبْحَـٰنَكَ مَا يَكُونُ لِىٓ أَنْ أَقُولَ مَا لَيْسَ لِى بِحَقٍّ ۚ إِن كُنتُ قُلْتُهُۥ فَقَدْ عَلِمْتَهُۥ ۚ تَعْلَمُ مَا فِى نَفْسِى وَلَآ أَعْلَمُ مَا فِى نَفْسِكَ ۚ إِنَّكَ أَنتَ عَلَّٰمُ ٱلْغُيُوبِ ﴿١١٦﴾\\
\textamh{117.\  } & مَا قُلْتُ لَهُمْ إِلَّا مَآ أَمَرْتَنِى بِهِۦٓ أَنِ ٱعْبُدُوا۟ ٱللَّهَ رَبِّى وَرَبَّكُمْ ۚ وَكُنتُ عَلَيْهِمْ شَهِيدًۭا مَّا دُمْتُ فِيهِمْ ۖ فَلَمَّا تَوَفَّيْتَنِى كُنتَ أَنتَ ٱلرَّقِيبَ عَلَيْهِمْ ۚ وَأَنتَ عَلَىٰ كُلِّ شَىْءٍۢ شَهِيدٌ ﴿١١٧﴾\\
\textamh{118.\  } & إِن تُعَذِّبْهُمْ فَإِنَّهُمْ عِبَادُكَ ۖ وَإِن تَغْفِرْ لَهُمْ فَإِنَّكَ أَنتَ ٱلْعَزِيزُ ٱلْحَكِيمُ ﴿١١٨﴾\\
\textamh{119.\  } & قَالَ ٱللَّهُ هَـٰذَا يَوْمُ يَنفَعُ ٱلصَّـٰدِقِينَ صِدْقُهُمْ ۚ لَهُمْ جَنَّـٰتٌۭ تَجْرِى مِن تَحْتِهَا ٱلْأَنْهَـٰرُ خَـٰلِدِينَ فِيهَآ أَبَدًۭا ۚ رَّضِىَ ٱللَّهُ عَنْهُمْ وَرَضُوا۟ عَنْهُ ۚ ذَٟلِكَ ٱلْفَوْزُ ٱلْعَظِيمُ ﴿١١٩﴾\\
\textamh{120.\  } & لِلَّهِ مُلْكُ ٱلسَّمَـٰوَٟتِ وَٱلْأَرْضِ وَمَا فِيهِنَّ ۚ وَهُوَ عَلَىٰ كُلِّ شَىْءٍۢ قَدِيرٌۢ ﴿١٢٠﴾\\
\end{longtable}
\clearpage

%% License: BSD style (Berkley) (i.e. Put the Copyright owner's name always)
%% Writer and Copyright (to): Bewketu(Bilal) Tadilo (2016-17)
\centering\section{\LR{\textamharic{ሱራቱ አልአነኣም -}  \RL{سوره  الأنعام}}}
\begin{longtable}{%
  @{}
    p{.5\textwidth}
  @{~~~~~~~~~~~~~}
    p{.5\textwidth}
    @{}
}
\nopagebreak
\textamh{ቢስሚላሂ አራህመኒ ራሂይም } &  بِسْمِ ٱللَّهِ ٱلرَّحْمَـٰنِ ٱلرَّحِيمِ\\
\textamh{1.\  } &  ٱلْحَمْدُ لِلَّهِ ٱلَّذِى خَلَقَ ٱلسَّمَـٰوَٟتِ وَٱلْأَرْضَ وَجَعَلَ ٱلظُّلُمَـٰتِ وَٱلنُّورَ ۖ ثُمَّ ٱلَّذِينَ كَفَرُوا۟ بِرَبِّهِمْ يَعْدِلُونَ ﴿١﴾\\
\textamh{2.\  } & هُوَ ٱلَّذِى خَلَقَكُم مِّن طِينٍۢ ثُمَّ قَضَىٰٓ أَجَلًۭا ۖ وَأَجَلٌۭ مُّسَمًّى عِندَهُۥ ۖ ثُمَّ أَنتُمْ تَمْتَرُونَ ﴿٢﴾\\
\textamh{3.\  } & وَهُوَ ٱللَّهُ فِى ٱلسَّمَـٰوَٟتِ وَفِى ٱلْأَرْضِ ۖ يَعْلَمُ سِرَّكُمْ وَجَهْرَكُمْ وَيَعْلَمُ مَا تَكْسِبُونَ ﴿٣﴾\\
\textamh{4.\  } & وَمَا تَأْتِيهِم مِّنْ ءَايَةٍۢ مِّنْ ءَايَـٰتِ رَبِّهِمْ إِلَّا كَانُوا۟ عَنْهَا مُعْرِضِينَ ﴿٤﴾\\
\textamh{5.\  } & فَقَدْ كَذَّبُوا۟ بِٱلْحَقِّ لَمَّا جَآءَهُمْ ۖ فَسَوْفَ يَأْتِيهِمْ أَنۢبَٰٓؤُا۟ مَا كَانُوا۟ بِهِۦ يَسْتَهْزِءُونَ ﴿٥﴾\\
\textamh{6.\  } & أَلَمْ يَرَوْا۟ كَمْ أَهْلَكْنَا مِن قَبْلِهِم مِّن قَرْنٍۢ مَّكَّنَّـٰهُمْ فِى ٱلْأَرْضِ مَا لَمْ نُمَكِّن لَّكُمْ وَأَرْسَلْنَا ٱلسَّمَآءَ عَلَيْهِم مِّدْرَارًۭا وَجَعَلْنَا ٱلْأَنْهَـٰرَ تَجْرِى مِن تَحْتِهِمْ فَأَهْلَكْنَـٰهُم بِذُنُوبِهِمْ وَأَنشَأْنَا مِنۢ بَعْدِهِمْ قَرْنًا ءَاخَرِينَ ﴿٦﴾\\
\textamh{7.\  } & وَلَوْ نَزَّلْنَا عَلَيْكَ كِتَـٰبًۭا فِى قِرْطَاسٍۢ فَلَمَسُوهُ بِأَيْدِيهِمْ لَقَالَ ٱلَّذِينَ كَفَرُوٓا۟ إِنْ هَـٰذَآ إِلَّا سِحْرٌۭ مُّبِينٌۭ ﴿٧﴾\\
\textamh{8.\  } & وَقَالُوا۟ لَوْلَآ أُنزِلَ عَلَيْهِ مَلَكٌۭ ۖ وَلَوْ أَنزَلْنَا مَلَكًۭا لَّقُضِىَ ٱلْأَمْرُ ثُمَّ لَا يُنظَرُونَ ﴿٨﴾\\
\textamh{9.\  } & وَلَوْ جَعَلْنَـٰهُ مَلَكًۭا لَّجَعَلْنَـٰهُ رَجُلًۭا وَلَلَبَسْنَا عَلَيْهِم مَّا يَلْبِسُونَ ﴿٩﴾\\
\textamh{10.\  } & وَلَقَدِ ٱسْتُهْزِئَ بِرُسُلٍۢ مِّن قَبْلِكَ فَحَاقَ بِٱلَّذِينَ سَخِرُوا۟ مِنْهُم مَّا كَانُوا۟ بِهِۦ يَسْتَهْزِءُونَ ﴿١٠﴾\\
\textamh{11.\  } & قُلْ سِيرُوا۟ فِى ٱلْأَرْضِ ثُمَّ ٱنظُرُوا۟ كَيْفَ كَانَ عَـٰقِبَةُ ٱلْمُكَذِّبِينَ ﴿١١﴾\\
\textamh{12.\  } & قُل لِّمَن مَّا فِى ٱلسَّمَـٰوَٟتِ وَٱلْأَرْضِ ۖ قُل لِّلَّهِ ۚ كَتَبَ عَلَىٰ نَفْسِهِ ٱلرَّحْمَةَ ۚ لَيَجْمَعَنَّكُمْ إِلَىٰ يَوْمِ ٱلْقِيَـٰمَةِ لَا رَيْبَ فِيهِ ۚ ٱلَّذِينَ خَسِرُوٓا۟ أَنفُسَهُمْ فَهُمْ لَا يُؤْمِنُونَ ﴿١٢﴾\\
\textamh{13.\  } & ۞ وَلَهُۥ مَا سَكَنَ فِى ٱلَّيْلِ وَٱلنَّهَارِ ۚ وَهُوَ ٱلسَّمِيعُ ٱلْعَلِيمُ ﴿١٣﴾\\
\textamh{14.\  } & قُلْ أَغَيْرَ ٱللَّهِ أَتَّخِذُ وَلِيًّۭا فَاطِرِ ٱلسَّمَـٰوَٟتِ وَٱلْأَرْضِ وَهُوَ يُطْعِمُ وَلَا يُطْعَمُ ۗ قُلْ إِنِّىٓ أُمِرْتُ أَنْ أَكُونَ أَوَّلَ مَنْ أَسْلَمَ ۖ وَلَا تَكُونَنَّ مِنَ ٱلْمُشْرِكِينَ ﴿١٤﴾\\
\textamh{15.\  } & قُلْ إِنِّىٓ أَخَافُ إِنْ عَصَيْتُ رَبِّى عَذَابَ يَوْمٍ عَظِيمٍۢ ﴿١٥﴾\\
\textamh{16.\  } & مَّن يُصْرَفْ عَنْهُ يَوْمَئِذٍۢ فَقَدْ رَحِمَهُۥ ۚ وَذَٟلِكَ ٱلْفَوْزُ ٱلْمُبِينُ ﴿١٦﴾\\
\textamh{17.\  } & وَإِن يَمْسَسْكَ ٱللَّهُ بِضُرٍّۢ فَلَا كَاشِفَ لَهُۥٓ إِلَّا هُوَ ۖ وَإِن يَمْسَسْكَ بِخَيْرٍۢ فَهُوَ عَلَىٰ كُلِّ شَىْءٍۢ قَدِيرٌۭ ﴿١٧﴾\\
\textamh{18.\  } & وَهُوَ ٱلْقَاهِرُ فَوْقَ عِبَادِهِۦ ۚ وَهُوَ ٱلْحَكِيمُ ٱلْخَبِيرُ ﴿١٨﴾\\
\textamh{19.\  } & قُلْ أَىُّ شَىْءٍ أَكْبَرُ شَهَـٰدَةًۭ ۖ قُلِ ٱللَّهُ ۖ شَهِيدٌۢ بَيْنِى وَبَيْنَكُمْ ۚ وَأُوحِىَ إِلَىَّ هَـٰذَا ٱلْقُرْءَانُ لِأُنذِرَكُم بِهِۦ وَمَنۢ بَلَغَ ۚ أَئِنَّكُمْ لَتَشْهَدُونَ أَنَّ مَعَ ٱللَّهِ ءَالِهَةً أُخْرَىٰ ۚ قُل لَّآ أَشْهَدُ ۚ قُلْ إِنَّمَا هُوَ إِلَـٰهٌۭ وَٟحِدٌۭ وَإِنَّنِى بَرِىٓءٌۭ مِّمَّا تُشْرِكُونَ ﴿١٩﴾\\
\textamh{20.\  } & ٱلَّذِينَ ءَاتَيْنَـٰهُمُ ٱلْكِتَـٰبَ يَعْرِفُونَهُۥ كَمَا يَعْرِفُونَ أَبْنَآءَهُمُ ۘ ٱلَّذِينَ خَسِرُوٓا۟ أَنفُسَهُمْ فَهُمْ لَا يُؤْمِنُونَ ﴿٢٠﴾\\
\textamh{21.\  } & وَمَنْ أَظْلَمُ مِمَّنِ ٱفْتَرَىٰ عَلَى ٱللَّهِ كَذِبًا أَوْ كَذَّبَ بِـَٔايَـٰتِهِۦٓ ۗ إِنَّهُۥ لَا يُفْلِحُ ٱلظَّـٰلِمُونَ ﴿٢١﴾\\
\textamh{22.\  } & وَيَوْمَ نَحْشُرُهُمْ جَمِيعًۭا ثُمَّ نَقُولُ لِلَّذِينَ أَشْرَكُوٓا۟ أَيْنَ شُرَكَآؤُكُمُ ٱلَّذِينَ كُنتُمْ تَزْعُمُونَ ﴿٢٢﴾\\
\textamh{23.\  } & ثُمَّ لَمْ تَكُن فِتْنَتُهُمْ إِلَّآ أَن قَالُوا۟ وَٱللَّهِ رَبِّنَا مَا كُنَّا مُشْرِكِينَ ﴿٢٣﴾\\
\textamh{24.\  } & ٱنظُرْ كَيْفَ كَذَبُوا۟ عَلَىٰٓ أَنفُسِهِمْ ۚ وَضَلَّ عَنْهُم مَّا كَانُوا۟ يَفْتَرُونَ ﴿٢٤﴾\\
\textamh{25.\  } & وَمِنْهُم مَّن يَسْتَمِعُ إِلَيْكَ ۖ وَجَعَلْنَا عَلَىٰ قُلُوبِهِمْ أَكِنَّةً أَن يَفْقَهُوهُ وَفِىٓ ءَاذَانِهِمْ وَقْرًۭا ۚ وَإِن يَرَوْا۟ كُلَّ ءَايَةٍۢ لَّا يُؤْمِنُوا۟ بِهَا ۚ حَتَّىٰٓ إِذَا جَآءُوكَ يُجَٰدِلُونَكَ يَقُولُ ٱلَّذِينَ كَفَرُوٓا۟ إِنْ هَـٰذَآ إِلَّآ أَسَـٰطِيرُ ٱلْأَوَّلِينَ ﴿٢٥﴾\\
\textamh{26.\  } & وَهُمْ يَنْهَوْنَ عَنْهُ وَيَنْـَٔوْنَ عَنْهُ ۖ وَإِن يُهْلِكُونَ إِلَّآ أَنفُسَهُمْ وَمَا يَشْعُرُونَ ﴿٢٦﴾\\
\textamh{27.\  } & وَلَوْ تَرَىٰٓ إِذْ وُقِفُوا۟ عَلَى ٱلنَّارِ فَقَالُوا۟ يَـٰلَيْتَنَا نُرَدُّ وَلَا نُكَذِّبَ بِـَٔايَـٰتِ رَبِّنَا وَنَكُونَ مِنَ ٱلْمُؤْمِنِينَ ﴿٢٧﴾\\
\textamh{28.\  } & بَلْ بَدَا لَهُم مَّا كَانُوا۟ يُخْفُونَ مِن قَبْلُ ۖ وَلَوْ رُدُّوا۟ لَعَادُوا۟ لِمَا نُهُوا۟ عَنْهُ وَإِنَّهُمْ لَكَـٰذِبُونَ ﴿٢٨﴾\\
\textamh{29.\  } & وَقَالُوٓا۟ إِنْ هِىَ إِلَّا حَيَاتُنَا ٱلدُّنْيَا وَمَا نَحْنُ بِمَبْعُوثِينَ ﴿٢٩﴾\\
\textamh{30.\  } & وَلَوْ تَرَىٰٓ إِذْ وُقِفُوا۟ عَلَىٰ رَبِّهِمْ ۚ قَالَ أَلَيْسَ هَـٰذَا بِٱلْحَقِّ ۚ قَالُوا۟ بَلَىٰ وَرَبِّنَا ۚ قَالَ فَذُوقُوا۟ ٱلْعَذَابَ بِمَا كُنتُمْ تَكْفُرُونَ ﴿٣٠﴾\\
\textamh{31.\  } & قَدْ خَسِرَ ٱلَّذِينَ كَذَّبُوا۟ بِلِقَآءِ ٱللَّهِ ۖ حَتَّىٰٓ إِذَا جَآءَتْهُمُ ٱلسَّاعَةُ بَغْتَةًۭ قَالُوا۟ يَـٰحَسْرَتَنَا عَلَىٰ مَا فَرَّطْنَا فِيهَا وَهُمْ يَحْمِلُونَ أَوْزَارَهُمْ عَلَىٰ ظُهُورِهِمْ ۚ أَلَا سَآءَ مَا يَزِرُونَ ﴿٣١﴾\\
\textamh{32.\  } & وَمَا ٱلْحَيَوٰةُ ٱلدُّنْيَآ إِلَّا لَعِبٌۭ وَلَهْوٌۭ ۖ وَلَلدَّارُ ٱلْءَاخِرَةُ خَيْرٌۭ لِّلَّذِينَ يَتَّقُونَ ۗ أَفَلَا تَعْقِلُونَ ﴿٣٢﴾\\
\textamh{33.\  } & قَدْ نَعْلَمُ إِنَّهُۥ لَيَحْزُنُكَ ٱلَّذِى يَقُولُونَ ۖ فَإِنَّهُمْ لَا يُكَذِّبُونَكَ وَلَـٰكِنَّ ٱلظَّـٰلِمِينَ بِـَٔايَـٰتِ ٱللَّهِ يَجْحَدُونَ ﴿٣٣﴾\\
\textamh{34.\  } & وَلَقَدْ كُذِّبَتْ رُسُلٌۭ مِّن قَبْلِكَ فَصَبَرُوا۟ عَلَىٰ مَا كُذِّبُوا۟ وَأُوذُوا۟ حَتَّىٰٓ أَتَىٰهُمْ نَصْرُنَا ۚ وَلَا مُبَدِّلَ لِكَلِمَـٰتِ ٱللَّهِ ۚ وَلَقَدْ جَآءَكَ مِن نَّبَإِى۟ ٱلْمُرْسَلِينَ ﴿٣٤﴾\\
\textamh{35.\  } & وَإِن كَانَ كَبُرَ عَلَيْكَ إِعْرَاضُهُمْ فَإِنِ ٱسْتَطَعْتَ أَن تَبْتَغِىَ نَفَقًۭا فِى ٱلْأَرْضِ أَوْ سُلَّمًۭا فِى ٱلسَّمَآءِ فَتَأْتِيَهُم بِـَٔايَةٍۢ ۚ وَلَوْ شَآءَ ٱللَّهُ لَجَمَعَهُمْ عَلَى ٱلْهُدَىٰ ۚ فَلَا تَكُونَنَّ مِنَ ٱلْجَٰهِلِينَ ﴿٣٥﴾\\
\textamh{36.\  } & ۞ إِنَّمَا يَسْتَجِيبُ ٱلَّذِينَ يَسْمَعُونَ ۘ وَٱلْمَوْتَىٰ يَبْعَثُهُمُ ٱللَّهُ ثُمَّ إِلَيْهِ يُرْجَعُونَ ﴿٣٦﴾\\
\textamh{37.\  } & وَقَالُوا۟ لَوْلَا نُزِّلَ عَلَيْهِ ءَايَةٌۭ مِّن رَّبِّهِۦ ۚ قُلْ إِنَّ ٱللَّهَ قَادِرٌ عَلَىٰٓ أَن يُنَزِّلَ ءَايَةًۭ وَلَـٰكِنَّ أَكْثَرَهُمْ لَا يَعْلَمُونَ ﴿٣٧﴾\\
\textamh{38.\  } & وَمَا مِن دَآبَّةٍۢ فِى ٱلْأَرْضِ وَلَا طَٰٓئِرٍۢ يَطِيرُ بِجَنَاحَيْهِ إِلَّآ أُمَمٌ أَمْثَالُكُم ۚ مَّا فَرَّطْنَا فِى ٱلْكِتَـٰبِ مِن شَىْءٍۢ ۚ ثُمَّ إِلَىٰ رَبِّهِمْ يُحْشَرُونَ ﴿٣٨﴾\\
\textamh{39.\  } & وَٱلَّذِينَ كَذَّبُوا۟ بِـَٔايَـٰتِنَا صُمٌّۭ وَبُكْمٌۭ فِى ٱلظُّلُمَـٰتِ ۗ مَن يَشَإِ ٱللَّهُ يُضْلِلْهُ وَمَن يَشَأْ يَجْعَلْهُ عَلَىٰ صِرَٰطٍۢ مُّسْتَقِيمٍۢ ﴿٣٩﴾\\
\textamh{40.\  } & قُلْ أَرَءَيْتَكُمْ إِنْ أَتَىٰكُمْ عَذَابُ ٱللَّهِ أَوْ أَتَتْكُمُ ٱلسَّاعَةُ أَغَيْرَ ٱللَّهِ تَدْعُونَ إِن كُنتُمْ صَـٰدِقِينَ ﴿٤٠﴾\\
\textamh{41.\  } & بَلْ إِيَّاهُ تَدْعُونَ فَيَكْشِفُ مَا تَدْعُونَ إِلَيْهِ إِن شَآءَ وَتَنسَوْنَ مَا تُشْرِكُونَ ﴿٤١﴾\\
\textamh{42.\  } & وَلَقَدْ أَرْسَلْنَآ إِلَىٰٓ أُمَمٍۢ مِّن قَبْلِكَ فَأَخَذْنَـٰهُم بِٱلْبَأْسَآءِ وَٱلضَّرَّآءِ لَعَلَّهُمْ يَتَضَرَّعُونَ ﴿٤٢﴾\\
\textamh{43.\  } & فَلَوْلَآ إِذْ جَآءَهُم بَأْسُنَا تَضَرَّعُوا۟ وَلَـٰكِن قَسَتْ قُلُوبُهُمْ وَزَيَّنَ لَهُمُ ٱلشَّيْطَٰنُ مَا كَانُوا۟ يَعْمَلُونَ ﴿٤٣﴾\\
\textamh{44.\  } & فَلَمَّا نَسُوا۟ مَا ذُكِّرُوا۟ بِهِۦ فَتَحْنَا عَلَيْهِمْ أَبْوَٟبَ كُلِّ شَىْءٍ حَتَّىٰٓ إِذَا فَرِحُوا۟ بِمَآ أُوتُوٓا۟ أَخَذْنَـٰهُم بَغْتَةًۭ فَإِذَا هُم مُّبْلِسُونَ ﴿٤٤﴾\\
\textamh{45.\  } & فَقُطِعَ دَابِرُ ٱلْقَوْمِ ٱلَّذِينَ ظَلَمُوا۟ ۚ وَٱلْحَمْدُ لِلَّهِ رَبِّ ٱلْعَـٰلَمِينَ ﴿٤٥﴾\\
\textamh{46.\  } & قُلْ أَرَءَيْتُمْ إِنْ أَخَذَ ٱللَّهُ سَمْعَكُمْ وَأَبْصَـٰرَكُمْ وَخَتَمَ عَلَىٰ قُلُوبِكُم مَّنْ إِلَـٰهٌ غَيْرُ ٱللَّهِ يَأْتِيكُم بِهِ ۗ ٱنظُرْ كَيْفَ نُصَرِّفُ ٱلْءَايَـٰتِ ثُمَّ هُمْ يَصْدِفُونَ ﴿٤٦﴾\\
\textamh{47.\  } & قُلْ أَرَءَيْتَكُمْ إِنْ أَتَىٰكُمْ عَذَابُ ٱللَّهِ بَغْتَةً أَوْ جَهْرَةً هَلْ يُهْلَكُ إِلَّا ٱلْقَوْمُ ٱلظَّـٰلِمُونَ ﴿٤٧﴾\\
\textamh{48.\  } & وَمَا نُرْسِلُ ٱلْمُرْسَلِينَ إِلَّا مُبَشِّرِينَ وَمُنذِرِينَ ۖ فَمَنْ ءَامَنَ وَأَصْلَحَ فَلَا خَوْفٌ عَلَيْهِمْ وَلَا هُمْ يَحْزَنُونَ ﴿٤٨﴾\\
\textamh{49.\  } & وَٱلَّذِينَ كَذَّبُوا۟ بِـَٔايَـٰتِنَا يَمَسُّهُمُ ٱلْعَذَابُ بِمَا كَانُوا۟ يَفْسُقُونَ ﴿٤٩﴾\\
\textamh{50.\  } & قُل لَّآ أَقُولُ لَكُمْ عِندِى خَزَآئِنُ ٱللَّهِ وَلَآ أَعْلَمُ ٱلْغَيْبَ وَلَآ أَقُولُ لَكُمْ إِنِّى مَلَكٌ ۖ إِنْ أَتَّبِعُ إِلَّا مَا يُوحَىٰٓ إِلَىَّ ۚ قُلْ هَلْ يَسْتَوِى ٱلْأَعْمَىٰ وَٱلْبَصِيرُ ۚ أَفَلَا تَتَفَكَّرُونَ ﴿٥٠﴾\\
\textamh{51.\  } & وَأَنذِرْ بِهِ ٱلَّذِينَ يَخَافُونَ أَن يُحْشَرُوٓا۟ إِلَىٰ رَبِّهِمْ ۙ لَيْسَ لَهُم مِّن دُونِهِۦ وَلِىٌّۭ وَلَا شَفِيعٌۭ لَّعَلَّهُمْ يَتَّقُونَ ﴿٥١﴾\\
\textamh{52.\  } & وَلَا تَطْرُدِ ٱلَّذِينَ يَدْعُونَ رَبَّهُم بِٱلْغَدَوٰةِ وَٱلْعَشِىِّ يُرِيدُونَ وَجْهَهُۥ ۖ مَا عَلَيْكَ مِنْ حِسَابِهِم مِّن شَىْءٍۢ وَمَا مِنْ حِسَابِكَ عَلَيْهِم مِّن شَىْءٍۢ فَتَطْرُدَهُمْ فَتَكُونَ مِنَ ٱلظَّـٰلِمِينَ ﴿٥٢﴾\\
\textamh{53.\  } & وَكَذَٟلِكَ فَتَنَّا بَعْضَهُم بِبَعْضٍۢ لِّيَقُولُوٓا۟ أَهَـٰٓؤُلَآءِ مَنَّ ٱللَّهُ عَلَيْهِم مِّنۢ بَيْنِنَآ ۗ أَلَيْسَ ٱللَّهُ بِأَعْلَمَ بِٱلشَّـٰكِرِينَ ﴿٥٣﴾\\
\textamh{54.\  } & وَإِذَا جَآءَكَ ٱلَّذِينَ يُؤْمِنُونَ بِـَٔايَـٰتِنَا فَقُلْ سَلَـٰمٌ عَلَيْكُمْ ۖ كَتَبَ رَبُّكُمْ عَلَىٰ نَفْسِهِ ٱلرَّحْمَةَ ۖ أَنَّهُۥ مَنْ عَمِلَ مِنكُمْ سُوٓءًۢا بِجَهَـٰلَةٍۢ ثُمَّ تَابَ مِنۢ بَعْدِهِۦ وَأَصْلَحَ فَأَنَّهُۥ غَفُورٌۭ رَّحِيمٌۭ ﴿٥٤﴾\\
\textamh{55.\  } & وَكَذَٟلِكَ نُفَصِّلُ ٱلْءَايَـٰتِ وَلِتَسْتَبِينَ سَبِيلُ ٱلْمُجْرِمِينَ ﴿٥٥﴾\\
\textamh{56.\  } & قُلْ إِنِّى نُهِيتُ أَنْ أَعْبُدَ ٱلَّذِينَ تَدْعُونَ مِن دُونِ ٱللَّهِ ۚ قُل لَّآ أَتَّبِعُ أَهْوَآءَكُمْ ۙ قَدْ ضَلَلْتُ إِذًۭا وَمَآ أَنَا۠ مِنَ ٱلْمُهْتَدِينَ ﴿٥٦﴾\\
\textamh{57.\  } & قُلْ إِنِّى عَلَىٰ بَيِّنَةٍۢ مِّن رَّبِّى وَكَذَّبْتُم بِهِۦ ۚ مَا عِندِى مَا تَسْتَعْجِلُونَ بِهِۦٓ ۚ إِنِ ٱلْحُكْمُ إِلَّا لِلَّهِ ۖ يَقُصُّ ٱلْحَقَّ ۖ وَهُوَ خَيْرُ ٱلْفَـٰصِلِينَ ﴿٥٧﴾\\
\textamh{58.\  } & قُل لَّوْ أَنَّ عِندِى مَا تَسْتَعْجِلُونَ بِهِۦ لَقُضِىَ ٱلْأَمْرُ بَيْنِى وَبَيْنَكُمْ ۗ وَٱللَّهُ أَعْلَمُ بِٱلظَّـٰلِمِينَ ﴿٥٨﴾\\
\textamh{59.\  } & ۞ وَعِندَهُۥ مَفَاتِحُ ٱلْغَيْبِ لَا يَعْلَمُهَآ إِلَّا هُوَ ۚ وَيَعْلَمُ مَا فِى ٱلْبَرِّ وَٱلْبَحْرِ ۚ وَمَا تَسْقُطُ مِن وَرَقَةٍ إِلَّا يَعْلَمُهَا وَلَا حَبَّةٍۢ فِى ظُلُمَـٰتِ ٱلْأَرْضِ وَلَا رَطْبٍۢ وَلَا يَابِسٍ إِلَّا فِى كِتَـٰبٍۢ مُّبِينٍۢ ﴿٥٩﴾\\
\textamh{60.\  } & وَهُوَ ٱلَّذِى يَتَوَفَّىٰكُم بِٱلَّيْلِ وَيَعْلَمُ مَا جَرَحْتُم بِٱلنَّهَارِ ثُمَّ يَبْعَثُكُمْ فِيهِ لِيُقْضَىٰٓ أَجَلٌۭ مُّسَمًّۭى ۖ ثُمَّ إِلَيْهِ مَرْجِعُكُمْ ثُمَّ يُنَبِّئُكُم بِمَا كُنتُمْ تَعْمَلُونَ ﴿٦٠﴾\\
\textamh{61.\  } & وَهُوَ ٱلْقَاهِرُ فَوْقَ عِبَادِهِۦ ۖ وَيُرْسِلُ عَلَيْكُمْ حَفَظَةً حَتَّىٰٓ إِذَا جَآءَ أَحَدَكُمُ ٱلْمَوْتُ تَوَفَّتْهُ رُسُلُنَا وَهُمْ لَا يُفَرِّطُونَ ﴿٦١﴾\\
\textamh{62.\  } & ثُمَّ رُدُّوٓا۟ إِلَى ٱللَّهِ مَوْلَىٰهُمُ ٱلْحَقِّ ۚ أَلَا لَهُ ٱلْحُكْمُ وَهُوَ أَسْرَعُ ٱلْحَـٰسِبِينَ ﴿٦٢﴾\\
\textamh{63.\  } & قُلْ مَن يُنَجِّيكُم مِّن ظُلُمَـٰتِ ٱلْبَرِّ وَٱلْبَحْرِ تَدْعُونَهُۥ تَضَرُّعًۭا وَخُفْيَةًۭ لَّئِنْ أَنجَىٰنَا مِنْ هَـٰذِهِۦ لَنَكُونَنَّ مِنَ ٱلشَّـٰكِرِينَ ﴿٦٣﴾\\
\textamh{64.\  } & قُلِ ٱللَّهُ يُنَجِّيكُم مِّنْهَا وَمِن كُلِّ كَرْبٍۢ ثُمَّ أَنتُمْ تُشْرِكُونَ ﴿٦٤﴾\\
\textamh{65.\  } & قُلْ هُوَ ٱلْقَادِرُ عَلَىٰٓ أَن يَبْعَثَ عَلَيْكُمْ عَذَابًۭا مِّن فَوْقِكُمْ أَوْ مِن تَحْتِ أَرْجُلِكُمْ أَوْ يَلْبِسَكُمْ شِيَعًۭا وَيُذِيقَ بَعْضَكُم بَأْسَ بَعْضٍ ۗ ٱنظُرْ كَيْفَ نُصَرِّفُ ٱلْءَايَـٰتِ لَعَلَّهُمْ يَفْقَهُونَ ﴿٦٥﴾\\
\textamh{66.\  } & وَكَذَّبَ بِهِۦ قَوْمُكَ وَهُوَ ٱلْحَقُّ ۚ قُل لَّسْتُ عَلَيْكُم بِوَكِيلٍۢ ﴿٦٦﴾\\
\textamh{67.\  } & لِّكُلِّ نَبَإٍۢ مُّسْتَقَرٌّۭ ۚ وَسَوْفَ تَعْلَمُونَ ﴿٦٧﴾\\
\textamh{68.\  } & وَإِذَا رَأَيْتَ ٱلَّذِينَ يَخُوضُونَ فِىٓ ءَايَـٰتِنَا فَأَعْرِضْ عَنْهُمْ حَتَّىٰ يَخُوضُوا۟ فِى حَدِيثٍ غَيْرِهِۦ ۚ وَإِمَّا يُنسِيَنَّكَ ٱلشَّيْطَٰنُ فَلَا تَقْعُدْ بَعْدَ ٱلذِّكْرَىٰ مَعَ ٱلْقَوْمِ ٱلظَّـٰلِمِينَ ﴿٦٨﴾\\
\textamh{69.\  } & وَمَا عَلَى ٱلَّذِينَ يَتَّقُونَ مِنْ حِسَابِهِم مِّن شَىْءٍۢ وَلَـٰكِن ذِكْرَىٰ لَعَلَّهُمْ يَتَّقُونَ ﴿٦٩﴾\\
\textamh{70.\  } & وَذَرِ ٱلَّذِينَ ٱتَّخَذُوا۟ دِينَهُمْ لَعِبًۭا وَلَهْوًۭا وَغَرَّتْهُمُ ٱلْحَيَوٰةُ ٱلدُّنْيَا ۚ وَذَكِّرْ بِهِۦٓ أَن تُبْسَلَ نَفْسٌۢ بِمَا كَسَبَتْ لَيْسَ لَهَا مِن دُونِ ٱللَّهِ وَلِىٌّۭ وَلَا شَفِيعٌۭ وَإِن تَعْدِلْ كُلَّ عَدْلٍۢ لَّا يُؤْخَذْ مِنْهَآ ۗ أُو۟لَـٰٓئِكَ ٱلَّذِينَ أُبْسِلُوا۟ بِمَا كَسَبُوا۟ ۖ لَهُمْ شَرَابٌۭ مِّنْ حَمِيمٍۢ وَعَذَابٌ أَلِيمٌۢ بِمَا كَانُوا۟ يَكْفُرُونَ ﴿٧٠﴾\\
\textamh{71.\  } & قُلْ أَنَدْعُوا۟ مِن دُونِ ٱللَّهِ مَا لَا يَنفَعُنَا وَلَا يَضُرُّنَا وَنُرَدُّ عَلَىٰٓ أَعْقَابِنَا بَعْدَ إِذْ هَدَىٰنَا ٱللَّهُ كَٱلَّذِى ٱسْتَهْوَتْهُ ٱلشَّيَـٰطِينُ فِى ٱلْأَرْضِ حَيْرَانَ لَهُۥٓ أَصْحَـٰبٌۭ يَدْعُونَهُۥٓ إِلَى ٱلْهُدَى ٱئْتِنَا ۗ قُلْ إِنَّ هُدَى ٱللَّهِ هُوَ ٱلْهُدَىٰ ۖ وَأُمِرْنَا لِنُسْلِمَ لِرَبِّ ٱلْعَـٰلَمِينَ ﴿٧١﴾\\
\textamh{72.\  } & وَأَنْ أَقِيمُوا۟ ٱلصَّلَوٰةَ وَٱتَّقُوهُ ۚ وَهُوَ ٱلَّذِىٓ إِلَيْهِ تُحْشَرُونَ ﴿٧٢﴾\\
\textamh{73.\  } & وَهُوَ ٱلَّذِى خَلَقَ ٱلسَّمَـٰوَٟتِ وَٱلْأَرْضَ بِٱلْحَقِّ ۖ وَيَوْمَ يَقُولُ كُن فَيَكُونُ ۚ قَوْلُهُ ٱلْحَقُّ ۚ وَلَهُ ٱلْمُلْكُ يَوْمَ يُنفَخُ فِى ٱلصُّورِ ۚ عَـٰلِمُ ٱلْغَيْبِ وَٱلشَّهَـٰدَةِ ۚ وَهُوَ ٱلْحَكِيمُ ٱلْخَبِيرُ ﴿٧٣﴾\\
\textamh{74.\  } & ۞ وَإِذْ قَالَ إِبْرَٰهِيمُ لِأَبِيهِ ءَازَرَ أَتَتَّخِذُ أَصْنَامًا ءَالِهَةً ۖ إِنِّىٓ أَرَىٰكَ وَقَوْمَكَ فِى ضَلَـٰلٍۢ مُّبِينٍۢ ﴿٧٤﴾\\
\textamh{75.\  } & وَكَذَٟلِكَ نُرِىٓ إِبْرَٰهِيمَ مَلَكُوتَ ٱلسَّمَـٰوَٟتِ وَٱلْأَرْضِ وَلِيَكُونَ مِنَ ٱلْمُوقِنِينَ ﴿٧٥﴾\\
\textamh{76.\  } & فَلَمَّا جَنَّ عَلَيْهِ ٱلَّيْلُ رَءَا كَوْكَبًۭا ۖ قَالَ هَـٰذَا رَبِّى ۖ فَلَمَّآ أَفَلَ قَالَ لَآ أُحِبُّ ٱلْءَافِلِينَ ﴿٧٦﴾\\
\textamh{77.\  } & فَلَمَّا رَءَا ٱلْقَمَرَ بَازِغًۭا قَالَ هَـٰذَا رَبِّى ۖ فَلَمَّآ أَفَلَ قَالَ لَئِن لَّمْ يَهْدِنِى رَبِّى لَأَكُونَنَّ مِنَ ٱلْقَوْمِ ٱلضَّآلِّينَ ﴿٧٧﴾\\
\textamh{78.\  } & فَلَمَّا رَءَا ٱلشَّمْسَ بَازِغَةًۭ قَالَ هَـٰذَا رَبِّى هَـٰذَآ أَكْبَرُ ۖ فَلَمَّآ أَفَلَتْ قَالَ يَـٰقَوْمِ إِنِّى بَرِىٓءٌۭ مِّمَّا تُشْرِكُونَ ﴿٧٨﴾\\
\textamh{79.\  } & إِنِّى وَجَّهْتُ وَجْهِىَ لِلَّذِى فَطَرَ ٱلسَّمَـٰوَٟتِ وَٱلْأَرْضَ حَنِيفًۭا ۖ وَمَآ أَنَا۠ مِنَ ٱلْمُشْرِكِينَ ﴿٧٩﴾\\
\textamh{80.\  } & وَحَآجَّهُۥ قَوْمُهُۥ ۚ قَالَ أَتُحَـٰٓجُّوٓنِّى فِى ٱللَّهِ وَقَدْ هَدَىٰنِ ۚ وَلَآ أَخَافُ مَا تُشْرِكُونَ بِهِۦٓ إِلَّآ أَن يَشَآءَ رَبِّى شَيْـًۭٔا ۗ وَسِعَ رَبِّى كُلَّ شَىْءٍ عِلْمًا ۗ أَفَلَا تَتَذَكَّرُونَ ﴿٨٠﴾\\
\textamh{81.\  } & وَكَيْفَ أَخَافُ مَآ أَشْرَكْتُمْ وَلَا تَخَافُونَ أَنَّكُمْ أَشْرَكْتُم بِٱللَّهِ مَا لَمْ يُنَزِّلْ بِهِۦ عَلَيْكُمْ سُلْطَٰنًۭا ۚ فَأَىُّ ٱلْفَرِيقَيْنِ أَحَقُّ بِٱلْأَمْنِ ۖ إِن كُنتُمْ تَعْلَمُونَ ﴿٨١﴾\\
\textamh{82.\  } & ٱلَّذِينَ ءَامَنُوا۟ وَلَمْ يَلْبِسُوٓا۟ إِيمَـٰنَهُم بِظُلْمٍ أُو۟لَـٰٓئِكَ لَهُمُ ٱلْأَمْنُ وَهُم مُّهْتَدُونَ ﴿٨٢﴾\\
\textamh{83.\  } & وَتِلْكَ حُجَّتُنَآ ءَاتَيْنَـٰهَآ إِبْرَٰهِيمَ عَلَىٰ قَوْمِهِۦ ۚ نَرْفَعُ دَرَجَٰتٍۢ مَّن نَّشَآءُ ۗ إِنَّ رَبَّكَ حَكِيمٌ عَلِيمٌۭ ﴿٨٣﴾\\
\textamh{84.\  } & وَوَهَبْنَا لَهُۥٓ إِسْحَـٰقَ وَيَعْقُوبَ ۚ كُلًّا هَدَيْنَا ۚ وَنُوحًا هَدَيْنَا مِن قَبْلُ ۖ وَمِن ذُرِّيَّتِهِۦ دَاوُۥدَ وَسُلَيْمَـٰنَ وَأَيُّوبَ وَيُوسُفَ وَمُوسَىٰ وَهَـٰرُونَ ۚ وَكَذَٟلِكَ نَجْزِى ٱلْمُحْسِنِينَ ﴿٨٤﴾\\
\textamh{85.\  } & وَزَكَرِيَّا وَيَحْيَىٰ وَعِيسَىٰ وَإِلْيَاسَ ۖ كُلٌّۭ مِّنَ ٱلصَّـٰلِحِينَ ﴿٨٥﴾\\
\textamh{86.\  } & وَإِسْمَـٰعِيلَ وَٱلْيَسَعَ وَيُونُسَ وَلُوطًۭا ۚ وَكُلًّۭا فَضَّلْنَا عَلَى ٱلْعَـٰلَمِينَ ﴿٨٦﴾\\
\textamh{87.\  } & وَمِنْ ءَابَآئِهِمْ وَذُرِّيَّٰتِهِمْ وَإِخْوَٟنِهِمْ ۖ وَٱجْتَبَيْنَـٰهُمْ وَهَدَيْنَـٰهُمْ إِلَىٰ صِرَٰطٍۢ مُّسْتَقِيمٍۢ ﴿٨٧﴾\\
\textamh{88.\  } & ذَٟلِكَ هُدَى ٱللَّهِ يَهْدِى بِهِۦ مَن يَشَآءُ مِنْ عِبَادِهِۦ ۚ وَلَوْ أَشْرَكُوا۟ لَحَبِطَ عَنْهُم مَّا كَانُوا۟ يَعْمَلُونَ ﴿٨٨﴾\\
\textamh{89.\  } & أُو۟لَـٰٓئِكَ ٱلَّذِينَ ءَاتَيْنَـٰهُمُ ٱلْكِتَـٰبَ وَٱلْحُكْمَ وَٱلنُّبُوَّةَ ۚ فَإِن يَكْفُرْ بِهَا هَـٰٓؤُلَآءِ فَقَدْ وَكَّلْنَا بِهَا قَوْمًۭا لَّيْسُوا۟ بِهَا بِكَـٰفِرِينَ ﴿٨٩﴾\\
\textamh{90.\  } & أُو۟لَـٰٓئِكَ ٱلَّذِينَ هَدَى ٱللَّهُ ۖ فَبِهُدَىٰهُمُ ٱقْتَدِهْ ۗ قُل لَّآ أَسْـَٔلُكُمْ عَلَيْهِ أَجْرًا ۖ إِنْ هُوَ إِلَّا ذِكْرَىٰ لِلْعَـٰلَمِينَ ﴿٩٠﴾\\
\textamh{91.\  } & وَمَا قَدَرُوا۟ ٱللَّهَ حَقَّ قَدْرِهِۦٓ إِذْ قَالُوا۟ مَآ أَنزَلَ ٱللَّهُ عَلَىٰ بَشَرٍۢ مِّن شَىْءٍۢ ۗ قُلْ مَنْ أَنزَلَ ٱلْكِتَـٰبَ ٱلَّذِى جَآءَ بِهِۦ مُوسَىٰ نُورًۭا وَهُدًۭى لِّلنَّاسِ ۖ تَجْعَلُونَهُۥ قَرَاطِيسَ تُبْدُونَهَا وَتُخْفُونَ كَثِيرًۭا ۖ وَعُلِّمْتُم مَّا لَمْ تَعْلَمُوٓا۟ أَنتُمْ وَلَآ ءَابَآؤُكُمْ ۖ قُلِ ٱللَّهُ ۖ ثُمَّ ذَرْهُمْ فِى خَوْضِهِمْ يَلْعَبُونَ ﴿٩١﴾\\
\textamh{92.\  } & وَهَـٰذَا كِتَـٰبٌ أَنزَلْنَـٰهُ مُبَارَكٌۭ مُّصَدِّقُ ٱلَّذِى بَيْنَ يَدَيْهِ وَلِتُنذِرَ أُمَّ ٱلْقُرَىٰ وَمَنْ حَوْلَهَا ۚ وَٱلَّذِينَ يُؤْمِنُونَ بِٱلْءَاخِرَةِ يُؤْمِنُونَ بِهِۦ ۖ وَهُمْ عَلَىٰ صَلَاتِهِمْ يُحَافِظُونَ ﴿٩٢﴾\\
\textamh{93.\  } & وَمَنْ أَظْلَمُ مِمَّنِ ٱفْتَرَىٰ عَلَى ٱللَّهِ كَذِبًا أَوْ قَالَ أُوحِىَ إِلَىَّ وَلَمْ يُوحَ إِلَيْهِ شَىْءٌۭ وَمَن قَالَ سَأُنزِلُ مِثْلَ مَآ أَنزَلَ ٱللَّهُ ۗ وَلَوْ تَرَىٰٓ إِذِ ٱلظَّـٰلِمُونَ فِى غَمَرَٰتِ ٱلْمَوْتِ وَٱلْمَلَـٰٓئِكَةُ بَاسِطُوٓا۟ أَيْدِيهِمْ أَخْرِجُوٓا۟ أَنفُسَكُمُ ۖ ٱلْيَوْمَ تُجْزَوْنَ عَذَابَ ٱلْهُونِ بِمَا كُنتُمْ تَقُولُونَ عَلَى ٱللَّهِ غَيْرَ ٱلْحَقِّ وَكُنتُمْ عَنْ ءَايَـٰتِهِۦ تَسْتَكْبِرُونَ ﴿٩٣﴾\\
\textamh{94.\  } & وَلَقَدْ جِئْتُمُونَا فُرَٰدَىٰ كَمَا خَلَقْنَـٰكُمْ أَوَّلَ مَرَّةٍۢ وَتَرَكْتُم مَّا خَوَّلْنَـٰكُمْ وَرَآءَ ظُهُورِكُمْ ۖ وَمَا نَرَىٰ مَعَكُمْ شُفَعَآءَكُمُ ٱلَّذِينَ زَعَمْتُمْ أَنَّهُمْ فِيكُمْ شُرَكَـٰٓؤُا۟ ۚ لَقَد تَّقَطَّعَ بَيْنَكُمْ وَضَلَّ عَنكُم مَّا كُنتُمْ تَزْعُمُونَ ﴿٩٤﴾\\
\textamh{95.\  } & ۞ إِنَّ ٱللَّهَ فَالِقُ ٱلْحَبِّ وَٱلنَّوَىٰ ۖ يُخْرِجُ ٱلْحَىَّ مِنَ ٱلْمَيِّتِ وَمُخْرِجُ ٱلْمَيِّتِ مِنَ ٱلْحَىِّ ۚ ذَٟلِكُمُ ٱللَّهُ ۖ فَأَنَّىٰ تُؤْفَكُونَ ﴿٩٥﴾\\
\textamh{96.\  } & فَالِقُ ٱلْإِصْبَاحِ وَجَعَلَ ٱلَّيْلَ سَكَنًۭا وَٱلشَّمْسَ وَٱلْقَمَرَ حُسْبَانًۭا ۚ ذَٟلِكَ تَقْدِيرُ ٱلْعَزِيزِ ٱلْعَلِيمِ ﴿٩٦﴾\\
\textamh{97.\  } & وَهُوَ ٱلَّذِى جَعَلَ لَكُمُ ٱلنُّجُومَ لِتَهْتَدُوا۟ بِهَا فِى ظُلُمَـٰتِ ٱلْبَرِّ وَٱلْبَحْرِ ۗ قَدْ فَصَّلْنَا ٱلْءَايَـٰتِ لِقَوْمٍۢ يَعْلَمُونَ ﴿٩٧﴾\\
\textamh{98.\  } & وَهُوَ ٱلَّذِىٓ أَنشَأَكُم مِّن نَّفْسٍۢ وَٟحِدَةٍۢ فَمُسْتَقَرٌّۭ وَمُسْتَوْدَعٌۭ ۗ قَدْ فَصَّلْنَا ٱلْءَايَـٰتِ لِقَوْمٍۢ يَفْقَهُونَ ﴿٩٨﴾\\
\textamh{99.\  } & وَهُوَ ٱلَّذِىٓ أَنزَلَ مِنَ ٱلسَّمَآءِ مَآءًۭ فَأَخْرَجْنَا بِهِۦ نَبَاتَ كُلِّ شَىْءٍۢ فَأَخْرَجْنَا مِنْهُ خَضِرًۭا نُّخْرِجُ مِنْهُ حَبًّۭا مُّتَرَاكِبًۭا وَمِنَ ٱلنَّخْلِ مِن طَلْعِهَا قِنْوَانٌۭ دَانِيَةٌۭ وَجَنَّـٰتٍۢ مِّنْ أَعْنَابٍۢ وَٱلزَّيْتُونَ وَٱلرُّمَّانَ مُشْتَبِهًۭا وَغَيْرَ مُتَشَـٰبِهٍ ۗ ٱنظُرُوٓا۟ إِلَىٰ ثَمَرِهِۦٓ إِذَآ أَثْمَرَ وَيَنْعِهِۦٓ ۚ إِنَّ فِى ذَٟلِكُمْ لَءَايَـٰتٍۢ لِّقَوْمٍۢ يُؤْمِنُونَ ﴿٩٩﴾\\
\textamh{100.\  } & وَجَعَلُوا۟ لِلَّهِ شُرَكَآءَ ٱلْجِنَّ وَخَلَقَهُمْ ۖ وَخَرَقُوا۟ لَهُۥ بَنِينَ وَبَنَـٰتٍۭ بِغَيْرِ عِلْمٍۢ ۚ سُبْحَـٰنَهُۥ وَتَعَـٰلَىٰ عَمَّا يَصِفُونَ ﴿١٠٠﴾\\
\textamh{101.\  } & بَدِيعُ ٱلسَّمَـٰوَٟتِ وَٱلْأَرْضِ ۖ أَنَّىٰ يَكُونُ لَهُۥ وَلَدٌۭ وَلَمْ تَكُن لَّهُۥ صَـٰحِبَةٌۭ ۖ وَخَلَقَ كُلَّ شَىْءٍۢ ۖ وَهُوَ بِكُلِّ شَىْءٍ عَلِيمٌۭ ﴿١٠١﴾\\
\textamh{102.\  } & ذَٟلِكُمُ ٱللَّهُ رَبُّكُمْ ۖ لَآ إِلَـٰهَ إِلَّا هُوَ ۖ خَـٰلِقُ كُلِّ شَىْءٍۢ فَٱعْبُدُوهُ ۚ وَهُوَ عَلَىٰ كُلِّ شَىْءٍۢ وَكِيلٌۭ ﴿١٠٢﴾\\
\textamh{103.\  } & لَّا تُدْرِكُهُ ٱلْأَبْصَـٰرُ وَهُوَ يُدْرِكُ ٱلْأَبْصَـٰرَ ۖ وَهُوَ ٱللَّطِيفُ ٱلْخَبِيرُ ﴿١٠٣﴾\\
\textamh{104.\  } & قَدْ جَآءَكُم بَصَآئِرُ مِن رَّبِّكُمْ ۖ فَمَنْ أَبْصَرَ فَلِنَفْسِهِۦ ۖ وَمَنْ عَمِىَ فَعَلَيْهَا ۚ وَمَآ أَنَا۠ عَلَيْكُم بِحَفِيظٍۢ ﴿١٠٤﴾\\
\textamh{105.\  } & وَكَذَٟلِكَ نُصَرِّفُ ٱلْءَايَـٰتِ وَلِيَقُولُوا۟ دَرَسْتَ وَلِنُبَيِّنَهُۥ لِقَوْمٍۢ يَعْلَمُونَ ﴿١٠٥﴾\\
\textamh{106.\  } & ٱتَّبِعْ مَآ أُوحِىَ إِلَيْكَ مِن رَّبِّكَ ۖ لَآ إِلَـٰهَ إِلَّا هُوَ ۖ وَأَعْرِضْ عَنِ ٱلْمُشْرِكِينَ ﴿١٠٦﴾\\
\textamh{107.\  } & وَلَوْ شَآءَ ٱللَّهُ مَآ أَشْرَكُوا۟ ۗ وَمَا جَعَلْنَـٰكَ عَلَيْهِمْ حَفِيظًۭا ۖ وَمَآ أَنتَ عَلَيْهِم بِوَكِيلٍۢ ﴿١٠٧﴾\\
\textamh{108.\  } & وَلَا تَسُبُّوا۟ ٱلَّذِينَ يَدْعُونَ مِن دُونِ ٱللَّهِ فَيَسُبُّوا۟ ٱللَّهَ عَدْوًۢا بِغَيْرِ عِلْمٍۢ ۗ كَذَٟلِكَ زَيَّنَّا لِكُلِّ أُمَّةٍ عَمَلَهُمْ ثُمَّ إِلَىٰ رَبِّهِم مَّرْجِعُهُمْ فَيُنَبِّئُهُم بِمَا كَانُوا۟ يَعْمَلُونَ ﴿١٠٨﴾\\
\textamh{109.\  } & وَأَقْسَمُوا۟ بِٱللَّهِ جَهْدَ أَيْمَـٰنِهِمْ لَئِن جَآءَتْهُمْ ءَايَةٌۭ لَّيُؤْمِنُنَّ بِهَا ۚ قُلْ إِنَّمَا ٱلْءَايَـٰتُ عِندَ ٱللَّهِ ۖ وَمَا يُشْعِرُكُمْ أَنَّهَآ إِذَا جَآءَتْ لَا يُؤْمِنُونَ ﴿١٠٩﴾\\
\textamh{110.\  } & وَنُقَلِّبُ أَفْـِٔدَتَهُمْ وَأَبْصَـٰرَهُمْ كَمَا لَمْ يُؤْمِنُوا۟ بِهِۦٓ أَوَّلَ مَرَّةٍۢ وَنَذَرُهُمْ فِى طُغْيَـٰنِهِمْ يَعْمَهُونَ ﴿١١٠﴾\\
\textamh{111.\  } & ۞ وَلَوْ أَنَّنَا نَزَّلْنَآ إِلَيْهِمُ ٱلْمَلَـٰٓئِكَةَ وَكَلَّمَهُمُ ٱلْمَوْتَىٰ وَحَشَرْنَا عَلَيْهِمْ كُلَّ شَىْءٍۢ قُبُلًۭا مَّا كَانُوا۟ لِيُؤْمِنُوٓا۟ إِلَّآ أَن يَشَآءَ ٱللَّهُ وَلَـٰكِنَّ أَكْثَرَهُمْ يَجْهَلُونَ ﴿١١١﴾\\
\textamh{112.\  } & وَكَذَٟلِكَ جَعَلْنَا لِكُلِّ نَبِىٍّ عَدُوًّۭا شَيَـٰطِينَ ٱلْإِنسِ وَٱلْجِنِّ يُوحِى بَعْضُهُمْ إِلَىٰ بَعْضٍۢ زُخْرُفَ ٱلْقَوْلِ غُرُورًۭا ۚ وَلَوْ شَآءَ رَبُّكَ مَا فَعَلُوهُ ۖ فَذَرْهُمْ وَمَا يَفْتَرُونَ ﴿١١٢﴾\\
\textamh{113.\  } & وَلِتَصْغَىٰٓ إِلَيْهِ أَفْـِٔدَةُ ٱلَّذِينَ لَا يُؤْمِنُونَ بِٱلْءَاخِرَةِ وَلِيَرْضَوْهُ وَلِيَقْتَرِفُوا۟ مَا هُم مُّقْتَرِفُونَ ﴿١١٣﴾\\
\textamh{114.\  } & أَفَغَيْرَ ٱللَّهِ أَبْتَغِى حَكَمًۭا وَهُوَ ٱلَّذِىٓ أَنزَلَ إِلَيْكُمُ ٱلْكِتَـٰبَ مُفَصَّلًۭا ۚ وَٱلَّذِينَ ءَاتَيْنَـٰهُمُ ٱلْكِتَـٰبَ يَعْلَمُونَ أَنَّهُۥ مُنَزَّلٌۭ مِّن رَّبِّكَ بِٱلْحَقِّ ۖ فَلَا تَكُونَنَّ مِنَ ٱلْمُمْتَرِينَ ﴿١١٤﴾\\
\textamh{115.\  } & وَتَمَّتْ كَلِمَتُ رَبِّكَ صِدْقًۭا وَعَدْلًۭا ۚ لَّا مُبَدِّلَ لِكَلِمَـٰتِهِۦ ۚ وَهُوَ ٱلسَّمِيعُ ٱلْعَلِيمُ ﴿١١٥﴾\\
\textamh{116.\  } & وَإِن تُطِعْ أَكْثَرَ مَن فِى ٱلْأَرْضِ يُضِلُّوكَ عَن سَبِيلِ ٱللَّهِ ۚ إِن يَتَّبِعُونَ إِلَّا ٱلظَّنَّ وَإِنْ هُمْ إِلَّا يَخْرُصُونَ ﴿١١٦﴾\\
\textamh{117.\  } & إِنَّ رَبَّكَ هُوَ أَعْلَمُ مَن يَضِلُّ عَن سَبِيلِهِۦ ۖ وَهُوَ أَعْلَمُ بِٱلْمُهْتَدِينَ ﴿١١٧﴾\\
\textamh{118.\  } & فَكُلُوا۟ مِمَّا ذُكِرَ ٱسْمُ ٱللَّهِ عَلَيْهِ إِن كُنتُم بِـَٔايَـٰتِهِۦ مُؤْمِنِينَ ﴿١١٨﴾\\
\textamh{119.\  } & وَمَا لَكُمْ أَلَّا تَأْكُلُوا۟ مِمَّا ذُكِرَ ٱسْمُ ٱللَّهِ عَلَيْهِ وَقَدْ فَصَّلَ لَكُم مَّا حَرَّمَ عَلَيْكُمْ إِلَّا مَا ٱضْطُرِرْتُمْ إِلَيْهِ ۗ وَإِنَّ كَثِيرًۭا لَّيُضِلُّونَ بِأَهْوَآئِهِم بِغَيْرِ عِلْمٍ ۗ إِنَّ رَبَّكَ هُوَ أَعْلَمُ بِٱلْمُعْتَدِينَ ﴿١١٩﴾\\
\textamh{120.\  } & وَذَرُوا۟ ظَـٰهِرَ ٱلْإِثْمِ وَبَاطِنَهُۥٓ ۚ إِنَّ ٱلَّذِينَ يَكْسِبُونَ ٱلْإِثْمَ سَيُجْزَوْنَ بِمَا كَانُوا۟ يَقْتَرِفُونَ ﴿١٢٠﴾\\
\textamh{121.\  } & وَلَا تَأْكُلُوا۟ مِمَّا لَمْ يُذْكَرِ ٱسْمُ ٱللَّهِ عَلَيْهِ وَإِنَّهُۥ لَفِسْقٌۭ ۗ وَإِنَّ ٱلشَّيَـٰطِينَ لَيُوحُونَ إِلَىٰٓ أَوْلِيَآئِهِمْ لِيُجَٰدِلُوكُمْ ۖ وَإِنْ أَطَعْتُمُوهُمْ إِنَّكُمْ لَمُشْرِكُونَ ﴿١٢١﴾\\
\textamh{122.\  } & أَوَمَن كَانَ مَيْتًۭا فَأَحْيَيْنَـٰهُ وَجَعَلْنَا لَهُۥ نُورًۭا يَمْشِى بِهِۦ فِى ٱلنَّاسِ كَمَن مَّثَلُهُۥ فِى ٱلظُّلُمَـٰتِ لَيْسَ بِخَارِجٍۢ مِّنْهَا ۚ كَذَٟلِكَ زُيِّنَ لِلْكَـٰفِرِينَ مَا كَانُوا۟ يَعْمَلُونَ ﴿١٢٢﴾\\
\textamh{123.\  } & وَكَذَٟلِكَ جَعَلْنَا فِى كُلِّ قَرْيَةٍ أَكَـٰبِرَ مُجْرِمِيهَا لِيَمْكُرُوا۟ فِيهَا ۖ وَمَا يَمْكُرُونَ إِلَّا بِأَنفُسِهِمْ وَمَا يَشْعُرُونَ ﴿١٢٣﴾\\
\textamh{124.\  } & وَإِذَا جَآءَتْهُمْ ءَايَةٌۭ قَالُوا۟ لَن نُّؤْمِنَ حَتَّىٰ نُؤْتَىٰ مِثْلَ مَآ أُوتِىَ رُسُلُ ٱللَّهِ ۘ ٱللَّهُ أَعْلَمُ حَيْثُ يَجْعَلُ رِسَالَتَهُۥ ۗ سَيُصِيبُ ٱلَّذِينَ أَجْرَمُوا۟ صَغَارٌ عِندَ ٱللَّهِ وَعَذَابٌۭ شَدِيدٌۢ بِمَا كَانُوا۟ يَمْكُرُونَ ﴿١٢٤﴾\\
\textamh{125.\  } & فَمَن يُرِدِ ٱللَّهُ أَن يَهْدِيَهُۥ يَشْرَحْ صَدْرَهُۥ لِلْإِسْلَـٰمِ ۖ وَمَن يُرِدْ أَن يُضِلَّهُۥ يَجْعَلْ صَدْرَهُۥ ضَيِّقًا حَرَجًۭا كَأَنَّمَا يَصَّعَّدُ فِى ٱلسَّمَآءِ ۚ كَذَٟلِكَ يَجْعَلُ ٱللَّهُ ٱلرِّجْسَ عَلَى ٱلَّذِينَ لَا يُؤْمِنُونَ ﴿١٢٥﴾\\
\textamh{126.\  } & وَهَـٰذَا صِرَٰطُ رَبِّكَ مُسْتَقِيمًۭا ۗ قَدْ فَصَّلْنَا ٱلْءَايَـٰتِ لِقَوْمٍۢ يَذَّكَّرُونَ ﴿١٢٦﴾\\
\textamh{127.\  } & ۞ لَهُمْ دَارُ ٱلسَّلَـٰمِ عِندَ رَبِّهِمْ ۖ وَهُوَ وَلِيُّهُم بِمَا كَانُوا۟ يَعْمَلُونَ ﴿١٢٧﴾\\
\textamh{128.\  } & وَيَوْمَ يَحْشُرُهُمْ جَمِيعًۭا يَـٰمَعْشَرَ ٱلْجِنِّ قَدِ ٱسْتَكْثَرْتُم مِّنَ ٱلْإِنسِ ۖ وَقَالَ أَوْلِيَآؤُهُم مِّنَ ٱلْإِنسِ رَبَّنَا ٱسْتَمْتَعَ بَعْضُنَا بِبَعْضٍۢ وَبَلَغْنَآ أَجَلَنَا ٱلَّذِىٓ أَجَّلْتَ لَنَا ۚ قَالَ ٱلنَّارُ مَثْوَىٰكُمْ خَـٰلِدِينَ فِيهَآ إِلَّا مَا شَآءَ ٱللَّهُ ۗ إِنَّ رَبَّكَ حَكِيمٌ عَلِيمٌۭ ﴿١٢٨﴾\\
\textamh{129.\  } & وَكَذَٟلِكَ نُوَلِّى بَعْضَ ٱلظَّـٰلِمِينَ بَعْضًۢا بِمَا كَانُوا۟ يَكْسِبُونَ ﴿١٢٩﴾\\
\textamh{130.\  } & يَـٰمَعْشَرَ ٱلْجِنِّ وَٱلْإِنسِ أَلَمْ يَأْتِكُمْ رُسُلٌۭ مِّنكُمْ يَقُصُّونَ عَلَيْكُمْ ءَايَـٰتِى وَيُنذِرُونَكُمْ لِقَآءَ يَوْمِكُمْ هَـٰذَا ۚ قَالُوا۟ شَهِدْنَا عَلَىٰٓ أَنفُسِنَا ۖ وَغَرَّتْهُمُ ٱلْحَيَوٰةُ ٱلدُّنْيَا وَشَهِدُوا۟ عَلَىٰٓ أَنفُسِهِمْ أَنَّهُمْ كَانُوا۟ كَـٰفِرِينَ ﴿١٣٠﴾\\
\textamh{131.\  } & ذَٟلِكَ أَن لَّمْ يَكُن رَّبُّكَ مُهْلِكَ ٱلْقُرَىٰ بِظُلْمٍۢ وَأَهْلُهَا غَٰفِلُونَ ﴿١٣١﴾\\
\textamh{132.\  } & وَلِكُلٍّۢ دَرَجَٰتٌۭ مِّمَّا عَمِلُوا۟ ۚ وَمَا رَبُّكَ بِغَٰفِلٍ عَمَّا يَعْمَلُونَ ﴿١٣٢﴾\\
\textamh{133.\  } & وَرَبُّكَ ٱلْغَنِىُّ ذُو ٱلرَّحْمَةِ ۚ إِن يَشَأْ يُذْهِبْكُمْ وَيَسْتَخْلِفْ مِنۢ بَعْدِكُم مَّا يَشَآءُ كَمَآ أَنشَأَكُم مِّن ذُرِّيَّةِ قَوْمٍ ءَاخَرِينَ ﴿١٣٣﴾\\
\textamh{134.\  } & إِنَّ مَا تُوعَدُونَ لَءَاتٍۢ ۖ وَمَآ أَنتُم بِمُعْجِزِينَ ﴿١٣٤﴾\\
\textamh{135.\  } & قُلْ يَـٰقَوْمِ ٱعْمَلُوا۟ عَلَىٰ مَكَانَتِكُمْ إِنِّى عَامِلٌۭ ۖ فَسَوْفَ تَعْلَمُونَ مَن تَكُونُ لَهُۥ عَـٰقِبَةُ ٱلدَّارِ ۗ إِنَّهُۥ لَا يُفْلِحُ ٱلظَّـٰلِمُونَ ﴿١٣٥﴾\\
\textamh{136.\  } & وَجَعَلُوا۟ لِلَّهِ مِمَّا ذَرَأَ مِنَ ٱلْحَرْثِ وَٱلْأَنْعَـٰمِ نَصِيبًۭا فَقَالُوا۟ هَـٰذَا لِلَّهِ بِزَعْمِهِمْ وَهَـٰذَا لِشُرَكَآئِنَا ۖ فَمَا كَانَ لِشُرَكَآئِهِمْ فَلَا يَصِلُ إِلَى ٱللَّهِ ۖ وَمَا كَانَ لِلَّهِ فَهُوَ يَصِلُ إِلَىٰ شُرَكَآئِهِمْ ۗ سَآءَ مَا يَحْكُمُونَ ﴿١٣٦﴾\\
\textamh{137.\  } & وَكَذَٟلِكَ زَيَّنَ لِكَثِيرٍۢ مِّنَ ٱلْمُشْرِكِينَ قَتْلَ أَوْلَـٰدِهِمْ شُرَكَآؤُهُمْ لِيُرْدُوهُمْ وَلِيَلْبِسُوا۟ عَلَيْهِمْ دِينَهُمْ ۖ وَلَوْ شَآءَ ٱللَّهُ مَا فَعَلُوهُ ۖ فَذَرْهُمْ وَمَا يَفْتَرُونَ ﴿١٣٧﴾\\
\textamh{138.\  } & وَقَالُوا۟ هَـٰذِهِۦٓ أَنْعَـٰمٌۭ وَحَرْثٌ حِجْرٌۭ لَّا يَطْعَمُهَآ إِلَّا مَن نَّشَآءُ بِزَعْمِهِمْ وَأَنْعَـٰمٌ حُرِّمَتْ ظُهُورُهَا وَأَنْعَـٰمٌۭ لَّا يَذْكُرُونَ ٱسْمَ ٱللَّهِ عَلَيْهَا ٱفْتِرَآءً عَلَيْهِ ۚ سَيَجْزِيهِم بِمَا كَانُوا۟ يَفْتَرُونَ ﴿١٣٨﴾\\
\textamh{139.\  } & وَقَالُوا۟ مَا فِى بُطُونِ هَـٰذِهِ ٱلْأَنْعَـٰمِ خَالِصَةٌۭ لِّذُكُورِنَا وَمُحَرَّمٌ عَلَىٰٓ أَزْوَٟجِنَا ۖ وَإِن يَكُن مَّيْتَةًۭ فَهُمْ فِيهِ شُرَكَآءُ ۚ سَيَجْزِيهِمْ وَصْفَهُمْ ۚ إِنَّهُۥ حَكِيمٌ عَلِيمٌۭ ﴿١٣٩﴾\\
\textamh{140.\  } & قَدْ خَسِرَ ٱلَّذِينَ قَتَلُوٓا۟ أَوْلَـٰدَهُمْ سَفَهًۢا بِغَيْرِ عِلْمٍۢ وَحَرَّمُوا۟ مَا رَزَقَهُمُ ٱللَّهُ ٱفْتِرَآءً عَلَى ٱللَّهِ ۚ قَدْ ضَلُّوا۟ وَمَا كَانُوا۟ مُهْتَدِينَ ﴿١٤٠﴾\\
\textamh{141.\  } & ۞ وَهُوَ ٱلَّذِىٓ أَنشَأَ جَنَّـٰتٍۢ مَّعْرُوشَـٰتٍۢ وَغَيْرَ مَعْرُوشَـٰتٍۢ وَٱلنَّخْلَ وَٱلزَّرْعَ مُخْتَلِفًا أُكُلُهُۥ وَٱلزَّيْتُونَ وَٱلرُّمَّانَ مُتَشَـٰبِهًۭا وَغَيْرَ مُتَشَـٰبِهٍۢ ۚ كُلُوا۟ مِن ثَمَرِهِۦٓ إِذَآ أَثْمَرَ وَءَاتُوا۟ حَقَّهُۥ يَوْمَ حَصَادِهِۦ ۖ وَلَا تُسْرِفُوٓا۟ ۚ إِنَّهُۥ لَا يُحِبُّ ٱلْمُسْرِفِينَ ﴿١٤١﴾\\
\textamh{142.\  } & وَمِنَ ٱلْأَنْعَـٰمِ حَمُولَةًۭ وَفَرْشًۭا ۚ كُلُوا۟ مِمَّا رَزَقَكُمُ ٱللَّهُ وَلَا تَتَّبِعُوا۟ خُطُوَٟتِ ٱلشَّيْطَٰنِ ۚ إِنَّهُۥ لَكُمْ عَدُوٌّۭ مُّبِينٌۭ ﴿١٤٢﴾\\
\textamh{143.\  } & ثَمَـٰنِيَةَ أَزْوَٟجٍۢ ۖ مِّنَ ٱلضَّأْنِ ٱثْنَيْنِ وَمِنَ ٱلْمَعْزِ ٱثْنَيْنِ ۗ قُلْ ءَآلذَّكَرَيْنِ حَرَّمَ أَمِ ٱلْأُنثَيَيْنِ أَمَّا ٱشْتَمَلَتْ عَلَيْهِ أَرْحَامُ ٱلْأُنثَيَيْنِ ۖ نَبِّـُٔونِى بِعِلْمٍ إِن كُنتُمْ صَـٰدِقِينَ ﴿١٤٣﴾\\
\textamh{144.\  } & وَمِنَ ٱلْإِبِلِ ٱثْنَيْنِ وَمِنَ ٱلْبَقَرِ ٱثْنَيْنِ ۗ قُلْ ءَآلذَّكَرَيْنِ حَرَّمَ أَمِ ٱلْأُنثَيَيْنِ أَمَّا ٱشْتَمَلَتْ عَلَيْهِ أَرْحَامُ ٱلْأُنثَيَيْنِ ۖ أَمْ كُنتُمْ شُهَدَآءَ إِذْ وَصَّىٰكُمُ ٱللَّهُ بِهَـٰذَا ۚ فَمَنْ أَظْلَمُ مِمَّنِ ٱفْتَرَىٰ عَلَى ٱللَّهِ كَذِبًۭا لِّيُضِلَّ ٱلنَّاسَ بِغَيْرِ عِلْمٍ ۗ إِنَّ ٱللَّهَ لَا يَهْدِى ٱلْقَوْمَ ٱلظَّـٰلِمِينَ ﴿١٤٤﴾\\
\textamh{145.\  } & قُل لَّآ أَجِدُ فِى مَآ أُوحِىَ إِلَىَّ مُحَرَّمًا عَلَىٰ طَاعِمٍۢ يَطْعَمُهُۥٓ إِلَّآ أَن يَكُونَ مَيْتَةً أَوْ دَمًۭا مَّسْفُوحًا أَوْ لَحْمَ خِنزِيرٍۢ فَإِنَّهُۥ رِجْسٌ أَوْ فِسْقًا أُهِلَّ لِغَيْرِ ٱللَّهِ بِهِۦ ۚ فَمَنِ ٱضْطُرَّ غَيْرَ بَاغٍۢ وَلَا عَادٍۢ فَإِنَّ رَبَّكَ غَفُورٌۭ رَّحِيمٌۭ ﴿١٤٥﴾\\
\textamh{146.\  } & وَعَلَى ٱلَّذِينَ هَادُوا۟ حَرَّمْنَا كُلَّ ذِى ظُفُرٍۢ ۖ وَمِنَ ٱلْبَقَرِ وَٱلْغَنَمِ حَرَّمْنَا عَلَيْهِمْ شُحُومَهُمَآ إِلَّا مَا حَمَلَتْ ظُهُورُهُمَآ أَوِ ٱلْحَوَايَآ أَوْ مَا ٱخْتَلَطَ بِعَظْمٍۢ ۚ ذَٟلِكَ جَزَيْنَـٰهُم بِبَغْيِهِمْ ۖ وَإِنَّا لَصَـٰدِقُونَ ﴿١٤٦﴾\\
\textamh{147.\  } & فَإِن كَذَّبُوكَ فَقُل رَّبُّكُمْ ذُو رَحْمَةٍۢ وَٟسِعَةٍۢ وَلَا يُرَدُّ بَأْسُهُۥ عَنِ ٱلْقَوْمِ ٱلْمُجْرِمِينَ ﴿١٤٧﴾\\
\textamh{148.\  } & سَيَقُولُ ٱلَّذِينَ أَشْرَكُوا۟ لَوْ شَآءَ ٱللَّهُ مَآ أَشْرَكْنَا وَلَآ ءَابَآؤُنَا وَلَا حَرَّمْنَا مِن شَىْءٍۢ ۚ كَذَٟلِكَ كَذَّبَ ٱلَّذِينَ مِن قَبْلِهِمْ حَتَّىٰ ذَاقُوا۟ بَأْسَنَا ۗ قُلْ هَلْ عِندَكُم مِّنْ عِلْمٍۢ فَتُخْرِجُوهُ لَنَآ ۖ إِن تَتَّبِعُونَ إِلَّا ٱلظَّنَّ وَإِنْ أَنتُمْ إِلَّا تَخْرُصُونَ ﴿١٤٨﴾\\
\textamh{149.\  } & قُلْ فَلِلَّهِ ٱلْحُجَّةُ ٱلْبَٰلِغَةُ ۖ فَلَوْ شَآءَ لَهَدَىٰكُمْ أَجْمَعِينَ ﴿١٤٩﴾\\
\textamh{150.\  } & قُلْ هَلُمَّ شُهَدَآءَكُمُ ٱلَّذِينَ يَشْهَدُونَ أَنَّ ٱللَّهَ حَرَّمَ هَـٰذَا ۖ فَإِن شَهِدُوا۟ فَلَا تَشْهَدْ مَعَهُمْ ۚ وَلَا تَتَّبِعْ أَهْوَآءَ ٱلَّذِينَ كَذَّبُوا۟ بِـَٔايَـٰتِنَا وَٱلَّذِينَ لَا يُؤْمِنُونَ بِٱلْءَاخِرَةِ وَهُم بِرَبِّهِمْ يَعْدِلُونَ ﴿١٥٠﴾\\
\textamh{151.\  } & ۞ قُلْ تَعَالَوْا۟ أَتْلُ مَا حَرَّمَ رَبُّكُمْ عَلَيْكُمْ ۖ أَلَّا تُشْرِكُوا۟ بِهِۦ شَيْـًۭٔا ۖ وَبِٱلْوَٟلِدَيْنِ إِحْسَـٰنًۭا ۖ وَلَا تَقْتُلُوٓا۟ أَوْلَـٰدَكُم مِّنْ إِمْلَـٰقٍۢ ۖ نَّحْنُ نَرْزُقُكُمْ وَإِيَّاهُمْ ۖ وَلَا تَقْرَبُوا۟ ٱلْفَوَٟحِشَ مَا ظَهَرَ مِنْهَا وَمَا بَطَنَ ۖ وَلَا تَقْتُلُوا۟ ٱلنَّفْسَ ٱلَّتِى حَرَّمَ ٱللَّهُ إِلَّا بِٱلْحَقِّ ۚ ذَٟلِكُمْ وَصَّىٰكُم بِهِۦ لَعَلَّكُمْ تَعْقِلُونَ ﴿١٥١﴾\\
\textamh{152.\  } & وَلَا تَقْرَبُوا۟ مَالَ ٱلْيَتِيمِ إِلَّا بِٱلَّتِى هِىَ أَحْسَنُ حَتَّىٰ يَبْلُغَ أَشُدَّهُۥ ۖ وَأَوْفُوا۟ ٱلْكَيْلَ وَٱلْمِيزَانَ بِٱلْقِسْطِ ۖ لَا نُكَلِّفُ نَفْسًا إِلَّا وُسْعَهَا ۖ وَإِذَا قُلْتُمْ فَٱعْدِلُوا۟ وَلَوْ كَانَ ذَا قُرْبَىٰ ۖ وَبِعَهْدِ ٱللَّهِ أَوْفُوا۟ ۚ ذَٟلِكُمْ وَصَّىٰكُم بِهِۦ لَعَلَّكُمْ تَذَكَّرُونَ ﴿١٥٢﴾\\
\textamh{153.\  } & وَأَنَّ هَـٰذَا صِرَٰطِى مُسْتَقِيمًۭا فَٱتَّبِعُوهُ ۖ وَلَا تَتَّبِعُوا۟ ٱلسُّبُلَ فَتَفَرَّقَ بِكُمْ عَن سَبِيلِهِۦ ۚ ذَٟلِكُمْ وَصَّىٰكُم بِهِۦ لَعَلَّكُمْ تَتَّقُونَ ﴿١٥٣﴾\\
\textamh{154.\  } & ثُمَّ ءَاتَيْنَا مُوسَى ٱلْكِتَـٰبَ تَمَامًا عَلَى ٱلَّذِىٓ أَحْسَنَ وَتَفْصِيلًۭا لِّكُلِّ شَىْءٍۢ وَهُدًۭى وَرَحْمَةًۭ لَّعَلَّهُم بِلِقَآءِ رَبِّهِمْ يُؤْمِنُونَ ﴿١٥٤﴾\\
\textamh{155.\  } & وَهَـٰذَا كِتَـٰبٌ أَنزَلْنَـٰهُ مُبَارَكٌۭ فَٱتَّبِعُوهُ وَٱتَّقُوا۟ لَعَلَّكُمْ تُرْحَمُونَ ﴿١٥٥﴾\\
\textamh{156.\  } & أَن تَقُولُوٓا۟ إِنَّمَآ أُنزِلَ ٱلْكِتَـٰبُ عَلَىٰ طَآئِفَتَيْنِ مِن قَبْلِنَا وَإِن كُنَّا عَن دِرَاسَتِهِمْ لَغَٰفِلِينَ ﴿١٥٦﴾\\
\textamh{157.\  } & أَوْ تَقُولُوا۟ لَوْ أَنَّآ أُنزِلَ عَلَيْنَا ٱلْكِتَـٰبُ لَكُنَّآ أَهْدَىٰ مِنْهُمْ ۚ فَقَدْ جَآءَكُم بَيِّنَةٌۭ مِّن رَّبِّكُمْ وَهُدًۭى وَرَحْمَةٌۭ ۚ فَمَنْ أَظْلَمُ مِمَّن كَذَّبَ بِـَٔايَـٰتِ ٱللَّهِ وَصَدَفَ عَنْهَا ۗ سَنَجْزِى ٱلَّذِينَ يَصْدِفُونَ عَنْ ءَايَـٰتِنَا سُوٓءَ ٱلْعَذَابِ بِمَا كَانُوا۟ يَصْدِفُونَ ﴿١٥٧﴾\\
\textamh{158.\  } & هَلْ يَنظُرُونَ إِلَّآ أَن تَأْتِيَهُمُ ٱلْمَلَـٰٓئِكَةُ أَوْ يَأْتِىَ رَبُّكَ أَوْ يَأْتِىَ بَعْضُ ءَايَـٰتِ رَبِّكَ ۗ يَوْمَ يَأْتِى بَعْضُ ءَايَـٰتِ رَبِّكَ لَا يَنفَعُ نَفْسًا إِيمَـٰنُهَا لَمْ تَكُنْ ءَامَنَتْ مِن قَبْلُ أَوْ كَسَبَتْ فِىٓ إِيمَـٰنِهَا خَيْرًۭا ۗ قُلِ ٱنتَظِرُوٓا۟ إِنَّا مُنتَظِرُونَ ﴿١٥٨﴾\\
\textamh{159.\  } & إِنَّ ٱلَّذِينَ فَرَّقُوا۟ دِينَهُمْ وَكَانُوا۟ شِيَعًۭا لَّسْتَ مِنْهُمْ فِى شَىْءٍ ۚ إِنَّمَآ أَمْرُهُمْ إِلَى ٱللَّهِ ثُمَّ يُنَبِّئُهُم بِمَا كَانُوا۟ يَفْعَلُونَ ﴿١٥٩﴾\\
\textamh{160.\  } & مَن جَآءَ بِٱلْحَسَنَةِ فَلَهُۥ عَشْرُ أَمْثَالِهَا ۖ وَمَن جَآءَ بِٱلسَّيِّئَةِ فَلَا يُجْزَىٰٓ إِلَّا مِثْلَهَا وَهُمْ لَا يُظْلَمُونَ ﴿١٦٠﴾\\
\textamh{161.\  } & قُلْ إِنَّنِى هَدَىٰنِى رَبِّىٓ إِلَىٰ صِرَٰطٍۢ مُّسْتَقِيمٍۢ دِينًۭا قِيَمًۭا مِّلَّةَ إِبْرَٰهِيمَ حَنِيفًۭا ۚ وَمَا كَانَ مِنَ ٱلْمُشْرِكِينَ ﴿١٦١﴾\\
\textamh{162.\  } & قُلْ إِنَّ صَلَاتِى وَنُسُكِى وَمَحْيَاىَ وَمَمَاتِى لِلَّهِ رَبِّ ٱلْعَـٰلَمِينَ ﴿١٦٢﴾\\
\textamh{163.\  } & لَا شَرِيكَ لَهُۥ ۖ وَبِذَٟلِكَ أُمِرْتُ وَأَنَا۠ أَوَّلُ ٱلْمُسْلِمِينَ ﴿١٦٣﴾\\
\textamh{164.\  } & قُلْ أَغَيْرَ ٱللَّهِ أَبْغِى رَبًّۭا وَهُوَ رَبُّ كُلِّ شَىْءٍۢ ۚ وَلَا تَكْسِبُ كُلُّ نَفْسٍ إِلَّا عَلَيْهَا ۚ وَلَا تَزِرُ وَازِرَةٌۭ وِزْرَ أُخْرَىٰ ۚ ثُمَّ إِلَىٰ رَبِّكُم مَّرْجِعُكُمْ فَيُنَبِّئُكُم بِمَا كُنتُمْ فِيهِ تَخْتَلِفُونَ ﴿١٦٤﴾\\
\textamh{165.\  } & وَهُوَ ٱلَّذِى جَعَلَكُمْ خَلَـٰٓئِفَ ٱلْأَرْضِ وَرَفَعَ بَعْضَكُمْ فَوْقَ بَعْضٍۢ دَرَجَٰتٍۢ لِّيَبْلُوَكُمْ فِى مَآ ءَاتَىٰكُمْ ۗ إِنَّ رَبَّكَ سَرِيعُ ٱلْعِقَابِ وَإِنَّهُۥ لَغَفُورٌۭ رَّحِيمٌۢ ﴿١٦٥﴾\\
\end{longtable}
\clearpage
%% License: BSD style (Berkley) (i.e. Put the Copyright owner's name always)
%% Writer and Copyright (to): Bewketu(Bilal) Tadilo (2016-17)
\centering\section{\LR{\textamharic{ሱራቱ አልአእራፍ -}  \RL{سوره  الأعراف}}}
\begin{longtable}{%
  @{}
    p{.5\textwidth}
  @{~~~~~~~~~~~~}
    p{.5\textwidth}
    @{}
}
\nopagebreak
\textamh{ቢስሚላሂ አራህመኒ ራሂይም } &  بِسْمِ ٱللَّهِ ٱلرَّحْمَـٰنِ ٱلرَّحِيمِ\\
\textamh{1.\  } &  الٓمٓصٓ ﴿١﴾\\
\textamh{2.\  } & كِتَـٰبٌ أُنزِلَ إِلَيْكَ فَلَا يَكُن فِى صَدْرِكَ حَرَجٌۭ مِّنْهُ لِتُنذِرَ بِهِۦ وَذِكْرَىٰ لِلْمُؤْمِنِينَ ﴿٢﴾\\
\textamh{3.\  } & ٱتَّبِعُوا۟ مَآ أُنزِلَ إِلَيْكُم مِّن رَّبِّكُمْ وَلَا تَتَّبِعُوا۟ مِن دُونِهِۦٓ أَوْلِيَآءَ ۗ قَلِيلًۭا مَّا تَذَكَّرُونَ ﴿٣﴾\\
\textamh{4.\  } & وَكَم مِّن قَرْيَةٍ أَهْلَكْنَـٰهَا فَجَآءَهَا بَأْسُنَا بَيَـٰتًا أَوْ هُمْ قَآئِلُونَ ﴿٤﴾\\
\textamh{5.\  } & فَمَا كَانَ دَعْوَىٰهُمْ إِذْ جَآءَهُم بَأْسُنَآ إِلَّآ أَن قَالُوٓا۟ إِنَّا كُنَّا ظَـٰلِمِينَ ﴿٥﴾\\
\textamh{6.\  } & فَلَنَسْـَٔلَنَّ ٱلَّذِينَ أُرْسِلَ إِلَيْهِمْ وَلَنَسْـَٔلَنَّ ٱلْمُرْسَلِينَ ﴿٦﴾\\
\textamh{7.\  } & فَلَنَقُصَّنَّ عَلَيْهِم بِعِلْمٍۢ ۖ وَمَا كُنَّا غَآئِبِينَ ﴿٧﴾\\
\textamh{8.\  } & وَٱلْوَزْنُ يَوْمَئِذٍ ٱلْحَقُّ ۚ فَمَن ثَقُلَتْ مَوَٟزِينُهُۥ فَأُو۟لَـٰٓئِكَ هُمُ ٱلْمُفْلِحُونَ ﴿٨﴾\\
\textamh{9.\  } & وَمَنْ خَفَّتْ مَوَٟزِينُهُۥ فَأُو۟لَـٰٓئِكَ ٱلَّذِينَ خَسِرُوٓا۟ أَنفُسَهُم بِمَا كَانُوا۟ بِـَٔايَـٰتِنَا يَظْلِمُونَ ﴿٩﴾\\
\textamh{10.\  } & وَلَقَدْ مَكَّنَّـٰكُمْ فِى ٱلْأَرْضِ وَجَعَلْنَا لَكُمْ فِيهَا مَعَـٰيِشَ ۗ قَلِيلًۭا مَّا تَشْكُرُونَ ﴿١٠﴾\\
\textamh{11.\  } & وَلَقَدْ خَلَقْنَـٰكُمْ ثُمَّ صَوَّرْنَـٰكُمْ ثُمَّ قُلْنَا لِلْمَلَـٰٓئِكَةِ ٱسْجُدُوا۟ لِءَادَمَ فَسَجَدُوٓا۟ إِلَّآ إِبْلِيسَ لَمْ يَكُن مِّنَ ٱلسَّٰجِدِينَ ﴿١١﴾\\
\textamh{12.\  } & قَالَ مَا مَنَعَكَ أَلَّا تَسْجُدَ إِذْ أَمَرْتُكَ ۖ قَالَ أَنَا۠ خَيْرٌۭ مِّنْهُ خَلَقْتَنِى مِن نَّارٍۢ وَخَلَقْتَهُۥ مِن طِينٍۢ ﴿١٢﴾\\
\textamh{13.\  } & قَالَ فَٱهْبِطْ مِنْهَا فَمَا يَكُونُ لَكَ أَن تَتَكَبَّرَ فِيهَا فَٱخْرُجْ إِنَّكَ مِنَ ٱلصَّـٰغِرِينَ ﴿١٣﴾\\
\textamh{14.\  } & قَالَ أَنظِرْنِىٓ إِلَىٰ يَوْمِ يُبْعَثُونَ ﴿١٤﴾\\
\textamh{15.\  } & قَالَ إِنَّكَ مِنَ ٱلْمُنظَرِينَ ﴿١٥﴾\\
\textamh{16.\  } & قَالَ فَبِمَآ أَغْوَيْتَنِى لَأَقْعُدَنَّ لَهُمْ صِرَٰطَكَ ٱلْمُسْتَقِيمَ ﴿١٦﴾\\
\textamh{17.\  } & ثُمَّ لَءَاتِيَنَّهُم مِّنۢ بَيْنِ أَيْدِيهِمْ وَمِنْ خَلْفِهِمْ وَعَنْ أَيْمَـٰنِهِمْ وَعَن شَمَآئِلِهِمْ ۖ وَلَا تَجِدُ أَكْثَرَهُمْ شَـٰكِرِينَ ﴿١٧﴾\\
\textamh{18.\  } & قَالَ ٱخْرُجْ مِنْهَا مَذْءُومًۭا مَّدْحُورًۭا ۖ لَّمَن تَبِعَكَ مِنْهُمْ لَأَمْلَأَنَّ جَهَنَّمَ مِنكُمْ أَجْمَعِينَ ﴿١٨﴾\\
\textamh{19.\  } & وَيَـٰٓـَٔادَمُ ٱسْكُنْ أَنتَ وَزَوْجُكَ ٱلْجَنَّةَ فَكُلَا مِنْ حَيْثُ شِئْتُمَا وَلَا تَقْرَبَا هَـٰذِهِ ٱلشَّجَرَةَ فَتَكُونَا مِنَ ٱلظَّـٰلِمِينَ ﴿١٩﴾\\
\textamh{20.\  } & فَوَسْوَسَ لَهُمَا ٱلشَّيْطَٰنُ لِيُبْدِىَ لَهُمَا مَا وُۥرِىَ عَنْهُمَا مِن سَوْءَٰتِهِمَا وَقَالَ مَا نَهَىٰكُمَا رَبُّكُمَا عَنْ هَـٰذِهِ ٱلشَّجَرَةِ إِلَّآ أَن تَكُونَا مَلَكَيْنِ أَوْ تَكُونَا مِنَ ٱلْخَـٰلِدِينَ ﴿٢٠﴾\\
\textamh{21.\  } & وَقَاسَمَهُمَآ إِنِّى لَكُمَا لَمِنَ ٱلنَّـٰصِحِينَ ﴿٢١﴾\\
\textamh{22.\  } & فَدَلَّىٰهُمَا بِغُرُورٍۢ ۚ فَلَمَّا ذَاقَا ٱلشَّجَرَةَ بَدَتْ لَهُمَا سَوْءَٰتُهُمَا وَطَفِقَا يَخْصِفَانِ عَلَيْهِمَا مِن وَرَقِ ٱلْجَنَّةِ ۖ وَنَادَىٰهُمَا رَبُّهُمَآ أَلَمْ أَنْهَكُمَا عَن تِلْكُمَا ٱلشَّجَرَةِ وَأَقُل لَّكُمَآ إِنَّ ٱلشَّيْطَٰنَ لَكُمَا عَدُوٌّۭ مُّبِينٌۭ ﴿٢٢﴾\\
\textamh{23.\  } & قَالَا رَبَّنَا ظَلَمْنَآ أَنفُسَنَا وَإِن لَّمْ تَغْفِرْ لَنَا وَتَرْحَمْنَا لَنَكُونَنَّ مِنَ ٱلْخَـٰسِرِينَ ﴿٢٣﴾\\
\textamh{24.\  } & قَالَ ٱهْبِطُوا۟ بَعْضُكُمْ لِبَعْضٍ عَدُوٌّۭ ۖ وَلَكُمْ فِى ٱلْأَرْضِ مُسْتَقَرٌّۭ وَمَتَـٰعٌ إِلَىٰ حِينٍۢ ﴿٢٤﴾\\
\textamh{25.\  } & قَالَ فِيهَا تَحْيَوْنَ وَفِيهَا تَمُوتُونَ وَمِنْهَا تُخْرَجُونَ ﴿٢٥﴾\\
\textamh{26.\  } & يَـٰبَنِىٓ ءَادَمَ قَدْ أَنزَلْنَا عَلَيْكُمْ لِبَاسًۭا يُوَٟرِى سَوْءَٰتِكُمْ وَرِيشًۭا ۖ وَلِبَاسُ ٱلتَّقْوَىٰ ذَٟلِكَ خَيْرٌۭ ۚ ذَٟلِكَ مِنْ ءَايَـٰتِ ٱللَّهِ لَعَلَّهُمْ يَذَّكَّرُونَ ﴿٢٦﴾\\
\textamh{27.\  } & يَـٰبَنِىٓ ءَادَمَ لَا يَفْتِنَنَّكُمُ ٱلشَّيْطَٰنُ كَمَآ أَخْرَجَ أَبَوَيْكُم مِّنَ ٱلْجَنَّةِ يَنزِعُ عَنْهُمَا لِبَاسَهُمَا لِيُرِيَهُمَا سَوْءَٰتِهِمَآ ۗ إِنَّهُۥ يَرَىٰكُمْ هُوَ وَقَبِيلُهُۥ مِنْ حَيْثُ لَا تَرَوْنَهُمْ ۗ إِنَّا جَعَلْنَا ٱلشَّيَـٰطِينَ أَوْلِيَآءَ لِلَّذِينَ لَا يُؤْمِنُونَ ﴿٢٧﴾\\
\textamh{28.\  } & وَإِذَا فَعَلُوا۟ فَـٰحِشَةًۭ قَالُوا۟ وَجَدْنَا عَلَيْهَآ ءَابَآءَنَا وَٱللَّهُ أَمَرَنَا بِهَا ۗ قُلْ إِنَّ ٱللَّهَ لَا يَأْمُرُ بِٱلْفَحْشَآءِ ۖ أَتَقُولُونَ عَلَى ٱللَّهِ مَا لَا تَعْلَمُونَ ﴿٢٨﴾\\
\textamh{29.\  } & قُلْ أَمَرَ رَبِّى بِٱلْقِسْطِ ۖ وَأَقِيمُوا۟ وُجُوهَكُمْ عِندَ كُلِّ مَسْجِدٍۢ وَٱدْعُوهُ مُخْلِصِينَ لَهُ ٱلدِّينَ ۚ كَمَا بَدَأَكُمْ تَعُودُونَ ﴿٢٩﴾\\
\textamh{30.\  } & فَرِيقًا هَدَىٰ وَفَرِيقًا حَقَّ عَلَيْهِمُ ٱلضَّلَـٰلَةُ ۗ إِنَّهُمُ ٱتَّخَذُوا۟ ٱلشَّيَـٰطِينَ أَوْلِيَآءَ مِن دُونِ ٱللَّهِ وَيَحْسَبُونَ أَنَّهُم مُّهْتَدُونَ ﴿٣٠﴾\\
\textamh{31.\  } & ۞ يَـٰبَنِىٓ ءَادَمَ خُذُوا۟ زِينَتَكُمْ عِندَ كُلِّ مَسْجِدٍۢ وَكُلُوا۟ وَٱشْرَبُوا۟ وَلَا تُسْرِفُوٓا۟ ۚ إِنَّهُۥ لَا يُحِبُّ ٱلْمُسْرِفِينَ ﴿٣١﴾\\
\textamh{32.\  } & قُلْ مَنْ حَرَّمَ زِينَةَ ٱللَّهِ ٱلَّتِىٓ أَخْرَجَ لِعِبَادِهِۦ وَٱلطَّيِّبَٰتِ مِنَ ٱلرِّزْقِ ۚ قُلْ هِىَ لِلَّذِينَ ءَامَنُوا۟ فِى ٱلْحَيَوٰةِ ٱلدُّنْيَا خَالِصَةًۭ يَوْمَ ٱلْقِيَـٰمَةِ ۗ كَذَٟلِكَ نُفَصِّلُ ٱلْءَايَـٰتِ لِقَوْمٍۢ يَعْلَمُونَ ﴿٣٢﴾\\
\textamh{33.\  } & قُلْ إِنَّمَا حَرَّمَ رَبِّىَ ٱلْفَوَٟحِشَ مَا ظَهَرَ مِنْهَا وَمَا بَطَنَ وَٱلْإِثْمَ وَٱلْبَغْىَ بِغَيْرِ ٱلْحَقِّ وَأَن تُشْرِكُوا۟ بِٱللَّهِ مَا لَمْ يُنَزِّلْ بِهِۦ سُلْطَٰنًۭا وَأَن تَقُولُوا۟ عَلَى ٱللَّهِ مَا لَا تَعْلَمُونَ ﴿٣٣﴾\\
\textamh{34.\  } & وَلِكُلِّ أُمَّةٍ أَجَلٌۭ ۖ فَإِذَا جَآءَ أَجَلُهُمْ لَا يَسْتَأْخِرُونَ سَاعَةًۭ ۖ وَلَا يَسْتَقْدِمُونَ ﴿٣٤﴾\\
\textamh{35.\  } & يَـٰبَنِىٓ ءَادَمَ إِمَّا يَأْتِيَنَّكُمْ رُسُلٌۭ مِّنكُمْ يَقُصُّونَ عَلَيْكُمْ ءَايَـٰتِى ۙ فَمَنِ ٱتَّقَىٰ وَأَصْلَحَ فَلَا خَوْفٌ عَلَيْهِمْ وَلَا هُمْ يَحْزَنُونَ ﴿٣٥﴾\\
\textamh{36.\  } & وَٱلَّذِينَ كَذَّبُوا۟ بِـَٔايَـٰتِنَا وَٱسْتَكْبَرُوا۟ عَنْهَآ أُو۟لَـٰٓئِكَ أَصْحَـٰبُ ٱلنَّارِ ۖ هُمْ فِيهَا خَـٰلِدُونَ ﴿٣٦﴾\\
\textamh{37.\  } & فَمَنْ أَظْلَمُ مِمَّنِ ٱفْتَرَىٰ عَلَى ٱللَّهِ كَذِبًا أَوْ كَذَّبَ بِـَٔايَـٰتِهِۦٓ ۚ أُو۟لَـٰٓئِكَ يَنَالُهُمْ نَصِيبُهُم مِّنَ ٱلْكِتَـٰبِ ۖ حَتَّىٰٓ إِذَا جَآءَتْهُمْ رُسُلُنَا يَتَوَفَّوْنَهُمْ قَالُوٓا۟ أَيْنَ مَا كُنتُمْ تَدْعُونَ مِن دُونِ ٱللَّهِ ۖ قَالُوا۟ ضَلُّوا۟ عَنَّا وَشَهِدُوا۟ عَلَىٰٓ أَنفُسِهِمْ أَنَّهُمْ كَانُوا۟ كَـٰفِرِينَ ﴿٣٧﴾\\
\textamh{38.\  } & قَالَ ٱدْخُلُوا۟ فِىٓ أُمَمٍۢ قَدْ خَلَتْ مِن قَبْلِكُم مِّنَ ٱلْجِنِّ وَٱلْإِنسِ فِى ٱلنَّارِ ۖ كُلَّمَا دَخَلَتْ أُمَّةٌۭ لَّعَنَتْ أُخْتَهَا ۖ حَتَّىٰٓ إِذَا ٱدَّارَكُوا۟ فِيهَا جَمِيعًۭا قَالَتْ أُخْرَىٰهُمْ لِأُولَىٰهُمْ رَبَّنَا هَـٰٓؤُلَآءِ أَضَلُّونَا فَـَٔاتِهِمْ عَذَابًۭا ضِعْفًۭا مِّنَ ٱلنَّارِ ۖ قَالَ لِكُلٍّۢ ضِعْفٌۭ وَلَـٰكِن لَّا تَعْلَمُونَ ﴿٣٨﴾\\
\textamh{39.\  } & وَقَالَتْ أُولَىٰهُمْ لِأُخْرَىٰهُمْ فَمَا كَانَ لَكُمْ عَلَيْنَا مِن فَضْلٍۢ فَذُوقُوا۟ ٱلْعَذَابَ بِمَا كُنتُمْ تَكْسِبُونَ ﴿٣٩﴾\\
\textamh{40.\  } & إِنَّ ٱلَّذِينَ كَذَّبُوا۟ بِـَٔايَـٰتِنَا وَٱسْتَكْبَرُوا۟ عَنْهَا لَا تُفَتَّحُ لَهُمْ أَبْوَٟبُ ٱلسَّمَآءِ وَلَا يَدْخُلُونَ ٱلْجَنَّةَ حَتَّىٰ يَلِجَ ٱلْجَمَلُ فِى سَمِّ ٱلْخِيَاطِ ۚ وَكَذَٟلِكَ نَجْزِى ٱلْمُجْرِمِينَ ﴿٤٠﴾\\
\textamh{41.\  } & لَهُم مِّن جَهَنَّمَ مِهَادٌۭ وَمِن فَوْقِهِمْ غَوَاشٍۢ ۚ وَكَذَٟلِكَ نَجْزِى ٱلظَّـٰلِمِينَ ﴿٤١﴾\\
\textamh{42.\  } & وَٱلَّذِينَ ءَامَنُوا۟ وَعَمِلُوا۟ ٱلصَّـٰلِحَـٰتِ لَا نُكَلِّفُ نَفْسًا إِلَّا وُسْعَهَآ أُو۟لَـٰٓئِكَ أَصْحَـٰبُ ٱلْجَنَّةِ ۖ هُمْ فِيهَا خَـٰلِدُونَ ﴿٤٢﴾\\
\textamh{43.\  } & وَنَزَعْنَا مَا فِى صُدُورِهِم مِّنْ غِلٍّۢ تَجْرِى مِن تَحْتِهِمُ ٱلْأَنْهَـٰرُ ۖ وَقَالُوا۟ ٱلْحَمْدُ لِلَّهِ ٱلَّذِى هَدَىٰنَا لِهَـٰذَا وَمَا كُنَّا لِنَهْتَدِىَ لَوْلَآ أَنْ هَدَىٰنَا ٱللَّهُ ۖ لَقَدْ جَآءَتْ رُسُلُ رَبِّنَا بِٱلْحَقِّ ۖ وَنُودُوٓا۟ أَن تِلْكُمُ ٱلْجَنَّةُ أُورِثْتُمُوهَا بِمَا كُنتُمْ تَعْمَلُونَ ﴿٤٣﴾\\
\textamh{44.\  } & وَنَادَىٰٓ أَصْحَـٰبُ ٱلْجَنَّةِ أَصْحَـٰبَ ٱلنَّارِ أَن قَدْ وَجَدْنَا مَا وَعَدَنَا رَبُّنَا حَقًّۭا فَهَلْ وَجَدتُّم مَّا وَعَدَ رَبُّكُمْ حَقًّۭا ۖ قَالُوا۟ نَعَمْ ۚ فَأَذَّنَ مُؤَذِّنٌۢ بَيْنَهُمْ أَن لَّعْنَةُ ٱللَّهِ عَلَى ٱلظَّـٰلِمِينَ ﴿٤٤﴾\\
\textamh{45.\  } & ٱلَّذِينَ يَصُدُّونَ عَن سَبِيلِ ٱللَّهِ وَيَبْغُونَهَا عِوَجًۭا وَهُم بِٱلْءَاخِرَةِ كَـٰفِرُونَ ﴿٤٥﴾\\
\textamh{46.\  } & وَبَيْنَهُمَا حِجَابٌۭ ۚ وَعَلَى ٱلْأَعْرَافِ رِجَالٌۭ يَعْرِفُونَ كُلًّۢا بِسِيمَىٰهُمْ ۚ وَنَادَوْا۟ أَصْحَـٰبَ ٱلْجَنَّةِ أَن سَلَـٰمٌ عَلَيْكُمْ ۚ لَمْ يَدْخُلُوهَا وَهُمْ يَطْمَعُونَ ﴿٤٦﴾\\
\textamh{47.\  } & ۞ وَإِذَا صُرِفَتْ أَبْصَـٰرُهُمْ تِلْقَآءَ أَصْحَـٰبِ ٱلنَّارِ قَالُوا۟ رَبَّنَا لَا تَجْعَلْنَا مَعَ ٱلْقَوْمِ ٱلظَّـٰلِمِينَ ﴿٤٧﴾\\
\textamh{48.\  } & وَنَادَىٰٓ أَصْحَـٰبُ ٱلْأَعْرَافِ رِجَالًۭا يَعْرِفُونَهُم بِسِيمَىٰهُمْ قَالُوا۟ مَآ أَغْنَىٰ عَنكُمْ جَمْعُكُمْ وَمَا كُنتُمْ تَسْتَكْبِرُونَ ﴿٤٨﴾\\
\textamh{49.\  } & أَهَـٰٓؤُلَآءِ ٱلَّذِينَ أَقْسَمْتُمْ لَا يَنَالُهُمُ ٱللَّهُ بِرَحْمَةٍ ۚ ٱدْخُلُوا۟ ٱلْجَنَّةَ لَا خَوْفٌ عَلَيْكُمْ وَلَآ أَنتُمْ تَحْزَنُونَ ﴿٤٩﴾\\
\textamh{50.\  } & وَنَادَىٰٓ أَصْحَـٰبُ ٱلنَّارِ أَصْحَـٰبَ ٱلْجَنَّةِ أَنْ أَفِيضُوا۟ عَلَيْنَا مِنَ ٱلْمَآءِ أَوْ مِمَّا رَزَقَكُمُ ٱللَّهُ ۚ قَالُوٓا۟ إِنَّ ٱللَّهَ حَرَّمَهُمَا عَلَى ٱلْكَـٰفِرِينَ ﴿٥٠﴾\\
\textamh{51.\  } & ٱلَّذِينَ ٱتَّخَذُوا۟ دِينَهُمْ لَهْوًۭا وَلَعِبًۭا وَغَرَّتْهُمُ ٱلْحَيَوٰةُ ٱلدُّنْيَا ۚ فَٱلْيَوْمَ نَنسَىٰهُمْ كَمَا نَسُوا۟ لِقَآءَ يَوْمِهِمْ هَـٰذَا وَمَا كَانُوا۟ بِـَٔايَـٰتِنَا يَجْحَدُونَ ﴿٥١﴾\\
\textamh{52.\  } & وَلَقَدْ جِئْنَـٰهُم بِكِتَـٰبٍۢ فَصَّلْنَـٰهُ عَلَىٰ عِلْمٍ هُدًۭى وَرَحْمَةًۭ لِّقَوْمٍۢ يُؤْمِنُونَ ﴿٥٢﴾\\
\textamh{53.\  } & هَلْ يَنظُرُونَ إِلَّا تَأْوِيلَهُۥ ۚ يَوْمَ يَأْتِى تَأْوِيلُهُۥ يَقُولُ ٱلَّذِينَ نَسُوهُ مِن قَبْلُ قَدْ جَآءَتْ رُسُلُ رَبِّنَا بِٱلْحَقِّ فَهَل لَّنَا مِن شُفَعَآءَ فَيَشْفَعُوا۟ لَنَآ أَوْ نُرَدُّ فَنَعْمَلَ غَيْرَ ٱلَّذِى كُنَّا نَعْمَلُ ۚ قَدْ خَسِرُوٓا۟ أَنفُسَهُمْ وَضَلَّ عَنْهُم مَّا كَانُوا۟ يَفْتَرُونَ ﴿٥٣﴾\\
\textamh{54.\  } & إِنَّ رَبَّكُمُ ٱللَّهُ ٱلَّذِى خَلَقَ ٱلسَّمَـٰوَٟتِ وَٱلْأَرْضَ فِى سِتَّةِ أَيَّامٍۢ ثُمَّ ٱسْتَوَىٰ عَلَى ٱلْعَرْشِ يُغْشِى ٱلَّيْلَ ٱلنَّهَارَ يَطْلُبُهُۥ حَثِيثًۭا وَٱلشَّمْسَ وَٱلْقَمَرَ وَٱلنُّجُومَ مُسَخَّرَٰتٍۭ بِأَمْرِهِۦٓ ۗ أَلَا لَهُ ٱلْخَلْقُ وَٱلْأَمْرُ ۗ تَبَارَكَ ٱللَّهُ رَبُّ ٱلْعَـٰلَمِينَ ﴿٥٤﴾\\
\textamh{55.\  } & ٱدْعُوا۟ رَبَّكُمْ تَضَرُّعًۭا وَخُفْيَةً ۚ إِنَّهُۥ لَا يُحِبُّ ٱلْمُعْتَدِينَ ﴿٥٥﴾\\
\textamh{56.\  } & وَلَا تُفْسِدُوا۟ فِى ٱلْأَرْضِ بَعْدَ إِصْلَـٰحِهَا وَٱدْعُوهُ خَوْفًۭا وَطَمَعًا ۚ إِنَّ رَحْمَتَ ٱللَّهِ قَرِيبٌۭ مِّنَ ٱلْمُحْسِنِينَ ﴿٥٦﴾\\
\textamh{57.\  } & وَهُوَ ٱلَّذِى يُرْسِلُ ٱلرِّيَـٰحَ بُشْرًۢا بَيْنَ يَدَىْ رَحْمَتِهِۦ ۖ حَتَّىٰٓ إِذَآ أَقَلَّتْ سَحَابًۭا ثِقَالًۭا سُقْنَـٰهُ لِبَلَدٍۢ مَّيِّتٍۢ فَأَنزَلْنَا بِهِ ٱلْمَآءَ فَأَخْرَجْنَا بِهِۦ مِن كُلِّ ٱلثَّمَرَٰتِ ۚ كَذَٟلِكَ نُخْرِجُ ٱلْمَوْتَىٰ لَعَلَّكُمْ تَذَكَّرُونَ ﴿٥٧﴾\\
\textamh{58.\  } & وَٱلْبَلَدُ ٱلطَّيِّبُ يَخْرُجُ نَبَاتُهُۥ بِإِذْنِ رَبِّهِۦ ۖ وَٱلَّذِى خَبُثَ لَا يَخْرُجُ إِلَّا نَكِدًۭا ۚ كَذَٟلِكَ نُصَرِّفُ ٱلْءَايَـٰتِ لِقَوْمٍۢ يَشْكُرُونَ ﴿٥٨﴾\\
\textamh{59.\  } & لَقَدْ أَرْسَلْنَا نُوحًا إِلَىٰ قَوْمِهِۦ فَقَالَ يَـٰقَوْمِ ٱعْبُدُوا۟ ٱللَّهَ مَا لَكُم مِّنْ إِلَـٰهٍ غَيْرُهُۥٓ إِنِّىٓ أَخَافُ عَلَيْكُمْ عَذَابَ يَوْمٍ عَظِيمٍۢ ﴿٥٩﴾\\
\textamh{60.\  } & قَالَ ٱلْمَلَأُ مِن قَوْمِهِۦٓ إِنَّا لَنَرَىٰكَ فِى ضَلَـٰلٍۢ مُّبِينٍۢ ﴿٦٠﴾\\
\textamh{61.\  } & قَالَ يَـٰقَوْمِ لَيْسَ بِى ضَلَـٰلَةٌۭ وَلَـٰكِنِّى رَسُولٌۭ مِّن رَّبِّ ٱلْعَـٰلَمِينَ ﴿٦١﴾\\
\textamh{62.\  } & أُبَلِّغُكُمْ رِسَـٰلَـٰتِ رَبِّى وَأَنصَحُ لَكُمْ وَأَعْلَمُ مِنَ ٱللَّهِ مَا لَا تَعْلَمُونَ ﴿٦٢﴾\\
\textamh{63.\  } & أَوَعَجِبْتُمْ أَن جَآءَكُمْ ذِكْرٌۭ مِّن رَّبِّكُمْ عَلَىٰ رَجُلٍۢ مِّنكُمْ لِيُنذِرَكُمْ وَلِتَتَّقُوا۟ وَلَعَلَّكُمْ تُرْحَمُونَ ﴿٦٣﴾\\
\textamh{64.\  } & فَكَذَّبُوهُ فَأَنجَيْنَـٰهُ وَٱلَّذِينَ مَعَهُۥ فِى ٱلْفُلْكِ وَأَغْرَقْنَا ٱلَّذِينَ كَذَّبُوا۟ بِـَٔايَـٰتِنَآ ۚ إِنَّهُمْ كَانُوا۟ قَوْمًا عَمِينَ ﴿٦٤﴾\\
\textamh{65.\  } & ۞ وَإِلَىٰ عَادٍ أَخَاهُمْ هُودًۭا ۗ قَالَ يَـٰقَوْمِ ٱعْبُدُوا۟ ٱللَّهَ مَا لَكُم مِّنْ إِلَـٰهٍ غَيْرُهُۥٓ ۚ أَفَلَا تَتَّقُونَ ﴿٦٥﴾\\
\textamh{66.\  } & قَالَ ٱلْمَلَأُ ٱلَّذِينَ كَفَرُوا۟ مِن قَوْمِهِۦٓ إِنَّا لَنَرَىٰكَ فِى سَفَاهَةٍۢ وَإِنَّا لَنَظُنُّكَ مِنَ ٱلْكَـٰذِبِينَ ﴿٦٦﴾\\
\textamh{67.\  } & قَالَ يَـٰقَوْمِ لَيْسَ بِى سَفَاهَةٌۭ وَلَـٰكِنِّى رَسُولٌۭ مِّن رَّبِّ ٱلْعَـٰلَمِينَ ﴿٦٧﴾\\
\textamh{68.\  } & أُبَلِّغُكُمْ رِسَـٰلَـٰتِ رَبِّى وَأَنَا۠ لَكُمْ نَاصِحٌ أَمِينٌ ﴿٦٨﴾\\
\textamh{69.\  } & أَوَعَجِبْتُمْ أَن جَآءَكُمْ ذِكْرٌۭ مِّن رَّبِّكُمْ عَلَىٰ رَجُلٍۢ مِّنكُمْ لِيُنذِرَكُمْ ۚ وَٱذْكُرُوٓا۟ إِذْ جَعَلَكُمْ خُلَفَآءَ مِنۢ بَعْدِ قَوْمِ نُوحٍۢ وَزَادَكُمْ فِى ٱلْخَلْقِ بَصْۜطَةًۭ ۖ فَٱذْكُرُوٓا۟ ءَالَآءَ ٱللَّهِ لَعَلَّكُمْ تُفْلِحُونَ ﴿٦٩﴾\\
\textamh{70.\  } & قَالُوٓا۟ أَجِئْتَنَا لِنَعْبُدَ ٱللَّهَ وَحْدَهُۥ وَنَذَرَ مَا كَانَ يَعْبُدُ ءَابَآؤُنَا ۖ فَأْتِنَا بِمَا تَعِدُنَآ إِن كُنتَ مِنَ ٱلصَّـٰدِقِينَ ﴿٧٠﴾\\
\textamh{71.\  } & قَالَ قَدْ وَقَعَ عَلَيْكُم مِّن رَّبِّكُمْ رِجْسٌۭ وَغَضَبٌ ۖ أَتُجَٰدِلُونَنِى فِىٓ أَسْمَآءٍۢ سَمَّيْتُمُوهَآ أَنتُمْ وَءَابَآؤُكُم مَّا نَزَّلَ ٱللَّهُ بِهَا مِن سُلْطَٰنٍۢ ۚ فَٱنتَظِرُوٓا۟ إِنِّى مَعَكُم مِّنَ ٱلْمُنتَظِرِينَ ﴿٧١﴾\\
\textamh{72.\  } & فَأَنجَيْنَـٰهُ وَٱلَّذِينَ مَعَهُۥ بِرَحْمَةٍۢ مِّنَّا وَقَطَعْنَا دَابِرَ ٱلَّذِينَ كَذَّبُوا۟ بِـَٔايَـٰتِنَا ۖ وَمَا كَانُوا۟ مُؤْمِنِينَ ﴿٧٢﴾\\
\textamh{73.\  } & وَإِلَىٰ ثَمُودَ أَخَاهُمْ صَـٰلِحًۭا ۗ قَالَ يَـٰقَوْمِ ٱعْبُدُوا۟ ٱللَّهَ مَا لَكُم مِّنْ إِلَـٰهٍ غَيْرُهُۥ ۖ قَدْ جَآءَتْكُم بَيِّنَةٌۭ مِّن رَّبِّكُمْ ۖ هَـٰذِهِۦ نَاقَةُ ٱللَّهِ لَكُمْ ءَايَةًۭ ۖ فَذَرُوهَا تَأْكُلْ فِىٓ أَرْضِ ٱللَّهِ ۖ وَلَا تَمَسُّوهَا بِسُوٓءٍۢ فَيَأْخُذَكُمْ عَذَابٌ أَلِيمٌۭ ﴿٧٣﴾\\
\textamh{74.\  } & وَٱذْكُرُوٓا۟ إِذْ جَعَلَكُمْ خُلَفَآءَ مِنۢ بَعْدِ عَادٍۢ وَبَوَّأَكُمْ فِى ٱلْأَرْضِ تَتَّخِذُونَ مِن سُهُولِهَا قُصُورًۭا وَتَنْحِتُونَ ٱلْجِبَالَ بُيُوتًۭا ۖ فَٱذْكُرُوٓا۟ ءَالَآءَ ٱللَّهِ وَلَا تَعْثَوْا۟ فِى ٱلْأَرْضِ مُفْسِدِينَ ﴿٧٤﴾\\
\textamh{75.\  } & قَالَ ٱلْمَلَأُ ٱلَّذِينَ ٱسْتَكْبَرُوا۟ مِن قَوْمِهِۦ لِلَّذِينَ ٱسْتُضْعِفُوا۟ لِمَنْ ءَامَنَ مِنْهُمْ أَتَعْلَمُونَ أَنَّ صَـٰلِحًۭا مُّرْسَلٌۭ مِّن رَّبِّهِۦ ۚ قَالُوٓا۟ إِنَّا بِمَآ أُرْسِلَ بِهِۦ مُؤْمِنُونَ ﴿٧٥﴾\\
\textamh{76.\  } & قَالَ ٱلَّذِينَ ٱسْتَكْبَرُوٓا۟ إِنَّا بِٱلَّذِىٓ ءَامَنتُم بِهِۦ كَـٰفِرُونَ ﴿٧٦﴾\\
\textamh{77.\  } & فَعَقَرُوا۟ ٱلنَّاقَةَ وَعَتَوْا۟ عَنْ أَمْرِ رَبِّهِمْ وَقَالُوا۟ يَـٰصَـٰلِحُ ٱئْتِنَا بِمَا تَعِدُنَآ إِن كُنتَ مِنَ ٱلْمُرْسَلِينَ ﴿٧٧﴾\\
\textamh{78.\  } & فَأَخَذَتْهُمُ ٱلرَّجْفَةُ فَأَصْبَحُوا۟ فِى دَارِهِمْ جَٰثِمِينَ ﴿٧٨﴾\\
\textamh{79.\  } & فَتَوَلَّىٰ عَنْهُمْ وَقَالَ يَـٰقَوْمِ لَقَدْ أَبْلَغْتُكُمْ رِسَالَةَ رَبِّى وَنَصَحْتُ لَكُمْ وَلَـٰكِن لَّا تُحِبُّونَ ٱلنَّـٰصِحِينَ ﴿٧٩﴾\\
\textamh{80.\  } & وَلُوطًا إِذْ قَالَ لِقَوْمِهِۦٓ أَتَأْتُونَ ٱلْفَـٰحِشَةَ مَا سَبَقَكُم بِهَا مِنْ أَحَدٍۢ مِّنَ ٱلْعَـٰلَمِينَ ﴿٨٠﴾\\
\textamh{81.\  } & إِنَّكُمْ لَتَأْتُونَ ٱلرِّجَالَ شَهْوَةًۭ مِّن دُونِ ٱلنِّسَآءِ ۚ بَلْ أَنتُمْ قَوْمٌۭ مُّسْرِفُونَ ﴿٨١﴾\\
\textamh{82.\  } & وَمَا كَانَ جَوَابَ قَوْمِهِۦٓ إِلَّآ أَن قَالُوٓا۟ أَخْرِجُوهُم مِّن قَرْيَتِكُمْ ۖ إِنَّهُمْ أُنَاسٌۭ يَتَطَهَّرُونَ ﴿٨٢﴾\\
\textamh{83.\  } & فَأَنجَيْنَـٰهُ وَأَهْلَهُۥٓ إِلَّا ٱمْرَأَتَهُۥ كَانَتْ مِنَ ٱلْغَٰبِرِينَ ﴿٨٣﴾\\
\textamh{84.\  } & وَأَمْطَرْنَا عَلَيْهِم مَّطَرًۭا ۖ فَٱنظُرْ كَيْفَ كَانَ عَـٰقِبَةُ ٱلْمُجْرِمِينَ ﴿٨٤﴾\\
\textamh{85.\  } & وَإِلَىٰ مَدْيَنَ أَخَاهُمْ شُعَيْبًۭا ۗ قَالَ يَـٰقَوْمِ ٱعْبُدُوا۟ ٱللَّهَ مَا لَكُم مِّنْ إِلَـٰهٍ غَيْرُهُۥ ۖ قَدْ جَآءَتْكُم بَيِّنَةٌۭ مِّن رَّبِّكُمْ ۖ فَأَوْفُوا۟ ٱلْكَيْلَ وَٱلْمِيزَانَ وَلَا تَبْخَسُوا۟ ٱلنَّاسَ أَشْيَآءَهُمْ وَلَا تُفْسِدُوا۟ فِى ٱلْأَرْضِ بَعْدَ إِصْلَـٰحِهَا ۚ ذَٟلِكُمْ خَيْرٌۭ لَّكُمْ إِن كُنتُم مُّؤْمِنِينَ ﴿٨٥﴾\\
\textamh{86.\  } & وَلَا تَقْعُدُوا۟ بِكُلِّ صِرَٰطٍۢ تُوعِدُونَ وَتَصُدُّونَ عَن سَبِيلِ ٱللَّهِ مَنْ ءَامَنَ بِهِۦ وَتَبْغُونَهَا عِوَجًۭا ۚ وَٱذْكُرُوٓا۟ إِذْ كُنتُمْ قَلِيلًۭا فَكَثَّرَكُمْ ۖ وَٱنظُرُوا۟ كَيْفَ كَانَ عَـٰقِبَةُ ٱلْمُفْسِدِينَ ﴿٨٦﴾\\
\textamh{87.\  } & وَإِن كَانَ طَآئِفَةٌۭ مِّنكُمْ ءَامَنُوا۟ بِٱلَّذِىٓ أُرْسِلْتُ بِهِۦ وَطَآئِفَةٌۭ لَّمْ يُؤْمِنُوا۟ فَٱصْبِرُوا۟ حَتَّىٰ يَحْكُمَ ٱللَّهُ بَيْنَنَا ۚ وَهُوَ خَيْرُ ٱلْحَـٰكِمِينَ ﴿٨٧﴾\\
\textamh{88.\  } & ۞ قَالَ ٱلْمَلَأُ ٱلَّذِينَ ٱسْتَكْبَرُوا۟ مِن قَوْمِهِۦ لَنُخْرِجَنَّكَ يَـٰشُعَيْبُ وَٱلَّذِينَ ءَامَنُوا۟ مَعَكَ مِن قَرْيَتِنَآ أَوْ لَتَعُودُنَّ فِى مِلَّتِنَا ۚ قَالَ أَوَلَوْ كُنَّا كَـٰرِهِينَ ﴿٨٨﴾\\
\textamh{89.\  } & قَدِ ٱفْتَرَيْنَا عَلَى ٱللَّهِ كَذِبًا إِنْ عُدْنَا فِى مِلَّتِكُم بَعْدَ إِذْ نَجَّىٰنَا ٱللَّهُ مِنْهَا ۚ وَمَا يَكُونُ لَنَآ أَن نَّعُودَ فِيهَآ إِلَّآ أَن يَشَآءَ ٱللَّهُ رَبُّنَا ۚ وَسِعَ رَبُّنَا كُلَّ شَىْءٍ عِلْمًا ۚ عَلَى ٱللَّهِ تَوَكَّلْنَا ۚ رَبَّنَا ٱفْتَحْ بَيْنَنَا وَبَيْنَ قَوْمِنَا بِٱلْحَقِّ وَأَنتَ خَيْرُ ٱلْفَـٰتِحِينَ ﴿٨٩﴾\\
\textamh{90.\  } & وَقَالَ ٱلْمَلَأُ ٱلَّذِينَ كَفَرُوا۟ مِن قَوْمِهِۦ لَئِنِ ٱتَّبَعْتُمْ شُعَيْبًا إِنَّكُمْ إِذًۭا لَّخَـٰسِرُونَ ﴿٩٠﴾\\
\textamh{91.\  } & فَأَخَذَتْهُمُ ٱلرَّجْفَةُ فَأَصْبَحُوا۟ فِى دَارِهِمْ جَٰثِمِينَ ﴿٩١﴾\\
\textamh{92.\  } & ٱلَّذِينَ كَذَّبُوا۟ شُعَيْبًۭا كَأَن لَّمْ يَغْنَوْا۟ فِيهَا ۚ ٱلَّذِينَ كَذَّبُوا۟ شُعَيْبًۭا كَانُوا۟ هُمُ ٱلْخَـٰسِرِينَ ﴿٩٢﴾\\
\textamh{93.\  } & فَتَوَلَّىٰ عَنْهُمْ وَقَالَ يَـٰقَوْمِ لَقَدْ أَبْلَغْتُكُمْ رِسَـٰلَـٰتِ رَبِّى وَنَصَحْتُ لَكُمْ ۖ فَكَيْفَ ءَاسَىٰ عَلَىٰ قَوْمٍۢ كَـٰفِرِينَ ﴿٩٣﴾\\
\textamh{94.\  } & وَمَآ أَرْسَلْنَا فِى قَرْيَةٍۢ مِّن نَّبِىٍّ إِلَّآ أَخَذْنَآ أَهْلَهَا بِٱلْبَأْسَآءِ وَٱلضَّرَّآءِ لَعَلَّهُمْ يَضَّرَّعُونَ ﴿٩٤﴾\\
\textamh{95.\  } & ثُمَّ بَدَّلْنَا مَكَانَ ٱلسَّيِّئَةِ ٱلْحَسَنَةَ حَتَّىٰ عَفَوا۟ وَّقَالُوا۟ قَدْ مَسَّ ءَابَآءَنَا ٱلضَّرَّآءُ وَٱلسَّرَّآءُ فَأَخَذْنَـٰهُم بَغْتَةًۭ وَهُمْ لَا يَشْعُرُونَ ﴿٩٥﴾\\
\textamh{96.\  } & وَلَوْ أَنَّ أَهْلَ ٱلْقُرَىٰٓ ءَامَنُوا۟ وَٱتَّقَوْا۟ لَفَتَحْنَا عَلَيْهِم بَرَكَـٰتٍۢ مِّنَ ٱلسَّمَآءِ وَٱلْأَرْضِ وَلَـٰكِن كَذَّبُوا۟ فَأَخَذْنَـٰهُم بِمَا كَانُوا۟ يَكْسِبُونَ ﴿٩٦﴾\\
\textamh{97.\  } & أَفَأَمِنَ أَهْلُ ٱلْقُرَىٰٓ أَن يَأْتِيَهُم بَأْسُنَا بَيَـٰتًۭا وَهُمْ نَآئِمُونَ ﴿٩٧﴾\\
\textamh{98.\  } & أَوَأَمِنَ أَهْلُ ٱلْقُرَىٰٓ أَن يَأْتِيَهُم بَأْسُنَا ضُحًۭى وَهُمْ يَلْعَبُونَ ﴿٩٨﴾\\
\textamh{99.\  } & أَفَأَمِنُوا۟ مَكْرَ ٱللَّهِ ۚ فَلَا يَأْمَنُ مَكْرَ ٱللَّهِ إِلَّا ٱلْقَوْمُ ٱلْخَـٰسِرُونَ ﴿٩٩﴾\\
\textamh{100.\  } & أَوَلَمْ يَهْدِ لِلَّذِينَ يَرِثُونَ ٱلْأَرْضَ مِنۢ بَعْدِ أَهْلِهَآ أَن لَّوْ نَشَآءُ أَصَبْنَـٰهُم بِذُنُوبِهِمْ ۚ وَنَطْبَعُ عَلَىٰ قُلُوبِهِمْ فَهُمْ لَا يَسْمَعُونَ ﴿١٠٠﴾\\
\textamh{101.\  } & تِلْكَ ٱلْقُرَىٰ نَقُصُّ عَلَيْكَ مِنْ أَنۢبَآئِهَا ۚ وَلَقَدْ جَآءَتْهُمْ رُسُلُهُم بِٱلْبَيِّنَـٰتِ فَمَا كَانُوا۟ لِيُؤْمِنُوا۟ بِمَا كَذَّبُوا۟ مِن قَبْلُ ۚ كَذَٟلِكَ يَطْبَعُ ٱللَّهُ عَلَىٰ قُلُوبِ ٱلْكَـٰفِرِينَ ﴿١٠١﴾\\
\textamh{102.\  } & وَمَا وَجَدْنَا لِأَكْثَرِهِم مِّنْ عَهْدٍۢ ۖ وَإِن وَجَدْنَآ أَكْثَرَهُمْ لَفَـٰسِقِينَ ﴿١٠٢﴾\\
\textamh{103.\  } & ثُمَّ بَعَثْنَا مِنۢ بَعْدِهِم مُّوسَىٰ بِـَٔايَـٰتِنَآ إِلَىٰ فِرْعَوْنَ وَمَلَإِي۟هِۦ فَظَلَمُوا۟ بِهَا ۖ فَٱنظُرْ كَيْفَ كَانَ عَـٰقِبَةُ ٱلْمُفْسِدِينَ ﴿١٠٣﴾\\
\textamh{104.\  } & وَقَالَ مُوسَىٰ يَـٰفِرْعَوْنُ إِنِّى رَسُولٌۭ مِّن رَّبِّ ٱلْعَـٰلَمِينَ ﴿١٠٤﴾\\
\textamh{105.\  } & حَقِيقٌ عَلَىٰٓ أَن لَّآ أَقُولَ عَلَى ٱللَّهِ إِلَّا ٱلْحَقَّ ۚ قَدْ جِئْتُكُم بِبَيِّنَةٍۢ مِّن رَّبِّكُمْ فَأَرْسِلْ مَعِىَ بَنِىٓ إِسْرَٰٓءِيلَ ﴿١٠٥﴾\\
\textamh{106.\  } & قَالَ إِن كُنتَ جِئْتَ بِـَٔايَةٍۢ فَأْتِ بِهَآ إِن كُنتَ مِنَ ٱلصَّـٰدِقِينَ ﴿١٠٦﴾\\
\textamh{107.\  } & فَأَلْقَىٰ عَصَاهُ فَإِذَا هِىَ ثُعْبَانٌۭ مُّبِينٌۭ ﴿١٠٧﴾\\
\textamh{108.\  } & وَنَزَعَ يَدَهُۥ فَإِذَا هِىَ بَيْضَآءُ لِلنَّـٰظِرِينَ ﴿١٠٨﴾\\
\textamh{109.\  } & قَالَ ٱلْمَلَأُ مِن قَوْمِ فِرْعَوْنَ إِنَّ هَـٰذَا لَسَـٰحِرٌ عَلِيمٌۭ ﴿١٠٩﴾\\
\textamh{110.\  } & يُرِيدُ أَن يُخْرِجَكُم مِّنْ أَرْضِكُمْ ۖ فَمَاذَا تَأْمُرُونَ ﴿١١٠﴾\\
\textamh{111.\  } & قَالُوٓا۟ أَرْجِهْ وَأَخَاهُ وَأَرْسِلْ فِى ٱلْمَدَآئِنِ حَـٰشِرِينَ ﴿١١١﴾\\
\textamh{112.\  } & يَأْتُوكَ بِكُلِّ سَـٰحِرٍ عَلِيمٍۢ ﴿١١٢﴾\\
\textamh{113.\  } & وَجَآءَ ٱلسَّحَرَةُ فِرْعَوْنَ قَالُوٓا۟ إِنَّ لَنَا لَأَجْرًا إِن كُنَّا نَحْنُ ٱلْغَٰلِبِينَ ﴿١١٣﴾\\
\textamh{114.\  } & قَالَ نَعَمْ وَإِنَّكُمْ لَمِنَ ٱلْمُقَرَّبِينَ ﴿١١٤﴾\\
\textamh{115.\  } & قَالُوا۟ يَـٰمُوسَىٰٓ إِمَّآ أَن تُلْقِىَ وَإِمَّآ أَن نَّكُونَ نَحْنُ ٱلْمُلْقِينَ ﴿١١٥﴾\\
\textamh{116.\  } & قَالَ أَلْقُوا۟ ۖ فَلَمَّآ أَلْقَوْا۟ سَحَرُوٓا۟ أَعْيُنَ ٱلنَّاسِ وَٱسْتَرْهَبُوهُمْ وَجَآءُو بِسِحْرٍ عَظِيمٍۢ ﴿١١٦﴾\\
\textamh{117.\  } & ۞ وَأَوْحَيْنَآ إِلَىٰ مُوسَىٰٓ أَنْ أَلْقِ عَصَاكَ ۖ فَإِذَا هِىَ تَلْقَفُ مَا يَأْفِكُونَ ﴿١١٧﴾\\
\textamh{118.\  } & فَوَقَعَ ٱلْحَقُّ وَبَطَلَ مَا كَانُوا۟ يَعْمَلُونَ ﴿١١٨﴾\\
\textamh{119.\  } & فَغُلِبُوا۟ هُنَالِكَ وَٱنقَلَبُوا۟ صَـٰغِرِينَ ﴿١١٩﴾\\
\textamh{120.\  } & وَأُلْقِىَ ٱلسَّحَرَةُ سَـٰجِدِينَ ﴿١٢٠﴾\\
\textamh{121.\  } & قَالُوٓا۟ ءَامَنَّا بِرَبِّ ٱلْعَـٰلَمِينَ ﴿١٢١﴾\\
\textamh{122.\  } & رَبِّ مُوسَىٰ وَهَـٰرُونَ ﴿١٢٢﴾\\
\textamh{123.\  } & قَالَ فِرْعَوْنُ ءَامَنتُم بِهِۦ قَبْلَ أَنْ ءَاذَنَ لَكُمْ ۖ إِنَّ هَـٰذَا لَمَكْرٌۭ مَّكَرْتُمُوهُ فِى ٱلْمَدِينَةِ لِتُخْرِجُوا۟ مِنْهَآ أَهْلَهَا ۖ فَسَوْفَ تَعْلَمُونَ ﴿١٢٣﴾\\
\textamh{124.\  } & لَأُقَطِّعَنَّ أَيْدِيَكُمْ وَأَرْجُلَكُم مِّنْ خِلَـٰفٍۢ ثُمَّ لَأُصَلِّبَنَّكُمْ أَجْمَعِينَ ﴿١٢٤﴾\\
\textamh{125.\  } & قَالُوٓا۟ إِنَّآ إِلَىٰ رَبِّنَا مُنقَلِبُونَ ﴿١٢٥﴾\\
\textamh{126.\  } & وَمَا تَنقِمُ مِنَّآ إِلَّآ أَنْ ءَامَنَّا بِـَٔايَـٰتِ رَبِّنَا لَمَّا جَآءَتْنَا ۚ رَبَّنَآ أَفْرِغْ عَلَيْنَا صَبْرًۭا وَتَوَفَّنَا مُسْلِمِينَ ﴿١٢٦﴾\\
\textamh{127.\  } & وَقَالَ ٱلْمَلَأُ مِن قَوْمِ فِرْعَوْنَ أَتَذَرُ مُوسَىٰ وَقَوْمَهُۥ لِيُفْسِدُوا۟ فِى ٱلْأَرْضِ وَيَذَرَكَ وَءَالِهَتَكَ ۚ قَالَ سَنُقَتِّلُ أَبْنَآءَهُمْ وَنَسْتَحْىِۦ نِسَآءَهُمْ وَإِنَّا فَوْقَهُمْ قَـٰهِرُونَ ﴿١٢٧﴾\\
\textamh{128.\  } & قَالَ مُوسَىٰ لِقَوْمِهِ ٱسْتَعِينُوا۟ بِٱللَّهِ وَٱصْبِرُوٓا۟ ۖ إِنَّ ٱلْأَرْضَ لِلَّهِ يُورِثُهَا مَن يَشَآءُ مِنْ عِبَادِهِۦ ۖ وَٱلْعَـٰقِبَةُ لِلْمُتَّقِينَ ﴿١٢٨﴾\\
\textamh{129.\  } & قَالُوٓا۟ أُوذِينَا مِن قَبْلِ أَن تَأْتِيَنَا وَمِنۢ بَعْدِ مَا جِئْتَنَا ۚ قَالَ عَسَىٰ رَبُّكُمْ أَن يُهْلِكَ عَدُوَّكُمْ وَيَسْتَخْلِفَكُمْ فِى ٱلْأَرْضِ فَيَنظُرَ كَيْفَ تَعْمَلُونَ ﴿١٢٩﴾\\
\textamh{130.\  } & وَلَقَدْ أَخَذْنَآ ءَالَ فِرْعَوْنَ بِٱلسِّنِينَ وَنَقْصٍۢ مِّنَ ٱلثَّمَرَٰتِ لَعَلَّهُمْ يَذَّكَّرُونَ ﴿١٣٠﴾\\
\textamh{131.\  } & فَإِذَا جَآءَتْهُمُ ٱلْحَسَنَةُ قَالُوا۟ لَنَا هَـٰذِهِۦ ۖ وَإِن تُصِبْهُمْ سَيِّئَةٌۭ يَطَّيَّرُوا۟ بِمُوسَىٰ وَمَن مَّعَهُۥٓ ۗ أَلَآ إِنَّمَا طَٰٓئِرُهُمْ عِندَ ٱللَّهِ وَلَـٰكِنَّ أَكْثَرَهُمْ لَا يَعْلَمُونَ ﴿١٣١﴾\\
\textamh{132.\  } & وَقَالُوا۟ مَهْمَا تَأْتِنَا بِهِۦ مِنْ ءَايَةٍۢ لِّتَسْحَرَنَا بِهَا فَمَا نَحْنُ لَكَ بِمُؤْمِنِينَ ﴿١٣٢﴾\\
\textamh{133.\  } & فَأَرْسَلْنَا عَلَيْهِمُ ٱلطُّوفَانَ وَٱلْجَرَادَ وَٱلْقُمَّلَ وَٱلضَّفَادِعَ وَٱلدَّمَ ءَايَـٰتٍۢ مُّفَصَّلَـٰتٍۢ فَٱسْتَكْبَرُوا۟ وَكَانُوا۟ قَوْمًۭا مُّجْرِمِينَ ﴿١٣٣﴾\\
\textamh{134.\  } & وَلَمَّا وَقَعَ عَلَيْهِمُ ٱلرِّجْزُ قَالُوا۟ يَـٰمُوسَى ٱدْعُ لَنَا رَبَّكَ بِمَا عَهِدَ عِندَكَ ۖ لَئِن كَشَفْتَ عَنَّا ٱلرِّجْزَ لَنُؤْمِنَنَّ لَكَ وَلَنُرْسِلَنَّ مَعَكَ بَنِىٓ إِسْرَٰٓءِيلَ ﴿١٣٤﴾\\
\textamh{135.\  } & فَلَمَّا كَشَفْنَا عَنْهُمُ ٱلرِّجْزَ إِلَىٰٓ أَجَلٍ هُم بَٰلِغُوهُ إِذَا هُمْ يَنكُثُونَ ﴿١٣٥﴾\\
\textamh{136.\  } & فَٱنتَقَمْنَا مِنْهُمْ فَأَغْرَقْنَـٰهُمْ فِى ٱلْيَمِّ بِأَنَّهُمْ كَذَّبُوا۟ بِـَٔايَـٰتِنَا وَكَانُوا۟ عَنْهَا غَٰفِلِينَ ﴿١٣٦﴾\\
\textamh{137.\  } & وَأَوْرَثْنَا ٱلْقَوْمَ ٱلَّذِينَ كَانُوا۟ يُسْتَضْعَفُونَ مَشَـٰرِقَ ٱلْأَرْضِ وَمَغَٰرِبَهَا ٱلَّتِى بَٰرَكْنَا فِيهَا ۖ وَتَمَّتْ كَلِمَتُ رَبِّكَ ٱلْحُسْنَىٰ عَلَىٰ بَنِىٓ إِسْرَٰٓءِيلَ بِمَا صَبَرُوا۟ ۖ وَدَمَّرْنَا مَا كَانَ يَصْنَعُ فِرْعَوْنُ وَقَوْمُهُۥ وَمَا كَانُوا۟ يَعْرِشُونَ ﴿١٣٧﴾\\
\textamh{138.\  } & وَجَٰوَزْنَا بِبَنِىٓ إِسْرَٰٓءِيلَ ٱلْبَحْرَ فَأَتَوْا۟ عَلَىٰ قَوْمٍۢ يَعْكُفُونَ عَلَىٰٓ أَصْنَامٍۢ لَّهُمْ ۚ قَالُوا۟ يَـٰمُوسَى ٱجْعَل لَّنَآ إِلَـٰهًۭا كَمَا لَهُمْ ءَالِهَةٌۭ ۚ قَالَ إِنَّكُمْ قَوْمٌۭ تَجْهَلُونَ ﴿١٣٨﴾\\
\textamh{139.\  } & إِنَّ هَـٰٓؤُلَآءِ مُتَبَّرٌۭ مَّا هُمْ فِيهِ وَبَٰطِلٌۭ مَّا كَانُوا۟ يَعْمَلُونَ ﴿١٣٩﴾\\
\textamh{140.\  } & قَالَ أَغَيْرَ ٱللَّهِ أَبْغِيكُمْ إِلَـٰهًۭا وَهُوَ فَضَّلَكُمْ عَلَى ٱلْعَـٰلَمِينَ ﴿١٤٠﴾\\
\textamh{141.\  } & وَإِذْ أَنجَيْنَـٰكُم مِّنْ ءَالِ فِرْعَوْنَ يَسُومُونَكُمْ سُوٓءَ ٱلْعَذَابِ ۖ يُقَتِّلُونَ أَبْنَآءَكُمْ وَيَسْتَحْيُونَ نِسَآءَكُمْ ۚ وَفِى ذَٟلِكُم بَلَآءٌۭ مِّن رَّبِّكُمْ عَظِيمٌۭ ﴿١٤١﴾\\
\textamh{142.\  } & ۞ وَوَٟعَدْنَا مُوسَىٰ ثَلَـٰثِينَ لَيْلَةًۭ وَأَتْمَمْنَـٰهَا بِعَشْرٍۢ فَتَمَّ مِيقَـٰتُ رَبِّهِۦٓ أَرْبَعِينَ لَيْلَةًۭ ۚ وَقَالَ مُوسَىٰ لِأَخِيهِ هَـٰرُونَ ٱخْلُفْنِى فِى قَوْمِى وَأَصْلِحْ وَلَا تَتَّبِعْ سَبِيلَ ٱلْمُفْسِدِينَ ﴿١٤٢﴾\\
\textamh{143.\  } & وَلَمَّا جَآءَ مُوسَىٰ لِمِيقَـٰتِنَا وَكَلَّمَهُۥ رَبُّهُۥ قَالَ رَبِّ أَرِنِىٓ أَنظُرْ إِلَيْكَ ۚ قَالَ لَن تَرَىٰنِى وَلَـٰكِنِ ٱنظُرْ إِلَى ٱلْجَبَلِ فَإِنِ ٱسْتَقَرَّ مَكَانَهُۥ فَسَوْفَ تَرَىٰنِى ۚ فَلَمَّا تَجَلَّىٰ رَبُّهُۥ لِلْجَبَلِ جَعَلَهُۥ دَكًّۭا وَخَرَّ مُوسَىٰ صَعِقًۭا ۚ فَلَمَّآ أَفَاقَ قَالَ سُبْحَـٰنَكَ تُبْتُ إِلَيْكَ وَأَنَا۠ أَوَّلُ ٱلْمُؤْمِنِينَ ﴿١٤٣﴾\\
\textamh{144.\  } & قَالَ يَـٰمُوسَىٰٓ إِنِّى ٱصْطَفَيْتُكَ عَلَى ٱلنَّاسِ بِرِسَـٰلَـٰتِى وَبِكَلَـٰمِى فَخُذْ مَآ ءَاتَيْتُكَ وَكُن مِّنَ ٱلشَّـٰكِرِينَ ﴿١٤٤﴾\\
\textamh{145.\  } & وَكَتَبْنَا لَهُۥ فِى ٱلْأَلْوَاحِ مِن كُلِّ شَىْءٍۢ مَّوْعِظَةًۭ وَتَفْصِيلًۭا لِّكُلِّ شَىْءٍۢ فَخُذْهَا بِقُوَّةٍۢ وَأْمُرْ قَوْمَكَ يَأْخُذُوا۟ بِأَحْسَنِهَا ۚ سَأُو۟رِيكُمْ دَارَ ٱلْفَـٰسِقِينَ ﴿١٤٥﴾\\
\textamh{146.\  } & سَأَصْرِفُ عَنْ ءَايَـٰتِىَ ٱلَّذِينَ يَتَكَبَّرُونَ فِى ٱلْأَرْضِ بِغَيْرِ ٱلْحَقِّ وَإِن يَرَوْا۟ كُلَّ ءَايَةٍۢ لَّا يُؤْمِنُوا۟ بِهَا وَإِن يَرَوْا۟ سَبِيلَ ٱلرُّشْدِ لَا يَتَّخِذُوهُ سَبِيلًۭا وَإِن يَرَوْا۟ سَبِيلَ ٱلْغَىِّ يَتَّخِذُوهُ سَبِيلًۭا ۚ ذَٟلِكَ بِأَنَّهُمْ كَذَّبُوا۟ بِـَٔايَـٰتِنَا وَكَانُوا۟ عَنْهَا غَٰفِلِينَ ﴿١٤٦﴾\\
\textamh{147.\  } & وَٱلَّذِينَ كَذَّبُوا۟ بِـَٔايَـٰتِنَا وَلِقَآءِ ٱلْءَاخِرَةِ حَبِطَتْ أَعْمَـٰلُهُمْ ۚ هَلْ يُجْزَوْنَ إِلَّا مَا كَانُوا۟ يَعْمَلُونَ ﴿١٤٧﴾\\
\textamh{148.\  } & وَٱتَّخَذَ قَوْمُ مُوسَىٰ مِنۢ بَعْدِهِۦ مِنْ حُلِيِّهِمْ عِجْلًۭا جَسَدًۭا لَّهُۥ خُوَارٌ ۚ أَلَمْ يَرَوْا۟ أَنَّهُۥ لَا يُكَلِّمُهُمْ وَلَا يَهْدِيهِمْ سَبِيلًا ۘ ٱتَّخَذُوهُ وَكَانُوا۟ ظَـٰلِمِينَ ﴿١٤٨﴾\\
\textamh{149.\  } & وَلَمَّا سُقِطَ فِىٓ أَيْدِيهِمْ وَرَأَوْا۟ أَنَّهُمْ قَدْ ضَلُّوا۟ قَالُوا۟ لَئِن لَّمْ يَرْحَمْنَا رَبُّنَا وَيَغْفِرْ لَنَا لَنَكُونَنَّ مِنَ ٱلْخَـٰسِرِينَ ﴿١٤٩﴾\\
\textamh{150.\  } & وَلَمَّا رَجَعَ مُوسَىٰٓ إِلَىٰ قَوْمِهِۦ غَضْبَٰنَ أَسِفًۭا قَالَ بِئْسَمَا خَلَفْتُمُونِى مِنۢ بَعْدِىٓ ۖ أَعَجِلْتُمْ أَمْرَ رَبِّكُمْ ۖ وَأَلْقَى ٱلْأَلْوَاحَ وَأَخَذَ بِرَأْسِ أَخِيهِ يَجُرُّهُۥٓ إِلَيْهِ ۚ قَالَ ٱبْنَ أُمَّ إِنَّ ٱلْقَوْمَ ٱسْتَضْعَفُونِى وَكَادُوا۟ يَقْتُلُونَنِى فَلَا تُشْمِتْ بِىَ ٱلْأَعْدَآءَ وَلَا تَجْعَلْنِى مَعَ ٱلْقَوْمِ ٱلظَّـٰلِمِينَ ﴿١٥٠﴾\\
\textamh{151.\  } & قَالَ رَبِّ ٱغْفِرْ لِى وَلِأَخِى وَأَدْخِلْنَا فِى رَحْمَتِكَ ۖ وَأَنتَ أَرْحَمُ ٱلرَّٟحِمِينَ ﴿١٥١﴾\\
\textamh{152.\  } & إِنَّ ٱلَّذِينَ ٱتَّخَذُوا۟ ٱلْعِجْلَ سَيَنَالُهُمْ غَضَبٌۭ مِّن رَّبِّهِمْ وَذِلَّةٌۭ فِى ٱلْحَيَوٰةِ ٱلدُّنْيَا ۚ وَكَذَٟلِكَ نَجْزِى ٱلْمُفْتَرِينَ ﴿١٥٢﴾\\
\textamh{153.\  } & وَٱلَّذِينَ عَمِلُوا۟ ٱلسَّيِّـَٔاتِ ثُمَّ تَابُوا۟ مِنۢ بَعْدِهَا وَءَامَنُوٓا۟ إِنَّ رَبَّكَ مِنۢ بَعْدِهَا لَغَفُورٌۭ رَّحِيمٌۭ ﴿١٥٣﴾\\
\textamh{154.\  } & وَلَمَّا سَكَتَ عَن مُّوسَى ٱلْغَضَبُ أَخَذَ ٱلْأَلْوَاحَ ۖ وَفِى نُسْخَتِهَا هُدًۭى وَرَحْمَةٌۭ لِّلَّذِينَ هُمْ لِرَبِّهِمْ يَرْهَبُونَ ﴿١٥٤﴾\\
\textamh{155.\  } & وَٱخْتَارَ مُوسَىٰ قَوْمَهُۥ سَبْعِينَ رَجُلًۭا لِّمِيقَـٰتِنَا ۖ فَلَمَّآ أَخَذَتْهُمُ ٱلرَّجْفَةُ قَالَ رَبِّ لَوْ شِئْتَ أَهْلَكْتَهُم مِّن قَبْلُ وَإِيَّٰىَ ۖ أَتُهْلِكُنَا بِمَا فَعَلَ ٱلسُّفَهَآءُ مِنَّآ ۖ إِنْ هِىَ إِلَّا فِتْنَتُكَ تُضِلُّ بِهَا مَن تَشَآءُ وَتَهْدِى مَن تَشَآءُ ۖ أَنتَ وَلِيُّنَا فَٱغْفِرْ لَنَا وَٱرْحَمْنَا ۖ وَأَنتَ خَيْرُ ٱلْغَٰفِرِينَ ﴿١٥٥﴾\\
\textamh{156.\  } & ۞ وَٱكْتُبْ لَنَا فِى هَـٰذِهِ ٱلدُّنْيَا حَسَنَةًۭ وَفِى ٱلْءَاخِرَةِ إِنَّا هُدْنَآ إِلَيْكَ ۚ قَالَ عَذَابِىٓ أُصِيبُ بِهِۦ مَنْ أَشَآءُ ۖ وَرَحْمَتِى وَسِعَتْ كُلَّ شَىْءٍۢ ۚ فَسَأَكْتُبُهَا لِلَّذِينَ يَتَّقُونَ وَيُؤْتُونَ ٱلزَّكَوٰةَ وَٱلَّذِينَ هُم بِـَٔايَـٰتِنَا يُؤْمِنُونَ ﴿١٥٦﴾\\
\textamh{157.\  } & ٱلَّذِينَ يَتَّبِعُونَ ٱلرَّسُولَ ٱلنَّبِىَّ ٱلْأُمِّىَّ ٱلَّذِى يَجِدُونَهُۥ مَكْتُوبًا عِندَهُمْ فِى ٱلتَّوْرَىٰةِ وَٱلْإِنجِيلِ يَأْمُرُهُم بِٱلْمَعْرُوفِ وَيَنْهَىٰهُمْ عَنِ ٱلْمُنكَرِ وَيُحِلُّ لَهُمُ ٱلطَّيِّبَٰتِ وَيُحَرِّمُ عَلَيْهِمُ ٱلْخَبَٰٓئِثَ وَيَضَعُ عَنْهُمْ إِصْرَهُمْ وَٱلْأَغْلَـٰلَ ٱلَّتِى كَانَتْ عَلَيْهِمْ ۚ فَٱلَّذِينَ ءَامَنُوا۟ بِهِۦ وَعَزَّرُوهُ وَنَصَرُوهُ وَٱتَّبَعُوا۟ ٱلنُّورَ ٱلَّذِىٓ أُنزِلَ مَعَهُۥٓ ۙ أُو۟لَـٰٓئِكَ هُمُ ٱلْمُفْلِحُونَ ﴿١٥٧﴾\\
\textamh{158.\  } & قُلْ يَـٰٓأَيُّهَا ٱلنَّاسُ إِنِّى رَسُولُ ٱللَّهِ إِلَيْكُمْ جَمِيعًا ٱلَّذِى لَهُۥ مُلْكُ ٱلسَّمَـٰوَٟتِ وَٱلْأَرْضِ ۖ لَآ إِلَـٰهَ إِلَّا هُوَ يُحْىِۦ وَيُمِيتُ ۖ فَـَٔامِنُوا۟ بِٱللَّهِ وَرَسُولِهِ ٱلنَّبِىِّ ٱلْأُمِّىِّ ٱلَّذِى يُؤْمِنُ بِٱللَّهِ وَكَلِمَـٰتِهِۦ وَٱتَّبِعُوهُ لَعَلَّكُمْ تَهْتَدُونَ ﴿١٥٨﴾\\
\textamh{159.\  } & وَمِن قَوْمِ مُوسَىٰٓ أُمَّةٌۭ يَهْدُونَ بِٱلْحَقِّ وَبِهِۦ يَعْدِلُونَ ﴿١٥٩﴾\\
\textamh{160.\  } & وَقَطَّعْنَـٰهُمُ ٱثْنَتَىْ عَشْرَةَ أَسْبَاطًا أُمَمًۭا ۚ وَأَوْحَيْنَآ إِلَىٰ مُوسَىٰٓ إِذِ ٱسْتَسْقَىٰهُ قَوْمُهُۥٓ أَنِ ٱضْرِب بِّعَصَاكَ ٱلْحَجَرَ ۖ فَٱنۢبَجَسَتْ مِنْهُ ٱثْنَتَا عَشْرَةَ عَيْنًۭا ۖ قَدْ عَلِمَ كُلُّ أُنَاسٍۢ مَّشْرَبَهُمْ ۚ وَظَلَّلْنَا عَلَيْهِمُ ٱلْغَمَـٰمَ وَأَنزَلْنَا عَلَيْهِمُ ٱلْمَنَّ وَٱلسَّلْوَىٰ ۖ كُلُوا۟ مِن طَيِّبَٰتِ مَا رَزَقْنَـٰكُمْ ۚ وَمَا ظَلَمُونَا وَلَـٰكِن كَانُوٓا۟ أَنفُسَهُمْ يَظْلِمُونَ ﴿١٦٠﴾\\
\textamh{161.\  } & وَإِذْ قِيلَ لَهُمُ ٱسْكُنُوا۟ هَـٰذِهِ ٱلْقَرْيَةَ وَكُلُوا۟ مِنْهَا حَيْثُ شِئْتُمْ وَقُولُوا۟ حِطَّةٌۭ وَٱدْخُلُوا۟ ٱلْبَابَ سُجَّدًۭا نَّغْفِرْ لَكُمْ خَطِيٓـَٰٔتِكُمْ ۚ سَنَزِيدُ ٱلْمُحْسِنِينَ ﴿١٦١﴾\\
\textamh{162.\  } & فَبَدَّلَ ٱلَّذِينَ ظَلَمُوا۟ مِنْهُمْ قَوْلًا غَيْرَ ٱلَّذِى قِيلَ لَهُمْ فَأَرْسَلْنَا عَلَيْهِمْ رِجْزًۭا مِّنَ ٱلسَّمَآءِ بِمَا كَانُوا۟ يَظْلِمُونَ ﴿١٦٢﴾\\
\textamh{163.\  } & وَسْـَٔلْهُمْ عَنِ ٱلْقَرْيَةِ ٱلَّتِى كَانَتْ حَاضِرَةَ ٱلْبَحْرِ إِذْ يَعْدُونَ فِى ٱلسَّبْتِ إِذْ تَأْتِيهِمْ حِيتَانُهُمْ يَوْمَ سَبْتِهِمْ شُرَّعًۭا وَيَوْمَ لَا يَسْبِتُونَ ۙ لَا تَأْتِيهِمْ ۚ كَذَٟلِكَ نَبْلُوهُم بِمَا كَانُوا۟ يَفْسُقُونَ ﴿١٦٣﴾\\
\textamh{164.\  } & وَإِذْ قَالَتْ أُمَّةٌۭ مِّنْهُمْ لِمَ تَعِظُونَ قَوْمًا ۙ ٱللَّهُ مُهْلِكُهُمْ أَوْ مُعَذِّبُهُمْ عَذَابًۭا شَدِيدًۭا ۖ قَالُوا۟ مَعْذِرَةً إِلَىٰ رَبِّكُمْ وَلَعَلَّهُمْ يَتَّقُونَ ﴿١٦٤﴾\\
\textamh{165.\  } & فَلَمَّا نَسُوا۟ مَا ذُكِّرُوا۟ بِهِۦٓ أَنجَيْنَا ٱلَّذِينَ يَنْهَوْنَ عَنِ ٱلسُّوٓءِ وَأَخَذْنَا ٱلَّذِينَ ظَلَمُوا۟ بِعَذَابٍۭ بَـِٔيسٍۭ بِمَا كَانُوا۟ يَفْسُقُونَ ﴿١٦٥﴾\\
\textamh{166.\  } & فَلَمَّا عَتَوْا۟ عَن مَّا نُهُوا۟ عَنْهُ قُلْنَا لَهُمْ كُونُوا۟ قِرَدَةً خَـٰسِـِٔينَ ﴿١٦٦﴾\\
\textamh{167.\  } & وَإِذْ تَأَذَّنَ رَبُّكَ لَيَبْعَثَنَّ عَلَيْهِمْ إِلَىٰ يَوْمِ ٱلْقِيَـٰمَةِ مَن يَسُومُهُمْ سُوٓءَ ٱلْعَذَابِ ۗ إِنَّ رَبَّكَ لَسَرِيعُ ٱلْعِقَابِ ۖ وَإِنَّهُۥ لَغَفُورٌۭ رَّحِيمٌۭ ﴿١٦٧﴾\\
\textamh{168.\  } & وَقَطَّعْنَـٰهُمْ فِى ٱلْأَرْضِ أُمَمًۭا ۖ مِّنْهُمُ ٱلصَّـٰلِحُونَ وَمِنْهُمْ دُونَ ذَٟلِكَ ۖ وَبَلَوْنَـٰهُم بِٱلْحَسَنَـٰتِ وَٱلسَّيِّـَٔاتِ لَعَلَّهُمْ يَرْجِعُونَ ﴿١٦٨﴾\\
\textamh{169.\  } & فَخَلَفَ مِنۢ بَعْدِهِمْ خَلْفٌۭ وَرِثُوا۟ ٱلْكِتَـٰبَ يَأْخُذُونَ عَرَضَ هَـٰذَا ٱلْأَدْنَىٰ وَيَقُولُونَ سَيُغْفَرُ لَنَا وَإِن يَأْتِهِمْ عَرَضٌۭ مِّثْلُهُۥ يَأْخُذُوهُ ۚ أَلَمْ يُؤْخَذْ عَلَيْهِم مِّيثَـٰقُ ٱلْكِتَـٰبِ أَن لَّا يَقُولُوا۟ عَلَى ٱللَّهِ إِلَّا ٱلْحَقَّ وَدَرَسُوا۟ مَا فِيهِ ۗ وَٱلدَّارُ ٱلْءَاخِرَةُ خَيْرٌۭ لِّلَّذِينَ يَتَّقُونَ ۗ أَفَلَا تَعْقِلُونَ ﴿١٦٩﴾\\
\textamh{170.\  } & وَٱلَّذِينَ يُمَسِّكُونَ بِٱلْكِتَـٰبِ وَأَقَامُوا۟ ٱلصَّلَوٰةَ إِنَّا لَا نُضِيعُ أَجْرَ ٱلْمُصْلِحِينَ ﴿١٧٠﴾\\
\textamh{171.\  } & ۞ وَإِذْ نَتَقْنَا ٱلْجَبَلَ فَوْقَهُمْ كَأَنَّهُۥ ظُلَّةٌۭ وَظَنُّوٓا۟ أَنَّهُۥ وَاقِعٌۢ بِهِمْ خُذُوا۟ مَآ ءَاتَيْنَـٰكُم بِقُوَّةٍۢ وَٱذْكُرُوا۟ مَا فِيهِ لَعَلَّكُمْ تَتَّقُونَ ﴿١٧١﴾\\
\textamh{172.\  } & وَإِذْ أَخَذَ رَبُّكَ مِنۢ بَنِىٓ ءَادَمَ مِن ظُهُورِهِمْ ذُرِّيَّتَهُمْ وَأَشْهَدَهُمْ عَلَىٰٓ أَنفُسِهِمْ أَلَسْتُ بِرَبِّكُمْ ۖ قَالُوا۟ بَلَىٰ ۛ شَهِدْنَآ ۛ أَن تَقُولُوا۟ يَوْمَ ٱلْقِيَـٰمَةِ إِنَّا كُنَّا عَنْ هَـٰذَا غَٰفِلِينَ ﴿١٧٢﴾\\
\textamh{173.\  } & أَوْ تَقُولُوٓا۟ إِنَّمَآ أَشْرَكَ ءَابَآؤُنَا مِن قَبْلُ وَكُنَّا ذُرِّيَّةًۭ مِّنۢ بَعْدِهِمْ ۖ أَفَتُهْلِكُنَا بِمَا فَعَلَ ٱلْمُبْطِلُونَ ﴿١٧٣﴾\\
\textamh{174.\  } & وَكَذَٟلِكَ نُفَصِّلُ ٱلْءَايَـٰتِ وَلَعَلَّهُمْ يَرْجِعُونَ ﴿١٧٤﴾\\
\textamh{175.\  } & وَٱتْلُ عَلَيْهِمْ نَبَأَ ٱلَّذِىٓ ءَاتَيْنَـٰهُ ءَايَـٰتِنَا فَٱنسَلَخَ مِنْهَا فَأَتْبَعَهُ ٱلشَّيْطَٰنُ فَكَانَ مِنَ ٱلْغَاوِينَ ﴿١٧٥﴾\\
\textamh{176.\  } & وَلَوْ شِئْنَا لَرَفَعْنَـٰهُ بِهَا وَلَـٰكِنَّهُۥٓ أَخْلَدَ إِلَى ٱلْأَرْضِ وَٱتَّبَعَ هَوَىٰهُ ۚ فَمَثَلُهُۥ كَمَثَلِ ٱلْكَلْبِ إِن تَحْمِلْ عَلَيْهِ يَلْهَثْ أَوْ تَتْرُكْهُ يَلْهَث ۚ ذَّٰلِكَ مَثَلُ ٱلْقَوْمِ ٱلَّذِينَ كَذَّبُوا۟ بِـَٔايَـٰتِنَا ۚ فَٱقْصُصِ ٱلْقَصَصَ لَعَلَّهُمْ يَتَفَكَّرُونَ ﴿١٧٦﴾\\
\textamh{177.\  } & سَآءَ مَثَلًا ٱلْقَوْمُ ٱلَّذِينَ كَذَّبُوا۟ بِـَٔايَـٰتِنَا وَأَنفُسَهُمْ كَانُوا۟ يَظْلِمُونَ ﴿١٧٧﴾\\
\textamh{178.\  } & مَن يَهْدِ ٱللَّهُ فَهُوَ ٱلْمُهْتَدِى ۖ وَمَن يُضْلِلْ فَأُو۟لَـٰٓئِكَ هُمُ ٱلْخَـٰسِرُونَ ﴿١٧٨﴾\\
\textamh{179.\  } & وَلَقَدْ ذَرَأْنَا لِجَهَنَّمَ كَثِيرًۭا مِّنَ ٱلْجِنِّ وَٱلْإِنسِ ۖ لَهُمْ قُلُوبٌۭ لَّا يَفْقَهُونَ بِهَا وَلَهُمْ أَعْيُنٌۭ لَّا يُبْصِرُونَ بِهَا وَلَهُمْ ءَاذَانٌۭ لَّا يَسْمَعُونَ بِهَآ ۚ أُو۟لَـٰٓئِكَ كَٱلْأَنْعَـٰمِ بَلْ هُمْ أَضَلُّ ۚ أُو۟لَـٰٓئِكَ هُمُ ٱلْغَٰفِلُونَ ﴿١٧٩﴾\\
\textamh{180.\  } & وَلِلَّهِ ٱلْأَسْمَآءُ ٱلْحُسْنَىٰ فَٱدْعُوهُ بِهَا ۖ وَذَرُوا۟ ٱلَّذِينَ يُلْحِدُونَ فِىٓ أَسْمَـٰٓئِهِۦ ۚ سَيُجْزَوْنَ مَا كَانُوا۟ يَعْمَلُونَ ﴿١٨٠﴾\\
\textamh{181.\  } & وَمِمَّنْ خَلَقْنَآ أُمَّةٌۭ يَهْدُونَ بِٱلْحَقِّ وَبِهِۦ يَعْدِلُونَ ﴿١٨١﴾\\
\textamh{182.\  } & وَٱلَّذِينَ كَذَّبُوا۟ بِـَٔايَـٰتِنَا سَنَسْتَدْرِجُهُم مِّنْ حَيْثُ لَا يَعْلَمُونَ ﴿١٨٢﴾\\
\textamh{183.\  } & وَأُمْلِى لَهُمْ ۚ إِنَّ كَيْدِى مَتِينٌ ﴿١٨٣﴾\\
\textamh{184.\  } & أَوَلَمْ يَتَفَكَّرُوا۟ ۗ مَا بِصَاحِبِهِم مِّن جِنَّةٍ ۚ إِنْ هُوَ إِلَّا نَذِيرٌۭ مُّبِينٌ ﴿١٨٤﴾\\
\textamh{185.\  } & أَوَلَمْ يَنظُرُوا۟ فِى مَلَكُوتِ ٱلسَّمَـٰوَٟتِ وَٱلْأَرْضِ وَمَا خَلَقَ ٱللَّهُ مِن شَىْءٍۢ وَأَنْ عَسَىٰٓ أَن يَكُونَ قَدِ ٱقْتَرَبَ أَجَلُهُمْ ۖ فَبِأَىِّ حَدِيثٍۭ بَعْدَهُۥ يُؤْمِنُونَ ﴿١٨٥﴾\\
\textamh{186.\  } & مَن يُضْلِلِ ٱللَّهُ فَلَا هَادِىَ لَهُۥ ۚ وَيَذَرُهُمْ فِى طُغْيَـٰنِهِمْ يَعْمَهُونَ ﴿١٨٦﴾\\
\textamh{187.\  } & يَسْـَٔلُونَكَ عَنِ ٱلسَّاعَةِ أَيَّانَ مُرْسَىٰهَا ۖ قُلْ إِنَّمَا عِلْمُهَا عِندَ رَبِّى ۖ لَا يُجَلِّيهَا لِوَقْتِهَآ إِلَّا هُوَ ۚ ثَقُلَتْ فِى ٱلسَّمَـٰوَٟتِ وَٱلْأَرْضِ ۚ لَا تَأْتِيكُمْ إِلَّا بَغْتَةًۭ ۗ يَسْـَٔلُونَكَ كَأَنَّكَ حَفِىٌّ عَنْهَا ۖ قُلْ إِنَّمَا عِلْمُهَا عِندَ ٱللَّهِ وَلَـٰكِنَّ أَكْثَرَ ٱلنَّاسِ لَا يَعْلَمُونَ ﴿١٨٧﴾\\
\textamh{188.\  } & قُل لَّآ أَمْلِكُ لِنَفْسِى نَفْعًۭا وَلَا ضَرًّا إِلَّا مَا شَآءَ ٱللَّهُ ۚ وَلَوْ كُنتُ أَعْلَمُ ٱلْغَيْبَ لَٱسْتَكْثَرْتُ مِنَ ٱلْخَيْرِ وَمَا مَسَّنِىَ ٱلسُّوٓءُ ۚ إِنْ أَنَا۠ إِلَّا نَذِيرٌۭ وَبَشِيرٌۭ لِّقَوْمٍۢ يُؤْمِنُونَ ﴿١٨٨﴾\\
\textamh{189.\  } & ۞ هُوَ ٱلَّذِى خَلَقَكُم مِّن نَّفْسٍۢ وَٟحِدَةٍۢ وَجَعَلَ مِنْهَا زَوْجَهَا لِيَسْكُنَ إِلَيْهَا ۖ فَلَمَّا تَغَشَّىٰهَا حَمَلَتْ حَمْلًا خَفِيفًۭا فَمَرَّتْ بِهِۦ ۖ فَلَمَّآ أَثْقَلَت دَّعَوَا ٱللَّهَ رَبَّهُمَا لَئِنْ ءَاتَيْتَنَا صَـٰلِحًۭا لَّنَكُونَنَّ مِنَ ٱلشَّـٰكِرِينَ ﴿١٨٩﴾\\
\textamh{190.\  } & فَلَمَّآ ءَاتَىٰهُمَا صَـٰلِحًۭا جَعَلَا لَهُۥ شُرَكَآءَ فِيمَآ ءَاتَىٰهُمَا ۚ فَتَعَـٰلَى ٱللَّهُ عَمَّا يُشْرِكُونَ ﴿١٩٠﴾\\
\textamh{191.\  } & أَيُشْرِكُونَ مَا لَا يَخْلُقُ شَيْـًۭٔا وَهُمْ يُخْلَقُونَ ﴿١٩١﴾\\
\textamh{192.\  } & وَلَا يَسْتَطِيعُونَ لَهُمْ نَصْرًۭا وَلَآ أَنفُسَهُمْ يَنصُرُونَ ﴿١٩٢﴾\\
\textamh{193.\  } & وَإِن تَدْعُوهُمْ إِلَى ٱلْهُدَىٰ لَا يَتَّبِعُوكُمْ ۚ سَوَآءٌ عَلَيْكُمْ أَدَعَوْتُمُوهُمْ أَمْ أَنتُمْ صَـٰمِتُونَ ﴿١٩٣﴾\\
\textamh{194.\  } & إِنَّ ٱلَّذِينَ تَدْعُونَ مِن دُونِ ٱللَّهِ عِبَادٌ أَمْثَالُكُمْ ۖ فَٱدْعُوهُمْ فَلْيَسْتَجِيبُوا۟ لَكُمْ إِن كُنتُمْ صَـٰدِقِينَ ﴿١٩٤﴾\\
\textamh{195.\  } & أَلَهُمْ أَرْجُلٌۭ يَمْشُونَ بِهَآ ۖ أَمْ لَهُمْ أَيْدٍۢ يَبْطِشُونَ بِهَآ ۖ أَمْ لَهُمْ أَعْيُنٌۭ يُبْصِرُونَ بِهَآ ۖ أَمْ لَهُمْ ءَاذَانٌۭ يَسْمَعُونَ بِهَا ۗ قُلِ ٱدْعُوا۟ شُرَكَآءَكُمْ ثُمَّ كِيدُونِ فَلَا تُنظِرُونِ ﴿١٩٥﴾\\
\textamh{196.\  } & إِنَّ وَلِۦِّىَ ٱللَّهُ ٱلَّذِى نَزَّلَ ٱلْكِتَـٰبَ ۖ وَهُوَ يَتَوَلَّى ٱلصَّـٰلِحِينَ ﴿١٩٦﴾\\
\textamh{197.\  } & وَٱلَّذِينَ تَدْعُونَ مِن دُونِهِۦ لَا يَسْتَطِيعُونَ نَصْرَكُمْ وَلَآ أَنفُسَهُمْ يَنصُرُونَ ﴿١٩٧﴾\\
\textamh{198.\  } & وَإِن تَدْعُوهُمْ إِلَى ٱلْهُدَىٰ لَا يَسْمَعُوا۟ ۖ وَتَرَىٰهُمْ يَنظُرُونَ إِلَيْكَ وَهُمْ لَا يُبْصِرُونَ ﴿١٩٨﴾\\
\textamh{199.\  } & خُذِ ٱلْعَفْوَ وَأْمُرْ بِٱلْعُرْفِ وَأَعْرِضْ عَنِ ٱلْجَٰهِلِينَ ﴿١٩٩﴾\\
\textamh{200.\  } & وَإِمَّا يَنزَغَنَّكَ مِنَ ٱلشَّيْطَٰنِ نَزْغٌۭ فَٱسْتَعِذْ بِٱللَّهِ ۚ إِنَّهُۥ سَمِيعٌ عَلِيمٌ ﴿٢٠٠﴾\\
\textamh{201.\  } & إِنَّ ٱلَّذِينَ ٱتَّقَوْا۟ إِذَا مَسَّهُمْ طَٰٓئِفٌۭ مِّنَ ٱلشَّيْطَٰنِ تَذَكَّرُوا۟ فَإِذَا هُم مُّبْصِرُونَ ﴿٢٠١﴾\\
\textamh{202.\  } & وَإِخْوَٟنُهُمْ يَمُدُّونَهُمْ فِى ٱلْغَىِّ ثُمَّ لَا يُقْصِرُونَ ﴿٢٠٢﴾\\
\textamh{203.\  } & وَإِذَا لَمْ تَأْتِهِم بِـَٔايَةٍۢ قَالُوا۟ لَوْلَا ٱجْتَبَيْتَهَا ۚ قُلْ إِنَّمَآ أَتَّبِعُ مَا يُوحَىٰٓ إِلَىَّ مِن رَّبِّى ۚ هَـٰذَا بَصَآئِرُ مِن رَّبِّكُمْ وَهُدًۭى وَرَحْمَةٌۭ لِّقَوْمٍۢ يُؤْمِنُونَ ﴿٢٠٣﴾\\
\textamh{204.\  } & وَإِذَا قُرِئَ ٱلْقُرْءَانُ فَٱسْتَمِعُوا۟ لَهُۥ وَأَنصِتُوا۟ لَعَلَّكُمْ تُرْحَمُونَ ﴿٢٠٤﴾\\
\textamh{205.\  } & وَٱذْكُر رَّبَّكَ فِى نَفْسِكَ تَضَرُّعًۭا وَخِيفَةًۭ وَدُونَ ٱلْجَهْرِ مِنَ ٱلْقَوْلِ بِٱلْغُدُوِّ وَٱلْءَاصَالِ وَلَا تَكُن مِّنَ ٱلْغَٰفِلِينَ ﴿٢٠٥﴾\\
\textamh{206.\  } & إِنَّ ٱلَّذِينَ عِندَ رَبِّكَ لَا يَسْتَكْبِرُونَ عَنْ عِبَادَتِهِۦ وَيُسَبِّحُونَهُۥ وَلَهُۥ يَسْجُدُونَ ۩ ﴿٢٠٦﴾\\
\end{longtable}
\clearpage
%% License: BSD style (Berkley) (i.e. Put the Copyright owner's name always)
%% Writer and Copyright (to): Bewketu(Bilal) Tadilo (2016-17)
\centering\section{\LR{\textamharic{ሱራቱ አልአንፋል -}  \RL{سوره  الأنفال}}}
\begin{longtable}{%
  @{}
    p{.5\textwidth}
  @{~~~~~~~~~~~~~}
    p{.5\textwidth}
    @{}
}
\nopagebreak
\textamh{ቢስሚላሂ አራህመኒ ራሂይም } &  بِسْمِ ٱللَّهِ ٱلرَّحْمَـٰنِ ٱلرَّحِيمِ\\
\textamh{1.\  } &  يَسْـَٔلُونَكَ عَنِ ٱلْأَنفَالِ ۖ قُلِ ٱلْأَنفَالُ لِلَّهِ وَٱلرَّسُولِ ۖ فَٱتَّقُوا۟ ٱللَّهَ وَأَصْلِحُوا۟ ذَاتَ بَيْنِكُمْ ۖ وَأَطِيعُوا۟ ٱللَّهَ وَرَسُولَهُۥٓ إِن كُنتُم مُّؤْمِنِينَ ﴿١﴾\\
\textamh{2.\  } & إِنَّمَا ٱلْمُؤْمِنُونَ ٱلَّذِينَ إِذَا ذُكِرَ ٱللَّهُ وَجِلَتْ قُلُوبُهُمْ وَإِذَا تُلِيَتْ عَلَيْهِمْ ءَايَـٰتُهُۥ زَادَتْهُمْ إِيمَـٰنًۭا وَعَلَىٰ رَبِّهِمْ يَتَوَكَّلُونَ ﴿٢﴾\\
\textamh{3.\  } & ٱلَّذِينَ يُقِيمُونَ ٱلصَّلَوٰةَ وَمِمَّا رَزَقْنَـٰهُمْ يُنفِقُونَ ﴿٣﴾\\
\textamh{4.\  } & أُو۟لَـٰٓئِكَ هُمُ ٱلْمُؤْمِنُونَ حَقًّۭا ۚ لَّهُمْ دَرَجَٰتٌ عِندَ رَبِّهِمْ وَمَغْفِرَةٌۭ وَرِزْقٌۭ كَرِيمٌۭ ﴿٤﴾\\
\textamh{5.\  } & كَمَآ أَخْرَجَكَ رَبُّكَ مِنۢ بَيْتِكَ بِٱلْحَقِّ وَإِنَّ فَرِيقًۭا مِّنَ ٱلْمُؤْمِنِينَ لَكَـٰرِهُونَ ﴿٥﴾\\
\textamh{6.\  } & يُجَٰدِلُونَكَ فِى ٱلْحَقِّ بَعْدَمَا تَبَيَّنَ كَأَنَّمَا يُسَاقُونَ إِلَى ٱلْمَوْتِ وَهُمْ يَنظُرُونَ ﴿٦﴾\\
\textamh{7.\  } & وَإِذْ يَعِدُكُمُ ٱللَّهُ إِحْدَى ٱلطَّآئِفَتَيْنِ أَنَّهَا لَكُمْ وَتَوَدُّونَ أَنَّ غَيْرَ ذَاتِ ٱلشَّوْكَةِ تَكُونُ لَكُمْ وَيُرِيدُ ٱللَّهُ أَن يُحِقَّ ٱلْحَقَّ بِكَلِمَـٰتِهِۦ وَيَقْطَعَ دَابِرَ ٱلْكَـٰفِرِينَ ﴿٧﴾\\
\textamh{8.\  } & لِيُحِقَّ ٱلْحَقَّ وَيُبْطِلَ ٱلْبَٰطِلَ وَلَوْ كَرِهَ ٱلْمُجْرِمُونَ ﴿٨﴾\\
\textamh{9.\  } & إِذْ تَسْتَغِيثُونَ رَبَّكُمْ فَٱسْتَجَابَ لَكُمْ أَنِّى مُمِدُّكُم بِأَلْفٍۢ مِّنَ ٱلْمَلَـٰٓئِكَةِ مُرْدِفِينَ ﴿٩﴾\\
\textamh{10.\  } & وَمَا جَعَلَهُ ٱللَّهُ إِلَّا بُشْرَىٰ وَلِتَطْمَئِنَّ بِهِۦ قُلُوبُكُمْ ۚ وَمَا ٱلنَّصْرُ إِلَّا مِنْ عِندِ ٱللَّهِ ۚ إِنَّ ٱللَّهَ عَزِيزٌ حَكِيمٌ ﴿١٠﴾\\
\textamh{11.\  } & إِذْ يُغَشِّيكُمُ ٱلنُّعَاسَ أَمَنَةًۭ مِّنْهُ وَيُنَزِّلُ عَلَيْكُم مِّنَ ٱلسَّمَآءِ مَآءًۭ لِّيُطَهِّرَكُم بِهِۦ وَيُذْهِبَ عَنكُمْ رِجْزَ ٱلشَّيْطَٰنِ وَلِيَرْبِطَ عَلَىٰ قُلُوبِكُمْ وَيُثَبِّتَ بِهِ ٱلْأَقْدَامَ ﴿١١﴾\\
\textamh{12.\  } & إِذْ يُوحِى رَبُّكَ إِلَى ٱلْمَلَـٰٓئِكَةِ أَنِّى مَعَكُمْ فَثَبِّتُوا۟ ٱلَّذِينَ ءَامَنُوا۟ ۚ سَأُلْقِى فِى قُلُوبِ ٱلَّذِينَ كَفَرُوا۟ ٱلرُّعْبَ فَٱضْرِبُوا۟ فَوْقَ ٱلْأَعْنَاقِ وَٱضْرِبُوا۟ مِنْهُمْ كُلَّ بَنَانٍۢ ﴿١٢﴾\\
\textamh{13.\  } & ذَٟلِكَ بِأَنَّهُمْ شَآقُّوا۟ ٱللَّهَ وَرَسُولَهُۥ ۚ وَمَن يُشَاقِقِ ٱللَّهَ وَرَسُولَهُۥ فَإِنَّ ٱللَّهَ شَدِيدُ ٱلْعِقَابِ ﴿١٣﴾\\
\textamh{14.\  } & ذَٟلِكُمْ فَذُوقُوهُ وَأَنَّ لِلْكَـٰفِرِينَ عَذَابَ ٱلنَّارِ ﴿١٤﴾\\
\textamh{15.\  } & يَـٰٓأَيُّهَا ٱلَّذِينَ ءَامَنُوٓا۟ إِذَا لَقِيتُمُ ٱلَّذِينَ كَفَرُوا۟ زَحْفًۭا فَلَا تُوَلُّوهُمُ ٱلْأَدْبَارَ ﴿١٥﴾\\
\textamh{16.\  } & وَمَن يُوَلِّهِمْ يَوْمَئِذٍۢ دُبُرَهُۥٓ إِلَّا مُتَحَرِّفًۭا لِّقِتَالٍ أَوْ مُتَحَيِّزًا إِلَىٰ فِئَةٍۢ فَقَدْ بَآءَ بِغَضَبٍۢ مِّنَ ٱللَّهِ وَمَأْوَىٰهُ جَهَنَّمُ ۖ وَبِئْسَ ٱلْمَصِيرُ ﴿١٦﴾\\
\textamh{17.\  } & فَلَمْ تَقْتُلُوهُمْ وَلَـٰكِنَّ ٱللَّهَ قَتَلَهُمْ ۚ وَمَا رَمَيْتَ إِذْ رَمَيْتَ وَلَـٰكِنَّ ٱللَّهَ رَمَىٰ ۚ وَلِيُبْلِىَ ٱلْمُؤْمِنِينَ مِنْهُ بَلَآءً حَسَنًا ۚ إِنَّ ٱللَّهَ سَمِيعٌ عَلِيمٌۭ ﴿١٧﴾\\
\textamh{18.\  } & ذَٟلِكُمْ وَأَنَّ ٱللَّهَ مُوهِنُ كَيْدِ ٱلْكَـٰفِرِينَ ﴿١٨﴾\\
\textamh{19.\  } & إِن تَسْتَفْتِحُوا۟ فَقَدْ جَآءَكُمُ ٱلْفَتْحُ ۖ وَإِن تَنتَهُوا۟ فَهُوَ خَيْرٌۭ لَّكُمْ ۖ وَإِن تَعُودُوا۟ نَعُدْ وَلَن تُغْنِىَ عَنكُمْ فِئَتُكُمْ شَيْـًۭٔا وَلَوْ كَثُرَتْ وَأَنَّ ٱللَّهَ مَعَ ٱلْمُؤْمِنِينَ ﴿١٩﴾\\
\textamh{20.\  } & يَـٰٓأَيُّهَا ٱلَّذِينَ ءَامَنُوٓا۟ أَطِيعُوا۟ ٱللَّهَ وَرَسُولَهُۥ وَلَا تَوَلَّوْا۟ عَنْهُ وَأَنتُمْ تَسْمَعُونَ ﴿٢٠﴾\\
\textamh{21.\  } & وَلَا تَكُونُوا۟ كَٱلَّذِينَ قَالُوا۟ سَمِعْنَا وَهُمْ لَا يَسْمَعُونَ ﴿٢١﴾\\
\textamh{22.\  } & ۞ إِنَّ شَرَّ ٱلدَّوَآبِّ عِندَ ٱللَّهِ ٱلصُّمُّ ٱلْبُكْمُ ٱلَّذِينَ لَا يَعْقِلُونَ ﴿٢٢﴾\\
\textamh{23.\  } & وَلَوْ عَلِمَ ٱللَّهُ فِيهِمْ خَيْرًۭا لَّأَسْمَعَهُمْ ۖ وَلَوْ أَسْمَعَهُمْ لَتَوَلَّوا۟ وَّهُم مُّعْرِضُونَ ﴿٢٣﴾\\
\textamh{24.\  } & يَـٰٓأَيُّهَا ٱلَّذِينَ ءَامَنُوا۟ ٱسْتَجِيبُوا۟ لِلَّهِ وَلِلرَّسُولِ إِذَا دَعَاكُمْ لِمَا يُحْيِيكُمْ ۖ وَٱعْلَمُوٓا۟ أَنَّ ٱللَّهَ يَحُولُ بَيْنَ ٱلْمَرْءِ وَقَلْبِهِۦ وَأَنَّهُۥٓ إِلَيْهِ تُحْشَرُونَ ﴿٢٤﴾\\
\textamh{25.\  } & وَٱتَّقُوا۟ فِتْنَةًۭ لَّا تُصِيبَنَّ ٱلَّذِينَ ظَلَمُوا۟ مِنكُمْ خَآصَّةًۭ ۖ وَٱعْلَمُوٓا۟ أَنَّ ٱللَّهَ شَدِيدُ ٱلْعِقَابِ ﴿٢٥﴾\\
\textamh{26.\  } & وَٱذْكُرُوٓا۟ إِذْ أَنتُمْ قَلِيلٌۭ مُّسْتَضْعَفُونَ فِى ٱلْأَرْضِ تَخَافُونَ أَن يَتَخَطَّفَكُمُ ٱلنَّاسُ فَـَٔاوَىٰكُمْ وَأَيَّدَكُم بِنَصْرِهِۦ وَرَزَقَكُم مِّنَ ٱلطَّيِّبَٰتِ لَعَلَّكُمْ تَشْكُرُونَ ﴿٢٦﴾\\
\textamh{27.\  } & يَـٰٓأَيُّهَا ٱلَّذِينَ ءَامَنُوا۟ لَا تَخُونُوا۟ ٱللَّهَ وَٱلرَّسُولَ وَتَخُونُوٓا۟ أَمَـٰنَـٰتِكُمْ وَأَنتُمْ تَعْلَمُونَ ﴿٢٧﴾\\
\textamh{28.\  } & وَٱعْلَمُوٓا۟ أَنَّمَآ أَمْوَٟلُكُمْ وَأَوْلَـٰدُكُمْ فِتْنَةٌۭ وَأَنَّ ٱللَّهَ عِندَهُۥٓ أَجْرٌ عَظِيمٌۭ ﴿٢٨﴾\\
\textamh{29.\  } & يَـٰٓأَيُّهَا ٱلَّذِينَ ءَامَنُوٓا۟ إِن تَتَّقُوا۟ ٱللَّهَ يَجْعَل لَّكُمْ فُرْقَانًۭا وَيُكَفِّرْ عَنكُمْ سَيِّـَٔاتِكُمْ وَيَغْفِرْ لَكُمْ ۗ وَٱللَّهُ ذُو ٱلْفَضْلِ ٱلْعَظِيمِ ﴿٢٩﴾\\
\textamh{30.\  } & وَإِذْ يَمْكُرُ بِكَ ٱلَّذِينَ كَفَرُوا۟ لِيُثْبِتُوكَ أَوْ يَقْتُلُوكَ أَوْ يُخْرِجُوكَ ۚ وَيَمْكُرُونَ وَيَمْكُرُ ٱللَّهُ ۖ وَٱللَّهُ خَيْرُ ٱلْمَـٰكِرِينَ ﴿٣٠﴾\\
\textamh{31.\  } & وَإِذَا تُتْلَىٰ عَلَيْهِمْ ءَايَـٰتُنَا قَالُوا۟ قَدْ سَمِعْنَا لَوْ نَشَآءُ لَقُلْنَا مِثْلَ هَـٰذَآ ۙ إِنْ هَـٰذَآ إِلَّآ أَسَـٰطِيرُ ٱلْأَوَّلِينَ ﴿٣١﴾\\
\textamh{32.\  } & وَإِذْ قَالُوا۟ ٱللَّهُمَّ إِن كَانَ هَـٰذَا هُوَ ٱلْحَقَّ مِنْ عِندِكَ فَأَمْطِرْ عَلَيْنَا حِجَارَةًۭ مِّنَ ٱلسَّمَآءِ أَوِ ٱئْتِنَا بِعَذَابٍ أَلِيمٍۢ ﴿٣٢﴾\\
\textamh{33.\  } & وَمَا كَانَ ٱللَّهُ لِيُعَذِّبَهُمْ وَأَنتَ فِيهِمْ ۚ وَمَا كَانَ ٱللَّهُ مُعَذِّبَهُمْ وَهُمْ يَسْتَغْفِرُونَ ﴿٣٣﴾\\
\textamh{34.\  } & وَمَا لَهُمْ أَلَّا يُعَذِّبَهُمُ ٱللَّهُ وَهُمْ يَصُدُّونَ عَنِ ٱلْمَسْجِدِ ٱلْحَرَامِ وَمَا كَانُوٓا۟ أَوْلِيَآءَهُۥٓ ۚ إِنْ أَوْلِيَآؤُهُۥٓ إِلَّا ٱلْمُتَّقُونَ وَلَـٰكِنَّ أَكْثَرَهُمْ لَا يَعْلَمُونَ ﴿٣٤﴾\\
\textamh{35.\  } & وَمَا كَانَ صَلَاتُهُمْ عِندَ ٱلْبَيْتِ إِلَّا مُكَآءًۭ وَتَصْدِيَةًۭ ۚ فَذُوقُوا۟ ٱلْعَذَابَ بِمَا كُنتُمْ تَكْفُرُونَ ﴿٣٥﴾\\
\textamh{36.\  } & إِنَّ ٱلَّذِينَ كَفَرُوا۟ يُنفِقُونَ أَمْوَٟلَهُمْ لِيَصُدُّوا۟ عَن سَبِيلِ ٱللَّهِ ۚ فَسَيُنفِقُونَهَا ثُمَّ تَكُونُ عَلَيْهِمْ حَسْرَةًۭ ثُمَّ يُغْلَبُونَ ۗ وَٱلَّذِينَ كَفَرُوٓا۟ إِلَىٰ جَهَنَّمَ يُحْشَرُونَ ﴿٣٦﴾\\
\textamh{37.\  } & لِيَمِيزَ ٱللَّهُ ٱلْخَبِيثَ مِنَ ٱلطَّيِّبِ وَيَجْعَلَ ٱلْخَبِيثَ بَعْضَهُۥ عَلَىٰ بَعْضٍۢ فَيَرْكُمَهُۥ جَمِيعًۭا فَيَجْعَلَهُۥ فِى جَهَنَّمَ ۚ أُو۟لَـٰٓئِكَ هُمُ ٱلْخَـٰسِرُونَ ﴿٣٧﴾\\
\textamh{38.\  } & قُل لِّلَّذِينَ كَفَرُوٓا۟ إِن يَنتَهُوا۟ يُغْفَرْ لَهُم مَّا قَدْ سَلَفَ وَإِن يَعُودُوا۟ فَقَدْ مَضَتْ سُنَّتُ ٱلْأَوَّلِينَ ﴿٣٨﴾\\
\textamh{39.\  } & وَقَـٰتِلُوهُمْ حَتَّىٰ لَا تَكُونَ فِتْنَةٌۭ وَيَكُونَ ٱلدِّينُ كُلُّهُۥ لِلَّهِ ۚ فَإِنِ ٱنتَهَوْا۟ فَإِنَّ ٱللَّهَ بِمَا يَعْمَلُونَ بَصِيرٌۭ ﴿٣٩﴾\\
\textamh{40.\  } & وَإِن تَوَلَّوْا۟ فَٱعْلَمُوٓا۟ أَنَّ ٱللَّهَ مَوْلَىٰكُمْ ۚ نِعْمَ ٱلْمَوْلَىٰ وَنِعْمَ ٱلنَّصِيرُ ﴿٤٠﴾\\
\textamh{41.\  } & ۞ وَٱعْلَمُوٓا۟ أَنَّمَا غَنِمْتُم مِّن شَىْءٍۢ فَأَنَّ لِلَّهِ خُمُسَهُۥ وَلِلرَّسُولِ وَلِذِى ٱلْقُرْبَىٰ وَٱلْيَتَـٰمَىٰ وَٱلْمَسَـٰكِينِ وَٱبْنِ ٱلسَّبِيلِ إِن كُنتُمْ ءَامَنتُم بِٱللَّهِ وَمَآ أَنزَلْنَا عَلَىٰ عَبْدِنَا يَوْمَ ٱلْفُرْقَانِ يَوْمَ ٱلْتَقَى ٱلْجَمْعَانِ ۗ وَٱللَّهُ عَلَىٰ كُلِّ شَىْءٍۢ قَدِيرٌ ﴿٤١﴾\\
\textamh{42.\  } & إِذْ أَنتُم بِٱلْعُدْوَةِ ٱلدُّنْيَا وَهُم بِٱلْعُدْوَةِ ٱلْقُصْوَىٰ وَٱلرَّكْبُ أَسْفَلَ مِنكُمْ ۚ وَلَوْ تَوَاعَدتُّمْ لَٱخْتَلَفْتُمْ فِى ٱلْمِيعَـٰدِ ۙ وَلَـٰكِن لِّيَقْضِىَ ٱللَّهُ أَمْرًۭا كَانَ مَفْعُولًۭا لِّيَهْلِكَ مَنْ هَلَكَ عَنۢ بَيِّنَةٍۢ وَيَحْيَىٰ مَنْ حَىَّ عَنۢ بَيِّنَةٍۢ ۗ وَإِنَّ ٱللَّهَ لَسَمِيعٌ عَلِيمٌ ﴿٤٢﴾\\
\textamh{43.\  } & إِذْ يُرِيكَهُمُ ٱللَّهُ فِى مَنَامِكَ قَلِيلًۭا ۖ وَلَوْ أَرَىٰكَهُمْ كَثِيرًۭا لَّفَشِلْتُمْ وَلَتَنَـٰزَعْتُمْ فِى ٱلْأَمْرِ وَلَـٰكِنَّ ٱللَّهَ سَلَّمَ ۗ إِنَّهُۥ عَلِيمٌۢ بِذَاتِ ٱلصُّدُورِ ﴿٤٣﴾\\
\textamh{44.\  } & وَإِذْ يُرِيكُمُوهُمْ إِذِ ٱلْتَقَيْتُمْ فِىٓ أَعْيُنِكُمْ قَلِيلًۭا وَيُقَلِّلُكُمْ فِىٓ أَعْيُنِهِمْ لِيَقْضِىَ ٱللَّهُ أَمْرًۭا كَانَ مَفْعُولًۭا ۗ وَإِلَى ٱللَّهِ تُرْجَعُ ٱلْأُمُورُ ﴿٤٤﴾\\
\textamh{45.\  } & يَـٰٓأَيُّهَا ٱلَّذِينَ ءَامَنُوٓا۟ إِذَا لَقِيتُمْ فِئَةًۭ فَٱثْبُتُوا۟ وَٱذْكُرُوا۟ ٱللَّهَ كَثِيرًۭا لَّعَلَّكُمْ تُفْلِحُونَ ﴿٤٥﴾\\
\textamh{46.\  } & وَأَطِيعُوا۟ ٱللَّهَ وَرَسُولَهُۥ وَلَا تَنَـٰزَعُوا۟ فَتَفْشَلُوا۟ وَتَذْهَبَ رِيحُكُمْ ۖ وَٱصْبِرُوٓا۟ ۚ إِنَّ ٱللَّهَ مَعَ ٱلصَّـٰبِرِينَ ﴿٤٦﴾\\
\textamh{47.\  } & وَلَا تَكُونُوا۟ كَٱلَّذِينَ خَرَجُوا۟ مِن دِيَـٰرِهِم بَطَرًۭا وَرِئَآءَ ٱلنَّاسِ وَيَصُدُّونَ عَن سَبِيلِ ٱللَّهِ ۚ وَٱللَّهُ بِمَا يَعْمَلُونَ مُحِيطٌۭ ﴿٤٧﴾\\
\textamh{48.\  } & وَإِذْ زَيَّنَ لَهُمُ ٱلشَّيْطَٰنُ أَعْمَـٰلَهُمْ وَقَالَ لَا غَالِبَ لَكُمُ ٱلْيَوْمَ مِنَ ٱلنَّاسِ وَإِنِّى جَارٌۭ لَّكُمْ ۖ فَلَمَّا تَرَآءَتِ ٱلْفِئَتَانِ نَكَصَ عَلَىٰ عَقِبَيْهِ وَقَالَ إِنِّى بَرِىٓءٌۭ مِّنكُمْ إِنِّىٓ أَرَىٰ مَا لَا تَرَوْنَ إِنِّىٓ أَخَافُ ٱللَّهَ ۚ وَٱللَّهُ شَدِيدُ ٱلْعِقَابِ ﴿٤٨﴾\\
\textamh{49.\  } & إِذْ يَقُولُ ٱلْمُنَـٰفِقُونَ وَٱلَّذِينَ فِى قُلُوبِهِم مَّرَضٌ غَرَّ هَـٰٓؤُلَآءِ دِينُهُمْ ۗ وَمَن يَتَوَكَّلْ عَلَى ٱللَّهِ فَإِنَّ ٱللَّهَ عَزِيزٌ حَكِيمٌۭ ﴿٤٩﴾\\
\textamh{50.\  } & وَلَوْ تَرَىٰٓ إِذْ يَتَوَفَّى ٱلَّذِينَ كَفَرُوا۟ ۙ ٱلْمَلَـٰٓئِكَةُ يَضْرِبُونَ وُجُوهَهُمْ وَأَدْبَٰرَهُمْ وَذُوقُوا۟ عَذَابَ ٱلْحَرِيقِ ﴿٥٠﴾\\
\textamh{51.\  } & ذَٟلِكَ بِمَا قَدَّمَتْ أَيْدِيكُمْ وَأَنَّ ٱللَّهَ لَيْسَ بِظَلَّٰمٍۢ لِّلْعَبِيدِ ﴿٥١﴾\\
\textamh{52.\  } & كَدَأْبِ ءَالِ فِرْعَوْنَ ۙ وَٱلَّذِينَ مِن قَبْلِهِمْ ۚ كَفَرُوا۟ بِـَٔايَـٰتِ ٱللَّهِ فَأَخَذَهُمُ ٱللَّهُ بِذُنُوبِهِمْ ۗ إِنَّ ٱللَّهَ قَوِىٌّۭ شَدِيدُ ٱلْعِقَابِ ﴿٥٢﴾\\
\textamh{53.\  } & ذَٟلِكَ بِأَنَّ ٱللَّهَ لَمْ يَكُ مُغَيِّرًۭا نِّعْمَةً أَنْعَمَهَا عَلَىٰ قَوْمٍ حَتَّىٰ يُغَيِّرُوا۟ مَا بِأَنفُسِهِمْ ۙ وَأَنَّ ٱللَّهَ سَمِيعٌ عَلِيمٌۭ ﴿٥٣﴾\\
\textamh{54.\  } & كَدَأْبِ ءَالِ فِرْعَوْنَ ۙ وَٱلَّذِينَ مِن قَبْلِهِمْ ۚ كَذَّبُوا۟ بِـَٔايَـٰتِ رَبِّهِمْ فَأَهْلَكْنَـٰهُم بِذُنُوبِهِمْ وَأَغْرَقْنَآ ءَالَ فِرْعَوْنَ ۚ وَكُلٌّۭ كَانُوا۟ ظَـٰلِمِينَ ﴿٥٤﴾\\
\textamh{55.\  } & إِنَّ شَرَّ ٱلدَّوَآبِّ عِندَ ٱللَّهِ ٱلَّذِينَ كَفَرُوا۟ فَهُمْ لَا يُؤْمِنُونَ ﴿٥٥﴾\\
\textamh{56.\  } & ٱلَّذِينَ عَـٰهَدتَّ مِنْهُمْ ثُمَّ يَنقُضُونَ عَهْدَهُمْ فِى كُلِّ مَرَّةٍۢ وَهُمْ لَا يَتَّقُونَ ﴿٥٦﴾\\
\textamh{57.\  } & فَإِمَّا تَثْقَفَنَّهُمْ فِى ٱلْحَرْبِ فَشَرِّدْ بِهِم مَّنْ خَلْفَهُمْ لَعَلَّهُمْ يَذَّكَّرُونَ ﴿٥٧﴾\\
\textamh{58.\  } & وَإِمَّا تَخَافَنَّ مِن قَوْمٍ خِيَانَةًۭ فَٱنۢبِذْ إِلَيْهِمْ عَلَىٰ سَوَآءٍ ۚ إِنَّ ٱللَّهَ لَا يُحِبُّ ٱلْخَآئِنِينَ ﴿٥٨﴾\\
\textamh{59.\  } & وَلَا يَحْسَبَنَّ ٱلَّذِينَ كَفَرُوا۟ سَبَقُوٓا۟ ۚ إِنَّهُمْ لَا يُعْجِزُونَ ﴿٥٩﴾\\
\textamh{60.\  } & وَأَعِدُّوا۟ لَهُم مَّا ٱسْتَطَعْتُم مِّن قُوَّةٍۢ وَمِن رِّبَاطِ ٱلْخَيْلِ تُرْهِبُونَ بِهِۦ عَدُوَّ ٱللَّهِ وَعَدُوَّكُمْ وَءَاخَرِينَ مِن دُونِهِمْ لَا تَعْلَمُونَهُمُ ٱللَّهُ يَعْلَمُهُمْ ۚ وَمَا تُنفِقُوا۟ مِن شَىْءٍۢ فِى سَبِيلِ ٱللَّهِ يُوَفَّ إِلَيْكُمْ وَأَنتُمْ لَا تُظْلَمُونَ ﴿٦٠﴾\\
\textamh{61.\  } & ۞ وَإِن جَنَحُوا۟ لِلسَّلْمِ فَٱجْنَحْ لَهَا وَتَوَكَّلْ عَلَى ٱللَّهِ ۚ إِنَّهُۥ هُوَ ٱلسَّمِيعُ ٱلْعَلِيمُ ﴿٦١﴾\\
\textamh{62.\  } & وَإِن يُرِيدُوٓا۟ أَن يَخْدَعُوكَ فَإِنَّ حَسْبَكَ ٱللَّهُ ۚ هُوَ ٱلَّذِىٓ أَيَّدَكَ بِنَصْرِهِۦ وَبِٱلْمُؤْمِنِينَ ﴿٦٢﴾\\
\textamh{63.\  } & وَأَلَّفَ بَيْنَ قُلُوبِهِمْ ۚ لَوْ أَنفَقْتَ مَا فِى ٱلْأَرْضِ جَمِيعًۭا مَّآ أَلَّفْتَ بَيْنَ قُلُوبِهِمْ وَلَـٰكِنَّ ٱللَّهَ أَلَّفَ بَيْنَهُمْ ۚ إِنَّهُۥ عَزِيزٌ حَكِيمٌۭ ﴿٦٣﴾\\
\textamh{64.\  } & يَـٰٓأَيُّهَا ٱلنَّبِىُّ حَسْبُكَ ٱللَّهُ وَمَنِ ٱتَّبَعَكَ مِنَ ٱلْمُؤْمِنِينَ ﴿٦٤﴾\\
\textamh{65.\  } & يَـٰٓأَيُّهَا ٱلنَّبِىُّ حَرِّضِ ٱلْمُؤْمِنِينَ عَلَى ٱلْقِتَالِ ۚ إِن يَكُن مِّنكُمْ عِشْرُونَ صَـٰبِرُونَ يَغْلِبُوا۟ مِا۟ئَتَيْنِ ۚ وَإِن يَكُن مِّنكُم مِّا۟ئَةٌۭ يَغْلِبُوٓا۟ أَلْفًۭا مِّنَ ٱلَّذِينَ كَفَرُوا۟ بِأَنَّهُمْ قَوْمٌۭ لَّا يَفْقَهُونَ ﴿٦٥﴾\\
\textamh{66.\  } & ٱلْـَٰٔنَ خَفَّفَ ٱللَّهُ عَنكُمْ وَعَلِمَ أَنَّ فِيكُمْ ضَعْفًۭا ۚ فَإِن يَكُن مِّنكُم مِّا۟ئَةٌۭ صَابِرَةٌۭ يَغْلِبُوا۟ مِا۟ئَتَيْنِ ۚ وَإِن يَكُن مِّنكُمْ أَلْفٌۭ يَغْلِبُوٓا۟ أَلْفَيْنِ بِإِذْنِ ٱللَّهِ ۗ وَٱللَّهُ مَعَ ٱلصَّـٰبِرِينَ ﴿٦٦﴾\\
\textamh{67.\  } & مَا كَانَ لِنَبِىٍّ أَن يَكُونَ لَهُۥٓ أَسْرَىٰ حَتَّىٰ يُثْخِنَ فِى ٱلْأَرْضِ ۚ تُرِيدُونَ عَرَضَ ٱلدُّنْيَا وَٱللَّهُ يُرِيدُ ٱلْءَاخِرَةَ ۗ وَٱللَّهُ عَزِيزٌ حَكِيمٌۭ ﴿٦٧﴾\\
\textamh{68.\  } & لَّوْلَا كِتَـٰبٌۭ مِّنَ ٱللَّهِ سَبَقَ لَمَسَّكُمْ فِيمَآ أَخَذْتُمْ عَذَابٌ عَظِيمٌۭ ﴿٦٨﴾\\
\textamh{69.\  } & فَكُلُوا۟ مِمَّا غَنِمْتُمْ حَلَـٰلًۭا طَيِّبًۭا ۚ وَٱتَّقُوا۟ ٱللَّهَ ۚ إِنَّ ٱللَّهَ غَفُورٌۭ رَّحِيمٌۭ ﴿٦٩﴾\\
\textamh{70.\  } & يَـٰٓأَيُّهَا ٱلنَّبِىُّ قُل لِّمَن فِىٓ أَيْدِيكُم مِّنَ ٱلْأَسْرَىٰٓ إِن يَعْلَمِ ٱللَّهُ فِى قُلُوبِكُمْ خَيْرًۭا يُؤْتِكُمْ خَيْرًۭا مِّمَّآ أُخِذَ مِنكُمْ وَيَغْفِرْ لَكُمْ ۗ وَٱللَّهُ غَفُورٌۭ رَّحِيمٌۭ ﴿٧٠﴾\\
\textamh{71.\  } & وَإِن يُرِيدُوا۟ خِيَانَتَكَ فَقَدْ خَانُوا۟ ٱللَّهَ مِن قَبْلُ فَأَمْكَنَ مِنْهُمْ ۗ وَٱللَّهُ عَلِيمٌ حَكِيمٌ ﴿٧١﴾\\
\textamh{72.\  } & إِنَّ ٱلَّذِينَ ءَامَنُوا۟ وَهَاجَرُوا۟ وَجَٰهَدُوا۟ بِأَمْوَٟلِهِمْ وَأَنفُسِهِمْ فِى سَبِيلِ ٱللَّهِ وَٱلَّذِينَ ءَاوَوا۟ وَّنَصَرُوٓا۟ أُو۟لَـٰٓئِكَ بَعْضُهُمْ أَوْلِيَآءُ بَعْضٍۢ ۚ وَٱلَّذِينَ ءَامَنُوا۟ وَلَمْ يُهَاجِرُوا۟ مَا لَكُم مِّن وَلَـٰيَتِهِم مِّن شَىْءٍ حَتَّىٰ يُهَاجِرُوا۟ ۚ وَإِنِ ٱسْتَنصَرُوكُمْ فِى ٱلدِّينِ فَعَلَيْكُمُ ٱلنَّصْرُ إِلَّا عَلَىٰ قَوْمٍۭ بَيْنَكُمْ وَبَيْنَهُم مِّيثَـٰقٌۭ ۗ وَٱللَّهُ بِمَا تَعْمَلُونَ بَصِيرٌۭ ﴿٧٢﴾\\
\textamh{73.\  } & وَٱلَّذِينَ كَفَرُوا۟ بَعْضُهُمْ أَوْلِيَآءُ بَعْضٍ ۚ إِلَّا تَفْعَلُوهُ تَكُن فِتْنَةٌۭ فِى ٱلْأَرْضِ وَفَسَادٌۭ كَبِيرٌۭ ﴿٧٣﴾\\
\textamh{74.\  } & وَٱلَّذِينَ ءَامَنُوا۟ وَهَاجَرُوا۟ وَجَٰهَدُوا۟ فِى سَبِيلِ ٱللَّهِ وَٱلَّذِينَ ءَاوَوا۟ وَّنَصَرُوٓا۟ أُو۟لَـٰٓئِكَ هُمُ ٱلْمُؤْمِنُونَ حَقًّۭا ۚ لَّهُم مَّغْفِرَةٌۭ وَرِزْقٌۭ كَرِيمٌۭ ﴿٧٤﴾\\
\textamh{75.\  } & وَٱلَّذِينَ ءَامَنُوا۟ مِنۢ بَعْدُ وَهَاجَرُوا۟ وَجَٰهَدُوا۟ مَعَكُمْ فَأُو۟لَـٰٓئِكَ مِنكُمْ ۚ وَأُو۟لُوا۟ ٱلْأَرْحَامِ بَعْضُهُمْ أَوْلَىٰ بِبَعْضٍۢ فِى كِتَـٰبِ ٱللَّهِ ۗ إِنَّ ٱللَّهَ بِكُلِّ شَىْءٍ عَلِيمٌۢ ﴿٧٥﴾\\
\end{longtable}
\clearpage
%% License: BSD style (Berkley) (i.e. Put the Copyright owner's name always)
%% Writer and Copyright (to): Bewketu(Bilal) Tadilo (2016-17)
\centering\section{\LR{\textamharic{ሱራቱ አተውባ -}  \RL{سوره  التوبة}}}
\begin{longtable}{%
  @{}
    p{.5\textwidth}
  @{~~~~~~~~~~~~~}
    p{.5\textwidth}
    @{}
}
\textamh{1.\  } & بَرَآءَةٌۭ مِّنَ ٱللَّهِ وَرَسُولِهِۦٓ إِلَى ٱلَّذِينَ عَـٰهَدتُّم مِّنَ ٱلْمُشْرِكِينَ ﴿١﴾\\
\textamh{2.\  } & فَسِيحُوا۟ فِى ٱلْأَرْضِ أَرْبَعَةَ أَشْهُرٍۢ وَٱعْلَمُوٓا۟ أَنَّكُمْ غَيْرُ مُعْجِزِى ٱللَّهِ ۙ وَأَنَّ ٱللَّهَ مُخْزِى ٱلْكَـٰفِرِينَ ﴿٢﴾\\
\textamh{3.\  } & وَأَذَٟنٌۭ مِّنَ ٱللَّهِ وَرَسُولِهِۦٓ إِلَى ٱلنَّاسِ يَوْمَ ٱلْحَجِّ ٱلْأَكْبَرِ أَنَّ ٱللَّهَ بَرِىٓءٌۭ مِّنَ ٱلْمُشْرِكِينَ ۙ وَرَسُولُهُۥ ۚ فَإِن تُبْتُمْ فَهُوَ خَيْرٌۭ لَّكُمْ ۖ وَإِن تَوَلَّيْتُمْ فَٱعْلَمُوٓا۟ أَنَّكُمْ غَيْرُ مُعْجِزِى ٱللَّهِ ۗ وَبَشِّرِ ٱلَّذِينَ كَفَرُوا۟ بِعَذَابٍ أَلِيمٍ ﴿٣﴾\\
\textamh{4.\  } & إِلَّا ٱلَّذِينَ عَـٰهَدتُّم مِّنَ ٱلْمُشْرِكِينَ ثُمَّ لَمْ يَنقُصُوكُمْ شَيْـًۭٔا وَلَمْ يُظَـٰهِرُوا۟ عَلَيْكُمْ أَحَدًۭا فَأَتِمُّوٓا۟ إِلَيْهِمْ عَهْدَهُمْ إِلَىٰ مُدَّتِهِمْ ۚ إِنَّ ٱللَّهَ يُحِبُّ ٱلْمُتَّقِينَ ﴿٤﴾\\
\textamh{5.\  } & فَإِذَا ٱنسَلَخَ ٱلْأَشْهُرُ ٱلْحُرُمُ فَٱقْتُلُوا۟ ٱلْمُشْرِكِينَ حَيْثُ وَجَدتُّمُوهُمْ وَخُذُوهُمْ وَٱحْصُرُوهُمْ وَٱقْعُدُوا۟ لَهُمْ كُلَّ مَرْصَدٍۢ ۚ فَإِن تَابُوا۟ وَأَقَامُوا۟ ٱلصَّلَوٰةَ وَءَاتَوُا۟ ٱلزَّكَوٰةَ فَخَلُّوا۟ سَبِيلَهُمْ ۚ إِنَّ ٱللَّهَ غَفُورٌۭ رَّحِيمٌۭ ﴿٥﴾\\
\textamh{6.\  } & وَإِنْ أَحَدٌۭ مِّنَ ٱلْمُشْرِكِينَ ٱسْتَجَارَكَ فَأَجِرْهُ حَتَّىٰ يَسْمَعَ كَلَـٰمَ ٱللَّهِ ثُمَّ أَبْلِغْهُ مَأْمَنَهُۥ ۚ ذَٟلِكَ بِأَنَّهُمْ قَوْمٌۭ لَّا يَعْلَمُونَ ﴿٦﴾\\
\textamh{7.\  } & كَيْفَ يَكُونُ لِلْمُشْرِكِينَ عَهْدٌ عِندَ ٱللَّهِ وَعِندَ رَسُولِهِۦٓ إِلَّا ٱلَّذِينَ عَـٰهَدتُّمْ عِندَ ٱلْمَسْجِدِ ٱلْحَرَامِ ۖ فَمَا ٱسْتَقَـٰمُوا۟ لَكُمْ فَٱسْتَقِيمُوا۟ لَهُمْ ۚ إِنَّ ٱللَّهَ يُحِبُّ ٱلْمُتَّقِينَ ﴿٧﴾\\
\textamh{8.\  } & كَيْفَ وَإِن يَظْهَرُوا۟ عَلَيْكُمْ لَا يَرْقُبُوا۟ فِيكُمْ إِلًّۭا وَلَا ذِمَّةًۭ ۚ يُرْضُونَكُم بِأَفْوَٟهِهِمْ وَتَأْبَىٰ قُلُوبُهُمْ وَأَكْثَرُهُمْ فَـٰسِقُونَ ﴿٨﴾\\
\textamh{9.\  } & ٱشْتَرَوْا۟ بِـَٔايَـٰتِ ٱللَّهِ ثَمَنًۭا قَلِيلًۭا فَصَدُّوا۟ عَن سَبِيلِهِۦٓ ۚ إِنَّهُمْ سَآءَ مَا كَانُوا۟ يَعْمَلُونَ ﴿٩﴾\\
\textamh{10.\  } & لَا يَرْقُبُونَ فِى مُؤْمِنٍ إِلًّۭا وَلَا ذِمَّةًۭ ۚ وَأُو۟لَـٰٓئِكَ هُمُ ٱلْمُعْتَدُونَ ﴿١٠﴾\\
\textamh{11.\  } & فَإِن تَابُوا۟ وَأَقَامُوا۟ ٱلصَّلَوٰةَ وَءَاتَوُا۟ ٱلزَّكَوٰةَ فَإِخْوَٟنُكُمْ فِى ٱلدِّينِ ۗ وَنُفَصِّلُ ٱلْءَايَـٰتِ لِقَوْمٍۢ يَعْلَمُونَ ﴿١١﴾\\
\textamh{12.\  } & وَإِن نَّكَثُوٓا۟ أَيْمَـٰنَهُم مِّنۢ بَعْدِ عَهْدِهِمْ وَطَعَنُوا۟ فِى دِينِكُمْ فَقَـٰتِلُوٓا۟ أَئِمَّةَ ٱلْكُفْرِ ۙ إِنَّهُمْ لَآ أَيْمَـٰنَ لَهُمْ لَعَلَّهُمْ يَنتَهُونَ ﴿١٢﴾\\
\textamh{13.\  } & أَلَا تُقَـٰتِلُونَ قَوْمًۭا نَّكَثُوٓا۟ أَيْمَـٰنَهُمْ وَهَمُّوا۟ بِإِخْرَاجِ ٱلرَّسُولِ وَهُم بَدَءُوكُمْ أَوَّلَ مَرَّةٍ ۚ أَتَخْشَوْنَهُمْ ۚ فَٱللَّهُ أَحَقُّ أَن تَخْشَوْهُ إِن كُنتُم مُّؤْمِنِينَ ﴿١٣﴾\\
\textamh{14.\  } & قَـٰتِلُوهُمْ يُعَذِّبْهُمُ ٱللَّهُ بِأَيْدِيكُمْ وَيُخْزِهِمْ وَيَنصُرْكُمْ عَلَيْهِمْ وَيَشْفِ صُدُورَ قَوْمٍۢ مُّؤْمِنِينَ ﴿١٤﴾\\
\textamh{15.\  } & وَيُذْهِبْ غَيْظَ قُلُوبِهِمْ ۗ وَيَتُوبُ ٱللَّهُ عَلَىٰ مَن يَشَآءُ ۗ وَٱللَّهُ عَلِيمٌ حَكِيمٌ ﴿١٥﴾\\
\textamh{16.\  } & أَمْ حَسِبْتُمْ أَن تُتْرَكُوا۟ وَلَمَّا يَعْلَمِ ٱللَّهُ ٱلَّذِينَ جَٰهَدُوا۟ مِنكُمْ وَلَمْ يَتَّخِذُوا۟ مِن دُونِ ٱللَّهِ وَلَا رَسُولِهِۦ وَلَا ٱلْمُؤْمِنِينَ وَلِيجَةًۭ ۚ وَٱللَّهُ خَبِيرٌۢ بِمَا تَعْمَلُونَ ﴿١٦﴾\\
\textamh{17.\  } & مَا كَانَ لِلْمُشْرِكِينَ أَن يَعْمُرُوا۟ مَسَـٰجِدَ ٱللَّهِ شَـٰهِدِينَ عَلَىٰٓ أَنفُسِهِم بِٱلْكُفْرِ ۚ أُو۟لَـٰٓئِكَ حَبِطَتْ أَعْمَـٰلُهُمْ وَفِى ٱلنَّارِ هُمْ خَـٰلِدُونَ ﴿١٧﴾\\
\textamh{18.\  } & إِنَّمَا يَعْمُرُ مَسَـٰجِدَ ٱللَّهِ مَنْ ءَامَنَ بِٱللَّهِ وَٱلْيَوْمِ ٱلْءَاخِرِ وَأَقَامَ ٱلصَّلَوٰةَ وَءَاتَى ٱلزَّكَوٰةَ وَلَمْ يَخْشَ إِلَّا ٱللَّهَ ۖ فَعَسَىٰٓ أُو۟لَـٰٓئِكَ أَن يَكُونُوا۟ مِنَ ٱلْمُهْتَدِينَ ﴿١٨﴾\\
\textamh{19.\  } & ۞ أَجَعَلْتُمْ سِقَايَةَ ٱلْحَآجِّ وَعِمَارَةَ ٱلْمَسْجِدِ ٱلْحَرَامِ كَمَنْ ءَامَنَ بِٱللَّهِ وَٱلْيَوْمِ ٱلْءَاخِرِ وَجَٰهَدَ فِى سَبِيلِ ٱللَّهِ ۚ لَا يَسْتَوُۥنَ عِندَ ٱللَّهِ ۗ وَٱللَّهُ لَا يَهْدِى ٱلْقَوْمَ ٱلظَّـٰلِمِينَ ﴿١٩﴾\\
\textamh{20.\  } & ٱلَّذِينَ ءَامَنُوا۟ وَهَاجَرُوا۟ وَجَٰهَدُوا۟ فِى سَبِيلِ ٱللَّهِ بِأَمْوَٟلِهِمْ وَأَنفُسِهِمْ أَعْظَمُ دَرَجَةً عِندَ ٱللَّهِ ۚ وَأُو۟لَـٰٓئِكَ هُمُ ٱلْفَآئِزُونَ ﴿٢٠﴾\\
\textamh{21.\  } & يُبَشِّرُهُمْ رَبُّهُم بِرَحْمَةٍۢ مِّنْهُ وَرِضْوَٟنٍۢ وَجَنَّـٰتٍۢ لَّهُمْ فِيهَا نَعِيمٌۭ مُّقِيمٌ ﴿٢١﴾\\
\textamh{22.\  } & خَـٰلِدِينَ فِيهَآ أَبَدًا ۚ إِنَّ ٱللَّهَ عِندَهُۥٓ أَجْرٌ عَظِيمٌۭ ﴿٢٢﴾\\
\textamh{23.\  } & يَـٰٓأَيُّهَا ٱلَّذِينَ ءَامَنُوا۟ لَا تَتَّخِذُوٓا۟ ءَابَآءَكُمْ وَإِخْوَٟنَكُمْ أَوْلِيَآءَ إِنِ ٱسْتَحَبُّوا۟ ٱلْكُفْرَ عَلَى ٱلْإِيمَـٰنِ ۚ وَمَن يَتَوَلَّهُم مِّنكُمْ فَأُو۟لَـٰٓئِكَ هُمُ ٱلظَّـٰلِمُونَ ﴿٢٣﴾\\
\textamh{24.\  } & قُلْ إِن كَانَ ءَابَآؤُكُمْ وَأَبْنَآؤُكُمْ وَإِخْوَٟنُكُمْ وَأَزْوَٟجُكُمْ وَعَشِيرَتُكُمْ وَأَمْوَٟلٌ ٱقْتَرَفْتُمُوهَا وَتِجَٰرَةٌۭ تَخْشَوْنَ كَسَادَهَا وَمَسَـٰكِنُ تَرْضَوْنَهَآ أَحَبَّ إِلَيْكُم مِّنَ ٱللَّهِ وَرَسُولِهِۦ وَجِهَادٍۢ فِى سَبِيلِهِۦ فَتَرَبَّصُوا۟ حَتَّىٰ يَأْتِىَ ٱللَّهُ بِأَمْرِهِۦ ۗ وَٱللَّهُ لَا يَهْدِى ٱلْقَوْمَ ٱلْفَـٰسِقِينَ ﴿٢٤﴾\\
\textamh{25.\  } & لَقَدْ نَصَرَكُمُ ٱللَّهُ فِى مَوَاطِنَ كَثِيرَةٍۢ ۙ وَيَوْمَ حُنَيْنٍ ۙ إِذْ أَعْجَبَتْكُمْ كَثْرَتُكُمْ فَلَمْ تُغْنِ عَنكُمْ شَيْـًۭٔا وَضَاقَتْ عَلَيْكُمُ ٱلْأَرْضُ بِمَا رَحُبَتْ ثُمَّ وَلَّيْتُم مُّدْبِرِينَ ﴿٢٥﴾\\
\textamh{26.\  } & ثُمَّ أَنزَلَ ٱللَّهُ سَكِينَتَهُۥ عَلَىٰ رَسُولِهِۦ وَعَلَى ٱلْمُؤْمِنِينَ وَأَنزَلَ جُنُودًۭا لَّمْ تَرَوْهَا وَعَذَّبَ ٱلَّذِينَ كَفَرُوا۟ ۚ وَذَٟلِكَ جَزَآءُ ٱلْكَـٰفِرِينَ ﴿٢٦﴾\\
\textamh{27.\  } & ثُمَّ يَتُوبُ ٱللَّهُ مِنۢ بَعْدِ ذَٟلِكَ عَلَىٰ مَن يَشَآءُ ۗ وَٱللَّهُ غَفُورٌۭ رَّحِيمٌۭ ﴿٢٧﴾\\
\textamh{28.\  } & يَـٰٓأَيُّهَا ٱلَّذِينَ ءَامَنُوٓا۟ إِنَّمَا ٱلْمُشْرِكُونَ نَجَسٌۭ فَلَا يَقْرَبُوا۟ ٱلْمَسْجِدَ ٱلْحَرَامَ بَعْدَ عَامِهِمْ هَـٰذَا ۚ وَإِنْ خِفْتُمْ عَيْلَةًۭ فَسَوْفَ يُغْنِيكُمُ ٱللَّهُ مِن فَضْلِهِۦٓ إِن شَآءَ ۚ إِنَّ ٱللَّهَ عَلِيمٌ حَكِيمٌۭ ﴿٢٨﴾\\
\textamh{29.\  } & قَـٰتِلُوا۟ ٱلَّذِينَ لَا يُؤْمِنُونَ بِٱللَّهِ وَلَا بِٱلْيَوْمِ ٱلْءَاخِرِ وَلَا يُحَرِّمُونَ مَا حَرَّمَ ٱللَّهُ وَرَسُولُهُۥ وَلَا يَدِينُونَ دِينَ ٱلْحَقِّ مِنَ ٱلَّذِينَ أُوتُوا۟ ٱلْكِتَـٰبَ حَتَّىٰ يُعْطُوا۟ ٱلْجِزْيَةَ عَن يَدٍۢ وَهُمْ صَـٰغِرُونَ ﴿٢٩﴾\\
\textamh{30.\  } & وَقَالَتِ ٱلْيَهُودُ عُزَيْرٌ ٱبْنُ ٱللَّهِ وَقَالَتِ ٱلنَّصَـٰرَى ٱلْمَسِيحُ ٱبْنُ ٱللَّهِ ۖ ذَٟلِكَ قَوْلُهُم بِأَفْوَٟهِهِمْ ۖ يُضَٰهِـُٔونَ قَوْلَ ٱلَّذِينَ كَفَرُوا۟ مِن قَبْلُ ۚ قَـٰتَلَهُمُ ٱللَّهُ ۚ أَنَّىٰ يُؤْفَكُونَ ﴿٣٠﴾\\
\textamh{31.\  } & ٱتَّخَذُوٓا۟ أَحْبَارَهُمْ وَرُهْبَٰنَهُمْ أَرْبَابًۭا مِّن دُونِ ٱللَّهِ وَٱلْمَسِيحَ ٱبْنَ مَرْيَمَ وَمَآ أُمِرُوٓا۟ إِلَّا لِيَعْبُدُوٓا۟ إِلَـٰهًۭا وَٟحِدًۭا ۖ لَّآ إِلَـٰهَ إِلَّا هُوَ ۚ سُبْحَـٰنَهُۥ عَمَّا يُشْرِكُونَ ﴿٣١﴾\\
\textamh{32.\  } & يُرِيدُونَ أَن يُطْفِـُٔوا۟ نُورَ ٱللَّهِ بِأَفْوَٟهِهِمْ وَيَأْبَى ٱللَّهُ إِلَّآ أَن يُتِمَّ نُورَهُۥ وَلَوْ كَرِهَ ٱلْكَـٰفِرُونَ ﴿٣٢﴾\\
\textamh{33.\  } & هُوَ ٱلَّذِىٓ أَرْسَلَ رَسُولَهُۥ بِٱلْهُدَىٰ وَدِينِ ٱلْحَقِّ لِيُظْهِرَهُۥ عَلَى ٱلدِّينِ كُلِّهِۦ وَلَوْ كَرِهَ ٱلْمُشْرِكُونَ ﴿٣٣﴾\\
\textamh{34.\  } & ۞ يَـٰٓأَيُّهَا ٱلَّذِينَ ءَامَنُوٓا۟ إِنَّ كَثِيرًۭا مِّنَ ٱلْأَحْبَارِ وَٱلرُّهْبَانِ لَيَأْكُلُونَ أَمْوَٟلَ ٱلنَّاسِ بِٱلْبَٰطِلِ وَيَصُدُّونَ عَن سَبِيلِ ٱللَّهِ ۗ وَٱلَّذِينَ يَكْنِزُونَ ٱلذَّهَبَ وَٱلْفِضَّةَ وَلَا يُنفِقُونَهَا فِى سَبِيلِ ٱللَّهِ فَبَشِّرْهُم بِعَذَابٍ أَلِيمٍۢ ﴿٣٤﴾\\
\textamh{35.\  } & يَوْمَ يُحْمَىٰ عَلَيْهَا فِى نَارِ جَهَنَّمَ فَتُكْوَىٰ بِهَا جِبَاهُهُمْ وَجُنُوبُهُمْ وَظُهُورُهُمْ ۖ هَـٰذَا مَا كَنَزْتُمْ لِأَنفُسِكُمْ فَذُوقُوا۟ مَا كُنتُمْ تَكْنِزُونَ ﴿٣٥﴾\\
\textamh{36.\  } & إِنَّ عِدَّةَ ٱلشُّهُورِ عِندَ ٱللَّهِ ٱثْنَا عَشَرَ شَهْرًۭا فِى كِتَـٰبِ ٱللَّهِ يَوْمَ خَلَقَ ٱلسَّمَـٰوَٟتِ وَٱلْأَرْضَ مِنْهَآ أَرْبَعَةٌ حُرُمٌۭ ۚ ذَٟلِكَ ٱلدِّينُ ٱلْقَيِّمُ ۚ فَلَا تَظْلِمُوا۟ فِيهِنَّ أَنفُسَكُمْ ۚ وَقَـٰتِلُوا۟ ٱلْمُشْرِكِينَ كَآفَّةًۭ كَمَا يُقَـٰتِلُونَكُمْ كَآفَّةًۭ ۚ وَٱعْلَمُوٓا۟ أَنَّ ٱللَّهَ مَعَ ٱلْمُتَّقِينَ ﴿٣٦﴾\\
\textamh{37.\  } & إِنَّمَا ٱلنَّسِىٓءُ زِيَادَةٌۭ فِى ٱلْكُفْرِ ۖ يُضَلُّ بِهِ ٱلَّذِينَ كَفَرُوا۟ يُحِلُّونَهُۥ عَامًۭا وَيُحَرِّمُونَهُۥ عَامًۭا لِّيُوَاطِـُٔوا۟ عِدَّةَ مَا حَرَّمَ ٱللَّهُ فَيُحِلُّوا۟ مَا حَرَّمَ ٱللَّهُ ۚ زُيِّنَ لَهُمْ سُوٓءُ أَعْمَـٰلِهِمْ ۗ وَٱللَّهُ لَا يَهْدِى ٱلْقَوْمَ ٱلْكَـٰفِرِينَ ﴿٣٧﴾\\
\textamh{38.\  } & يَـٰٓأَيُّهَا ٱلَّذِينَ ءَامَنُوا۟ مَا لَكُمْ إِذَا قِيلَ لَكُمُ ٱنفِرُوا۟ فِى سَبِيلِ ٱللَّهِ ٱثَّاقَلْتُمْ إِلَى ٱلْأَرْضِ ۚ أَرَضِيتُم بِٱلْحَيَوٰةِ ٱلدُّنْيَا مِنَ ٱلْءَاخِرَةِ ۚ فَمَا مَتَـٰعُ ٱلْحَيَوٰةِ ٱلدُّنْيَا فِى ٱلْءَاخِرَةِ إِلَّا قَلِيلٌ ﴿٣٨﴾\\
\textamh{39.\  } & إِلَّا تَنفِرُوا۟ يُعَذِّبْكُمْ عَذَابًا أَلِيمًۭا وَيَسْتَبْدِلْ قَوْمًا غَيْرَكُمْ وَلَا تَضُرُّوهُ شَيْـًۭٔا ۗ وَٱللَّهُ عَلَىٰ كُلِّ شَىْءٍۢ قَدِيرٌ ﴿٣٩﴾\\
\textamh{40.\  } & إِلَّا تَنصُرُوهُ فَقَدْ نَصَرَهُ ٱللَّهُ إِذْ أَخْرَجَهُ ٱلَّذِينَ كَفَرُوا۟ ثَانِىَ ٱثْنَيْنِ إِذْ هُمَا فِى ٱلْغَارِ إِذْ يَقُولُ لِصَـٰحِبِهِۦ لَا تَحْزَنْ إِنَّ ٱللَّهَ مَعَنَا ۖ فَأَنزَلَ ٱللَّهُ سَكِينَتَهُۥ عَلَيْهِ وَأَيَّدَهُۥ بِجُنُودٍۢ لَّمْ تَرَوْهَا وَجَعَلَ كَلِمَةَ ٱلَّذِينَ كَفَرُوا۟ ٱلسُّفْلَىٰ ۗ وَكَلِمَةُ ٱللَّهِ هِىَ ٱلْعُلْيَا ۗ وَٱللَّهُ عَزِيزٌ حَكِيمٌ ﴿٤٠﴾\\
\textamh{41.\  } & ٱنفِرُوا۟ خِفَافًۭا وَثِقَالًۭا وَجَٰهِدُوا۟ بِأَمْوَٟلِكُمْ وَأَنفُسِكُمْ فِى سَبِيلِ ٱللَّهِ ۚ ذَٟلِكُمْ خَيْرٌۭ لَّكُمْ إِن كُنتُمْ تَعْلَمُونَ ﴿٤١﴾\\
\textamh{42.\  } & لَوْ كَانَ عَرَضًۭا قَرِيبًۭا وَسَفَرًۭا قَاصِدًۭا لَّٱتَّبَعُوكَ وَلَـٰكِنۢ بَعُدَتْ عَلَيْهِمُ ٱلشُّقَّةُ ۚ وَسَيَحْلِفُونَ بِٱللَّهِ لَوِ ٱسْتَطَعْنَا لَخَرَجْنَا مَعَكُمْ يُهْلِكُونَ أَنفُسَهُمْ وَٱللَّهُ يَعْلَمُ إِنَّهُمْ لَكَـٰذِبُونَ ﴿٤٢﴾\\
\textamh{43.\  } & عَفَا ٱللَّهُ عَنكَ لِمَ أَذِنتَ لَهُمْ حَتَّىٰ يَتَبَيَّنَ لَكَ ٱلَّذِينَ صَدَقُوا۟ وَتَعْلَمَ ٱلْكَـٰذِبِينَ ﴿٤٣﴾\\
\textamh{44.\  } & لَا يَسْتَـْٔذِنُكَ ٱلَّذِينَ يُؤْمِنُونَ بِٱللَّهِ وَٱلْيَوْمِ ٱلْءَاخِرِ أَن يُجَٰهِدُوا۟ بِأَمْوَٟلِهِمْ وَأَنفُسِهِمْ ۗ وَٱللَّهُ عَلِيمٌۢ بِٱلْمُتَّقِينَ ﴿٤٤﴾\\
\textamh{45.\  } & إِنَّمَا يَسْتَـْٔذِنُكَ ٱلَّذِينَ لَا يُؤْمِنُونَ بِٱللَّهِ وَٱلْيَوْمِ ٱلْءَاخِرِ وَٱرْتَابَتْ قُلُوبُهُمْ فَهُمْ فِى رَيْبِهِمْ يَتَرَدَّدُونَ ﴿٤٥﴾\\
\textamh{46.\  } & ۞ وَلَوْ أَرَادُوا۟ ٱلْخُرُوجَ لَأَعَدُّوا۟ لَهُۥ عُدَّةًۭ وَلَـٰكِن كَرِهَ ٱللَّهُ ٱنۢبِعَاثَهُمْ فَثَبَّطَهُمْ وَقِيلَ ٱقْعُدُوا۟ مَعَ ٱلْقَـٰعِدِينَ ﴿٤٦﴾\\
\textamh{47.\  } & لَوْ خَرَجُوا۟ فِيكُم مَّا زَادُوكُمْ إِلَّا خَبَالًۭا وَلَأَوْضَعُوا۟ خِلَـٰلَكُمْ يَبْغُونَكُمُ ٱلْفِتْنَةَ وَفِيكُمْ سَمَّٰعُونَ لَهُمْ ۗ وَٱللَّهُ عَلِيمٌۢ بِٱلظَّـٰلِمِينَ ﴿٤٧﴾\\
\textamh{48.\  } & لَقَدِ ٱبْتَغَوُا۟ ٱلْفِتْنَةَ مِن قَبْلُ وَقَلَّبُوا۟ لَكَ ٱلْأُمُورَ حَتَّىٰ جَآءَ ٱلْحَقُّ وَظَهَرَ أَمْرُ ٱللَّهِ وَهُمْ كَـٰرِهُونَ ﴿٤٨﴾\\
\textamh{49.\  } & وَمِنْهُم مَّن يَقُولُ ٱئْذَن لِّى وَلَا تَفْتِنِّىٓ ۚ أَلَا فِى ٱلْفِتْنَةِ سَقَطُوا۟ ۗ وَإِنَّ جَهَنَّمَ لَمُحِيطَةٌۢ بِٱلْكَـٰفِرِينَ ﴿٤٩﴾\\
\textamh{50.\  } & إِن تُصِبْكَ حَسَنَةٌۭ تَسُؤْهُمْ ۖ وَإِن تُصِبْكَ مُصِيبَةٌۭ يَقُولُوا۟ قَدْ أَخَذْنَآ أَمْرَنَا مِن قَبْلُ وَيَتَوَلَّوا۟ وَّهُمْ فَرِحُونَ ﴿٥٠﴾\\
\textamh{51.\  } & قُل لَّن يُصِيبَنَآ إِلَّا مَا كَتَبَ ٱللَّهُ لَنَا هُوَ مَوْلَىٰنَا ۚ وَعَلَى ٱللَّهِ فَلْيَتَوَكَّلِ ٱلْمُؤْمِنُونَ ﴿٥١﴾\\
\textamh{52.\  } & قُلْ هَلْ تَرَبَّصُونَ بِنَآ إِلَّآ إِحْدَى ٱلْحُسْنَيَيْنِ ۖ وَنَحْنُ نَتَرَبَّصُ بِكُمْ أَن يُصِيبَكُمُ ٱللَّهُ بِعَذَابٍۢ مِّنْ عِندِهِۦٓ أَوْ بِأَيْدِينَا ۖ فَتَرَبَّصُوٓا۟ إِنَّا مَعَكُم مُّتَرَبِّصُونَ ﴿٥٢﴾\\
\textamh{53.\  } & قُلْ أَنفِقُوا۟ طَوْعًا أَوْ كَرْهًۭا لَّن يُتَقَبَّلَ مِنكُمْ ۖ إِنَّكُمْ كُنتُمْ قَوْمًۭا فَـٰسِقِينَ ﴿٥٣﴾\\
\textamh{54.\  } & وَمَا مَنَعَهُمْ أَن تُقْبَلَ مِنْهُمْ نَفَقَـٰتُهُمْ إِلَّآ أَنَّهُمْ كَفَرُوا۟ بِٱللَّهِ وَبِرَسُولِهِۦ وَلَا يَأْتُونَ ٱلصَّلَوٰةَ إِلَّا وَهُمْ كُسَالَىٰ وَلَا يُنفِقُونَ إِلَّا وَهُمْ كَـٰرِهُونَ ﴿٥٤﴾\\
\textamh{55.\  } & فَلَا تُعْجِبْكَ أَمْوَٟلُهُمْ وَلَآ أَوْلَـٰدُهُمْ ۚ إِنَّمَا يُرِيدُ ٱللَّهُ لِيُعَذِّبَهُم بِهَا فِى ٱلْحَيَوٰةِ ٱلدُّنْيَا وَتَزْهَقَ أَنفُسُهُمْ وَهُمْ كَـٰفِرُونَ ﴿٥٥﴾\\
\textamh{56.\  } & وَيَحْلِفُونَ بِٱللَّهِ إِنَّهُمْ لَمِنكُمْ وَمَا هُم مِّنكُمْ وَلَـٰكِنَّهُمْ قَوْمٌۭ يَفْرَقُونَ ﴿٥٦﴾\\
\textamh{57.\  } & لَوْ يَجِدُونَ مَلْجَـًٔا أَوْ مَغَٰرَٰتٍ أَوْ مُدَّخَلًۭا لَّوَلَّوْا۟ إِلَيْهِ وَهُمْ يَجْمَحُونَ ﴿٥٧﴾\\
\textamh{58.\  } & وَمِنْهُم مَّن يَلْمِزُكَ فِى ٱلصَّدَقَـٰتِ فَإِنْ أُعْطُوا۟ مِنْهَا رَضُوا۟ وَإِن لَّمْ يُعْطَوْا۟ مِنْهَآ إِذَا هُمْ يَسْخَطُونَ ﴿٥٨﴾\\
\textamh{59.\  } & وَلَوْ أَنَّهُمْ رَضُوا۟ مَآ ءَاتَىٰهُمُ ٱللَّهُ وَرَسُولُهُۥ وَقَالُوا۟ حَسْبُنَا ٱللَّهُ سَيُؤْتِينَا ٱللَّهُ مِن فَضْلِهِۦ وَرَسُولُهُۥٓ إِنَّآ إِلَى ٱللَّهِ رَٰغِبُونَ ﴿٥٩﴾\\
\textamh{60.\  } & ۞ إِنَّمَا ٱلصَّدَقَـٰتُ لِلْفُقَرَآءِ وَٱلْمَسَـٰكِينِ وَٱلْعَـٰمِلِينَ عَلَيْهَا وَٱلْمُؤَلَّفَةِ قُلُوبُهُمْ وَفِى ٱلرِّقَابِ وَٱلْغَٰرِمِينَ وَفِى سَبِيلِ ٱللَّهِ وَٱبْنِ ٱلسَّبِيلِ ۖ فَرِيضَةًۭ مِّنَ ٱللَّهِ ۗ وَٱللَّهُ عَلِيمٌ حَكِيمٌۭ ﴿٦٠﴾\\
\textamh{61.\  } & وَمِنْهُمُ ٱلَّذِينَ يُؤْذُونَ ٱلنَّبِىَّ وَيَقُولُونَ هُوَ أُذُنٌۭ ۚ قُلْ أُذُنُ خَيْرٍۢ لَّكُمْ يُؤْمِنُ بِٱللَّهِ وَيُؤْمِنُ لِلْمُؤْمِنِينَ وَرَحْمَةٌۭ لِّلَّذِينَ ءَامَنُوا۟ مِنكُمْ ۚ وَٱلَّذِينَ يُؤْذُونَ رَسُولَ ٱللَّهِ لَهُمْ عَذَابٌ أَلِيمٌۭ ﴿٦١﴾\\
\textamh{62.\  } & يَحْلِفُونَ بِٱللَّهِ لَكُمْ لِيُرْضُوكُمْ وَٱللَّهُ وَرَسُولُهُۥٓ أَحَقُّ أَن يُرْضُوهُ إِن كَانُوا۟ مُؤْمِنِينَ ﴿٦٢﴾\\
\textamh{63.\  } & أَلَمْ يَعْلَمُوٓا۟ أَنَّهُۥ مَن يُحَادِدِ ٱللَّهَ وَرَسُولَهُۥ فَأَنَّ لَهُۥ نَارَ جَهَنَّمَ خَـٰلِدًۭا فِيهَا ۚ ذَٟلِكَ ٱلْخِزْىُ ٱلْعَظِيمُ ﴿٦٣﴾\\
\textamh{64.\  } & يَحْذَرُ ٱلْمُنَـٰفِقُونَ أَن تُنَزَّلَ عَلَيْهِمْ سُورَةٌۭ تُنَبِّئُهُم بِمَا فِى قُلُوبِهِمْ ۚ قُلِ ٱسْتَهْزِءُوٓا۟ إِنَّ ٱللَّهَ مُخْرِجٌۭ مَّا تَحْذَرُونَ ﴿٦٤﴾\\
\textamh{65.\  } & وَلَئِن سَأَلْتَهُمْ لَيَقُولُنَّ إِنَّمَا كُنَّا نَخُوضُ وَنَلْعَبُ ۚ قُلْ أَبِٱللَّهِ وَءَايَـٰتِهِۦ وَرَسُولِهِۦ كُنتُمْ تَسْتَهْزِءُونَ ﴿٦٥﴾\\
\textamh{66.\  } & لَا تَعْتَذِرُوا۟ قَدْ كَفَرْتُم بَعْدَ إِيمَـٰنِكُمْ ۚ إِن نَّعْفُ عَن طَآئِفَةٍۢ مِّنكُمْ نُعَذِّبْ طَآئِفَةًۢ بِأَنَّهُمْ كَانُوا۟ مُجْرِمِينَ ﴿٦٦﴾\\
\textamh{67.\  } & ٱلْمُنَـٰفِقُونَ وَٱلْمُنَـٰفِقَـٰتُ بَعْضُهُم مِّنۢ بَعْضٍۢ ۚ يَأْمُرُونَ بِٱلْمُنكَرِ وَيَنْهَوْنَ عَنِ ٱلْمَعْرُوفِ وَيَقْبِضُونَ أَيْدِيَهُمْ ۚ نَسُوا۟ ٱللَّهَ فَنَسِيَهُمْ ۗ إِنَّ ٱلْمُنَـٰفِقِينَ هُمُ ٱلْفَـٰسِقُونَ ﴿٦٧﴾\\
\textamh{68.\  } & وَعَدَ ٱللَّهُ ٱلْمُنَـٰفِقِينَ وَٱلْمُنَـٰفِقَـٰتِ وَٱلْكُفَّارَ نَارَ جَهَنَّمَ خَـٰلِدِينَ فِيهَا ۚ هِىَ حَسْبُهُمْ ۚ وَلَعَنَهُمُ ٱللَّهُ ۖ وَلَهُمْ عَذَابٌۭ مُّقِيمٌۭ ﴿٦٨﴾\\
\textamh{69.\  } & كَٱلَّذِينَ مِن قَبْلِكُمْ كَانُوٓا۟ أَشَدَّ مِنكُمْ قُوَّةًۭ وَأَكْثَرَ أَمْوَٟلًۭا وَأَوْلَـٰدًۭا فَٱسْتَمْتَعُوا۟ بِخَلَـٰقِهِمْ فَٱسْتَمْتَعْتُم بِخَلَـٰقِكُمْ كَمَا ٱسْتَمْتَعَ ٱلَّذِينَ مِن قَبْلِكُم بِخَلَـٰقِهِمْ وَخُضْتُمْ كَٱلَّذِى خَاضُوٓا۟ ۚ أُو۟لَـٰٓئِكَ حَبِطَتْ أَعْمَـٰلُهُمْ فِى ٱلدُّنْيَا وَٱلْءَاخِرَةِ ۖ وَأُو۟لَـٰٓئِكَ هُمُ ٱلْخَـٰسِرُونَ ﴿٦٩﴾\\
\textamh{70.\  } & أَلَمْ يَأْتِهِمْ نَبَأُ ٱلَّذِينَ مِن قَبْلِهِمْ قَوْمِ نُوحٍۢ وَعَادٍۢ وَثَمُودَ وَقَوْمِ إِبْرَٰهِيمَ وَأَصْحَـٰبِ مَدْيَنَ وَٱلْمُؤْتَفِكَـٰتِ ۚ أَتَتْهُمْ رُسُلُهُم بِٱلْبَيِّنَـٰتِ ۖ فَمَا كَانَ ٱللَّهُ لِيَظْلِمَهُمْ وَلَـٰكِن كَانُوٓا۟ أَنفُسَهُمْ يَظْلِمُونَ ﴿٧٠﴾\\
\textamh{71.\  } & وَٱلْمُؤْمِنُونَ وَٱلْمُؤْمِنَـٰتُ بَعْضُهُمْ أَوْلِيَآءُ بَعْضٍۢ ۚ يَأْمُرُونَ بِٱلْمَعْرُوفِ وَيَنْهَوْنَ عَنِ ٱلْمُنكَرِ وَيُقِيمُونَ ٱلصَّلَوٰةَ وَيُؤْتُونَ ٱلزَّكَوٰةَ وَيُطِيعُونَ ٱللَّهَ وَرَسُولَهُۥٓ ۚ أُو۟لَـٰٓئِكَ سَيَرْحَمُهُمُ ٱللَّهُ ۗ إِنَّ ٱللَّهَ عَزِيزٌ حَكِيمٌۭ ﴿٧١﴾\\
\textamh{72.\  } & وَعَدَ ٱللَّهُ ٱلْمُؤْمِنِينَ وَٱلْمُؤْمِنَـٰتِ جَنَّـٰتٍۢ تَجْرِى مِن تَحْتِهَا ٱلْأَنْهَـٰرُ خَـٰلِدِينَ فِيهَا وَمَسَـٰكِنَ طَيِّبَةًۭ فِى جَنَّـٰتِ عَدْنٍۢ ۚ وَرِضْوَٟنٌۭ مِّنَ ٱللَّهِ أَكْبَرُ ۚ ذَٟلِكَ هُوَ ٱلْفَوْزُ ٱلْعَظِيمُ ﴿٧٢﴾\\
\textamh{73.\  } & يَـٰٓأَيُّهَا ٱلنَّبِىُّ جَٰهِدِ ٱلْكُفَّارَ وَٱلْمُنَـٰفِقِينَ وَٱغْلُظْ عَلَيْهِمْ ۚ وَمَأْوَىٰهُمْ جَهَنَّمُ ۖ وَبِئْسَ ٱلْمَصِيرُ ﴿٧٣﴾\\
\textamh{74.\  } & يَحْلِفُونَ بِٱللَّهِ مَا قَالُوا۟ وَلَقَدْ قَالُوا۟ كَلِمَةَ ٱلْكُفْرِ وَكَفَرُوا۟ بَعْدَ إِسْلَـٰمِهِمْ وَهَمُّوا۟ بِمَا لَمْ يَنَالُوا۟ ۚ وَمَا نَقَمُوٓا۟ إِلَّآ أَنْ أَغْنَىٰهُمُ ٱللَّهُ وَرَسُولُهُۥ مِن فَضْلِهِۦ ۚ فَإِن يَتُوبُوا۟ يَكُ خَيْرًۭا لَّهُمْ ۖ وَإِن يَتَوَلَّوْا۟ يُعَذِّبْهُمُ ٱللَّهُ عَذَابًا أَلِيمًۭا فِى ٱلدُّنْيَا وَٱلْءَاخِرَةِ ۚ وَمَا لَهُمْ فِى ٱلْأَرْضِ مِن وَلِىٍّۢ وَلَا نَصِيرٍۢ ﴿٧٤﴾\\
\textamh{75.\  } & ۞ وَمِنْهُم مَّنْ عَـٰهَدَ ٱللَّهَ لَئِنْ ءَاتَىٰنَا مِن فَضْلِهِۦ لَنَصَّدَّقَنَّ وَلَنَكُونَنَّ مِنَ ٱلصَّـٰلِحِينَ ﴿٧٥﴾\\
\textamh{76.\  } & فَلَمَّآ ءَاتَىٰهُم مِّن فَضْلِهِۦ بَخِلُوا۟ بِهِۦ وَتَوَلَّوا۟ وَّهُم مُّعْرِضُونَ ﴿٧٦﴾\\
\textamh{77.\  } & فَأَعْقَبَهُمْ نِفَاقًۭا فِى قُلُوبِهِمْ إِلَىٰ يَوْمِ يَلْقَوْنَهُۥ بِمَآ أَخْلَفُوا۟ ٱللَّهَ مَا وَعَدُوهُ وَبِمَا كَانُوا۟ يَكْذِبُونَ ﴿٧٧﴾\\
\textamh{78.\  } & أَلَمْ يَعْلَمُوٓا۟ أَنَّ ٱللَّهَ يَعْلَمُ سِرَّهُمْ وَنَجْوَىٰهُمْ وَأَنَّ ٱللَّهَ عَلَّٰمُ ٱلْغُيُوبِ ﴿٧٨﴾\\
\textamh{79.\  } & ٱلَّذِينَ يَلْمِزُونَ ٱلْمُطَّوِّعِينَ مِنَ ٱلْمُؤْمِنِينَ فِى ٱلصَّدَقَـٰتِ وَٱلَّذِينَ لَا يَجِدُونَ إِلَّا جُهْدَهُمْ فَيَسْخَرُونَ مِنْهُمْ ۙ سَخِرَ ٱللَّهُ مِنْهُمْ وَلَهُمْ عَذَابٌ أَلِيمٌ ﴿٧٩﴾\\
\textamh{80.\  } & ٱسْتَغْفِرْ لَهُمْ أَوْ لَا تَسْتَغْفِرْ لَهُمْ إِن تَسْتَغْفِرْ لَهُمْ سَبْعِينَ مَرَّةًۭ فَلَن يَغْفِرَ ٱللَّهُ لَهُمْ ۚ ذَٟلِكَ بِأَنَّهُمْ كَفَرُوا۟ بِٱللَّهِ وَرَسُولِهِۦ ۗ وَٱللَّهُ لَا يَهْدِى ٱلْقَوْمَ ٱلْفَـٰسِقِينَ ﴿٨٠﴾\\
\textamh{81.\  } & فَرِحَ ٱلْمُخَلَّفُونَ بِمَقْعَدِهِمْ خِلَـٰفَ رَسُولِ ٱللَّهِ وَكَرِهُوٓا۟ أَن يُجَٰهِدُوا۟ بِأَمْوَٟلِهِمْ وَأَنفُسِهِمْ فِى سَبِيلِ ٱللَّهِ وَقَالُوا۟ لَا تَنفِرُوا۟ فِى ٱلْحَرِّ ۗ قُلْ نَارُ جَهَنَّمَ أَشَدُّ حَرًّۭا ۚ لَّوْ كَانُوا۟ يَفْقَهُونَ ﴿٨١﴾\\
\textamh{82.\  } & فَلْيَضْحَكُوا۟ قَلِيلًۭا وَلْيَبْكُوا۟ كَثِيرًۭا جَزَآءًۢ بِمَا كَانُوا۟ يَكْسِبُونَ ﴿٨٢﴾\\
\textamh{83.\  } & فَإِن رَّجَعَكَ ٱللَّهُ إِلَىٰ طَآئِفَةٍۢ مِّنْهُمْ فَٱسْتَـْٔذَنُوكَ لِلْخُرُوجِ فَقُل لَّن تَخْرُجُوا۟ مَعِىَ أَبَدًۭا وَلَن تُقَـٰتِلُوا۟ مَعِىَ عَدُوًّا ۖ إِنَّكُمْ رَضِيتُم بِٱلْقُعُودِ أَوَّلَ مَرَّةٍۢ فَٱقْعُدُوا۟ مَعَ ٱلْخَـٰلِفِينَ ﴿٨٣﴾\\
\textamh{84.\  } & وَلَا تُصَلِّ عَلَىٰٓ أَحَدٍۢ مِّنْهُم مَّاتَ أَبَدًۭا وَلَا تَقُمْ عَلَىٰ قَبْرِهِۦٓ ۖ إِنَّهُمْ كَفَرُوا۟ بِٱللَّهِ وَرَسُولِهِۦ وَمَاتُوا۟ وَهُمْ فَـٰسِقُونَ ﴿٨٤﴾\\
\textamh{85.\  } & وَلَا تُعْجِبْكَ أَمْوَٟلُهُمْ وَأَوْلَـٰدُهُمْ ۚ إِنَّمَا يُرِيدُ ٱللَّهُ أَن يُعَذِّبَهُم بِهَا فِى ٱلدُّنْيَا وَتَزْهَقَ أَنفُسُهُمْ وَهُمْ كَـٰفِرُونَ ﴿٨٥﴾\\
\textamh{86.\  } & وَإِذَآ أُنزِلَتْ سُورَةٌ أَنْ ءَامِنُوا۟ بِٱللَّهِ وَجَٰهِدُوا۟ مَعَ رَسُولِهِ ٱسْتَـْٔذَنَكَ أُو۟لُوا۟ ٱلطَّوْلِ مِنْهُمْ وَقَالُوا۟ ذَرْنَا نَكُن مَّعَ ٱلْقَـٰعِدِينَ ﴿٨٦﴾\\
\textamh{87.\  } & رَضُوا۟ بِأَن يَكُونُوا۟ مَعَ ٱلْخَوَالِفِ وَطُبِعَ عَلَىٰ قُلُوبِهِمْ فَهُمْ لَا يَفْقَهُونَ ﴿٨٧﴾\\
\textamh{88.\  } & لَـٰكِنِ ٱلرَّسُولُ وَٱلَّذِينَ ءَامَنُوا۟ مَعَهُۥ جَٰهَدُوا۟ بِأَمْوَٟلِهِمْ وَأَنفُسِهِمْ ۚ وَأُو۟لَـٰٓئِكَ لَهُمُ ٱلْخَيْرَٰتُ ۖ وَأُو۟لَـٰٓئِكَ هُمُ ٱلْمُفْلِحُونَ ﴿٨٨﴾\\
\textamh{89.\  } & أَعَدَّ ٱللَّهُ لَهُمْ جَنَّـٰتٍۢ تَجْرِى مِن تَحْتِهَا ٱلْأَنْهَـٰرُ خَـٰلِدِينَ فِيهَا ۚ ذَٟلِكَ ٱلْفَوْزُ ٱلْعَظِيمُ ﴿٨٩﴾\\
\textamh{90.\  } & وَجَآءَ ٱلْمُعَذِّرُونَ مِنَ ٱلْأَعْرَابِ لِيُؤْذَنَ لَهُمْ وَقَعَدَ ٱلَّذِينَ كَذَبُوا۟ ٱللَّهَ وَرَسُولَهُۥ ۚ سَيُصِيبُ ٱلَّذِينَ كَفَرُوا۟ مِنْهُمْ عَذَابٌ أَلِيمٌۭ ﴿٩٠﴾\\
\textamh{91.\  } & لَّيْسَ عَلَى ٱلضُّعَفَآءِ وَلَا عَلَى ٱلْمَرْضَىٰ وَلَا عَلَى ٱلَّذِينَ لَا يَجِدُونَ مَا يُنفِقُونَ حَرَجٌ إِذَا نَصَحُوا۟ لِلَّهِ وَرَسُولِهِۦ ۚ مَا عَلَى ٱلْمُحْسِنِينَ مِن سَبِيلٍۢ ۚ وَٱللَّهُ غَفُورٌۭ رَّحِيمٌۭ ﴿٩١﴾\\
\textamh{92.\  } & وَلَا عَلَى ٱلَّذِينَ إِذَا مَآ أَتَوْكَ لِتَحْمِلَهُمْ قُلْتَ لَآ أَجِدُ مَآ أَحْمِلُكُمْ عَلَيْهِ تَوَلَّوا۟ وَّأَعْيُنُهُمْ تَفِيضُ مِنَ ٱلدَّمْعِ حَزَنًا أَلَّا يَجِدُوا۟ مَا يُنفِقُونَ ﴿٩٢﴾\\
\textamh{93.\  } & ۞ إِنَّمَا ٱلسَّبِيلُ عَلَى ٱلَّذِينَ يَسْتَـْٔذِنُونَكَ وَهُمْ أَغْنِيَآءُ ۚ رَضُوا۟ بِأَن يَكُونُوا۟ مَعَ ٱلْخَوَالِفِ وَطَبَعَ ٱللَّهُ عَلَىٰ قُلُوبِهِمْ فَهُمْ لَا يَعْلَمُونَ ﴿٩٣﴾\\
\textamh{94.\  } & يَعْتَذِرُونَ إِلَيْكُمْ إِذَا رَجَعْتُمْ إِلَيْهِمْ ۚ قُل لَّا تَعْتَذِرُوا۟ لَن نُّؤْمِنَ لَكُمْ قَدْ نَبَّأَنَا ٱللَّهُ مِنْ أَخْبَارِكُمْ ۚ وَسَيَرَى ٱللَّهُ عَمَلَكُمْ وَرَسُولُهُۥ ثُمَّ تُرَدُّونَ إِلَىٰ عَـٰلِمِ ٱلْغَيْبِ وَٱلشَّهَـٰدَةِ فَيُنَبِّئُكُم بِمَا كُنتُمْ تَعْمَلُونَ ﴿٩٤﴾\\
\textamh{95.\  } & سَيَحْلِفُونَ بِٱللَّهِ لَكُمْ إِذَا ٱنقَلَبْتُمْ إِلَيْهِمْ لِتُعْرِضُوا۟ عَنْهُمْ ۖ فَأَعْرِضُوا۟ عَنْهُمْ ۖ إِنَّهُمْ رِجْسٌۭ ۖ وَمَأْوَىٰهُمْ جَهَنَّمُ جَزَآءًۢ بِمَا كَانُوا۟ يَكْسِبُونَ ﴿٩٥﴾\\
\textamh{96.\  } & يَحْلِفُونَ لَكُمْ لِتَرْضَوْا۟ عَنْهُمْ ۖ فَإِن تَرْضَوْا۟ عَنْهُمْ فَإِنَّ ٱللَّهَ لَا يَرْضَىٰ عَنِ ٱلْقَوْمِ ٱلْفَـٰسِقِينَ ﴿٩٦﴾\\
\textamh{97.\  } & ٱلْأَعْرَابُ أَشَدُّ كُفْرًۭا وَنِفَاقًۭا وَأَجْدَرُ أَلَّا يَعْلَمُوا۟ حُدُودَ مَآ أَنزَلَ ٱللَّهُ عَلَىٰ رَسُولِهِۦ ۗ وَٱللَّهُ عَلِيمٌ حَكِيمٌۭ ﴿٩٧﴾\\
\textamh{98.\  } & وَمِنَ ٱلْأَعْرَابِ مَن يَتَّخِذُ مَا يُنفِقُ مَغْرَمًۭا وَيَتَرَبَّصُ بِكُمُ ٱلدَّوَآئِرَ ۚ عَلَيْهِمْ دَآئِرَةُ ٱلسَّوْءِ ۗ وَٱللَّهُ سَمِيعٌ عَلِيمٌۭ ﴿٩٨﴾\\
\textamh{99.\  } & وَمِنَ ٱلْأَعْرَابِ مَن يُؤْمِنُ بِٱللَّهِ وَٱلْيَوْمِ ٱلْءَاخِرِ وَيَتَّخِذُ مَا يُنفِقُ قُرُبَٰتٍ عِندَ ٱللَّهِ وَصَلَوَٟتِ ٱلرَّسُولِ ۚ أَلَآ إِنَّهَا قُرْبَةٌۭ لَّهُمْ ۚ سَيُدْخِلُهُمُ ٱللَّهُ فِى رَحْمَتِهِۦٓ ۗ إِنَّ ٱللَّهَ غَفُورٌۭ رَّحِيمٌۭ ﴿٩٩﴾\\
\textamh{100.\  } & وَٱلسَّٰبِقُونَ ٱلْأَوَّلُونَ مِنَ ٱلْمُهَـٰجِرِينَ وَٱلْأَنصَارِ وَٱلَّذِينَ ٱتَّبَعُوهُم بِإِحْسَـٰنٍۢ رَّضِىَ ٱللَّهُ عَنْهُمْ وَرَضُوا۟ عَنْهُ وَأَعَدَّ لَهُمْ جَنَّـٰتٍۢ تَجْرِى تَحْتَهَا ٱلْأَنْهَـٰرُ خَـٰلِدِينَ فِيهَآ أَبَدًۭا ۚ ذَٟلِكَ ٱلْفَوْزُ ٱلْعَظِيمُ ﴿١٠٠﴾\\
\textamh{101.\  } & وَمِمَّنْ حَوْلَكُم مِّنَ ٱلْأَعْرَابِ مُنَـٰفِقُونَ ۖ وَمِنْ أَهْلِ ٱلْمَدِينَةِ ۖ مَرَدُوا۟ عَلَى ٱلنِّفَاقِ لَا تَعْلَمُهُمْ ۖ نَحْنُ نَعْلَمُهُمْ ۚ سَنُعَذِّبُهُم مَّرَّتَيْنِ ثُمَّ يُرَدُّونَ إِلَىٰ عَذَابٍ عَظِيمٍۢ ﴿١٠١﴾\\
\textamh{102.\  } & وَءَاخَرُونَ ٱعْتَرَفُوا۟ بِذُنُوبِهِمْ خَلَطُوا۟ عَمَلًۭا صَـٰلِحًۭا وَءَاخَرَ سَيِّئًا عَسَى ٱللَّهُ أَن يَتُوبَ عَلَيْهِمْ ۚ إِنَّ ٱللَّهَ غَفُورٌۭ رَّحِيمٌ ﴿١٠٢﴾\\
\textamh{103.\  } & خُذْ مِنْ أَمْوَٟلِهِمْ صَدَقَةًۭ تُطَهِّرُهُمْ وَتُزَكِّيهِم بِهَا وَصَلِّ عَلَيْهِمْ ۖ إِنَّ صَلَوٰتَكَ سَكَنٌۭ لَّهُمْ ۗ وَٱللَّهُ سَمِيعٌ عَلِيمٌ ﴿١٠٣﴾\\
\textamh{104.\  } & أَلَمْ يَعْلَمُوٓا۟ أَنَّ ٱللَّهَ هُوَ يَقْبَلُ ٱلتَّوْبَةَ عَنْ عِبَادِهِۦ وَيَأْخُذُ ٱلصَّدَقَـٰتِ وَأَنَّ ٱللَّهَ هُوَ ٱلتَّوَّابُ ٱلرَّحِيمُ ﴿١٠٤﴾\\
\textamh{105.\  } & وَقُلِ ٱعْمَلُوا۟ فَسَيَرَى ٱللَّهُ عَمَلَكُمْ وَرَسُولُهُۥ وَٱلْمُؤْمِنُونَ ۖ وَسَتُرَدُّونَ إِلَىٰ عَـٰلِمِ ٱلْغَيْبِ وَٱلشَّهَـٰدَةِ فَيُنَبِّئُكُم بِمَا كُنتُمْ تَعْمَلُونَ ﴿١٠٥﴾\\
\textamh{106.\  } & وَءَاخَرُونَ مُرْجَوْنَ لِأَمْرِ ٱللَّهِ إِمَّا يُعَذِّبُهُمْ وَإِمَّا يَتُوبُ عَلَيْهِمْ ۗ وَٱللَّهُ عَلِيمٌ حَكِيمٌۭ ﴿١٠٦﴾\\
\textamh{107.\  } & وَٱلَّذِينَ ٱتَّخَذُوا۟ مَسْجِدًۭا ضِرَارًۭا وَكُفْرًۭا وَتَفْرِيقًۢا بَيْنَ ٱلْمُؤْمِنِينَ وَإِرْصَادًۭا لِّمَنْ حَارَبَ ٱللَّهَ وَرَسُولَهُۥ مِن قَبْلُ ۚ وَلَيَحْلِفُنَّ إِنْ أَرَدْنَآ إِلَّا ٱلْحُسْنَىٰ ۖ وَٱللَّهُ يَشْهَدُ إِنَّهُمْ لَكَـٰذِبُونَ ﴿١٠٧﴾\\
\textamh{108.\  } & لَا تَقُمْ فِيهِ أَبَدًۭا ۚ لَّمَسْجِدٌ أُسِّسَ عَلَى ٱلتَّقْوَىٰ مِنْ أَوَّلِ يَوْمٍ أَحَقُّ أَن تَقُومَ فِيهِ ۚ فِيهِ رِجَالٌۭ يُحِبُّونَ أَن يَتَطَهَّرُوا۟ ۚ وَٱللَّهُ يُحِبُّ ٱلْمُطَّهِّرِينَ ﴿١٠٨﴾\\
\textamh{109.\  } & أَفَمَنْ أَسَّسَ بُنْيَـٰنَهُۥ عَلَىٰ تَقْوَىٰ مِنَ ٱللَّهِ وَرِضْوَٟنٍ خَيْرٌ أَم مَّنْ أَسَّسَ بُنْيَـٰنَهُۥ عَلَىٰ شَفَا جُرُفٍ هَارٍۢ فَٱنْهَارَ بِهِۦ فِى نَارِ جَهَنَّمَ ۗ وَٱللَّهُ لَا يَهْدِى ٱلْقَوْمَ ٱلظَّـٰلِمِينَ ﴿١٠٩﴾\\
\textamh{110.\  } & لَا يَزَالُ بُنْيَـٰنُهُمُ ٱلَّذِى بَنَوْا۟ رِيبَةًۭ فِى قُلُوبِهِمْ إِلَّآ أَن تَقَطَّعَ قُلُوبُهُمْ ۗ وَٱللَّهُ عَلِيمٌ حَكِيمٌ ﴿١١٠﴾\\
\textamh{111.\  } & ۞ إِنَّ ٱللَّهَ ٱشْتَرَىٰ مِنَ ٱلْمُؤْمِنِينَ أَنفُسَهُمْ وَأَمْوَٟلَهُم بِأَنَّ لَهُمُ ٱلْجَنَّةَ ۚ يُقَـٰتِلُونَ فِى سَبِيلِ ٱللَّهِ فَيَقْتُلُونَ وَيُقْتَلُونَ ۖ وَعْدًا عَلَيْهِ حَقًّۭا فِى ٱلتَّوْرَىٰةِ وَٱلْإِنجِيلِ وَٱلْقُرْءَانِ ۚ وَمَنْ أَوْفَىٰ بِعَهْدِهِۦ مِنَ ٱللَّهِ ۚ فَٱسْتَبْشِرُوا۟ بِبَيْعِكُمُ ٱلَّذِى بَايَعْتُم بِهِۦ ۚ وَذَٟلِكَ هُوَ ٱلْفَوْزُ ٱلْعَظِيمُ ﴿١١١﴾\\
\textamh{112.\  } & ٱلتَّٰٓئِبُونَ ٱلْعَـٰبِدُونَ ٱلْحَـٰمِدُونَ ٱلسَّٰٓئِحُونَ ٱلرَّٟكِعُونَ ٱلسَّٰجِدُونَ ٱلْءَامِرُونَ بِٱلْمَعْرُوفِ وَٱلنَّاهُونَ عَنِ ٱلْمُنكَرِ وَٱلْحَـٰفِظُونَ لِحُدُودِ ٱللَّهِ ۗ وَبَشِّرِ ٱلْمُؤْمِنِينَ ﴿١١٢﴾\\
\textamh{113.\  } & مَا كَانَ لِلنَّبِىِّ وَٱلَّذِينَ ءَامَنُوٓا۟ أَن يَسْتَغْفِرُوا۟ لِلْمُشْرِكِينَ وَلَوْ كَانُوٓا۟ أُو۟لِى قُرْبَىٰ مِنۢ بَعْدِ مَا تَبَيَّنَ لَهُمْ أَنَّهُمْ أَصْحَـٰبُ ٱلْجَحِيمِ ﴿١١٣﴾\\
\textamh{114.\  } & وَمَا كَانَ ٱسْتِغْفَارُ إِبْرَٰهِيمَ لِأَبِيهِ إِلَّا عَن مَّوْعِدَةٍۢ وَعَدَهَآ إِيَّاهُ فَلَمَّا تَبَيَّنَ لَهُۥٓ أَنَّهُۥ عَدُوٌّۭ لِّلَّهِ تَبَرَّأَ مِنْهُ ۚ إِنَّ إِبْرَٰهِيمَ لَأَوَّٰهٌ حَلِيمٌۭ ﴿١١٤﴾\\
\textamh{115.\  } & وَمَا كَانَ ٱللَّهُ لِيُضِلَّ قَوْمًۢا بَعْدَ إِذْ هَدَىٰهُمْ حَتَّىٰ يُبَيِّنَ لَهُم مَّا يَتَّقُونَ ۚ إِنَّ ٱللَّهَ بِكُلِّ شَىْءٍ عَلِيمٌ ﴿١١٥﴾\\
\textamh{116.\  } & إِنَّ ٱللَّهَ لَهُۥ مُلْكُ ٱلسَّمَـٰوَٟتِ وَٱلْأَرْضِ ۖ يُحْىِۦ وَيُمِيتُ ۚ وَمَا لَكُم مِّن دُونِ ٱللَّهِ مِن وَلِىٍّۢ وَلَا نَصِيرٍۢ ﴿١١٦﴾\\
\textamh{117.\  } & لَّقَد تَّابَ ٱللَّهُ عَلَى ٱلنَّبِىِّ وَٱلْمُهَـٰجِرِينَ وَٱلْأَنصَارِ ٱلَّذِينَ ٱتَّبَعُوهُ فِى سَاعَةِ ٱلْعُسْرَةِ مِنۢ بَعْدِ مَا كَادَ يَزِيغُ قُلُوبُ فَرِيقٍۢ مِّنْهُمْ ثُمَّ تَابَ عَلَيْهِمْ ۚ إِنَّهُۥ بِهِمْ رَءُوفٌۭ رَّحِيمٌۭ ﴿١١٧﴾\\
\textamh{118.\  } & وَعَلَى ٱلثَّلَـٰثَةِ ٱلَّذِينَ خُلِّفُوا۟ حَتَّىٰٓ إِذَا ضَاقَتْ عَلَيْهِمُ ٱلْأَرْضُ بِمَا رَحُبَتْ وَضَاقَتْ عَلَيْهِمْ أَنفُسُهُمْ وَظَنُّوٓا۟ أَن لَّا مَلْجَأَ مِنَ ٱللَّهِ إِلَّآ إِلَيْهِ ثُمَّ تَابَ عَلَيْهِمْ لِيَتُوبُوٓا۟ ۚ إِنَّ ٱللَّهَ هُوَ ٱلتَّوَّابُ ٱلرَّحِيمُ ﴿١١٨﴾\\
\textamh{119.\  } & يَـٰٓأَيُّهَا ٱلَّذِينَ ءَامَنُوا۟ ٱتَّقُوا۟ ٱللَّهَ وَكُونُوا۟ مَعَ ٱلصَّـٰدِقِينَ ﴿١١٩﴾\\
\textamh{120.\  } & مَا كَانَ لِأَهْلِ ٱلْمَدِينَةِ وَمَنْ حَوْلَهُم مِّنَ ٱلْأَعْرَابِ أَن يَتَخَلَّفُوا۟ عَن رَّسُولِ ٱللَّهِ وَلَا يَرْغَبُوا۟ بِأَنفُسِهِمْ عَن نَّفْسِهِۦ ۚ ذَٟلِكَ بِأَنَّهُمْ لَا يُصِيبُهُمْ ظَمَأٌۭ وَلَا نَصَبٌۭ وَلَا مَخْمَصَةٌۭ فِى سَبِيلِ ٱللَّهِ وَلَا يَطَـُٔونَ مَوْطِئًۭا يَغِيظُ ٱلْكُفَّارَ وَلَا يَنَالُونَ مِنْ عَدُوٍّۢ نَّيْلًا إِلَّا كُتِبَ لَهُم بِهِۦ عَمَلٌۭ صَـٰلِحٌ ۚ إِنَّ ٱللَّهَ لَا يُضِيعُ أَجْرَ ٱلْمُحْسِنِينَ ﴿١٢٠﴾\\
\textamh{121.\  } & وَلَا يُنفِقُونَ نَفَقَةًۭ صَغِيرَةًۭ وَلَا كَبِيرَةًۭ وَلَا يَقْطَعُونَ وَادِيًا إِلَّا كُتِبَ لَهُمْ لِيَجْزِيَهُمُ ٱللَّهُ أَحْسَنَ مَا كَانُوا۟ يَعْمَلُونَ ﴿١٢١﴾\\
\textamh{122.\  } & ۞ وَمَا كَانَ ٱلْمُؤْمِنُونَ لِيَنفِرُوا۟ كَآفَّةًۭ ۚ فَلَوْلَا نَفَرَ مِن كُلِّ فِرْقَةٍۢ مِّنْهُمْ طَآئِفَةٌۭ لِّيَتَفَقَّهُوا۟ فِى ٱلدِّينِ وَلِيُنذِرُوا۟ قَوْمَهُمْ إِذَا رَجَعُوٓا۟ إِلَيْهِمْ لَعَلَّهُمْ يَحْذَرُونَ ﴿١٢٢﴾\\
\textamh{123.\  } & يَـٰٓأَيُّهَا ٱلَّذِينَ ءَامَنُوا۟ قَـٰتِلُوا۟ ٱلَّذِينَ يَلُونَكُم مِّنَ ٱلْكُفَّارِ وَلْيَجِدُوا۟ فِيكُمْ غِلْظَةًۭ ۚ وَٱعْلَمُوٓا۟ أَنَّ ٱللَّهَ مَعَ ٱلْمُتَّقِينَ ﴿١٢٣﴾\\
\textamh{124.\  } & وَإِذَا مَآ أُنزِلَتْ سُورَةٌۭ فَمِنْهُم مَّن يَقُولُ أَيُّكُمْ زَادَتْهُ هَـٰذِهِۦٓ إِيمَـٰنًۭا ۚ فَأَمَّا ٱلَّذِينَ ءَامَنُوا۟ فَزَادَتْهُمْ إِيمَـٰنًۭا وَهُمْ يَسْتَبْشِرُونَ ﴿١٢٤﴾\\
\textamh{125.\  } & وَأَمَّا ٱلَّذِينَ فِى قُلُوبِهِم مَّرَضٌۭ فَزَادَتْهُمْ رِجْسًا إِلَىٰ رِجْسِهِمْ وَمَاتُوا۟ وَهُمْ كَـٰفِرُونَ ﴿١٢٥﴾\\
\textamh{126.\  } & أَوَلَا يَرَوْنَ أَنَّهُمْ يُفْتَنُونَ فِى كُلِّ عَامٍۢ مَّرَّةً أَوْ مَرَّتَيْنِ ثُمَّ لَا يَتُوبُونَ وَلَا هُمْ يَذَّكَّرُونَ ﴿١٢٦﴾\\
\textamh{127.\  } & وَإِذَا مَآ أُنزِلَتْ سُورَةٌۭ نَّظَرَ بَعْضُهُمْ إِلَىٰ بَعْضٍ هَلْ يَرَىٰكُم مِّنْ أَحَدٍۢ ثُمَّ ٱنصَرَفُوا۟ ۚ صَرَفَ ٱللَّهُ قُلُوبَهُم بِأَنَّهُمْ قَوْمٌۭ لَّا يَفْقَهُونَ ﴿١٢٧﴾\\
\textamh{128.\  } & لَقَدْ جَآءَكُمْ رَسُولٌۭ مِّنْ أَنفُسِكُمْ عَزِيزٌ عَلَيْهِ مَا عَنِتُّمْ حَرِيصٌ عَلَيْكُم بِٱلْمُؤْمِنِينَ رَءُوفٌۭ رَّحِيمٌۭ ﴿١٢٨﴾\\
\textamh{129.\  } & فَإِن تَوَلَّوْا۟ فَقُلْ حَسْبِىَ ٱللَّهُ لَآ إِلَـٰهَ إِلَّا هُوَ ۖ عَلَيْهِ تَوَكَّلْتُ ۖ وَهُوَ رَبُّ ٱلْعَرْشِ ٱلْعَظِيمِ ﴿١٢٩﴾\\
\end{longtable} \newpage

%% License: BSD style (Berkley) (i.e. Put the Copyright owner's name always)
%% Writer and Copyright (to): Bewketu(Bilal) Tadilo (2016-17)
\centering\section{\LR{\textamharic{ሱራቱ ዩኑስ -}  \RL{سوره  يونس}}}
\begin{longtable}{%
  @{}
    p{.5\textwidth}
  @{~~~~~~~~~~~~~}
    p{.5\textwidth}
    @{}
}
\nopagebreak
\textamh{\ \ \ \ \ \  ቢስሚላሂ አራህመኒ ራሂይም } &  بِسْمِ ٱللَّهِ ٱلرَّحْمَـٰنِ ٱلرَّحِيمِ\\
\textamh{1.\  } &  الٓر ۚ تِلْكَ ءَايَـٰتُ ٱلْكِتَـٰبِ ٱلْحَكِيمِ ﴿١﴾\\
\textamh{2.\  } & أَكَانَ لِلنَّاسِ عَجَبًا أَنْ أَوْحَيْنَآ إِلَىٰ رَجُلٍۢ مِّنْهُمْ أَنْ أَنذِرِ ٱلنَّاسَ وَبَشِّرِ ٱلَّذِينَ ءَامَنُوٓا۟ أَنَّ لَهُمْ قَدَمَ صِدْقٍ عِندَ رَبِّهِمْ ۗ قَالَ ٱلْكَـٰفِرُونَ إِنَّ هَـٰذَا لَسَـٰحِرٌۭ مُّبِينٌ ﴿٢﴾\\
\textamh{3.\  } & إِنَّ رَبَّكُمُ ٱللَّهُ ٱلَّذِى خَلَقَ ٱلسَّمَـٰوَٟتِ وَٱلْأَرْضَ فِى سِتَّةِ أَيَّامٍۢ ثُمَّ ٱسْتَوَىٰ عَلَى ٱلْعَرْشِ ۖ يُدَبِّرُ ٱلْأَمْرَ ۖ مَا مِن شَفِيعٍ إِلَّا مِنۢ بَعْدِ إِذْنِهِۦ ۚ ذَٟلِكُمُ ٱللَّهُ رَبُّكُمْ فَٱعْبُدُوهُ ۚ أَفَلَا تَذَكَّرُونَ ﴿٣﴾\\
\textamh{4.\  } & إِلَيْهِ مَرْجِعُكُمْ جَمِيعًۭا ۖ وَعْدَ ٱللَّهِ حَقًّا ۚ إِنَّهُۥ يَبْدَؤُا۟ ٱلْخَلْقَ ثُمَّ يُعِيدُهُۥ لِيَجْزِىَ ٱلَّذِينَ ءَامَنُوا۟ وَعَمِلُوا۟ ٱلصَّـٰلِحَـٰتِ بِٱلْقِسْطِ ۚ وَٱلَّذِينَ كَفَرُوا۟ لَهُمْ شَرَابٌۭ مِّنْ حَمِيمٍۢ وَعَذَابٌ أَلِيمٌۢ بِمَا كَانُوا۟ يَكْفُرُونَ ﴿٤﴾\\
\textamh{5.\  } & هُوَ ٱلَّذِى جَعَلَ ٱلشَّمْسَ ضِيَآءًۭ وَٱلْقَمَرَ نُورًۭا وَقَدَّرَهُۥ مَنَازِلَ لِتَعْلَمُوا۟ عَدَدَ ٱلسِّنِينَ وَٱلْحِسَابَ ۚ مَا خَلَقَ ٱللَّهُ ذَٟلِكَ إِلَّا بِٱلْحَقِّ ۚ يُفَصِّلُ ٱلْءَايَـٰتِ لِقَوْمٍۢ يَعْلَمُونَ ﴿٥﴾\\
\textamh{6.\  } & إِنَّ فِى ٱخْتِلَـٰفِ ٱلَّيْلِ وَٱلنَّهَارِ وَمَا خَلَقَ ٱللَّهُ فِى ٱلسَّمَـٰوَٟتِ وَٱلْأَرْضِ لَءَايَـٰتٍۢ لِّقَوْمٍۢ يَتَّقُونَ ﴿٦﴾\\
\textamh{7.\  } & إِنَّ ٱلَّذِينَ لَا يَرْجُونَ لِقَآءَنَا وَرَضُوا۟ بِٱلْحَيَوٰةِ ٱلدُّنْيَا وَٱطْمَأَنُّوا۟ بِهَا وَٱلَّذِينَ هُمْ عَنْ ءَايَـٰتِنَا غَٰفِلُونَ ﴿٧﴾\\
\textamh{8.\  } & أُو۟لَـٰٓئِكَ مَأْوَىٰهُمُ ٱلنَّارُ بِمَا كَانُوا۟ يَكْسِبُونَ ﴿٨﴾\\
\textamh{9.\  } & إِنَّ ٱلَّذِينَ ءَامَنُوا۟ وَعَمِلُوا۟ ٱلصَّـٰلِحَـٰتِ يَهْدِيهِمْ رَبُّهُم بِإِيمَـٰنِهِمْ ۖ تَجْرِى مِن تَحْتِهِمُ ٱلْأَنْهَـٰرُ فِى جَنَّـٰتِ ٱلنَّعِيمِ ﴿٩﴾\\
\textamh{10.\  } & دَعْوَىٰهُمْ فِيهَا سُبْحَـٰنَكَ ٱللَّهُمَّ وَتَحِيَّتُهُمْ فِيهَا سَلَـٰمٌۭ ۚ وَءَاخِرُ دَعْوَىٰهُمْ أَنِ ٱلْحَمْدُ لِلَّهِ رَبِّ ٱلْعَـٰلَمِينَ ﴿١٠﴾\\
\textamh{11.\  } & ۞ وَلَوْ يُعَجِّلُ ٱللَّهُ لِلنَّاسِ ٱلشَّرَّ ٱسْتِعْجَالَهُم بِٱلْخَيْرِ لَقُضِىَ إِلَيْهِمْ أَجَلُهُمْ ۖ فَنَذَرُ ٱلَّذِينَ لَا يَرْجُونَ لِقَآءَنَا فِى طُغْيَـٰنِهِمْ يَعْمَهُونَ ﴿١١﴾\\
\textamh{12.\  } & وَإِذَا مَسَّ ٱلْإِنسَـٰنَ ٱلضُّرُّ دَعَانَا لِجَنۢبِهِۦٓ أَوْ قَاعِدًا أَوْ قَآئِمًۭا فَلَمَّا كَشَفْنَا عَنْهُ ضُرَّهُۥ مَرَّ كَأَن لَّمْ يَدْعُنَآ إِلَىٰ ضُرٍّۢ مَّسَّهُۥ ۚ كَذَٟلِكَ زُيِّنَ لِلْمُسْرِفِينَ مَا كَانُوا۟ يَعْمَلُونَ ﴿١٢﴾\\
\textamh{13.\  } & وَلَقَدْ أَهْلَكْنَا ٱلْقُرُونَ مِن قَبْلِكُمْ لَمَّا ظَلَمُوا۟ ۙ وَجَآءَتْهُمْ رُسُلُهُم بِٱلْبَيِّنَـٰتِ وَمَا كَانُوا۟ لِيُؤْمِنُوا۟ ۚ كَذَٟلِكَ نَجْزِى ٱلْقَوْمَ ٱلْمُجْرِمِينَ ﴿١٣﴾\\
\textamh{14.\  } & ثُمَّ جَعَلْنَـٰكُمْ خَلَـٰٓئِفَ فِى ٱلْأَرْضِ مِنۢ بَعْدِهِمْ لِنَنظُرَ كَيْفَ تَعْمَلُونَ ﴿١٤﴾\\
\textamh{15.\  } & وَإِذَا تُتْلَىٰ عَلَيْهِمْ ءَايَاتُنَا بَيِّنَـٰتٍۢ ۙ قَالَ ٱلَّذِينَ لَا يَرْجُونَ لِقَآءَنَا ٱئْتِ بِقُرْءَانٍ غَيْرِ هَـٰذَآ أَوْ بَدِّلْهُ ۚ قُلْ مَا يَكُونُ لِىٓ أَنْ أُبَدِّلَهُۥ مِن تِلْقَآئِ نَفْسِىٓ ۖ إِنْ أَتَّبِعُ إِلَّا مَا يُوحَىٰٓ إِلَىَّ ۖ إِنِّىٓ أَخَافُ إِنْ عَصَيْتُ رَبِّى عَذَابَ يَوْمٍ عَظِيمٍۢ ﴿١٥﴾\\
\textamh{16.\  } & قُل لَّوْ شَآءَ ٱللَّهُ مَا تَلَوْتُهُۥ عَلَيْكُمْ وَلَآ أَدْرَىٰكُم بِهِۦ ۖ فَقَدْ لَبِثْتُ فِيكُمْ عُمُرًۭا مِّن قَبْلِهِۦٓ ۚ أَفَلَا تَعْقِلُونَ ﴿١٦﴾\\
\textamh{17.\  } & فَمَنْ أَظْلَمُ مِمَّنِ ٱفْتَرَىٰ عَلَى ٱللَّهِ كَذِبًا أَوْ كَذَّبَ بِـَٔايَـٰتِهِۦٓ ۚ إِنَّهُۥ لَا يُفْلِحُ ٱلْمُجْرِمُونَ ﴿١٧﴾\\
\textamh{18.\  } & وَيَعْبُدُونَ مِن دُونِ ٱللَّهِ مَا لَا يَضُرُّهُمْ وَلَا يَنفَعُهُمْ وَيَقُولُونَ هَـٰٓؤُلَآءِ شُفَعَـٰٓؤُنَا عِندَ ٱللَّهِ ۚ قُلْ أَتُنَبِّـُٔونَ ٱللَّهَ بِمَا لَا يَعْلَمُ فِى ٱلسَّمَـٰوَٟتِ وَلَا فِى ٱلْأَرْضِ ۚ سُبْحَـٰنَهُۥ وَتَعَـٰلَىٰ عَمَّا يُشْرِكُونَ ﴿١٨﴾\\
\textamh{19.\  } & وَمَا كَانَ ٱلنَّاسُ إِلَّآ أُمَّةًۭ وَٟحِدَةًۭ فَٱخْتَلَفُوا۟ ۚ وَلَوْلَا كَلِمَةٌۭ سَبَقَتْ مِن رَّبِّكَ لَقُضِىَ بَيْنَهُمْ فِيمَا فِيهِ يَخْتَلِفُونَ ﴿١٩﴾\\
\textamh{20.\  } & وَيَقُولُونَ لَوْلَآ أُنزِلَ عَلَيْهِ ءَايَةٌۭ مِّن رَّبِّهِۦ ۖ فَقُلْ إِنَّمَا ٱلْغَيْبُ لِلَّهِ فَٱنتَظِرُوٓا۟ إِنِّى مَعَكُم مِّنَ ٱلْمُنتَظِرِينَ ﴿٢٠﴾\\
\textamh{21.\  } & وَإِذَآ أَذَقْنَا ٱلنَّاسَ رَحْمَةًۭ مِّنۢ بَعْدِ ضَرَّآءَ مَسَّتْهُمْ إِذَا لَهُم مَّكْرٌۭ فِىٓ ءَايَاتِنَا ۚ قُلِ ٱللَّهُ أَسْرَعُ مَكْرًا ۚ إِنَّ رُسُلَنَا يَكْتُبُونَ مَا تَمْكُرُونَ ﴿٢١﴾\\
\textamh{22.\  } & هُوَ ٱلَّذِى يُسَيِّرُكُمْ فِى ٱلْبَرِّ وَٱلْبَحْرِ ۖ حَتَّىٰٓ إِذَا كُنتُمْ فِى ٱلْفُلْكِ وَجَرَيْنَ بِهِم بِرِيحٍۢ طَيِّبَةٍۢ وَفَرِحُوا۟ بِهَا جَآءَتْهَا رِيحٌ عَاصِفٌۭ وَجَآءَهُمُ ٱلْمَوْجُ مِن كُلِّ مَكَانٍۢ وَظَنُّوٓا۟ أَنَّهُمْ أُحِيطَ بِهِمْ ۙ دَعَوُا۟ ٱللَّهَ مُخْلِصِينَ لَهُ ٱلدِّينَ لَئِنْ أَنجَيْتَنَا مِنْ هَـٰذِهِۦ لَنَكُونَنَّ مِنَ ٱلشَّـٰكِرِينَ ﴿٢٢﴾\\
\textamh{23.\  } & فَلَمَّآ أَنجَىٰهُمْ إِذَا هُمْ يَبْغُونَ فِى ٱلْأَرْضِ بِغَيْرِ ٱلْحَقِّ ۗ يَـٰٓأَيُّهَا ٱلنَّاسُ إِنَّمَا بَغْيُكُمْ عَلَىٰٓ أَنفُسِكُم ۖ مَّتَـٰعَ ٱلْحَيَوٰةِ ٱلدُّنْيَا ۖ ثُمَّ إِلَيْنَا مَرْجِعُكُمْ فَنُنَبِّئُكُم بِمَا كُنتُمْ تَعْمَلُونَ ﴿٢٣﴾\\
\textamh{24.\  } & إِنَّمَا مَثَلُ ٱلْحَيَوٰةِ ٱلدُّنْيَا كَمَآءٍ أَنزَلْنَـٰهُ مِنَ ٱلسَّمَآءِ فَٱخْتَلَطَ بِهِۦ نَبَاتُ ٱلْأَرْضِ مِمَّا يَأْكُلُ ٱلنَّاسُ وَٱلْأَنْعَـٰمُ حَتَّىٰٓ إِذَآ أَخَذَتِ ٱلْأَرْضُ زُخْرُفَهَا وَٱزَّيَّنَتْ وَظَنَّ أَهْلُهَآ أَنَّهُمْ قَـٰدِرُونَ عَلَيْهَآ أَتَىٰهَآ أَمْرُنَا لَيْلًا أَوْ نَهَارًۭا فَجَعَلْنَـٰهَا حَصِيدًۭا كَأَن لَّمْ تَغْنَ بِٱلْأَمْسِ ۚ كَذَٟلِكَ نُفَصِّلُ ٱلْءَايَـٰتِ لِقَوْمٍۢ يَتَفَكَّرُونَ ﴿٢٤﴾\\
\textamh{25.\  } & وَٱللَّهُ يَدْعُوٓا۟ إِلَىٰ دَارِ ٱلسَّلَـٰمِ وَيَهْدِى مَن يَشَآءُ إِلَىٰ صِرَٰطٍۢ مُّسْتَقِيمٍۢ ﴿٢٥﴾\\
\textamh{26.\  } & ۞ لِّلَّذِينَ أَحْسَنُوا۟ ٱلْحُسْنَىٰ وَزِيَادَةٌۭ ۖ وَلَا يَرْهَقُ وُجُوهَهُمْ قَتَرٌۭ وَلَا ذِلَّةٌ ۚ أُو۟لَـٰٓئِكَ أَصْحَـٰبُ ٱلْجَنَّةِ ۖ هُمْ فِيهَا خَـٰلِدُونَ ﴿٢٦﴾\\
\textamh{27.\  } & وَٱلَّذِينَ كَسَبُوا۟ ٱلسَّيِّـَٔاتِ جَزَآءُ سَيِّئَةٍۭ بِمِثْلِهَا وَتَرْهَقُهُمْ ذِلَّةٌۭ ۖ مَّا لَهُم مِّنَ ٱللَّهِ مِنْ عَاصِمٍۢ ۖ كَأَنَّمَآ أُغْشِيَتْ وُجُوهُهُمْ قِطَعًۭا مِّنَ ٱلَّيْلِ مُظْلِمًا ۚ أُو۟لَـٰٓئِكَ أَصْحَـٰبُ ٱلنَّارِ ۖ هُمْ فِيهَا خَـٰلِدُونَ ﴿٢٧﴾\\
\textamh{28.\  } & وَيَوْمَ نَحْشُرُهُمْ جَمِيعًۭا ثُمَّ نَقُولُ لِلَّذِينَ أَشْرَكُوا۟ مَكَانَكُمْ أَنتُمْ وَشُرَكَآؤُكُمْ ۚ فَزَيَّلْنَا بَيْنَهُمْ ۖ وَقَالَ شُرَكَآؤُهُم مَّا كُنتُمْ إِيَّانَا تَعْبُدُونَ ﴿٢٨﴾\\
\textamh{29.\  } & فَكَفَىٰ بِٱللَّهِ شَهِيدًۢا بَيْنَنَا وَبَيْنَكُمْ إِن كُنَّا عَنْ عِبَادَتِكُمْ لَغَٰفِلِينَ ﴿٢٩﴾\\
\textamh{30.\  } & هُنَالِكَ تَبْلُوا۟ كُلُّ نَفْسٍۢ مَّآ أَسْلَفَتْ ۚ وَرُدُّوٓا۟ إِلَى ٱللَّهِ مَوْلَىٰهُمُ ٱلْحَقِّ ۖ وَضَلَّ عَنْهُم مَّا كَانُوا۟ يَفْتَرُونَ ﴿٣٠﴾\\
\textamh{31.\  } & قُلْ مَن يَرْزُقُكُم مِّنَ ٱلسَّمَآءِ وَٱلْأَرْضِ أَمَّن يَمْلِكُ ٱلسَّمْعَ وَٱلْأَبْصَـٰرَ وَمَن يُخْرِجُ ٱلْحَىَّ مِنَ ٱلْمَيِّتِ وَيُخْرِجُ ٱلْمَيِّتَ مِنَ ٱلْحَىِّ وَمَن يُدَبِّرُ ٱلْأَمْرَ ۚ فَسَيَقُولُونَ ٱللَّهُ ۚ فَقُلْ أَفَلَا تَتَّقُونَ ﴿٣١﴾\\
\textamh{32.\  } & فَذَٟلِكُمُ ٱللَّهُ رَبُّكُمُ ٱلْحَقُّ ۖ فَمَاذَا بَعْدَ ٱلْحَقِّ إِلَّا ٱلضَّلَـٰلُ ۖ فَأَنَّىٰ تُصْرَفُونَ ﴿٣٢﴾\\
\textamh{33.\  } & كَذَٟلِكَ حَقَّتْ كَلِمَتُ رَبِّكَ عَلَى ٱلَّذِينَ فَسَقُوٓا۟ أَنَّهُمْ لَا يُؤْمِنُونَ ﴿٣٣﴾\\
\textamh{34.\  } & قُلْ هَلْ مِن شُرَكَآئِكُم مَّن يَبْدَؤُا۟ ٱلْخَلْقَ ثُمَّ يُعِيدُهُۥ ۚ قُلِ ٱللَّهُ يَبْدَؤُا۟ ٱلْخَلْقَ ثُمَّ يُعِيدُهُۥ ۖ فَأَنَّىٰ تُؤْفَكُونَ ﴿٣٤﴾\\
\textamh{35.\  } & قُلْ هَلْ مِن شُرَكَآئِكُم مَّن يَهْدِىٓ إِلَى ٱلْحَقِّ ۚ قُلِ ٱللَّهُ يَهْدِى لِلْحَقِّ ۗ أَفَمَن يَهْدِىٓ إِلَى ٱلْحَقِّ أَحَقُّ أَن يُتَّبَعَ أَمَّن لَّا يَهِدِّىٓ إِلَّآ أَن يُهْدَىٰ ۖ فَمَا لَكُمْ كَيْفَ تَحْكُمُونَ ﴿٣٥﴾\\
\textamh{36.\  } & وَمَا يَتَّبِعُ أَكْثَرُهُمْ إِلَّا ظَنًّا ۚ إِنَّ ٱلظَّنَّ لَا يُغْنِى مِنَ ٱلْحَقِّ شَيْـًٔا ۚ إِنَّ ٱللَّهَ عَلِيمٌۢ بِمَا يَفْعَلُونَ ﴿٣٦﴾\\
\textamh{37.\  } & وَمَا كَانَ هَـٰذَا ٱلْقُرْءَانُ أَن يُفْتَرَىٰ مِن دُونِ ٱللَّهِ وَلَـٰكِن تَصْدِيقَ ٱلَّذِى بَيْنَ يَدَيْهِ وَتَفْصِيلَ ٱلْكِتَـٰبِ لَا رَيْبَ فِيهِ مِن رَّبِّ ٱلْعَـٰلَمِينَ ﴿٣٧﴾\\
\textamh{38.\  } & أَمْ يَقُولُونَ ٱفْتَرَىٰهُ ۖ قُلْ فَأْتُوا۟ بِسُورَةٍۢ مِّثْلِهِۦ وَٱدْعُوا۟ مَنِ ٱسْتَطَعْتُم مِّن دُونِ ٱللَّهِ إِن كُنتُمْ صَـٰدِقِينَ ﴿٣٨﴾\\
\textamh{39.\  } & بَلْ كَذَّبُوا۟ بِمَا لَمْ يُحِيطُوا۟ بِعِلْمِهِۦ وَلَمَّا يَأْتِهِمْ تَأْوِيلُهُۥ ۚ كَذَٟلِكَ كَذَّبَ ٱلَّذِينَ مِن قَبْلِهِمْ ۖ فَٱنظُرْ كَيْفَ كَانَ عَـٰقِبَةُ ٱلظَّـٰلِمِينَ ﴿٣٩﴾\\
\textamh{40.\  } & وَمِنْهُم مَّن يُؤْمِنُ بِهِۦ وَمِنْهُم مَّن لَّا يُؤْمِنُ بِهِۦ ۚ وَرَبُّكَ أَعْلَمُ بِٱلْمُفْسِدِينَ ﴿٤٠﴾\\
\textamh{41.\  } & وَإِن كَذَّبُوكَ فَقُل لِّى عَمَلِى وَلَكُمْ عَمَلُكُمْ ۖ أَنتُم بَرِيٓـُٔونَ مِمَّآ أَعْمَلُ وَأَنَا۠ بَرِىٓءٌۭ مِّمَّا تَعْمَلُونَ ﴿٤١﴾\\
\textamh{42.\  } & وَمِنْهُم مَّن يَسْتَمِعُونَ إِلَيْكَ ۚ أَفَأَنتَ تُسْمِعُ ٱلصُّمَّ وَلَوْ كَانُوا۟ لَا يَعْقِلُونَ ﴿٤٢﴾\\
\textamh{43.\  } & وَمِنْهُم مَّن يَنظُرُ إِلَيْكَ ۚ أَفَأَنتَ تَهْدِى ٱلْعُمْىَ وَلَوْ كَانُوا۟ لَا يُبْصِرُونَ ﴿٤٣﴾\\
\textamh{44.\  } & إِنَّ ٱللَّهَ لَا يَظْلِمُ ٱلنَّاسَ شَيْـًۭٔا وَلَـٰكِنَّ ٱلنَّاسَ أَنفُسَهُمْ يَظْلِمُونَ ﴿٤٤﴾\\
\textamh{45.\  } & وَيَوْمَ يَحْشُرُهُمْ كَأَن لَّمْ يَلْبَثُوٓا۟ إِلَّا سَاعَةًۭ مِّنَ ٱلنَّهَارِ يَتَعَارَفُونَ بَيْنَهُمْ ۚ قَدْ خَسِرَ ٱلَّذِينَ كَذَّبُوا۟ بِلِقَآءِ ٱللَّهِ وَمَا كَانُوا۟ مُهْتَدِينَ ﴿٤٥﴾\\
\textamh{46.\  } & وَإِمَّا نُرِيَنَّكَ بَعْضَ ٱلَّذِى نَعِدُهُمْ أَوْ نَتَوَفَّيَنَّكَ فَإِلَيْنَا مَرْجِعُهُمْ ثُمَّ ٱللَّهُ شَهِيدٌ عَلَىٰ مَا يَفْعَلُونَ ﴿٤٦﴾\\
\textamh{47.\  } & وَلِكُلِّ أُمَّةٍۢ رَّسُولٌۭ ۖ فَإِذَا جَآءَ رَسُولُهُمْ قُضِىَ بَيْنَهُم بِٱلْقِسْطِ وَهُمْ لَا يُظْلَمُونَ ﴿٤٧﴾\\
\textamh{48.\  } & وَيَقُولُونَ مَتَىٰ هَـٰذَا ٱلْوَعْدُ إِن كُنتُمْ صَـٰدِقِينَ ﴿٤٨﴾\\
\textamh{49.\  } & قُل لَّآ أَمْلِكُ لِنَفْسِى ضَرًّۭا وَلَا نَفْعًا إِلَّا مَا شَآءَ ٱللَّهُ ۗ لِكُلِّ أُمَّةٍ أَجَلٌ ۚ إِذَا جَآءَ أَجَلُهُمْ فَلَا يَسْتَـْٔخِرُونَ سَاعَةًۭ ۖ وَلَا يَسْتَقْدِمُونَ ﴿٤٩﴾\\
\textamh{50.\  } & قُلْ أَرَءَيْتُمْ إِنْ أَتَىٰكُمْ عَذَابُهُۥ بَيَـٰتًا أَوْ نَهَارًۭا مَّاذَا يَسْتَعْجِلُ مِنْهُ ٱلْمُجْرِمُونَ ﴿٥٠﴾\\
\textamh{51.\  } & أَثُمَّ إِذَا مَا وَقَعَ ءَامَنتُم بِهِۦٓ ۚ ءَآلْـَٰٔنَ وَقَدْ كُنتُم بِهِۦ تَسْتَعْجِلُونَ ﴿٥١﴾\\
\textamh{52.\  } & ثُمَّ قِيلَ لِلَّذِينَ ظَلَمُوا۟ ذُوقُوا۟ عَذَابَ ٱلْخُلْدِ هَلْ تُجْزَوْنَ إِلَّا بِمَا كُنتُمْ تَكْسِبُونَ ﴿٥٢﴾\\
\textamh{53.\  } & ۞ وَيَسْتَنۢبِـُٔونَكَ أَحَقٌّ هُوَ ۖ قُلْ إِى وَرَبِّىٓ إِنَّهُۥ لَحَقٌّۭ ۖ وَمَآ أَنتُم بِمُعْجِزِينَ ﴿٥٣﴾\\
\textamh{54.\  } & وَلَوْ أَنَّ لِكُلِّ نَفْسٍۢ ظَلَمَتْ مَا فِى ٱلْأَرْضِ لَٱفْتَدَتْ بِهِۦ ۗ وَأَسَرُّوا۟ ٱلنَّدَامَةَ لَمَّا رَأَوُا۟ ٱلْعَذَابَ ۖ وَقُضِىَ بَيْنَهُم بِٱلْقِسْطِ ۚ وَهُمْ لَا يُظْلَمُونَ ﴿٥٤﴾\\
\textamh{55.\  } & أَلَآ إِنَّ لِلَّهِ مَا فِى ٱلسَّمَـٰوَٟتِ وَٱلْأَرْضِ ۗ أَلَآ إِنَّ وَعْدَ ٱللَّهِ حَقٌّۭ وَلَـٰكِنَّ أَكْثَرَهُمْ لَا يَعْلَمُونَ ﴿٥٥﴾\\
\textamh{56.\  } & هُوَ يُحْىِۦ وَيُمِيتُ وَإِلَيْهِ تُرْجَعُونَ ﴿٥٦﴾\\
\textamh{57.\  } & يَـٰٓأَيُّهَا ٱلنَّاسُ قَدْ جَآءَتْكُم مَّوْعِظَةٌۭ مِّن رَّبِّكُمْ وَشِفَآءٌۭ لِّمَا فِى ٱلصُّدُورِ وَهُدًۭى وَرَحْمَةٌۭ لِّلْمُؤْمِنِينَ ﴿٥٧﴾\\
\textamh{58.\  } & قُلْ بِفَضْلِ ٱللَّهِ وَبِرَحْمَتِهِۦ فَبِذَٟلِكَ فَلْيَفْرَحُوا۟ هُوَ خَيْرٌۭ مِّمَّا يَجْمَعُونَ ﴿٥٨﴾\\
\textamh{59.\  } & قُلْ أَرَءَيْتُم مَّآ أَنزَلَ ٱللَّهُ لَكُم مِّن رِّزْقٍۢ فَجَعَلْتُم مِّنْهُ حَرَامًۭا وَحَلَـٰلًۭا قُلْ ءَآللَّهُ أَذِنَ لَكُمْ ۖ أَمْ عَلَى ٱللَّهِ تَفْتَرُونَ ﴿٥٩﴾\\
\textamh{60.\  } & وَمَا ظَنُّ ٱلَّذِينَ يَفْتَرُونَ عَلَى ٱللَّهِ ٱلْكَذِبَ يَوْمَ ٱلْقِيَـٰمَةِ ۗ إِنَّ ٱللَّهَ لَذُو فَضْلٍ عَلَى ٱلنَّاسِ وَلَـٰكِنَّ أَكْثَرَهُمْ لَا يَشْكُرُونَ ﴿٦٠﴾\\
\textamh{61.\  } & وَمَا تَكُونُ فِى شَأْنٍۢ وَمَا تَتْلُوا۟ مِنْهُ مِن قُرْءَانٍۢ وَلَا تَعْمَلُونَ مِنْ عَمَلٍ إِلَّا كُنَّا عَلَيْكُمْ شُهُودًا إِذْ تُفِيضُونَ فِيهِ ۚ وَمَا يَعْزُبُ عَن رَّبِّكَ مِن مِّثْقَالِ ذَرَّةٍۢ فِى ٱلْأَرْضِ وَلَا فِى ٱلسَّمَآءِ وَلَآ أَصْغَرَ مِن ذَٟلِكَ وَلَآ أَكْبَرَ إِلَّا فِى كِتَـٰبٍۢ مُّبِينٍ ﴿٦١﴾\\
\textamh{62.\  } & أَلَآ إِنَّ أَوْلِيَآءَ ٱللَّهِ لَا خَوْفٌ عَلَيْهِمْ وَلَا هُمْ يَحْزَنُونَ ﴿٦٢﴾\\
\textamh{63.\  } & ٱلَّذِينَ ءَامَنُوا۟ وَكَانُوا۟ يَتَّقُونَ ﴿٦٣﴾\\
\textamh{64.\  } & لَهُمُ ٱلْبُشْرَىٰ فِى ٱلْحَيَوٰةِ ٱلدُّنْيَا وَفِى ٱلْءَاخِرَةِ ۚ لَا تَبْدِيلَ لِكَلِمَـٰتِ ٱللَّهِ ۚ ذَٟلِكَ هُوَ ٱلْفَوْزُ ٱلْعَظِيمُ ﴿٦٤﴾\\
\textamh{65.\  } & وَلَا يَحْزُنكَ قَوْلُهُمْ ۘ إِنَّ ٱلْعِزَّةَ لِلَّهِ جَمِيعًا ۚ هُوَ ٱلسَّمِيعُ ٱلْعَلِيمُ ﴿٦٥﴾\\
\textamh{66.\  } & أَلَآ إِنَّ لِلَّهِ مَن فِى ٱلسَّمَـٰوَٟتِ وَمَن فِى ٱلْأَرْضِ ۗ وَمَا يَتَّبِعُ ٱلَّذِينَ يَدْعُونَ مِن دُونِ ٱللَّهِ شُرَكَآءَ ۚ إِن يَتَّبِعُونَ إِلَّا ٱلظَّنَّ وَإِنْ هُمْ إِلَّا يَخْرُصُونَ ﴿٦٦﴾\\
\textamh{67.\  } & هُوَ ٱلَّذِى جَعَلَ لَكُمُ ٱلَّيْلَ لِتَسْكُنُوا۟ فِيهِ وَٱلنَّهَارَ مُبْصِرًا ۚ إِنَّ فِى ذَٟلِكَ لَءَايَـٰتٍۢ لِّقَوْمٍۢ يَسْمَعُونَ ﴿٦٧﴾\\
\textamh{68.\  } & قَالُوا۟ ٱتَّخَذَ ٱللَّهُ وَلَدًۭا ۗ سُبْحَـٰنَهُۥ ۖ هُوَ ٱلْغَنِىُّ ۖ لَهُۥ مَا فِى ٱلسَّمَـٰوَٟتِ وَمَا فِى ٱلْأَرْضِ ۚ إِنْ عِندَكُم مِّن سُلْطَٰنٍۭ بِهَـٰذَآ ۚ أَتَقُولُونَ عَلَى ٱللَّهِ مَا لَا تَعْلَمُونَ ﴿٦٨﴾\\
\textamh{69.\  } & قُلْ إِنَّ ٱلَّذِينَ يَفْتَرُونَ عَلَى ٱللَّهِ ٱلْكَذِبَ لَا يُفْلِحُونَ ﴿٦٩﴾\\
\textamh{70.\  } & مَتَـٰعٌۭ فِى ٱلدُّنْيَا ثُمَّ إِلَيْنَا مَرْجِعُهُمْ ثُمَّ نُذِيقُهُمُ ٱلْعَذَابَ ٱلشَّدِيدَ بِمَا كَانُوا۟ يَكْفُرُونَ ﴿٧٠﴾\\
\textamh{71.\  } & ۞ وَٱتْلُ عَلَيْهِمْ نَبَأَ نُوحٍ إِذْ قَالَ لِقَوْمِهِۦ يَـٰقَوْمِ إِن كَانَ كَبُرَ عَلَيْكُم مَّقَامِى وَتَذْكِيرِى بِـَٔايَـٰتِ ٱللَّهِ فَعَلَى ٱللَّهِ تَوَكَّلْتُ فَأَجْمِعُوٓا۟ أَمْرَكُمْ وَشُرَكَآءَكُمْ ثُمَّ لَا يَكُنْ أَمْرُكُمْ عَلَيْكُمْ غُمَّةًۭ ثُمَّ ٱقْضُوٓا۟ إِلَىَّ وَلَا تُنظِرُونِ ﴿٧١﴾\\
\textamh{72.\  } & فَإِن تَوَلَّيْتُمْ فَمَا سَأَلْتُكُم مِّنْ أَجْرٍ ۖ إِنْ أَجْرِىَ إِلَّا عَلَى ٱللَّهِ ۖ وَأُمِرْتُ أَنْ أَكُونَ مِنَ ٱلْمُسْلِمِينَ ﴿٧٢﴾\\
\textamh{73.\  } & فَكَذَّبُوهُ فَنَجَّيْنَـٰهُ وَمَن مَّعَهُۥ فِى ٱلْفُلْكِ وَجَعَلْنَـٰهُمْ خَلَـٰٓئِفَ وَأَغْرَقْنَا ٱلَّذِينَ كَذَّبُوا۟ بِـَٔايَـٰتِنَا ۖ فَٱنظُرْ كَيْفَ كَانَ عَـٰقِبَةُ ٱلْمُنذَرِينَ ﴿٧٣﴾\\
\textamh{74.\  } & ثُمَّ بَعَثْنَا مِنۢ بَعْدِهِۦ رُسُلًا إِلَىٰ قَوْمِهِمْ فَجَآءُوهُم بِٱلْبَيِّنَـٰتِ فَمَا كَانُوا۟ لِيُؤْمِنُوا۟ بِمَا كَذَّبُوا۟ بِهِۦ مِن قَبْلُ ۚ كَذَٟلِكَ نَطْبَعُ عَلَىٰ قُلُوبِ ٱلْمُعْتَدِينَ ﴿٧٤﴾\\
\textamh{75.\  } & ثُمَّ بَعَثْنَا مِنۢ بَعْدِهِم مُّوسَىٰ وَهَـٰرُونَ إِلَىٰ فِرْعَوْنَ وَمَلَإِي۟هِۦ بِـَٔايَـٰتِنَا فَٱسْتَكْبَرُوا۟ وَكَانُوا۟ قَوْمًۭا مُّجْرِمِينَ ﴿٧٥﴾\\
\textamh{76.\  } & فَلَمَّا جَآءَهُمُ ٱلْحَقُّ مِنْ عِندِنَا قَالُوٓا۟ إِنَّ هَـٰذَا لَسِحْرٌۭ مُّبِينٌۭ ﴿٧٦﴾\\
\textamh{77.\  } & قَالَ مُوسَىٰٓ أَتَقُولُونَ لِلْحَقِّ لَمَّا جَآءَكُمْ ۖ أَسِحْرٌ هَـٰذَا وَلَا يُفْلِحُ ٱلسَّٰحِرُونَ ﴿٧٧﴾\\
\textamh{78.\  } & قَالُوٓا۟ أَجِئْتَنَا لِتَلْفِتَنَا عَمَّا وَجَدْنَا عَلَيْهِ ءَابَآءَنَا وَتَكُونَ لَكُمَا ٱلْكِبْرِيَآءُ فِى ٱلْأَرْضِ وَمَا نَحْنُ لَكُمَا بِمُؤْمِنِينَ ﴿٧٨﴾\\
\textamh{79.\  } & وَقَالَ فِرْعَوْنُ ٱئْتُونِى بِكُلِّ سَـٰحِرٍ عَلِيمٍۢ ﴿٧٩﴾\\
\textamh{80.\  } & فَلَمَّا جَآءَ ٱلسَّحَرَةُ قَالَ لَهُم مُّوسَىٰٓ أَلْقُوا۟ مَآ أَنتُم مُّلْقُونَ ﴿٨٠﴾\\
\textamh{81.\  } & فَلَمَّآ أَلْقَوْا۟ قَالَ مُوسَىٰ مَا جِئْتُم بِهِ ٱلسِّحْرُ ۖ إِنَّ ٱللَّهَ سَيُبْطِلُهُۥٓ ۖ إِنَّ ٱللَّهَ لَا يُصْلِحُ عَمَلَ ٱلْمُفْسِدِينَ ﴿٨١﴾\\
\textamh{82.\  } & وَيُحِقُّ ٱللَّهُ ٱلْحَقَّ بِكَلِمَـٰتِهِۦ وَلَوْ كَرِهَ ٱلْمُجْرِمُونَ ﴿٨٢﴾\\
\textamh{83.\  } & فَمَآ ءَامَنَ لِمُوسَىٰٓ إِلَّا ذُرِّيَّةٌۭ مِّن قَوْمِهِۦ عَلَىٰ خَوْفٍۢ مِّن فِرْعَوْنَ وَمَلَإِي۟هِمْ أَن يَفْتِنَهُمْ ۚ وَإِنَّ فِرْعَوْنَ لَعَالٍۢ فِى ٱلْأَرْضِ وَإِنَّهُۥ لَمِنَ ٱلْمُسْرِفِينَ ﴿٨٣﴾\\
\textamh{84.\  } & وَقَالَ مُوسَىٰ يَـٰقَوْمِ إِن كُنتُمْ ءَامَنتُم بِٱللَّهِ فَعَلَيْهِ تَوَكَّلُوٓا۟ إِن كُنتُم مُّسْلِمِينَ ﴿٨٤﴾\\
\textamh{85.\  } & فَقَالُوا۟ عَلَى ٱللَّهِ تَوَكَّلْنَا رَبَّنَا لَا تَجْعَلْنَا فِتْنَةًۭ لِّلْقَوْمِ ٱلظَّـٰلِمِينَ ﴿٨٥﴾\\
\textamh{86.\  } & وَنَجِّنَا بِرَحْمَتِكَ مِنَ ٱلْقَوْمِ ٱلْكَـٰفِرِينَ ﴿٨٦﴾\\
\textamh{87.\  } & وَأَوْحَيْنَآ إِلَىٰ مُوسَىٰ وَأَخِيهِ أَن تَبَوَّءَا لِقَوْمِكُمَا بِمِصْرَ بُيُوتًۭا وَٱجْعَلُوا۟ بُيُوتَكُمْ قِبْلَةًۭ وَأَقِيمُوا۟ ٱلصَّلَوٰةَ ۗ وَبَشِّرِ ٱلْمُؤْمِنِينَ ﴿٨٧﴾\\
\textamh{88.\  } & وَقَالَ مُوسَىٰ رَبَّنَآ إِنَّكَ ءَاتَيْتَ فِرْعَوْنَ وَمَلَأَهُۥ زِينَةًۭ وَأَمْوَٟلًۭا فِى ٱلْحَيَوٰةِ ٱلدُّنْيَا رَبَّنَا لِيُضِلُّوا۟ عَن سَبِيلِكَ ۖ رَبَّنَا ٱطْمِسْ عَلَىٰٓ أَمْوَٟلِهِمْ وَٱشْدُدْ عَلَىٰ قُلُوبِهِمْ فَلَا يُؤْمِنُوا۟ حَتَّىٰ يَرَوُا۟ ٱلْعَذَابَ ٱلْأَلِيمَ ﴿٨٨﴾\\
\textamh{89.\  } & قَالَ قَدْ أُجِيبَت دَّعْوَتُكُمَا فَٱسْتَقِيمَا وَلَا تَتَّبِعَآنِّ سَبِيلَ ٱلَّذِينَ لَا يَعْلَمُونَ ﴿٨٩﴾\\
\textamh{90.\  } & ۞ وَجَٰوَزْنَا بِبَنِىٓ إِسْرَٰٓءِيلَ ٱلْبَحْرَ فَأَتْبَعَهُمْ فِرْعَوْنُ وَجُنُودُهُۥ بَغْيًۭا وَعَدْوًا ۖ حَتَّىٰٓ إِذَآ أَدْرَكَهُ ٱلْغَرَقُ قَالَ ءَامَنتُ أَنَّهُۥ لَآ إِلَـٰهَ إِلَّا ٱلَّذِىٓ ءَامَنَتْ بِهِۦ بَنُوٓا۟ إِسْرَٰٓءِيلَ وَأَنَا۠ مِنَ ٱلْمُسْلِمِينَ ﴿٩٠﴾\\
\textamh{91.\  } & ءَآلْـَٰٔنَ وَقَدْ عَصَيْتَ قَبْلُ وَكُنتَ مِنَ ٱلْمُفْسِدِينَ ﴿٩١﴾\\
\textamh{92.\  } & فَٱلْيَوْمَ نُنَجِّيكَ بِبَدَنِكَ لِتَكُونَ لِمَنْ خَلْفَكَ ءَايَةًۭ ۚ وَإِنَّ كَثِيرًۭا مِّنَ ٱلنَّاسِ عَنْ ءَايَـٰتِنَا لَغَٰفِلُونَ ﴿٩٢﴾\\
\textamh{93.\  } & وَلَقَدْ بَوَّأْنَا بَنِىٓ إِسْرَٰٓءِيلَ مُبَوَّأَ صِدْقٍۢ وَرَزَقْنَـٰهُم مِّنَ ٱلطَّيِّبَٰتِ فَمَا ٱخْتَلَفُوا۟ حَتَّىٰ جَآءَهُمُ ٱلْعِلْمُ ۚ إِنَّ رَبَّكَ يَقْضِى بَيْنَهُمْ يَوْمَ ٱلْقِيَـٰمَةِ فِيمَا كَانُوا۟ فِيهِ يَخْتَلِفُونَ ﴿٩٣﴾\\
\textamh{94.\  } & فَإِن كُنتَ فِى شَكٍّۢ مِّمَّآ أَنزَلْنَآ إِلَيْكَ فَسْـَٔلِ ٱلَّذِينَ يَقْرَءُونَ ٱلْكِتَـٰبَ مِن قَبْلِكَ ۚ لَقَدْ جَآءَكَ ٱلْحَقُّ مِن رَّبِّكَ فَلَا تَكُونَنَّ مِنَ ٱلْمُمْتَرِينَ ﴿٩٤﴾\\
\textamh{95.\  } & وَلَا تَكُونَنَّ مِنَ ٱلَّذِينَ كَذَّبُوا۟ بِـَٔايَـٰتِ ٱللَّهِ فَتَكُونَ مِنَ ٱلْخَـٰسِرِينَ ﴿٩٥﴾\\
\textamh{96.\  } & إِنَّ ٱلَّذِينَ حَقَّتْ عَلَيْهِمْ كَلِمَتُ رَبِّكَ لَا يُؤْمِنُونَ ﴿٩٦﴾\\
\textamh{97.\  } & وَلَوْ جَآءَتْهُمْ كُلُّ ءَايَةٍ حَتَّىٰ يَرَوُا۟ ٱلْعَذَابَ ٱلْأَلِيمَ ﴿٩٧﴾\\
\textamh{98.\  } & فَلَوْلَا كَانَتْ قَرْيَةٌ ءَامَنَتْ فَنَفَعَهَآ إِيمَـٰنُهَآ إِلَّا قَوْمَ يُونُسَ لَمَّآ ءَامَنُوا۟ كَشَفْنَا عَنْهُمْ عَذَابَ ٱلْخِزْىِ فِى ٱلْحَيَوٰةِ ٱلدُّنْيَا وَمَتَّعْنَـٰهُمْ إِلَىٰ حِينٍۢ ﴿٩٨﴾\\
\textamh{99.\  } & وَلَوْ شَآءَ رَبُّكَ لَءَامَنَ مَن فِى ٱلْأَرْضِ كُلُّهُمْ جَمِيعًا ۚ أَفَأَنتَ تُكْرِهُ ٱلنَّاسَ حَتَّىٰ يَكُونُوا۟ مُؤْمِنِينَ ﴿٩٩﴾\\
\textamh{100.\  } & وَمَا كَانَ لِنَفْسٍ أَن تُؤْمِنَ إِلَّا بِإِذْنِ ٱللَّهِ ۚ وَيَجْعَلُ ٱلرِّجْسَ عَلَى ٱلَّذِينَ لَا يَعْقِلُونَ ﴿١٠٠﴾\\
\textamh{101.\  } & قُلِ ٱنظُرُوا۟ مَاذَا فِى ٱلسَّمَـٰوَٟتِ وَٱلْأَرْضِ ۚ وَمَا تُغْنِى ٱلْءَايَـٰتُ وَٱلنُّذُرُ عَن قَوْمٍۢ لَّا يُؤْمِنُونَ ﴿١٠١﴾\\
\textamh{102.\  } & فَهَلْ يَنتَظِرُونَ إِلَّا مِثْلَ أَيَّامِ ٱلَّذِينَ خَلَوْا۟ مِن قَبْلِهِمْ ۚ قُلْ فَٱنتَظِرُوٓا۟ إِنِّى مَعَكُم مِّنَ ٱلْمُنتَظِرِينَ ﴿١٠٢﴾\\
\textamh{103.\  } & ثُمَّ نُنَجِّى رُسُلَنَا وَٱلَّذِينَ ءَامَنُوا۟ ۚ كَذَٟلِكَ حَقًّا عَلَيْنَا نُنجِ ٱلْمُؤْمِنِينَ ﴿١٠٣﴾\\
\textamh{104.\  } & قُلْ يَـٰٓأَيُّهَا ٱلنَّاسُ إِن كُنتُمْ فِى شَكٍّۢ مِّن دِينِى فَلَآ أَعْبُدُ ٱلَّذِينَ تَعْبُدُونَ مِن دُونِ ٱللَّهِ وَلَـٰكِنْ أَعْبُدُ ٱللَّهَ ٱلَّذِى يَتَوَفَّىٰكُمْ ۖ وَأُمِرْتُ أَنْ أَكُونَ مِنَ ٱلْمُؤْمِنِينَ ﴿١٠٤﴾\\
\textamh{105.\  } & وَأَنْ أَقِمْ وَجْهَكَ لِلدِّينِ حَنِيفًۭا وَلَا تَكُونَنَّ مِنَ ٱلْمُشْرِكِينَ ﴿١٠٥﴾\\
\textamh{106.\  } & وَلَا تَدْعُ مِن دُونِ ٱللَّهِ مَا لَا يَنفَعُكَ وَلَا يَضُرُّكَ ۖ فَإِن فَعَلْتَ فَإِنَّكَ إِذًۭا مِّنَ ٱلظَّـٰلِمِينَ ﴿١٠٦﴾\\
\textamh{107.\  } & وَإِن يَمْسَسْكَ ٱللَّهُ بِضُرٍّۢ فَلَا كَاشِفَ لَهُۥٓ إِلَّا هُوَ ۖ وَإِن يُرِدْكَ بِخَيْرٍۢ فَلَا رَآدَّ لِفَضْلِهِۦ ۚ يُصِيبُ بِهِۦ مَن يَشَآءُ مِنْ عِبَادِهِۦ ۚ وَهُوَ ٱلْغَفُورُ ٱلرَّحِيمُ ﴿١٠٧﴾\\
\textamh{108.\  } & قُلْ يَـٰٓأَيُّهَا ٱلنَّاسُ قَدْ جَآءَكُمُ ٱلْحَقُّ مِن رَّبِّكُمْ ۖ فَمَنِ ٱهْتَدَىٰ فَإِنَّمَا يَهْتَدِى لِنَفْسِهِۦ ۖ وَمَن ضَلَّ فَإِنَّمَا يَضِلُّ عَلَيْهَا ۖ وَمَآ أَنَا۠ عَلَيْكُم بِوَكِيلٍۢ ﴿١٠٨﴾\\
\textamh{109.\  } & وَٱتَّبِعْ مَا يُوحَىٰٓ إِلَيْكَ وَٱصْبِرْ حَتَّىٰ يَحْكُمَ ٱللَّهُ ۚ وَهُوَ خَيْرُ ٱلْحَـٰكِمِينَ ﴿١٠٩﴾\\
\end{longtable} \newpage

%% License: BSD style (Berkley) (i.e. Put the Copyright owner's name always)
%% Writer and Copyright (to): Bewketu(Bilal) Tadilo (2016-17)
\centering\section{\LR{\textamharic{ሱራቱ ሁድ -}  \RL{سوره  هود}}}
\begin{longtable}{%
  @{}
    p{.5\textwidth}
  @{~~~~~~~~~~~~}
    p{.5\textwidth}
    @{}
}
\nopagebreak
\textamh{ቢስሚላሂ አራህመኒ ራሂይም } &  بِسْمِ ٱللَّهِ ٱلرَّحْمَـٰنِ ٱلرَّحِيمِ\\
\textamh{1.\  } &  الٓر ۚ كِتَـٰبٌ أُحْكِمَتْ ءَايَـٰتُهُۥ ثُمَّ فُصِّلَتْ مِن لَّدُنْ حَكِيمٍ خَبِيرٍ ﴿١﴾\\
\textamh{2.\  } & أَلَّا تَعْبُدُوٓا۟ إِلَّا ٱللَّهَ ۚ إِنَّنِى لَكُم مِّنْهُ نَذِيرٌۭ وَبَشِيرٌۭ ﴿٢﴾\\
\textamh{3.\  } & وَأَنِ ٱسْتَغْفِرُوا۟ رَبَّكُمْ ثُمَّ تُوبُوٓا۟ إِلَيْهِ يُمَتِّعْكُم مَّتَـٰعًا حَسَنًا إِلَىٰٓ أَجَلٍۢ مُّسَمًّۭى وَيُؤْتِ كُلَّ ذِى فَضْلٍۢ فَضْلَهُۥ ۖ وَإِن تَوَلَّوْا۟ فَإِنِّىٓ أَخَافُ عَلَيْكُمْ عَذَابَ يَوْمٍۢ كَبِيرٍ ﴿٣﴾\\
\textamh{4.\  } & إِلَى ٱللَّهِ مَرْجِعُكُمْ ۖ وَهُوَ عَلَىٰ كُلِّ شَىْءٍۢ قَدِيرٌ ﴿٤﴾\\
\textamh{5.\  } & أَلَآ إِنَّهُمْ يَثْنُونَ صُدُورَهُمْ لِيَسْتَخْفُوا۟ مِنْهُ ۚ أَلَا حِينَ يَسْتَغْشُونَ ثِيَابَهُمْ يَعْلَمُ مَا يُسِرُّونَ وَمَا يُعْلِنُونَ ۚ إِنَّهُۥ عَلِيمٌۢ بِذَاتِ ٱلصُّدُورِ ﴿٥﴾\\
\textamh{6.\  } & ۞ وَمَا مِن دَآبَّةٍۢ فِى ٱلْأَرْضِ إِلَّا عَلَى ٱللَّهِ رِزْقُهَا وَيَعْلَمُ مُسْتَقَرَّهَا وَمُسْتَوْدَعَهَا ۚ كُلٌّۭ فِى كِتَـٰبٍۢ مُّبِينٍۢ ﴿٦﴾\\
\textamh{7.\  } & وَهُوَ ٱلَّذِى خَلَقَ ٱلسَّمَـٰوَٟتِ وَٱلْأَرْضَ فِى سِتَّةِ أَيَّامٍۢ وَكَانَ عَرْشُهُۥ عَلَى ٱلْمَآءِ لِيَبْلُوَكُمْ أَيُّكُمْ أَحْسَنُ عَمَلًۭا ۗ وَلَئِن قُلْتَ إِنَّكُم مَّبْعُوثُونَ مِنۢ بَعْدِ ٱلْمَوْتِ لَيَقُولَنَّ ٱلَّذِينَ كَفَرُوٓا۟ إِنْ هَـٰذَآ إِلَّا سِحْرٌۭ مُّبِينٌۭ ﴿٧﴾\\
\textamh{8.\  } & وَلَئِنْ أَخَّرْنَا عَنْهُمُ ٱلْعَذَابَ إِلَىٰٓ أُمَّةٍۢ مَّعْدُودَةٍۢ لَّيَقُولُنَّ مَا يَحْبِسُهُۥٓ ۗ أَلَا يَوْمَ يَأْتِيهِمْ لَيْسَ مَصْرُوفًا عَنْهُمْ وَحَاقَ بِهِم مَّا كَانُوا۟ بِهِۦ يَسْتَهْزِءُونَ ﴿٨﴾\\
\textamh{9.\  } & وَلَئِنْ أَذَقْنَا ٱلْإِنسَـٰنَ مِنَّا رَحْمَةًۭ ثُمَّ نَزَعْنَـٰهَا مِنْهُ إِنَّهُۥ لَيَـُٔوسٌۭ كَفُورٌۭ ﴿٩﴾\\
\textamh{10.\  } & وَلَئِنْ أَذَقْنَـٰهُ نَعْمَآءَ بَعْدَ ضَرَّآءَ مَسَّتْهُ لَيَقُولَنَّ ذَهَبَ ٱلسَّيِّـَٔاتُ عَنِّىٓ ۚ إِنَّهُۥ لَفَرِحٌۭ فَخُورٌ ﴿١٠﴾\\
\textamh{11.\  } & إِلَّا ٱلَّذِينَ صَبَرُوا۟ وَعَمِلُوا۟ ٱلصَّـٰلِحَـٰتِ أُو۟لَـٰٓئِكَ لَهُم مَّغْفِرَةٌۭ وَأَجْرٌۭ كَبِيرٌۭ ﴿١١﴾\\
\textamh{12.\  } & فَلَعَلَّكَ تَارِكٌۢ بَعْضَ مَا يُوحَىٰٓ إِلَيْكَ وَضَآئِقٌۢ بِهِۦ صَدْرُكَ أَن يَقُولُوا۟ لَوْلَآ أُنزِلَ عَلَيْهِ كَنزٌ أَوْ جَآءَ مَعَهُۥ مَلَكٌ ۚ إِنَّمَآ أَنتَ نَذِيرٌۭ ۚ وَٱللَّهُ عَلَىٰ كُلِّ شَىْءٍۢ وَكِيلٌ ﴿١٢﴾\\
\textamh{13.\  } & أَمْ يَقُولُونَ ٱفْتَرَىٰهُ ۖ قُلْ فَأْتُوا۟ بِعَشْرِ سُوَرٍۢ مِّثْلِهِۦ مُفْتَرَيَـٰتٍۢ وَٱدْعُوا۟ مَنِ ٱسْتَطَعْتُم مِّن دُونِ ٱللَّهِ إِن كُنتُمْ صَـٰدِقِينَ ﴿١٣﴾\\
\textamh{14.\  } & فَإِلَّمْ يَسْتَجِيبُوا۟ لَكُمْ فَٱعْلَمُوٓا۟ أَنَّمَآ أُنزِلَ بِعِلْمِ ٱللَّهِ وَأَن لَّآ إِلَـٰهَ إِلَّا هُوَ ۖ فَهَلْ أَنتُم مُّسْلِمُونَ ﴿١٤﴾\\
\textamh{15.\  } & مَن كَانَ يُرِيدُ ٱلْحَيَوٰةَ ٱلدُّنْيَا وَزِينَتَهَا نُوَفِّ إِلَيْهِمْ أَعْمَـٰلَهُمْ فِيهَا وَهُمْ فِيهَا لَا يُبْخَسُونَ ﴿١٥﴾\\
\textamh{16.\  } & أُو۟لَـٰٓئِكَ ٱلَّذِينَ لَيْسَ لَهُمْ فِى ٱلْءَاخِرَةِ إِلَّا ٱلنَّارُ ۖ وَحَبِطَ مَا صَنَعُوا۟ فِيهَا وَبَٰطِلٌۭ مَّا كَانُوا۟ يَعْمَلُونَ ﴿١٦﴾\\
\textamh{17.\  } & أَفَمَن كَانَ عَلَىٰ بَيِّنَةٍۢ مِّن رَّبِّهِۦ وَيَتْلُوهُ شَاهِدٌۭ مِّنْهُ وَمِن قَبْلِهِۦ كِتَـٰبُ مُوسَىٰٓ إِمَامًۭا وَرَحْمَةً ۚ أُو۟لَـٰٓئِكَ يُؤْمِنُونَ بِهِۦ ۚ وَمَن يَكْفُرْ بِهِۦ مِنَ ٱلْأَحْزَابِ فَٱلنَّارُ مَوْعِدُهُۥ ۚ فَلَا تَكُ فِى مِرْيَةٍۢ مِّنْهُ ۚ إِنَّهُ ٱلْحَقُّ مِن رَّبِّكَ وَلَـٰكِنَّ أَكْثَرَ ٱلنَّاسِ لَا يُؤْمِنُونَ ﴿١٧﴾\\
\textamh{18.\  } & وَمَنْ أَظْلَمُ مِمَّنِ ٱفْتَرَىٰ عَلَى ٱللَّهِ كَذِبًا ۚ أُو۟لَـٰٓئِكَ يُعْرَضُونَ عَلَىٰ رَبِّهِمْ وَيَقُولُ ٱلْأَشْهَـٰدُ هَـٰٓؤُلَآءِ ٱلَّذِينَ كَذَبُوا۟ عَلَىٰ رَبِّهِمْ ۚ أَلَا لَعْنَةُ ٱللَّهِ عَلَى ٱلظَّـٰلِمِينَ ﴿١٨﴾\\
\textamh{19.\  } & ٱلَّذِينَ يَصُدُّونَ عَن سَبِيلِ ٱللَّهِ وَيَبْغُونَهَا عِوَجًۭا وَهُم بِٱلْءَاخِرَةِ هُمْ كَـٰفِرُونَ ﴿١٩﴾\\
\textamh{20.\  } & أُو۟لَـٰٓئِكَ لَمْ يَكُونُوا۟ مُعْجِزِينَ فِى ٱلْأَرْضِ وَمَا كَانَ لَهُم مِّن دُونِ ٱللَّهِ مِنْ أَوْلِيَآءَ ۘ يُضَٰعَفُ لَهُمُ ٱلْعَذَابُ ۚ مَا كَانُوا۟ يَسْتَطِيعُونَ ٱلسَّمْعَ وَمَا كَانُوا۟ يُبْصِرُونَ ﴿٢٠﴾\\
\textamh{21.\  } & أُو۟لَـٰٓئِكَ ٱلَّذِينَ خَسِرُوٓا۟ أَنفُسَهُمْ وَضَلَّ عَنْهُم مَّا كَانُوا۟ يَفْتَرُونَ ﴿٢١﴾\\
\textamh{22.\  } & لَا جَرَمَ أَنَّهُمْ فِى ٱلْءَاخِرَةِ هُمُ ٱلْأَخْسَرُونَ ﴿٢٢﴾\\
\textamh{23.\  } & إِنَّ ٱلَّذِينَ ءَامَنُوا۟ وَعَمِلُوا۟ ٱلصَّـٰلِحَـٰتِ وَأَخْبَتُوٓا۟ إِلَىٰ رَبِّهِمْ أُو۟لَـٰٓئِكَ أَصْحَـٰبُ ٱلْجَنَّةِ ۖ هُمْ فِيهَا خَـٰلِدُونَ ﴿٢٣﴾\\
\textamh{24.\  } & ۞ مَثَلُ ٱلْفَرِيقَيْنِ كَٱلْأَعْمَىٰ وَٱلْأَصَمِّ وَٱلْبَصِيرِ وَٱلسَّمِيعِ ۚ هَلْ يَسْتَوِيَانِ مَثَلًا ۚ أَفَلَا تَذَكَّرُونَ ﴿٢٤﴾\\
\textamh{25.\  } & وَلَقَدْ أَرْسَلْنَا نُوحًا إِلَىٰ قَوْمِهِۦٓ إِنِّى لَكُمْ نَذِيرٌۭ مُّبِينٌ ﴿٢٥﴾\\
\textamh{26.\  } & أَن لَّا تَعْبُدُوٓا۟ إِلَّا ٱللَّهَ ۖ إِنِّىٓ أَخَافُ عَلَيْكُمْ عَذَابَ يَوْمٍ أَلِيمٍۢ ﴿٢٦﴾\\
\textamh{27.\  } & فَقَالَ ٱلْمَلَأُ ٱلَّذِينَ كَفَرُوا۟ مِن قَوْمِهِۦ مَا نَرَىٰكَ إِلَّا بَشَرًۭا مِّثْلَنَا وَمَا نَرَىٰكَ ٱتَّبَعَكَ إِلَّا ٱلَّذِينَ هُمْ أَرَاذِلُنَا بَادِىَ ٱلرَّأْىِ وَمَا نَرَىٰ لَكُمْ عَلَيْنَا مِن فَضْلٍۭ بَلْ نَظُنُّكُمْ كَـٰذِبِينَ ﴿٢٧﴾\\
\textamh{28.\  } & قَالَ يَـٰقَوْمِ أَرَءَيْتُمْ إِن كُنتُ عَلَىٰ بَيِّنَةٍۢ مِّن رَّبِّى وَءَاتَىٰنِى رَحْمَةًۭ مِّنْ عِندِهِۦ فَعُمِّيَتْ عَلَيْكُمْ أَنُلْزِمُكُمُوهَا وَأَنتُمْ لَهَا كَـٰرِهُونَ ﴿٢٨﴾\\
\textamh{29.\  } & وَيَـٰقَوْمِ لَآ أَسْـَٔلُكُمْ عَلَيْهِ مَالًا ۖ إِنْ أَجْرِىَ إِلَّا عَلَى ٱللَّهِ ۚ وَمَآ أَنَا۠ بِطَارِدِ ٱلَّذِينَ ءَامَنُوٓا۟ ۚ إِنَّهُم مُّلَـٰقُوا۟ رَبِّهِمْ وَلَـٰكِنِّىٓ أَرَىٰكُمْ قَوْمًۭا تَجْهَلُونَ ﴿٢٩﴾\\
\textamh{30.\  } & وَيَـٰقَوْمِ مَن يَنصُرُنِى مِنَ ٱللَّهِ إِن طَرَدتُّهُمْ ۚ أَفَلَا تَذَكَّرُونَ ﴿٣٠﴾\\
\textamh{31.\  } & وَلَآ أَقُولُ لَكُمْ عِندِى خَزَآئِنُ ٱللَّهِ وَلَآ أَعْلَمُ ٱلْغَيْبَ وَلَآ أَقُولُ إِنِّى مَلَكٌۭ وَلَآ أَقُولُ لِلَّذِينَ تَزْدَرِىٓ أَعْيُنُكُمْ لَن يُؤْتِيَهُمُ ٱللَّهُ خَيْرًا ۖ ٱللَّهُ أَعْلَمُ بِمَا فِىٓ أَنفُسِهِمْ ۖ إِنِّىٓ إِذًۭا لَّمِنَ ٱلظَّـٰلِمِينَ ﴿٣١﴾\\
\textamh{32.\  } & قَالُوا۟ يَـٰنُوحُ قَدْ جَٰدَلْتَنَا فَأَكْثَرْتَ جِدَٟلَنَا فَأْتِنَا بِمَا تَعِدُنَآ إِن كُنتَ مِنَ ٱلصَّـٰدِقِينَ ﴿٣٢﴾\\
\textamh{33.\  } & قَالَ إِنَّمَا يَأْتِيكُم بِهِ ٱللَّهُ إِن شَآءَ وَمَآ أَنتُم بِمُعْجِزِينَ ﴿٣٣﴾\\
\textamh{34.\  } & وَلَا يَنفَعُكُمْ نُصْحِىٓ إِنْ أَرَدتُّ أَنْ أَنصَحَ لَكُمْ إِن كَانَ ٱللَّهُ يُرِيدُ أَن يُغْوِيَكُمْ ۚ هُوَ رَبُّكُمْ وَإِلَيْهِ تُرْجَعُونَ ﴿٣٤﴾\\
\textamh{35.\  } & أَمْ يَقُولُونَ ٱفْتَرَىٰهُ ۖ قُلْ إِنِ ٱفْتَرَيْتُهُۥ فَعَلَىَّ إِجْرَامِى وَأَنَا۠ بَرِىٓءٌۭ مِّمَّا تُجْرِمُونَ ﴿٣٥﴾\\
\textamh{36.\  } & وَأُوحِىَ إِلَىٰ نُوحٍ أَنَّهُۥ لَن يُؤْمِنَ مِن قَوْمِكَ إِلَّا مَن قَدْ ءَامَنَ فَلَا تَبْتَئِسْ بِمَا كَانُوا۟ يَفْعَلُونَ ﴿٣٦﴾\\
\textamh{37.\  } & وَٱصْنَعِ ٱلْفُلْكَ بِأَعْيُنِنَا وَوَحْيِنَا وَلَا تُخَـٰطِبْنِى فِى ٱلَّذِينَ ظَلَمُوٓا۟ ۚ إِنَّهُم مُّغْرَقُونَ ﴿٣٧﴾\\
\textamh{38.\  } & وَيَصْنَعُ ٱلْفُلْكَ وَكُلَّمَا مَرَّ عَلَيْهِ مَلَأٌۭ مِّن قَوْمِهِۦ سَخِرُوا۟ مِنْهُ ۚ قَالَ إِن تَسْخَرُوا۟ مِنَّا فَإِنَّا نَسْخَرُ مِنكُمْ كَمَا تَسْخَرُونَ ﴿٣٨﴾\\
\textamh{39.\  } & فَسَوْفَ تَعْلَمُونَ مَن يَأْتِيهِ عَذَابٌۭ يُخْزِيهِ وَيَحِلُّ عَلَيْهِ عَذَابٌۭ مُّقِيمٌ ﴿٣٩﴾\\
\textamh{40.\  } & حَتَّىٰٓ إِذَا جَآءَ أَمْرُنَا وَفَارَ ٱلتَّنُّورُ قُلْنَا ٱحْمِلْ فِيهَا مِن كُلٍّۢ زَوْجَيْنِ ٱثْنَيْنِ وَأَهْلَكَ إِلَّا مَن سَبَقَ عَلَيْهِ ٱلْقَوْلُ وَمَنْ ءَامَنَ ۚ وَمَآ ءَامَنَ مَعَهُۥٓ إِلَّا قَلِيلٌۭ ﴿٤٠﴾\\
\textamh{41.\  } & ۞ وَقَالَ ٱرْكَبُوا۟ فِيهَا بِسْمِ ٱللَّهِ مَجْر۪ىٰهَا وَمُرْسَىٰهَآ ۚ إِنَّ رَبِّى لَغَفُورٌۭ رَّحِيمٌۭ ﴿٤١﴾\\
\textamh{42.\  } & وَهِىَ تَجْرِى بِهِمْ فِى مَوْجٍۢ كَٱلْجِبَالِ وَنَادَىٰ نُوحٌ ٱبْنَهُۥ وَكَانَ فِى مَعْزِلٍۢ يَـٰبُنَىَّ ٱرْكَب مَّعَنَا وَلَا تَكُن مَّعَ ٱلْكَـٰفِرِينَ ﴿٤٢﴾\\
\textamh{43.\  } & قَالَ سَـَٔاوِىٓ إِلَىٰ جَبَلٍۢ يَعْصِمُنِى مِنَ ٱلْمَآءِ ۚ قَالَ لَا عَاصِمَ ٱلْيَوْمَ مِنْ أَمْرِ ٱللَّهِ إِلَّا مَن رَّحِمَ ۚ وَحَالَ بَيْنَهُمَا ٱلْمَوْجُ فَكَانَ مِنَ ٱلْمُغْرَقِينَ ﴿٤٣﴾\\
\textamh{44.\  } & وَقِيلَ يَـٰٓأَرْضُ ٱبْلَعِى مَآءَكِ وَيَـٰسَمَآءُ أَقْلِعِى وَغِيضَ ٱلْمَآءُ وَقُضِىَ ٱلْأَمْرُ وَٱسْتَوَتْ عَلَى ٱلْجُودِىِّ ۖ وَقِيلَ بُعْدًۭا لِّلْقَوْمِ ٱلظَّـٰلِمِينَ ﴿٤٤﴾\\
\textamh{45.\  } & وَنَادَىٰ نُوحٌۭ رَّبَّهُۥ فَقَالَ رَبِّ إِنَّ ٱبْنِى مِنْ أَهْلِى وَإِنَّ وَعْدَكَ ٱلْحَقُّ وَأَنتَ أَحْكَمُ ٱلْحَـٰكِمِينَ ﴿٤٥﴾\\
\textamh{46.\  } & قَالَ يَـٰنُوحُ إِنَّهُۥ لَيْسَ مِنْ أَهْلِكَ ۖ إِنَّهُۥ عَمَلٌ غَيْرُ صَـٰلِحٍۢ ۖ فَلَا تَسْـَٔلْنِ مَا لَيْسَ لَكَ بِهِۦ عِلْمٌ ۖ إِنِّىٓ أَعِظُكَ أَن تَكُونَ مِنَ ٱلْجَٰهِلِينَ ﴿٤٦﴾\\
\textamh{47.\  } & قَالَ رَبِّ إِنِّىٓ أَعُوذُ بِكَ أَنْ أَسْـَٔلَكَ مَا لَيْسَ لِى بِهِۦ عِلْمٌۭ ۖ وَإِلَّا تَغْفِرْ لِى وَتَرْحَمْنِىٓ أَكُن مِّنَ ٱلْخَـٰسِرِينَ ﴿٤٧﴾\\
\textamh{48.\  } & قِيلَ يَـٰنُوحُ ٱهْبِطْ بِسَلَـٰمٍۢ مِّنَّا وَبَرَكَـٰتٍ عَلَيْكَ وَعَلَىٰٓ أُمَمٍۢ مِّمَّن مَّعَكَ ۚ وَأُمَمٌۭ سَنُمَتِّعُهُمْ ثُمَّ يَمَسُّهُم مِّنَّا عَذَابٌ أَلِيمٌۭ ﴿٤٨﴾\\
\textamh{49.\  } & تِلْكَ مِنْ أَنۢبَآءِ ٱلْغَيْبِ نُوحِيهَآ إِلَيْكَ ۖ مَا كُنتَ تَعْلَمُهَآ أَنتَ وَلَا قَوْمُكَ مِن قَبْلِ هَـٰذَا ۖ فَٱصْبِرْ ۖ إِنَّ ٱلْعَـٰقِبَةَ لِلْمُتَّقِينَ ﴿٤٩﴾\\
\textamh{50.\  } & وَإِلَىٰ عَادٍ أَخَاهُمْ هُودًۭا ۚ قَالَ يَـٰقَوْمِ ٱعْبُدُوا۟ ٱللَّهَ مَا لَكُم مِّنْ إِلَـٰهٍ غَيْرُهُۥٓ ۖ إِنْ أَنتُمْ إِلَّا مُفْتَرُونَ ﴿٥٠﴾\\
\textamh{51.\  } & يَـٰقَوْمِ لَآ أَسْـَٔلُكُمْ عَلَيْهِ أَجْرًا ۖ إِنْ أَجْرِىَ إِلَّا عَلَى ٱلَّذِى فَطَرَنِىٓ ۚ أَفَلَا تَعْقِلُونَ ﴿٥١﴾\\
\textamh{52.\  } & وَيَـٰقَوْمِ ٱسْتَغْفِرُوا۟ رَبَّكُمْ ثُمَّ تُوبُوٓا۟ إِلَيْهِ يُرْسِلِ ٱلسَّمَآءَ عَلَيْكُم مِّدْرَارًۭا وَيَزِدْكُمْ قُوَّةً إِلَىٰ قُوَّتِكُمْ وَلَا تَتَوَلَّوْا۟ مُجْرِمِينَ ﴿٥٢﴾\\
\textamh{53.\  } & قَالُوا۟ يَـٰهُودُ مَا جِئْتَنَا بِبَيِّنَةٍۢ وَمَا نَحْنُ بِتَارِكِىٓ ءَالِهَتِنَا عَن قَوْلِكَ وَمَا نَحْنُ لَكَ بِمُؤْمِنِينَ ﴿٥٣﴾\\
\textamh{54.\  } & إِن نَّقُولُ إِلَّا ٱعْتَرَىٰكَ بَعْضُ ءَالِهَتِنَا بِسُوٓءٍۢ ۗ قَالَ إِنِّىٓ أُشْهِدُ ٱللَّهَ وَٱشْهَدُوٓا۟ أَنِّى بَرِىٓءٌۭ مِّمَّا تُشْرِكُونَ ﴿٥٤﴾\\
\textamh{55.\  } & مِن دُونِهِۦ ۖ فَكِيدُونِى جَمِيعًۭا ثُمَّ لَا تُنظِرُونِ ﴿٥٥﴾\\
\textamh{56.\  } & إِنِّى تَوَكَّلْتُ عَلَى ٱللَّهِ رَبِّى وَرَبِّكُم ۚ مَّا مِن دَآبَّةٍ إِلَّا هُوَ ءَاخِذٌۢ بِنَاصِيَتِهَآ ۚ إِنَّ رَبِّى عَلَىٰ صِرَٰطٍۢ مُّسْتَقِيمٍۢ ﴿٥٦﴾\\
\textamh{57.\  } & فَإِن تَوَلَّوْا۟ فَقَدْ أَبْلَغْتُكُم مَّآ أُرْسِلْتُ بِهِۦٓ إِلَيْكُمْ ۚ وَيَسْتَخْلِفُ رَبِّى قَوْمًا غَيْرَكُمْ وَلَا تَضُرُّونَهُۥ شَيْـًٔا ۚ إِنَّ رَبِّى عَلَىٰ كُلِّ شَىْءٍ حَفِيظٌۭ ﴿٥٧﴾\\
\textamh{58.\  } & وَلَمَّا جَآءَ أَمْرُنَا نَجَّيْنَا هُودًۭا وَٱلَّذِينَ ءَامَنُوا۟ مَعَهُۥ بِرَحْمَةٍۢ مِّنَّا وَنَجَّيْنَـٰهُم مِّنْ عَذَابٍ غَلِيظٍۢ ﴿٥٨﴾\\
\textamh{59.\  } & وَتِلْكَ عَادٌۭ ۖ جَحَدُوا۟ بِـَٔايَـٰتِ رَبِّهِمْ وَعَصَوْا۟ رُسُلَهُۥ وَٱتَّبَعُوٓا۟ أَمْرَ كُلِّ جَبَّارٍ عَنِيدٍۢ ﴿٥٩﴾\\
\textamh{60.\  } & وَأُتْبِعُوا۟ فِى هَـٰذِهِ ٱلدُّنْيَا لَعْنَةًۭ وَيَوْمَ ٱلْقِيَـٰمَةِ ۗ أَلَآ إِنَّ عَادًۭا كَفَرُوا۟ رَبَّهُمْ ۗ أَلَا بُعْدًۭا لِّعَادٍۢ قَوْمِ هُودٍۢ ﴿٦٠﴾\\
\textamh{61.\  } & ۞ وَإِلَىٰ ثَمُودَ أَخَاهُمْ صَـٰلِحًۭا ۚ قَالَ يَـٰقَوْمِ ٱعْبُدُوا۟ ٱللَّهَ مَا لَكُم مِّنْ إِلَـٰهٍ غَيْرُهُۥ ۖ هُوَ أَنشَأَكُم مِّنَ ٱلْأَرْضِ وَٱسْتَعْمَرَكُمْ فِيهَا فَٱسْتَغْفِرُوهُ ثُمَّ تُوبُوٓا۟ إِلَيْهِ ۚ إِنَّ رَبِّى قَرِيبٌۭ مُّجِيبٌۭ ﴿٦١﴾\\
\textamh{62.\  } & قَالُوا۟ يَـٰصَـٰلِحُ قَدْ كُنتَ فِينَا مَرْجُوًّۭا قَبْلَ هَـٰذَآ ۖ أَتَنْهَىٰنَآ أَن نَّعْبُدَ مَا يَعْبُدُ ءَابَآؤُنَا وَإِنَّنَا لَفِى شَكٍّۢ مِّمَّا تَدْعُونَآ إِلَيْهِ مُرِيبٍۢ ﴿٦٢﴾\\
\textamh{63.\  } & قَالَ يَـٰقَوْمِ أَرَءَيْتُمْ إِن كُنتُ عَلَىٰ بَيِّنَةٍۢ مِّن رَّبِّى وَءَاتَىٰنِى مِنْهُ رَحْمَةًۭ فَمَن يَنصُرُنِى مِنَ ٱللَّهِ إِنْ عَصَيْتُهُۥ ۖ فَمَا تَزِيدُونَنِى غَيْرَ تَخْسِيرٍۢ ﴿٦٣﴾\\
\textamh{64.\  } & وَيَـٰقَوْمِ هَـٰذِهِۦ نَاقَةُ ٱللَّهِ لَكُمْ ءَايَةًۭ فَذَرُوهَا تَأْكُلْ فِىٓ أَرْضِ ٱللَّهِ وَلَا تَمَسُّوهَا بِسُوٓءٍۢ فَيَأْخُذَكُمْ عَذَابٌۭ قَرِيبٌۭ ﴿٦٤﴾\\
\textamh{65.\  } & فَعَقَرُوهَا فَقَالَ تَمَتَّعُوا۟ فِى دَارِكُمْ ثَلَـٰثَةَ أَيَّامٍۢ ۖ ذَٟلِكَ وَعْدٌ غَيْرُ مَكْذُوبٍۢ ﴿٦٥﴾\\
\textamh{66.\  } & فَلَمَّا جَآءَ أَمْرُنَا نَجَّيْنَا صَـٰلِحًۭا وَٱلَّذِينَ ءَامَنُوا۟ مَعَهُۥ بِرَحْمَةٍۢ مِّنَّا وَمِنْ خِزْىِ يَوْمِئِذٍ ۗ إِنَّ رَبَّكَ هُوَ ٱلْقَوِىُّ ٱلْعَزِيزُ ﴿٦٦﴾\\
\textamh{67.\  } & وَأَخَذَ ٱلَّذِينَ ظَلَمُوا۟ ٱلصَّيْحَةُ فَأَصْبَحُوا۟ فِى دِيَـٰرِهِمْ جَٰثِمِينَ ﴿٦٧﴾\\
\textamh{68.\  } & كَأَن لَّمْ يَغْنَوْا۟ فِيهَآ ۗ أَلَآ إِنَّ ثَمُودَا۟ كَفَرُوا۟ رَبَّهُمْ ۗ أَلَا بُعْدًۭا لِّثَمُودَ ﴿٦٨﴾\\
\textamh{69.\  } & وَلَقَدْ جَآءَتْ رُسُلُنَآ إِبْرَٰهِيمَ بِٱلْبُشْرَىٰ قَالُوا۟ سَلَـٰمًۭا ۖ قَالَ سَلَـٰمٌۭ ۖ فَمَا لَبِثَ أَن جَآءَ بِعِجْلٍ حَنِيذٍۢ ﴿٦٩﴾\\
\textamh{70.\  } & فَلَمَّا رَءَآ أَيْدِيَهُمْ لَا تَصِلُ إِلَيْهِ نَكِرَهُمْ وَأَوْجَسَ مِنْهُمْ خِيفَةًۭ ۚ قَالُوا۟ لَا تَخَفْ إِنَّآ أُرْسِلْنَآ إِلَىٰ قَوْمِ لُوطٍۢ ﴿٧٠﴾\\
\textamh{71.\  } & وَٱمْرَأَتُهُۥ قَآئِمَةٌۭ فَضَحِكَتْ فَبَشَّرْنَـٰهَا بِإِسْحَـٰقَ وَمِن وَرَآءِ إِسْحَـٰقَ يَعْقُوبَ ﴿٧١﴾\\
\textamh{72.\  } & قَالَتْ يَـٰوَيْلَتَىٰٓ ءَأَلِدُ وَأَنَا۠ عَجُوزٌۭ وَهَـٰذَا بَعْلِى شَيْخًا ۖ إِنَّ هَـٰذَا لَشَىْءٌ عَجِيبٌۭ ﴿٧٢﴾\\
\textamh{73.\  } & قَالُوٓا۟ أَتَعْجَبِينَ مِنْ أَمْرِ ٱللَّهِ ۖ رَحْمَتُ ٱللَّهِ وَبَرَكَـٰتُهُۥ عَلَيْكُمْ أَهْلَ ٱلْبَيْتِ ۚ إِنَّهُۥ حَمِيدٌۭ مَّجِيدٌۭ ﴿٧٣﴾\\
\textamh{74.\  } & فَلَمَّا ذَهَبَ عَنْ إِبْرَٰهِيمَ ٱلرَّوْعُ وَجَآءَتْهُ ٱلْبُشْرَىٰ يُجَٰدِلُنَا فِى قَوْمِ لُوطٍ ﴿٧٤﴾\\
\textamh{75.\  } & إِنَّ إِبْرَٰهِيمَ لَحَلِيمٌ أَوَّٰهٌۭ مُّنِيبٌۭ ﴿٧٥﴾\\
\textamh{76.\  } & يَـٰٓإِبْرَٰهِيمُ أَعْرِضْ عَنْ هَـٰذَآ ۖ إِنَّهُۥ قَدْ جَآءَ أَمْرُ رَبِّكَ ۖ وَإِنَّهُمْ ءَاتِيهِمْ عَذَابٌ غَيْرُ مَرْدُودٍۢ ﴿٧٦﴾\\
\textamh{77.\  } & وَلَمَّا جَآءَتْ رُسُلُنَا لُوطًۭا سِىٓءَ بِهِمْ وَضَاقَ بِهِمْ ذَرْعًۭا وَقَالَ هَـٰذَا يَوْمٌ عَصِيبٌۭ ﴿٧٧﴾\\
\textamh{78.\  } & وَجَآءَهُۥ قَوْمُهُۥ يُهْرَعُونَ إِلَيْهِ وَمِن قَبْلُ كَانُوا۟ يَعْمَلُونَ ٱلسَّيِّـَٔاتِ ۚ قَالَ يَـٰقَوْمِ هَـٰٓؤُلَآءِ بَنَاتِى هُنَّ أَطْهَرُ لَكُمْ ۖ فَٱتَّقُوا۟ ٱللَّهَ وَلَا تُخْزُونِ فِى ضَيْفِىٓ ۖ أَلَيْسَ مِنكُمْ رَجُلٌۭ رَّشِيدٌۭ ﴿٧٨﴾\\
\textamh{79.\  } & قَالُوا۟ لَقَدْ عَلِمْتَ مَا لَنَا فِى بَنَاتِكَ مِنْ حَقٍّۢ وَإِنَّكَ لَتَعْلَمُ مَا نُرِيدُ ﴿٧٩﴾\\
\textamh{80.\  } & قَالَ لَوْ أَنَّ لِى بِكُمْ قُوَّةً أَوْ ءَاوِىٓ إِلَىٰ رُكْنٍۢ شَدِيدٍۢ ﴿٨٠﴾\\
\textamh{81.\  } & قَالُوا۟ يَـٰلُوطُ إِنَّا رُسُلُ رَبِّكَ لَن يَصِلُوٓا۟ إِلَيْكَ ۖ فَأَسْرِ بِأَهْلِكَ بِقِطْعٍۢ مِّنَ ٱلَّيْلِ وَلَا يَلْتَفِتْ مِنكُمْ أَحَدٌ إِلَّا ٱمْرَأَتَكَ ۖ إِنَّهُۥ مُصِيبُهَا مَآ أَصَابَهُمْ ۚ إِنَّ مَوْعِدَهُمُ ٱلصُّبْحُ ۚ أَلَيْسَ ٱلصُّبْحُ بِقَرِيبٍۢ ﴿٨١﴾\\
\textamh{82.\  } & فَلَمَّا جَآءَ أَمْرُنَا جَعَلْنَا عَـٰلِيَهَا سَافِلَهَا وَأَمْطَرْنَا عَلَيْهَا حِجَارَةًۭ مِّن سِجِّيلٍۢ مَّنضُودٍۢ ﴿٨٢﴾\\
\textamh{83.\  } & مُّسَوَّمَةً عِندَ رَبِّكَ ۖ وَمَا هِىَ مِنَ ٱلظَّـٰلِمِينَ بِبَعِيدٍۢ ﴿٨٣﴾\\
\textamh{84.\  } & ۞ وَإِلَىٰ مَدْيَنَ أَخَاهُمْ شُعَيْبًۭا ۚ قَالَ يَـٰقَوْمِ ٱعْبُدُوا۟ ٱللَّهَ مَا لَكُم مِّنْ إِلَـٰهٍ غَيْرُهُۥ ۖ وَلَا تَنقُصُوا۟ ٱلْمِكْيَالَ وَٱلْمِيزَانَ ۚ إِنِّىٓ أَرَىٰكُم بِخَيْرٍۢ وَإِنِّىٓ أَخَافُ عَلَيْكُمْ عَذَابَ يَوْمٍۢ مُّحِيطٍۢ ﴿٨٤﴾\\
\textamh{85.\  } & وَيَـٰقَوْمِ أَوْفُوا۟ ٱلْمِكْيَالَ وَٱلْمِيزَانَ بِٱلْقِسْطِ ۖ وَلَا تَبْخَسُوا۟ ٱلنَّاسَ أَشْيَآءَهُمْ وَلَا تَعْثَوْا۟ فِى ٱلْأَرْضِ مُفْسِدِينَ ﴿٨٥﴾\\
\textamh{86.\  } & بَقِيَّتُ ٱللَّهِ خَيْرٌۭ لَّكُمْ إِن كُنتُم مُّؤْمِنِينَ ۚ وَمَآ أَنَا۠ عَلَيْكُم بِحَفِيظٍۢ ﴿٨٦﴾\\
\textamh{87.\  } & قَالُوا۟ يَـٰشُعَيْبُ أَصَلَوٰتُكَ تَأْمُرُكَ أَن نَّتْرُكَ مَا يَعْبُدُ ءَابَآؤُنَآ أَوْ أَن نَّفْعَلَ فِىٓ أَمْوَٟلِنَا مَا نَشَـٰٓؤُا۟ ۖ إِنَّكَ لَأَنتَ ٱلْحَلِيمُ ٱلرَّشِيدُ ﴿٨٧﴾\\
\textamh{88.\  } & قَالَ يَـٰقَوْمِ أَرَءَيْتُمْ إِن كُنتُ عَلَىٰ بَيِّنَةٍۢ مِّن رَّبِّى وَرَزَقَنِى مِنْهُ رِزْقًا حَسَنًۭا ۚ وَمَآ أُرِيدُ أَنْ أُخَالِفَكُمْ إِلَىٰ مَآ أَنْهَىٰكُمْ عَنْهُ ۚ إِنْ أُرِيدُ إِلَّا ٱلْإِصْلَـٰحَ مَا ٱسْتَطَعْتُ ۚ وَمَا تَوْفِيقِىٓ إِلَّا بِٱللَّهِ ۚ عَلَيْهِ تَوَكَّلْتُ وَإِلَيْهِ أُنِيبُ ﴿٨٨﴾\\
\textamh{89.\  } & وَيَـٰقَوْمِ لَا يَجْرِمَنَّكُمْ شِقَاقِىٓ أَن يُصِيبَكُم مِّثْلُ مَآ أَصَابَ قَوْمَ نُوحٍ أَوْ قَوْمَ هُودٍ أَوْ قَوْمَ صَـٰلِحٍۢ ۚ وَمَا قَوْمُ لُوطٍۢ مِّنكُم بِبَعِيدٍۢ ﴿٨٩﴾\\
\textamh{90.\  } & وَٱسْتَغْفِرُوا۟ رَبَّكُمْ ثُمَّ تُوبُوٓا۟ إِلَيْهِ ۚ إِنَّ رَبِّى رَحِيمٌۭ وَدُودٌۭ ﴿٩٠﴾\\
\textamh{91.\  } & قَالُوا۟ يَـٰشُعَيْبُ مَا نَفْقَهُ كَثِيرًۭا مِّمَّا تَقُولُ وَإِنَّا لَنَرَىٰكَ فِينَا ضَعِيفًۭا ۖ وَلَوْلَا رَهْطُكَ لَرَجَمْنَـٰكَ ۖ وَمَآ أَنتَ عَلَيْنَا بِعَزِيزٍۢ ﴿٩١﴾\\
\textamh{92.\  } & قَالَ يَـٰقَوْمِ أَرَهْطِىٓ أَعَزُّ عَلَيْكُم مِّنَ ٱللَّهِ وَٱتَّخَذْتُمُوهُ وَرَآءَكُمْ ظِهْرِيًّا ۖ إِنَّ رَبِّى بِمَا تَعْمَلُونَ مُحِيطٌۭ ﴿٩٢﴾\\
\textamh{93.\  } & وَيَـٰقَوْمِ ٱعْمَلُوا۟ عَلَىٰ مَكَانَتِكُمْ إِنِّى عَـٰمِلٌۭ ۖ سَوْفَ تَعْلَمُونَ مَن يَأْتِيهِ عَذَابٌۭ يُخْزِيهِ وَمَنْ هُوَ كَـٰذِبٌۭ ۖ وَٱرْتَقِبُوٓا۟ إِنِّى مَعَكُمْ رَقِيبٌۭ ﴿٩٣﴾\\
\textamh{94.\  } & وَلَمَّا جَآءَ أَمْرُنَا نَجَّيْنَا شُعَيْبًۭا وَٱلَّذِينَ ءَامَنُوا۟ مَعَهُۥ بِرَحْمَةٍۢ مِّنَّا وَأَخَذَتِ ٱلَّذِينَ ظَلَمُوا۟ ٱلصَّيْحَةُ فَأَصْبَحُوا۟ فِى دِيَـٰرِهِمْ جَٰثِمِينَ ﴿٩٤﴾\\
\textamh{95.\  } & كَأَن لَّمْ يَغْنَوْا۟ فِيهَآ ۗ أَلَا بُعْدًۭا لِّمَدْيَنَ كَمَا بَعِدَتْ ثَمُودُ ﴿٩٥﴾\\
\textamh{96.\  } & وَلَقَدْ أَرْسَلْنَا مُوسَىٰ بِـَٔايَـٰتِنَا وَسُلْطَٰنٍۢ مُّبِينٍ ﴿٩٦﴾\\
\textamh{97.\  } & إِلَىٰ فِرْعَوْنَ وَمَلَإِي۟هِۦ فَٱتَّبَعُوٓا۟ أَمْرَ فِرْعَوْنَ ۖ وَمَآ أَمْرُ فِرْعَوْنَ بِرَشِيدٍۢ ﴿٩٧﴾\\
\textamh{98.\  } & يَقْدُمُ قَوْمَهُۥ يَوْمَ ٱلْقِيَـٰمَةِ فَأَوْرَدَهُمُ ٱلنَّارَ ۖ وَبِئْسَ ٱلْوِرْدُ ٱلْمَوْرُودُ ﴿٩٨﴾\\
\textamh{99.\  } & وَأُتْبِعُوا۟ فِى هَـٰذِهِۦ لَعْنَةًۭ وَيَوْمَ ٱلْقِيَـٰمَةِ ۚ بِئْسَ ٱلرِّفْدُ ٱلْمَرْفُودُ ﴿٩٩﴾\\
\textamh{100.\  } & ذَٟلِكَ مِنْ أَنۢبَآءِ ٱلْقُرَىٰ نَقُصُّهُۥ عَلَيْكَ ۖ مِنْهَا قَآئِمٌۭ وَحَصِيدٌۭ ﴿١٠٠﴾\\
\textamh{101.\  } & وَمَا ظَلَمْنَـٰهُمْ وَلَـٰكِن ظَلَمُوٓا۟ أَنفُسَهُمْ ۖ فَمَآ أَغْنَتْ عَنْهُمْ ءَالِهَتُهُمُ ٱلَّتِى يَدْعُونَ مِن دُونِ ٱللَّهِ مِن شَىْءٍۢ لَّمَّا جَآءَ أَمْرُ رَبِّكَ ۖ وَمَا زَادُوهُمْ غَيْرَ تَتْبِيبٍۢ ﴿١٠١﴾\\
\textamh{102.\  } & وَكَذَٟلِكَ أَخْذُ رَبِّكَ إِذَآ أَخَذَ ٱلْقُرَىٰ وَهِىَ ظَـٰلِمَةٌ ۚ إِنَّ أَخْذَهُۥٓ أَلِيمٌۭ شَدِيدٌ ﴿١٠٢﴾\\
\textamh{103.\  } & إِنَّ فِى ذَٟلِكَ لَءَايَةًۭ لِّمَنْ خَافَ عَذَابَ ٱلْءَاخِرَةِ ۚ ذَٟلِكَ يَوْمٌۭ مَّجْمُوعٌۭ لَّهُ ٱلنَّاسُ وَذَٟلِكَ يَوْمٌۭ مَّشْهُودٌۭ ﴿١٠٣﴾\\
\textamh{104.\  } & وَمَا نُؤَخِّرُهُۥٓ إِلَّا لِأَجَلٍۢ مَّعْدُودٍۢ ﴿١٠٤﴾\\
\textamh{105.\  } & يَوْمَ يَأْتِ لَا تَكَلَّمُ نَفْسٌ إِلَّا بِإِذْنِهِۦ ۚ فَمِنْهُمْ شَقِىٌّۭ وَسَعِيدٌۭ ﴿١٠٥﴾\\
\textamh{106.\  } & فَأَمَّا ٱلَّذِينَ شَقُوا۟ فَفِى ٱلنَّارِ لَهُمْ فِيهَا زَفِيرٌۭ وَشَهِيقٌ ﴿١٠٦﴾\\
\textamh{107.\  } & خَـٰلِدِينَ فِيهَا مَا دَامَتِ ٱلسَّمَـٰوَٟتُ وَٱلْأَرْضُ إِلَّا مَا شَآءَ رَبُّكَ ۚ إِنَّ رَبَّكَ فَعَّالٌۭ لِّمَا يُرِيدُ ﴿١٠٧﴾\\
\textamh{108.\  } & ۞ وَأَمَّا ٱلَّذِينَ سُعِدُوا۟ فَفِى ٱلْجَنَّةِ خَـٰلِدِينَ فِيهَا مَا دَامَتِ ٱلسَّمَـٰوَٟتُ وَٱلْأَرْضُ إِلَّا مَا شَآءَ رَبُّكَ ۖ عَطَآءً غَيْرَ مَجْذُوذٍۢ ﴿١٠٨﴾\\
\textamh{109.\  } & فَلَا تَكُ فِى مِرْيَةٍۢ مِّمَّا يَعْبُدُ هَـٰٓؤُلَآءِ ۚ مَا يَعْبُدُونَ إِلَّا كَمَا يَعْبُدُ ءَابَآؤُهُم مِّن قَبْلُ ۚ وَإِنَّا لَمُوَفُّوهُمْ نَصِيبَهُمْ غَيْرَ مَنقُوصٍۢ ﴿١٠٩﴾\\
\textamh{110.\  } & وَلَقَدْ ءَاتَيْنَا مُوسَى ٱلْكِتَـٰبَ فَٱخْتُلِفَ فِيهِ ۚ وَلَوْلَا كَلِمَةٌۭ سَبَقَتْ مِن رَّبِّكَ لَقُضِىَ بَيْنَهُمْ ۚ وَإِنَّهُمْ لَفِى شَكٍّۢ مِّنْهُ مُرِيبٍۢ ﴿١١٠﴾\\
\textamh{111.\  } & وَإِنَّ كُلًّۭا لَّمَّا لَيُوَفِّيَنَّهُمْ رَبُّكَ أَعْمَـٰلَهُمْ ۚ إِنَّهُۥ بِمَا يَعْمَلُونَ خَبِيرٌۭ ﴿١١١﴾\\
\textamh{112.\  } & فَٱسْتَقِمْ كَمَآ أُمِرْتَ وَمَن تَابَ مَعَكَ وَلَا تَطْغَوْا۟ ۚ إِنَّهُۥ بِمَا تَعْمَلُونَ بَصِيرٌۭ ﴿١١٢﴾\\
\textamh{113.\  } & وَلَا تَرْكَنُوٓا۟ إِلَى ٱلَّذِينَ ظَلَمُوا۟ فَتَمَسَّكُمُ ٱلنَّارُ وَمَا لَكُم مِّن دُونِ ٱللَّهِ مِنْ أَوْلِيَآءَ ثُمَّ لَا تُنصَرُونَ ﴿١١٣﴾\\
\textamh{114.\  } & وَأَقِمِ ٱلصَّلَوٰةَ طَرَفَىِ ٱلنَّهَارِ وَزُلَفًۭا مِّنَ ٱلَّيْلِ ۚ إِنَّ ٱلْحَسَنَـٰتِ يُذْهِبْنَ ٱلسَّيِّـَٔاتِ ۚ ذَٟلِكَ ذِكْرَىٰ لِلذَّٰكِرِينَ ﴿١١٤﴾\\
\textamh{115.\  } & وَٱصْبِرْ فَإِنَّ ٱللَّهَ لَا يُضِيعُ أَجْرَ ٱلْمُحْسِنِينَ ﴿١١٥﴾\\
\textamh{116.\  } & فَلَوْلَا كَانَ مِنَ ٱلْقُرُونِ مِن قَبْلِكُمْ أُو۟لُوا۟ بَقِيَّةٍۢ يَنْهَوْنَ عَنِ ٱلْفَسَادِ فِى ٱلْأَرْضِ إِلَّا قَلِيلًۭا مِّمَّنْ أَنجَيْنَا مِنْهُمْ ۗ وَٱتَّبَعَ ٱلَّذِينَ ظَلَمُوا۟ مَآ أُتْرِفُوا۟ فِيهِ وَكَانُوا۟ مُجْرِمِينَ ﴿١١٦﴾\\
\textamh{117.\  } & وَمَا كَانَ رَبُّكَ لِيُهْلِكَ ٱلْقُرَىٰ بِظُلْمٍۢ وَأَهْلُهَا مُصْلِحُونَ ﴿١١٧﴾\\
\textamh{118.\  } & وَلَوْ شَآءَ رَبُّكَ لَجَعَلَ ٱلنَّاسَ أُمَّةًۭ وَٟحِدَةًۭ ۖ وَلَا يَزَالُونَ مُخْتَلِفِينَ ﴿١١٨﴾\\
\textamh{119.\  } & إِلَّا مَن رَّحِمَ رَبُّكَ ۚ وَلِذَٟلِكَ خَلَقَهُمْ ۗ وَتَمَّتْ كَلِمَةُ رَبِّكَ لَأَمْلَأَنَّ جَهَنَّمَ مِنَ ٱلْجِنَّةِ وَٱلنَّاسِ أَجْمَعِينَ ﴿١١٩﴾\\
\textamh{120.\  } & وَكُلًّۭا نَّقُصُّ عَلَيْكَ مِنْ أَنۢبَآءِ ٱلرُّسُلِ مَا نُثَبِّتُ بِهِۦ فُؤَادَكَ ۚ وَجَآءَكَ فِى هَـٰذِهِ ٱلْحَقُّ وَمَوْعِظَةٌۭ وَذِكْرَىٰ لِلْمُؤْمِنِينَ ﴿١٢٠﴾\\
\textamh{121.\  } & وَقُل لِّلَّذِينَ لَا يُؤْمِنُونَ ٱعْمَلُوا۟ عَلَىٰ مَكَانَتِكُمْ إِنَّا عَـٰمِلُونَ ﴿١٢١﴾\\
\textamh{122.\  } & وَٱنتَظِرُوٓا۟ إِنَّا مُنتَظِرُونَ ﴿١٢٢﴾\\
\textamh{123.\  } & وَلِلَّهِ غَيْبُ ٱلسَّمَـٰوَٟتِ وَٱلْأَرْضِ وَإِلَيْهِ يُرْجَعُ ٱلْأَمْرُ كُلُّهُۥ فَٱعْبُدْهُ وَتَوَكَّلْ عَلَيْهِ ۚ وَمَا رَبُّكَ بِغَٰفِلٍ عَمَّا تَعْمَلُونَ ﴿١٢٣﴾\\
\end{longtable}
\clearpage
%% License: BSD style (Berkley) (i.e. Put the Copyright owner's name always)
%% Writer and Copyright (to): Bewketu(Bilal) Tadilo (2016-17)
\centering\section{\LR{\textamharic{ሱራቱ ዩሱፍ -}  \RL{سوره  يوسف}}}
\begin{longtable}{%
  @{}
    p{.5\textwidth}
  @{~~~~~~~~~~~~~}
    p{.5\textwidth}
    @{}
}
\nopagebreak
\textamh{ቢስሚላሂ አራህመኒ ራሂይም } &  بِسْمِ ٱللَّهِ ٱلرَّحْمَـٰنِ ٱلرَّحِيمِ\\
\textamh{1.\  } &  الٓر ۚ تِلْكَ ءَايَـٰتُ ٱلْكِتَـٰبِ ٱلْمُبِينِ ﴿١﴾\\
\textamh{2.\  } & إِنَّآ أَنزَلْنَـٰهُ قُرْءَٰنًا عَرَبِيًّۭا لَّعَلَّكُمْ تَعْقِلُونَ ﴿٢﴾\\
\textamh{3.\  } & نَحْنُ نَقُصُّ عَلَيْكَ أَحْسَنَ ٱلْقَصَصِ بِمَآ أَوْحَيْنَآ إِلَيْكَ هَـٰذَا ٱلْقُرْءَانَ وَإِن كُنتَ مِن قَبْلِهِۦ لَمِنَ ٱلْغَٰفِلِينَ ﴿٣﴾\\
\textamh{4.\  } & إِذْ قَالَ يُوسُفُ لِأَبِيهِ يَـٰٓأَبَتِ إِنِّى رَأَيْتُ أَحَدَ عَشَرَ كَوْكَبًۭا وَٱلشَّمْسَ وَٱلْقَمَرَ رَأَيْتُهُمْ لِى سَـٰجِدِينَ ﴿٤﴾\\
\textamh{5.\  } & قَالَ يَـٰبُنَىَّ لَا تَقْصُصْ رُءْيَاكَ عَلَىٰٓ إِخْوَتِكَ فَيَكِيدُوا۟ لَكَ كَيْدًا ۖ إِنَّ ٱلشَّيْطَٰنَ لِلْإِنسَـٰنِ عَدُوٌّۭ مُّبِينٌۭ ﴿٥﴾\\
\textamh{6.\  } & وَكَذَٟلِكَ يَجْتَبِيكَ رَبُّكَ وَيُعَلِّمُكَ مِن تَأْوِيلِ ٱلْأَحَادِيثِ وَيُتِمُّ نِعْمَتَهُۥ عَلَيْكَ وَعَلَىٰٓ ءَالِ يَعْقُوبَ كَمَآ أَتَمَّهَا عَلَىٰٓ أَبَوَيْكَ مِن قَبْلُ إِبْرَٰهِيمَ وَإِسْحَـٰقَ ۚ إِنَّ رَبَّكَ عَلِيمٌ حَكِيمٌۭ ﴿٦﴾\\
\textamh{7.\  } & ۞ لَّقَدْ كَانَ فِى يُوسُفَ وَإِخْوَتِهِۦٓ ءَايَـٰتٌۭ لِّلسَّآئِلِينَ ﴿٧﴾\\
\textamh{8.\  } & إِذْ قَالُوا۟ لَيُوسُفُ وَأَخُوهُ أَحَبُّ إِلَىٰٓ أَبِينَا مِنَّا وَنَحْنُ عُصْبَةٌ إِنَّ أَبَانَا لَفِى ضَلَـٰلٍۢ مُّبِينٍ ﴿٨﴾\\
\textamh{9.\  } & ٱقْتُلُوا۟ يُوسُفَ أَوِ ٱطْرَحُوهُ أَرْضًۭا يَخْلُ لَكُمْ وَجْهُ أَبِيكُمْ وَتَكُونُوا۟ مِنۢ بَعْدِهِۦ قَوْمًۭا صَـٰلِحِينَ ﴿٩﴾\\
\textamh{10.\  } & قَالَ قَآئِلٌۭ مِّنْهُمْ لَا تَقْتُلُوا۟ يُوسُفَ وَأَلْقُوهُ فِى غَيَـٰبَتِ ٱلْجُبِّ يَلْتَقِطْهُ بَعْضُ ٱلسَّيَّارَةِ إِن كُنتُمْ فَـٰعِلِينَ ﴿١٠﴾\\
\textamh{11.\  } & قَالُوا۟ يَـٰٓأَبَانَا مَا لَكَ لَا تَأْمَ۫نَّا عَلَىٰ يُوسُفَ وَإِنَّا لَهُۥ لَنَـٰصِحُونَ ﴿١١﴾\\
\textamh{12.\  } & أَرْسِلْهُ مَعَنَا غَدًۭا يَرْتَعْ وَيَلْعَبْ وَإِنَّا لَهُۥ لَحَـٰفِظُونَ ﴿١٢﴾\\
\textamh{13.\  } & قَالَ إِنِّى لَيَحْزُنُنِىٓ أَن تَذْهَبُوا۟ بِهِۦ وَأَخَافُ أَن يَأْكُلَهُ ٱلذِّئْبُ وَأَنتُمْ عَنْهُ غَٰفِلُونَ ﴿١٣﴾\\
\textamh{14.\  } & قَالُوا۟ لَئِنْ أَكَلَهُ ٱلذِّئْبُ وَنَحْنُ عُصْبَةٌ إِنَّآ إِذًۭا لَّخَـٰسِرُونَ ﴿١٤﴾\\
\textamh{15.\  } & فَلَمَّا ذَهَبُوا۟ بِهِۦ وَأَجْمَعُوٓا۟ أَن يَجْعَلُوهُ فِى غَيَـٰبَتِ ٱلْجُبِّ ۚ وَأَوْحَيْنَآ إِلَيْهِ لَتُنَبِّئَنَّهُم بِأَمْرِهِمْ هَـٰذَا وَهُمْ لَا يَشْعُرُونَ ﴿١٥﴾\\
\textamh{16.\  } & وَجَآءُوٓ أَبَاهُمْ عِشَآءًۭ يَبْكُونَ ﴿١٦﴾\\
\textamh{17.\  } & قَالُوا۟ يَـٰٓأَبَانَآ إِنَّا ذَهَبْنَا نَسْتَبِقُ وَتَرَكْنَا يُوسُفَ عِندَ مَتَـٰعِنَا فَأَكَلَهُ ٱلذِّئْبُ ۖ وَمَآ أَنتَ بِمُؤْمِنٍۢ لَّنَا وَلَوْ كُنَّا صَـٰدِقِينَ ﴿١٧﴾\\
\textamh{18.\  } & وَجَآءُو عَلَىٰ قَمِيصِهِۦ بِدَمٍۢ كَذِبٍۢ ۚ قَالَ بَلْ سَوَّلَتْ لَكُمْ أَنفُسُكُمْ أَمْرًۭا ۖ فَصَبْرٌۭ جَمِيلٌۭ ۖ وَٱللَّهُ ٱلْمُسْتَعَانُ عَلَىٰ مَا تَصِفُونَ ﴿١٨﴾\\
\textamh{19.\  } & وَجَآءَتْ سَيَّارَةٌۭ فَأَرْسَلُوا۟ وَارِدَهُمْ فَأَدْلَىٰ دَلْوَهُۥ ۖ قَالَ يَـٰبُشْرَىٰ هَـٰذَا غُلَـٰمٌۭ ۚ وَأَسَرُّوهُ بِضَٰعَةًۭ ۚ وَٱللَّهُ عَلِيمٌۢ بِمَا يَعْمَلُونَ ﴿١٩﴾\\
\textamh{20.\  } & وَشَرَوْهُ بِثَمَنٍۭ بَخْسٍۢ دَرَٰهِمَ مَعْدُودَةٍۢ وَكَانُوا۟ فِيهِ مِنَ ٱلزَّٰهِدِينَ ﴿٢٠﴾\\
\textamh{21.\  } & وَقَالَ ٱلَّذِى ٱشْتَرَىٰهُ مِن مِّصْرَ لِٱمْرَأَتِهِۦٓ أَكْرِمِى مَثْوَىٰهُ عَسَىٰٓ أَن يَنفَعَنَآ أَوْ نَتَّخِذَهُۥ وَلَدًۭا ۚ وَكَذَٟلِكَ مَكَّنَّا لِيُوسُفَ فِى ٱلْأَرْضِ وَلِنُعَلِّمَهُۥ مِن تَأْوِيلِ ٱلْأَحَادِيثِ ۚ وَٱللَّهُ غَالِبٌ عَلَىٰٓ أَمْرِهِۦ وَلَـٰكِنَّ أَكْثَرَ ٱلنَّاسِ لَا يَعْلَمُونَ ﴿٢١﴾\\
\textamh{22.\  } & وَلَمَّا بَلَغَ أَشُدَّهُۥٓ ءَاتَيْنَـٰهُ حُكْمًۭا وَعِلْمًۭا ۚ وَكَذَٟلِكَ نَجْزِى ٱلْمُحْسِنِينَ ﴿٢٢﴾\\
\textamh{23.\  } & وَرَٰوَدَتْهُ ٱلَّتِى هُوَ فِى بَيْتِهَا عَن نَّفْسِهِۦ وَغَلَّقَتِ ٱلْأَبْوَٟبَ وَقَالَتْ هَيْتَ لَكَ ۚ قَالَ مَعَاذَ ٱللَّهِ ۖ إِنَّهُۥ رَبِّىٓ أَحْسَنَ مَثْوَاىَ ۖ إِنَّهُۥ لَا يُفْلِحُ ٱلظَّـٰلِمُونَ ﴿٢٣﴾\\
\textamh{24.\  } & وَلَقَدْ هَمَّتْ بِهِۦ ۖ وَهَمَّ بِهَا لَوْلَآ أَن رَّءَا بُرْهَـٰنَ رَبِّهِۦ ۚ كَذَٟلِكَ لِنَصْرِفَ عَنْهُ ٱلسُّوٓءَ وَٱلْفَحْشَآءَ ۚ إِنَّهُۥ مِنْ عِبَادِنَا ٱلْمُخْلَصِينَ ﴿٢٤﴾\\
\textamh{25.\  } & وَٱسْتَبَقَا ٱلْبَابَ وَقَدَّتْ قَمِيصَهُۥ مِن دُبُرٍۢ وَأَلْفَيَا سَيِّدَهَا لَدَا ٱلْبَابِ ۚ قَالَتْ مَا جَزَآءُ مَنْ أَرَادَ بِأَهْلِكَ سُوٓءًا إِلَّآ أَن يُسْجَنَ أَوْ عَذَابٌ أَلِيمٌۭ ﴿٢٥﴾\\
\textamh{26.\  } & قَالَ هِىَ رَٰوَدَتْنِى عَن نَّفْسِى ۚ وَشَهِدَ شَاهِدٌۭ مِّنْ أَهْلِهَآ إِن كَانَ قَمِيصُهُۥ قُدَّ مِن قُبُلٍۢ فَصَدَقَتْ وَهُوَ مِنَ ٱلْكَـٰذِبِينَ ﴿٢٦﴾\\
\textamh{27.\  } & وَإِن كَانَ قَمِيصُهُۥ قُدَّ مِن دُبُرٍۢ فَكَذَبَتْ وَهُوَ مِنَ ٱلصَّـٰدِقِينَ ﴿٢٧﴾\\
\textamh{28.\  } & فَلَمَّا رَءَا قَمِيصَهُۥ قُدَّ مِن دُبُرٍۢ قَالَ إِنَّهُۥ مِن كَيْدِكُنَّ ۖ إِنَّ كَيْدَكُنَّ عَظِيمٌۭ ﴿٢٨﴾\\
\textamh{29.\  } & يُوسُفُ أَعْرِضْ عَنْ هَـٰذَا ۚ وَٱسْتَغْفِرِى لِذَنۢبِكِ ۖ إِنَّكِ كُنتِ مِنَ ٱلْخَاطِـِٔينَ ﴿٢٩﴾\\
\textamh{30.\  } & ۞ وَقَالَ نِسْوَةٌۭ فِى ٱلْمَدِينَةِ ٱمْرَأَتُ ٱلْعَزِيزِ تُرَٰوِدُ فَتَىٰهَا عَن نَّفْسِهِۦ ۖ قَدْ شَغَفَهَا حُبًّا ۖ إِنَّا لَنَرَىٰهَا فِى ضَلَـٰلٍۢ مُّبِينٍۢ ﴿٣٠﴾\\
\textamh{31.\  } & فَلَمَّا سَمِعَتْ بِمَكْرِهِنَّ أَرْسَلَتْ إِلَيْهِنَّ وَأَعْتَدَتْ لَهُنَّ مُتَّكَـًۭٔا وَءَاتَتْ كُلَّ وَٟحِدَةٍۢ مِّنْهُنَّ سِكِّينًۭا وَقَالَتِ ٱخْرُجْ عَلَيْهِنَّ ۖ فَلَمَّا رَأَيْنَهُۥٓ أَكْبَرْنَهُۥ وَقَطَّعْنَ أَيْدِيَهُنَّ وَقُلْنَ حَـٰشَ لِلَّهِ مَا هَـٰذَا بَشَرًا إِنْ هَـٰذَآ إِلَّا مَلَكٌۭ كَرِيمٌۭ ﴿٣١﴾\\
\textamh{32.\  } & قَالَتْ فَذَٟلِكُنَّ ٱلَّذِى لُمْتُنَّنِى فِيهِ ۖ وَلَقَدْ رَٰوَدتُّهُۥ عَن نَّفْسِهِۦ فَٱسْتَعْصَمَ ۖ وَلَئِن لَّمْ يَفْعَلْ مَآ ءَامُرُهُۥ لَيُسْجَنَنَّ وَلَيَكُونًۭا مِّنَ ٱلصَّـٰغِرِينَ ﴿٣٢﴾\\
\textamh{33.\  } & قَالَ رَبِّ ٱلسِّجْنُ أَحَبُّ إِلَىَّ مِمَّا يَدْعُونَنِىٓ إِلَيْهِ ۖ وَإِلَّا تَصْرِفْ عَنِّى كَيْدَهُنَّ أَصْبُ إِلَيْهِنَّ وَأَكُن مِّنَ ٱلْجَٰهِلِينَ ﴿٣٣﴾\\
\textamh{34.\  } & فَٱسْتَجَابَ لَهُۥ رَبُّهُۥ فَصَرَفَ عَنْهُ كَيْدَهُنَّ ۚ إِنَّهُۥ هُوَ ٱلسَّمِيعُ ٱلْعَلِيمُ ﴿٣٤﴾\\
\textamh{35.\  } & ثُمَّ بَدَا لَهُم مِّنۢ بَعْدِ مَا رَأَوُا۟ ٱلْءَايَـٰتِ لَيَسْجُنُنَّهُۥ حَتَّىٰ حِينٍۢ ﴿٣٥﴾\\
\textamh{36.\  } & وَدَخَلَ مَعَهُ ٱلسِّجْنَ فَتَيَانِ ۖ قَالَ أَحَدُهُمَآ إِنِّىٓ أَرَىٰنِىٓ أَعْصِرُ خَمْرًۭا ۖ وَقَالَ ٱلْءَاخَرُ إِنِّىٓ أَرَىٰنِىٓ أَحْمِلُ فَوْقَ رَأْسِى خُبْزًۭا تَأْكُلُ ٱلطَّيْرُ مِنْهُ ۖ نَبِّئْنَا بِتَأْوِيلِهِۦٓ ۖ إِنَّا نَرَىٰكَ مِنَ ٱلْمُحْسِنِينَ ﴿٣٦﴾\\
\textamh{37.\  } & قَالَ لَا يَأْتِيكُمَا طَعَامٌۭ تُرْزَقَانِهِۦٓ إِلَّا نَبَّأْتُكُمَا بِتَأْوِيلِهِۦ قَبْلَ أَن يَأْتِيَكُمَا ۚ ذَٟلِكُمَا مِمَّا عَلَّمَنِى رَبِّىٓ ۚ إِنِّى تَرَكْتُ مِلَّةَ قَوْمٍۢ لَّا يُؤْمِنُونَ بِٱللَّهِ وَهُم بِٱلْءَاخِرَةِ هُمْ كَـٰفِرُونَ ﴿٣٧﴾\\
\textamh{38.\  } & وَٱتَّبَعْتُ مِلَّةَ ءَابَآءِىٓ إِبْرَٰهِيمَ وَإِسْحَـٰقَ وَيَعْقُوبَ ۚ مَا كَانَ لَنَآ أَن نُّشْرِكَ بِٱللَّهِ مِن شَىْءٍۢ ۚ ذَٟلِكَ مِن فَضْلِ ٱللَّهِ عَلَيْنَا وَعَلَى ٱلنَّاسِ وَلَـٰكِنَّ أَكْثَرَ ٱلنَّاسِ لَا يَشْكُرُونَ ﴿٣٨﴾\\
\textamh{39.\  } & يَـٰصَىٰحِبَىِ ٱلسِّجْنِ ءَأَرْبَابٌۭ مُّتَفَرِّقُونَ خَيْرٌ أَمِ ٱللَّهُ ٱلْوَٟحِدُ ٱلْقَهَّارُ ﴿٣٩﴾\\
\textamh{40.\  } & مَا تَعْبُدُونَ مِن دُونِهِۦٓ إِلَّآ أَسْمَآءًۭ سَمَّيْتُمُوهَآ أَنتُمْ وَءَابَآؤُكُم مَّآ أَنزَلَ ٱللَّهُ بِهَا مِن سُلْطَٰنٍ ۚ إِنِ ٱلْحُكْمُ إِلَّا لِلَّهِ ۚ أَمَرَ أَلَّا تَعْبُدُوٓا۟ إِلَّآ إِيَّاهُ ۚ ذَٟلِكَ ٱلدِّينُ ٱلْقَيِّمُ وَلَـٰكِنَّ أَكْثَرَ ٱلنَّاسِ لَا يَعْلَمُونَ ﴿٤٠﴾\\
\textamh{41.\  } & يَـٰصَىٰحِبَىِ ٱلسِّجْنِ أَمَّآ أَحَدُكُمَا فَيَسْقِى رَبَّهُۥ خَمْرًۭا ۖ وَأَمَّا ٱلْءَاخَرُ فَيُصْلَبُ فَتَأْكُلُ ٱلطَّيْرُ مِن رَّأْسِهِۦ ۚ قُضِىَ ٱلْأَمْرُ ٱلَّذِى فِيهِ تَسْتَفْتِيَانِ ﴿٤١﴾\\
\textamh{42.\  } & وَقَالَ لِلَّذِى ظَنَّ أَنَّهُۥ نَاجٍۢ مِّنْهُمَا ٱذْكُرْنِى عِندَ رَبِّكَ فَأَنسَىٰهُ ٱلشَّيْطَٰنُ ذِكْرَ رَبِّهِۦ فَلَبِثَ فِى ٱلسِّجْنِ بِضْعَ سِنِينَ ﴿٤٢﴾\\
\textamh{43.\  } & وَقَالَ ٱلْمَلِكُ إِنِّىٓ أَرَىٰ سَبْعَ بَقَرَٰتٍۢ سِمَانٍۢ يَأْكُلُهُنَّ سَبْعٌ عِجَافٌۭ وَسَبْعَ سُنۢبُلَـٰتٍ خُضْرٍۢ وَأُخَرَ يَابِسَـٰتٍۢ ۖ يَـٰٓأَيُّهَا ٱلْمَلَأُ أَفْتُونِى فِى رُءْيَـٰىَ إِن كُنتُمْ لِلرُّءْيَا تَعْبُرُونَ ﴿٤٣﴾\\
\textamh{44.\  } & قَالُوٓا۟ أَضْغَٰثُ أَحْلَـٰمٍۢ ۖ وَمَا نَحْنُ بِتَأْوِيلِ ٱلْأَحْلَـٰمِ بِعَـٰلِمِينَ ﴿٤٤﴾\\
\textamh{45.\  } & وَقَالَ ٱلَّذِى نَجَا مِنْهُمَا وَٱدَّكَرَ بَعْدَ أُمَّةٍ أَنَا۠ أُنَبِّئُكُم بِتَأْوِيلِهِۦ فَأَرْسِلُونِ ﴿٤٥﴾\\
\textamh{46.\  } & يُوسُفُ أَيُّهَا ٱلصِّدِّيقُ أَفْتِنَا فِى سَبْعِ بَقَرَٰتٍۢ سِمَانٍۢ يَأْكُلُهُنَّ سَبْعٌ عِجَافٌۭ وَسَبْعِ سُنۢبُلَـٰتٍ خُضْرٍۢ وَأُخَرَ يَابِسَـٰتٍۢ لَّعَلِّىٓ أَرْجِعُ إِلَى ٱلنَّاسِ لَعَلَّهُمْ يَعْلَمُونَ ﴿٤٦﴾\\
\textamh{47.\  } & قَالَ تَزْرَعُونَ سَبْعَ سِنِينَ دَأَبًۭا فَمَا حَصَدتُّمْ فَذَرُوهُ فِى سُنۢبُلِهِۦٓ إِلَّا قَلِيلًۭا مِّمَّا تَأْكُلُونَ ﴿٤٧﴾\\
\textamh{48.\  } & ثُمَّ يَأْتِى مِنۢ بَعْدِ ذَٟلِكَ سَبْعٌۭ شِدَادٌۭ يَأْكُلْنَ مَا قَدَّمْتُمْ لَهُنَّ إِلَّا قَلِيلًۭا مِّمَّا تُحْصِنُونَ ﴿٤٨﴾\\
\textamh{49.\  } & ثُمَّ يَأْتِى مِنۢ بَعْدِ ذَٟلِكَ عَامٌۭ فِيهِ يُغَاثُ ٱلنَّاسُ وَفِيهِ يَعْصِرُونَ ﴿٤٩﴾\\
\textamh{50.\  } & وَقَالَ ٱلْمَلِكُ ٱئْتُونِى بِهِۦ ۖ فَلَمَّا جَآءَهُ ٱلرَّسُولُ قَالَ ٱرْجِعْ إِلَىٰ رَبِّكَ فَسْـَٔلْهُ مَا بَالُ ٱلنِّسْوَةِ ٱلَّٰتِى قَطَّعْنَ أَيْدِيَهُنَّ ۚ إِنَّ رَبِّى بِكَيْدِهِنَّ عَلِيمٌۭ ﴿٥٠﴾\\
\textamh{51.\  } & قَالَ مَا خَطْبُكُنَّ إِذْ رَٰوَدتُّنَّ يُوسُفَ عَن نَّفْسِهِۦ ۚ قُلْنَ حَـٰشَ لِلَّهِ مَا عَلِمْنَا عَلَيْهِ مِن سُوٓءٍۢ ۚ قَالَتِ ٱمْرَأَتُ ٱلْعَزِيزِ ٱلْـَٰٔنَ حَصْحَصَ ٱلْحَقُّ أَنَا۠ رَٰوَدتُّهُۥ عَن نَّفْسِهِۦ وَإِنَّهُۥ لَمِنَ ٱلصَّـٰدِقِينَ ﴿٥١﴾\\
\textamh{52.\  } & ذَٟلِكَ لِيَعْلَمَ أَنِّى لَمْ أَخُنْهُ بِٱلْغَيْبِ وَأَنَّ ٱللَّهَ لَا يَهْدِى كَيْدَ ٱلْخَآئِنِينَ ﴿٥٢﴾\\
\textamh{53.\  } & ۞ وَمَآ أُبَرِّئُ نَفْسِىٓ ۚ إِنَّ ٱلنَّفْسَ لَأَمَّارَةٌۢ بِٱلسُّوٓءِ إِلَّا مَا رَحِمَ رَبِّىٓ ۚ إِنَّ رَبِّى غَفُورٌۭ رَّحِيمٌۭ ﴿٥٣﴾\\
\textamh{54.\  } & وَقَالَ ٱلْمَلِكُ ٱئْتُونِى بِهِۦٓ أَسْتَخْلِصْهُ لِنَفْسِى ۖ فَلَمَّا كَلَّمَهُۥ قَالَ إِنَّكَ ٱلْيَوْمَ لَدَيْنَا مَكِينٌ أَمِينٌۭ ﴿٥٤﴾\\
\textamh{55.\  } & قَالَ ٱجْعَلْنِى عَلَىٰ خَزَآئِنِ ٱلْأَرْضِ ۖ إِنِّى حَفِيظٌ عَلِيمٌۭ ﴿٥٥﴾\\
\textamh{56.\  } & وَكَذَٟلِكَ مَكَّنَّا لِيُوسُفَ فِى ٱلْأَرْضِ يَتَبَوَّأُ مِنْهَا حَيْثُ يَشَآءُ ۚ نُصِيبُ بِرَحْمَتِنَا مَن نَّشَآءُ ۖ وَلَا نُضِيعُ أَجْرَ ٱلْمُحْسِنِينَ ﴿٥٦﴾\\
\textamh{57.\  } & وَلَأَجْرُ ٱلْءَاخِرَةِ خَيْرٌۭ لِّلَّذِينَ ءَامَنُوا۟ وَكَانُوا۟ يَتَّقُونَ ﴿٥٧﴾\\
\textamh{58.\  } & وَجَآءَ إِخْوَةُ يُوسُفَ فَدَخَلُوا۟ عَلَيْهِ فَعَرَفَهُمْ وَهُمْ لَهُۥ مُنكِرُونَ ﴿٥٨﴾\\
\textamh{59.\  } & وَلَمَّا جَهَّزَهُم بِجَهَازِهِمْ قَالَ ٱئْتُونِى بِأَخٍۢ لَّكُم مِّنْ أَبِيكُمْ ۚ أَلَا تَرَوْنَ أَنِّىٓ أُوفِى ٱلْكَيْلَ وَأَنَا۠ خَيْرُ ٱلْمُنزِلِينَ ﴿٥٩﴾\\
\textamh{60.\  } & فَإِن لَّمْ تَأْتُونِى بِهِۦ فَلَا كَيْلَ لَكُمْ عِندِى وَلَا تَقْرَبُونِ ﴿٦٠﴾\\
\textamh{61.\  } & قَالُوا۟ سَنُرَٰوِدُ عَنْهُ أَبَاهُ وَإِنَّا لَفَـٰعِلُونَ ﴿٦١﴾\\
\textamh{62.\  } & وَقَالَ لِفِتْيَـٰنِهِ ٱجْعَلُوا۟ بِضَٰعَتَهُمْ فِى رِحَالِهِمْ لَعَلَّهُمْ يَعْرِفُونَهَآ إِذَا ٱنقَلَبُوٓا۟ إِلَىٰٓ أَهْلِهِمْ لَعَلَّهُمْ يَرْجِعُونَ ﴿٦٢﴾\\
\textamh{63.\  } & فَلَمَّا رَجَعُوٓا۟ إِلَىٰٓ أَبِيهِمْ قَالُوا۟ يَـٰٓأَبَانَا مُنِعَ مِنَّا ٱلْكَيْلُ فَأَرْسِلْ مَعَنَآ أَخَانَا نَكْتَلْ وَإِنَّا لَهُۥ لَحَـٰفِظُونَ ﴿٦٣﴾\\
\textamh{64.\  } & قَالَ هَلْ ءَامَنُكُمْ عَلَيْهِ إِلَّا كَمَآ أَمِنتُكُمْ عَلَىٰٓ أَخِيهِ مِن قَبْلُ ۖ فَٱللَّهُ خَيْرٌ حَـٰفِظًۭا ۖ وَهُوَ أَرْحَمُ ٱلرَّٟحِمِينَ ﴿٦٤﴾\\
\textamh{65.\  } & وَلَمَّا فَتَحُوا۟ مَتَـٰعَهُمْ وَجَدُوا۟ بِضَٰعَتَهُمْ رُدَّتْ إِلَيْهِمْ ۖ قَالُوا۟ يَـٰٓأَبَانَا مَا نَبْغِى ۖ هَـٰذِهِۦ بِضَٰعَتُنَا رُدَّتْ إِلَيْنَا ۖ وَنَمِيرُ أَهْلَنَا وَنَحْفَظُ أَخَانَا وَنَزْدَادُ كَيْلَ بَعِيرٍۢ ۖ ذَٟلِكَ كَيْلٌۭ يَسِيرٌۭ ﴿٦٥﴾\\
\textamh{66.\  } & قَالَ لَنْ أُرْسِلَهُۥ مَعَكُمْ حَتَّىٰ تُؤْتُونِ مَوْثِقًۭا مِّنَ ٱللَّهِ لَتَأْتُنَّنِى بِهِۦٓ إِلَّآ أَن يُحَاطَ بِكُمْ ۖ فَلَمَّآ ءَاتَوْهُ مَوْثِقَهُمْ قَالَ ٱللَّهُ عَلَىٰ مَا نَقُولُ وَكِيلٌۭ ﴿٦٦﴾\\
\textamh{67.\  } & وَقَالَ يَـٰبَنِىَّ لَا تَدْخُلُوا۟ مِنۢ بَابٍۢ وَٟحِدٍۢ وَٱدْخُلُوا۟ مِنْ أَبْوَٟبٍۢ مُّتَفَرِّقَةٍۢ ۖ وَمَآ أُغْنِى عَنكُم مِّنَ ٱللَّهِ مِن شَىْءٍ ۖ إِنِ ٱلْحُكْمُ إِلَّا لِلَّهِ ۖ عَلَيْهِ تَوَكَّلْتُ ۖ وَعَلَيْهِ فَلْيَتَوَكَّلِ ٱلْمُتَوَكِّلُونَ ﴿٦٧﴾\\
\textamh{68.\  } & وَلَمَّا دَخَلُوا۟ مِنْ حَيْثُ أَمَرَهُمْ أَبُوهُم مَّا كَانَ يُغْنِى عَنْهُم مِّنَ ٱللَّهِ مِن شَىْءٍ إِلَّا حَاجَةًۭ فِى نَفْسِ يَعْقُوبَ قَضَىٰهَا ۚ وَإِنَّهُۥ لَذُو عِلْمٍۢ لِّمَا عَلَّمْنَـٰهُ وَلَـٰكِنَّ أَكْثَرَ ٱلنَّاسِ لَا يَعْلَمُونَ ﴿٦٨﴾\\
\textamh{69.\  } & وَلَمَّا دَخَلُوا۟ عَلَىٰ يُوسُفَ ءَاوَىٰٓ إِلَيْهِ أَخَاهُ ۖ قَالَ إِنِّىٓ أَنَا۠ أَخُوكَ فَلَا تَبْتَئِسْ بِمَا كَانُوا۟ يَعْمَلُونَ ﴿٦٩﴾\\
\textamh{70.\  } & فَلَمَّا جَهَّزَهُم بِجَهَازِهِمْ جَعَلَ ٱلسِّقَايَةَ فِى رَحْلِ أَخِيهِ ثُمَّ أَذَّنَ مُؤَذِّنٌ أَيَّتُهَا ٱلْعِيرُ إِنَّكُمْ لَسَـٰرِقُونَ ﴿٧٠﴾\\
\textamh{71.\  } & قَالُوا۟ وَأَقْبَلُوا۟ عَلَيْهِم مَّاذَا تَفْقِدُونَ ﴿٧١﴾\\
\textamh{72.\  } & قَالُوا۟ نَفْقِدُ صُوَاعَ ٱلْمَلِكِ وَلِمَن جَآءَ بِهِۦ حِمْلُ بَعِيرٍۢ وَأَنَا۠ بِهِۦ زَعِيمٌۭ ﴿٧٢﴾\\
\textamh{73.\  } & قَالُوا۟ تَٱللَّهِ لَقَدْ عَلِمْتُم مَّا جِئْنَا لِنُفْسِدَ فِى ٱلْأَرْضِ وَمَا كُنَّا سَـٰرِقِينَ ﴿٧٣﴾\\
\textamh{74.\  } & قَالُوا۟ فَمَا جَزَٰٓؤُهُۥٓ إِن كُنتُمْ كَـٰذِبِينَ ﴿٧٤﴾\\
\textamh{75.\  } & قَالُوا۟ جَزَٰٓؤُهُۥ مَن وُجِدَ فِى رَحْلِهِۦ فَهُوَ جَزَٰٓؤُهُۥ ۚ كَذَٟلِكَ نَجْزِى ٱلظَّـٰلِمِينَ ﴿٧٥﴾\\
\textamh{76.\  } & فَبَدَأَ بِأَوْعِيَتِهِمْ قَبْلَ وِعَآءِ أَخِيهِ ثُمَّ ٱسْتَخْرَجَهَا مِن وِعَآءِ أَخِيهِ ۚ كَذَٟلِكَ كِدْنَا لِيُوسُفَ ۖ مَا كَانَ لِيَأْخُذَ أَخَاهُ فِى دِينِ ٱلْمَلِكِ إِلَّآ أَن يَشَآءَ ٱللَّهُ ۚ نَرْفَعُ دَرَجَٰتٍۢ مَّن نَّشَآءُ ۗ وَفَوْقَ كُلِّ ذِى عِلْمٍ عَلِيمٌۭ ﴿٧٦﴾\\
\textamh{77.\  } & ۞ قَالُوٓا۟ إِن يَسْرِقْ فَقَدْ سَرَقَ أَخٌۭ لَّهُۥ مِن قَبْلُ ۚ فَأَسَرَّهَا يُوسُفُ فِى نَفْسِهِۦ وَلَمْ يُبْدِهَا لَهُمْ ۚ قَالَ أَنتُمْ شَرٌّۭ مَّكَانًۭا ۖ وَٱللَّهُ أَعْلَمُ بِمَا تَصِفُونَ ﴿٧٧﴾\\
\textamh{78.\  } & قَالُوا۟ يَـٰٓأَيُّهَا ٱلْعَزِيزُ إِنَّ لَهُۥٓ أَبًۭا شَيْخًۭا كَبِيرًۭا فَخُذْ أَحَدَنَا مَكَانَهُۥٓ ۖ إِنَّا نَرَىٰكَ مِنَ ٱلْمُحْسِنِينَ ﴿٧٨﴾\\
\textamh{79.\  } & قَالَ مَعَاذَ ٱللَّهِ أَن نَّأْخُذَ إِلَّا مَن وَجَدْنَا مَتَـٰعَنَا عِندَهُۥٓ إِنَّآ إِذًۭا لَّظَـٰلِمُونَ ﴿٧٩﴾\\
\textamh{80.\  } & فَلَمَّا ٱسْتَيْـَٔسُوا۟ مِنْهُ خَلَصُوا۟ نَجِيًّۭا ۖ قَالَ كَبِيرُهُمْ أَلَمْ تَعْلَمُوٓا۟ أَنَّ أَبَاكُمْ قَدْ أَخَذَ عَلَيْكُم مَّوْثِقًۭا مِّنَ ٱللَّهِ وَمِن قَبْلُ مَا فَرَّطتُمْ فِى يُوسُفَ ۖ فَلَنْ أَبْرَحَ ٱلْأَرْضَ حَتَّىٰ يَأْذَنَ لِىٓ أَبِىٓ أَوْ يَحْكُمَ ٱللَّهُ لِى ۖ وَهُوَ خَيْرُ ٱلْحَـٰكِمِينَ ﴿٨٠﴾\\
\textamh{81.\  } & ٱرْجِعُوٓا۟ إِلَىٰٓ أَبِيكُمْ فَقُولُوا۟ يَـٰٓأَبَانَآ إِنَّ ٱبْنَكَ سَرَقَ وَمَا شَهِدْنَآ إِلَّا بِمَا عَلِمْنَا وَمَا كُنَّا لِلْغَيْبِ حَـٰفِظِينَ ﴿٨١﴾\\
\textamh{82.\  } & وَسْـَٔلِ ٱلْقَرْيَةَ ٱلَّتِى كُنَّا فِيهَا وَٱلْعِيرَ ٱلَّتِىٓ أَقْبَلْنَا فِيهَا ۖ وَإِنَّا لَصَـٰدِقُونَ ﴿٨٢﴾\\
\textamh{83.\  } & قَالَ بَلْ سَوَّلَتْ لَكُمْ أَنفُسُكُمْ أَمْرًۭا ۖ فَصَبْرٌۭ جَمِيلٌ ۖ عَسَى ٱللَّهُ أَن يَأْتِيَنِى بِهِمْ جَمِيعًا ۚ إِنَّهُۥ هُوَ ٱلْعَلِيمُ ٱلْحَكِيمُ ﴿٨٣﴾\\
\textamh{84.\  } & وَتَوَلَّىٰ عَنْهُمْ وَقَالَ يَـٰٓأَسَفَىٰ عَلَىٰ يُوسُفَ وَٱبْيَضَّتْ عَيْنَاهُ مِنَ ٱلْحُزْنِ فَهُوَ كَظِيمٌۭ ﴿٨٤﴾\\
\textamh{85.\  } & قَالُوا۟ تَٱللَّهِ تَفْتَؤُا۟ تَذْكُرُ يُوسُفَ حَتَّىٰ تَكُونَ حَرَضًا أَوْ تَكُونَ مِنَ ٱلْهَـٰلِكِينَ ﴿٨٥﴾\\
\textamh{86.\  } & قَالَ إِنَّمَآ أَشْكُوا۟ بَثِّى وَحُزْنِىٓ إِلَى ٱللَّهِ وَأَعْلَمُ مِنَ ٱللَّهِ مَا لَا تَعْلَمُونَ ﴿٨٦﴾\\
\textamh{87.\  } & يَـٰبَنِىَّ ٱذْهَبُوا۟ فَتَحَسَّسُوا۟ مِن يُوسُفَ وَأَخِيهِ وَلَا تَا۟يْـَٔسُوا۟ مِن رَّوْحِ ٱللَّهِ ۖ إِنَّهُۥ لَا يَا۟يْـَٔسُ مِن رَّوْحِ ٱللَّهِ إِلَّا ٱلْقَوْمُ ٱلْكَـٰفِرُونَ ﴿٨٧﴾\\
\textamh{88.\  } & فَلَمَّا دَخَلُوا۟ عَلَيْهِ قَالُوا۟ يَـٰٓأَيُّهَا ٱلْعَزِيزُ مَسَّنَا وَأَهْلَنَا ٱلضُّرُّ وَجِئْنَا بِبِضَٰعَةٍۢ مُّزْجَىٰةٍۢ فَأَوْفِ لَنَا ٱلْكَيْلَ وَتَصَدَّقْ عَلَيْنَآ ۖ إِنَّ ٱللَّهَ يَجْزِى ٱلْمُتَصَدِّقِينَ ﴿٨٨﴾\\
\textamh{89.\  } & قَالَ هَلْ عَلِمْتُم مَّا فَعَلْتُم بِيُوسُفَ وَأَخِيهِ إِذْ أَنتُمْ جَٰهِلُونَ ﴿٨٩﴾\\
\textamh{90.\  } & قَالُوٓا۟ أَءِنَّكَ لَأَنتَ يُوسُفُ ۖ قَالَ أَنَا۠ يُوسُفُ وَهَـٰذَآ أَخِى ۖ قَدْ مَنَّ ٱللَّهُ عَلَيْنَآ ۖ إِنَّهُۥ مَن يَتَّقِ وَيَصْبِرْ فَإِنَّ ٱللَّهَ لَا يُضِيعُ أَجْرَ ٱلْمُحْسِنِينَ ﴿٩٠﴾\\
\textamh{91.\  } & قَالُوا۟ تَٱللَّهِ لَقَدْ ءَاثَرَكَ ٱللَّهُ عَلَيْنَا وَإِن كُنَّا لَخَـٰطِـِٔينَ ﴿٩١﴾\\
\textamh{92.\  } & قَالَ لَا تَثْرِيبَ عَلَيْكُمُ ٱلْيَوْمَ ۖ يَغْفِرُ ٱللَّهُ لَكُمْ ۖ وَهُوَ أَرْحَمُ ٱلرَّٟحِمِينَ ﴿٩٢﴾\\
\textamh{93.\  } & ٱذْهَبُوا۟ بِقَمِيصِى هَـٰذَا فَأَلْقُوهُ عَلَىٰ وَجْهِ أَبِى يَأْتِ بَصِيرًۭا وَأْتُونِى بِأَهْلِكُمْ أَجْمَعِينَ ﴿٩٣﴾\\
\textamh{94.\  } & وَلَمَّا فَصَلَتِ ٱلْعِيرُ قَالَ أَبُوهُمْ إِنِّى لَأَجِدُ رِيحَ يُوسُفَ ۖ لَوْلَآ أَن تُفَنِّدُونِ ﴿٩٤﴾\\
\textamh{95.\  } & قَالُوا۟ تَٱللَّهِ إِنَّكَ لَفِى ضَلَـٰلِكَ ٱلْقَدِيمِ ﴿٩٥﴾\\
\textamh{96.\  } & فَلَمَّآ أَن جَآءَ ٱلْبَشِيرُ أَلْقَىٰهُ عَلَىٰ وَجْهِهِۦ فَٱرْتَدَّ بَصِيرًۭا ۖ قَالَ أَلَمْ أَقُل لَّكُمْ إِنِّىٓ أَعْلَمُ مِنَ ٱللَّهِ مَا لَا تَعْلَمُونَ ﴿٩٦﴾\\
\textamh{97.\  } & قَالُوا۟ يَـٰٓأَبَانَا ٱسْتَغْفِرْ لَنَا ذُنُوبَنَآ إِنَّا كُنَّا خَـٰطِـِٔينَ ﴿٩٧﴾\\
\textamh{98.\  } & قَالَ سَوْفَ أَسْتَغْفِرُ لَكُمْ رَبِّىٓ ۖ إِنَّهُۥ هُوَ ٱلْغَفُورُ ٱلرَّحِيمُ ﴿٩٨﴾\\
\textamh{99.\  } & فَلَمَّا دَخَلُوا۟ عَلَىٰ يُوسُفَ ءَاوَىٰٓ إِلَيْهِ أَبَوَيْهِ وَقَالَ ٱدْخُلُوا۟ مِصْرَ إِن شَآءَ ٱللَّهُ ءَامِنِينَ ﴿٩٩﴾\\
\textamh{100.\  } & وَرَفَعَ أَبَوَيْهِ عَلَى ٱلْعَرْشِ وَخَرُّوا۟ لَهُۥ سُجَّدًۭا ۖ وَقَالَ يَـٰٓأَبَتِ هَـٰذَا تَأْوِيلُ رُءْيَـٰىَ مِن قَبْلُ قَدْ جَعَلَهَا رَبِّى حَقًّۭا ۖ وَقَدْ أَحْسَنَ بِىٓ إِذْ أَخْرَجَنِى مِنَ ٱلسِّجْنِ وَجَآءَ بِكُم مِّنَ ٱلْبَدْوِ مِنۢ بَعْدِ أَن نَّزَغَ ٱلشَّيْطَٰنُ بَيْنِى وَبَيْنَ إِخْوَتِىٓ ۚ إِنَّ رَبِّى لَطِيفٌۭ لِّمَا يَشَآءُ ۚ إِنَّهُۥ هُوَ ٱلْعَلِيمُ ٱلْحَكِيمُ ﴿١٠٠﴾\\
\textamh{101.\  } & ۞ رَبِّ قَدْ ءَاتَيْتَنِى مِنَ ٱلْمُلْكِ وَعَلَّمْتَنِى مِن تَأْوِيلِ ٱلْأَحَادِيثِ ۚ فَاطِرَ ٱلسَّمَـٰوَٟتِ وَٱلْأَرْضِ أَنتَ وَلِىِّۦ فِى ٱلدُّنْيَا وَٱلْءَاخِرَةِ ۖ تَوَفَّنِى مُسْلِمًۭا وَأَلْحِقْنِى بِٱلصَّـٰلِحِينَ ﴿١٠١﴾\\
\textamh{102.\  } & ذَٟلِكَ مِنْ أَنۢبَآءِ ٱلْغَيْبِ نُوحِيهِ إِلَيْكَ ۖ وَمَا كُنتَ لَدَيْهِمْ إِذْ أَجْمَعُوٓا۟ أَمْرَهُمْ وَهُمْ يَمْكُرُونَ ﴿١٠٢﴾\\
\textamh{103.\  } & وَمَآ أَكْثَرُ ٱلنَّاسِ وَلَوْ حَرَصْتَ بِمُؤْمِنِينَ ﴿١٠٣﴾\\
\textamh{104.\  } & وَمَا تَسْـَٔلُهُمْ عَلَيْهِ مِنْ أَجْرٍ ۚ إِنْ هُوَ إِلَّا ذِكْرٌۭ لِّلْعَـٰلَمِينَ ﴿١٠٤﴾\\
\textamh{105.\  } & وَكَأَيِّن مِّنْ ءَايَةٍۢ فِى ٱلسَّمَـٰوَٟتِ وَٱلْأَرْضِ يَمُرُّونَ عَلَيْهَا وَهُمْ عَنْهَا مُعْرِضُونَ ﴿١٠٥﴾\\
\textamh{106.\  } & وَمَا يُؤْمِنُ أَكْثَرُهُم بِٱللَّهِ إِلَّا وَهُم مُّشْرِكُونَ ﴿١٠٦﴾\\
\textamh{107.\  } & أَفَأَمِنُوٓا۟ أَن تَأْتِيَهُمْ غَٰشِيَةٌۭ مِّنْ عَذَابِ ٱللَّهِ أَوْ تَأْتِيَهُمُ ٱلسَّاعَةُ بَغْتَةًۭ وَهُمْ لَا يَشْعُرُونَ ﴿١٠٧﴾\\
\textamh{108.\  } & قُلْ هَـٰذِهِۦ سَبِيلِىٓ أَدْعُوٓا۟ إِلَى ٱللَّهِ ۚ عَلَىٰ بَصِيرَةٍ أَنَا۠ وَمَنِ ٱتَّبَعَنِى ۖ وَسُبْحَـٰنَ ٱللَّهِ وَمَآ أَنَا۠ مِنَ ٱلْمُشْرِكِينَ ﴿١٠٨﴾\\
\textamh{109.\  } & وَمَآ أَرْسَلْنَا مِن قَبْلِكَ إِلَّا رِجَالًۭا نُّوحِىٓ إِلَيْهِم مِّنْ أَهْلِ ٱلْقُرَىٰٓ ۗ أَفَلَمْ يَسِيرُوا۟ فِى ٱلْأَرْضِ فَيَنظُرُوا۟ كَيْفَ كَانَ عَـٰقِبَةُ ٱلَّذِينَ مِن قَبْلِهِمْ ۗ وَلَدَارُ ٱلْءَاخِرَةِ خَيْرٌۭ لِّلَّذِينَ ٱتَّقَوْا۟ ۗ أَفَلَا تَعْقِلُونَ ﴿١٠٩﴾\\
\textamh{110.\  } & حَتَّىٰٓ إِذَا ٱسْتَيْـَٔسَ ٱلرُّسُلُ وَظَنُّوٓا۟ أَنَّهُمْ قَدْ كُذِبُوا۟ جَآءَهُمْ نَصْرُنَا فَنُجِّىَ مَن نَّشَآءُ ۖ وَلَا يُرَدُّ بَأْسُنَا عَنِ ٱلْقَوْمِ ٱلْمُجْرِمِينَ ﴿١١٠﴾\\
\textamh{111.\  } & لَقَدْ كَانَ فِى قَصَصِهِمْ عِبْرَةٌۭ لِّأُو۟لِى ٱلْأَلْبَٰبِ ۗ مَا كَانَ حَدِيثًۭا يُفْتَرَىٰ وَلَـٰكِن تَصْدِيقَ ٱلَّذِى بَيْنَ يَدَيْهِ وَتَفْصِيلَ كُلِّ شَىْءٍۢ وَهُدًۭى وَرَحْمَةًۭ لِّقَوْمٍۢ يُؤْمِنُونَ ﴿١١١﴾\\
\end{longtable}
\clearpage
%% License: BSD style (Berkley) (i.e. Put the Copyright owner's name always)
%% Writer and Copyright (to): Bewketu(Bilal) Tadilo (2016-17)
\centering\section{\LR{\textamharic{ሱራቱ አልርኣድ -}  \RL{سوره  الرعد}}}
\begin{longtable}{%
  @{}
    p{.5\textwidth}
  @{~~~~~~~~~~~~~}
    p{.5\textwidth}
    @{}
}
\nopagebreak
\textamh{\ \ \ \ \ \  ቢስሚላሂ አራህመኒ ራሂይም } &  بِسْمِ ٱللَّهِ ٱلرَّحْمَـٰنِ ٱلرَّحِيمِ\\
\textamh{1.\  } &  الٓمٓر ۚ تِلْكَ ءَايَـٰتُ ٱلْكِتَـٰبِ ۗ وَٱلَّذِىٓ أُنزِلَ إِلَيْكَ مِن رَّبِّكَ ٱلْحَقُّ وَلَـٰكِنَّ أَكْثَرَ ٱلنَّاسِ لَا يُؤْمِنُونَ ﴿١﴾\\
\textamh{2.\  } & ٱللَّهُ ٱلَّذِى رَفَعَ ٱلسَّمَـٰوَٟتِ بِغَيْرِ عَمَدٍۢ تَرَوْنَهَا ۖ ثُمَّ ٱسْتَوَىٰ عَلَى ٱلْعَرْشِ ۖ وَسَخَّرَ ٱلشَّمْسَ وَٱلْقَمَرَ ۖ كُلٌّۭ يَجْرِى لِأَجَلٍۢ مُّسَمًّۭى ۚ يُدَبِّرُ ٱلْأَمْرَ يُفَصِّلُ ٱلْءَايَـٰتِ لَعَلَّكُم بِلِقَآءِ رَبِّكُمْ تُوقِنُونَ ﴿٢﴾\\
\textamh{3.\  } & وَهُوَ ٱلَّذِى مَدَّ ٱلْأَرْضَ وَجَعَلَ فِيهَا رَوَٟسِىَ وَأَنْهَـٰرًۭا ۖ وَمِن كُلِّ ٱلثَّمَرَٰتِ جَعَلَ فِيهَا زَوْجَيْنِ ٱثْنَيْنِ ۖ يُغْشِى ٱلَّيْلَ ٱلنَّهَارَ ۚ إِنَّ فِى ذَٟلِكَ لَءَايَـٰتٍۢ لِّقَوْمٍۢ يَتَفَكَّرُونَ ﴿٣﴾\\
\textamh{4.\  } & وَفِى ٱلْأَرْضِ قِطَعٌۭ مُّتَجَٰوِرَٰتٌۭ وَجَنَّـٰتٌۭ مِّنْ أَعْنَـٰبٍۢ وَزَرْعٌۭ وَنَخِيلٌۭ صِنْوَانٌۭ وَغَيْرُ صِنْوَانٍۢ يُسْقَىٰ بِمَآءٍۢ وَٟحِدٍۢ وَنُفَضِّلُ بَعْضَهَا عَلَىٰ بَعْضٍۢ فِى ٱلْأُكُلِ ۚ إِنَّ فِى ذَٟلِكَ لَءَايَـٰتٍۢ لِّقَوْمٍۢ يَعْقِلُونَ ﴿٤﴾\\
\textamh{5.\  } & ۞ وَإِن تَعْجَبْ فَعَجَبٌۭ قَوْلُهُمْ أَءِذَا كُنَّا تُرَٰبًا أَءِنَّا لَفِى خَلْقٍۢ جَدِيدٍ ۗ أُو۟لَـٰٓئِكَ ٱلَّذِينَ كَفَرُوا۟ بِرَبِّهِمْ ۖ وَأُو۟لَـٰٓئِكَ ٱلْأَغْلَـٰلُ فِىٓ أَعْنَاقِهِمْ ۖ وَأُو۟لَـٰٓئِكَ أَصْحَـٰبُ ٱلنَّارِ ۖ هُمْ فِيهَا خَـٰلِدُونَ ﴿٥﴾\\
\textamh{6.\  } & وَيَسْتَعْجِلُونَكَ بِٱلسَّيِّئَةِ قَبْلَ ٱلْحَسَنَةِ وَقَدْ خَلَتْ مِن قَبْلِهِمُ ٱلْمَثُلَـٰتُ ۗ وَإِنَّ رَبَّكَ لَذُو مَغْفِرَةٍۢ لِّلنَّاسِ عَلَىٰ ظُلْمِهِمْ ۖ وَإِنَّ رَبَّكَ لَشَدِيدُ ٱلْعِقَابِ ﴿٦﴾\\
\textamh{7.\  } & وَيَقُولُ ٱلَّذِينَ كَفَرُوا۟ لَوْلَآ أُنزِلَ عَلَيْهِ ءَايَةٌۭ مِّن رَّبِّهِۦٓ ۗ إِنَّمَآ أَنتَ مُنذِرٌۭ ۖ وَلِكُلِّ قَوْمٍ هَادٍ ﴿٧﴾\\
\textamh{8.\  } & ٱللَّهُ يَعْلَمُ مَا تَحْمِلُ كُلُّ أُنثَىٰ وَمَا تَغِيضُ ٱلْأَرْحَامُ وَمَا تَزْدَادُ ۖ وَكُلُّ شَىْءٍ عِندَهُۥ بِمِقْدَارٍ ﴿٨﴾\\
\textamh{9.\  } & عَـٰلِمُ ٱلْغَيْبِ وَٱلشَّهَـٰدَةِ ٱلْكَبِيرُ ٱلْمُتَعَالِ ﴿٩﴾\\
\textamh{10.\  } & سَوَآءٌۭ مِّنكُم مَّنْ أَسَرَّ ٱلْقَوْلَ وَمَن جَهَرَ بِهِۦ وَمَنْ هُوَ مُسْتَخْفٍۭ بِٱلَّيْلِ وَسَارِبٌۢ بِٱلنَّهَارِ ﴿١٠﴾\\
\textamh{11.\  } & لَهُۥ مُعَقِّبَٰتٌۭ مِّنۢ بَيْنِ يَدَيْهِ وَمِنْ خَلْفِهِۦ يَحْفَظُونَهُۥ مِنْ أَمْرِ ٱللَّهِ ۗ إِنَّ ٱللَّهَ لَا يُغَيِّرُ مَا بِقَوْمٍ حَتَّىٰ يُغَيِّرُوا۟ مَا بِأَنفُسِهِمْ ۗ وَإِذَآ أَرَادَ ٱللَّهُ بِقَوْمٍۢ سُوٓءًۭا فَلَا مَرَدَّ لَهُۥ ۚ وَمَا لَهُم مِّن دُونِهِۦ مِن وَالٍ ﴿١١﴾\\
\textamh{12.\  } & هُوَ ٱلَّذِى يُرِيكُمُ ٱلْبَرْقَ خَوْفًۭا وَطَمَعًۭا وَيُنشِئُ ٱلسَّحَابَ ٱلثِّقَالَ ﴿١٢﴾\\
\textamh{13.\  } & وَيُسَبِّحُ ٱلرَّعْدُ بِحَمْدِهِۦ وَٱلْمَلَـٰٓئِكَةُ مِنْ خِيفَتِهِۦ وَيُرْسِلُ ٱلصَّوَٟعِقَ فَيُصِيبُ بِهَا مَن يَشَآءُ وَهُمْ يُجَٰدِلُونَ فِى ٱللَّهِ وَهُوَ شَدِيدُ ٱلْمِحَالِ ﴿١٣﴾\\
\textamh{14.\  } & لَهُۥ دَعْوَةُ ٱلْحَقِّ ۖ وَٱلَّذِينَ يَدْعُونَ مِن دُونِهِۦ لَا يَسْتَجِيبُونَ لَهُم بِشَىْءٍ إِلَّا كَبَٰسِطِ كَفَّيْهِ إِلَى ٱلْمَآءِ لِيَبْلُغَ فَاهُ وَمَا هُوَ بِبَٰلِغِهِۦ ۚ وَمَا دُعَآءُ ٱلْكَـٰفِرِينَ إِلَّا فِى ضَلَـٰلٍۢ ﴿١٤﴾\\
\textamh{15.\  } & وَلِلَّهِ يَسْجُدُ مَن فِى ٱلسَّمَـٰوَٟتِ وَٱلْأَرْضِ طَوْعًۭا وَكَرْهًۭا وَظِلَـٰلُهُم بِٱلْغُدُوِّ وَٱلْءَاصَالِ ۩ ﴿١٥﴾\\
\textamh{16.\  } & قُلْ مَن رَّبُّ ٱلسَّمَـٰوَٟتِ وَٱلْأَرْضِ قُلِ ٱللَّهُ ۚ قُلْ أَفَٱتَّخَذْتُم مِّن دُونِهِۦٓ أَوْلِيَآءَ لَا يَمْلِكُونَ لِأَنفُسِهِمْ نَفْعًۭا وَلَا ضَرًّۭا ۚ قُلْ هَلْ يَسْتَوِى ٱلْأَعْمَىٰ وَٱلْبَصِيرُ أَمْ هَلْ تَسْتَوِى ٱلظُّلُمَـٰتُ وَٱلنُّورُ ۗ أَمْ جَعَلُوا۟ لِلَّهِ شُرَكَآءَ خَلَقُوا۟ كَخَلْقِهِۦ فَتَشَـٰبَهَ ٱلْخَلْقُ عَلَيْهِمْ ۚ قُلِ ٱللَّهُ خَـٰلِقُ كُلِّ شَىْءٍۢ وَهُوَ ٱلْوَٟحِدُ ٱلْقَهَّٰرُ ﴿١٦﴾\\
\textamh{17.\  } & أَنزَلَ مِنَ ٱلسَّمَآءِ مَآءًۭ فَسَالَتْ أَوْدِيَةٌۢ بِقَدَرِهَا فَٱحْتَمَلَ ٱلسَّيْلُ زَبَدًۭا رَّابِيًۭا ۚ وَمِمَّا يُوقِدُونَ عَلَيْهِ فِى ٱلنَّارِ ٱبْتِغَآءَ حِلْيَةٍ أَوْ مَتَـٰعٍۢ زَبَدٌۭ مِّثْلُهُۥ ۚ كَذَٟلِكَ يَضْرِبُ ٱللَّهُ ٱلْحَقَّ وَٱلْبَٰطِلَ ۚ فَأَمَّا ٱلزَّبَدُ فَيَذْهَبُ جُفَآءًۭ ۖ وَأَمَّا مَا يَنفَعُ ٱلنَّاسَ فَيَمْكُثُ فِى ٱلْأَرْضِ ۚ كَذَٟلِكَ يَضْرِبُ ٱللَّهُ ٱلْأَمْثَالَ ﴿١٧﴾\\
\textamh{18.\  } & لِلَّذِينَ ٱسْتَجَابُوا۟ لِرَبِّهِمُ ٱلْحُسْنَىٰ ۚ وَٱلَّذِينَ لَمْ يَسْتَجِيبُوا۟ لَهُۥ لَوْ أَنَّ لَهُم مَّا فِى ٱلْأَرْضِ جَمِيعًۭا وَمِثْلَهُۥ مَعَهُۥ لَٱفْتَدَوْا۟ بِهِۦٓ ۚ أُو۟لَـٰٓئِكَ لَهُمْ سُوٓءُ ٱلْحِسَابِ وَمَأْوَىٰهُمْ جَهَنَّمُ ۖ وَبِئْسَ ٱلْمِهَادُ ﴿١٨﴾\\
\textamh{19.\  } & ۞ أَفَمَن يَعْلَمُ أَنَّمَآ أُنزِلَ إِلَيْكَ مِن رَّبِّكَ ٱلْحَقُّ كَمَنْ هُوَ أَعْمَىٰٓ ۚ إِنَّمَا يَتَذَكَّرُ أُو۟لُوا۟ ٱلْأَلْبَٰبِ ﴿١٩﴾\\
\textamh{20.\  } & ٱلَّذِينَ يُوفُونَ بِعَهْدِ ٱللَّهِ وَلَا يَنقُضُونَ ٱلْمِيثَـٰقَ ﴿٢٠﴾\\
\textamh{21.\  } & وَٱلَّذِينَ يَصِلُونَ مَآ أَمَرَ ٱللَّهُ بِهِۦٓ أَن يُوصَلَ وَيَخْشَوْنَ رَبَّهُمْ وَيَخَافُونَ سُوٓءَ ٱلْحِسَابِ ﴿٢١﴾\\
\textamh{22.\  } & وَٱلَّذِينَ صَبَرُوا۟ ٱبْتِغَآءَ وَجْهِ رَبِّهِمْ وَأَقَامُوا۟ ٱلصَّلَوٰةَ وَأَنفَقُوا۟ مِمَّا رَزَقْنَـٰهُمْ سِرًّۭا وَعَلَانِيَةًۭ وَيَدْرَءُونَ بِٱلْحَسَنَةِ ٱلسَّيِّئَةَ أُو۟لَـٰٓئِكَ لَهُمْ عُقْبَى ٱلدَّارِ ﴿٢٢﴾\\
\textamh{23.\  } & جَنَّـٰتُ عَدْنٍۢ يَدْخُلُونَهَا وَمَن صَلَحَ مِنْ ءَابَآئِهِمْ وَأَزْوَٟجِهِمْ وَذُرِّيَّٰتِهِمْ ۖ وَٱلْمَلَـٰٓئِكَةُ يَدْخُلُونَ عَلَيْهِم مِّن كُلِّ بَابٍۢ ﴿٢٣﴾\\
\textamh{24.\  } & سَلَـٰمٌ عَلَيْكُم بِمَا صَبَرْتُمْ ۚ فَنِعْمَ عُقْبَى ٱلدَّارِ ﴿٢٤﴾\\
\textamh{25.\  } & وَٱلَّذِينَ يَنقُضُونَ عَهْدَ ٱللَّهِ مِنۢ بَعْدِ مِيثَـٰقِهِۦ وَيَقْطَعُونَ مَآ أَمَرَ ٱللَّهُ بِهِۦٓ أَن يُوصَلَ وَيُفْسِدُونَ فِى ٱلْأَرْضِ ۙ أُو۟لَـٰٓئِكَ لَهُمُ ٱللَّعْنَةُ وَلَهُمْ سُوٓءُ ٱلدَّارِ ﴿٢٥﴾\\
\textamh{26.\  } & ٱللَّهُ يَبْسُطُ ٱلرِّزْقَ لِمَن يَشَآءُ وَيَقْدِرُ ۚ وَفَرِحُوا۟ بِٱلْحَيَوٰةِ ٱلدُّنْيَا وَمَا ٱلْحَيَوٰةُ ٱلدُّنْيَا فِى ٱلْءَاخِرَةِ إِلَّا مَتَـٰعٌۭ ﴿٢٦﴾\\
\textamh{27.\  } & وَيَقُولُ ٱلَّذِينَ كَفَرُوا۟ لَوْلَآ أُنزِلَ عَلَيْهِ ءَايَةٌۭ مِّن رَّبِّهِۦ ۗ قُلْ إِنَّ ٱللَّهَ يُضِلُّ مَن يَشَآءُ وَيَهْدِىٓ إِلَيْهِ مَنْ أَنَابَ ﴿٢٧﴾\\
\textamh{28.\  } & ٱلَّذِينَ ءَامَنُوا۟ وَتَطْمَئِنُّ قُلُوبُهُم بِذِكْرِ ٱللَّهِ ۗ أَلَا بِذِكْرِ ٱللَّهِ تَطْمَئِنُّ ٱلْقُلُوبُ ﴿٢٨﴾\\
\textamh{29.\  } & ٱلَّذِينَ ءَامَنُوا۟ وَعَمِلُوا۟ ٱلصَّـٰلِحَـٰتِ طُوبَىٰ لَهُمْ وَحُسْنُ مَـَٔابٍۢ ﴿٢٩﴾\\
\textamh{30.\  } & كَذَٟلِكَ أَرْسَلْنَـٰكَ فِىٓ أُمَّةٍۢ قَدْ خَلَتْ مِن قَبْلِهَآ أُمَمٌۭ لِّتَتْلُوَا۟ عَلَيْهِمُ ٱلَّذِىٓ أَوْحَيْنَآ إِلَيْكَ وَهُمْ يَكْفُرُونَ بِٱلرَّحْمَـٰنِ ۚ قُلْ هُوَ رَبِّى لَآ إِلَـٰهَ إِلَّا هُوَ عَلَيْهِ تَوَكَّلْتُ وَإِلَيْهِ مَتَابِ ﴿٣٠﴾\\
\textamh{31.\  } & وَلَوْ أَنَّ قُرْءَانًۭا سُيِّرَتْ بِهِ ٱلْجِبَالُ أَوْ قُطِّعَتْ بِهِ ٱلْأَرْضُ أَوْ كُلِّمَ بِهِ ٱلْمَوْتَىٰ ۗ بَل لِّلَّهِ ٱلْأَمْرُ جَمِيعًا ۗ أَفَلَمْ يَا۟يْـَٔسِ ٱلَّذِينَ ءَامَنُوٓا۟ أَن لَّوْ يَشَآءُ ٱللَّهُ لَهَدَى ٱلنَّاسَ جَمِيعًۭا ۗ وَلَا يَزَالُ ٱلَّذِينَ كَفَرُوا۟ تُصِيبُهُم بِمَا صَنَعُوا۟ قَارِعَةٌ أَوْ تَحُلُّ قَرِيبًۭا مِّن دَارِهِمْ حَتَّىٰ يَأْتِىَ وَعْدُ ٱللَّهِ ۚ إِنَّ ٱللَّهَ لَا يُخْلِفُ ٱلْمِيعَادَ ﴿٣١﴾\\
\textamh{32.\  } & وَلَقَدِ ٱسْتُهْزِئَ بِرُسُلٍۢ مِّن قَبْلِكَ فَأَمْلَيْتُ لِلَّذِينَ كَفَرُوا۟ ثُمَّ أَخَذْتُهُمْ ۖ فَكَيْفَ كَانَ عِقَابِ ﴿٣٢﴾\\
\textamh{33.\  } & أَفَمَنْ هُوَ قَآئِمٌ عَلَىٰ كُلِّ نَفْسٍۭ بِمَا كَسَبَتْ ۗ وَجَعَلُوا۟ لِلَّهِ شُرَكَآءَ قُلْ سَمُّوهُمْ ۚ أَمْ تُنَبِّـُٔونَهُۥ بِمَا لَا يَعْلَمُ فِى ٱلْأَرْضِ أَم بِظَـٰهِرٍۢ مِّنَ ٱلْقَوْلِ ۗ بَلْ زُيِّنَ لِلَّذِينَ كَفَرُوا۟ مَكْرُهُمْ وَصُدُّوا۟ عَنِ ٱلسَّبِيلِ ۗ وَمَن يُضْلِلِ ٱللَّهُ فَمَا لَهُۥ مِنْ هَادٍۢ ﴿٣٣﴾\\
\textamh{34.\  } & لَّهُمْ عَذَابٌۭ فِى ٱلْحَيَوٰةِ ٱلدُّنْيَا ۖ وَلَعَذَابُ ٱلْءَاخِرَةِ أَشَقُّ ۖ وَمَا لَهُم مِّنَ ٱللَّهِ مِن وَاقٍۢ ﴿٣٤﴾\\
\textamh{35.\  } & ۞ مَّثَلُ ٱلْجَنَّةِ ٱلَّتِى وُعِدَ ٱلْمُتَّقُونَ ۖ تَجْرِى مِن تَحْتِهَا ٱلْأَنْهَـٰرُ ۖ أُكُلُهَا دَآئِمٌۭ وَظِلُّهَا ۚ تِلْكَ عُقْبَى ٱلَّذِينَ ٱتَّقَوا۟ ۖ وَّعُقْبَى ٱلْكَـٰفِرِينَ ٱلنَّارُ ﴿٣٥﴾\\
\textamh{36.\  } & وَٱلَّذِينَ ءَاتَيْنَـٰهُمُ ٱلْكِتَـٰبَ يَفْرَحُونَ بِمَآ أُنزِلَ إِلَيْكَ ۖ وَمِنَ ٱلْأَحْزَابِ مَن يُنكِرُ بَعْضَهُۥ ۚ قُلْ إِنَّمَآ أُمِرْتُ أَنْ أَعْبُدَ ٱللَّهَ وَلَآ أُشْرِكَ بِهِۦٓ ۚ إِلَيْهِ أَدْعُوا۟ وَإِلَيْهِ مَـَٔابِ ﴿٣٦﴾\\
\textamh{37.\  } & وَكَذَٟلِكَ أَنزَلْنَـٰهُ حُكْمًا عَرَبِيًّۭا ۚ وَلَئِنِ ٱتَّبَعْتَ أَهْوَآءَهُم بَعْدَمَا جَآءَكَ مِنَ ٱلْعِلْمِ مَا لَكَ مِنَ ٱللَّهِ مِن وَلِىٍّۢ وَلَا وَاقٍۢ ﴿٣٧﴾\\
\textamh{38.\  } & وَلَقَدْ أَرْسَلْنَا رُسُلًۭا مِّن قَبْلِكَ وَجَعَلْنَا لَهُمْ أَزْوَٟجًۭا وَذُرِّيَّةًۭ ۚ وَمَا كَانَ لِرَسُولٍ أَن يَأْتِىَ بِـَٔايَةٍ إِلَّا بِإِذْنِ ٱللَّهِ ۗ لِكُلِّ أَجَلٍۢ كِتَابٌۭ ﴿٣٨﴾\\
\textamh{39.\  } & يَمْحُوا۟ ٱللَّهُ مَا يَشَآءُ وَيُثْبِتُ ۖ وَعِندَهُۥٓ أُمُّ ٱلْكِتَـٰبِ ﴿٣٩﴾\\
\textamh{40.\  } & وَإِن مَّا نُرِيَنَّكَ بَعْضَ ٱلَّذِى نَعِدُهُمْ أَوْ نَتَوَفَّيَنَّكَ فَإِنَّمَا عَلَيْكَ ٱلْبَلَـٰغُ وَعَلَيْنَا ٱلْحِسَابُ ﴿٤٠﴾\\
\textamh{41.\  } & أَوَلَمْ يَرَوْا۟ أَنَّا نَأْتِى ٱلْأَرْضَ نَنقُصُهَا مِنْ أَطْرَافِهَا ۚ وَٱللَّهُ يَحْكُمُ لَا مُعَقِّبَ لِحُكْمِهِۦ ۚ وَهُوَ سَرِيعُ ٱلْحِسَابِ ﴿٤١﴾\\
\textamh{42.\  } & وَقَدْ مَكَرَ ٱلَّذِينَ مِن قَبْلِهِمْ فَلِلَّهِ ٱلْمَكْرُ جَمِيعًۭا ۖ يَعْلَمُ مَا تَكْسِبُ كُلُّ نَفْسٍۢ ۗ وَسَيَعْلَمُ ٱلْكُفَّٰرُ لِمَنْ عُقْبَى ٱلدَّارِ ﴿٤٢﴾\\
\textamh{43.\  } & وَيَقُولُ ٱلَّذِينَ كَفَرُوا۟ لَسْتَ مُرْسَلًۭا ۚ قُلْ كَفَىٰ بِٱللَّهِ شَهِيدًۢا بَيْنِى وَبَيْنَكُمْ وَمَنْ عِندَهُۥ عِلْمُ ٱلْكِتَـٰبِ ﴿٤٣﴾\\
\end{longtable} \newpage

%% License: BSD style (Berkley) (i.e. Put the Copyright owner's name always)
%% Writer and Copyright (to): Bewketu(Bilal) Tadilo (2016-17)
\centering\section{\LR{\textamharic{ሱራቱ ኢብራሂም -}  \RL{سوره  ابراهيم}}}
\begin{longtable}{%
  @{}
    p{.5\textwidth}
  @{~~~~~~~~~~~~}
    p{.5\textwidth}
    @{}
}
\nopagebreak
\textamh{ቢስሚላሂ አራህመኒ ራሂይም } &  بِسْمِ ٱللَّهِ ٱلرَّحْمَـٰنِ ٱلرَّحِيمِ\\
\textamh{1.\  } &  الٓر ۚ كِتَـٰبٌ أَنزَلْنَـٰهُ إِلَيْكَ لِتُخْرِجَ ٱلنَّاسَ مِنَ ٱلظُّلُمَـٰتِ إِلَى ٱلنُّورِ بِإِذْنِ رَبِّهِمْ إِلَىٰ صِرَٰطِ ٱلْعَزِيزِ ٱلْحَمِيدِ ﴿١﴾\\
\textamh{2.\  } & ٱللَّهِ ٱلَّذِى لَهُۥ مَا فِى ٱلسَّمَـٰوَٟتِ وَمَا فِى ٱلْأَرْضِ ۗ وَوَيْلٌۭ لِّلْكَـٰفِرِينَ مِنْ عَذَابٍۢ شَدِيدٍ ﴿٢﴾\\
\textamh{3.\  } & ٱلَّذِينَ يَسْتَحِبُّونَ ٱلْحَيَوٰةَ ٱلدُّنْيَا عَلَى ٱلْءَاخِرَةِ وَيَصُدُّونَ عَن سَبِيلِ ٱللَّهِ وَيَبْغُونَهَا عِوَجًا ۚ أُو۟لَـٰٓئِكَ فِى ضَلَـٰلٍۭ بَعِيدٍۢ ﴿٣﴾\\
\textamh{4.\  } & وَمَآ أَرْسَلْنَا مِن رَّسُولٍ إِلَّا بِلِسَانِ قَوْمِهِۦ لِيُبَيِّنَ لَهُمْ ۖ فَيُضِلُّ ٱللَّهُ مَن يَشَآءُ وَيَهْدِى مَن يَشَآءُ ۚ وَهُوَ ٱلْعَزِيزُ ٱلْحَكِيمُ ﴿٤﴾\\
\textamh{5.\  } & وَلَقَدْ أَرْسَلْنَا مُوسَىٰ بِـَٔايَـٰتِنَآ أَنْ أَخْرِجْ قَوْمَكَ مِنَ ٱلظُّلُمَـٰتِ إِلَى ٱلنُّورِ وَذَكِّرْهُم بِأَيَّىٰمِ ٱللَّهِ ۚ إِنَّ فِى ذَٟلِكَ لَءَايَـٰتٍۢ لِّكُلِّ صَبَّارٍۢ شَكُورٍۢ ﴿٥﴾\\
\textamh{6.\  } & وَإِذْ قَالَ مُوسَىٰ لِقَوْمِهِ ٱذْكُرُوا۟ نِعْمَةَ ٱللَّهِ عَلَيْكُمْ إِذْ أَنجَىٰكُم مِّنْ ءَالِ فِرْعَوْنَ يَسُومُونَكُمْ سُوٓءَ ٱلْعَذَابِ وَيُذَبِّحُونَ أَبْنَآءَكُمْ وَيَسْتَحْيُونَ نِسَآءَكُمْ ۚ وَفِى ذَٟلِكُم بَلَآءٌۭ مِّن رَّبِّكُمْ عَظِيمٌۭ ﴿٦﴾\\
\textamh{7.\  } & وَإِذْ تَأَذَّنَ رَبُّكُمْ لَئِن شَكَرْتُمْ لَأَزِيدَنَّكُمْ ۖ وَلَئِن كَفَرْتُمْ إِنَّ عَذَابِى لَشَدِيدٌۭ ﴿٧﴾\\
\textamh{8.\  } & وَقَالَ مُوسَىٰٓ إِن تَكْفُرُوٓا۟ أَنتُمْ وَمَن فِى ٱلْأَرْضِ جَمِيعًۭا فَإِنَّ ٱللَّهَ لَغَنِىٌّ حَمِيدٌ ﴿٨﴾\\
\textamh{9.\  } & أَلَمْ يَأْتِكُمْ نَبَؤُا۟ ٱلَّذِينَ مِن قَبْلِكُمْ قَوْمِ نُوحٍۢ وَعَادٍۢ وَثَمُودَ ۛ وَٱلَّذِينَ مِنۢ بَعْدِهِمْ ۛ لَا يَعْلَمُهُمْ إِلَّا ٱللَّهُ ۚ جَآءَتْهُمْ رُسُلُهُم بِٱلْبَيِّنَـٰتِ فَرَدُّوٓا۟ أَيْدِيَهُمْ فِىٓ أَفْوَٟهِهِمْ وَقَالُوٓا۟ إِنَّا كَفَرْنَا بِمَآ أُرْسِلْتُم بِهِۦ وَإِنَّا لَفِى شَكٍّۢ مِّمَّا تَدْعُونَنَآ إِلَيْهِ مُرِيبٍۢ ﴿٩﴾\\
\textamh{10.\  } & ۞ قَالَتْ رُسُلُهُمْ أَفِى ٱللَّهِ شَكٌّۭ فَاطِرِ ٱلسَّمَـٰوَٟتِ وَٱلْأَرْضِ ۖ يَدْعُوكُمْ لِيَغْفِرَ لَكُم مِّن ذُنُوبِكُمْ وَيُؤَخِّرَكُمْ إِلَىٰٓ أَجَلٍۢ مُّسَمًّۭى ۚ قَالُوٓا۟ إِنْ أَنتُمْ إِلَّا بَشَرٌۭ مِّثْلُنَا تُرِيدُونَ أَن تَصُدُّونَا عَمَّا كَانَ يَعْبُدُ ءَابَآؤُنَا فَأْتُونَا بِسُلْطَٰنٍۢ مُّبِينٍۢ ﴿١٠﴾\\
\textamh{11.\  } & قَالَتْ لَهُمْ رُسُلُهُمْ إِن نَّحْنُ إِلَّا بَشَرٌۭ مِّثْلُكُمْ وَلَـٰكِنَّ ٱللَّهَ يَمُنُّ عَلَىٰ مَن يَشَآءُ مِنْ عِبَادِهِۦ ۖ وَمَا كَانَ لَنَآ أَن نَّأْتِيَكُم بِسُلْطَٰنٍ إِلَّا بِإِذْنِ ٱللَّهِ ۚ وَعَلَى ٱللَّهِ فَلْيَتَوَكَّلِ ٱلْمُؤْمِنُونَ ﴿١١﴾\\
\textamh{12.\  } & وَمَا لَنَآ أَلَّا نَتَوَكَّلَ عَلَى ٱللَّهِ وَقَدْ هَدَىٰنَا سُبُلَنَا ۚ وَلَنَصْبِرَنَّ عَلَىٰ مَآ ءَاذَيْتُمُونَا ۚ وَعَلَى ٱللَّهِ فَلْيَتَوَكَّلِ ٱلْمُتَوَكِّلُونَ ﴿١٢﴾\\
\textamh{13.\  } & وَقَالَ ٱلَّذِينَ كَفَرُوا۟ لِرُسُلِهِمْ لَنُخْرِجَنَّكُم مِّنْ أَرْضِنَآ أَوْ لَتَعُودُنَّ فِى مِلَّتِنَا ۖ فَأَوْحَىٰٓ إِلَيْهِمْ رَبُّهُمْ لَنُهْلِكَنَّ ٱلظَّـٰلِمِينَ ﴿١٣﴾\\
\textamh{14.\  } & وَلَنُسْكِنَنَّكُمُ ٱلْأَرْضَ مِنۢ بَعْدِهِمْ ۚ ذَٟلِكَ لِمَنْ خَافَ مَقَامِى وَخَافَ وَعِيدِ ﴿١٤﴾\\
\textamh{15.\  } & وَٱسْتَفْتَحُوا۟ وَخَابَ كُلُّ جَبَّارٍ عَنِيدٍۢ ﴿١٥﴾\\
\textamh{16.\  } & مِّن وَرَآئِهِۦ جَهَنَّمُ وَيُسْقَىٰ مِن مَّآءٍۢ صَدِيدٍۢ ﴿١٦﴾\\
\textamh{17.\  } & يَتَجَرَّعُهُۥ وَلَا يَكَادُ يُسِيغُهُۥ وَيَأْتِيهِ ٱلْمَوْتُ مِن كُلِّ مَكَانٍۢ وَمَا هُوَ بِمَيِّتٍۢ ۖ وَمِن وَرَآئِهِۦ عَذَابٌ غَلِيظٌۭ ﴿١٧﴾\\
\textamh{18.\  } & مَّثَلُ ٱلَّذِينَ كَفَرُوا۟ بِرَبِّهِمْ ۖ أَعْمَـٰلُهُمْ كَرَمَادٍ ٱشْتَدَّتْ بِهِ ٱلرِّيحُ فِى يَوْمٍ عَاصِفٍۢ ۖ لَّا يَقْدِرُونَ مِمَّا كَسَبُوا۟ عَلَىٰ شَىْءٍۢ ۚ ذَٟلِكَ هُوَ ٱلضَّلَـٰلُ ٱلْبَعِيدُ ﴿١٨﴾\\
\textamh{19.\  } & أَلَمْ تَرَ أَنَّ ٱللَّهَ خَلَقَ ٱلسَّمَـٰوَٟتِ وَٱلْأَرْضَ بِٱلْحَقِّ ۚ إِن يَشَأْ يُذْهِبْكُمْ وَيَأْتِ بِخَلْقٍۢ جَدِيدٍۢ ﴿١٩﴾\\
\textamh{20.\  } & وَمَا ذَٟلِكَ عَلَى ٱللَّهِ بِعَزِيزٍۢ ﴿٢٠﴾\\
\textamh{21.\  } & وَبَرَزُوا۟ لِلَّهِ جَمِيعًۭا فَقَالَ ٱلضُّعَفَـٰٓؤُا۟ لِلَّذِينَ ٱسْتَكْبَرُوٓا۟ إِنَّا كُنَّا لَكُمْ تَبَعًۭا فَهَلْ أَنتُم مُّغْنُونَ عَنَّا مِنْ عَذَابِ ٱللَّهِ مِن شَىْءٍۢ ۚ قَالُوا۟ لَوْ هَدَىٰنَا ٱللَّهُ لَهَدَيْنَـٰكُمْ ۖ سَوَآءٌ عَلَيْنَآ أَجَزِعْنَآ أَمْ صَبَرْنَا مَا لَنَا مِن مَّحِيصٍۢ ﴿٢١﴾\\
\textamh{22.\  } & وَقَالَ ٱلشَّيْطَٰنُ لَمَّا قُضِىَ ٱلْأَمْرُ إِنَّ ٱللَّهَ وَعَدَكُمْ وَعْدَ ٱلْحَقِّ وَوَعَدتُّكُمْ فَأَخْلَفْتُكُمْ ۖ وَمَا كَانَ لِىَ عَلَيْكُم مِّن سُلْطَٰنٍ إِلَّآ أَن دَعَوْتُكُمْ فَٱسْتَجَبْتُمْ لِى ۖ فَلَا تَلُومُونِى وَلُومُوٓا۟ أَنفُسَكُم ۖ مَّآ أَنَا۠ بِمُصْرِخِكُمْ وَمَآ أَنتُم بِمُصْرِخِىَّ ۖ إِنِّى كَفَرْتُ بِمَآ أَشْرَكْتُمُونِ مِن قَبْلُ ۗ إِنَّ ٱلظَّـٰلِمِينَ لَهُمْ عَذَابٌ أَلِيمٌۭ ﴿٢٢﴾\\
\textamh{23.\  } & وَأُدْخِلَ ٱلَّذِينَ ءَامَنُوا۟ وَعَمِلُوا۟ ٱلصَّـٰلِحَـٰتِ جَنَّـٰتٍۢ تَجْرِى مِن تَحْتِهَا ٱلْأَنْهَـٰرُ خَـٰلِدِينَ فِيهَا بِإِذْنِ رَبِّهِمْ ۖ تَحِيَّتُهُمْ فِيهَا سَلَـٰمٌ ﴿٢٣﴾\\
\textamh{24.\  } & أَلَمْ تَرَ كَيْفَ ضَرَبَ ٱللَّهُ مَثَلًۭا كَلِمَةًۭ طَيِّبَةًۭ كَشَجَرَةٍۢ طَيِّبَةٍ أَصْلُهَا ثَابِتٌۭ وَفَرْعُهَا فِى ٱلسَّمَآءِ ﴿٢٤﴾\\
\textamh{25.\  } & تُؤْتِىٓ أُكُلَهَا كُلَّ حِينٍۭ بِإِذْنِ رَبِّهَا ۗ وَيَضْرِبُ ٱللَّهُ ٱلْأَمْثَالَ لِلنَّاسِ لَعَلَّهُمْ يَتَذَكَّرُونَ ﴿٢٥﴾\\
\textamh{26.\  } & وَمَثَلُ كَلِمَةٍ خَبِيثَةٍۢ كَشَجَرَةٍ خَبِيثَةٍ ٱجْتُثَّتْ مِن فَوْقِ ٱلْأَرْضِ مَا لَهَا مِن قَرَارٍۢ ﴿٢٦﴾\\
\textamh{27.\  } & يُثَبِّتُ ٱللَّهُ ٱلَّذِينَ ءَامَنُوا۟ بِٱلْقَوْلِ ٱلثَّابِتِ فِى ٱلْحَيَوٰةِ ٱلدُّنْيَا وَفِى ٱلْءَاخِرَةِ ۖ وَيُضِلُّ ٱللَّهُ ٱلظَّـٰلِمِينَ ۚ وَيَفْعَلُ ٱللَّهُ مَا يَشَآءُ ﴿٢٧﴾\\
\textamh{28.\  } & ۞ أَلَمْ تَرَ إِلَى ٱلَّذِينَ بَدَّلُوا۟ نِعْمَتَ ٱللَّهِ كُفْرًۭا وَأَحَلُّوا۟ قَوْمَهُمْ دَارَ ٱلْبَوَارِ ﴿٢٨﴾\\
\textamh{29.\  } & جَهَنَّمَ يَصْلَوْنَهَا ۖ وَبِئْسَ ٱلْقَرَارُ ﴿٢٩﴾\\
\textamh{30.\  } & وَجَعَلُوا۟ لِلَّهِ أَندَادًۭا لِّيُضِلُّوا۟ عَن سَبِيلِهِۦ ۗ قُلْ تَمَتَّعُوا۟ فَإِنَّ مَصِيرَكُمْ إِلَى ٱلنَّارِ ﴿٣٠﴾\\
\textamh{31.\  } & قُل لِّعِبَادِىَ ٱلَّذِينَ ءَامَنُوا۟ يُقِيمُوا۟ ٱلصَّلَوٰةَ وَيُنفِقُوا۟ مِمَّا رَزَقْنَـٰهُمْ سِرًّۭا وَعَلَانِيَةًۭ مِّن قَبْلِ أَن يَأْتِىَ يَوْمٌۭ لَّا بَيْعٌۭ فِيهِ وَلَا خِلَـٰلٌ ﴿٣١﴾\\
\textamh{32.\  } & ٱللَّهُ ٱلَّذِى خَلَقَ ٱلسَّمَـٰوَٟتِ وَٱلْأَرْضَ وَأَنزَلَ مِنَ ٱلسَّمَآءِ مَآءًۭ فَأَخْرَجَ بِهِۦ مِنَ ٱلثَّمَرَٰتِ رِزْقًۭا لَّكُمْ ۖ وَسَخَّرَ لَكُمُ ٱلْفُلْكَ لِتَجْرِىَ فِى ٱلْبَحْرِ بِأَمْرِهِۦ ۖ وَسَخَّرَ لَكُمُ ٱلْأَنْهَـٰرَ ﴿٣٢﴾\\
\textamh{33.\  } & وَسَخَّرَ لَكُمُ ٱلشَّمْسَ وَٱلْقَمَرَ دَآئِبَيْنِ ۖ وَسَخَّرَ لَكُمُ ٱلَّيْلَ وَٱلنَّهَارَ ﴿٣٣﴾\\
\textamh{34.\  } & وَءَاتَىٰكُم مِّن كُلِّ مَا سَأَلْتُمُوهُ ۚ وَإِن تَعُدُّوا۟ نِعْمَتَ ٱللَّهِ لَا تُحْصُوهَآ ۗ إِنَّ ٱلْإِنسَـٰنَ لَظَلُومٌۭ كَفَّارٌۭ ﴿٣٤﴾\\
\textamh{35.\  } & وَإِذْ قَالَ إِبْرَٰهِيمُ رَبِّ ٱجْعَلْ هَـٰذَا ٱلْبَلَدَ ءَامِنًۭا وَٱجْنُبْنِى وَبَنِىَّ أَن نَّعْبُدَ ٱلْأَصْنَامَ ﴿٣٥﴾\\
\textamh{36.\  } & رَبِّ إِنَّهُنَّ أَضْلَلْنَ كَثِيرًۭا مِّنَ ٱلنَّاسِ ۖ فَمَن تَبِعَنِى فَإِنَّهُۥ مِنِّى ۖ وَمَنْ عَصَانِى فَإِنَّكَ غَفُورٌۭ رَّحِيمٌۭ ﴿٣٦﴾\\
\textamh{37.\  } & رَّبَّنَآ إِنِّىٓ أَسْكَنتُ مِن ذُرِّيَّتِى بِوَادٍ غَيْرِ ذِى زَرْعٍ عِندَ بَيْتِكَ ٱلْمُحَرَّمِ رَبَّنَا لِيُقِيمُوا۟ ٱلصَّلَوٰةَ فَٱجْعَلْ أَفْـِٔدَةًۭ مِّنَ ٱلنَّاسِ تَهْوِىٓ إِلَيْهِمْ وَٱرْزُقْهُم مِّنَ ٱلثَّمَرَٰتِ لَعَلَّهُمْ يَشْكُرُونَ ﴿٣٧﴾\\
\textamh{38.\  } & رَبَّنَآ إِنَّكَ تَعْلَمُ مَا نُخْفِى وَمَا نُعْلِنُ ۗ وَمَا يَخْفَىٰ عَلَى ٱللَّهِ مِن شَىْءٍۢ فِى ٱلْأَرْضِ وَلَا فِى ٱلسَّمَآءِ ﴿٣٨﴾\\
\textamh{39.\  } & ٱلْحَمْدُ لِلَّهِ ٱلَّذِى وَهَبَ لِى عَلَى ٱلْكِبَرِ إِسْمَـٰعِيلَ وَإِسْحَـٰقَ ۚ إِنَّ رَبِّى لَسَمِيعُ ٱلدُّعَآءِ ﴿٣٩﴾\\
\textamh{40.\  } & رَبِّ ٱجْعَلْنِى مُقِيمَ ٱلصَّلَوٰةِ وَمِن ذُرِّيَّتِى ۚ رَبَّنَا وَتَقَبَّلْ دُعَآءِ ﴿٤٠﴾\\
\textamh{41.\  } & رَبَّنَا ٱغْفِرْ لِى وَلِوَٟلِدَىَّ وَلِلْمُؤْمِنِينَ يَوْمَ يَقُومُ ٱلْحِسَابُ ﴿٤١﴾\\
\textamh{42.\  } & وَلَا تَحْسَبَنَّ ٱللَّهَ غَٰفِلًا عَمَّا يَعْمَلُ ٱلظَّـٰلِمُونَ ۚ إِنَّمَا يُؤَخِّرُهُمْ لِيَوْمٍۢ تَشْخَصُ فِيهِ ٱلْأَبْصَـٰرُ ﴿٤٢﴾\\
\textamh{43.\  } & مُهْطِعِينَ مُقْنِعِى رُءُوسِهِمْ لَا يَرْتَدُّ إِلَيْهِمْ طَرْفُهُمْ ۖ وَأَفْـِٔدَتُهُمْ هَوَآءٌۭ ﴿٤٣﴾\\
\textamh{44.\  } & وَأَنذِرِ ٱلنَّاسَ يَوْمَ يَأْتِيهِمُ ٱلْعَذَابُ فَيَقُولُ ٱلَّذِينَ ظَلَمُوا۟ رَبَّنَآ أَخِّرْنَآ إِلَىٰٓ أَجَلٍۢ قَرِيبٍۢ نُّجِبْ دَعْوَتَكَ وَنَتَّبِعِ ٱلرُّسُلَ ۗ أَوَلَمْ تَكُونُوٓا۟ أَقْسَمْتُم مِّن قَبْلُ مَا لَكُم مِّن زَوَالٍۢ ﴿٤٤﴾\\
\textamh{45.\  } & وَسَكَنتُمْ فِى مَسَـٰكِنِ ٱلَّذِينَ ظَلَمُوٓا۟ أَنفُسَهُمْ وَتَبَيَّنَ لَكُمْ كَيْفَ فَعَلْنَا بِهِمْ وَضَرَبْنَا لَكُمُ ٱلْأَمْثَالَ ﴿٤٥﴾\\
\textamh{46.\  } & وَقَدْ مَكَرُوا۟ مَكْرَهُمْ وَعِندَ ٱللَّهِ مَكْرُهُمْ وَإِن كَانَ مَكْرُهُمْ لِتَزُولَ مِنْهُ ٱلْجِبَالُ ﴿٤٦﴾\\
\textamh{47.\  } & فَلَا تَحْسَبَنَّ ٱللَّهَ مُخْلِفَ وَعْدِهِۦ رُسُلَهُۥٓ ۗ إِنَّ ٱللَّهَ عَزِيزٌۭ ذُو ٱنتِقَامٍۢ ﴿٤٧﴾\\
\textamh{48.\  } & يَوْمَ تُبَدَّلُ ٱلْأَرْضُ غَيْرَ ٱلْأَرْضِ وَٱلسَّمَـٰوَٟتُ ۖ وَبَرَزُوا۟ لِلَّهِ ٱلْوَٟحِدِ ٱلْقَهَّارِ ﴿٤٨﴾\\
\textamh{49.\  } & وَتَرَى ٱلْمُجْرِمِينَ يَوْمَئِذٍۢ مُّقَرَّنِينَ فِى ٱلْأَصْفَادِ ﴿٤٩﴾\\
\textamh{50.\  } & سَرَابِيلُهُم مِّن قَطِرَانٍۢ وَتَغْشَىٰ وُجُوهَهُمُ ٱلنَّارُ ﴿٥٠﴾\\
\textamh{51.\  } & لِيَجْزِىَ ٱللَّهُ كُلَّ نَفْسٍۢ مَّا كَسَبَتْ ۚ إِنَّ ٱللَّهَ سَرِيعُ ٱلْحِسَابِ ﴿٥١﴾\\
\textamh{52.\  } & هَـٰذَا بَلَـٰغٌۭ لِّلنَّاسِ وَلِيُنذَرُوا۟ بِهِۦ وَلِيَعْلَمُوٓا۟ أَنَّمَا هُوَ إِلَـٰهٌۭ وَٟحِدٌۭ وَلِيَذَّكَّرَ أُو۟لُوا۟ ٱلْأَلْبَٰبِ ﴿٥٢﴾\\
\end{longtable}
\clearpage
%% License: BSD style (Berkley) (i.e. Put the Copyright owner's name always)
%% Writer and Copyright (to): Bewketu(Bilal) Tadilo (2016-17)
\begin{center}\section{\LR{\textamhsec{ሱራቱ አልሂጅር -}  \textarabic{سوره  الحجر}}}\end{center}
\begin{longtable}{%
  @{}
    p{.5\textwidth}
  @{~~~}
    p{.5\textwidth}
    @{}
}
\textamh{ቢስሚላሂ አራህመኒ ራሂይም } &  \mytextarabic{بِسْمِ ٱللَّهِ ٱلرَّحْمَـٰنِ ٱلرَّحِيمِ}\\
\textamh{1.\  } & \mytextarabic{ الٓر ۚ تِلْكَ ءَايَـٰتُ ٱلْكِتَـٰبِ وَقُرْءَانٍۢ مُّبِينٍۢ ﴿١﴾}\\
\textamh{2.\  } & \mytextarabic{رُّبَمَا يَوَدُّ ٱلَّذِينَ كَفَرُوا۟ لَوْ كَانُوا۟ مُسْلِمِينَ ﴿٢﴾}\\
\textamh{3.\  } & \mytextarabic{ذَرْهُمْ يَأْكُلُوا۟ وَيَتَمَتَّعُوا۟ وَيُلْهِهِمُ ٱلْأَمَلُ ۖ فَسَوْفَ يَعْلَمُونَ ﴿٣﴾}\\
\textamh{4.\  } & \mytextarabic{وَمَآ أَهْلَكْنَا مِن قَرْيَةٍ إِلَّا وَلَهَا كِتَابٌۭ مَّعْلُومٌۭ ﴿٤﴾}\\
\textamh{5.\  } & \mytextarabic{مَّا تَسْبِقُ مِنْ أُمَّةٍ أَجَلَهَا وَمَا يَسْتَـْٔخِرُونَ ﴿٥﴾}\\
\textamh{6.\  } & \mytextarabic{وَقَالُوا۟ يَـٰٓأَيُّهَا ٱلَّذِى نُزِّلَ عَلَيْهِ ٱلذِّكْرُ إِنَّكَ لَمَجْنُونٌۭ ﴿٦﴾}\\
\textamh{7.\  } & \mytextarabic{لَّوْ مَا تَأْتِينَا بِٱلْمَلَـٰٓئِكَةِ إِن كُنتَ مِنَ ٱلصَّـٰدِقِينَ ﴿٧﴾}\\
\textamh{8.\  } & \mytextarabic{مَا نُنَزِّلُ ٱلْمَلَـٰٓئِكَةَ إِلَّا بِٱلْحَقِّ وَمَا كَانُوٓا۟ إِذًۭا مُّنظَرِينَ ﴿٨﴾}\\
\textamh{9.\  } & \mytextarabic{إِنَّا نَحْنُ نَزَّلْنَا ٱلذِّكْرَ وَإِنَّا لَهُۥ لَحَـٰفِظُونَ ﴿٩﴾}\\
\textamh{10.\  } & \mytextarabic{وَلَقَدْ أَرْسَلْنَا مِن قَبْلِكَ فِى شِيَعِ ٱلْأَوَّلِينَ ﴿١٠﴾}\\
\textamh{11.\  } & \mytextarabic{وَمَا يَأْتِيهِم مِّن رَّسُولٍ إِلَّا كَانُوا۟ بِهِۦ يَسْتَهْزِءُونَ ﴿١١﴾}\\
\textamh{12.\  } & \mytextarabic{كَذَٟلِكَ نَسْلُكُهُۥ فِى قُلُوبِ ٱلْمُجْرِمِينَ ﴿١٢﴾}\\
\textamh{13.\  } & \mytextarabic{لَا يُؤْمِنُونَ بِهِۦ ۖ وَقَدْ خَلَتْ سُنَّةُ ٱلْأَوَّلِينَ ﴿١٣﴾}\\
\textamh{14.\  } & \mytextarabic{وَلَوْ فَتَحْنَا عَلَيْهِم بَابًۭا مِّنَ ٱلسَّمَآءِ فَظَلُّوا۟ فِيهِ يَعْرُجُونَ ﴿١٤﴾}\\
\textamh{15.\  } & \mytextarabic{لَقَالُوٓا۟ إِنَّمَا سُكِّرَتْ أَبْصَـٰرُنَا بَلْ نَحْنُ قَوْمٌۭ مَّسْحُورُونَ ﴿١٥﴾}\\
\textamh{16.\  } & \mytextarabic{وَلَقَدْ جَعَلْنَا فِى ٱلسَّمَآءِ بُرُوجًۭا وَزَيَّنَّـٰهَا لِلنَّـٰظِرِينَ ﴿١٦﴾}\\
\textamh{17.\  } & \mytextarabic{وَحَفِظْنَـٰهَا مِن كُلِّ شَيْطَٰنٍۢ رَّجِيمٍ ﴿١٧﴾}\\
\textamh{18.\  } & \mytextarabic{إِلَّا مَنِ ٱسْتَرَقَ ٱلسَّمْعَ فَأَتْبَعَهُۥ شِهَابٌۭ مُّبِينٌۭ ﴿١٨﴾}\\
\textamh{19.\  } & \mytextarabic{وَٱلْأَرْضَ مَدَدْنَـٰهَا وَأَلْقَيْنَا فِيهَا رَوَٟسِىَ وَأَنۢبَتْنَا فِيهَا مِن كُلِّ شَىْءٍۢ مَّوْزُونٍۢ ﴿١٩﴾}\\
\textamh{20.\  } & \mytextarabic{وَجَعَلْنَا لَكُمْ فِيهَا مَعَـٰيِشَ وَمَن لَّسْتُمْ لَهُۥ بِرَٰزِقِينَ ﴿٢٠﴾}\\
\textamh{21.\  } & \mytextarabic{وَإِن مِّن شَىْءٍ إِلَّا عِندَنَا خَزَآئِنُهُۥ وَمَا نُنَزِّلُهُۥٓ إِلَّا بِقَدَرٍۢ مَّعْلُومٍۢ ﴿٢١﴾}\\
\textamh{22.\  } & \mytextarabic{وَأَرْسَلْنَا ٱلرِّيَـٰحَ لَوَٟقِحَ فَأَنزَلْنَا مِنَ ٱلسَّمَآءِ مَآءًۭ فَأَسْقَيْنَـٰكُمُوهُ وَمَآ أَنتُمْ لَهُۥ بِخَـٰزِنِينَ ﴿٢٢﴾}\\
\textamh{23.\  } & \mytextarabic{وَإِنَّا لَنَحْنُ نُحْىِۦ وَنُمِيتُ وَنَحْنُ ٱلْوَٟرِثُونَ ﴿٢٣﴾}\\
\textamh{24.\  } & \mytextarabic{وَلَقَدْ عَلِمْنَا ٱلْمُسْتَقْدِمِينَ مِنكُمْ وَلَقَدْ عَلِمْنَا ٱلْمُسْتَـْٔخِرِينَ ﴿٢٤﴾}\\
\textamh{25.\  } & \mytextarabic{وَإِنَّ رَبَّكَ هُوَ يَحْشُرُهُمْ ۚ إِنَّهُۥ حَكِيمٌ عَلِيمٌۭ ﴿٢٥﴾}\\
\textamh{26.\  } & \mytextarabic{وَلَقَدْ خَلَقْنَا ٱلْإِنسَـٰنَ مِن صَلْصَـٰلٍۢ مِّنْ حَمَإٍۢ مَّسْنُونٍۢ ﴿٢٦﴾}\\
\textamh{27.\  } & \mytextarabic{وَٱلْجَآنَّ خَلَقْنَـٰهُ مِن قَبْلُ مِن نَّارِ ٱلسَّمُومِ ﴿٢٧﴾}\\
\textamh{28.\  } & \mytextarabic{وَإِذْ قَالَ رَبُّكَ لِلْمَلَـٰٓئِكَةِ إِنِّى خَـٰلِقٌۢ بَشَرًۭا مِّن صَلْصَـٰلٍۢ مِّنْ حَمَإٍۢ مَّسْنُونٍۢ ﴿٢٨﴾}\\
\textamh{29.\  } & \mytextarabic{فَإِذَا سَوَّيْتُهُۥ وَنَفَخْتُ فِيهِ مِن رُّوحِى فَقَعُوا۟ لَهُۥ سَـٰجِدِينَ ﴿٢٩﴾}\\
\textamh{30.\  } & \mytextarabic{فَسَجَدَ ٱلْمَلَـٰٓئِكَةُ كُلُّهُمْ أَجْمَعُونَ ﴿٣٠﴾}\\
\textamh{31.\  } & \mytextarabic{إِلَّآ إِبْلِيسَ أَبَىٰٓ أَن يَكُونَ مَعَ ٱلسَّٰجِدِينَ ﴿٣١﴾}\\
\textamh{32.\  } & \mytextarabic{قَالَ يَـٰٓإِبْلِيسُ مَا لَكَ أَلَّا تَكُونَ مَعَ ٱلسَّٰجِدِينَ ﴿٣٢﴾}\\
\textamh{33.\  } & \mytextarabic{قَالَ لَمْ أَكُن لِّأَسْجُدَ لِبَشَرٍ خَلَقْتَهُۥ مِن صَلْصَـٰلٍۢ مِّنْ حَمَإٍۢ مَّسْنُونٍۢ ﴿٣٣﴾}\\
\textamh{34.\  } & \mytextarabic{قَالَ فَٱخْرُجْ مِنْهَا فَإِنَّكَ رَجِيمٌۭ ﴿٣٤﴾}\\
\textamh{35.\  } & \mytextarabic{وَإِنَّ عَلَيْكَ ٱللَّعْنَةَ إِلَىٰ يَوْمِ ٱلدِّينِ ﴿٣٥﴾}\\
\textamh{36.\  } & \mytextarabic{قَالَ رَبِّ فَأَنظِرْنِىٓ إِلَىٰ يَوْمِ يُبْعَثُونَ ﴿٣٦﴾}\\
\textamh{37.\  } & \mytextarabic{قَالَ فَإِنَّكَ مِنَ ٱلْمُنظَرِينَ ﴿٣٧﴾}\\
\textamh{38.\  } & \mytextarabic{إِلَىٰ يَوْمِ ٱلْوَقْتِ ٱلْمَعْلُومِ ﴿٣٨﴾}\\
\textamh{39.\  } & \mytextarabic{قَالَ رَبِّ بِمَآ أَغْوَيْتَنِى لَأُزَيِّنَنَّ لَهُمْ فِى ٱلْأَرْضِ وَلَأُغْوِيَنَّهُمْ أَجْمَعِينَ ﴿٣٩﴾}\\
\textamh{40.\  } & \mytextarabic{إِلَّا عِبَادَكَ مِنْهُمُ ٱلْمُخْلَصِينَ ﴿٤٠﴾}\\
\textamh{41.\  } & \mytextarabic{قَالَ هَـٰذَا صِرَٰطٌ عَلَىَّ مُسْتَقِيمٌ ﴿٤١﴾}\\
\textamh{42.\  } & \mytextarabic{إِنَّ عِبَادِى لَيْسَ لَكَ عَلَيْهِمْ سُلْطَٰنٌ إِلَّا مَنِ ٱتَّبَعَكَ مِنَ ٱلْغَاوِينَ ﴿٤٢﴾}\\
\textamh{43.\  } & \mytextarabic{وَإِنَّ جَهَنَّمَ لَمَوْعِدُهُمْ أَجْمَعِينَ ﴿٤٣﴾}\\
\textamh{44.\  } & \mytextarabic{لَهَا سَبْعَةُ أَبْوَٟبٍۢ لِّكُلِّ بَابٍۢ مِّنْهُمْ جُزْءٌۭ مَّقْسُومٌ ﴿٤٤﴾}\\
\textamh{45.\  } & \mytextarabic{إِنَّ ٱلْمُتَّقِينَ فِى جَنَّـٰتٍۢ وَعُيُونٍ ﴿٤٥﴾}\\
\textamh{46.\  } & \mytextarabic{ٱدْخُلُوهَا بِسَلَـٰمٍ ءَامِنِينَ ﴿٤٦﴾}\\
\textamh{47.\  } & \mytextarabic{وَنَزَعْنَا مَا فِى صُدُورِهِم مِّنْ غِلٍّ إِخْوَٟنًا عَلَىٰ سُرُرٍۢ مُّتَقَـٰبِلِينَ ﴿٤٧﴾}\\
\textamh{48.\  } & \mytextarabic{لَا يَمَسُّهُمْ فِيهَا نَصَبٌۭ وَمَا هُم مِّنْهَا بِمُخْرَجِينَ ﴿٤٨﴾}\\
\textamh{49.\  } & \mytextarabic{۞ نَبِّئْ عِبَادِىٓ أَنِّىٓ أَنَا ٱلْغَفُورُ ٱلرَّحِيمُ ﴿٤٩﴾}\\
\textamh{50.\  } & \mytextarabic{وَأَنَّ عَذَابِى هُوَ ٱلْعَذَابُ ٱلْأَلِيمُ ﴿٥٠﴾}\\
\textamh{51.\  } & \mytextarabic{وَنَبِّئْهُمْ عَن ضَيْفِ إِبْرَٰهِيمَ ﴿٥١﴾}\\
\textamh{52.\  } & \mytextarabic{إِذْ دَخَلُوا۟ عَلَيْهِ فَقَالُوا۟ سَلَـٰمًۭا قَالَ إِنَّا مِنكُمْ وَجِلُونَ ﴿٥٢﴾}\\
\textamh{53.\  } & \mytextarabic{قَالُوا۟ لَا تَوْجَلْ إِنَّا نُبَشِّرُكَ بِغُلَـٰمٍ عَلِيمٍۢ ﴿٥٣﴾}\\
\textamh{54.\  } & \mytextarabic{قَالَ أَبَشَّرْتُمُونِى عَلَىٰٓ أَن مَّسَّنِىَ ٱلْكِبَرُ فَبِمَ تُبَشِّرُونَ ﴿٥٤﴾}\\
\textamh{55.\  } & \mytextarabic{قَالُوا۟ بَشَّرْنَـٰكَ بِٱلْحَقِّ فَلَا تَكُن مِّنَ ٱلْقَـٰنِطِينَ ﴿٥٥﴾}\\
\textamh{56.\  } & \mytextarabic{قَالَ وَمَن يَقْنَطُ مِن رَّحْمَةِ رَبِّهِۦٓ إِلَّا ٱلضَّآلُّونَ ﴿٥٦﴾}\\
\textamh{57.\  } & \mytextarabic{قَالَ فَمَا خَطْبُكُمْ أَيُّهَا ٱلْمُرْسَلُونَ ﴿٥٧﴾}\\
\textamh{58.\  } & \mytextarabic{قَالُوٓا۟ إِنَّآ أُرْسِلْنَآ إِلَىٰ قَوْمٍۢ مُّجْرِمِينَ ﴿٥٨﴾}\\
\textamh{59.\  } & \mytextarabic{إِلَّآ ءَالَ لُوطٍ إِنَّا لَمُنَجُّوهُمْ أَجْمَعِينَ ﴿٥٩﴾}\\
\textamh{60.\  } & \mytextarabic{إِلَّا ٱمْرَأَتَهُۥ قَدَّرْنَآ ۙ إِنَّهَا لَمِنَ ٱلْغَٰبِرِينَ ﴿٦٠﴾}\\
\textamh{61.\  } & \mytextarabic{فَلَمَّا جَآءَ ءَالَ لُوطٍ ٱلْمُرْسَلُونَ ﴿٦١﴾}\\
\textamh{62.\  } & \mytextarabic{قَالَ إِنَّكُمْ قَوْمٌۭ مُّنكَرُونَ ﴿٦٢﴾}\\
\textamh{63.\  } & \mytextarabic{قَالُوا۟ بَلْ جِئْنَـٰكَ بِمَا كَانُوا۟ فِيهِ يَمْتَرُونَ ﴿٦٣﴾}\\
\textamh{64.\  } & \mytextarabic{وَأَتَيْنَـٰكَ بِٱلْحَقِّ وَإِنَّا لَصَـٰدِقُونَ ﴿٦٤﴾}\\
\textamh{65.\  } & \mytextarabic{فَأَسْرِ بِأَهْلِكَ بِقِطْعٍۢ مِّنَ ٱلَّيْلِ وَٱتَّبِعْ أَدْبَٰرَهُمْ وَلَا يَلْتَفِتْ مِنكُمْ أَحَدٌۭ وَٱمْضُوا۟ حَيْثُ تُؤْمَرُونَ ﴿٦٥﴾}\\
\textamh{66.\  } & \mytextarabic{وَقَضَيْنَآ إِلَيْهِ ذَٟلِكَ ٱلْأَمْرَ أَنَّ دَابِرَ هَـٰٓؤُلَآءِ مَقْطُوعٌۭ مُّصْبِحِينَ ﴿٦٦﴾}\\
\textamh{67.\  } & \mytextarabic{وَجَآءَ أَهْلُ ٱلْمَدِينَةِ يَسْتَبْشِرُونَ ﴿٦٧﴾}\\
\textamh{68.\  } & \mytextarabic{قَالَ إِنَّ هَـٰٓؤُلَآءِ ضَيْفِى فَلَا تَفْضَحُونِ ﴿٦٨﴾}\\
\textamh{69.\  } & \mytextarabic{وَٱتَّقُوا۟ ٱللَّهَ وَلَا تُخْزُونِ ﴿٦٩﴾}\\
\textamh{70.\  } & \mytextarabic{قَالُوٓا۟ أَوَلَمْ نَنْهَكَ عَنِ ٱلْعَـٰلَمِينَ ﴿٧٠﴾}\\
\textamh{71.\  } & \mytextarabic{قَالَ هَـٰٓؤُلَآءِ بَنَاتِىٓ إِن كُنتُمْ فَـٰعِلِينَ ﴿٧١﴾}\\
\textamh{72.\  } & \mytextarabic{لَعَمْرُكَ إِنَّهُمْ لَفِى سَكْرَتِهِمْ يَعْمَهُونَ ﴿٧٢﴾}\\
\textamh{73.\  } & \mytextarabic{فَأَخَذَتْهُمُ ٱلصَّيْحَةُ مُشْرِقِينَ ﴿٧٣﴾}\\
\textamh{74.\  } & \mytextarabic{فَجَعَلْنَا عَـٰلِيَهَا سَافِلَهَا وَأَمْطَرْنَا عَلَيْهِمْ حِجَارَةًۭ مِّن سِجِّيلٍ ﴿٧٤﴾}\\
\textamh{75.\  } & \mytextarabic{إِنَّ فِى ذَٟلِكَ لَءَايَـٰتٍۢ لِّلْمُتَوَسِّمِينَ ﴿٧٥﴾}\\
\textamh{76.\  } & \mytextarabic{وَإِنَّهَا لَبِسَبِيلٍۢ مُّقِيمٍ ﴿٧٦﴾}\\
\textamh{77.\  } & \mytextarabic{إِنَّ فِى ذَٟلِكَ لَءَايَةًۭ لِّلْمُؤْمِنِينَ ﴿٧٧﴾}\\
\textamh{78.\  } & \mytextarabic{وَإِن كَانَ أَصْحَـٰبُ ٱلْأَيْكَةِ لَظَـٰلِمِينَ ﴿٧٨﴾}\\
\textamh{79.\  } & \mytextarabic{فَٱنتَقَمْنَا مِنْهُمْ وَإِنَّهُمَا لَبِإِمَامٍۢ مُّبِينٍۢ ﴿٧٩﴾}\\
\textamh{80.\  } & \mytextarabic{وَلَقَدْ كَذَّبَ أَصْحَـٰبُ ٱلْحِجْرِ ٱلْمُرْسَلِينَ ﴿٨٠﴾}\\
\textamh{81.\  } & \mytextarabic{وَءَاتَيْنَـٰهُمْ ءَايَـٰتِنَا فَكَانُوا۟ عَنْهَا مُعْرِضِينَ ﴿٨١﴾}\\
\textamh{82.\  } & \mytextarabic{وَكَانُوا۟ يَنْحِتُونَ مِنَ ٱلْجِبَالِ بُيُوتًا ءَامِنِينَ ﴿٨٢﴾}\\
\textamh{83.\  } & \mytextarabic{فَأَخَذَتْهُمُ ٱلصَّيْحَةُ مُصْبِحِينَ ﴿٨٣﴾}\\
\textamh{84.\  } & \mytextarabic{فَمَآ أَغْنَىٰ عَنْهُم مَّا كَانُوا۟ يَكْسِبُونَ ﴿٨٤﴾}\\
\textamh{85.\  } & \mytextarabic{وَمَا خَلَقْنَا ٱلسَّمَـٰوَٟتِ وَٱلْأَرْضَ وَمَا بَيْنَهُمَآ إِلَّا بِٱلْحَقِّ ۗ وَإِنَّ ٱلسَّاعَةَ لَءَاتِيَةٌۭ ۖ فَٱصْفَحِ ٱلصَّفْحَ ٱلْجَمِيلَ ﴿٨٥﴾}\\
\textamh{86.\  } & \mytextarabic{إِنَّ رَبَّكَ هُوَ ٱلْخَلَّٰقُ ٱلْعَلِيمُ ﴿٨٦﴾}\\
\textamh{87.\  } & \mytextarabic{وَلَقَدْ ءَاتَيْنَـٰكَ سَبْعًۭا مِّنَ ٱلْمَثَانِى وَٱلْقُرْءَانَ ٱلْعَظِيمَ ﴿٨٧﴾}\\
\textamh{88.\  } & \mytextarabic{لَا تَمُدَّنَّ عَيْنَيْكَ إِلَىٰ مَا مَتَّعْنَا بِهِۦٓ أَزْوَٟجًۭا مِّنْهُمْ وَلَا تَحْزَنْ عَلَيْهِمْ وَٱخْفِضْ جَنَاحَكَ لِلْمُؤْمِنِينَ ﴿٨٨﴾}\\
\textamh{89.\  } & \mytextarabic{وَقُلْ إِنِّىٓ أَنَا ٱلنَّذِيرُ ٱلْمُبِينُ ﴿٨٩﴾}\\
\textamh{90.\  } & \mytextarabic{كَمَآ أَنزَلْنَا عَلَى ٱلْمُقْتَسِمِينَ ﴿٩٠﴾}\\
\textamh{91.\  } & \mytextarabic{ٱلَّذِينَ جَعَلُوا۟ ٱلْقُرْءَانَ عِضِينَ ﴿٩١﴾}\\
\textamh{92.\  } & \mytextarabic{فَوَرَبِّكَ لَنَسْـَٔلَنَّهُمْ أَجْمَعِينَ ﴿٩٢﴾}\\
\textamh{93.\  } & \mytextarabic{عَمَّا كَانُوا۟ يَعْمَلُونَ ﴿٩٣﴾}\\
\textamh{94.\  } & \mytextarabic{فَٱصْدَعْ بِمَا تُؤْمَرُ وَأَعْرِضْ عَنِ ٱلْمُشْرِكِينَ ﴿٩٤﴾}\\
\textamh{95.\  } & \mytextarabic{إِنَّا كَفَيْنَـٰكَ ٱلْمُسْتَهْزِءِينَ ﴿٩٥﴾}\\
\textamh{96.\  } & \mytextarabic{ٱلَّذِينَ يَجْعَلُونَ مَعَ ٱللَّهِ إِلَـٰهًا ءَاخَرَ ۚ فَسَوْفَ يَعْلَمُونَ ﴿٩٦﴾}\\
\textamh{97.\  } & \mytextarabic{وَلَقَدْ نَعْلَمُ أَنَّكَ يَضِيقُ صَدْرُكَ بِمَا يَقُولُونَ ﴿٩٧﴾}\\
\textamh{98.\  } & \mytextarabic{فَسَبِّحْ بِحَمْدِ رَبِّكَ وَكُن مِّنَ ٱلسَّٰجِدِينَ ﴿٩٨﴾}\\
\textamh{99.\  } & \mytextarabic{وَٱعْبُدْ رَبَّكَ حَتَّىٰ يَأْتِيَكَ ٱلْيَقِينُ ﴿٩٩﴾}\\
\end{longtable}
\clearpage
%% License: BSD style (Berkley) (i.e. Put the Copyright owner's name always)
%% Writer and Copyright (to): Bewketu(Bilal) Tadilo (2016-17)
\centering\section{\LR{\textamharic{ሱራቱ አንነህል -}  \RL{سوره  النحل}}}
\begin{longtable}{%
  @{}
    p{.5\textwidth}
  @{~~~~~~~~~~~~~}
    p{.5\textwidth}
    @{}
}
\nopagebreak
\textamh{ቢስሚላሂ አራህመኒ ራሂይም } &  بِسْمِ ٱللَّهِ ٱلرَّحْمَـٰنِ ٱلرَّحِيمِ\\
\textamh{1.\  } &  أَتَىٰٓ أَمْرُ ٱللَّهِ فَلَا تَسْتَعْجِلُوهُ ۚ سُبْحَـٰنَهُۥ وَتَعَـٰلَىٰ عَمَّا يُشْرِكُونَ ﴿١﴾\\
\textamh{2.\  } & يُنَزِّلُ ٱلْمَلَـٰٓئِكَةَ بِٱلرُّوحِ مِنْ أَمْرِهِۦ عَلَىٰ مَن يَشَآءُ مِنْ عِبَادِهِۦٓ أَنْ أَنذِرُوٓا۟ أَنَّهُۥ لَآ إِلَـٰهَ إِلَّآ أَنَا۠ فَٱتَّقُونِ ﴿٢﴾\\
\textamh{3.\  } & خَلَقَ ٱلسَّمَـٰوَٟتِ وَٱلْأَرْضَ بِٱلْحَقِّ ۚ تَعَـٰلَىٰ عَمَّا يُشْرِكُونَ ﴿٣﴾\\
\textamh{4.\  } & خَلَقَ ٱلْإِنسَـٰنَ مِن نُّطْفَةٍۢ فَإِذَا هُوَ خَصِيمٌۭ مُّبِينٌۭ ﴿٤﴾\\
\textamh{5.\  } & وَٱلْأَنْعَـٰمَ خَلَقَهَا ۗ لَكُمْ فِيهَا دِفْءٌۭ وَمَنَـٰفِعُ وَمِنْهَا تَأْكُلُونَ ﴿٥﴾\\
\textamh{6.\  } & وَلَكُمْ فِيهَا جَمَالٌ حِينَ تُرِيحُونَ وَحِينَ تَسْرَحُونَ ﴿٦﴾\\
\textamh{7.\  } & وَتَحْمِلُ أَثْقَالَكُمْ إِلَىٰ بَلَدٍۢ لَّمْ تَكُونُوا۟ بَٰلِغِيهِ إِلَّا بِشِقِّ ٱلْأَنفُسِ ۚ إِنَّ رَبَّكُمْ لَرَءُوفٌۭ رَّحِيمٌۭ ﴿٧﴾\\
\textamh{8.\  } & وَٱلْخَيْلَ وَٱلْبِغَالَ وَٱلْحَمِيرَ لِتَرْكَبُوهَا وَزِينَةًۭ ۚ وَيَخْلُقُ مَا لَا تَعْلَمُونَ ﴿٨﴾\\
\textamh{9.\  } & وَعَلَى ٱللَّهِ قَصْدُ ٱلسَّبِيلِ وَمِنْهَا جَآئِرٌۭ ۚ وَلَوْ شَآءَ لَهَدَىٰكُمْ أَجْمَعِينَ ﴿٩﴾\\
\textamh{10.\  } & هُوَ ٱلَّذِىٓ أَنزَلَ مِنَ ٱلسَّمَآءِ مَآءًۭ ۖ لَّكُم مِّنْهُ شَرَابٌۭ وَمِنْهُ شَجَرٌۭ فِيهِ تُسِيمُونَ ﴿١٠﴾\\
\textamh{11.\  } & يُنۢبِتُ لَكُم بِهِ ٱلزَّرْعَ وَٱلزَّيْتُونَ وَٱلنَّخِيلَ وَٱلْأَعْنَـٰبَ وَمِن كُلِّ ٱلثَّمَرَٰتِ ۗ إِنَّ فِى ذَٟلِكَ لَءَايَةًۭ لِّقَوْمٍۢ يَتَفَكَّرُونَ ﴿١١﴾\\
\textamh{12.\  } & وَسَخَّرَ لَكُمُ ٱلَّيْلَ وَٱلنَّهَارَ وَٱلشَّمْسَ وَٱلْقَمَرَ ۖ وَٱلنُّجُومُ مُسَخَّرَٰتٌۢ بِأَمْرِهِۦٓ ۗ إِنَّ فِى ذَٟلِكَ لَءَايَـٰتٍۢ لِّقَوْمٍۢ يَعْقِلُونَ ﴿١٢﴾\\
\textamh{13.\  } & وَمَا ذَرَأَ لَكُمْ فِى ٱلْأَرْضِ مُخْتَلِفًا أَلْوَٟنُهُۥٓ ۗ إِنَّ فِى ذَٟلِكَ لَءَايَةًۭ لِّقَوْمٍۢ يَذَّكَّرُونَ ﴿١٣﴾\\
\textamh{14.\  } & وَهُوَ ٱلَّذِى سَخَّرَ ٱلْبَحْرَ لِتَأْكُلُوا۟ مِنْهُ لَحْمًۭا طَرِيًّۭا وَتَسْتَخْرِجُوا۟ مِنْهُ حِلْيَةًۭ تَلْبَسُونَهَا وَتَرَى ٱلْفُلْكَ مَوَاخِرَ فِيهِ وَلِتَبْتَغُوا۟ مِن فَضْلِهِۦ وَلَعَلَّكُمْ تَشْكُرُونَ ﴿١٤﴾\\
\textamh{15.\  } & وَأَلْقَىٰ فِى ٱلْأَرْضِ رَوَٟسِىَ أَن تَمِيدَ بِكُمْ وَأَنْهَـٰرًۭا وَسُبُلًۭا لَّعَلَّكُمْ تَهْتَدُونَ ﴿١٥﴾\\
\textamh{16.\  } & وَعَلَـٰمَـٰتٍۢ ۚ وَبِٱلنَّجْمِ هُمْ يَهْتَدُونَ ﴿١٦﴾\\
\textamh{17.\  } & أَفَمَن يَخْلُقُ كَمَن لَّا يَخْلُقُ ۗ أَفَلَا تَذَكَّرُونَ ﴿١٧﴾\\
\textamh{18.\  } & وَإِن تَعُدُّوا۟ نِعْمَةَ ٱللَّهِ لَا تُحْصُوهَآ ۗ إِنَّ ٱللَّهَ لَغَفُورٌۭ رَّحِيمٌۭ ﴿١٨﴾\\
\textamh{19.\  } & وَٱللَّهُ يَعْلَمُ مَا تُسِرُّونَ وَمَا تُعْلِنُونَ ﴿١٩﴾\\
\textamh{20.\  } & وَٱلَّذِينَ يَدْعُونَ مِن دُونِ ٱللَّهِ لَا يَخْلُقُونَ شَيْـًۭٔا وَهُمْ يُخْلَقُونَ ﴿٢٠﴾\\
\textamh{21.\  } & أَمْوَٟتٌ غَيْرُ أَحْيَآءٍۢ ۖ وَمَا يَشْعُرُونَ أَيَّانَ يُبْعَثُونَ ﴿٢١﴾\\
\textamh{22.\  } & إِلَـٰهُكُمْ إِلَـٰهٌۭ وَٟحِدٌۭ ۚ فَٱلَّذِينَ لَا يُؤْمِنُونَ بِٱلْءَاخِرَةِ قُلُوبُهُم مُّنكِرَةٌۭ وَهُم مُّسْتَكْبِرُونَ ﴿٢٢﴾\\
\textamh{23.\  } & لَا جَرَمَ أَنَّ ٱللَّهَ يَعْلَمُ مَا يُسِرُّونَ وَمَا يُعْلِنُونَ ۚ إِنَّهُۥ لَا يُحِبُّ ٱلْمُسْتَكْبِرِينَ ﴿٢٣﴾\\
\textamh{24.\  } & وَإِذَا قِيلَ لَهُم مَّاذَآ أَنزَلَ رَبُّكُمْ ۙ قَالُوٓا۟ أَسَـٰطِيرُ ٱلْأَوَّلِينَ ﴿٢٤﴾\\
\textamh{25.\  } & لِيَحْمِلُوٓا۟ أَوْزَارَهُمْ كَامِلَةًۭ يَوْمَ ٱلْقِيَـٰمَةِ ۙ وَمِنْ أَوْزَارِ ٱلَّذِينَ يُضِلُّونَهُم بِغَيْرِ عِلْمٍ ۗ أَلَا سَآءَ مَا يَزِرُونَ ﴿٢٥﴾\\
\textamh{26.\  } & قَدْ مَكَرَ ٱلَّذِينَ مِن قَبْلِهِمْ فَأَتَى ٱللَّهُ بُنْيَـٰنَهُم مِّنَ ٱلْقَوَاعِدِ فَخَرَّ عَلَيْهِمُ ٱلسَّقْفُ مِن فَوْقِهِمْ وَأَتَىٰهُمُ ٱلْعَذَابُ مِنْ حَيْثُ لَا يَشْعُرُونَ ﴿٢٦﴾\\
\textamh{27.\  } & ثُمَّ يَوْمَ ٱلْقِيَـٰمَةِ يُخْزِيهِمْ وَيَقُولُ أَيْنَ شُرَكَآءِىَ ٱلَّذِينَ كُنتُمْ تُشَـٰٓقُّونَ فِيهِمْ ۚ قَالَ ٱلَّذِينَ أُوتُوا۟ ٱلْعِلْمَ إِنَّ ٱلْخِزْىَ ٱلْيَوْمَ وَٱلسُّوٓءَ عَلَى ٱلْكَـٰفِرِينَ ﴿٢٧﴾\\
\textamh{28.\  } & ٱلَّذِينَ تَتَوَفَّىٰهُمُ ٱلْمَلَـٰٓئِكَةُ ظَالِمِىٓ أَنفُسِهِمْ ۖ فَأَلْقَوُا۟ ٱلسَّلَمَ مَا كُنَّا نَعْمَلُ مِن سُوٓءٍۭ ۚ بَلَىٰٓ إِنَّ ٱللَّهَ عَلِيمٌۢ بِمَا كُنتُمْ تَعْمَلُونَ ﴿٢٨﴾\\
\textamh{29.\  } & فَٱدْخُلُوٓا۟ أَبْوَٟبَ جَهَنَّمَ خَـٰلِدِينَ فِيهَا ۖ فَلَبِئْسَ مَثْوَى ٱلْمُتَكَبِّرِينَ ﴿٢٩﴾\\
\textamh{30.\  } & ۞ وَقِيلَ لِلَّذِينَ ٱتَّقَوْا۟ مَاذَآ أَنزَلَ رَبُّكُمْ ۚ قَالُوا۟ خَيْرًۭا ۗ لِّلَّذِينَ أَحْسَنُوا۟ فِى هَـٰذِهِ ٱلدُّنْيَا حَسَنَةٌۭ ۚ وَلَدَارُ ٱلْءَاخِرَةِ خَيْرٌۭ ۚ وَلَنِعْمَ دَارُ ٱلْمُتَّقِينَ ﴿٣٠﴾\\
\textamh{31.\  } & جَنَّـٰتُ عَدْنٍۢ يَدْخُلُونَهَا تَجْرِى مِن تَحْتِهَا ٱلْأَنْهَـٰرُ ۖ لَهُمْ فِيهَا مَا يَشَآءُونَ ۚ كَذَٟلِكَ يَجْزِى ٱللَّهُ ٱلْمُتَّقِينَ ﴿٣١﴾\\
\textamh{32.\  } & ٱلَّذِينَ تَتَوَفَّىٰهُمُ ٱلْمَلَـٰٓئِكَةُ طَيِّبِينَ ۙ يَقُولُونَ سَلَـٰمٌ عَلَيْكُمُ ٱدْخُلُوا۟ ٱلْجَنَّةَ بِمَا كُنتُمْ تَعْمَلُونَ ﴿٣٢﴾\\
\textamh{33.\  } & هَلْ يَنظُرُونَ إِلَّآ أَن تَأْتِيَهُمُ ٱلْمَلَـٰٓئِكَةُ أَوْ يَأْتِىَ أَمْرُ رَبِّكَ ۚ كَذَٟلِكَ فَعَلَ ٱلَّذِينَ مِن قَبْلِهِمْ ۚ وَمَا ظَلَمَهُمُ ٱللَّهُ وَلَـٰكِن كَانُوٓا۟ أَنفُسَهُمْ يَظْلِمُونَ ﴿٣٣﴾\\
\textamh{34.\  } & فَأَصَابَهُمْ سَيِّـَٔاتُ مَا عَمِلُوا۟ وَحَاقَ بِهِم مَّا كَانُوا۟ بِهِۦ يَسْتَهْزِءُونَ ﴿٣٤﴾\\
\textamh{35.\  } & وَقَالَ ٱلَّذِينَ أَشْرَكُوا۟ لَوْ شَآءَ ٱللَّهُ مَا عَبَدْنَا مِن دُونِهِۦ مِن شَىْءٍۢ نَّحْنُ وَلَآ ءَابَآؤُنَا وَلَا حَرَّمْنَا مِن دُونِهِۦ مِن شَىْءٍۢ ۚ كَذَٟلِكَ فَعَلَ ٱلَّذِينَ مِن قَبْلِهِمْ ۚ فَهَلْ عَلَى ٱلرُّسُلِ إِلَّا ٱلْبَلَـٰغُ ٱلْمُبِينُ ﴿٣٥﴾\\
\textamh{36.\  } & وَلَقَدْ بَعَثْنَا فِى كُلِّ أُمَّةٍۢ رَّسُولًا أَنِ ٱعْبُدُوا۟ ٱللَّهَ وَٱجْتَنِبُوا۟ ٱلطَّٰغُوتَ ۖ فَمِنْهُم مَّنْ هَدَى ٱللَّهُ وَمِنْهُم مَّنْ حَقَّتْ عَلَيْهِ ٱلضَّلَـٰلَةُ ۚ فَسِيرُوا۟ فِى ٱلْأَرْضِ فَٱنظُرُوا۟ كَيْفَ كَانَ عَـٰقِبَةُ ٱلْمُكَذِّبِينَ ﴿٣٦﴾\\
\textamh{37.\  } & إِن تَحْرِصْ عَلَىٰ هُدَىٰهُمْ فَإِنَّ ٱللَّهَ لَا يَهْدِى مَن يُضِلُّ ۖ وَمَا لَهُم مِّن نَّـٰصِرِينَ ﴿٣٧﴾\\
\textamh{38.\  } & وَأَقْسَمُوا۟ بِٱللَّهِ جَهْدَ أَيْمَـٰنِهِمْ ۙ لَا يَبْعَثُ ٱللَّهُ مَن يَمُوتُ ۚ بَلَىٰ وَعْدًا عَلَيْهِ حَقًّۭا وَلَـٰكِنَّ أَكْثَرَ ٱلنَّاسِ لَا يَعْلَمُونَ ﴿٣٨﴾\\
\textamh{39.\  } & لِيُبَيِّنَ لَهُمُ ٱلَّذِى يَخْتَلِفُونَ فِيهِ وَلِيَعْلَمَ ٱلَّذِينَ كَفَرُوٓا۟ أَنَّهُمْ كَانُوا۟ كَـٰذِبِينَ ﴿٣٩﴾\\
\textamh{40.\  } & إِنَّمَا قَوْلُنَا لِشَىْءٍ إِذَآ أَرَدْنَـٰهُ أَن نَّقُولَ لَهُۥ كُن فَيَكُونُ ﴿٤٠﴾\\
\textamh{41.\  } & وَٱلَّذِينَ هَاجَرُوا۟ فِى ٱللَّهِ مِنۢ بَعْدِ مَا ظُلِمُوا۟ لَنُبَوِّئَنَّهُمْ فِى ٱلدُّنْيَا حَسَنَةًۭ ۖ وَلَأَجْرُ ٱلْءَاخِرَةِ أَكْبَرُ ۚ لَوْ كَانُوا۟ يَعْلَمُونَ ﴿٤١﴾\\
\textamh{42.\  } & ٱلَّذِينَ صَبَرُوا۟ وَعَلَىٰ رَبِّهِمْ يَتَوَكَّلُونَ ﴿٤٢﴾\\
\textamh{43.\  } & وَمَآ أَرْسَلْنَا مِن قَبْلِكَ إِلَّا رِجَالًۭا نُّوحِىٓ إِلَيْهِمْ ۚ فَسْـَٔلُوٓا۟ أَهْلَ ٱلذِّكْرِ إِن كُنتُمْ لَا تَعْلَمُونَ ﴿٤٣﴾\\
\textamh{44.\  } & بِٱلْبَيِّنَـٰتِ وَٱلزُّبُرِ ۗ وَأَنزَلْنَآ إِلَيْكَ ٱلذِّكْرَ لِتُبَيِّنَ لِلنَّاسِ مَا نُزِّلَ إِلَيْهِمْ وَلَعَلَّهُمْ يَتَفَكَّرُونَ ﴿٤٤﴾\\
\textamh{45.\  } & أَفَأَمِنَ ٱلَّذِينَ مَكَرُوا۟ ٱلسَّيِّـَٔاتِ أَن يَخْسِفَ ٱللَّهُ بِهِمُ ٱلْأَرْضَ أَوْ يَأْتِيَهُمُ ٱلْعَذَابُ مِنْ حَيْثُ لَا يَشْعُرُونَ ﴿٤٥﴾\\
\textamh{46.\  } & أَوْ يَأْخُذَهُمْ فِى تَقَلُّبِهِمْ فَمَا هُم بِمُعْجِزِينَ ﴿٤٦﴾\\
\textamh{47.\  } & أَوْ يَأْخُذَهُمْ عَلَىٰ تَخَوُّفٍۢ فَإِنَّ رَبَّكُمْ لَرَءُوفٌۭ رَّحِيمٌ ﴿٤٧﴾\\
\textamh{48.\  } & أَوَلَمْ يَرَوْا۟ إِلَىٰ مَا خَلَقَ ٱللَّهُ مِن شَىْءٍۢ يَتَفَيَّؤُا۟ ظِلَـٰلُهُۥ عَنِ ٱلْيَمِينِ وَٱلشَّمَآئِلِ سُجَّدًۭا لِّلَّهِ وَهُمْ دَٟخِرُونَ ﴿٤٨﴾\\
\textamh{49.\  } & وَلِلَّهِ يَسْجُدُ مَا فِى ٱلسَّمَـٰوَٟتِ وَمَا فِى ٱلْأَرْضِ مِن دَآبَّةٍۢ وَٱلْمَلَـٰٓئِكَةُ وَهُمْ لَا يَسْتَكْبِرُونَ ﴿٤٩﴾\\
\textamh{50.\  } & يَخَافُونَ رَبَّهُم مِّن فَوْقِهِمْ وَيَفْعَلُونَ مَا يُؤْمَرُونَ ۩ ﴿٥٠﴾\\
\textamh{51.\  } & ۞ وَقَالَ ٱللَّهُ لَا تَتَّخِذُوٓا۟ إِلَـٰهَيْنِ ٱثْنَيْنِ ۖ إِنَّمَا هُوَ إِلَـٰهٌۭ وَٟحِدٌۭ ۖ فَإِيَّٰىَ فَٱرْهَبُونِ ﴿٥١﴾\\
\textamh{52.\  } & وَلَهُۥ مَا فِى ٱلسَّمَـٰوَٟتِ وَٱلْأَرْضِ وَلَهُ ٱلدِّينُ وَاصِبًا ۚ أَفَغَيْرَ ٱللَّهِ تَتَّقُونَ ﴿٥٢﴾\\
\textamh{53.\  } & وَمَا بِكُم مِّن نِّعْمَةٍۢ فَمِنَ ٱللَّهِ ۖ ثُمَّ إِذَا مَسَّكُمُ ٱلضُّرُّ فَإِلَيْهِ تَجْـَٔرُونَ ﴿٥٣﴾\\
\textamh{54.\  } & ثُمَّ إِذَا كَشَفَ ٱلضُّرَّ عَنكُمْ إِذَا فَرِيقٌۭ مِّنكُم بِرَبِّهِمْ يُشْرِكُونَ ﴿٥٤﴾\\
\textamh{55.\  } & لِيَكْفُرُوا۟ بِمَآ ءَاتَيْنَـٰهُمْ ۚ فَتَمَتَّعُوا۟ ۖ فَسَوْفَ تَعْلَمُونَ ﴿٥٥﴾\\
\textamh{56.\  } & وَيَجْعَلُونَ لِمَا لَا يَعْلَمُونَ نَصِيبًۭا مِّمَّا رَزَقْنَـٰهُمْ ۗ تَٱللَّهِ لَتُسْـَٔلُنَّ عَمَّا كُنتُمْ تَفْتَرُونَ ﴿٥٦﴾\\
\textamh{57.\  } & وَيَجْعَلُونَ لِلَّهِ ٱلْبَنَـٰتِ سُبْحَـٰنَهُۥ ۙ وَلَهُم مَّا يَشْتَهُونَ ﴿٥٧﴾\\
\textamh{58.\  } & وَإِذَا بُشِّرَ أَحَدُهُم بِٱلْأُنثَىٰ ظَلَّ وَجْهُهُۥ مُسْوَدًّۭا وَهُوَ كَظِيمٌۭ ﴿٥٨﴾\\
\textamh{59.\  } & يَتَوَٟرَىٰ مِنَ ٱلْقَوْمِ مِن سُوٓءِ مَا بُشِّرَ بِهِۦٓ ۚ أَيُمْسِكُهُۥ عَلَىٰ هُونٍ أَمْ يَدُسُّهُۥ فِى ٱلتُّرَابِ ۗ أَلَا سَآءَ مَا يَحْكُمُونَ ﴿٥٩﴾\\
\textamh{60.\  } & لِلَّذِينَ لَا يُؤْمِنُونَ بِٱلْءَاخِرَةِ مَثَلُ ٱلسَّوْءِ ۖ وَلِلَّهِ ٱلْمَثَلُ ٱلْأَعْلَىٰ ۚ وَهُوَ ٱلْعَزِيزُ ٱلْحَكِيمُ ﴿٦٠﴾\\
\textamh{61.\  } & وَلَوْ يُؤَاخِذُ ٱللَّهُ ٱلنَّاسَ بِظُلْمِهِم مَّا تَرَكَ عَلَيْهَا مِن دَآبَّةٍۢ وَلَـٰكِن يُؤَخِّرُهُمْ إِلَىٰٓ أَجَلٍۢ مُّسَمًّۭى ۖ فَإِذَا جَآءَ أَجَلُهُمْ لَا يَسْتَـْٔخِرُونَ سَاعَةًۭ ۖ وَلَا يَسْتَقْدِمُونَ ﴿٦١﴾\\
\textamh{62.\  } & وَيَجْعَلُونَ لِلَّهِ مَا يَكْرَهُونَ وَتَصِفُ أَلْسِنَتُهُمُ ٱلْكَذِبَ أَنَّ لَهُمُ ٱلْحُسْنَىٰ ۖ لَا جَرَمَ أَنَّ لَهُمُ ٱلنَّارَ وَأَنَّهُم مُّفْرَطُونَ ﴿٦٢﴾\\
\textamh{63.\  } & تَٱللَّهِ لَقَدْ أَرْسَلْنَآ إِلَىٰٓ أُمَمٍۢ مِّن قَبْلِكَ فَزَيَّنَ لَهُمُ ٱلشَّيْطَٰنُ أَعْمَـٰلَهُمْ فَهُوَ وَلِيُّهُمُ ٱلْيَوْمَ وَلَهُمْ عَذَابٌ أَلِيمٌۭ ﴿٦٣﴾\\
\textamh{64.\  } & وَمَآ أَنزَلْنَا عَلَيْكَ ٱلْكِتَـٰبَ إِلَّا لِتُبَيِّنَ لَهُمُ ٱلَّذِى ٱخْتَلَفُوا۟ فِيهِ ۙ وَهُدًۭى وَرَحْمَةًۭ لِّقَوْمٍۢ يُؤْمِنُونَ ﴿٦٤﴾\\
\textamh{65.\  } & وَٱللَّهُ أَنزَلَ مِنَ ٱلسَّمَآءِ مَآءًۭ فَأَحْيَا بِهِ ٱلْأَرْضَ بَعْدَ مَوْتِهَآ ۚ إِنَّ فِى ذَٟلِكَ لَءَايَةًۭ لِّقَوْمٍۢ يَسْمَعُونَ ﴿٦٥﴾\\
\textamh{66.\  } & وَإِنَّ لَكُمْ فِى ٱلْأَنْعَـٰمِ لَعِبْرَةًۭ ۖ نُّسْقِيكُم مِّمَّا فِى بُطُونِهِۦ مِنۢ بَيْنِ فَرْثٍۢ وَدَمٍۢ لَّبَنًا خَالِصًۭا سَآئِغًۭا لِّلشَّـٰرِبِينَ ﴿٦٦﴾\\
\textamh{67.\  } & وَمِن ثَمَرَٰتِ ٱلنَّخِيلِ وَٱلْأَعْنَـٰبِ تَتَّخِذُونَ مِنْهُ سَكَرًۭا وَرِزْقًا حَسَنًا ۗ إِنَّ فِى ذَٟلِكَ لَءَايَةًۭ لِّقَوْمٍۢ يَعْقِلُونَ ﴿٦٧﴾\\
\textamh{68.\  } & وَأَوْحَىٰ رَبُّكَ إِلَى ٱلنَّحْلِ أَنِ ٱتَّخِذِى مِنَ ٱلْجِبَالِ بُيُوتًۭا وَمِنَ ٱلشَّجَرِ وَمِمَّا يَعْرِشُونَ ﴿٦٨﴾\\
\textamh{69.\  } & ثُمَّ كُلِى مِن كُلِّ ٱلثَّمَرَٰتِ فَٱسْلُكِى سُبُلَ رَبِّكِ ذُلُلًۭا ۚ يَخْرُجُ مِنۢ بُطُونِهَا شَرَابٌۭ مُّخْتَلِفٌ أَلْوَٟنُهُۥ فِيهِ شِفَآءٌۭ لِّلنَّاسِ ۗ إِنَّ فِى ذَٟلِكَ لَءَايَةًۭ لِّقَوْمٍۢ يَتَفَكَّرُونَ ﴿٦٩﴾\\
\textamh{70.\  } & وَٱللَّهُ خَلَقَكُمْ ثُمَّ يَتَوَفَّىٰكُمْ ۚ وَمِنكُم مَّن يُرَدُّ إِلَىٰٓ أَرْذَلِ ٱلْعُمُرِ لِكَىْ لَا يَعْلَمَ بَعْدَ عِلْمٍۢ شَيْـًٔا ۚ إِنَّ ٱللَّهَ عَلِيمٌۭ قَدِيرٌۭ ﴿٧٠﴾\\
\textamh{71.\  } & وَٱللَّهُ فَضَّلَ بَعْضَكُمْ عَلَىٰ بَعْضٍۢ فِى ٱلرِّزْقِ ۚ فَمَا ٱلَّذِينَ فُضِّلُوا۟ بِرَآدِّى رِزْقِهِمْ عَلَىٰ مَا مَلَكَتْ أَيْمَـٰنُهُمْ فَهُمْ فِيهِ سَوَآءٌ ۚ أَفَبِنِعْمَةِ ٱللَّهِ يَجْحَدُونَ ﴿٧١﴾\\
\textamh{72.\  } & وَٱللَّهُ جَعَلَ لَكُم مِّنْ أَنفُسِكُمْ أَزْوَٟجًۭا وَجَعَلَ لَكُم مِّنْ أَزْوَٟجِكُم بَنِينَ وَحَفَدَةًۭ وَرَزَقَكُم مِّنَ ٱلطَّيِّبَٰتِ ۚ أَفَبِٱلْبَٰطِلِ يُؤْمِنُونَ وَبِنِعْمَتِ ٱللَّهِ هُمْ يَكْفُرُونَ ﴿٧٢﴾\\
\textamh{73.\  } & وَيَعْبُدُونَ مِن دُونِ ٱللَّهِ مَا لَا يَمْلِكُ لَهُمْ رِزْقًۭا مِّنَ ٱلسَّمَـٰوَٟتِ وَٱلْأَرْضِ شَيْـًۭٔا وَلَا يَسْتَطِيعُونَ ﴿٧٣﴾\\
\textamh{74.\  } & فَلَا تَضْرِبُوا۟ لِلَّهِ ٱلْأَمْثَالَ ۚ إِنَّ ٱللَّهَ يَعْلَمُ وَأَنتُمْ لَا تَعْلَمُونَ ﴿٧٤﴾\\
\textamh{75.\  } & ۞ ضَرَبَ ٱللَّهُ مَثَلًا عَبْدًۭا مَّمْلُوكًۭا لَّا يَقْدِرُ عَلَىٰ شَىْءٍۢ وَمَن رَّزَقْنَـٰهُ مِنَّا رِزْقًا حَسَنًۭا فَهُوَ يُنفِقُ مِنْهُ سِرًّۭا وَجَهْرًا ۖ هَلْ يَسْتَوُۥنَ ۚ ٱلْحَمْدُ لِلَّهِ ۚ بَلْ أَكْثَرُهُمْ لَا يَعْلَمُونَ ﴿٧٥﴾\\
\textamh{76.\  } & وَضَرَبَ ٱللَّهُ مَثَلًۭا رَّجُلَيْنِ أَحَدُهُمَآ أَبْكَمُ لَا يَقْدِرُ عَلَىٰ شَىْءٍۢ وَهُوَ كَلٌّ عَلَىٰ مَوْلَىٰهُ أَيْنَمَا يُوَجِّههُّ لَا يَأْتِ بِخَيْرٍ ۖ هَلْ يَسْتَوِى هُوَ وَمَن يَأْمُرُ بِٱلْعَدْلِ ۙ وَهُوَ عَلَىٰ صِرَٰطٍۢ مُّسْتَقِيمٍۢ ﴿٧٦﴾\\
\textamh{77.\  } & وَلِلَّهِ غَيْبُ ٱلسَّمَـٰوَٟتِ وَٱلْأَرْضِ ۚ وَمَآ أَمْرُ ٱلسَّاعَةِ إِلَّا كَلَمْحِ ٱلْبَصَرِ أَوْ هُوَ أَقْرَبُ ۚ إِنَّ ٱللَّهَ عَلَىٰ كُلِّ شَىْءٍۢ قَدِيرٌۭ ﴿٧٧﴾\\
\textamh{78.\  } & وَٱللَّهُ أَخْرَجَكُم مِّنۢ بُطُونِ أُمَّهَـٰتِكُمْ لَا تَعْلَمُونَ شَيْـًۭٔا وَجَعَلَ لَكُمُ ٱلسَّمْعَ وَٱلْأَبْصَـٰرَ وَٱلْأَفْـِٔدَةَ ۙ لَعَلَّكُمْ تَشْكُرُونَ ﴿٧٨﴾\\
\textamh{79.\  } & أَلَمْ يَرَوْا۟ إِلَى ٱلطَّيْرِ مُسَخَّرَٰتٍۢ فِى جَوِّ ٱلسَّمَآءِ مَا يُمْسِكُهُنَّ إِلَّا ٱللَّهُ ۗ إِنَّ فِى ذَٟلِكَ لَءَايَـٰتٍۢ لِّقَوْمٍۢ يُؤْمِنُونَ ﴿٧٩﴾\\
\textamh{80.\  } & وَٱللَّهُ جَعَلَ لَكُم مِّنۢ بُيُوتِكُمْ سَكَنًۭا وَجَعَلَ لَكُم مِّن جُلُودِ ٱلْأَنْعَـٰمِ بُيُوتًۭا تَسْتَخِفُّونَهَا يَوْمَ ظَعْنِكُمْ وَيَوْمَ إِقَامَتِكُمْ ۙ وَمِنْ أَصْوَافِهَا وَأَوْبَارِهَا وَأَشْعَارِهَآ أَثَـٰثًۭا وَمَتَـٰعًا إِلَىٰ حِينٍۢ ﴿٨٠﴾\\
\textamh{81.\  } & وَٱللَّهُ جَعَلَ لَكُم مِّمَّا خَلَقَ ظِلَـٰلًۭا وَجَعَلَ لَكُم مِّنَ ٱلْجِبَالِ أَكْنَـٰنًۭا وَجَعَلَ لَكُمْ سَرَٰبِيلَ تَقِيكُمُ ٱلْحَرَّ وَسَرَٰبِيلَ تَقِيكُم بَأْسَكُمْ ۚ كَذَٟلِكَ يُتِمُّ نِعْمَتَهُۥ عَلَيْكُمْ لَعَلَّكُمْ تُسْلِمُونَ ﴿٨١﴾\\
\textamh{82.\  } & فَإِن تَوَلَّوْا۟ فَإِنَّمَا عَلَيْكَ ٱلْبَلَـٰغُ ٱلْمُبِينُ ﴿٨٢﴾\\
\textamh{83.\  } & يَعْرِفُونَ نِعْمَتَ ٱللَّهِ ثُمَّ يُنكِرُونَهَا وَأَكْثَرُهُمُ ٱلْكَـٰفِرُونَ ﴿٨٣﴾\\
\textamh{84.\  } & وَيَوْمَ نَبْعَثُ مِن كُلِّ أُمَّةٍۢ شَهِيدًۭا ثُمَّ لَا يُؤْذَنُ لِلَّذِينَ كَفَرُوا۟ وَلَا هُمْ يُسْتَعْتَبُونَ ﴿٨٤﴾\\
\textamh{85.\  } & وَإِذَا رَءَا ٱلَّذِينَ ظَلَمُوا۟ ٱلْعَذَابَ فَلَا يُخَفَّفُ عَنْهُمْ وَلَا هُمْ يُنظَرُونَ ﴿٨٥﴾\\
\textamh{86.\  } & وَإِذَا رَءَا ٱلَّذِينَ أَشْرَكُوا۟ شُرَكَآءَهُمْ قَالُوا۟ رَبَّنَا هَـٰٓؤُلَآءِ شُرَكَآؤُنَا ٱلَّذِينَ كُنَّا نَدْعُوا۟ مِن دُونِكَ ۖ فَأَلْقَوْا۟ إِلَيْهِمُ ٱلْقَوْلَ إِنَّكُمْ لَكَـٰذِبُونَ ﴿٨٦﴾\\
\textamh{87.\  } & وَأَلْقَوْا۟ إِلَى ٱللَّهِ يَوْمَئِذٍ ٱلسَّلَمَ ۖ وَضَلَّ عَنْهُم مَّا كَانُوا۟ يَفْتَرُونَ ﴿٨٧﴾\\
\textamh{88.\  } & ٱلَّذِينَ كَفَرُوا۟ وَصَدُّوا۟ عَن سَبِيلِ ٱللَّهِ زِدْنَـٰهُمْ عَذَابًۭا فَوْقَ ٱلْعَذَابِ بِمَا كَانُوا۟ يُفْسِدُونَ ﴿٨٨﴾\\
\textamh{89.\  } & وَيَوْمَ نَبْعَثُ فِى كُلِّ أُمَّةٍۢ شَهِيدًا عَلَيْهِم مِّنْ أَنفُسِهِمْ ۖ وَجِئْنَا بِكَ شَهِيدًا عَلَىٰ هَـٰٓؤُلَآءِ ۚ وَنَزَّلْنَا عَلَيْكَ ٱلْكِتَـٰبَ تِبْيَـٰنًۭا لِّكُلِّ شَىْءٍۢ وَهُدًۭى وَرَحْمَةًۭ وَبُشْرَىٰ لِلْمُسْلِمِينَ ﴿٨٩﴾\\
\textamh{90.\  } & ۞ إِنَّ ٱللَّهَ يَأْمُرُ بِٱلْعَدْلِ وَٱلْإِحْسَـٰنِ وَإِيتَآئِ ذِى ٱلْقُرْبَىٰ وَيَنْهَىٰ عَنِ ٱلْفَحْشَآءِ وَٱلْمُنكَرِ وَٱلْبَغْىِ ۚ يَعِظُكُمْ لَعَلَّكُمْ تَذَكَّرُونَ ﴿٩٠﴾\\
\textamh{91.\  } & وَأَوْفُوا۟ بِعَهْدِ ٱللَّهِ إِذَا عَـٰهَدتُّمْ وَلَا تَنقُضُوا۟ ٱلْأَيْمَـٰنَ بَعْدَ تَوْكِيدِهَا وَقَدْ جَعَلْتُمُ ٱللَّهَ عَلَيْكُمْ كَفِيلًا ۚ إِنَّ ٱللَّهَ يَعْلَمُ مَا تَفْعَلُونَ ﴿٩١﴾\\
\textamh{92.\  } & وَلَا تَكُونُوا۟ كَٱلَّتِى نَقَضَتْ غَزْلَهَا مِنۢ بَعْدِ قُوَّةٍ أَنكَـٰثًۭا تَتَّخِذُونَ أَيْمَـٰنَكُمْ دَخَلًۢا بَيْنَكُمْ أَن تَكُونَ أُمَّةٌ هِىَ أَرْبَىٰ مِنْ أُمَّةٍ ۚ إِنَّمَا يَبْلُوكُمُ ٱللَّهُ بِهِۦ ۚ وَلَيُبَيِّنَنَّ لَكُمْ يَوْمَ ٱلْقِيَـٰمَةِ مَا كُنتُمْ فِيهِ تَخْتَلِفُونَ ﴿٩٢﴾\\
\textamh{93.\  } & وَلَوْ شَآءَ ٱللَّهُ لَجَعَلَكُمْ أُمَّةًۭ وَٟحِدَةًۭ وَلَـٰكِن يُضِلُّ مَن يَشَآءُ وَيَهْدِى مَن يَشَآءُ ۚ وَلَتُسْـَٔلُنَّ عَمَّا كُنتُمْ تَعْمَلُونَ ﴿٩٣﴾\\
\textamh{94.\  } & وَلَا تَتَّخِذُوٓا۟ أَيْمَـٰنَكُمْ دَخَلًۢا بَيْنَكُمْ فَتَزِلَّ قَدَمٌۢ بَعْدَ ثُبُوتِهَا وَتَذُوقُوا۟ ٱلسُّوٓءَ بِمَا صَدَدتُّمْ عَن سَبِيلِ ٱللَّهِ ۖ وَلَكُمْ عَذَابٌ عَظِيمٌۭ ﴿٩٤﴾\\
\textamh{95.\  } & وَلَا تَشْتَرُوا۟ بِعَهْدِ ٱللَّهِ ثَمَنًۭا قَلِيلًا ۚ إِنَّمَا عِندَ ٱللَّهِ هُوَ خَيْرٌۭ لَّكُمْ إِن كُنتُمْ تَعْلَمُونَ ﴿٩٥﴾\\
\textamh{96.\  } & مَا عِندَكُمْ يَنفَدُ ۖ وَمَا عِندَ ٱللَّهِ بَاقٍۢ ۗ وَلَنَجْزِيَنَّ ٱلَّذِينَ صَبَرُوٓا۟ أَجْرَهُم بِأَحْسَنِ مَا كَانُوا۟ يَعْمَلُونَ ﴿٩٦﴾\\
\textamh{97.\  } & مَنْ عَمِلَ صَـٰلِحًۭا مِّن ذَكَرٍ أَوْ أُنثَىٰ وَهُوَ مُؤْمِنٌۭ فَلَنُحْيِيَنَّهُۥ حَيَوٰةًۭ طَيِّبَةًۭ ۖ وَلَنَجْزِيَنَّهُمْ أَجْرَهُم بِأَحْسَنِ مَا كَانُوا۟ يَعْمَلُونَ ﴿٩٧﴾\\
\textamh{98.\  } & فَإِذَا قَرَأْتَ ٱلْقُرْءَانَ فَٱسْتَعِذْ بِٱللَّهِ مِنَ ٱلشَّيْطَٰنِ ٱلرَّجِيمِ ﴿٩٨﴾\\
\textamh{99.\  } & إِنَّهُۥ لَيْسَ لَهُۥ سُلْطَٰنٌ عَلَى ٱلَّذِينَ ءَامَنُوا۟ وَعَلَىٰ رَبِّهِمْ يَتَوَكَّلُونَ ﴿٩٩﴾\\
\textamh{100.\  } & إِنَّمَا سُلْطَٰنُهُۥ عَلَى ٱلَّذِينَ يَتَوَلَّوْنَهُۥ وَٱلَّذِينَ هُم بِهِۦ مُشْرِكُونَ ﴿١٠٠﴾\\
\textamh{101.\  } & وَإِذَا بَدَّلْنَآ ءَايَةًۭ مَّكَانَ ءَايَةٍۢ ۙ وَٱللَّهُ أَعْلَمُ بِمَا يُنَزِّلُ قَالُوٓا۟ إِنَّمَآ أَنتَ مُفْتَرٍۭ ۚ بَلْ أَكْثَرُهُمْ لَا يَعْلَمُونَ ﴿١٠١﴾\\
\textamh{102.\  } & قُلْ نَزَّلَهُۥ رُوحُ ٱلْقُدُسِ مِن رَّبِّكَ بِٱلْحَقِّ لِيُثَبِّتَ ٱلَّذِينَ ءَامَنُوا۟ وَهُدًۭى وَبُشْرَىٰ لِلْمُسْلِمِينَ ﴿١٠٢﴾\\
\textamh{103.\  } & وَلَقَدْ نَعْلَمُ أَنَّهُمْ يَقُولُونَ إِنَّمَا يُعَلِّمُهُۥ بَشَرٌۭ ۗ لِّسَانُ ٱلَّذِى يُلْحِدُونَ إِلَيْهِ أَعْجَمِىٌّۭ وَهَـٰذَا لِسَانٌ عَرَبِىٌّۭ مُّبِينٌ ﴿١٠٣﴾\\
\textamh{104.\  } & إِنَّ ٱلَّذِينَ لَا يُؤْمِنُونَ بِـَٔايَـٰتِ ٱللَّهِ لَا يَهْدِيهِمُ ٱللَّهُ وَلَهُمْ عَذَابٌ أَلِيمٌ ﴿١٠٤﴾\\
\textamh{105.\  } & إِنَّمَا يَفْتَرِى ٱلْكَذِبَ ٱلَّذِينَ لَا يُؤْمِنُونَ بِـَٔايَـٰتِ ٱللَّهِ ۖ وَأُو۟لَـٰٓئِكَ هُمُ ٱلْكَـٰذِبُونَ ﴿١٠٥﴾\\
\textamh{106.\  } & مَن كَفَرَ بِٱللَّهِ مِنۢ بَعْدِ إِيمَـٰنِهِۦٓ إِلَّا مَنْ أُكْرِهَ وَقَلْبُهُۥ مُطْمَئِنٌّۢ بِٱلْإِيمَـٰنِ وَلَـٰكِن مَّن شَرَحَ بِٱلْكُفْرِ صَدْرًۭا فَعَلَيْهِمْ غَضَبٌۭ مِّنَ ٱللَّهِ وَلَهُمْ عَذَابٌ عَظِيمٌۭ ﴿١٠٦﴾\\
\textamh{107.\  } & ذَٟلِكَ بِأَنَّهُمُ ٱسْتَحَبُّوا۟ ٱلْحَيَوٰةَ ٱلدُّنْيَا عَلَى ٱلْءَاخِرَةِ وَأَنَّ ٱللَّهَ لَا يَهْدِى ٱلْقَوْمَ ٱلْكَـٰفِرِينَ ﴿١٠٧﴾\\
\textamh{108.\  } & أُو۟لَـٰٓئِكَ ٱلَّذِينَ طَبَعَ ٱللَّهُ عَلَىٰ قُلُوبِهِمْ وَسَمْعِهِمْ وَأَبْصَـٰرِهِمْ ۖ وَأُو۟لَـٰٓئِكَ هُمُ ٱلْغَٰفِلُونَ ﴿١٠٨﴾\\
\textamh{109.\  } & لَا جَرَمَ أَنَّهُمْ فِى ٱلْءَاخِرَةِ هُمُ ٱلْخَـٰسِرُونَ ﴿١٠٩﴾\\
\textamh{110.\  } & ثُمَّ إِنَّ رَبَّكَ لِلَّذِينَ هَاجَرُوا۟ مِنۢ بَعْدِ مَا فُتِنُوا۟ ثُمَّ جَٰهَدُوا۟ وَصَبَرُوٓا۟ إِنَّ رَبَّكَ مِنۢ بَعْدِهَا لَغَفُورٌۭ رَّحِيمٌۭ ﴿١١٠﴾\\
\textamh{111.\  } & ۞ يَوْمَ تَأْتِى كُلُّ نَفْسٍۢ تُجَٰدِلُ عَن نَّفْسِهَا وَتُوَفَّىٰ كُلُّ نَفْسٍۢ مَّا عَمِلَتْ وَهُمْ لَا يُظْلَمُونَ ﴿١١١﴾\\
\textamh{112.\  } & وَضَرَبَ ٱللَّهُ مَثَلًۭا قَرْيَةًۭ كَانَتْ ءَامِنَةًۭ مُّطْمَئِنَّةًۭ يَأْتِيهَا رِزْقُهَا رَغَدًۭا مِّن كُلِّ مَكَانٍۢ فَكَفَرَتْ بِأَنْعُمِ ٱللَّهِ فَأَذَٟقَهَا ٱللَّهُ لِبَاسَ ٱلْجُوعِ وَٱلْخَوْفِ بِمَا كَانُوا۟ يَصْنَعُونَ ﴿١١٢﴾\\
\textamh{113.\  } & وَلَقَدْ جَآءَهُمْ رَسُولٌۭ مِّنْهُمْ فَكَذَّبُوهُ فَأَخَذَهُمُ ٱلْعَذَابُ وَهُمْ ظَـٰلِمُونَ ﴿١١٣﴾\\
\textamh{114.\  } & فَكُلُوا۟ مِمَّا رَزَقَكُمُ ٱللَّهُ حَلَـٰلًۭا طَيِّبًۭا وَٱشْكُرُوا۟ نِعْمَتَ ٱللَّهِ إِن كُنتُمْ إِيَّاهُ تَعْبُدُونَ ﴿١١٤﴾\\
\textamh{115.\  } & إِنَّمَا حَرَّمَ عَلَيْكُمُ ٱلْمَيْتَةَ وَٱلدَّمَ وَلَحْمَ ٱلْخِنزِيرِ وَمَآ أُهِلَّ لِغَيْرِ ٱللَّهِ بِهِۦ ۖ فَمَنِ ٱضْطُرَّ غَيْرَ بَاغٍۢ وَلَا عَادٍۢ فَإِنَّ ٱللَّهَ غَفُورٌۭ رَّحِيمٌۭ ﴿١١٥﴾\\
\textamh{116.\  } & وَلَا تَقُولُوا۟ لِمَا تَصِفُ أَلْسِنَتُكُمُ ٱلْكَذِبَ هَـٰذَا حَلَـٰلٌۭ وَهَـٰذَا حَرَامٌۭ لِّتَفْتَرُوا۟ عَلَى ٱللَّهِ ٱلْكَذِبَ ۚ إِنَّ ٱلَّذِينَ يَفْتَرُونَ عَلَى ٱللَّهِ ٱلْكَذِبَ لَا يُفْلِحُونَ ﴿١١٦﴾\\
\textamh{117.\  } & مَتَـٰعٌۭ قَلِيلٌۭ وَلَهُمْ عَذَابٌ أَلِيمٌۭ ﴿١١٧﴾\\
\textamh{118.\  } & وَعَلَى ٱلَّذِينَ هَادُوا۟ حَرَّمْنَا مَا قَصَصْنَا عَلَيْكَ مِن قَبْلُ ۖ وَمَا ظَلَمْنَـٰهُمْ وَلَـٰكِن كَانُوٓا۟ أَنفُسَهُمْ يَظْلِمُونَ ﴿١١٨﴾\\
\textamh{119.\  } & ثُمَّ إِنَّ رَبَّكَ لِلَّذِينَ عَمِلُوا۟ ٱلسُّوٓءَ بِجَهَـٰلَةٍۢ ثُمَّ تَابُوا۟ مِنۢ بَعْدِ ذَٟلِكَ وَأَصْلَحُوٓا۟ إِنَّ رَبَّكَ مِنۢ بَعْدِهَا لَغَفُورٌۭ رَّحِيمٌ ﴿١١٩﴾\\
\textamh{120.\  } & إِنَّ إِبْرَٰهِيمَ كَانَ أُمَّةًۭ قَانِتًۭا لِّلَّهِ حَنِيفًۭا وَلَمْ يَكُ مِنَ ٱلْمُشْرِكِينَ ﴿١٢٠﴾\\
\textamh{121.\  } & شَاكِرًۭا لِّأَنْعُمِهِ ۚ ٱجْتَبَىٰهُ وَهَدَىٰهُ إِلَىٰ صِرَٰطٍۢ مُّسْتَقِيمٍۢ ﴿١٢١﴾\\
\textamh{122.\  } & وَءَاتَيْنَـٰهُ فِى ٱلدُّنْيَا حَسَنَةًۭ ۖ وَإِنَّهُۥ فِى ٱلْءَاخِرَةِ لَمِنَ ٱلصَّـٰلِحِينَ ﴿١٢٢﴾\\
\textamh{123.\  } & ثُمَّ أَوْحَيْنَآ إِلَيْكَ أَنِ ٱتَّبِعْ مِلَّةَ إِبْرَٰهِيمَ حَنِيفًۭا ۖ وَمَا كَانَ مِنَ ٱلْمُشْرِكِينَ ﴿١٢٣﴾\\
\textamh{124.\  } & إِنَّمَا جُعِلَ ٱلسَّبْتُ عَلَى ٱلَّذِينَ ٱخْتَلَفُوا۟ فِيهِ ۚ وَإِنَّ رَبَّكَ لَيَحْكُمُ بَيْنَهُمْ يَوْمَ ٱلْقِيَـٰمَةِ فِيمَا كَانُوا۟ فِيهِ يَخْتَلِفُونَ ﴿١٢٤﴾\\
\textamh{125.\  } & ٱدْعُ إِلَىٰ سَبِيلِ رَبِّكَ بِٱلْحِكْمَةِ وَٱلْمَوْعِظَةِ ٱلْحَسَنَةِ ۖ وَجَٰدِلْهُم بِٱلَّتِى هِىَ أَحْسَنُ ۚ إِنَّ رَبَّكَ هُوَ أَعْلَمُ بِمَن ضَلَّ عَن سَبِيلِهِۦ ۖ وَهُوَ أَعْلَمُ بِٱلْمُهْتَدِينَ ﴿١٢٥﴾\\
\textamh{126.\  } & وَإِنْ عَاقَبْتُمْ فَعَاقِبُوا۟ بِمِثْلِ مَا عُوقِبْتُم بِهِۦ ۖ وَلَئِن صَبَرْتُمْ لَهُوَ خَيْرٌۭ لِّلصَّـٰبِرِينَ ﴿١٢٦﴾\\
\textamh{127.\  } & وَٱصْبِرْ وَمَا صَبْرُكَ إِلَّا بِٱللَّهِ ۚ وَلَا تَحْزَنْ عَلَيْهِمْ وَلَا تَكُ فِى ضَيْقٍۢ مِّمَّا يَمْكُرُونَ ﴿١٢٧﴾\\
\textamh{128.\  } & إِنَّ ٱللَّهَ مَعَ ٱلَّذِينَ ٱتَّقَوا۟ وَّٱلَّذِينَ هُم مُّحْسِنُونَ ﴿١٢٨﴾\\
\end{longtable}
\clearpage
%% License: BSD style (Berkley) (i.e. Put the Copyright owner's name always)
%% Writer and Copyright (to): Bewketu(Bilal) Tadilo (2016-17)
\centering\section{\LR{\textamharic{ሱራቱ አልኢስራኣ -}  \RL{سوره  الإسراء}}}
\begin{longtable}{%
  @{}
    p{.5\textwidth}
  @{~~~~~~~~~~~~}
    p{.5\textwidth}
    @{}
}
\nopagebreak
\textamh{ቢስሚላሂ አራህመኒ ራሂይም } &  بِسْمِ ٱللَّهِ ٱلرَّحْمَـٰنِ ٱلرَّحِيمِ\\
\textamh{1.\  } &  سُبْحَـٰنَ ٱلَّذِىٓ أَسْرَىٰ بِعَبْدِهِۦ لَيْلًۭا مِّنَ ٱلْمَسْجِدِ ٱلْحَرَامِ إِلَى ٱلْمَسْجِدِ ٱلْأَقْصَا ٱلَّذِى بَٰرَكْنَا حَوْلَهُۥ لِنُرِيَهُۥ مِنْ ءَايَـٰتِنَآ ۚ إِنَّهُۥ هُوَ ٱلسَّمِيعُ ٱلْبَصِيرُ ﴿١﴾\\
\textamh{2.\  } & وَءَاتَيْنَا مُوسَى ٱلْكِتَـٰبَ وَجَعَلْنَـٰهُ هُدًۭى لِّبَنِىٓ إِسْرَٰٓءِيلَ أَلَّا تَتَّخِذُوا۟ مِن دُونِى وَكِيلًۭا ﴿٢﴾\\
\textamh{3.\  } & ذُرِّيَّةَ مَنْ حَمَلْنَا مَعَ نُوحٍ ۚ إِنَّهُۥ كَانَ عَبْدًۭا شَكُورًۭا ﴿٣﴾\\
\textamh{4.\  } & وَقَضَيْنَآ إِلَىٰ بَنِىٓ إِسْرَٰٓءِيلَ فِى ٱلْكِتَـٰبِ لَتُفْسِدُنَّ فِى ٱلْأَرْضِ مَرَّتَيْنِ وَلَتَعْلُنَّ عُلُوًّۭا كَبِيرًۭا ﴿٤﴾\\
\textamh{5.\  } & فَإِذَا جَآءَ وَعْدُ أُولَىٰهُمَا بَعَثْنَا عَلَيْكُمْ عِبَادًۭا لَّنَآ أُو۟لِى بَأْسٍۢ شَدِيدٍۢ فَجَاسُوا۟ خِلَـٰلَ ٱلدِّيَارِ ۚ وَكَانَ وَعْدًۭا مَّفْعُولًۭا ﴿٥﴾\\
\textamh{6.\  } & ثُمَّ رَدَدْنَا لَكُمُ ٱلْكَرَّةَ عَلَيْهِمْ وَأَمْدَدْنَـٰكُم بِأَمْوَٟلٍۢ وَبَنِينَ وَجَعَلْنَـٰكُمْ أَكْثَرَ نَفِيرًا ﴿٦﴾\\
\textamh{7.\  } & إِنْ أَحْسَنتُمْ أَحْسَنتُمْ لِأَنفُسِكُمْ ۖ وَإِنْ أَسَأْتُمْ فَلَهَا ۚ فَإِذَا جَآءَ وَعْدُ ٱلْءَاخِرَةِ لِيَسُۥٓـُٔوا۟ وُجُوهَكُمْ وَلِيَدْخُلُوا۟ ٱلْمَسْجِدَ كَمَا دَخَلُوهُ أَوَّلَ مَرَّةٍۢ وَلِيُتَبِّرُوا۟ مَا عَلَوْا۟ تَتْبِيرًا ﴿٧﴾\\
\textamh{8.\  } & عَسَىٰ رَبُّكُمْ أَن يَرْحَمَكُمْ ۚ وَإِنْ عُدتُّمْ عُدْنَا ۘ وَجَعَلْنَا جَهَنَّمَ لِلْكَـٰفِرِينَ حَصِيرًا ﴿٨﴾\\
\textamh{9.\  } & إِنَّ هَـٰذَا ٱلْقُرْءَانَ يَهْدِى لِلَّتِى هِىَ أَقْوَمُ وَيُبَشِّرُ ٱلْمُؤْمِنِينَ ٱلَّذِينَ يَعْمَلُونَ ٱلصَّـٰلِحَـٰتِ أَنَّ لَهُمْ أَجْرًۭا كَبِيرًۭا ﴿٩﴾\\
\textamh{10.\  } & وَأَنَّ ٱلَّذِينَ لَا يُؤْمِنُونَ بِٱلْءَاخِرَةِ أَعْتَدْنَا لَهُمْ عَذَابًا أَلِيمًۭا ﴿١٠﴾\\
\textamh{11.\  } & وَيَدْعُ ٱلْإِنسَـٰنُ بِٱلشَّرِّ دُعَآءَهُۥ بِٱلْخَيْرِ ۖ وَكَانَ ٱلْإِنسَـٰنُ عَجُولًۭا ﴿١١﴾\\
\textamh{12.\  } & وَجَعَلْنَا ٱلَّيْلَ وَٱلنَّهَارَ ءَايَتَيْنِ ۖ فَمَحَوْنَآ ءَايَةَ ٱلَّيْلِ وَجَعَلْنَآ ءَايَةَ ٱلنَّهَارِ مُبْصِرَةًۭ لِّتَبْتَغُوا۟ فَضْلًۭا مِّن رَّبِّكُمْ وَلِتَعْلَمُوا۟ عَدَدَ ٱلسِّنِينَ وَٱلْحِسَابَ ۚ وَكُلَّ شَىْءٍۢ فَصَّلْنَـٰهُ تَفْصِيلًۭا ﴿١٢﴾\\
\textamh{13.\  } & وَكُلَّ إِنسَـٰنٍ أَلْزَمْنَـٰهُ طَٰٓئِرَهُۥ فِى عُنُقِهِۦ ۖ وَنُخْرِجُ لَهُۥ يَوْمَ ٱلْقِيَـٰمَةِ كِتَـٰبًۭا يَلْقَىٰهُ مَنشُورًا ﴿١٣﴾\\
\textamh{14.\  } & ٱقْرَأْ كِتَـٰبَكَ كَفَىٰ بِنَفْسِكَ ٱلْيَوْمَ عَلَيْكَ حَسِيبًۭا ﴿١٤﴾\\
\textamh{15.\  } & مَّنِ ٱهْتَدَىٰ فَإِنَّمَا يَهْتَدِى لِنَفْسِهِۦ ۖ وَمَن ضَلَّ فَإِنَّمَا يَضِلُّ عَلَيْهَا ۚ وَلَا تَزِرُ وَازِرَةٌۭ وِزْرَ أُخْرَىٰ ۗ وَمَا كُنَّا مُعَذِّبِينَ حَتَّىٰ نَبْعَثَ رَسُولًۭا ﴿١٥﴾\\
\textamh{16.\  } & وَإِذَآ أَرَدْنَآ أَن نُّهْلِكَ قَرْيَةً أَمَرْنَا مُتْرَفِيهَا فَفَسَقُوا۟ فِيهَا فَحَقَّ عَلَيْهَا ٱلْقَوْلُ فَدَمَّرْنَـٰهَا تَدْمِيرًۭا ﴿١٦﴾\\
\textamh{17.\  } & وَكَمْ أَهْلَكْنَا مِنَ ٱلْقُرُونِ مِنۢ بَعْدِ نُوحٍۢ ۗ وَكَفَىٰ بِرَبِّكَ بِذُنُوبِ عِبَادِهِۦ خَبِيرًۢا بَصِيرًۭا ﴿١٧﴾\\
\textamh{18.\  } & مَّن كَانَ يُرِيدُ ٱلْعَاجِلَةَ عَجَّلْنَا لَهُۥ فِيهَا مَا نَشَآءُ لِمَن نُّرِيدُ ثُمَّ جَعَلْنَا لَهُۥ جَهَنَّمَ يَصْلَىٰهَا مَذْمُومًۭا مَّدْحُورًۭا ﴿١٨﴾\\
\textamh{19.\  } & وَمَنْ أَرَادَ ٱلْءَاخِرَةَ وَسَعَىٰ لَهَا سَعْيَهَا وَهُوَ مُؤْمِنٌۭ فَأُو۟لَـٰٓئِكَ كَانَ سَعْيُهُم مَّشْكُورًۭا ﴿١٩﴾\\
\textamh{20.\  } & كُلًّۭا نُّمِدُّ هَـٰٓؤُلَآءِ وَهَـٰٓؤُلَآءِ مِنْ عَطَآءِ رَبِّكَ ۚ وَمَا كَانَ عَطَآءُ رَبِّكَ مَحْظُورًا ﴿٢٠﴾\\
\textamh{21.\  } & ٱنظُرْ كَيْفَ فَضَّلْنَا بَعْضَهُمْ عَلَىٰ بَعْضٍۢ ۚ وَلَلْءَاخِرَةُ أَكْبَرُ دَرَجَٰتٍۢ وَأَكْبَرُ تَفْضِيلًۭا ﴿٢١﴾\\
\textamh{22.\  } & لَّا تَجْعَلْ مَعَ ٱللَّهِ إِلَـٰهًا ءَاخَرَ فَتَقْعُدَ مَذْمُومًۭا مَّخْذُولًۭا ﴿٢٢﴾\\
\textamh{23.\  } & ۞ وَقَضَىٰ رَبُّكَ أَلَّا تَعْبُدُوٓا۟ إِلَّآ إِيَّاهُ وَبِٱلْوَٟلِدَيْنِ إِحْسَـٰنًا ۚ إِمَّا يَبْلُغَنَّ عِندَكَ ٱلْكِبَرَ أَحَدُهُمَآ أَوْ كِلَاهُمَا فَلَا تَقُل لَّهُمَآ أُفٍّۢ وَلَا تَنْهَرْهُمَا وَقُل لَّهُمَا قَوْلًۭا كَرِيمًۭا ﴿٢٣﴾\\
\textamh{24.\  } & وَٱخْفِضْ لَهُمَا جَنَاحَ ٱلذُّلِّ مِنَ ٱلرَّحْمَةِ وَقُل رَّبِّ ٱرْحَمْهُمَا كَمَا رَبَّيَانِى صَغِيرًۭا ﴿٢٤﴾\\
\textamh{25.\  } & رَّبُّكُمْ أَعْلَمُ بِمَا فِى نُفُوسِكُمْ ۚ إِن تَكُونُوا۟ صَـٰلِحِينَ فَإِنَّهُۥ كَانَ لِلْأَوَّٰبِينَ غَفُورًۭا ﴿٢٥﴾\\
\textamh{26.\  } & وَءَاتِ ذَا ٱلْقُرْبَىٰ حَقَّهُۥ وَٱلْمِسْكِينَ وَٱبْنَ ٱلسَّبِيلِ وَلَا تُبَذِّرْ تَبْذِيرًا ﴿٢٦﴾\\
\textamh{27.\  } & إِنَّ ٱلْمُبَذِّرِينَ كَانُوٓا۟ إِخْوَٟنَ ٱلشَّيَـٰطِينِ ۖ وَكَانَ ٱلشَّيْطَٰنُ لِرَبِّهِۦ كَفُورًۭا ﴿٢٧﴾\\
\textamh{28.\  } & وَإِمَّا تُعْرِضَنَّ عَنْهُمُ ٱبْتِغَآءَ رَحْمَةٍۢ مِّن رَّبِّكَ تَرْجُوهَا فَقُل لَّهُمْ قَوْلًۭا مَّيْسُورًۭا ﴿٢٨﴾\\
\textamh{29.\  } & وَلَا تَجْعَلْ يَدَكَ مَغْلُولَةً إِلَىٰ عُنُقِكَ وَلَا تَبْسُطْهَا كُلَّ ٱلْبَسْطِ فَتَقْعُدَ مَلُومًۭا مَّحْسُورًا ﴿٢٩﴾\\
\textamh{30.\  } & إِنَّ رَبَّكَ يَبْسُطُ ٱلرِّزْقَ لِمَن يَشَآءُ وَيَقْدِرُ ۚ إِنَّهُۥ كَانَ بِعِبَادِهِۦ خَبِيرًۢا بَصِيرًۭا ﴿٣٠﴾\\
\textamh{31.\  } & وَلَا تَقْتُلُوٓا۟ أَوْلَـٰدَكُمْ خَشْيَةَ إِمْلَـٰقٍۢ ۖ نَّحْنُ نَرْزُقُهُمْ وَإِيَّاكُمْ ۚ إِنَّ قَتْلَهُمْ كَانَ خِطْـًۭٔا كَبِيرًۭا ﴿٣١﴾\\
\textamh{32.\  } & وَلَا تَقْرَبُوا۟ ٱلزِّنَىٰٓ ۖ إِنَّهُۥ كَانَ فَـٰحِشَةًۭ وَسَآءَ سَبِيلًۭا ﴿٣٢﴾\\
\textamh{33.\  } & وَلَا تَقْتُلُوا۟ ٱلنَّفْسَ ٱلَّتِى حَرَّمَ ٱللَّهُ إِلَّا بِٱلْحَقِّ ۗ وَمَن قُتِلَ مَظْلُومًۭا فَقَدْ جَعَلْنَا لِوَلِيِّهِۦ سُلْطَٰنًۭا فَلَا يُسْرِف فِّى ٱلْقَتْلِ ۖ إِنَّهُۥ كَانَ مَنصُورًۭا ﴿٣٣﴾\\
\textamh{34.\  } & وَلَا تَقْرَبُوا۟ مَالَ ٱلْيَتِيمِ إِلَّا بِٱلَّتِى هِىَ أَحْسَنُ حَتَّىٰ يَبْلُغَ أَشُدَّهُۥ ۚ وَأَوْفُوا۟ بِٱلْعَهْدِ ۖ إِنَّ ٱلْعَهْدَ كَانَ مَسْـُٔولًۭا ﴿٣٤﴾\\
\textamh{35.\  } & وَأَوْفُوا۟ ٱلْكَيْلَ إِذَا كِلْتُمْ وَزِنُوا۟ بِٱلْقِسْطَاسِ ٱلْمُسْتَقِيمِ ۚ ذَٟلِكَ خَيْرٌۭ وَأَحْسَنُ تَأْوِيلًۭا ﴿٣٥﴾\\
\textamh{36.\  } & وَلَا تَقْفُ مَا لَيْسَ لَكَ بِهِۦ عِلْمٌ ۚ إِنَّ ٱلسَّمْعَ وَٱلْبَصَرَ وَٱلْفُؤَادَ كُلُّ أُو۟لَـٰٓئِكَ كَانَ عَنْهُ مَسْـُٔولًۭا ﴿٣٦﴾\\
\textamh{37.\  } & وَلَا تَمْشِ فِى ٱلْأَرْضِ مَرَحًا ۖ إِنَّكَ لَن تَخْرِقَ ٱلْأَرْضَ وَلَن تَبْلُغَ ٱلْجِبَالَ طُولًۭا ﴿٣٧﴾\\
\textamh{38.\  } & كُلُّ ذَٟلِكَ كَانَ سَيِّئُهُۥ عِندَ رَبِّكَ مَكْرُوهًۭا ﴿٣٨﴾\\
\textamh{39.\  } & ذَٟلِكَ مِمَّآ أَوْحَىٰٓ إِلَيْكَ رَبُّكَ مِنَ ٱلْحِكْمَةِ ۗ وَلَا تَجْعَلْ مَعَ ٱللَّهِ إِلَـٰهًا ءَاخَرَ فَتُلْقَىٰ فِى جَهَنَّمَ مَلُومًۭا مَّدْحُورًا ﴿٣٩﴾\\
\textamh{40.\  } & أَفَأَصْفَىٰكُمْ رَبُّكُم بِٱلْبَنِينَ وَٱتَّخَذَ مِنَ ٱلْمَلَـٰٓئِكَةِ إِنَـٰثًا ۚ إِنَّكُمْ لَتَقُولُونَ قَوْلًا عَظِيمًۭا ﴿٤٠﴾\\
\textamh{41.\  } & وَلَقَدْ صَرَّفْنَا فِى هَـٰذَا ٱلْقُرْءَانِ لِيَذَّكَّرُوا۟ وَمَا يَزِيدُهُمْ إِلَّا نُفُورًۭا ﴿٤١﴾\\
\textamh{42.\  } & قُل لَّوْ كَانَ مَعَهُۥٓ ءَالِهَةٌۭ كَمَا يَقُولُونَ إِذًۭا لَّٱبْتَغَوْا۟ إِلَىٰ ذِى ٱلْعَرْشِ سَبِيلًۭا ﴿٤٢﴾\\
\textamh{43.\  } & سُبْحَـٰنَهُۥ وَتَعَـٰلَىٰ عَمَّا يَقُولُونَ عُلُوًّۭا كَبِيرًۭا ﴿٤٣﴾\\
\textamh{44.\  } & تُسَبِّحُ لَهُ ٱلسَّمَـٰوَٟتُ ٱلسَّبْعُ وَٱلْأَرْضُ وَمَن فِيهِنَّ ۚ وَإِن مِّن شَىْءٍ إِلَّا يُسَبِّحُ بِحَمْدِهِۦ وَلَـٰكِن لَّا تَفْقَهُونَ تَسْبِيحَهُمْ ۗ إِنَّهُۥ كَانَ حَلِيمًا غَفُورًۭا ﴿٤٤﴾\\
\textamh{45.\  } & وَإِذَا قَرَأْتَ ٱلْقُرْءَانَ جَعَلْنَا بَيْنَكَ وَبَيْنَ ٱلَّذِينَ لَا يُؤْمِنُونَ بِٱلْءَاخِرَةِ حِجَابًۭا مَّسْتُورًۭا ﴿٤٥﴾\\
\textamh{46.\  } & وَجَعَلْنَا عَلَىٰ قُلُوبِهِمْ أَكِنَّةً أَن يَفْقَهُوهُ وَفِىٓ ءَاذَانِهِمْ وَقْرًۭا ۚ وَإِذَا ذَكَرْتَ رَبَّكَ فِى ٱلْقُرْءَانِ وَحْدَهُۥ وَلَّوْا۟ عَلَىٰٓ أَدْبَٰرِهِمْ نُفُورًۭا ﴿٤٦﴾\\
\textamh{47.\  } & نَّحْنُ أَعْلَمُ بِمَا يَسْتَمِعُونَ بِهِۦٓ إِذْ يَسْتَمِعُونَ إِلَيْكَ وَإِذْ هُمْ نَجْوَىٰٓ إِذْ يَقُولُ ٱلظَّـٰلِمُونَ إِن تَتَّبِعُونَ إِلَّا رَجُلًۭا مَّسْحُورًا ﴿٤٧﴾\\
\textamh{48.\  } & ٱنظُرْ كَيْفَ ضَرَبُوا۟ لَكَ ٱلْأَمْثَالَ فَضَلُّوا۟ فَلَا يَسْتَطِيعُونَ سَبِيلًۭا ﴿٤٨﴾\\
\textamh{49.\  } & وَقَالُوٓا۟ أَءِذَا كُنَّا عِظَـٰمًۭا وَرُفَـٰتًا أَءِنَّا لَمَبْعُوثُونَ خَلْقًۭا جَدِيدًۭا ﴿٤٩﴾\\
\textamh{50.\  } & ۞ قُلْ كُونُوا۟ حِجَارَةً أَوْ حَدِيدًا ﴿٥٠﴾\\
\textamh{51.\  } & أَوْ خَلْقًۭا مِّمَّا يَكْبُرُ فِى صُدُورِكُمْ ۚ فَسَيَقُولُونَ مَن يُعِيدُنَا ۖ قُلِ ٱلَّذِى فَطَرَكُمْ أَوَّلَ مَرَّةٍۢ ۚ فَسَيُنْغِضُونَ إِلَيْكَ رُءُوسَهُمْ وَيَقُولُونَ مَتَىٰ هُوَ ۖ قُلْ عَسَىٰٓ أَن يَكُونَ قَرِيبًۭا ﴿٥١﴾\\
\textamh{52.\  } & يَوْمَ يَدْعُوكُمْ فَتَسْتَجِيبُونَ بِحَمْدِهِۦ وَتَظُنُّونَ إِن لَّبِثْتُمْ إِلَّا قَلِيلًۭا ﴿٥٢﴾\\
\textamh{53.\  } & وَقُل لِّعِبَادِى يَقُولُوا۟ ٱلَّتِى هِىَ أَحْسَنُ ۚ إِنَّ ٱلشَّيْطَٰنَ يَنزَغُ بَيْنَهُمْ ۚ إِنَّ ٱلشَّيْطَٰنَ كَانَ لِلْإِنسَـٰنِ عَدُوًّۭا مُّبِينًۭا ﴿٥٣﴾\\
\textamh{54.\  } & رَّبُّكُمْ أَعْلَمُ بِكُمْ ۖ إِن يَشَأْ يَرْحَمْكُمْ أَوْ إِن يَشَأْ يُعَذِّبْكُمْ ۚ وَمَآ أَرْسَلْنَـٰكَ عَلَيْهِمْ وَكِيلًۭا ﴿٥٤﴾\\
\textamh{55.\  } & وَرَبُّكَ أَعْلَمُ بِمَن فِى ٱلسَّمَـٰوَٟتِ وَٱلْأَرْضِ ۗ وَلَقَدْ فَضَّلْنَا بَعْضَ ٱلنَّبِيِّۦنَ عَلَىٰ بَعْضٍۢ ۖ وَءَاتَيْنَا دَاوُۥدَ زَبُورًۭا ﴿٥٥﴾\\
\textamh{56.\  } & قُلِ ٱدْعُوا۟ ٱلَّذِينَ زَعَمْتُم مِّن دُونِهِۦ فَلَا يَمْلِكُونَ كَشْفَ ٱلضُّرِّ عَنكُمْ وَلَا تَحْوِيلًا ﴿٥٦﴾\\
\textamh{57.\  } & أُو۟لَـٰٓئِكَ ٱلَّذِينَ يَدْعُونَ يَبْتَغُونَ إِلَىٰ رَبِّهِمُ ٱلْوَسِيلَةَ أَيُّهُمْ أَقْرَبُ وَيَرْجُونَ رَحْمَتَهُۥ وَيَخَافُونَ عَذَابَهُۥٓ ۚ إِنَّ عَذَابَ رَبِّكَ كَانَ مَحْذُورًۭا ﴿٥٧﴾\\
\textamh{58.\  } & وَإِن مِّن قَرْيَةٍ إِلَّا نَحْنُ مُهْلِكُوهَا قَبْلَ يَوْمِ ٱلْقِيَـٰمَةِ أَوْ مُعَذِّبُوهَا عَذَابًۭا شَدِيدًۭا ۚ كَانَ ذَٟلِكَ فِى ٱلْكِتَـٰبِ مَسْطُورًۭا ﴿٥٨﴾\\
\textamh{59.\  } & وَمَا مَنَعَنَآ أَن نُّرْسِلَ بِٱلْءَايَـٰتِ إِلَّآ أَن كَذَّبَ بِهَا ٱلْأَوَّلُونَ ۚ وَءَاتَيْنَا ثَمُودَ ٱلنَّاقَةَ مُبْصِرَةًۭ فَظَلَمُوا۟ بِهَا ۚ وَمَا نُرْسِلُ بِٱلْءَايَـٰتِ إِلَّا تَخْوِيفًۭا ﴿٥٩﴾\\
\textamh{60.\  } & وَإِذْ قُلْنَا لَكَ إِنَّ رَبَّكَ أَحَاطَ بِٱلنَّاسِ ۚ وَمَا جَعَلْنَا ٱلرُّءْيَا ٱلَّتِىٓ أَرَيْنَـٰكَ إِلَّا فِتْنَةًۭ لِّلنَّاسِ وَٱلشَّجَرَةَ ٱلْمَلْعُونَةَ فِى ٱلْقُرْءَانِ ۚ وَنُخَوِّفُهُمْ فَمَا يَزِيدُهُمْ إِلَّا طُغْيَـٰنًۭا كَبِيرًۭا ﴿٦٠﴾\\
\textamh{61.\  } & وَإِذْ قُلْنَا لِلْمَلَـٰٓئِكَةِ ٱسْجُدُوا۟ لِءَادَمَ فَسَجَدُوٓا۟ إِلَّآ إِبْلِيسَ قَالَ ءَأَسْجُدُ لِمَنْ خَلَقْتَ طِينًۭا ﴿٦١﴾\\
\textamh{62.\  } & قَالَ أَرَءَيْتَكَ هَـٰذَا ٱلَّذِى كَرَّمْتَ عَلَىَّ لَئِنْ أَخَّرْتَنِ إِلَىٰ يَوْمِ ٱلْقِيَـٰمَةِ لَأَحْتَنِكَنَّ ذُرِّيَّتَهُۥٓ إِلَّا قَلِيلًۭا ﴿٦٢﴾\\
\textamh{63.\  } & قَالَ ٱذْهَبْ فَمَن تَبِعَكَ مِنْهُمْ فَإِنَّ جَهَنَّمَ جَزَآؤُكُمْ جَزَآءًۭ مَّوْفُورًۭا ﴿٦٣﴾\\
\textamh{64.\  } & وَٱسْتَفْزِزْ مَنِ ٱسْتَطَعْتَ مِنْهُم بِصَوْتِكَ وَأَجْلِبْ عَلَيْهِم بِخَيْلِكَ وَرَجِلِكَ وَشَارِكْهُمْ فِى ٱلْأَمْوَٟلِ وَٱلْأَوْلَـٰدِ وَعِدْهُمْ ۚ وَمَا يَعِدُهُمُ ٱلشَّيْطَٰنُ إِلَّا غُرُورًا ﴿٦٤﴾\\
\textamh{65.\  } & إِنَّ عِبَادِى لَيْسَ لَكَ عَلَيْهِمْ سُلْطَٰنٌۭ ۚ وَكَفَىٰ بِرَبِّكَ وَكِيلًۭا ﴿٦٥﴾\\
\textamh{66.\  } & رَّبُّكُمُ ٱلَّذِى يُزْجِى لَكُمُ ٱلْفُلْكَ فِى ٱلْبَحْرِ لِتَبْتَغُوا۟ مِن فَضْلِهِۦٓ ۚ إِنَّهُۥ كَانَ بِكُمْ رَحِيمًۭا ﴿٦٦﴾\\
\textamh{67.\  } & وَإِذَا مَسَّكُمُ ٱلضُّرُّ فِى ٱلْبَحْرِ ضَلَّ مَن تَدْعُونَ إِلَّآ إِيَّاهُ ۖ فَلَمَّا نَجَّىٰكُمْ إِلَى ٱلْبَرِّ أَعْرَضْتُمْ ۚ وَكَانَ ٱلْإِنسَـٰنُ كَفُورًا ﴿٦٧﴾\\
\textamh{68.\  } & أَفَأَمِنتُمْ أَن يَخْسِفَ بِكُمْ جَانِبَ ٱلْبَرِّ أَوْ يُرْسِلَ عَلَيْكُمْ حَاصِبًۭا ثُمَّ لَا تَجِدُوا۟ لَكُمْ وَكِيلًا ﴿٦٨﴾\\
\textamh{69.\  } & أَمْ أَمِنتُمْ أَن يُعِيدَكُمْ فِيهِ تَارَةً أُخْرَىٰ فَيُرْسِلَ عَلَيْكُمْ قَاصِفًۭا مِّنَ ٱلرِّيحِ فَيُغْرِقَكُم بِمَا كَفَرْتُمْ ۙ ثُمَّ لَا تَجِدُوا۟ لَكُمْ عَلَيْنَا بِهِۦ تَبِيعًۭا ﴿٦٩﴾\\
\textamh{70.\  } & ۞ وَلَقَدْ كَرَّمْنَا بَنِىٓ ءَادَمَ وَحَمَلْنَـٰهُمْ فِى ٱلْبَرِّ وَٱلْبَحْرِ وَرَزَقْنَـٰهُم مِّنَ ٱلطَّيِّبَٰتِ وَفَضَّلْنَـٰهُمْ عَلَىٰ كَثِيرٍۢ مِّمَّنْ خَلَقْنَا تَفْضِيلًۭا ﴿٧٠﴾\\
\textamh{71.\  } & يَوْمَ نَدْعُوا۟ كُلَّ أُنَاسٍۭ بِإِمَـٰمِهِمْ ۖ فَمَنْ أُوتِىَ كِتَـٰبَهُۥ بِيَمِينِهِۦ فَأُو۟لَـٰٓئِكَ يَقْرَءُونَ كِتَـٰبَهُمْ وَلَا يُظْلَمُونَ فَتِيلًۭا ﴿٧١﴾\\
\textamh{72.\  } & وَمَن كَانَ فِى هَـٰذِهِۦٓ أَعْمَىٰ فَهُوَ فِى ٱلْءَاخِرَةِ أَعْمَىٰ وَأَضَلُّ سَبِيلًۭا ﴿٧٢﴾\\
\textamh{73.\  } & وَإِن كَادُوا۟ لَيَفْتِنُونَكَ عَنِ ٱلَّذِىٓ أَوْحَيْنَآ إِلَيْكَ لِتَفْتَرِىَ عَلَيْنَا غَيْرَهُۥ ۖ وَإِذًۭا لَّٱتَّخَذُوكَ خَلِيلًۭا ﴿٧٣﴾\\
\textamh{74.\  } & وَلَوْلَآ أَن ثَبَّتْنَـٰكَ لَقَدْ كِدتَّ تَرْكَنُ إِلَيْهِمْ شَيْـًۭٔا قَلِيلًا ﴿٧٤﴾\\
\textamh{75.\  } & إِذًۭا لَّأَذَقْنَـٰكَ ضِعْفَ ٱلْحَيَوٰةِ وَضِعْفَ ٱلْمَمَاتِ ثُمَّ لَا تَجِدُ لَكَ عَلَيْنَا نَصِيرًۭا ﴿٧٥﴾\\
\textamh{76.\  } & وَإِن كَادُوا۟ لَيَسْتَفِزُّونَكَ مِنَ ٱلْأَرْضِ لِيُخْرِجُوكَ مِنْهَا ۖ وَإِذًۭا لَّا يَلْبَثُونَ خِلَـٰفَكَ إِلَّا قَلِيلًۭا ﴿٧٦﴾\\
\textamh{77.\  } & سُنَّةَ مَن قَدْ أَرْسَلْنَا قَبْلَكَ مِن رُّسُلِنَا ۖ وَلَا تَجِدُ لِسُنَّتِنَا تَحْوِيلًا ﴿٧٧﴾\\
\textamh{78.\  } & أَقِمِ ٱلصَّلَوٰةَ لِدُلُوكِ ٱلشَّمْسِ إِلَىٰ غَسَقِ ٱلَّيْلِ وَقُرْءَانَ ٱلْفَجْرِ ۖ إِنَّ قُرْءَانَ ٱلْفَجْرِ كَانَ مَشْهُودًۭا ﴿٧٨﴾\\
\textamh{79.\  } & وَمِنَ ٱلَّيْلِ فَتَهَجَّدْ بِهِۦ نَافِلَةًۭ لَّكَ عَسَىٰٓ أَن يَبْعَثَكَ رَبُّكَ مَقَامًۭا مَّحْمُودًۭا ﴿٧٩﴾\\
\textamh{80.\  } & وَقُل رَّبِّ أَدْخِلْنِى مُدْخَلَ صِدْقٍۢ وَأَخْرِجْنِى مُخْرَجَ صِدْقٍۢ وَٱجْعَل لِّى مِن لَّدُنكَ سُلْطَٰنًۭا نَّصِيرًۭا ﴿٨٠﴾\\
\textamh{81.\  } & وَقُلْ جَآءَ ٱلْحَقُّ وَزَهَقَ ٱلْبَٰطِلُ ۚ إِنَّ ٱلْبَٰطِلَ كَانَ زَهُوقًۭا ﴿٨١﴾\\
\textamh{82.\  } & وَنُنَزِّلُ مِنَ ٱلْقُرْءَانِ مَا هُوَ شِفَآءٌۭ وَرَحْمَةٌۭ لِّلْمُؤْمِنِينَ ۙ وَلَا يَزِيدُ ٱلظَّـٰلِمِينَ إِلَّا خَسَارًۭا ﴿٨٢﴾\\
\textamh{83.\  } & وَإِذَآ أَنْعَمْنَا عَلَى ٱلْإِنسَـٰنِ أَعْرَضَ وَنَـَٔا بِجَانِبِهِۦ ۖ وَإِذَا مَسَّهُ ٱلشَّرُّ كَانَ يَـُٔوسًۭا ﴿٨٣﴾\\
\textamh{84.\  } & قُلْ كُلٌّۭ يَعْمَلُ عَلَىٰ شَاكِلَتِهِۦ فَرَبُّكُمْ أَعْلَمُ بِمَنْ هُوَ أَهْدَىٰ سَبِيلًۭا ﴿٨٤﴾\\
\textamh{85.\  } & وَيَسْـَٔلُونَكَ عَنِ ٱلرُّوحِ ۖ قُلِ ٱلرُّوحُ مِنْ أَمْرِ رَبِّى وَمَآ أُوتِيتُم مِّنَ ٱلْعِلْمِ إِلَّا قَلِيلًۭا ﴿٨٥﴾\\
\textamh{86.\  } & وَلَئِن شِئْنَا لَنَذْهَبَنَّ بِٱلَّذِىٓ أَوْحَيْنَآ إِلَيْكَ ثُمَّ لَا تَجِدُ لَكَ بِهِۦ عَلَيْنَا وَكِيلًا ﴿٨٦﴾\\
\textamh{87.\  } & إِلَّا رَحْمَةًۭ مِّن رَّبِّكَ ۚ إِنَّ فَضْلَهُۥ كَانَ عَلَيْكَ كَبِيرًۭا ﴿٨٧﴾\\
\textamh{88.\  } & قُل لَّئِنِ ٱجْتَمَعَتِ ٱلْإِنسُ وَٱلْجِنُّ عَلَىٰٓ أَن يَأْتُوا۟ بِمِثْلِ هَـٰذَا ٱلْقُرْءَانِ لَا يَأْتُونَ بِمِثْلِهِۦ وَلَوْ كَانَ بَعْضُهُمْ لِبَعْضٍۢ ظَهِيرًۭا ﴿٨٨﴾\\
\textamh{89.\  } & وَلَقَدْ صَرَّفْنَا لِلنَّاسِ فِى هَـٰذَا ٱلْقُرْءَانِ مِن كُلِّ مَثَلٍۢ فَأَبَىٰٓ أَكْثَرُ ٱلنَّاسِ إِلَّا كُفُورًۭا ﴿٨٩﴾\\
\textamh{90.\  } & وَقَالُوا۟ لَن نُّؤْمِنَ لَكَ حَتَّىٰ تَفْجُرَ لَنَا مِنَ ٱلْأَرْضِ يَنۢبُوعًا ﴿٩٠﴾\\
\textamh{91.\  } & أَوْ تَكُونَ لَكَ جَنَّةٌۭ مِّن نَّخِيلٍۢ وَعِنَبٍۢ فَتُفَجِّرَ ٱلْأَنْهَـٰرَ خِلَـٰلَهَا تَفْجِيرًا ﴿٩١﴾\\
\textamh{92.\  } & أَوْ تُسْقِطَ ٱلسَّمَآءَ كَمَا زَعَمْتَ عَلَيْنَا كِسَفًا أَوْ تَأْتِىَ بِٱللَّهِ وَٱلْمَلَـٰٓئِكَةِ قَبِيلًا ﴿٩٢﴾\\
\textamh{93.\  } & أَوْ يَكُونَ لَكَ بَيْتٌۭ مِّن زُخْرُفٍ أَوْ تَرْقَىٰ فِى ٱلسَّمَآءِ وَلَن نُّؤْمِنَ لِرُقِيِّكَ حَتَّىٰ تُنَزِّلَ عَلَيْنَا كِتَـٰبًۭا نَّقْرَؤُهُۥ ۗ قُلْ سُبْحَانَ رَبِّى هَلْ كُنتُ إِلَّا بَشَرًۭا رَّسُولًۭا ﴿٩٣﴾\\
\textamh{94.\  } & وَمَا مَنَعَ ٱلنَّاسَ أَن يُؤْمِنُوٓا۟ إِذْ جَآءَهُمُ ٱلْهُدَىٰٓ إِلَّآ أَن قَالُوٓا۟ أَبَعَثَ ٱللَّهُ بَشَرًۭا رَّسُولًۭا ﴿٩٤﴾\\
\textamh{95.\  } & قُل لَّوْ كَانَ فِى ٱلْأَرْضِ مَلَـٰٓئِكَةٌۭ يَمْشُونَ مُطْمَئِنِّينَ لَنَزَّلْنَا عَلَيْهِم مِّنَ ٱلسَّمَآءِ مَلَكًۭا رَّسُولًۭا ﴿٩٥﴾\\
\textamh{96.\  } & قُلْ كَفَىٰ بِٱللَّهِ شَهِيدًۢا بَيْنِى وَبَيْنَكُمْ ۚ إِنَّهُۥ كَانَ بِعِبَادِهِۦ خَبِيرًۢا بَصِيرًۭا ﴿٩٦﴾\\
\textamh{97.\  } & وَمَن يَهْدِ ٱللَّهُ فَهُوَ ٱلْمُهْتَدِ ۖ وَمَن يُضْلِلْ فَلَن تَجِدَ لَهُمْ أَوْلِيَآءَ مِن دُونِهِۦ ۖ وَنَحْشُرُهُمْ يَوْمَ ٱلْقِيَـٰمَةِ عَلَىٰ وُجُوهِهِمْ عُمْيًۭا وَبُكْمًۭا وَصُمًّۭا ۖ مَّأْوَىٰهُمْ جَهَنَّمُ ۖ كُلَّمَا خَبَتْ زِدْنَـٰهُمْ سَعِيرًۭا ﴿٩٧﴾\\
\textamh{98.\  } & ذَٟلِكَ جَزَآؤُهُم بِأَنَّهُمْ كَفَرُوا۟ بِـَٔايَـٰتِنَا وَقَالُوٓا۟ أَءِذَا كُنَّا عِظَـٰمًۭا وَرُفَـٰتًا أَءِنَّا لَمَبْعُوثُونَ خَلْقًۭا جَدِيدًا ﴿٩٨﴾\\
\textamh{99.\  } & ۞ أَوَلَمْ يَرَوْا۟ أَنَّ ٱللَّهَ ٱلَّذِى خَلَقَ ٱلسَّمَـٰوَٟتِ وَٱلْأَرْضَ قَادِرٌ عَلَىٰٓ أَن يَخْلُقَ مِثْلَهُمْ وَجَعَلَ لَهُمْ أَجَلًۭا لَّا رَيْبَ فِيهِ فَأَبَى ٱلظَّـٰلِمُونَ إِلَّا كُفُورًۭا ﴿٩٩﴾\\
\textamh{100.\  } & قُل لَّوْ أَنتُمْ تَمْلِكُونَ خَزَآئِنَ رَحْمَةِ رَبِّىٓ إِذًۭا لَّأَمْسَكْتُمْ خَشْيَةَ ٱلْإِنفَاقِ ۚ وَكَانَ ٱلْإِنسَـٰنُ قَتُورًۭا ﴿١٠٠﴾\\
\textamh{101.\  } & وَلَقَدْ ءَاتَيْنَا مُوسَىٰ تِسْعَ ءَايَـٰتٍۭ بَيِّنَـٰتٍۢ ۖ فَسْـَٔلْ بَنِىٓ إِسْرَٰٓءِيلَ إِذْ جَآءَهُمْ فَقَالَ لَهُۥ فِرْعَوْنُ إِنِّى لَأَظُنُّكَ يَـٰمُوسَىٰ مَسْحُورًۭا ﴿١٠١﴾\\
\textamh{102.\  } & قَالَ لَقَدْ عَلِمْتَ مَآ أَنزَلَ هَـٰٓؤُلَآءِ إِلَّا رَبُّ ٱلسَّمَـٰوَٟتِ وَٱلْأَرْضِ بَصَآئِرَ وَإِنِّى لَأَظُنُّكَ يَـٰفِرْعَوْنُ مَثْبُورًۭا ﴿١٠٢﴾\\
\textamh{103.\  } & فَأَرَادَ أَن يَسْتَفِزَّهُم مِّنَ ٱلْأَرْضِ فَأَغْرَقْنَـٰهُ وَمَن مَّعَهُۥ جَمِيعًۭا ﴿١٠٣﴾\\
\textamh{104.\  } & وَقُلْنَا مِنۢ بَعْدِهِۦ لِبَنِىٓ إِسْرَٰٓءِيلَ ٱسْكُنُوا۟ ٱلْأَرْضَ فَإِذَا جَآءَ وَعْدُ ٱلْءَاخِرَةِ جِئْنَا بِكُمْ لَفِيفًۭا ﴿١٠٤﴾\\
\textamh{105.\  } & وَبِٱلْحَقِّ أَنزَلْنَـٰهُ وَبِٱلْحَقِّ نَزَلَ ۗ وَمَآ أَرْسَلْنَـٰكَ إِلَّا مُبَشِّرًۭا وَنَذِيرًۭا ﴿١٠٥﴾\\
\textamh{106.\  } & وَقُرْءَانًۭا فَرَقْنَـٰهُ لِتَقْرَأَهُۥ عَلَى ٱلنَّاسِ عَلَىٰ مُكْثٍۢ وَنَزَّلْنَـٰهُ تَنزِيلًۭا ﴿١٠٦﴾\\
\textamh{107.\  } & قُلْ ءَامِنُوا۟ بِهِۦٓ أَوْ لَا تُؤْمِنُوٓا۟ ۚ إِنَّ ٱلَّذِينَ أُوتُوا۟ ٱلْعِلْمَ مِن قَبْلِهِۦٓ إِذَا يُتْلَىٰ عَلَيْهِمْ يَخِرُّونَ لِلْأَذْقَانِ سُجَّدًۭا ﴿١٠٧﴾\\
\textamh{108.\  } & وَيَقُولُونَ سُبْحَـٰنَ رَبِّنَآ إِن كَانَ وَعْدُ رَبِّنَا لَمَفْعُولًۭا ﴿١٠٨﴾\\
\textamh{109.\  } & وَيَخِرُّونَ لِلْأَذْقَانِ يَبْكُونَ وَيَزِيدُهُمْ خُشُوعًۭا ۩ ﴿١٠٩﴾\\
\textamh{110.\  } & قُلِ ٱدْعُوا۟ ٱللَّهَ أَوِ ٱدْعُوا۟ ٱلرَّحْمَـٰنَ ۖ أَيًّۭا مَّا تَدْعُوا۟ فَلَهُ ٱلْأَسْمَآءُ ٱلْحُسْنَىٰ ۚ وَلَا تَجْهَرْ بِصَلَاتِكَ وَلَا تُخَافِتْ بِهَا وَٱبْتَغِ بَيْنَ ذَٟلِكَ سَبِيلًۭا ﴿١١٠﴾\\
\textamh{111.\  } & وَقُلِ ٱلْحَمْدُ لِلَّهِ ٱلَّذِى لَمْ يَتَّخِذْ وَلَدًۭا وَلَمْ يَكُن لَّهُۥ شَرِيكٌۭ فِى ٱلْمُلْكِ وَلَمْ يَكُن لَّهُۥ وَلِىٌّۭ مِّنَ ٱلذُّلِّ ۖ وَكَبِّرْهُ تَكْبِيرًۢا ﴿١١١﴾\\
\end{longtable}
\clearpage
%% License: BSD style (Berkley) (i.e. Put the Copyright owner's name always)
%% Writer and Copyright (to): Bewketu(Bilal) Tadilo (2016-17)
\centering\section{\LR{\textamharic{ሱራቱ አልካህፍ -}  \RL{سوره  الكهف}}}
\begin{longtable}{%
  @{}
    p{.5\textwidth}
  @{~~~~~~~~~~~~~}
    p{.5\textwidth}
    @{}
}
\nopagebreak
\textamh{ቢስሚላሂ አራህመኒ ራሂይም } &  بِسْمِ ٱللَّهِ ٱلرَّحْمَـٰنِ ٱلرَّحِيمِ\\
\textamh{1.\  } &  ٱلْحَمْدُ لِلَّهِ ٱلَّذِىٓ أَنزَلَ عَلَىٰ عَبْدِهِ ٱلْكِتَـٰبَ وَلَمْ يَجْعَل لَّهُۥ عِوَجَا ۜ ﴿١﴾\\
\textamh{2.\  } & قَيِّمًۭا لِّيُنذِرَ بَأْسًۭا شَدِيدًۭا مِّن لَّدُنْهُ وَيُبَشِّرَ ٱلْمُؤْمِنِينَ ٱلَّذِينَ يَعْمَلُونَ ٱلصَّـٰلِحَـٰتِ أَنَّ لَهُمْ أَجْرًا حَسَنًۭا ﴿٢﴾\\
\textamh{3.\  } & مَّٰكِثِينَ فِيهِ أَبَدًۭا ﴿٣﴾\\
\textamh{4.\  } & وَيُنذِرَ ٱلَّذِينَ قَالُوا۟ ٱتَّخَذَ ٱللَّهُ وَلَدًۭا ﴿٤﴾\\
\textamh{5.\  } & مَّا لَهُم بِهِۦ مِنْ عِلْمٍۢ وَلَا لِءَابَآئِهِمْ ۚ كَبُرَتْ كَلِمَةًۭ تَخْرُجُ مِنْ أَفْوَٟهِهِمْ ۚ إِن يَقُولُونَ إِلَّا كَذِبًۭا ﴿٥﴾\\
\textamh{6.\  } & فَلَعَلَّكَ بَٰخِعٌۭ نَّفْسَكَ عَلَىٰٓ ءَاثَـٰرِهِمْ إِن لَّمْ يُؤْمِنُوا۟ بِهَـٰذَا ٱلْحَدِيثِ أَسَفًا ﴿٦﴾\\
\textamh{7.\  } & إِنَّا جَعَلْنَا مَا عَلَى ٱلْأَرْضِ زِينَةًۭ لَّهَا لِنَبْلُوَهُمْ أَيُّهُمْ أَحْسَنُ عَمَلًۭا ﴿٧﴾\\
\textamh{8.\  } & وَإِنَّا لَجَٰعِلُونَ مَا عَلَيْهَا صَعِيدًۭا جُرُزًا ﴿٨﴾\\
\textamh{9.\  } & أَمْ حَسِبْتَ أَنَّ أَصْحَـٰبَ ٱلْكَهْفِ وَٱلرَّقِيمِ كَانُوا۟ مِنْ ءَايَـٰتِنَا عَجَبًا ﴿٩﴾\\
\textamh{10.\  } & إِذْ أَوَى ٱلْفِتْيَةُ إِلَى ٱلْكَهْفِ فَقَالُوا۟ رَبَّنَآ ءَاتِنَا مِن لَّدُنكَ رَحْمَةًۭ وَهَيِّئْ لَنَا مِنْ أَمْرِنَا رَشَدًۭا ﴿١٠﴾\\
\textamh{11.\  } & فَضَرَبْنَا عَلَىٰٓ ءَاذَانِهِمْ فِى ٱلْكَهْفِ سِنِينَ عَدَدًۭا ﴿١١﴾\\
\textamh{12.\  } & ثُمَّ بَعَثْنَـٰهُمْ لِنَعْلَمَ أَىُّ ٱلْحِزْبَيْنِ أَحْصَىٰ لِمَا لَبِثُوٓا۟ أَمَدًۭا ﴿١٢﴾\\
\textamh{13.\  } & نَّحْنُ نَقُصُّ عَلَيْكَ نَبَأَهُم بِٱلْحَقِّ ۚ إِنَّهُمْ فِتْيَةٌ ءَامَنُوا۟ بِرَبِّهِمْ وَزِدْنَـٰهُمْ هُدًۭى ﴿١٣﴾\\
\textamh{14.\  } & وَرَبَطْنَا عَلَىٰ قُلُوبِهِمْ إِذْ قَامُوا۟ فَقَالُوا۟ رَبُّنَا رَبُّ ٱلسَّمَـٰوَٟتِ وَٱلْأَرْضِ لَن نَّدْعُوَا۟ مِن دُونِهِۦٓ إِلَـٰهًۭا ۖ لَّقَدْ قُلْنَآ إِذًۭا شَطَطًا ﴿١٤﴾\\
\textamh{15.\  } & هَـٰٓؤُلَآءِ قَوْمُنَا ٱتَّخَذُوا۟ مِن دُونِهِۦٓ ءَالِهَةًۭ ۖ لَّوْلَا يَأْتُونَ عَلَيْهِم بِسُلْطَٰنٍۭ بَيِّنٍۢ ۖ فَمَنْ أَظْلَمُ مِمَّنِ ٱفْتَرَىٰ عَلَى ٱللَّهِ كَذِبًۭا ﴿١٥﴾\\
\textamh{16.\  } & وَإِذِ ٱعْتَزَلْتُمُوهُمْ وَمَا يَعْبُدُونَ إِلَّا ٱللَّهَ فَأْوُۥٓا۟ إِلَى ٱلْكَهْفِ يَنشُرْ لَكُمْ رَبُّكُم مِّن رَّحْمَتِهِۦ وَيُهَيِّئْ لَكُم مِّنْ أَمْرِكُم مِّرْفَقًۭا ﴿١٦﴾\\
\textamh{17.\  } & ۞ وَتَرَى ٱلشَّمْسَ إِذَا طَلَعَت تَّزَٰوَرُ عَن كَهْفِهِمْ ذَاتَ ٱلْيَمِينِ وَإِذَا غَرَبَت تَّقْرِضُهُمْ ذَاتَ ٱلشِّمَالِ وَهُمْ فِى فَجْوَةٍۢ مِّنْهُ ۚ ذَٟلِكَ مِنْ ءَايَـٰتِ ٱللَّهِ ۗ مَن يَهْدِ ٱللَّهُ فَهُوَ ٱلْمُهْتَدِ ۖ وَمَن يُضْلِلْ فَلَن تَجِدَ لَهُۥ وَلِيًّۭا مُّرْشِدًۭا ﴿١٧﴾\\
\textamh{18.\  } & وَتَحْسَبُهُمْ أَيْقَاظًۭا وَهُمْ رُقُودٌۭ ۚ وَنُقَلِّبُهُمْ ذَاتَ ٱلْيَمِينِ وَذَاتَ ٱلشِّمَالِ ۖ وَكَلْبُهُم بَٰسِطٌۭ ذِرَاعَيْهِ بِٱلْوَصِيدِ ۚ لَوِ ٱطَّلَعْتَ عَلَيْهِمْ لَوَلَّيْتَ مِنْهُمْ فِرَارًۭا وَلَمُلِئْتَ مِنْهُمْ رُعْبًۭا ﴿١٨﴾\\
\textamh{19.\  } & وَكَذَٟلِكَ بَعَثْنَـٰهُمْ لِيَتَسَآءَلُوا۟ بَيْنَهُمْ ۚ قَالَ قَآئِلٌۭ مِّنْهُمْ كَمْ لَبِثْتُمْ ۖ قَالُوا۟ لَبِثْنَا يَوْمًا أَوْ بَعْضَ يَوْمٍۢ ۚ قَالُوا۟ رَبُّكُمْ أَعْلَمُ بِمَا لَبِثْتُمْ فَٱبْعَثُوٓا۟ أَحَدَكُم بِوَرِقِكُمْ هَـٰذِهِۦٓ إِلَى ٱلْمَدِينَةِ فَلْيَنظُرْ أَيُّهَآ أَزْكَىٰ طَعَامًۭا فَلْيَأْتِكُم بِرِزْقٍۢ مِّنْهُ وَلْيَتَلَطَّفْ وَلَا يُشْعِرَنَّ بِكُمْ أَحَدًا ﴿١٩﴾\\
\textamh{20.\  } & إِنَّهُمْ إِن يَظْهَرُوا۟ عَلَيْكُمْ يَرْجُمُوكُمْ أَوْ يُعِيدُوكُمْ فِى مِلَّتِهِمْ وَلَن تُفْلِحُوٓا۟ إِذًا أَبَدًۭا ﴿٢٠﴾\\
\textamh{21.\  } & وَكَذَٟلِكَ أَعْثَرْنَا عَلَيْهِمْ لِيَعْلَمُوٓا۟ أَنَّ وَعْدَ ٱللَّهِ حَقٌّۭ وَأَنَّ ٱلسَّاعَةَ لَا رَيْبَ فِيهَآ إِذْ يَتَنَـٰزَعُونَ بَيْنَهُمْ أَمْرَهُمْ ۖ فَقَالُوا۟ ٱبْنُوا۟ عَلَيْهِم بُنْيَـٰنًۭا ۖ رَّبُّهُمْ أَعْلَمُ بِهِمْ ۚ قَالَ ٱلَّذِينَ غَلَبُوا۟ عَلَىٰٓ أَمْرِهِمْ لَنَتَّخِذَنَّ عَلَيْهِم مَّسْجِدًۭا ﴿٢١﴾\\
\textamh{22.\  } & سَيَقُولُونَ ثَلَـٰثَةٌۭ رَّابِعُهُمْ كَلْبُهُمْ وَيَقُولُونَ خَمْسَةٌۭ سَادِسُهُمْ كَلْبُهُمْ رَجْمًۢا بِٱلْغَيْبِ ۖ وَيَقُولُونَ سَبْعَةٌۭ وَثَامِنُهُمْ كَلْبُهُمْ ۚ قُل رَّبِّىٓ أَعْلَمُ بِعِدَّتِهِم مَّا يَعْلَمُهُمْ إِلَّا قَلِيلٌۭ ۗ فَلَا تُمَارِ فِيهِمْ إِلَّا مِرَآءًۭ ظَـٰهِرًۭا وَلَا تَسْتَفْتِ فِيهِم مِّنْهُمْ أَحَدًۭا ﴿٢٢﴾\\
\textamh{23.\  } & وَلَا تَقُولَنَّ لِشَا۟ىْءٍ إِنِّى فَاعِلٌۭ ذَٟلِكَ غَدًا ﴿٢٣﴾\\
\textamh{24.\  } & إِلَّآ أَن يَشَآءَ ٱللَّهُ ۚ وَٱذْكُر رَّبَّكَ إِذَا نَسِيتَ وَقُلْ عَسَىٰٓ أَن يَهْدِيَنِ رَبِّى لِأَقْرَبَ مِنْ هَـٰذَا رَشَدًۭا ﴿٢٤﴾\\
\textamh{25.\  } & وَلَبِثُوا۟ فِى كَهْفِهِمْ ثَلَـٰثَ مِا۟ئَةٍۢ سِنِينَ وَٱزْدَادُوا۟ تِسْعًۭا ﴿٢٥﴾\\
\textamh{26.\  } & قُلِ ٱللَّهُ أَعْلَمُ بِمَا لَبِثُوا۟ ۖ لَهُۥ غَيْبُ ٱلسَّمَـٰوَٟتِ وَٱلْأَرْضِ ۖ أَبْصِرْ بِهِۦ وَأَسْمِعْ ۚ مَا لَهُم مِّن دُونِهِۦ مِن وَلِىٍّۢ وَلَا يُشْرِكُ فِى حُكْمِهِۦٓ أَحَدًۭا ﴿٢٦﴾\\
\textamh{27.\  } & وَٱتْلُ مَآ أُوحِىَ إِلَيْكَ مِن كِتَابِ رَبِّكَ ۖ لَا مُبَدِّلَ لِكَلِمَـٰتِهِۦ وَلَن تَجِدَ مِن دُونِهِۦ مُلْتَحَدًۭا ﴿٢٧﴾\\
\textamh{28.\  } & وَٱصْبِرْ نَفْسَكَ مَعَ ٱلَّذِينَ يَدْعُونَ رَبَّهُم بِٱلْغَدَوٰةِ وَٱلْعَشِىِّ يُرِيدُونَ وَجْهَهُۥ ۖ وَلَا تَعْدُ عَيْنَاكَ عَنْهُمْ تُرِيدُ زِينَةَ ٱلْحَيَوٰةِ ٱلدُّنْيَا ۖ وَلَا تُطِعْ مَنْ أَغْفَلْنَا قَلْبَهُۥ عَن ذِكْرِنَا وَٱتَّبَعَ هَوَىٰهُ وَكَانَ أَمْرُهُۥ فُرُطًۭا ﴿٢٨﴾\\
\textamh{29.\  } & وَقُلِ ٱلْحَقُّ مِن رَّبِّكُمْ ۖ فَمَن شَآءَ فَلْيُؤْمِن وَمَن شَآءَ فَلْيَكْفُرْ ۚ إِنَّآ أَعْتَدْنَا لِلظَّـٰلِمِينَ نَارًا أَحَاطَ بِهِمْ سُرَادِقُهَا ۚ وَإِن يَسْتَغِيثُوا۟ يُغَاثُوا۟ بِمَآءٍۢ كَٱلْمُهْلِ يَشْوِى ٱلْوُجُوهَ ۚ بِئْسَ ٱلشَّرَابُ وَسَآءَتْ مُرْتَفَقًا ﴿٢٩﴾\\
\textamh{30.\  } & إِنَّ ٱلَّذِينَ ءَامَنُوا۟ وَعَمِلُوا۟ ٱلصَّـٰلِحَـٰتِ إِنَّا لَا نُضِيعُ أَجْرَ مَنْ أَحْسَنَ عَمَلًا ﴿٣٠﴾\\
\textamh{31.\  } & أُو۟لَـٰٓئِكَ لَهُمْ جَنَّـٰتُ عَدْنٍۢ تَجْرِى مِن تَحْتِهِمُ ٱلْأَنْهَـٰرُ يُحَلَّوْنَ فِيهَا مِنْ أَسَاوِرَ مِن ذَهَبٍۢ وَيَلْبَسُونَ ثِيَابًا خُضْرًۭا مِّن سُندُسٍۢ وَإِسْتَبْرَقٍۢ مُّتَّكِـِٔينَ فِيهَا عَلَى ٱلْأَرَآئِكِ ۚ نِعْمَ ٱلثَّوَابُ وَحَسُنَتْ مُرْتَفَقًۭا ﴿٣١﴾\\
\textamh{32.\  } & ۞ وَٱضْرِبْ لَهُم مَّثَلًۭا رَّجُلَيْنِ جَعَلْنَا لِأَحَدِهِمَا جَنَّتَيْنِ مِنْ أَعْنَـٰبٍۢ وَحَفَفْنَـٰهُمَا بِنَخْلٍۢ وَجَعَلْنَا بَيْنَهُمَا زَرْعًۭا ﴿٣٢﴾\\
\textamh{33.\  } & كِلْتَا ٱلْجَنَّتَيْنِ ءَاتَتْ أُكُلَهَا وَلَمْ تَظْلِم مِّنْهُ شَيْـًۭٔا ۚ وَفَجَّرْنَا خِلَـٰلَهُمَا نَهَرًۭا ﴿٣٣﴾\\
\textamh{34.\  } & وَكَانَ لَهُۥ ثَمَرٌۭ فَقَالَ لِصَـٰحِبِهِۦ وَهُوَ يُحَاوِرُهُۥٓ أَنَا۠ أَكْثَرُ مِنكَ مَالًۭا وَأَعَزُّ نَفَرًۭا ﴿٣٤﴾\\
\textamh{35.\  } & وَدَخَلَ جَنَّتَهُۥ وَهُوَ ظَالِمٌۭ لِّنَفْسِهِۦ قَالَ مَآ أَظُنُّ أَن تَبِيدَ هَـٰذِهِۦٓ أَبَدًۭا ﴿٣٥﴾\\
\textamh{36.\  } & وَمَآ أَظُنُّ ٱلسَّاعَةَ قَآئِمَةًۭ وَلَئِن رُّدِدتُّ إِلَىٰ رَبِّى لَأَجِدَنَّ خَيْرًۭا مِّنْهَا مُنقَلَبًۭا ﴿٣٦﴾\\
\textamh{37.\  } & قَالَ لَهُۥ صَاحِبُهُۥ وَهُوَ يُحَاوِرُهُۥٓ أَكَفَرْتَ بِٱلَّذِى خَلَقَكَ مِن تُرَابٍۢ ثُمَّ مِن نُّطْفَةٍۢ ثُمَّ سَوَّىٰكَ رَجُلًۭا ﴿٣٧﴾\\
\textamh{38.\  } & لَّٰكِنَّا۠ هُوَ ٱللَّهُ رَبِّى وَلَآ أُشْرِكُ بِرَبِّىٓ أَحَدًۭا ﴿٣٨﴾\\
\textamh{39.\  } & وَلَوْلَآ إِذْ دَخَلْتَ جَنَّتَكَ قُلْتَ مَا شَآءَ ٱللَّهُ لَا قُوَّةَ إِلَّا بِٱللَّهِ ۚ إِن تَرَنِ أَنَا۠ أَقَلَّ مِنكَ مَالًۭا وَوَلَدًۭا ﴿٣٩﴾\\
\textamh{40.\  } & فَعَسَىٰ رَبِّىٓ أَن يُؤْتِيَنِ خَيْرًۭا مِّن جَنَّتِكَ وَيُرْسِلَ عَلَيْهَا حُسْبَانًۭا مِّنَ ٱلسَّمَآءِ فَتُصْبِحَ صَعِيدًۭا زَلَقًا ﴿٤٠﴾\\
\textamh{41.\  } & أَوْ يُصْبِحَ مَآؤُهَا غَوْرًۭا فَلَن تَسْتَطِيعَ لَهُۥ طَلَبًۭا ﴿٤١﴾\\
\textamh{42.\  } & وَأُحِيطَ بِثَمَرِهِۦ فَأَصْبَحَ يُقَلِّبُ كَفَّيْهِ عَلَىٰ مَآ أَنفَقَ فِيهَا وَهِىَ خَاوِيَةٌ عَلَىٰ عُرُوشِهَا وَيَقُولُ يَـٰلَيْتَنِى لَمْ أُشْرِكْ بِرَبِّىٓ أَحَدًۭا ﴿٤٢﴾\\
\textamh{43.\  } & وَلَمْ تَكُن لَّهُۥ فِئَةٌۭ يَنصُرُونَهُۥ مِن دُونِ ٱللَّهِ وَمَا كَانَ مُنتَصِرًا ﴿٤٣﴾\\
\textamh{44.\  } & هُنَالِكَ ٱلْوَلَـٰيَةُ لِلَّهِ ٱلْحَقِّ ۚ هُوَ خَيْرٌۭ ثَوَابًۭا وَخَيْرٌ عُقْبًۭا ﴿٤٤﴾\\
\textamh{45.\  } & وَٱضْرِبْ لَهُم مَّثَلَ ٱلْحَيَوٰةِ ٱلدُّنْيَا كَمَآءٍ أَنزَلْنَـٰهُ مِنَ ٱلسَّمَآءِ فَٱخْتَلَطَ بِهِۦ نَبَاتُ ٱلْأَرْضِ فَأَصْبَحَ هَشِيمًۭا تَذْرُوهُ ٱلرِّيَـٰحُ ۗ وَكَانَ ٱللَّهُ عَلَىٰ كُلِّ شَىْءٍۢ مُّقْتَدِرًا ﴿٤٥﴾\\
\textamh{46.\  } & ٱلْمَالُ وَٱلْبَنُونَ زِينَةُ ٱلْحَيَوٰةِ ٱلدُّنْيَا ۖ وَٱلْبَٰقِيَـٰتُ ٱلصَّـٰلِحَـٰتُ خَيْرٌ عِندَ رَبِّكَ ثَوَابًۭا وَخَيْرٌ أَمَلًۭا ﴿٤٦﴾\\
\textamh{47.\  } & وَيَوْمَ نُسَيِّرُ ٱلْجِبَالَ وَتَرَى ٱلْأَرْضَ بَارِزَةًۭ وَحَشَرْنَـٰهُمْ فَلَمْ نُغَادِرْ مِنْهُمْ أَحَدًۭا ﴿٤٧﴾\\
\textamh{48.\  } & وَعُرِضُوا۟ عَلَىٰ رَبِّكَ صَفًّۭا لَّقَدْ جِئْتُمُونَا كَمَا خَلَقْنَـٰكُمْ أَوَّلَ مَرَّةٍۭ ۚ بَلْ زَعَمْتُمْ أَلَّن نَّجْعَلَ لَكُم مَّوْعِدًۭا ﴿٤٨﴾\\
\textamh{49.\  } & وَوُضِعَ ٱلْكِتَـٰبُ فَتَرَى ٱلْمُجْرِمِينَ مُشْفِقِينَ مِمَّا فِيهِ وَيَقُولُونَ يَـٰوَيْلَتَنَا مَالِ هَـٰذَا ٱلْكِتَـٰبِ لَا يُغَادِرُ صَغِيرَةًۭ وَلَا كَبِيرَةً إِلَّآ أَحْصَىٰهَا ۚ وَوَجَدُوا۟ مَا عَمِلُوا۟ حَاضِرًۭا ۗ وَلَا يَظْلِمُ رَبُّكَ أَحَدًۭا ﴿٤٩﴾\\
\textamh{50.\  } & وَإِذْ قُلْنَا لِلْمَلَـٰٓئِكَةِ ٱسْجُدُوا۟ لِءَادَمَ فَسَجَدُوٓا۟ إِلَّآ إِبْلِيسَ كَانَ مِنَ ٱلْجِنِّ فَفَسَقَ عَنْ أَمْرِ رَبِّهِۦٓ ۗ أَفَتَتَّخِذُونَهُۥ وَذُرِّيَّتَهُۥٓ أَوْلِيَآءَ مِن دُونِى وَهُمْ لَكُمْ عَدُوٌّۢ ۚ بِئْسَ لِلظَّـٰلِمِينَ بَدَلًۭا ﴿٥٠﴾\\
\textamh{51.\  } & ۞ مَّآ أَشْهَدتُّهُمْ خَلْقَ ٱلسَّمَـٰوَٟتِ وَٱلْأَرْضِ وَلَا خَلْقَ أَنفُسِهِمْ وَمَا كُنتُ مُتَّخِذَ ٱلْمُضِلِّينَ عَضُدًۭا ﴿٥١﴾\\
\textamh{52.\  } & وَيَوْمَ يَقُولُ نَادُوا۟ شُرَكَآءِىَ ٱلَّذِينَ زَعَمْتُمْ فَدَعَوْهُمْ فَلَمْ يَسْتَجِيبُوا۟ لَهُمْ وَجَعَلْنَا بَيْنَهُم مَّوْبِقًۭا ﴿٥٢﴾\\
\textamh{53.\  } & وَرَءَا ٱلْمُجْرِمُونَ ٱلنَّارَ فَظَنُّوٓا۟ أَنَّهُم مُّوَاقِعُوهَا وَلَمْ يَجِدُوا۟ عَنْهَا مَصْرِفًۭا ﴿٥٣﴾\\
\textamh{54.\  } & وَلَقَدْ صَرَّفْنَا فِى هَـٰذَا ٱلْقُرْءَانِ لِلنَّاسِ مِن كُلِّ مَثَلٍۢ ۚ وَكَانَ ٱلْإِنسَـٰنُ أَكْثَرَ شَىْءٍۢ جَدَلًۭا ﴿٥٤﴾\\
\textamh{55.\  } & وَمَا مَنَعَ ٱلنَّاسَ أَن يُؤْمِنُوٓا۟ إِذْ جَآءَهُمُ ٱلْهُدَىٰ وَيَسْتَغْفِرُوا۟ رَبَّهُمْ إِلَّآ أَن تَأْتِيَهُمْ سُنَّةُ ٱلْأَوَّلِينَ أَوْ يَأْتِيَهُمُ ٱلْعَذَابُ قُبُلًۭا ﴿٥٥﴾\\
\textamh{56.\  } & وَمَا نُرْسِلُ ٱلْمُرْسَلِينَ إِلَّا مُبَشِّرِينَ وَمُنذِرِينَ ۚ وَيُجَٰدِلُ ٱلَّذِينَ كَفَرُوا۟ بِٱلْبَٰطِلِ لِيُدْحِضُوا۟ بِهِ ٱلْحَقَّ ۖ وَٱتَّخَذُوٓا۟ ءَايَـٰتِى وَمَآ أُنذِرُوا۟ هُزُوًۭا ﴿٥٦﴾\\
\textamh{57.\  } & وَمَنْ أَظْلَمُ مِمَّن ذُكِّرَ بِـَٔايَـٰتِ رَبِّهِۦ فَأَعْرَضَ عَنْهَا وَنَسِىَ مَا قَدَّمَتْ يَدَاهُ ۚ إِنَّا جَعَلْنَا عَلَىٰ قُلُوبِهِمْ أَكِنَّةً أَن يَفْقَهُوهُ وَفِىٓ ءَاذَانِهِمْ وَقْرًۭا ۖ وَإِن تَدْعُهُمْ إِلَى ٱلْهُدَىٰ فَلَن يَهْتَدُوٓا۟ إِذًا أَبَدًۭا ﴿٥٧﴾\\
\textamh{58.\  } & وَرَبُّكَ ٱلْغَفُورُ ذُو ٱلرَّحْمَةِ ۖ لَوْ يُؤَاخِذُهُم بِمَا كَسَبُوا۟ لَعَجَّلَ لَهُمُ ٱلْعَذَابَ ۚ بَل لَّهُم مَّوْعِدٌۭ لَّن يَجِدُوا۟ مِن دُونِهِۦ مَوْئِلًۭا ﴿٥٨﴾\\
\textamh{59.\  } & وَتِلْكَ ٱلْقُرَىٰٓ أَهْلَكْنَـٰهُمْ لَمَّا ظَلَمُوا۟ وَجَعَلْنَا لِمَهْلِكِهِم مَّوْعِدًۭا ﴿٥٩﴾\\
\textamh{60.\  } & وَإِذْ قَالَ مُوسَىٰ لِفَتَىٰهُ لَآ أَبْرَحُ حَتَّىٰٓ أَبْلُغَ مَجْمَعَ ٱلْبَحْرَيْنِ أَوْ أَمْضِىَ حُقُبًۭا ﴿٦٠﴾\\
\textamh{61.\  } & فَلَمَّا بَلَغَا مَجْمَعَ بَيْنِهِمَا نَسِيَا حُوتَهُمَا فَٱتَّخَذَ سَبِيلَهُۥ فِى ٱلْبَحْرِ سَرَبًۭا ﴿٦١﴾\\
\textamh{62.\  } & فَلَمَّا جَاوَزَا قَالَ لِفَتَىٰهُ ءَاتِنَا غَدَآءَنَا لَقَدْ لَقِينَا مِن سَفَرِنَا هَـٰذَا نَصَبًۭا ﴿٦٢﴾\\
\textamh{63.\  } & قَالَ أَرَءَيْتَ إِذْ أَوَيْنَآ إِلَى ٱلصَّخْرَةِ فَإِنِّى نَسِيتُ ٱلْحُوتَ وَمَآ أَنسَىٰنِيهُ إِلَّا ٱلشَّيْطَٰنُ أَنْ أَذْكُرَهُۥ ۚ وَٱتَّخَذَ سَبِيلَهُۥ فِى ٱلْبَحْرِ عَجَبًۭا ﴿٦٣﴾\\
\textamh{64.\  } & قَالَ ذَٟلِكَ مَا كُنَّا نَبْغِ ۚ فَٱرْتَدَّا عَلَىٰٓ ءَاثَارِهِمَا قَصَصًۭا ﴿٦٤﴾\\
\textamh{65.\  } & فَوَجَدَا عَبْدًۭا مِّنْ عِبَادِنَآ ءَاتَيْنَـٰهُ رَحْمَةًۭ مِّنْ عِندِنَا وَعَلَّمْنَـٰهُ مِن لَّدُنَّا عِلْمًۭا ﴿٦٥﴾\\
\textamh{66.\  } & قَالَ لَهُۥ مُوسَىٰ هَلْ أَتَّبِعُكَ عَلَىٰٓ أَن تُعَلِّمَنِ مِمَّا عُلِّمْتَ رُشْدًۭا ﴿٦٦﴾\\
\textamh{67.\  } & قَالَ إِنَّكَ لَن تَسْتَطِيعَ مَعِىَ صَبْرًۭا ﴿٦٧﴾\\
\textamh{68.\  } & وَكَيْفَ تَصْبِرُ عَلَىٰ مَا لَمْ تُحِطْ بِهِۦ خُبْرًۭا ﴿٦٨﴾\\
\textamh{69.\  } & قَالَ سَتَجِدُنِىٓ إِن شَآءَ ٱللَّهُ صَابِرًۭا وَلَآ أَعْصِى لَكَ أَمْرًۭا ﴿٦٩﴾\\
\textamh{70.\  } & قَالَ فَإِنِ ٱتَّبَعْتَنِى فَلَا تَسْـَٔلْنِى عَن شَىْءٍ حَتَّىٰٓ أُحْدِثَ لَكَ مِنْهُ ذِكْرًۭا ﴿٧٠﴾\\
\textamh{71.\  } & فَٱنطَلَقَا حَتَّىٰٓ إِذَا رَكِبَا فِى ٱلسَّفِينَةِ خَرَقَهَا ۖ قَالَ أَخَرَقْتَهَا لِتُغْرِقَ أَهْلَهَا لَقَدْ جِئْتَ شَيْـًٔا إِمْرًۭا ﴿٧١﴾\\
\textamh{72.\  } & قَالَ أَلَمْ أَقُلْ إِنَّكَ لَن تَسْتَطِيعَ مَعِىَ صَبْرًۭا ﴿٧٢﴾\\
\textamh{73.\  } & قَالَ لَا تُؤَاخِذْنِى بِمَا نَسِيتُ وَلَا تُرْهِقْنِى مِنْ أَمْرِى عُسْرًۭا ﴿٧٣﴾\\
\textamh{74.\  } & فَٱنطَلَقَا حَتَّىٰٓ إِذَا لَقِيَا غُلَـٰمًۭا فَقَتَلَهُۥ قَالَ أَقَتَلْتَ نَفْسًۭا زَكِيَّةًۢ بِغَيْرِ نَفْسٍۢ لَّقَدْ جِئْتَ شَيْـًۭٔا نُّكْرًۭا ﴿٧٤﴾\\
\textamh{75.\  } & ۞ قَالَ أَلَمْ أَقُل لَّكَ إِنَّكَ لَن تَسْتَطِيعَ مَعِىَ صَبْرًۭا ﴿٧٥﴾\\
\textamh{76.\  } & قَالَ إِن سَأَلْتُكَ عَن شَىْءٍۭ بَعْدَهَا فَلَا تُصَـٰحِبْنِى ۖ قَدْ بَلَغْتَ مِن لَّدُنِّى عُذْرًۭا ﴿٧٦﴾\\
\textamh{77.\  } & فَٱنطَلَقَا حَتَّىٰٓ إِذَآ أَتَيَآ أَهْلَ قَرْيَةٍ ٱسْتَطْعَمَآ أَهْلَهَا فَأَبَوْا۟ أَن يُضَيِّفُوهُمَا فَوَجَدَا فِيهَا جِدَارًۭا يُرِيدُ أَن يَنقَضَّ فَأَقَامَهُۥ ۖ قَالَ لَوْ شِئْتَ لَتَّخَذْتَ عَلَيْهِ أَجْرًۭا ﴿٧٧﴾\\
\textamh{78.\  } & قَالَ هَـٰذَا فِرَاقُ بَيْنِى وَبَيْنِكَ ۚ سَأُنَبِّئُكَ بِتَأْوِيلِ مَا لَمْ تَسْتَطِع عَّلَيْهِ صَبْرًا ﴿٧٨﴾\\
\textamh{79.\  } & أَمَّا ٱلسَّفِينَةُ فَكَانَتْ لِمَسَـٰكِينَ يَعْمَلُونَ فِى ٱلْبَحْرِ فَأَرَدتُّ أَنْ أَعِيبَهَا وَكَانَ وَرَآءَهُم مَّلِكٌۭ يَأْخُذُ كُلَّ سَفِينَةٍ غَصْبًۭا ﴿٧٩﴾\\
\textamh{80.\  } & وَأَمَّا ٱلْغُلَـٰمُ فَكَانَ أَبَوَاهُ مُؤْمِنَيْنِ فَخَشِينَآ أَن يُرْهِقَهُمَا طُغْيَـٰنًۭا وَكُفْرًۭا ﴿٨٠﴾\\
\textamh{81.\  } & فَأَرَدْنَآ أَن يُبْدِلَهُمَا رَبُّهُمَا خَيْرًۭا مِّنْهُ زَكَوٰةًۭ وَأَقْرَبَ رُحْمًۭا ﴿٨١﴾\\
\textamh{82.\  } & وَأَمَّا ٱلْجِدَارُ فَكَانَ لِغُلَـٰمَيْنِ يَتِيمَيْنِ فِى ٱلْمَدِينَةِ وَكَانَ تَحْتَهُۥ كَنزٌۭ لَّهُمَا وَكَانَ أَبُوهُمَا صَـٰلِحًۭا فَأَرَادَ رَبُّكَ أَن يَبْلُغَآ أَشُدَّهُمَا وَيَسْتَخْرِجَا كَنزَهُمَا رَحْمَةًۭ مِّن رَّبِّكَ ۚ وَمَا فَعَلْتُهُۥ عَنْ أَمْرِى ۚ ذَٟلِكَ تَأْوِيلُ مَا لَمْ تَسْطِع عَّلَيْهِ صَبْرًۭا ﴿٨٢﴾\\
\textamh{83.\  } & وَيَسْـَٔلُونَكَ عَن ذِى ٱلْقَرْنَيْنِ ۖ قُلْ سَأَتْلُوا۟ عَلَيْكُم مِّنْهُ ذِكْرًا ﴿٨٣﴾\\
\textamh{84.\  } & إِنَّا مَكَّنَّا لَهُۥ فِى ٱلْأَرْضِ وَءَاتَيْنَـٰهُ مِن كُلِّ شَىْءٍۢ سَبَبًۭا ﴿٨٤﴾\\
\textamh{85.\  } & فَأَتْبَعَ سَبَبًا ﴿٨٥﴾\\
\textamh{86.\  } & حَتَّىٰٓ إِذَا بَلَغَ مَغْرِبَ ٱلشَّمْسِ وَجَدَهَا تَغْرُبُ فِى عَيْنٍ حَمِئَةٍۢ وَوَجَدَ عِندَهَا قَوْمًۭا ۗ قُلْنَا يَـٰذَا ٱلْقَرْنَيْنِ إِمَّآ أَن تُعَذِّبَ وَإِمَّآ أَن تَتَّخِذَ فِيهِمْ حُسْنًۭا ﴿٨٦﴾\\
\textamh{87.\  } & قَالَ أَمَّا مَن ظَلَمَ فَسَوْفَ نُعَذِّبُهُۥ ثُمَّ يُرَدُّ إِلَىٰ رَبِّهِۦ فَيُعَذِّبُهُۥ عَذَابًۭا نُّكْرًۭا ﴿٨٧﴾\\
\textamh{88.\  } & وَأَمَّا مَنْ ءَامَنَ وَعَمِلَ صَـٰلِحًۭا فَلَهُۥ جَزَآءً ٱلْحُسْنَىٰ ۖ وَسَنَقُولُ لَهُۥ مِنْ أَمْرِنَا يُسْرًۭا ﴿٨٨﴾\\
\textamh{89.\  } & ثُمَّ أَتْبَعَ سَبَبًا ﴿٨٩﴾\\
\textamh{90.\  } & حَتَّىٰٓ إِذَا بَلَغَ مَطْلِعَ ٱلشَّمْسِ وَجَدَهَا تَطْلُعُ عَلَىٰ قَوْمٍۢ لَّمْ نَجْعَل لَّهُم مِّن دُونِهَا سِتْرًۭا ﴿٩٠﴾\\
\textamh{91.\  } & كَذَٟلِكَ وَقَدْ أَحَطْنَا بِمَا لَدَيْهِ خُبْرًۭا ﴿٩١﴾\\
\textamh{92.\  } & ثُمَّ أَتْبَعَ سَبَبًا ﴿٩٢﴾\\
\textamh{93.\  } & حَتَّىٰٓ إِذَا بَلَغَ بَيْنَ ٱلسَّدَّيْنِ وَجَدَ مِن دُونِهِمَا قَوْمًۭا لَّا يَكَادُونَ يَفْقَهُونَ قَوْلًۭا ﴿٩٣﴾\\
\textamh{94.\  } & قَالُوا۟ يَـٰذَا ٱلْقَرْنَيْنِ إِنَّ يَأْجُوجَ وَمَأْجُوجَ مُفْسِدُونَ فِى ٱلْأَرْضِ فَهَلْ نَجْعَلُ لَكَ خَرْجًا عَلَىٰٓ أَن تَجْعَلَ بَيْنَنَا وَبَيْنَهُمْ سَدًّۭا ﴿٩٤﴾\\
\textamh{95.\  } & قَالَ مَا مَكَّنِّى فِيهِ رَبِّى خَيْرٌۭ فَأَعِينُونِى بِقُوَّةٍ أَجْعَلْ بَيْنَكُمْ وَبَيْنَهُمْ رَدْمًا ﴿٩٥﴾\\
\textamh{96.\  } & ءَاتُونِى زُبَرَ ٱلْحَدِيدِ ۖ حَتَّىٰٓ إِذَا سَاوَىٰ بَيْنَ ٱلصَّدَفَيْنِ قَالَ ٱنفُخُوا۟ ۖ حَتَّىٰٓ إِذَا جَعَلَهُۥ نَارًۭا قَالَ ءَاتُونِىٓ أُفْرِغْ عَلَيْهِ قِطْرًۭا ﴿٩٦﴾\\
\textamh{97.\  } & فَمَا ٱسْطَٰعُوٓا۟ أَن يَظْهَرُوهُ وَمَا ٱسْتَطَٰعُوا۟ لَهُۥ نَقْبًۭا ﴿٩٧﴾\\
\textamh{98.\  } & قَالَ هَـٰذَا رَحْمَةٌۭ مِّن رَّبِّى ۖ فَإِذَا جَآءَ وَعْدُ رَبِّى جَعَلَهُۥ دَكَّآءَ ۖ وَكَانَ وَعْدُ رَبِّى حَقًّۭا ﴿٩٨﴾\\
\textamh{99.\  } & ۞ وَتَرَكْنَا بَعْضَهُمْ يَوْمَئِذٍۢ يَمُوجُ فِى بَعْضٍۢ ۖ وَنُفِخَ فِى ٱلصُّورِ فَجَمَعْنَـٰهُمْ جَمْعًۭا ﴿٩٩﴾\\
\textamh{100.\  } & وَعَرَضْنَا جَهَنَّمَ يَوْمَئِذٍۢ لِّلْكَـٰفِرِينَ عَرْضًا ﴿١٠٠﴾\\
\textamh{101.\  } & ٱلَّذِينَ كَانَتْ أَعْيُنُهُمْ فِى غِطَآءٍ عَن ذِكْرِى وَكَانُوا۟ لَا يَسْتَطِيعُونَ سَمْعًا ﴿١٠١﴾\\
\textamh{102.\  } & أَفَحَسِبَ ٱلَّذِينَ كَفَرُوٓا۟ أَن يَتَّخِذُوا۟ عِبَادِى مِن دُونِىٓ أَوْلِيَآءَ ۚ إِنَّآ أَعْتَدْنَا جَهَنَّمَ لِلْكَـٰفِرِينَ نُزُلًۭا ﴿١٠٢﴾\\
\textamh{103.\  } & قُلْ هَلْ نُنَبِّئُكُم بِٱلْأَخْسَرِينَ أَعْمَـٰلًا ﴿١٠٣﴾\\
\textamh{104.\  } & ٱلَّذِينَ ضَلَّ سَعْيُهُمْ فِى ٱلْحَيَوٰةِ ٱلدُّنْيَا وَهُمْ يَحْسَبُونَ أَنَّهُمْ يُحْسِنُونَ صُنْعًا ﴿١٠٤﴾\\
\textamh{105.\  } & أُو۟لَـٰٓئِكَ ٱلَّذِينَ كَفَرُوا۟ بِـَٔايَـٰتِ رَبِّهِمْ وَلِقَآئِهِۦ فَحَبِطَتْ أَعْمَـٰلُهُمْ فَلَا نُقِيمُ لَهُمْ يَوْمَ ٱلْقِيَـٰمَةِ وَزْنًۭا ﴿١٠٥﴾\\
\textamh{106.\  } & ذَٟلِكَ جَزَآؤُهُمْ جَهَنَّمُ بِمَا كَفَرُوا۟ وَٱتَّخَذُوٓا۟ ءَايَـٰتِى وَرُسُلِى هُزُوًا ﴿١٠٦﴾\\
\textamh{107.\  } & إِنَّ ٱلَّذِينَ ءَامَنُوا۟ وَعَمِلُوا۟ ٱلصَّـٰلِحَـٰتِ كَانَتْ لَهُمْ جَنَّـٰتُ ٱلْفِرْدَوْسِ نُزُلًا ﴿١٠٧﴾\\
\textamh{108.\  } & خَـٰلِدِينَ فِيهَا لَا يَبْغُونَ عَنْهَا حِوَلًۭا ﴿١٠٨﴾\\
\textamh{109.\  } & قُل لَّوْ كَانَ ٱلْبَحْرُ مِدَادًۭا لِّكَلِمَـٰتِ رَبِّى لَنَفِدَ ٱلْبَحْرُ قَبْلَ أَن تَنفَدَ كَلِمَـٰتُ رَبِّى وَلَوْ جِئْنَا بِمِثْلِهِۦ مَدَدًۭا ﴿١٠٩﴾\\
\textamh{110.\  } & قُلْ إِنَّمَآ أَنَا۠ بَشَرٌۭ مِّثْلُكُمْ يُوحَىٰٓ إِلَىَّ أَنَّمَآ إِلَـٰهُكُمْ إِلَـٰهٌۭ وَٟحِدٌۭ ۖ فَمَن كَانَ يَرْجُوا۟ لِقَآءَ رَبِّهِۦ فَلْيَعْمَلْ عَمَلًۭا صَـٰلِحًۭا وَلَا يُشْرِكْ بِعِبَادَةِ رَبِّهِۦٓ أَحَدًۢا ﴿١١٠﴾\\
\end{longtable}
\clearpage
%% License: BSD style (Berkley) (i.e. Put the Copyright owner's name always)
%% Writer and Copyright (to): Bewketu(Bilal) Tadilo (2016-17)
\centering\section{\LR{\textamharic{ሱራቱ ማሪያም -}  \RL{سوره  مريم}}}
\begin{longtable}{%
  @{}
    p{.5\textwidth}
  @{~~~~~~~~~~~~}
    p{.5\textwidth}
    @{}
}
\nopagebreak
\textamh{ቢስሚላሂ አራህመኒ ራሂይም } &  بِسْمِ ٱللَّهِ ٱلرَّحْمَـٰنِ ٱلرَّحِيمِ\\
\textamh{1.\  } &  كٓهيعٓصٓ ﴿١﴾\\
\textamh{2.\  } & ذِكْرُ رَحْمَتِ رَبِّكَ عَبْدَهُۥ زَكَرِيَّآ ﴿٢﴾\\
\textamh{3.\  } & إِذْ نَادَىٰ رَبَّهُۥ نِدَآءً خَفِيًّۭا ﴿٣﴾\\
\textamh{4.\  } & قَالَ رَبِّ إِنِّى وَهَنَ ٱلْعَظْمُ مِنِّى وَٱشْتَعَلَ ٱلرَّأْسُ شَيْبًۭا وَلَمْ أَكُنۢ بِدُعَآئِكَ رَبِّ شَقِيًّۭا ﴿٤﴾\\
\textamh{5.\  } & وَإِنِّى خِفْتُ ٱلْمَوَٟلِىَ مِن وَرَآءِى وَكَانَتِ ٱمْرَأَتِى عَاقِرًۭا فَهَبْ لِى مِن لَّدُنكَ وَلِيًّۭا ﴿٥﴾\\
\textamh{6.\  } & يَرِثُنِى وَيَرِثُ مِنْ ءَالِ يَعْقُوبَ ۖ وَٱجْعَلْهُ رَبِّ رَضِيًّۭا ﴿٦﴾\\
\textamh{7.\  } & يَـٰزَكَرِيَّآ إِنَّا نُبَشِّرُكَ بِغُلَـٰمٍ ٱسْمُهُۥ يَحْيَىٰ لَمْ نَجْعَل لَّهُۥ مِن قَبْلُ سَمِيًّۭا ﴿٧﴾\\
\textamh{8.\  } & قَالَ رَبِّ أَنَّىٰ يَكُونُ لِى غُلَـٰمٌۭ وَكَانَتِ ٱمْرَأَتِى عَاقِرًۭا وَقَدْ بَلَغْتُ مِنَ ٱلْكِبَرِ عِتِيًّۭا ﴿٨﴾\\
\textamh{9.\  } & قَالَ كَذَٟلِكَ قَالَ رَبُّكَ هُوَ عَلَىَّ هَيِّنٌۭ وَقَدْ خَلَقْتُكَ مِن قَبْلُ وَلَمْ تَكُ شَيْـًۭٔا ﴿٩﴾\\
\textamh{10.\  } & قَالَ رَبِّ ٱجْعَل لِّىٓ ءَايَةًۭ ۚ قَالَ ءَايَتُكَ أَلَّا تُكَلِّمَ ٱلنَّاسَ ثَلَـٰثَ لَيَالٍۢ سَوِيًّۭا ﴿١٠﴾\\
\textamh{11.\  } & فَخَرَجَ عَلَىٰ قَوْمِهِۦ مِنَ ٱلْمِحْرَابِ فَأَوْحَىٰٓ إِلَيْهِمْ أَن سَبِّحُوا۟ بُكْرَةًۭ وَعَشِيًّۭا ﴿١١﴾\\
\textamh{12.\  } & يَـٰيَحْيَىٰ خُذِ ٱلْكِتَـٰبَ بِقُوَّةٍۢ ۖ وَءَاتَيْنَـٰهُ ٱلْحُكْمَ صَبِيًّۭا ﴿١٢﴾\\
\textamh{13.\  } & وَحَنَانًۭا مِّن لَّدُنَّا وَزَكَوٰةًۭ ۖ وَكَانَ تَقِيًّۭا ﴿١٣﴾\\
\textamh{14.\  } & وَبَرًّۢا بِوَٟلِدَيْهِ وَلَمْ يَكُن جَبَّارًا عَصِيًّۭا ﴿١٤﴾\\
\textamh{15.\  } & وَسَلَـٰمٌ عَلَيْهِ يَوْمَ وُلِدَ وَيَوْمَ يَمُوتُ وَيَوْمَ يُبْعَثُ حَيًّۭا ﴿١٥﴾\\
\textamh{16.\  } & وَٱذْكُرْ فِى ٱلْكِتَـٰبِ مَرْيَمَ إِذِ ٱنتَبَذَتْ مِنْ أَهْلِهَا مَكَانًۭا شَرْقِيًّۭا ﴿١٦﴾\\
\textamh{17.\  } & فَٱتَّخَذَتْ مِن دُونِهِمْ حِجَابًۭا فَأَرْسَلْنَآ إِلَيْهَا رُوحَنَا فَتَمَثَّلَ لَهَا بَشَرًۭا سَوِيًّۭا ﴿١٧﴾\\
\textamh{18.\  } & قَالَتْ إِنِّىٓ أَعُوذُ بِٱلرَّحْمَـٰنِ مِنكَ إِن كُنتَ تَقِيًّۭا ﴿١٨﴾\\
\textamh{19.\  } & قَالَ إِنَّمَآ أَنَا۠ رَسُولُ رَبِّكِ لِأَهَبَ لَكِ غُلَـٰمًۭا زَكِيًّۭا ﴿١٩﴾\\
\textamh{20.\  } & قَالَتْ أَنَّىٰ يَكُونُ لِى غُلَـٰمٌۭ وَلَمْ يَمْسَسْنِى بَشَرٌۭ وَلَمْ أَكُ بَغِيًّۭا ﴿٢٠﴾\\
\textamh{21.\  } & قَالَ كَذَٟلِكِ قَالَ رَبُّكِ هُوَ عَلَىَّ هَيِّنٌۭ ۖ وَلِنَجْعَلَهُۥٓ ءَايَةًۭ لِّلنَّاسِ وَرَحْمَةًۭ مِّنَّا ۚ وَكَانَ أَمْرًۭا مَّقْضِيًّۭا ﴿٢١﴾\\
\textamh{22.\  } & ۞ فَحَمَلَتْهُ فَٱنتَبَذَتْ بِهِۦ مَكَانًۭا قَصِيًّۭا ﴿٢٢﴾\\
\textamh{23.\  } & فَأَجَآءَهَا ٱلْمَخَاضُ إِلَىٰ جِذْعِ ٱلنَّخْلَةِ قَالَتْ يَـٰلَيْتَنِى مِتُّ قَبْلَ هَـٰذَا وَكُنتُ نَسْيًۭا مَّنسِيًّۭا ﴿٢٣﴾\\
\textamh{24.\  } & فَنَادَىٰهَا مِن تَحْتِهَآ أَلَّا تَحْزَنِى قَدْ جَعَلَ رَبُّكِ تَحْتَكِ سَرِيًّۭا ﴿٢٤﴾\\
\textamh{25.\  } & وَهُزِّىٓ إِلَيْكِ بِجِذْعِ ٱلنَّخْلَةِ تُسَـٰقِطْ عَلَيْكِ رُطَبًۭا جَنِيًّۭا ﴿٢٥﴾\\
\textamh{26.\  } & فَكُلِى وَٱشْرَبِى وَقَرِّى عَيْنًۭا ۖ فَإِمَّا تَرَيِنَّ مِنَ ٱلْبَشَرِ أَحَدًۭا فَقُولِىٓ إِنِّى نَذَرْتُ لِلرَّحْمَـٰنِ صَوْمًۭا فَلَنْ أُكَلِّمَ ٱلْيَوْمَ إِنسِيًّۭا ﴿٢٦﴾\\
\textamh{27.\  } & فَأَتَتْ بِهِۦ قَوْمَهَا تَحْمِلُهُۥ ۖ قَالُوا۟ يَـٰمَرْيَمُ لَقَدْ جِئْتِ شَيْـًۭٔا فَرِيًّۭا ﴿٢٧﴾\\
\textamh{28.\  } & يَـٰٓأُخْتَ هَـٰرُونَ مَا كَانَ أَبُوكِ ٱمْرَأَ سَوْءٍۢ وَمَا كَانَتْ أُمُّكِ بَغِيًّۭا ﴿٢٨﴾\\
\textamh{29.\  } & فَأَشَارَتْ إِلَيْهِ ۖ قَالُوا۟ كَيْفَ نُكَلِّمُ مَن كَانَ فِى ٱلْمَهْدِ صَبِيًّۭا ﴿٢٩﴾\\
\textamh{30.\  } & قَالَ إِنِّى عَبْدُ ٱللَّهِ ءَاتَىٰنِىَ ٱلْكِتَـٰبَ وَجَعَلَنِى نَبِيًّۭا ﴿٣٠﴾\\
\textamh{31.\  } & وَجَعَلَنِى مُبَارَكًا أَيْنَ مَا كُنتُ وَأَوْصَـٰنِى بِٱلصَّلَوٰةِ وَٱلزَّكَوٰةِ مَا دُمْتُ حَيًّۭا ﴿٣١﴾\\
\textamh{32.\  } & وَبَرًّۢا بِوَٟلِدَتِى وَلَمْ يَجْعَلْنِى جَبَّارًۭا شَقِيًّۭا ﴿٣٢﴾\\
\textamh{33.\  } & وَٱلسَّلَـٰمُ عَلَىَّ يَوْمَ وُلِدتُّ وَيَوْمَ أَمُوتُ وَيَوْمَ أُبْعَثُ حَيًّۭا ﴿٣٣﴾\\
\textamh{34.\  } & ذَٟلِكَ عِيسَى ٱبْنُ مَرْيَمَ ۚ قَوْلَ ٱلْحَقِّ ٱلَّذِى فِيهِ يَمْتَرُونَ ﴿٣٤﴾\\
\textamh{35.\  } & مَا كَانَ لِلَّهِ أَن يَتَّخِذَ مِن وَلَدٍۢ ۖ سُبْحَـٰنَهُۥٓ ۚ إِذَا قَضَىٰٓ أَمْرًۭا فَإِنَّمَا يَقُولُ لَهُۥ كُن فَيَكُونُ ﴿٣٥﴾\\
\textamh{36.\  } & وَإِنَّ ٱللَّهَ رَبِّى وَرَبُّكُمْ فَٱعْبُدُوهُ ۚ هَـٰذَا صِرَٰطٌۭ مُّسْتَقِيمٌۭ ﴿٣٦﴾\\
\textamh{37.\  } & فَٱخْتَلَفَ ٱلْأَحْزَابُ مِنۢ بَيْنِهِمْ ۖ فَوَيْلٌۭ لِّلَّذِينَ كَفَرُوا۟ مِن مَّشْهَدِ يَوْمٍ عَظِيمٍ ﴿٣٧﴾\\
\textamh{38.\  } & أَسْمِعْ بِهِمْ وَأَبْصِرْ يَوْمَ يَأْتُونَنَا ۖ لَـٰكِنِ ٱلظَّـٰلِمُونَ ٱلْيَوْمَ فِى ضَلَـٰلٍۢ مُّبِينٍۢ ﴿٣٨﴾\\
\textamh{39.\  } & وَأَنذِرْهُمْ يَوْمَ ٱلْحَسْرَةِ إِذْ قُضِىَ ٱلْأَمْرُ وَهُمْ فِى غَفْلَةٍۢ وَهُمْ لَا يُؤْمِنُونَ ﴿٣٩﴾\\
\textamh{40.\  } & إِنَّا نَحْنُ نَرِثُ ٱلْأَرْضَ وَمَنْ عَلَيْهَا وَإِلَيْنَا يُرْجَعُونَ ﴿٤٠﴾\\
\textamh{41.\  } & وَٱذْكُرْ فِى ٱلْكِتَـٰبِ إِبْرَٰهِيمَ ۚ إِنَّهُۥ كَانَ صِدِّيقًۭا نَّبِيًّا ﴿٤١﴾\\
\textamh{42.\  } & إِذْ قَالَ لِأَبِيهِ يَـٰٓأَبَتِ لِمَ تَعْبُدُ مَا لَا يَسْمَعُ وَلَا يُبْصِرُ وَلَا يُغْنِى عَنكَ شَيْـًۭٔا ﴿٤٢﴾\\
\textamh{43.\  } & يَـٰٓأَبَتِ إِنِّى قَدْ جَآءَنِى مِنَ ٱلْعِلْمِ مَا لَمْ يَأْتِكَ فَٱتَّبِعْنِىٓ أَهْدِكَ صِرَٰطًۭا سَوِيًّۭا ﴿٤٣﴾\\
\textamh{44.\  } & يَـٰٓأَبَتِ لَا تَعْبُدِ ٱلشَّيْطَٰنَ ۖ إِنَّ ٱلشَّيْطَٰنَ كَانَ لِلرَّحْمَـٰنِ عَصِيًّۭا ﴿٤٤﴾\\
\textamh{45.\  } & يَـٰٓأَبَتِ إِنِّىٓ أَخَافُ أَن يَمَسَّكَ عَذَابٌۭ مِّنَ ٱلرَّحْمَـٰنِ فَتَكُونَ لِلشَّيْطَٰنِ وَلِيًّۭا ﴿٤٥﴾\\
\textamh{46.\  } & قَالَ أَرَاغِبٌ أَنتَ عَنْ ءَالِهَتِى يَـٰٓإِبْرَٰهِيمُ ۖ لَئِن لَّمْ تَنتَهِ لَأَرْجُمَنَّكَ ۖ وَٱهْجُرْنِى مَلِيًّۭا ﴿٤٦﴾\\
\textamh{47.\  } & قَالَ سَلَـٰمٌ عَلَيْكَ ۖ سَأَسْتَغْفِرُ لَكَ رَبِّىٓ ۖ إِنَّهُۥ كَانَ بِى حَفِيًّۭا ﴿٤٧﴾\\
\textamh{48.\  } & وَأَعْتَزِلُكُمْ وَمَا تَدْعُونَ مِن دُونِ ٱللَّهِ وَأَدْعُوا۟ رَبِّى عَسَىٰٓ أَلَّآ أَكُونَ بِدُعَآءِ رَبِّى شَقِيًّۭا ﴿٤٨﴾\\
\textamh{49.\  } & فَلَمَّا ٱعْتَزَلَهُمْ وَمَا يَعْبُدُونَ مِن دُونِ ٱللَّهِ وَهَبْنَا لَهُۥٓ إِسْحَـٰقَ وَيَعْقُوبَ ۖ وَكُلًّۭا جَعَلْنَا نَبِيًّۭا ﴿٤٩﴾\\
\textamh{50.\  } & وَوَهَبْنَا لَهُم مِّن رَّحْمَتِنَا وَجَعَلْنَا لَهُمْ لِسَانَ صِدْقٍ عَلِيًّۭا ﴿٥٠﴾\\
\textamh{51.\  } & وَٱذْكُرْ فِى ٱلْكِتَـٰبِ مُوسَىٰٓ ۚ إِنَّهُۥ كَانَ مُخْلَصًۭا وَكَانَ رَسُولًۭا نَّبِيًّۭا ﴿٥١﴾\\
\textamh{52.\  } & وَنَـٰدَيْنَـٰهُ مِن جَانِبِ ٱلطُّورِ ٱلْأَيْمَنِ وَقَرَّبْنَـٰهُ نَجِيًّۭا ﴿٥٢﴾\\
\textamh{53.\  } & وَوَهَبْنَا لَهُۥ مِن رَّحْمَتِنَآ أَخَاهُ هَـٰرُونَ نَبِيًّۭا ﴿٥٣﴾\\
\textamh{54.\  } & وَٱذْكُرْ فِى ٱلْكِتَـٰبِ إِسْمَـٰعِيلَ ۚ إِنَّهُۥ كَانَ صَادِقَ ٱلْوَعْدِ وَكَانَ رَسُولًۭا نَّبِيًّۭا ﴿٥٤﴾\\
\textamh{55.\  } & وَكَانَ يَأْمُرُ أَهْلَهُۥ بِٱلصَّلَوٰةِ وَٱلزَّكَوٰةِ وَكَانَ عِندَ رَبِّهِۦ مَرْضِيًّۭا ﴿٥٥﴾\\
\textamh{56.\  } & وَٱذْكُرْ فِى ٱلْكِتَـٰبِ إِدْرِيسَ ۚ إِنَّهُۥ كَانَ صِدِّيقًۭا نَّبِيًّۭا ﴿٥٦﴾\\
\textamh{57.\  } & وَرَفَعْنَـٰهُ مَكَانًا عَلِيًّا ﴿٥٧﴾\\
\textamh{58.\  } & أُو۟لَـٰٓئِكَ ٱلَّذِينَ أَنْعَمَ ٱللَّهُ عَلَيْهِم مِّنَ ٱلنَّبِيِّۦنَ مِن ذُرِّيَّةِ ءَادَمَ وَمِمَّنْ حَمَلْنَا مَعَ نُوحٍۢ وَمِن ذُرِّيَّةِ إِبْرَٰهِيمَ وَإِسْرَٰٓءِيلَ وَمِمَّنْ هَدَيْنَا وَٱجْتَبَيْنَآ ۚ إِذَا تُتْلَىٰ عَلَيْهِمْ ءَايَـٰتُ ٱلرَّحْمَـٰنِ خَرُّوا۟ سُجَّدًۭا وَبُكِيًّۭا ۩ ﴿٥٨﴾\\
\textamh{59.\  } & ۞ فَخَلَفَ مِنۢ بَعْدِهِمْ خَلْفٌ أَضَاعُوا۟ ٱلصَّلَوٰةَ وَٱتَّبَعُوا۟ ٱلشَّهَوَٟتِ ۖ فَسَوْفَ يَلْقَوْنَ غَيًّا ﴿٥٩﴾\\
\textamh{60.\  } & إِلَّا مَن تَابَ وَءَامَنَ وَعَمِلَ صَـٰلِحًۭا فَأُو۟لَـٰٓئِكَ يَدْخُلُونَ ٱلْجَنَّةَ وَلَا يُظْلَمُونَ شَيْـًۭٔا ﴿٦٠﴾\\
\textamh{61.\  } & جَنَّـٰتِ عَدْنٍ ٱلَّتِى وَعَدَ ٱلرَّحْمَـٰنُ عِبَادَهُۥ بِٱلْغَيْبِ ۚ إِنَّهُۥ كَانَ وَعْدُهُۥ مَأْتِيًّۭا ﴿٦١﴾\\
\textamh{62.\  } & لَّا يَسْمَعُونَ فِيهَا لَغْوًا إِلَّا سَلَـٰمًۭا ۖ وَلَهُمْ رِزْقُهُمْ فِيهَا بُكْرَةًۭ وَعَشِيًّۭا ﴿٦٢﴾\\
\textamh{63.\  } & تِلْكَ ٱلْجَنَّةُ ٱلَّتِى نُورِثُ مِنْ عِبَادِنَا مَن كَانَ تَقِيًّۭا ﴿٦٣﴾\\
\textamh{64.\  } & وَمَا نَتَنَزَّلُ إِلَّا بِأَمْرِ رَبِّكَ ۖ لَهُۥ مَا بَيْنَ أَيْدِينَا وَمَا خَلْفَنَا وَمَا بَيْنَ ذَٟلِكَ ۚ وَمَا كَانَ رَبُّكَ نَسِيًّۭا ﴿٦٤﴾\\
\textamh{65.\  } & رَّبُّ ٱلسَّمَـٰوَٟتِ وَٱلْأَرْضِ وَمَا بَيْنَهُمَا فَٱعْبُدْهُ وَٱصْطَبِرْ لِعِبَٰدَتِهِۦ ۚ هَلْ تَعْلَمُ لَهُۥ سَمِيًّۭا ﴿٦٥﴾\\
\textamh{66.\  } & وَيَقُولُ ٱلْإِنسَـٰنُ أَءِذَا مَا مِتُّ لَسَوْفَ أُخْرَجُ حَيًّا ﴿٦٦﴾\\
\textamh{67.\  } & أَوَلَا يَذْكُرُ ٱلْإِنسَـٰنُ أَنَّا خَلَقْنَـٰهُ مِن قَبْلُ وَلَمْ يَكُ شَيْـًۭٔا ﴿٦٧﴾\\
\textamh{68.\  } & فَوَرَبِّكَ لَنَحْشُرَنَّهُمْ وَٱلشَّيَـٰطِينَ ثُمَّ لَنُحْضِرَنَّهُمْ حَوْلَ جَهَنَّمَ جِثِيًّۭا ﴿٦٨﴾\\
\textamh{69.\  } & ثُمَّ لَنَنزِعَنَّ مِن كُلِّ شِيعَةٍ أَيُّهُمْ أَشَدُّ عَلَى ٱلرَّحْمَـٰنِ عِتِيًّۭا ﴿٦٩﴾\\
\textamh{70.\  } & ثُمَّ لَنَحْنُ أَعْلَمُ بِٱلَّذِينَ هُمْ أَوْلَىٰ بِهَا صِلِيًّۭا ﴿٧٠﴾\\
\textamh{71.\  } & وَإِن مِّنكُمْ إِلَّا وَارِدُهَا ۚ كَانَ عَلَىٰ رَبِّكَ حَتْمًۭا مَّقْضِيًّۭا ﴿٧١﴾\\
\textamh{72.\  } & ثُمَّ نُنَجِّى ٱلَّذِينَ ٱتَّقَوا۟ وَّنَذَرُ ٱلظَّـٰلِمِينَ فِيهَا جِثِيًّۭا ﴿٧٢﴾\\
\textamh{73.\  } & وَإِذَا تُتْلَىٰ عَلَيْهِمْ ءَايَـٰتُنَا بَيِّنَـٰتٍۢ قَالَ ٱلَّذِينَ كَفَرُوا۟ لِلَّذِينَ ءَامَنُوٓا۟ أَىُّ ٱلْفَرِيقَيْنِ خَيْرٌۭ مَّقَامًۭا وَأَحْسَنُ نَدِيًّۭا ﴿٧٣﴾\\
\textamh{74.\  } & وَكَمْ أَهْلَكْنَا قَبْلَهُم مِّن قَرْنٍ هُمْ أَحْسَنُ أَثَـٰثًۭا وَرِءْيًۭا ﴿٧٤﴾\\
\textamh{75.\  } & قُلْ مَن كَانَ فِى ٱلضَّلَـٰلَةِ فَلْيَمْدُدْ لَهُ ٱلرَّحْمَـٰنُ مَدًّا ۚ حَتَّىٰٓ إِذَا رَأَوْا۟ مَا يُوعَدُونَ إِمَّا ٱلْعَذَابَ وَإِمَّا ٱلسَّاعَةَ فَسَيَعْلَمُونَ مَنْ هُوَ شَرٌّۭ مَّكَانًۭا وَأَضْعَفُ جُندًۭا ﴿٧٥﴾\\
\textamh{76.\  } & وَيَزِيدُ ٱللَّهُ ٱلَّذِينَ ٱهْتَدَوْا۟ هُدًۭى ۗ وَٱلْبَٰقِيَـٰتُ ٱلصَّـٰلِحَـٰتُ خَيْرٌ عِندَ رَبِّكَ ثَوَابًۭا وَخَيْرٌۭ مَّرَدًّا ﴿٧٦﴾\\
\textamh{77.\  } & أَفَرَءَيْتَ ٱلَّذِى كَفَرَ بِـَٔايَـٰتِنَا وَقَالَ لَأُوتَيَنَّ مَالًۭا وَوَلَدًا ﴿٧٧﴾\\
\textamh{78.\  } & أَطَّلَعَ ٱلْغَيْبَ أَمِ ٱتَّخَذَ عِندَ ٱلرَّحْمَـٰنِ عَهْدًۭا ﴿٧٨﴾\\
\textamh{79.\  } & كَلَّا ۚ سَنَكْتُبُ مَا يَقُولُ وَنَمُدُّ لَهُۥ مِنَ ٱلْعَذَابِ مَدًّۭا ﴿٧٩﴾\\
\textamh{80.\  } & وَنَرِثُهُۥ مَا يَقُولُ وَيَأْتِينَا فَرْدًۭا ﴿٨٠﴾\\
\textamh{81.\  } & وَٱتَّخَذُوا۟ مِن دُونِ ٱللَّهِ ءَالِهَةًۭ لِّيَكُونُوا۟ لَهُمْ عِزًّۭا ﴿٨١﴾\\
\textamh{82.\  } & كَلَّا ۚ سَيَكْفُرُونَ بِعِبَادَتِهِمْ وَيَكُونُونَ عَلَيْهِمْ ضِدًّا ﴿٨٢﴾\\
\textamh{83.\  } & أَلَمْ تَرَ أَنَّآ أَرْسَلْنَا ٱلشَّيَـٰطِينَ عَلَى ٱلْكَـٰفِرِينَ تَؤُزُّهُمْ أَزًّۭا ﴿٨٣﴾\\
\textamh{84.\  } & فَلَا تَعْجَلْ عَلَيْهِمْ ۖ إِنَّمَا نَعُدُّ لَهُمْ عَدًّۭا ﴿٨٤﴾\\
\textamh{85.\  } & يَوْمَ نَحْشُرُ ٱلْمُتَّقِينَ إِلَى ٱلرَّحْمَـٰنِ وَفْدًۭا ﴿٨٥﴾\\
\textamh{86.\  } & وَنَسُوقُ ٱلْمُجْرِمِينَ إِلَىٰ جَهَنَّمَ وِرْدًۭا ﴿٨٦﴾\\
\textamh{87.\  } & لَّا يَمْلِكُونَ ٱلشَّفَـٰعَةَ إِلَّا مَنِ ٱتَّخَذَ عِندَ ٱلرَّحْمَـٰنِ عَهْدًۭا ﴿٨٧﴾\\
\textamh{88.\  } & وَقَالُوا۟ ٱتَّخَذَ ٱلرَّحْمَـٰنُ وَلَدًۭا ﴿٨٨﴾\\
\textamh{89.\  } & لَّقَدْ جِئْتُمْ شَيْـًٔا إِدًّۭا ﴿٨٩﴾\\
\textamh{90.\  } & تَكَادُ ٱلسَّمَـٰوَٟتُ يَتَفَطَّرْنَ مِنْهُ وَتَنشَقُّ ٱلْأَرْضُ وَتَخِرُّ ٱلْجِبَالُ هَدًّا ﴿٩٠﴾\\
\textamh{91.\  } & أَن دَعَوْا۟ لِلرَّحْمَـٰنِ وَلَدًۭا ﴿٩١﴾\\
\textamh{92.\  } & وَمَا يَنۢبَغِى لِلرَّحْمَـٰنِ أَن يَتَّخِذَ وَلَدًا ﴿٩٢﴾\\
\textamh{93.\  } & إِن كُلُّ مَن فِى ٱلسَّمَـٰوَٟتِ وَٱلْأَرْضِ إِلَّآ ءَاتِى ٱلرَّحْمَـٰنِ عَبْدًۭا ﴿٩٣﴾\\
\textamh{94.\  } & لَّقَدْ أَحْصَىٰهُمْ وَعَدَّهُمْ عَدًّۭا ﴿٩٤﴾\\
\textamh{95.\  } & وَكُلُّهُمْ ءَاتِيهِ يَوْمَ ٱلْقِيَـٰمَةِ فَرْدًا ﴿٩٥﴾\\
\textamh{96.\  } & إِنَّ ٱلَّذِينَ ءَامَنُوا۟ وَعَمِلُوا۟ ٱلصَّـٰلِحَـٰتِ سَيَجْعَلُ لَهُمُ ٱلرَّحْمَـٰنُ وُدًّۭا ﴿٩٦﴾\\
\textamh{97.\  } & فَإِنَّمَا يَسَّرْنَـٰهُ بِلِسَانِكَ لِتُبَشِّرَ بِهِ ٱلْمُتَّقِينَ وَتُنذِرَ بِهِۦ قَوْمًۭا لُّدًّۭا ﴿٩٧﴾\\
\textamh{98.\  } & وَكَمْ أَهْلَكْنَا قَبْلَهُم مِّن قَرْنٍ هَلْ تُحِسُّ مِنْهُم مِّنْ أَحَدٍ أَوْ تَسْمَعُ لَهُمْ رِكْزًۢا ﴿٩٨﴾\\
\end{longtable}
\clearpage
%% License: BSD style (Berkley) (i.e. Put the Copyright owner's name always)
%% Writer and Copyright (to): Bewketu(Bilal) Tadilo (2016-17)
\centering\section{\LR{\textamharic{ሱራቱ ጣሃ -}  \RL{سوره  طه}}}
\begin{longtable}{%
  @{}
    p{.5\textwidth}
  @{~~~~~~~~~~~~~}
    p{.5\textwidth}
    @{}
}
\nopagebreak
\textamh{\ \ \ \ \ \  ቢስሚላሂ አራህመኒ ራሂይም } &  بِسْمِ ٱللَّهِ ٱلرَّحْمَـٰنِ ٱلرَّحِيمِ\\
\textamh{1.\  } &  طه ﴿١﴾\\
\textamh{2.\  } & مَآ أَنزَلْنَا عَلَيْكَ ٱلْقُرْءَانَ لِتَشْقَىٰٓ ﴿٢﴾\\
\textamh{3.\  } & إِلَّا تَذْكِرَةًۭ لِّمَن يَخْشَىٰ ﴿٣﴾\\
\textamh{4.\  } & تَنزِيلًۭا مِّمَّنْ خَلَقَ ٱلْأَرْضَ وَٱلسَّمَـٰوَٟتِ ٱلْعُلَى ﴿٤﴾\\
\textamh{5.\  } & ٱلرَّحْمَـٰنُ عَلَى ٱلْعَرْشِ ٱسْتَوَىٰ ﴿٥﴾\\
\textamh{6.\  } & لَهُۥ مَا فِى ٱلسَّمَـٰوَٟتِ وَمَا فِى ٱلْأَرْضِ وَمَا بَيْنَهُمَا وَمَا تَحْتَ ٱلثَّرَىٰ ﴿٦﴾\\
\textamh{7.\  } & وَإِن تَجْهَرْ بِٱلْقَوْلِ فَإِنَّهُۥ يَعْلَمُ ٱلسِّرَّ وَأَخْفَى ﴿٧﴾\\
\textamh{8.\  } & ٱللَّهُ لَآ إِلَـٰهَ إِلَّا هُوَ ۖ لَهُ ٱلْأَسْمَآءُ ٱلْحُسْنَىٰ ﴿٨﴾\\
\textamh{9.\  } & وَهَلْ أَتَىٰكَ حَدِيثُ مُوسَىٰٓ ﴿٩﴾\\
\textamh{10.\  } & إِذْ رَءَا نَارًۭا فَقَالَ لِأَهْلِهِ ٱمْكُثُوٓا۟ إِنِّىٓ ءَانَسْتُ نَارًۭا لَّعَلِّىٓ ءَاتِيكُم مِّنْهَا بِقَبَسٍ أَوْ أَجِدُ عَلَى ٱلنَّارِ هُدًۭى ﴿١٠﴾\\
\textamh{11.\  } & فَلَمَّآ أَتَىٰهَا نُودِىَ يَـٰمُوسَىٰٓ ﴿١١﴾\\
\textamh{12.\  } & إِنِّىٓ أَنَا۠ رَبُّكَ فَٱخْلَعْ نَعْلَيْكَ ۖ إِنَّكَ بِٱلْوَادِ ٱلْمُقَدَّسِ طُوًۭى ﴿١٢﴾\\
\textamh{13.\  } & وَأَنَا ٱخْتَرْتُكَ فَٱسْتَمِعْ لِمَا يُوحَىٰٓ ﴿١٣﴾\\
\textamh{14.\  } & إِنَّنِىٓ أَنَا ٱللَّهُ لَآ إِلَـٰهَ إِلَّآ أَنَا۠ فَٱعْبُدْنِى وَأَقِمِ ٱلصَّلَوٰةَ لِذِكْرِىٓ ﴿١٤﴾\\
\textamh{15.\  } & إِنَّ ٱلسَّاعَةَ ءَاتِيَةٌ أَكَادُ أُخْفِيهَا لِتُجْزَىٰ كُلُّ نَفْسٍۭ بِمَا تَسْعَىٰ ﴿١٥﴾\\
\textamh{16.\  } & فَلَا يَصُدَّنَّكَ عَنْهَا مَن لَّا يُؤْمِنُ بِهَا وَٱتَّبَعَ هَوَىٰهُ فَتَرْدَىٰ ﴿١٦﴾\\
\textamh{17.\  } & وَمَا تِلْكَ بِيَمِينِكَ يَـٰمُوسَىٰ ﴿١٧﴾\\
\textamh{18.\  } & قَالَ هِىَ عَصَاىَ أَتَوَكَّؤُا۟ عَلَيْهَا وَأَهُشُّ بِهَا عَلَىٰ غَنَمِى وَلِىَ فِيهَا مَـَٔارِبُ أُخْرَىٰ ﴿١٨﴾\\
\textamh{19.\  } & قَالَ أَلْقِهَا يَـٰمُوسَىٰ ﴿١٩﴾\\
\textamh{20.\  } & فَأَلْقَىٰهَا فَإِذَا هِىَ حَيَّةٌۭ تَسْعَىٰ ﴿٢٠﴾\\
\textamh{21.\  } & قَالَ خُذْهَا وَلَا تَخَفْ ۖ سَنُعِيدُهَا سِيرَتَهَا ٱلْأُولَىٰ ﴿٢١﴾\\
\textamh{22.\  } & وَٱضْمُمْ يَدَكَ إِلَىٰ جَنَاحِكَ تَخْرُجْ بَيْضَآءَ مِنْ غَيْرِ سُوٓءٍ ءَايَةً أُخْرَىٰ ﴿٢٢﴾\\
\textamh{23.\  } & لِنُرِيَكَ مِنْ ءَايَـٰتِنَا ٱلْكُبْرَى ﴿٢٣﴾\\
\textamh{24.\  } & ٱذْهَبْ إِلَىٰ فِرْعَوْنَ إِنَّهُۥ طَغَىٰ ﴿٢٤﴾\\
\textamh{25.\  } & قَالَ رَبِّ ٱشْرَحْ لِى صَدْرِى ﴿٢٥﴾\\
\textamh{26.\  } & وَيَسِّرْ لِىٓ أَمْرِى ﴿٢٦﴾\\
\textamh{27.\  } & وَٱحْلُلْ عُقْدَةًۭ مِّن لِّسَانِى ﴿٢٧﴾\\
\textamh{28.\  } & يَفْقَهُوا۟ قَوْلِى ﴿٢٨﴾\\
\textamh{29.\  } & وَٱجْعَل لِّى وَزِيرًۭا مِّنْ أَهْلِى ﴿٢٩﴾\\
\textamh{30.\  } & هَـٰرُونَ أَخِى ﴿٣٠﴾\\
\textamh{31.\  } & ٱشْدُدْ بِهِۦٓ أَزْرِى ﴿٣١﴾\\
\textamh{32.\  } & وَأَشْرِكْهُ فِىٓ أَمْرِى ﴿٣٢﴾\\
\textamh{33.\  } & كَىْ نُسَبِّحَكَ كَثِيرًۭا ﴿٣٣﴾\\
\textamh{34.\  } & وَنَذْكُرَكَ كَثِيرًا ﴿٣٤﴾\\
\textamh{35.\  } & إِنَّكَ كُنتَ بِنَا بَصِيرًۭا ﴿٣٥﴾\\
\textamh{36.\  } & قَالَ قَدْ أُوتِيتَ سُؤْلَكَ يَـٰمُوسَىٰ ﴿٣٦﴾\\
\textamh{37.\  } & وَلَقَدْ مَنَنَّا عَلَيْكَ مَرَّةً أُخْرَىٰٓ ﴿٣٧﴾\\
\textamh{38.\  } & إِذْ أَوْحَيْنَآ إِلَىٰٓ أُمِّكَ مَا يُوحَىٰٓ ﴿٣٨﴾\\
\textamh{39.\  } & أَنِ ٱقْذِفِيهِ فِى ٱلتَّابُوتِ فَٱقْذِفِيهِ فِى ٱلْيَمِّ فَلْيُلْقِهِ ٱلْيَمُّ بِٱلسَّاحِلِ يَأْخُذْهُ عَدُوٌّۭ لِّى وَعَدُوٌّۭ لَّهُۥ ۚ وَأَلْقَيْتُ عَلَيْكَ مَحَبَّةًۭ مِّنِّى وَلِتُصْنَعَ عَلَىٰ عَيْنِىٓ ﴿٣٩﴾\\
\textamh{40.\  } & إِذْ تَمْشِىٓ أُخْتُكَ فَتَقُولُ هَلْ أَدُلُّكُمْ عَلَىٰ مَن يَكْفُلُهُۥ ۖ فَرَجَعْنَـٰكَ إِلَىٰٓ أُمِّكَ كَىْ تَقَرَّ عَيْنُهَا وَلَا تَحْزَنَ ۚ وَقَتَلْتَ نَفْسًۭا فَنَجَّيْنَـٰكَ مِنَ ٱلْغَمِّ وَفَتَنَّـٰكَ فُتُونًۭا ۚ فَلَبِثْتَ سِنِينَ فِىٓ أَهْلِ مَدْيَنَ ثُمَّ جِئْتَ عَلَىٰ قَدَرٍۢ يَـٰمُوسَىٰ ﴿٤٠﴾\\
\textamh{41.\  } & وَٱصْطَنَعْتُكَ لِنَفْسِى ﴿٤١﴾\\
\textamh{42.\  } & ٱذْهَبْ أَنتَ وَأَخُوكَ بِـَٔايَـٰتِى وَلَا تَنِيَا فِى ذِكْرِى ﴿٤٢﴾\\
\textamh{43.\  } & ٱذْهَبَآ إِلَىٰ فِرْعَوْنَ إِنَّهُۥ طَغَىٰ ﴿٤٣﴾\\
\textamh{44.\  } & فَقُولَا لَهُۥ قَوْلًۭا لَّيِّنًۭا لَّعَلَّهُۥ يَتَذَكَّرُ أَوْ يَخْشَىٰ ﴿٤٤﴾\\
\textamh{45.\  } & قَالَا رَبَّنَآ إِنَّنَا نَخَافُ أَن يَفْرُطَ عَلَيْنَآ أَوْ أَن يَطْغَىٰ ﴿٤٥﴾\\
\textamh{46.\  } & قَالَ لَا تَخَافَآ ۖ إِنَّنِى مَعَكُمَآ أَسْمَعُ وَأَرَىٰ ﴿٤٦﴾\\
\textamh{47.\  } & فَأْتِيَاهُ فَقُولَآ إِنَّا رَسُولَا رَبِّكَ فَأَرْسِلْ مَعَنَا بَنِىٓ إِسْرَٰٓءِيلَ وَلَا تُعَذِّبْهُمْ ۖ قَدْ جِئْنَـٰكَ بِـَٔايَةٍۢ مِّن رَّبِّكَ ۖ وَٱلسَّلَـٰمُ عَلَىٰ مَنِ ٱتَّبَعَ ٱلْهُدَىٰٓ ﴿٤٧﴾\\
\textamh{48.\  } & إِنَّا قَدْ أُوحِىَ إِلَيْنَآ أَنَّ ٱلْعَذَابَ عَلَىٰ مَن كَذَّبَ وَتَوَلَّىٰ ﴿٤٨﴾\\
\textamh{49.\  } & قَالَ فَمَن رَّبُّكُمَا يَـٰمُوسَىٰ ﴿٤٩﴾\\
\textamh{50.\  } & قَالَ رَبُّنَا ٱلَّذِىٓ أَعْطَىٰ كُلَّ شَىْءٍ خَلْقَهُۥ ثُمَّ هَدَىٰ ﴿٥٠﴾\\
\textamh{51.\  } & قَالَ فَمَا بَالُ ٱلْقُرُونِ ٱلْأُولَىٰ ﴿٥١﴾\\
\textamh{52.\  } & قَالَ عِلْمُهَا عِندَ رَبِّى فِى كِتَـٰبٍۢ ۖ لَّا يَضِلُّ رَبِّى وَلَا يَنسَى ﴿٥٢﴾\\
\textamh{53.\  } & ٱلَّذِى جَعَلَ لَكُمُ ٱلْأَرْضَ مَهْدًۭا وَسَلَكَ لَكُمْ فِيهَا سُبُلًۭا وَأَنزَلَ مِنَ ٱلسَّمَآءِ مَآءًۭ فَأَخْرَجْنَا بِهِۦٓ أَزْوَٟجًۭا مِّن نَّبَاتٍۢ شَتَّىٰ ﴿٥٣﴾\\
\textamh{54.\  } & كُلُوا۟ وَٱرْعَوْا۟ أَنْعَـٰمَكُمْ ۗ إِنَّ فِى ذَٟلِكَ لَءَايَـٰتٍۢ لِّأُو۟لِى ٱلنُّهَىٰ ﴿٥٤﴾\\
\textamh{55.\  } & ۞ مِنْهَا خَلَقْنَـٰكُمْ وَفِيهَا نُعِيدُكُمْ وَمِنْهَا نُخْرِجُكُمْ تَارَةً أُخْرَىٰ ﴿٥٥﴾\\
\textamh{56.\  } & وَلَقَدْ أَرَيْنَـٰهُ ءَايَـٰتِنَا كُلَّهَا فَكَذَّبَ وَأَبَىٰ ﴿٥٦﴾\\
\textamh{57.\  } & قَالَ أَجِئْتَنَا لِتُخْرِجَنَا مِنْ أَرْضِنَا بِسِحْرِكَ يَـٰمُوسَىٰ ﴿٥٧﴾\\
\textamh{58.\  } & فَلَنَأْتِيَنَّكَ بِسِحْرٍۢ مِّثْلِهِۦ فَٱجْعَلْ بَيْنَنَا وَبَيْنَكَ مَوْعِدًۭا لَّا نُخْلِفُهُۥ نَحْنُ وَلَآ أَنتَ مَكَانًۭا سُوًۭى ﴿٥٨﴾\\
\textamh{59.\  } & قَالَ مَوْعِدُكُمْ يَوْمُ ٱلزِّينَةِ وَأَن يُحْشَرَ ٱلنَّاسُ ضُحًۭى ﴿٥٩﴾\\
\textamh{60.\  } & فَتَوَلَّىٰ فِرْعَوْنُ فَجَمَعَ كَيْدَهُۥ ثُمَّ أَتَىٰ ﴿٦٠﴾\\
\textamh{61.\  } & قَالَ لَهُم مُّوسَىٰ وَيْلَكُمْ لَا تَفْتَرُوا۟ عَلَى ٱللَّهِ كَذِبًۭا فَيُسْحِتَكُم بِعَذَابٍۢ ۖ وَقَدْ خَابَ مَنِ ٱفْتَرَىٰ ﴿٦١﴾\\
\textamh{62.\  } & فَتَنَـٰزَعُوٓا۟ أَمْرَهُم بَيْنَهُمْ وَأَسَرُّوا۟ ٱلنَّجْوَىٰ ﴿٦٢﴾\\
\textamh{63.\  } & قَالُوٓا۟ إِنْ هَـٰذَٟنِ لَسَـٰحِرَٰنِ يُرِيدَانِ أَن يُخْرِجَاكُم مِّنْ أَرْضِكُم بِسِحْرِهِمَا وَيَذْهَبَا بِطَرِيقَتِكُمُ ٱلْمُثْلَىٰ ﴿٦٣﴾\\
\textamh{64.\  } & فَأَجْمِعُوا۟ كَيْدَكُمْ ثُمَّ ٱئْتُوا۟ صَفًّۭا ۚ وَقَدْ أَفْلَحَ ٱلْيَوْمَ مَنِ ٱسْتَعْلَىٰ ﴿٦٤﴾\\
\textamh{65.\  } & قَالُوا۟ يَـٰمُوسَىٰٓ إِمَّآ أَن تُلْقِىَ وَإِمَّآ أَن نَّكُونَ أَوَّلَ مَنْ أَلْقَىٰ ﴿٦٥﴾\\
\textamh{66.\  } & قَالَ بَلْ أَلْقُوا۟ ۖ فَإِذَا حِبَالُهُمْ وَعِصِيُّهُمْ يُخَيَّلُ إِلَيْهِ مِن سِحْرِهِمْ أَنَّهَا تَسْعَىٰ ﴿٦٦﴾\\
\textamh{67.\  } & فَأَوْجَسَ فِى نَفْسِهِۦ خِيفَةًۭ مُّوسَىٰ ﴿٦٧﴾\\
\textamh{68.\  } & قُلْنَا لَا تَخَفْ إِنَّكَ أَنتَ ٱلْأَعْلَىٰ ﴿٦٨﴾\\
\textamh{69.\  } & وَأَلْقِ مَا فِى يَمِينِكَ تَلْقَفْ مَا صَنَعُوٓا۟ ۖ إِنَّمَا صَنَعُوا۟ كَيْدُ سَـٰحِرٍۢ ۖ وَلَا يُفْلِحُ ٱلسَّاحِرُ حَيْثُ أَتَىٰ ﴿٦٩﴾\\
\textamh{70.\  } & فَأُلْقِىَ ٱلسَّحَرَةُ سُجَّدًۭا قَالُوٓا۟ ءَامَنَّا بِرَبِّ هَـٰرُونَ وَمُوسَىٰ ﴿٧٠﴾\\
\textamh{71.\  } & قَالَ ءَامَنتُمْ لَهُۥ قَبْلَ أَنْ ءَاذَنَ لَكُمْ ۖ إِنَّهُۥ لَكَبِيرُكُمُ ٱلَّذِى عَلَّمَكُمُ ٱلسِّحْرَ ۖ فَلَأُقَطِّعَنَّ أَيْدِيَكُمْ وَأَرْجُلَكُم مِّنْ خِلَـٰفٍۢ وَلَأُصَلِّبَنَّكُمْ فِى جُذُوعِ ٱلنَّخْلِ وَلَتَعْلَمُنَّ أَيُّنَآ أَشَدُّ عَذَابًۭا وَأَبْقَىٰ ﴿٧١﴾\\
\textamh{72.\  } & قَالُوا۟ لَن نُّؤْثِرَكَ عَلَىٰ مَا جَآءَنَا مِنَ ٱلْبَيِّنَـٰتِ وَٱلَّذِى فَطَرَنَا ۖ فَٱقْضِ مَآ أَنتَ قَاضٍ ۖ إِنَّمَا تَقْضِى هَـٰذِهِ ٱلْحَيَوٰةَ ٱلدُّنْيَآ ﴿٧٢﴾\\
\textamh{73.\  } & إِنَّآ ءَامَنَّا بِرَبِّنَا لِيَغْفِرَ لَنَا خَطَٰيَـٰنَا وَمَآ أَكْرَهْتَنَا عَلَيْهِ مِنَ ٱلسِّحْرِ ۗ وَٱللَّهُ خَيْرٌۭ وَأَبْقَىٰٓ ﴿٧٣﴾\\
\textamh{74.\  } & إِنَّهُۥ مَن يَأْتِ رَبَّهُۥ مُجْرِمًۭا فَإِنَّ لَهُۥ جَهَنَّمَ لَا يَمُوتُ فِيهَا وَلَا يَحْيَىٰ ﴿٧٤﴾\\
\textamh{75.\  } & وَمَن يَأْتِهِۦ مُؤْمِنًۭا قَدْ عَمِلَ ٱلصَّـٰلِحَـٰتِ فَأُو۟لَـٰٓئِكَ لَهُمُ ٱلدَّرَجَٰتُ ٱلْعُلَىٰ ﴿٧٥﴾\\
\textamh{76.\  } & جَنَّـٰتُ عَدْنٍۢ تَجْرِى مِن تَحْتِهَا ٱلْأَنْهَـٰرُ خَـٰلِدِينَ فِيهَا ۚ وَذَٟلِكَ جَزَآءُ مَن تَزَكَّىٰ ﴿٧٦﴾\\
\textamh{77.\  } & وَلَقَدْ أَوْحَيْنَآ إِلَىٰ مُوسَىٰٓ أَنْ أَسْرِ بِعِبَادِى فَٱضْرِبْ لَهُمْ طَرِيقًۭا فِى ٱلْبَحْرِ يَبَسًۭا لَّا تَخَـٰفُ دَرَكًۭا وَلَا تَخْشَىٰ ﴿٧٧﴾\\
\textamh{78.\  } & فَأَتْبَعَهُمْ فِرْعَوْنُ بِجُنُودِهِۦ فَغَشِيَهُم مِّنَ ٱلْيَمِّ مَا غَشِيَهُمْ ﴿٧٨﴾\\
\textamh{79.\  } & وَأَضَلَّ فِرْعَوْنُ قَوْمَهُۥ وَمَا هَدَىٰ ﴿٧٩﴾\\
\textamh{80.\  } & يَـٰبَنِىٓ إِسْرَٰٓءِيلَ قَدْ أَنجَيْنَـٰكُم مِّنْ عَدُوِّكُمْ وَوَٟعَدْنَـٰكُمْ جَانِبَ ٱلطُّورِ ٱلْأَيْمَنَ وَنَزَّلْنَا عَلَيْكُمُ ٱلْمَنَّ وَٱلسَّلْوَىٰ ﴿٨٠﴾\\
\textamh{81.\  } & كُلُوا۟ مِن طَيِّبَٰتِ مَا رَزَقْنَـٰكُمْ وَلَا تَطْغَوْا۟ فِيهِ فَيَحِلَّ عَلَيْكُمْ غَضَبِى ۖ وَمَن يَحْلِلْ عَلَيْهِ غَضَبِى فَقَدْ هَوَىٰ ﴿٨١﴾\\
\textamh{82.\  } & وَإِنِّى لَغَفَّارٌۭ لِّمَن تَابَ وَءَامَنَ وَعَمِلَ صَـٰلِحًۭا ثُمَّ ٱهْتَدَىٰ ﴿٨٢﴾\\
\textamh{83.\  } & ۞ وَمَآ أَعْجَلَكَ عَن قَوْمِكَ يَـٰمُوسَىٰ ﴿٨٣﴾\\
\textamh{84.\  } & قَالَ هُمْ أُو۟لَآءِ عَلَىٰٓ أَثَرِى وَعَجِلْتُ إِلَيْكَ رَبِّ لِتَرْضَىٰ ﴿٨٤﴾\\
\textamh{85.\  } & قَالَ فَإِنَّا قَدْ فَتَنَّا قَوْمَكَ مِنۢ بَعْدِكَ وَأَضَلَّهُمُ ٱلسَّامِرِىُّ ﴿٨٥﴾\\
\textamh{86.\  } & فَرَجَعَ مُوسَىٰٓ إِلَىٰ قَوْمِهِۦ غَضْبَٰنَ أَسِفًۭا ۚ قَالَ يَـٰقَوْمِ أَلَمْ يَعِدْكُمْ رَبُّكُمْ وَعْدًا حَسَنًا ۚ أَفَطَالَ عَلَيْكُمُ ٱلْعَهْدُ أَمْ أَرَدتُّمْ أَن يَحِلَّ عَلَيْكُمْ غَضَبٌۭ مِّن رَّبِّكُمْ فَأَخْلَفْتُم مَّوْعِدِى ﴿٨٦﴾\\
\textamh{87.\  } & قَالُوا۟ مَآ أَخْلَفْنَا مَوْعِدَكَ بِمَلْكِنَا وَلَـٰكِنَّا حُمِّلْنَآ أَوْزَارًۭا مِّن زِينَةِ ٱلْقَوْمِ فَقَذَفْنَـٰهَا فَكَذَٟلِكَ أَلْقَى ٱلسَّامِرِىُّ ﴿٨٧﴾\\
\textamh{88.\  } & فَأَخْرَجَ لَهُمْ عِجْلًۭا جَسَدًۭا لَّهُۥ خُوَارٌۭ فَقَالُوا۟ هَـٰذَآ إِلَـٰهُكُمْ وَإِلَـٰهُ مُوسَىٰ فَنَسِىَ ﴿٨٨﴾\\
\textamh{89.\  } & أَفَلَا يَرَوْنَ أَلَّا يَرْجِعُ إِلَيْهِمْ قَوْلًۭا وَلَا يَمْلِكُ لَهُمْ ضَرًّۭا وَلَا نَفْعًۭا ﴿٨٩﴾\\
\textamh{90.\  } & وَلَقَدْ قَالَ لَهُمْ هَـٰرُونُ مِن قَبْلُ يَـٰقَوْمِ إِنَّمَا فُتِنتُم بِهِۦ ۖ وَإِنَّ رَبَّكُمُ ٱلرَّحْمَـٰنُ فَٱتَّبِعُونِى وَأَطِيعُوٓا۟ أَمْرِى ﴿٩٠﴾\\
\textamh{91.\  } & قَالُوا۟ لَن نَّبْرَحَ عَلَيْهِ عَـٰكِفِينَ حَتَّىٰ يَرْجِعَ إِلَيْنَا مُوسَىٰ ﴿٩١﴾\\
\textamh{92.\  } & قَالَ يَـٰهَـٰرُونُ مَا مَنَعَكَ إِذْ رَأَيْتَهُمْ ضَلُّوٓا۟ ﴿٩٢﴾\\
\textamh{93.\  } & أَلَّا تَتَّبِعَنِ ۖ أَفَعَصَيْتَ أَمْرِى ﴿٩٣﴾\\
\textamh{94.\  } & قَالَ يَبْنَؤُمَّ لَا تَأْخُذْ بِلِحْيَتِى وَلَا بِرَأْسِىٓ ۖ إِنِّى خَشِيتُ أَن تَقُولَ فَرَّقْتَ بَيْنَ بَنِىٓ إِسْرَٰٓءِيلَ وَلَمْ تَرْقُبْ قَوْلِى ﴿٩٤﴾\\
\textamh{95.\  } & قَالَ فَمَا خَطْبُكَ يَـٰسَـٰمِرِىُّ ﴿٩٥﴾\\
\textamh{96.\  } & قَالَ بَصُرْتُ بِمَا لَمْ يَبْصُرُوا۟ بِهِۦ فَقَبَضْتُ قَبْضَةًۭ مِّنْ أَثَرِ ٱلرَّسُولِ فَنَبَذْتُهَا وَكَذَٟلِكَ سَوَّلَتْ لِى نَفْسِى ﴿٩٦﴾\\
\textamh{97.\  } & قَالَ فَٱذْهَبْ فَإِنَّ لَكَ فِى ٱلْحَيَوٰةِ أَن تَقُولَ لَا مِسَاسَ ۖ وَإِنَّ لَكَ مَوْعِدًۭا لَّن تُخْلَفَهُۥ ۖ وَٱنظُرْ إِلَىٰٓ إِلَـٰهِكَ ٱلَّذِى ظَلْتَ عَلَيْهِ عَاكِفًۭا ۖ لَّنُحَرِّقَنَّهُۥ ثُمَّ لَنَنسِفَنَّهُۥ فِى ٱلْيَمِّ نَسْفًا ﴿٩٧﴾\\
\textamh{98.\  } & إِنَّمَآ إِلَـٰهُكُمُ ٱللَّهُ ٱلَّذِى لَآ إِلَـٰهَ إِلَّا هُوَ ۚ وَسِعَ كُلَّ شَىْءٍ عِلْمًۭا ﴿٩٨﴾\\
\textamh{99.\  } & كَذَٟلِكَ نَقُصُّ عَلَيْكَ مِنْ أَنۢبَآءِ مَا قَدْ سَبَقَ ۚ وَقَدْ ءَاتَيْنَـٰكَ مِن لَّدُنَّا ذِكْرًۭا ﴿٩٩﴾\\
\textamh{100.\  } & مَّنْ أَعْرَضَ عَنْهُ فَإِنَّهُۥ يَحْمِلُ يَوْمَ ٱلْقِيَـٰمَةِ وِزْرًا ﴿١٠٠﴾\\
\textamh{101.\  } & خَـٰلِدِينَ فِيهِ ۖ وَسَآءَ لَهُمْ يَوْمَ ٱلْقِيَـٰمَةِ حِمْلًۭا ﴿١٠١﴾\\
\textamh{102.\  } & يَوْمَ يُنفَخُ فِى ٱلصُّورِ ۚ وَنَحْشُرُ ٱلْمُجْرِمِينَ يَوْمَئِذٍۢ زُرْقًۭا ﴿١٠٢﴾\\
\textamh{103.\  } & يَتَخَـٰفَتُونَ بَيْنَهُمْ إِن لَّبِثْتُمْ إِلَّا عَشْرًۭا ﴿١٠٣﴾\\
\textamh{104.\  } & نَّحْنُ أَعْلَمُ بِمَا يَقُولُونَ إِذْ يَقُولُ أَمْثَلُهُمْ طَرِيقَةً إِن لَّبِثْتُمْ إِلَّا يَوْمًۭا ﴿١٠٤﴾\\
\textamh{105.\  } & وَيَسْـَٔلُونَكَ عَنِ ٱلْجِبَالِ فَقُلْ يَنسِفُهَا رَبِّى نَسْفًۭا ﴿١٠٥﴾\\
\textamh{106.\  } & فَيَذَرُهَا قَاعًۭا صَفْصَفًۭا ﴿١٠٦﴾\\
\textamh{107.\  } & لَّا تَرَىٰ فِيهَا عِوَجًۭا وَلَآ أَمْتًۭا ﴿١٠٧﴾\\
\textamh{108.\  } & يَوْمَئِذٍۢ يَتَّبِعُونَ ٱلدَّاعِىَ لَا عِوَجَ لَهُۥ ۖ وَخَشَعَتِ ٱلْأَصْوَاتُ لِلرَّحْمَـٰنِ فَلَا تَسْمَعُ إِلَّا هَمْسًۭا ﴿١٠٨﴾\\
\textamh{109.\  } & يَوْمَئِذٍۢ لَّا تَنفَعُ ٱلشَّفَـٰعَةُ إِلَّا مَنْ أَذِنَ لَهُ ٱلرَّحْمَـٰنُ وَرَضِىَ لَهُۥ قَوْلًۭا ﴿١٠٩﴾\\
\textamh{110.\  } & يَعْلَمُ مَا بَيْنَ أَيْدِيهِمْ وَمَا خَلْفَهُمْ وَلَا يُحِيطُونَ بِهِۦ عِلْمًۭا ﴿١١٠﴾\\
\textamh{111.\  } & ۞ وَعَنَتِ ٱلْوُجُوهُ لِلْحَىِّ ٱلْقَيُّومِ ۖ وَقَدْ خَابَ مَنْ حَمَلَ ظُلْمًۭا ﴿١١١﴾\\
\textamh{112.\  } & وَمَن يَعْمَلْ مِنَ ٱلصَّـٰلِحَـٰتِ وَهُوَ مُؤْمِنٌۭ فَلَا يَخَافُ ظُلْمًۭا وَلَا هَضْمًۭا ﴿١١٢﴾\\
\textamh{113.\  } & وَكَذَٟلِكَ أَنزَلْنَـٰهُ قُرْءَانًا عَرَبِيًّۭا وَصَرَّفْنَا فِيهِ مِنَ ٱلْوَعِيدِ لَعَلَّهُمْ يَتَّقُونَ أَوْ يُحْدِثُ لَهُمْ ذِكْرًۭا ﴿١١٣﴾\\
\textamh{114.\  } & فَتَعَـٰلَى ٱللَّهُ ٱلْمَلِكُ ٱلْحَقُّ ۗ وَلَا تَعْجَلْ بِٱلْقُرْءَانِ مِن قَبْلِ أَن يُقْضَىٰٓ إِلَيْكَ وَحْيُهُۥ ۖ وَقُل رَّبِّ زِدْنِى عِلْمًۭا ﴿١١٤﴾\\
\textamh{115.\  } & وَلَقَدْ عَهِدْنَآ إِلَىٰٓ ءَادَمَ مِن قَبْلُ فَنَسِىَ وَلَمْ نَجِدْ لَهُۥ عَزْمًۭا ﴿١١٥﴾\\
\textamh{116.\  } & وَإِذْ قُلْنَا لِلْمَلَـٰٓئِكَةِ ٱسْجُدُوا۟ لِءَادَمَ فَسَجَدُوٓا۟ إِلَّآ إِبْلِيسَ أَبَىٰ ﴿١١٦﴾\\
\textamh{117.\  } & فَقُلْنَا يَـٰٓـَٔادَمُ إِنَّ هَـٰذَا عَدُوٌّۭ لَّكَ وَلِزَوْجِكَ فَلَا يُخْرِجَنَّكُمَا مِنَ ٱلْجَنَّةِ فَتَشْقَىٰٓ ﴿١١٧﴾\\
\textamh{118.\  } & إِنَّ لَكَ أَلَّا تَجُوعَ فِيهَا وَلَا تَعْرَىٰ ﴿١١٨﴾\\
\textamh{119.\  } & وَأَنَّكَ لَا تَظْمَؤُا۟ فِيهَا وَلَا تَضْحَىٰ ﴿١١٩﴾\\
\textamh{120.\  } & فَوَسْوَسَ إِلَيْهِ ٱلشَّيْطَٰنُ قَالَ يَـٰٓـَٔادَمُ هَلْ أَدُلُّكَ عَلَىٰ شَجَرَةِ ٱلْخُلْدِ وَمُلْكٍۢ لَّا يَبْلَىٰ ﴿١٢٠﴾\\
\textamh{121.\  } & فَأَكَلَا مِنْهَا فَبَدَتْ لَهُمَا سَوْءَٰتُهُمَا وَطَفِقَا يَخْصِفَانِ عَلَيْهِمَا مِن وَرَقِ ٱلْجَنَّةِ ۚ وَعَصَىٰٓ ءَادَمُ رَبَّهُۥ فَغَوَىٰ ﴿١٢١﴾\\
\textamh{122.\  } & ثُمَّ ٱجْتَبَٰهُ رَبُّهُۥ فَتَابَ عَلَيْهِ وَهَدَىٰ ﴿١٢٢﴾\\
\textamh{123.\  } & قَالَ ٱهْبِطَا مِنْهَا جَمِيعًۢا ۖ بَعْضُكُمْ لِبَعْضٍ عَدُوٌّۭ ۖ فَإِمَّا يَأْتِيَنَّكُم مِّنِّى هُدًۭى فَمَنِ ٱتَّبَعَ هُدَاىَ فَلَا يَضِلُّ وَلَا يَشْقَىٰ ﴿١٢٣﴾\\
\textamh{124.\  } & وَمَنْ أَعْرَضَ عَن ذِكْرِى فَإِنَّ لَهُۥ مَعِيشَةًۭ ضَنكًۭا وَنَحْشُرُهُۥ يَوْمَ ٱلْقِيَـٰمَةِ أَعْمَىٰ ﴿١٢٤﴾\\
\textamh{125.\  } & قَالَ رَبِّ لِمَ حَشَرْتَنِىٓ أَعْمَىٰ وَقَدْ كُنتُ بَصِيرًۭا ﴿١٢٥﴾\\
\textamh{126.\  } & قَالَ كَذَٟلِكَ أَتَتْكَ ءَايَـٰتُنَا فَنَسِيتَهَا ۖ وَكَذَٟلِكَ ٱلْيَوْمَ تُنسَىٰ ﴿١٢٦﴾\\
\textamh{127.\  } & وَكَذَٟلِكَ نَجْزِى مَنْ أَسْرَفَ وَلَمْ يُؤْمِنۢ بِـَٔايَـٰتِ رَبِّهِۦ ۚ وَلَعَذَابُ ٱلْءَاخِرَةِ أَشَدُّ وَأَبْقَىٰٓ ﴿١٢٧﴾\\
\textamh{128.\  } & أَفَلَمْ يَهْدِ لَهُمْ كَمْ أَهْلَكْنَا قَبْلَهُم مِّنَ ٱلْقُرُونِ يَمْشُونَ فِى مَسَـٰكِنِهِمْ ۗ إِنَّ فِى ذَٟلِكَ لَءَايَـٰتٍۢ لِّأُو۟لِى ٱلنُّهَىٰ ﴿١٢٨﴾\\
\textamh{129.\  } & وَلَوْلَا كَلِمَةٌۭ سَبَقَتْ مِن رَّبِّكَ لَكَانَ لِزَامًۭا وَأَجَلٌۭ مُّسَمًّۭى ﴿١٢٩﴾\\
\textamh{130.\  } & فَٱصْبِرْ عَلَىٰ مَا يَقُولُونَ وَسَبِّحْ بِحَمْدِ رَبِّكَ قَبْلَ طُلُوعِ ٱلشَّمْسِ وَقَبْلَ غُرُوبِهَا ۖ وَمِنْ ءَانَآئِ ٱلَّيْلِ فَسَبِّحْ وَأَطْرَافَ ٱلنَّهَارِ لَعَلَّكَ تَرْضَىٰ ﴿١٣٠﴾\\
\textamh{131.\  } & وَلَا تَمُدَّنَّ عَيْنَيْكَ إِلَىٰ مَا مَتَّعْنَا بِهِۦٓ أَزْوَٟجًۭا مِّنْهُمْ زَهْرَةَ ٱلْحَيَوٰةِ ٱلدُّنْيَا لِنَفْتِنَهُمْ فِيهِ ۚ وَرِزْقُ رَبِّكَ خَيْرٌۭ وَأَبْقَىٰ ﴿١٣١﴾\\
\textamh{132.\  } & وَأْمُرْ أَهْلَكَ بِٱلصَّلَوٰةِ وَٱصْطَبِرْ عَلَيْهَا ۖ لَا نَسْـَٔلُكَ رِزْقًۭا ۖ نَّحْنُ نَرْزُقُكَ ۗ وَٱلْعَـٰقِبَةُ لِلتَّقْوَىٰ ﴿١٣٢﴾\\
\textamh{133.\  } & وَقَالُوا۟ لَوْلَا يَأْتِينَا بِـَٔايَةٍۢ مِّن رَّبِّهِۦٓ ۚ أَوَلَمْ تَأْتِهِم بَيِّنَةُ مَا فِى ٱلصُّحُفِ ٱلْأُولَىٰ ﴿١٣٣﴾\\
\textamh{134.\  } & وَلَوْ أَنَّآ أَهْلَكْنَـٰهُم بِعَذَابٍۢ مِّن قَبْلِهِۦ لَقَالُوا۟ رَبَّنَا لَوْلَآ أَرْسَلْتَ إِلَيْنَا رَسُولًۭا فَنَتَّبِعَ ءَايَـٰتِكَ مِن قَبْلِ أَن نَّذِلَّ وَنَخْزَىٰ ﴿١٣٤﴾\\
\textamh{135.\  } & قُلْ كُلٌّۭ مُّتَرَبِّصٌۭ فَتَرَبَّصُوا۟ ۖ فَسَتَعْلَمُونَ مَنْ أَصْحَـٰبُ ٱلصِّرَٰطِ ٱلسَّوِىِّ وَمَنِ ٱهْتَدَىٰ ﴿١٣٥﴾\\
\end{longtable} \newpage

%% License: BSD style (Berkley) (i.e. Put the Copyright owner's name always)
%% Writer and Copyright (to): Bewketu(Bilal) Tadilo (2016-17)
\centering\section{\LR{\textamharic{ሱራቱ አልአንቢያ -}  \RL{سوره  الأنبياء}}}
\begin{longtable}{%
  @{}
    p{.5\textwidth}
  @{~~~~~~~~~~~~}
    p{.5\textwidth}
    @{}
}
\nopagebreak
\textamh{ቢስሚላሂ አራህመኒ ራሂይም } &  بِسْمِ ٱللَّهِ ٱلرَّحْمَـٰنِ ٱلرَّحِيمِ\\
\textamh{1.\  } &  ٱقْتَرَبَ لِلنَّاسِ حِسَابُهُمْ وَهُمْ فِى غَفْلَةٍۢ مُّعْرِضُونَ ﴿١﴾\\
\textamh{2.\  } & مَا يَأْتِيهِم مِّن ذِكْرٍۢ مِّن رَّبِّهِم مُّحْدَثٍ إِلَّا ٱسْتَمَعُوهُ وَهُمْ يَلْعَبُونَ ﴿٢﴾\\
\textamh{3.\  } & لَاهِيَةًۭ قُلُوبُهُمْ ۗ وَأَسَرُّوا۟ ٱلنَّجْوَى ٱلَّذِينَ ظَلَمُوا۟ هَلْ هَـٰذَآ إِلَّا بَشَرٌۭ مِّثْلُكُمْ ۖ أَفَتَأْتُونَ ٱلسِّحْرَ وَأَنتُمْ تُبْصِرُونَ ﴿٣﴾\\
\textamh{4.\  } & قَالَ رَبِّى يَعْلَمُ ٱلْقَوْلَ فِى ٱلسَّمَآءِ وَٱلْأَرْضِ ۖ وَهُوَ ٱلسَّمِيعُ ٱلْعَلِيمُ ﴿٤﴾\\
\textamh{5.\  } & بَلْ قَالُوٓا۟ أَضْغَٰثُ أَحْلَـٰمٍۭ بَلِ ٱفْتَرَىٰهُ بَلْ هُوَ شَاعِرٌۭ فَلْيَأْتِنَا بِـَٔايَةٍۢ كَمَآ أُرْسِلَ ٱلْأَوَّلُونَ ﴿٥﴾\\
\textamh{6.\  } & مَآ ءَامَنَتْ قَبْلَهُم مِّن قَرْيَةٍ أَهْلَكْنَـٰهَآ ۖ أَفَهُمْ يُؤْمِنُونَ ﴿٦﴾\\
\textamh{7.\  } & وَمَآ أَرْسَلْنَا قَبْلَكَ إِلَّا رِجَالًۭا نُّوحِىٓ إِلَيْهِمْ ۖ فَسْـَٔلُوٓا۟ أَهْلَ ٱلذِّكْرِ إِن كُنتُمْ لَا تَعْلَمُونَ ﴿٧﴾\\
\textamh{8.\  } & وَمَا جَعَلْنَـٰهُمْ جَسَدًۭا لَّا يَأْكُلُونَ ٱلطَّعَامَ وَمَا كَانُوا۟ خَـٰلِدِينَ ﴿٨﴾\\
\textamh{9.\  } & ثُمَّ صَدَقْنَـٰهُمُ ٱلْوَعْدَ فَأَنجَيْنَـٰهُمْ وَمَن نَّشَآءُ وَأَهْلَكْنَا ٱلْمُسْرِفِينَ ﴿٩﴾\\
\textamh{10.\  } & لَقَدْ أَنزَلْنَآ إِلَيْكُمْ كِتَـٰبًۭا فِيهِ ذِكْرُكُمْ ۖ أَفَلَا تَعْقِلُونَ ﴿١٠﴾\\
\textamh{11.\  } & وَكَمْ قَصَمْنَا مِن قَرْيَةٍۢ كَانَتْ ظَالِمَةًۭ وَأَنشَأْنَا بَعْدَهَا قَوْمًا ءَاخَرِينَ ﴿١١﴾\\
\textamh{12.\  } & فَلَمَّآ أَحَسُّوا۟ بَأْسَنَآ إِذَا هُم مِّنْهَا يَرْكُضُونَ ﴿١٢﴾\\
\textamh{13.\  } & لَا تَرْكُضُوا۟ وَٱرْجِعُوٓا۟ إِلَىٰ مَآ أُتْرِفْتُمْ فِيهِ وَمَسَـٰكِنِكُمْ لَعَلَّكُمْ تُسْـَٔلُونَ ﴿١٣﴾\\
\textamh{14.\  } & قَالُوا۟ يَـٰوَيْلَنَآ إِنَّا كُنَّا ظَـٰلِمِينَ ﴿١٤﴾\\
\textamh{15.\  } & فَمَا زَالَت تِّلْكَ دَعْوَىٰهُمْ حَتَّىٰ جَعَلْنَـٰهُمْ حَصِيدًا خَـٰمِدِينَ ﴿١٥﴾\\
\textamh{16.\  } & وَمَا خَلَقْنَا ٱلسَّمَآءَ وَٱلْأَرْضَ وَمَا بَيْنَهُمَا لَـٰعِبِينَ ﴿١٦﴾\\
\textamh{17.\  } & لَوْ أَرَدْنَآ أَن نَّتَّخِذَ لَهْوًۭا لَّٱتَّخَذْنَـٰهُ مِن لَّدُنَّآ إِن كُنَّا فَـٰعِلِينَ ﴿١٧﴾\\
\textamh{18.\  } & بَلْ نَقْذِفُ بِٱلْحَقِّ عَلَى ٱلْبَٰطِلِ فَيَدْمَغُهُۥ فَإِذَا هُوَ زَاهِقٌۭ ۚ وَلَكُمُ ٱلْوَيْلُ مِمَّا تَصِفُونَ ﴿١٨﴾\\
\textamh{19.\  } & وَلَهُۥ مَن فِى ٱلسَّمَـٰوَٟتِ وَٱلْأَرْضِ ۚ وَمَنْ عِندَهُۥ لَا يَسْتَكْبِرُونَ عَنْ عِبَادَتِهِۦ وَلَا يَسْتَحْسِرُونَ ﴿١٩﴾\\
\textamh{20.\  } & يُسَبِّحُونَ ٱلَّيْلَ وَٱلنَّهَارَ لَا يَفْتُرُونَ ﴿٢٠﴾\\
\textamh{21.\  } & أَمِ ٱتَّخَذُوٓا۟ ءَالِهَةًۭ مِّنَ ٱلْأَرْضِ هُمْ يُنشِرُونَ ﴿٢١﴾\\
\textamh{22.\  } & لَوْ كَانَ فِيهِمَآ ءَالِهَةٌ إِلَّا ٱللَّهُ لَفَسَدَتَا ۚ فَسُبْحَـٰنَ ٱللَّهِ رَبِّ ٱلْعَرْشِ عَمَّا يَصِفُونَ ﴿٢٢﴾\\
\textamh{23.\  } & لَا يُسْـَٔلُ عَمَّا يَفْعَلُ وَهُمْ يُسْـَٔلُونَ ﴿٢٣﴾\\
\textamh{24.\  } & أَمِ ٱتَّخَذُوا۟ مِن دُونِهِۦٓ ءَالِهَةًۭ ۖ قُلْ هَاتُوا۟ بُرْهَـٰنَكُمْ ۖ هَـٰذَا ذِكْرُ مَن مَّعِىَ وَذِكْرُ مَن قَبْلِى ۗ بَلْ أَكْثَرُهُمْ لَا يَعْلَمُونَ ٱلْحَقَّ ۖ فَهُم مُّعْرِضُونَ ﴿٢٤﴾\\
\textamh{25.\  } & وَمَآ أَرْسَلْنَا مِن قَبْلِكَ مِن رَّسُولٍ إِلَّا نُوحِىٓ إِلَيْهِ أَنَّهُۥ لَآ إِلَـٰهَ إِلَّآ أَنَا۠ فَٱعْبُدُونِ ﴿٢٥﴾\\
\textamh{26.\  } & وَقَالُوا۟ ٱتَّخَذَ ٱلرَّحْمَـٰنُ وَلَدًۭا ۗ سُبْحَـٰنَهُۥ ۚ بَلْ عِبَادٌۭ مُّكْرَمُونَ ﴿٢٦﴾\\
\textamh{27.\  } & لَا يَسْبِقُونَهُۥ بِٱلْقَوْلِ وَهُم بِأَمْرِهِۦ يَعْمَلُونَ ﴿٢٧﴾\\
\textamh{28.\  } & يَعْلَمُ مَا بَيْنَ أَيْدِيهِمْ وَمَا خَلْفَهُمْ وَلَا يَشْفَعُونَ إِلَّا لِمَنِ ٱرْتَضَىٰ وَهُم مِّنْ خَشْيَتِهِۦ مُشْفِقُونَ ﴿٢٨﴾\\
\textamh{29.\  } & ۞ وَمَن يَقُلْ مِنْهُمْ إِنِّىٓ إِلَـٰهٌۭ مِّن دُونِهِۦ فَذَٟلِكَ نَجْزِيهِ جَهَنَّمَ ۚ كَذَٟلِكَ نَجْزِى ٱلظَّـٰلِمِينَ ﴿٢٩﴾\\
\textamh{30.\  } & أَوَلَمْ يَرَ ٱلَّذِينَ كَفَرُوٓا۟ أَنَّ ٱلسَّمَـٰوَٟتِ وَٱلْأَرْضَ كَانَتَا رَتْقًۭا فَفَتَقْنَـٰهُمَا ۖ وَجَعَلْنَا مِنَ ٱلْمَآءِ كُلَّ شَىْءٍ حَىٍّ ۖ أَفَلَا يُؤْمِنُونَ ﴿٣٠﴾\\
\textamh{31.\  } & وَجَعَلْنَا فِى ٱلْأَرْضِ رَوَٟسِىَ أَن تَمِيدَ بِهِمْ وَجَعَلْنَا فِيهَا فِجَاجًۭا سُبُلًۭا لَّعَلَّهُمْ يَهْتَدُونَ ﴿٣١﴾\\
\textamh{32.\  } & وَجَعَلْنَا ٱلسَّمَآءَ سَقْفًۭا مَّحْفُوظًۭا ۖ وَهُمْ عَنْ ءَايَـٰتِهَا مُعْرِضُونَ ﴿٣٢﴾\\
\textamh{33.\  } & وَهُوَ ٱلَّذِى خَلَقَ ٱلَّيْلَ وَٱلنَّهَارَ وَٱلشَّمْسَ وَٱلْقَمَرَ ۖ كُلٌّۭ فِى فَلَكٍۢ يَسْبَحُونَ ﴿٣٣﴾\\
\textamh{34.\  } & وَمَا جَعَلْنَا لِبَشَرٍۢ مِّن قَبْلِكَ ٱلْخُلْدَ ۖ أَفَإِي۟ن مِّتَّ فَهُمُ ٱلْخَـٰلِدُونَ ﴿٣٤﴾\\
\textamh{35.\  } & كُلُّ نَفْسٍۢ ذَآئِقَةُ ٱلْمَوْتِ ۗ وَنَبْلُوكُم بِٱلشَّرِّ وَٱلْخَيْرِ فِتْنَةًۭ ۖ وَإِلَيْنَا تُرْجَعُونَ ﴿٣٥﴾\\
\textamh{36.\  } & وَإِذَا رَءَاكَ ٱلَّذِينَ كَفَرُوٓا۟ إِن يَتَّخِذُونَكَ إِلَّا هُزُوًا أَهَـٰذَا ٱلَّذِى يَذْكُرُ ءَالِهَتَكُمْ وَهُم بِذِكْرِ ٱلرَّحْمَـٰنِ هُمْ كَـٰفِرُونَ ﴿٣٦﴾\\
\textamh{37.\  } & خُلِقَ ٱلْإِنسَـٰنُ مِنْ عَجَلٍۢ ۚ سَأُو۟رِيكُمْ ءَايَـٰتِى فَلَا تَسْتَعْجِلُونِ ﴿٣٧﴾\\
\textamh{38.\  } & وَيَقُولُونَ مَتَىٰ هَـٰذَا ٱلْوَعْدُ إِن كُنتُمْ صَـٰدِقِينَ ﴿٣٨﴾\\
\textamh{39.\  } & لَوْ يَعْلَمُ ٱلَّذِينَ كَفَرُوا۟ حِينَ لَا يَكُفُّونَ عَن وُجُوهِهِمُ ٱلنَّارَ وَلَا عَن ظُهُورِهِمْ وَلَا هُمْ يُنصَرُونَ ﴿٣٩﴾\\
\textamh{40.\  } & بَلْ تَأْتِيهِم بَغْتَةًۭ فَتَبْهَتُهُمْ فَلَا يَسْتَطِيعُونَ رَدَّهَا وَلَا هُمْ يُنظَرُونَ ﴿٤٠﴾\\
\textamh{41.\  } & وَلَقَدِ ٱسْتُهْزِئَ بِرُسُلٍۢ مِّن قَبْلِكَ فَحَاقَ بِٱلَّذِينَ سَخِرُوا۟ مِنْهُم مَّا كَانُوا۟ بِهِۦ يَسْتَهْزِءُونَ ﴿٤١﴾\\
\textamh{42.\  } & قُلْ مَن يَكْلَؤُكُم بِٱلَّيْلِ وَٱلنَّهَارِ مِنَ ٱلرَّحْمَـٰنِ ۗ بَلْ هُمْ عَن ذِكْرِ رَبِّهِم مُّعْرِضُونَ ﴿٤٢﴾\\
\textamh{43.\  } & أَمْ لَهُمْ ءَالِهَةٌۭ تَمْنَعُهُم مِّن دُونِنَا ۚ لَا يَسْتَطِيعُونَ نَصْرَ أَنفُسِهِمْ وَلَا هُم مِّنَّا يُصْحَبُونَ ﴿٤٣﴾\\
\textamh{44.\  } & بَلْ مَتَّعْنَا هَـٰٓؤُلَآءِ وَءَابَآءَهُمْ حَتَّىٰ طَالَ عَلَيْهِمُ ٱلْعُمُرُ ۗ أَفَلَا يَرَوْنَ أَنَّا نَأْتِى ٱلْأَرْضَ نَنقُصُهَا مِنْ أَطْرَافِهَآ ۚ أَفَهُمُ ٱلْغَٰلِبُونَ ﴿٤٤﴾\\
\textamh{45.\  } & قُلْ إِنَّمَآ أُنذِرُكُم بِٱلْوَحْىِ ۚ وَلَا يَسْمَعُ ٱلصُّمُّ ٱلدُّعَآءَ إِذَا مَا يُنذَرُونَ ﴿٤٥﴾\\
\textamh{46.\  } & وَلَئِن مَّسَّتْهُمْ نَفْحَةٌۭ مِّنْ عَذَابِ رَبِّكَ لَيَقُولُنَّ يَـٰوَيْلَنَآ إِنَّا كُنَّا ظَـٰلِمِينَ ﴿٤٦﴾\\
\textamh{47.\  } & وَنَضَعُ ٱلْمَوَٟزِينَ ٱلْقِسْطَ لِيَوْمِ ٱلْقِيَـٰمَةِ فَلَا تُظْلَمُ نَفْسٌۭ شَيْـًۭٔا ۖ وَإِن كَانَ مِثْقَالَ حَبَّةٍۢ مِّنْ خَرْدَلٍ أَتَيْنَا بِهَا ۗ وَكَفَىٰ بِنَا حَـٰسِبِينَ ﴿٤٧﴾\\
\textamh{48.\  } & وَلَقَدْ ءَاتَيْنَا مُوسَىٰ وَهَـٰرُونَ ٱلْفُرْقَانَ وَضِيَآءًۭ وَذِكْرًۭا لِّلْمُتَّقِينَ ﴿٤٨﴾\\
\textamh{49.\  } & ٱلَّذِينَ يَخْشَوْنَ رَبَّهُم بِٱلْغَيْبِ وَهُم مِّنَ ٱلسَّاعَةِ مُشْفِقُونَ ﴿٤٩﴾\\
\textamh{50.\  } & وَهَـٰذَا ذِكْرٌۭ مُّبَارَكٌ أَنزَلْنَـٰهُ ۚ أَفَأَنتُمْ لَهُۥ مُنكِرُونَ ﴿٥٠﴾\\
\textamh{51.\  } & ۞ وَلَقَدْ ءَاتَيْنَآ إِبْرَٰهِيمَ رُشْدَهُۥ مِن قَبْلُ وَكُنَّا بِهِۦ عَـٰلِمِينَ ﴿٥١﴾\\
\textamh{52.\  } & إِذْ قَالَ لِأَبِيهِ وَقَوْمِهِۦ مَا هَـٰذِهِ ٱلتَّمَاثِيلُ ٱلَّتِىٓ أَنتُمْ لَهَا عَـٰكِفُونَ ﴿٥٢﴾\\
\textamh{53.\  } & قَالُوا۟ وَجَدْنَآ ءَابَآءَنَا لَهَا عَـٰبِدِينَ ﴿٥٣﴾\\
\textamh{54.\  } & قَالَ لَقَدْ كُنتُمْ أَنتُمْ وَءَابَآؤُكُمْ فِى ضَلَـٰلٍۢ مُّبِينٍۢ ﴿٥٤﴾\\
\textamh{55.\  } & قَالُوٓا۟ أَجِئْتَنَا بِٱلْحَقِّ أَمْ أَنتَ مِنَ ٱللَّٰعِبِينَ ﴿٥٥﴾\\
\textamh{56.\  } & قَالَ بَل رَّبُّكُمْ رَبُّ ٱلسَّمَـٰوَٟتِ وَٱلْأَرْضِ ٱلَّذِى فَطَرَهُنَّ وَأَنَا۠ عَلَىٰ ذَٟلِكُم مِّنَ ٱلشَّـٰهِدِينَ ﴿٥٦﴾\\
\textamh{57.\  } & وَتَٱللَّهِ لَأَكِيدَنَّ أَصْنَـٰمَكُم بَعْدَ أَن تُوَلُّوا۟ مُدْبِرِينَ ﴿٥٧﴾\\
\textamh{58.\  } & فَجَعَلَهُمْ جُذَٟذًا إِلَّا كَبِيرًۭا لَّهُمْ لَعَلَّهُمْ إِلَيْهِ يَرْجِعُونَ ﴿٥٨﴾\\
\textamh{59.\  } & قَالُوا۟ مَن فَعَلَ هَـٰذَا بِـَٔالِهَتِنَآ إِنَّهُۥ لَمِنَ ٱلظَّـٰلِمِينَ ﴿٥٩﴾\\
\textamh{60.\  } & قَالُوا۟ سَمِعْنَا فَتًۭى يَذْكُرُهُمْ يُقَالُ لَهُۥٓ إِبْرَٰهِيمُ ﴿٦٠﴾\\
\textamh{61.\  } & قَالُوا۟ فَأْتُوا۟ بِهِۦ عَلَىٰٓ أَعْيُنِ ٱلنَّاسِ لَعَلَّهُمْ يَشْهَدُونَ ﴿٦١﴾\\
\textamh{62.\  } & قَالُوٓا۟ ءَأَنتَ فَعَلْتَ هَـٰذَا بِـَٔالِهَتِنَا يَـٰٓإِبْرَٰهِيمُ ﴿٦٢﴾\\
\textamh{63.\  } & قَالَ بَلْ فَعَلَهُۥ كَبِيرُهُمْ هَـٰذَا فَسْـَٔلُوهُمْ إِن كَانُوا۟ يَنطِقُونَ ﴿٦٣﴾\\
\textamh{64.\  } & فَرَجَعُوٓا۟ إِلَىٰٓ أَنفُسِهِمْ فَقَالُوٓا۟ إِنَّكُمْ أَنتُمُ ٱلظَّـٰلِمُونَ ﴿٦٤﴾\\
\textamh{65.\  } & ثُمَّ نُكِسُوا۟ عَلَىٰ رُءُوسِهِمْ لَقَدْ عَلِمْتَ مَا هَـٰٓؤُلَآءِ يَنطِقُونَ ﴿٦٥﴾\\
\textamh{66.\  } & قَالَ أَفَتَعْبُدُونَ مِن دُونِ ٱللَّهِ مَا لَا يَنفَعُكُمْ شَيْـًۭٔا وَلَا يَضُرُّكُمْ ﴿٦٦﴾\\
\textamh{67.\  } & أُفٍّۢ لَّكُمْ وَلِمَا تَعْبُدُونَ مِن دُونِ ٱللَّهِ ۖ أَفَلَا تَعْقِلُونَ ﴿٦٧﴾\\
\textamh{68.\  } & قَالُوا۟ حَرِّقُوهُ وَٱنصُرُوٓا۟ ءَالِهَتَكُمْ إِن كُنتُمْ فَـٰعِلِينَ ﴿٦٨﴾\\
\textamh{69.\  } & قُلْنَا يَـٰنَارُ كُونِى بَرْدًۭا وَسَلَـٰمًا عَلَىٰٓ إِبْرَٰهِيمَ ﴿٦٩﴾\\
\textamh{70.\  } & وَأَرَادُوا۟ بِهِۦ كَيْدًۭا فَجَعَلْنَـٰهُمُ ٱلْأَخْسَرِينَ ﴿٧٠﴾\\
\textamh{71.\  } & وَنَجَّيْنَـٰهُ وَلُوطًا إِلَى ٱلْأَرْضِ ٱلَّتِى بَٰرَكْنَا فِيهَا لِلْعَـٰلَمِينَ ﴿٧١﴾\\
\textamh{72.\  } & وَوَهَبْنَا لَهُۥٓ إِسْحَـٰقَ وَيَعْقُوبَ نَافِلَةًۭ ۖ وَكُلًّۭا جَعَلْنَا صَـٰلِحِينَ ﴿٧٢﴾\\
\textamh{73.\  } & وَجَعَلْنَـٰهُمْ أَئِمَّةًۭ يَهْدُونَ بِأَمْرِنَا وَأَوْحَيْنَآ إِلَيْهِمْ فِعْلَ ٱلْخَيْرَٰتِ وَإِقَامَ ٱلصَّلَوٰةِ وَإِيتَآءَ ٱلزَّكَوٰةِ ۖ وَكَانُوا۟ لَنَا عَـٰبِدِينَ ﴿٧٣﴾\\
\textamh{74.\  } & وَلُوطًا ءَاتَيْنَـٰهُ حُكْمًۭا وَعِلْمًۭا وَنَجَّيْنَـٰهُ مِنَ ٱلْقَرْيَةِ ٱلَّتِى كَانَت تَّعْمَلُ ٱلْخَبَٰٓئِثَ ۗ إِنَّهُمْ كَانُوا۟ قَوْمَ سَوْءٍۢ فَـٰسِقِينَ ﴿٧٤﴾\\
\textamh{75.\  } & وَأَدْخَلْنَـٰهُ فِى رَحْمَتِنَآ ۖ إِنَّهُۥ مِنَ ٱلصَّـٰلِحِينَ ﴿٧٥﴾\\
\textamh{76.\  } & وَنُوحًا إِذْ نَادَىٰ مِن قَبْلُ فَٱسْتَجَبْنَا لَهُۥ فَنَجَّيْنَـٰهُ وَأَهْلَهُۥ مِنَ ٱلْكَرْبِ ٱلْعَظِيمِ ﴿٧٦﴾\\
\textamh{77.\  } & وَنَصَرْنَـٰهُ مِنَ ٱلْقَوْمِ ٱلَّذِينَ كَذَّبُوا۟ بِـَٔايَـٰتِنَآ ۚ إِنَّهُمْ كَانُوا۟ قَوْمَ سَوْءٍۢ فَأَغْرَقْنَـٰهُمْ أَجْمَعِينَ ﴿٧٧﴾\\
\textamh{78.\  } & وَدَاوُۥدَ وَسُلَيْمَـٰنَ إِذْ يَحْكُمَانِ فِى ٱلْحَرْثِ إِذْ نَفَشَتْ فِيهِ غَنَمُ ٱلْقَوْمِ وَكُنَّا لِحُكْمِهِمْ شَـٰهِدِينَ ﴿٧٨﴾\\
\textamh{79.\  } & فَفَهَّمْنَـٰهَا سُلَيْمَـٰنَ ۚ وَكُلًّا ءَاتَيْنَا حُكْمًۭا وَعِلْمًۭا ۚ وَسَخَّرْنَا مَعَ دَاوُۥدَ ٱلْجِبَالَ يُسَبِّحْنَ وَٱلطَّيْرَ ۚ وَكُنَّا فَـٰعِلِينَ ﴿٧٩﴾\\
\textamh{80.\  } & وَعَلَّمْنَـٰهُ صَنْعَةَ لَبُوسٍۢ لَّكُمْ لِتُحْصِنَكُم مِّنۢ بَأْسِكُمْ ۖ فَهَلْ أَنتُمْ شَـٰكِرُونَ ﴿٨٠﴾\\
\textamh{81.\  } & وَلِسُلَيْمَـٰنَ ٱلرِّيحَ عَاصِفَةًۭ تَجْرِى بِأَمْرِهِۦٓ إِلَى ٱلْأَرْضِ ٱلَّتِى بَٰرَكْنَا فِيهَا ۚ وَكُنَّا بِكُلِّ شَىْءٍ عَـٰلِمِينَ ﴿٨١﴾\\
\textamh{82.\  } & وَمِنَ ٱلشَّيَـٰطِينِ مَن يَغُوصُونَ لَهُۥ وَيَعْمَلُونَ عَمَلًۭا دُونَ ذَٟلِكَ ۖ وَكُنَّا لَهُمْ حَـٰفِظِينَ ﴿٨٢﴾\\
\textamh{83.\  } & ۞ وَأَيُّوبَ إِذْ نَادَىٰ رَبَّهُۥٓ أَنِّى مَسَّنِىَ ٱلضُّرُّ وَأَنتَ أَرْحَمُ ٱلرَّٟحِمِينَ ﴿٨٣﴾\\
\textamh{84.\  } & فَٱسْتَجَبْنَا لَهُۥ فَكَشَفْنَا مَا بِهِۦ مِن ضُرٍّۢ ۖ وَءَاتَيْنَـٰهُ أَهْلَهُۥ وَمِثْلَهُم مَّعَهُمْ رَحْمَةًۭ مِّنْ عِندِنَا وَذِكْرَىٰ لِلْعَـٰبِدِينَ ﴿٨٤﴾\\
\textamh{85.\  } & وَإِسْمَـٰعِيلَ وَإِدْرِيسَ وَذَا ٱلْكِفْلِ ۖ كُلٌّۭ مِّنَ ٱلصَّـٰبِرِينَ ﴿٨٥﴾\\
\textamh{86.\  } & وَأَدْخَلْنَـٰهُمْ فِى رَحْمَتِنَآ ۖ إِنَّهُم مِّنَ ٱلصَّـٰلِحِينَ ﴿٨٦﴾\\
\textamh{87.\  } & وَذَا ٱلنُّونِ إِذ ذَّهَبَ مُغَٰضِبًۭا فَظَنَّ أَن لَّن نَّقْدِرَ عَلَيْهِ فَنَادَىٰ فِى ٱلظُّلُمَـٰتِ أَن لَّآ إِلَـٰهَ إِلَّآ أَنتَ سُبْحَـٰنَكَ إِنِّى كُنتُ مِنَ ٱلظَّـٰلِمِينَ ﴿٨٧﴾\\
\textamh{88.\  } & فَٱسْتَجَبْنَا لَهُۥ وَنَجَّيْنَـٰهُ مِنَ ٱلْغَمِّ ۚ وَكَذَٟلِكَ نُۨجِى ٱلْمُؤْمِنِينَ ﴿٨٨﴾\\
\textamh{89.\  } & وَزَكَرِيَّآ إِذْ نَادَىٰ رَبَّهُۥ رَبِّ لَا تَذَرْنِى فَرْدًۭا وَأَنتَ خَيْرُ ٱلْوَٟرِثِينَ ﴿٨٩﴾\\
\textamh{90.\  } & فَٱسْتَجَبْنَا لَهُۥ وَوَهَبْنَا لَهُۥ يَحْيَىٰ وَأَصْلَحْنَا لَهُۥ زَوْجَهُۥٓ ۚ إِنَّهُمْ كَانُوا۟ يُسَـٰرِعُونَ فِى ٱلْخَيْرَٰتِ وَيَدْعُونَنَا رَغَبًۭا وَرَهَبًۭا ۖ وَكَانُوا۟ لَنَا خَـٰشِعِينَ ﴿٩٠﴾\\
\textamh{91.\  } & وَٱلَّتِىٓ أَحْصَنَتْ فَرْجَهَا فَنَفَخْنَا فِيهَا مِن رُّوحِنَا وَجَعَلْنَـٰهَا وَٱبْنَهَآ ءَايَةًۭ لِّلْعَـٰلَمِينَ ﴿٩١﴾\\
\textamh{92.\  } & إِنَّ هَـٰذِهِۦٓ أُمَّتُكُمْ أُمَّةًۭ وَٟحِدَةًۭ وَأَنَا۠ رَبُّكُمْ فَٱعْبُدُونِ ﴿٩٢﴾\\
\textamh{93.\  } & وَتَقَطَّعُوٓا۟ أَمْرَهُم بَيْنَهُمْ ۖ كُلٌّ إِلَيْنَا رَٰجِعُونَ ﴿٩٣﴾\\
\textamh{94.\  } & فَمَن يَعْمَلْ مِنَ ٱلصَّـٰلِحَـٰتِ وَهُوَ مُؤْمِنٌۭ فَلَا كُفْرَانَ لِسَعْيِهِۦ وَإِنَّا لَهُۥ كَـٰتِبُونَ ﴿٩٤﴾\\
\textamh{95.\  } & وَحَرَٰمٌ عَلَىٰ قَرْيَةٍ أَهْلَكْنَـٰهَآ أَنَّهُمْ لَا يَرْجِعُونَ ﴿٩٥﴾\\
\textamh{96.\  } & حَتَّىٰٓ إِذَا فُتِحَتْ يَأْجُوجُ وَمَأْجُوجُ وَهُم مِّن كُلِّ حَدَبٍۢ يَنسِلُونَ ﴿٩٦﴾\\
\textamh{97.\  } & وَٱقْتَرَبَ ٱلْوَعْدُ ٱلْحَقُّ فَإِذَا هِىَ شَـٰخِصَةٌ أَبْصَـٰرُ ٱلَّذِينَ كَفَرُوا۟ يَـٰوَيْلَنَا قَدْ كُنَّا فِى غَفْلَةٍۢ مِّنْ هَـٰذَا بَلْ كُنَّا ظَـٰلِمِينَ ﴿٩٧﴾\\
\textamh{98.\  } & إِنَّكُمْ وَمَا تَعْبُدُونَ مِن دُونِ ٱللَّهِ حَصَبُ جَهَنَّمَ أَنتُمْ لَهَا وَٟرِدُونَ ﴿٩٨﴾\\
\textamh{99.\  } & لَوْ كَانَ هَـٰٓؤُلَآءِ ءَالِهَةًۭ مَّا وَرَدُوهَا ۖ وَكُلٌّۭ فِيهَا خَـٰلِدُونَ ﴿٩٩﴾\\
\textamh{100.\  } & لَهُمْ فِيهَا زَفِيرٌۭ وَهُمْ فِيهَا لَا يَسْمَعُونَ ﴿١٠٠﴾\\
\textamh{101.\  } & إِنَّ ٱلَّذِينَ سَبَقَتْ لَهُم مِّنَّا ٱلْحُسْنَىٰٓ أُو۟لَـٰٓئِكَ عَنْهَا مُبْعَدُونَ ﴿١٠١﴾\\
\textamh{102.\  } & لَا يَسْمَعُونَ حَسِيسَهَا ۖ وَهُمْ فِى مَا ٱشْتَهَتْ أَنفُسُهُمْ خَـٰلِدُونَ ﴿١٠٢﴾\\
\textamh{103.\  } & لَا يَحْزُنُهُمُ ٱلْفَزَعُ ٱلْأَكْبَرُ وَتَتَلَقَّىٰهُمُ ٱلْمَلَـٰٓئِكَةُ هَـٰذَا يَوْمُكُمُ ٱلَّذِى كُنتُمْ تُوعَدُونَ ﴿١٠٣﴾\\
\textamh{104.\  } & يَوْمَ نَطْوِى ٱلسَّمَآءَ كَطَىِّ ٱلسِّجِلِّ لِلْكُتُبِ ۚ كَمَا بَدَأْنَآ أَوَّلَ خَلْقٍۢ نُّعِيدُهُۥ ۚ وَعْدًا عَلَيْنَآ ۚ إِنَّا كُنَّا فَـٰعِلِينَ ﴿١٠٤﴾\\
\textamh{105.\  } & وَلَقَدْ كَتَبْنَا فِى ٱلزَّبُورِ مِنۢ بَعْدِ ٱلذِّكْرِ أَنَّ ٱلْأَرْضَ يَرِثُهَا عِبَادِىَ ٱلصَّـٰلِحُونَ ﴿١٠٥﴾\\
\textamh{106.\  } & إِنَّ فِى هَـٰذَا لَبَلَـٰغًۭا لِّقَوْمٍ عَـٰبِدِينَ ﴿١٠٦﴾\\
\textamh{107.\  } & وَمَآ أَرْسَلْنَـٰكَ إِلَّا رَحْمَةًۭ لِّلْعَـٰلَمِينَ ﴿١٠٧﴾\\
\textamh{108.\  } & قُلْ إِنَّمَا يُوحَىٰٓ إِلَىَّ أَنَّمَآ إِلَـٰهُكُمْ إِلَـٰهٌۭ وَٟحِدٌۭ ۖ فَهَلْ أَنتُم مُّسْلِمُونَ ﴿١٠٨﴾\\
\textamh{109.\  } & فَإِن تَوَلَّوْا۟ فَقُلْ ءَاذَنتُكُمْ عَلَىٰ سَوَآءٍۢ ۖ وَإِنْ أَدْرِىٓ أَقَرِيبٌ أَم بَعِيدٌۭ مَّا تُوعَدُونَ ﴿١٠٩﴾\\
\textamh{110.\  } & إِنَّهُۥ يَعْلَمُ ٱلْجَهْرَ مِنَ ٱلْقَوْلِ وَيَعْلَمُ مَا تَكْتُمُونَ ﴿١١٠﴾\\
\textamh{111.\  } & وَإِنْ أَدْرِى لَعَلَّهُۥ فِتْنَةٌۭ لَّكُمْ وَمَتَـٰعٌ إِلَىٰ حِينٍۢ ﴿١١١﴾\\
\textamh{112.\  } & قَـٰلَ رَبِّ ٱحْكُم بِٱلْحَقِّ ۗ وَرَبُّنَا ٱلرَّحْمَـٰنُ ٱلْمُسْتَعَانُ عَلَىٰ مَا تَصِفُونَ ﴿١١٢﴾\\
\end{longtable}
\clearpage
%% License: BSD style (Berkley) (i.e. Put the Copyright owner's name always)
%% Writer and Copyright (to): Bewketu(Bilal) Tadilo (2016-17)
\begin{center}\section{\LR{\textamhsec{ሱራቱ አልሀጅ -}  \textarabic{سوره  الحج}}}\end{center}
\begin{longtable}{%
  @{}
    p{.5\textwidth}
  @{~~~}
    p{.5\textwidth}
    @{}
}
\textamh{ቢስሚላሂ አራህመኒ ራሂይም } &  \mytextarabic{بِسْمِ ٱللَّهِ ٱلرَّحْمَـٰنِ ٱلرَّحِيمِ}\\
\textamh{1.\  } & \mytextarabic{ يَـٰٓأَيُّهَا ٱلنَّاسُ ٱتَّقُوا۟ رَبَّكُمْ ۚ إِنَّ زَلْزَلَةَ ٱلسَّاعَةِ شَىْءٌ عَظِيمٌۭ ﴿١﴾}\\
\textamh{2.\  } & \mytextarabic{يَوْمَ تَرَوْنَهَا تَذْهَلُ كُلُّ مُرْضِعَةٍ عَمَّآ أَرْضَعَتْ وَتَضَعُ كُلُّ ذَاتِ حَمْلٍ حَمْلَهَا وَتَرَى ٱلنَّاسَ سُكَـٰرَىٰ وَمَا هُم بِسُكَـٰرَىٰ وَلَـٰكِنَّ عَذَابَ ٱللَّهِ شَدِيدٌۭ ﴿٢﴾}\\
\textamh{3.\  } & \mytextarabic{وَمِنَ ٱلنَّاسِ مَن يُجَٰدِلُ فِى ٱللَّهِ بِغَيْرِ عِلْمٍۢ وَيَتَّبِعُ كُلَّ شَيْطَٰنٍۢ مَّرِيدٍۢ ﴿٣﴾}\\
\textamh{4.\  } & \mytextarabic{كُتِبَ عَلَيْهِ أَنَّهُۥ مَن تَوَلَّاهُ فَأَنَّهُۥ يُضِلُّهُۥ وَيَهْدِيهِ إِلَىٰ عَذَابِ ٱلسَّعِيرِ ﴿٤﴾}\\
\textamh{5.\  } & \mytextarabic{يَـٰٓأَيُّهَا ٱلنَّاسُ إِن كُنتُمْ فِى رَيْبٍۢ مِّنَ ٱلْبَعْثِ فَإِنَّا خَلَقْنَـٰكُم مِّن تُرَابٍۢ ثُمَّ مِن نُّطْفَةٍۢ ثُمَّ مِنْ عَلَقَةٍۢ ثُمَّ مِن مُّضْغَةٍۢ مُّخَلَّقَةٍۢ وَغَيْرِ مُخَلَّقَةٍۢ لِّنُبَيِّنَ لَكُمْ ۚ وَنُقِرُّ فِى ٱلْأَرْحَامِ مَا نَشَآءُ إِلَىٰٓ أَجَلٍۢ مُّسَمًّۭى ثُمَّ نُخْرِجُكُمْ طِفْلًۭا ثُمَّ لِتَبْلُغُوٓا۟ أَشُدَّكُمْ ۖ وَمِنكُم مَّن يُتَوَفَّىٰ وَمِنكُم مَّن يُرَدُّ إِلَىٰٓ أَرْذَلِ ٱلْعُمُرِ لِكَيْلَا يَعْلَمَ مِنۢ بَعْدِ عِلْمٍۢ شَيْـًۭٔا ۚ وَتَرَى ٱلْأَرْضَ هَامِدَةًۭ فَإِذَآ أَنزَلْنَا عَلَيْهَا ٱلْمَآءَ ٱهْتَزَّتْ وَرَبَتْ وَأَنۢبَتَتْ مِن كُلِّ زَوْجٍۭ بَهِيجٍۢ ﴿٥﴾}\\
\textamh{6.\  } & \mytextarabic{ذَٟلِكَ بِأَنَّ ٱللَّهَ هُوَ ٱلْحَقُّ وَأَنَّهُۥ يُحْىِ ٱلْمَوْتَىٰ وَأَنَّهُۥ عَلَىٰ كُلِّ شَىْءٍۢ قَدِيرٌۭ ﴿٦﴾}\\
\textamh{7.\  } & \mytextarabic{وَأَنَّ ٱلسَّاعَةَ ءَاتِيَةٌۭ لَّا رَيْبَ فِيهَا وَأَنَّ ٱللَّهَ يَبْعَثُ مَن فِى ٱلْقُبُورِ ﴿٧﴾}\\
\textamh{8.\  } & \mytextarabic{وَمِنَ ٱلنَّاسِ مَن يُجَٰدِلُ فِى ٱللَّهِ بِغَيْرِ عِلْمٍۢ وَلَا هُدًۭى وَلَا كِتَـٰبٍۢ مُّنِيرٍۢ ﴿٨﴾}\\
\textamh{9.\  } & \mytextarabic{ثَانِىَ عِطْفِهِۦ لِيُضِلَّ عَن سَبِيلِ ٱللَّهِ ۖ لَهُۥ فِى ٱلدُّنْيَا خِزْىٌۭ ۖ وَنُذِيقُهُۥ يَوْمَ ٱلْقِيَـٰمَةِ عَذَابَ ٱلْحَرِيقِ ﴿٩﴾}\\
\textamh{10.\  } & \mytextarabic{ذَٟلِكَ بِمَا قَدَّمَتْ يَدَاكَ وَأَنَّ ٱللَّهَ لَيْسَ بِظَلَّٰمٍۢ لِّلْعَبِيدِ ﴿١٠﴾}\\
\textamh{11.\  } & \mytextarabic{وَمِنَ ٱلنَّاسِ مَن يَعْبُدُ ٱللَّهَ عَلَىٰ حَرْفٍۢ ۖ فَإِنْ أَصَابَهُۥ خَيْرٌ ٱطْمَأَنَّ بِهِۦ ۖ وَإِنْ أَصَابَتْهُ فِتْنَةٌ ٱنقَلَبَ عَلَىٰ وَجْهِهِۦ خَسِرَ ٱلدُّنْيَا وَٱلْءَاخِرَةَ ۚ ذَٟلِكَ هُوَ ٱلْخُسْرَانُ ٱلْمُبِينُ ﴿١١﴾}\\
\textamh{12.\  } & \mytextarabic{يَدْعُوا۟ مِن دُونِ ٱللَّهِ مَا لَا يَضُرُّهُۥ وَمَا لَا يَنفَعُهُۥ ۚ ذَٟلِكَ هُوَ ٱلضَّلَـٰلُ ٱلْبَعِيدُ ﴿١٢﴾}\\
\textamh{13.\  } & \mytextarabic{يَدْعُوا۟ لَمَن ضَرُّهُۥٓ أَقْرَبُ مِن نَّفْعِهِۦ ۚ لَبِئْسَ ٱلْمَوْلَىٰ وَلَبِئْسَ ٱلْعَشِيرُ ﴿١٣﴾}\\
\textamh{14.\  } & \mytextarabic{إِنَّ ٱللَّهَ يُدْخِلُ ٱلَّذِينَ ءَامَنُوا۟ وَعَمِلُوا۟ ٱلصَّـٰلِحَـٰتِ جَنَّـٰتٍۢ تَجْرِى مِن تَحْتِهَا ٱلْأَنْهَـٰرُ ۚ إِنَّ ٱللَّهَ يَفْعَلُ مَا يُرِيدُ ﴿١٤﴾}\\
\textamh{15.\  } & \mytextarabic{مَن كَانَ يَظُنُّ أَن لَّن يَنصُرَهُ ٱللَّهُ فِى ٱلدُّنْيَا وَٱلْءَاخِرَةِ فَلْيَمْدُدْ بِسَبَبٍ إِلَى ٱلسَّمَآءِ ثُمَّ لْيَقْطَعْ فَلْيَنظُرْ هَلْ يُذْهِبَنَّ كَيْدُهُۥ مَا يَغِيظُ ﴿١٥﴾}\\
\textamh{16.\  } & \mytextarabic{وَكَذَٟلِكَ أَنزَلْنَـٰهُ ءَايَـٰتٍۭ بَيِّنَـٰتٍۢ وَأَنَّ ٱللَّهَ يَهْدِى مَن يُرِيدُ ﴿١٦﴾}\\
\textamh{17.\  } & \mytextarabic{إِنَّ ٱلَّذِينَ ءَامَنُوا۟ وَٱلَّذِينَ هَادُوا۟ وَٱلصَّـٰبِـِٔينَ وَٱلنَّصَـٰرَىٰ وَٱلْمَجُوسَ وَٱلَّذِينَ أَشْرَكُوٓا۟ إِنَّ ٱللَّهَ يَفْصِلُ بَيْنَهُمْ يَوْمَ ٱلْقِيَـٰمَةِ ۚ إِنَّ ٱللَّهَ عَلَىٰ كُلِّ شَىْءٍۢ شَهِيدٌ ﴿١٧﴾}\\
\textamh{18.\  } & \mytextarabic{أَلَمْ تَرَ أَنَّ ٱللَّهَ يَسْجُدُ لَهُۥ مَن فِى ٱلسَّمَـٰوَٟتِ وَمَن فِى ٱلْأَرْضِ وَٱلشَّمْسُ وَٱلْقَمَرُ وَٱلنُّجُومُ وَٱلْجِبَالُ وَٱلشَّجَرُ وَٱلدَّوَآبُّ وَكَثِيرٌۭ مِّنَ ٱلنَّاسِ ۖ وَكَثِيرٌ حَقَّ عَلَيْهِ ٱلْعَذَابُ ۗ وَمَن يُهِنِ ٱللَّهُ فَمَا لَهُۥ مِن مُّكْرِمٍ ۚ إِنَّ ٱللَّهَ يَفْعَلُ مَا يَشَآءُ ۩ ﴿١٨﴾}\\
\textamh{19.\  } & \mytextarabic{۞ هَـٰذَانِ خَصْمَانِ ٱخْتَصَمُوا۟ فِى رَبِّهِمْ ۖ فَٱلَّذِينَ كَفَرُوا۟ قُطِّعَتْ لَهُمْ ثِيَابٌۭ مِّن نَّارٍۢ يُصَبُّ مِن فَوْقِ رُءُوسِهِمُ ٱلْحَمِيمُ ﴿١٩﴾}\\
\textamh{20.\  } & \mytextarabic{يُصْهَرُ بِهِۦ مَا فِى بُطُونِهِمْ وَٱلْجُلُودُ ﴿٢٠﴾}\\
\textamh{21.\  } & \mytextarabic{وَلَهُم مَّقَـٰمِعُ مِنْ حَدِيدٍۢ ﴿٢١﴾}\\
\textamh{22.\  } & \mytextarabic{كُلَّمَآ أَرَادُوٓا۟ أَن يَخْرُجُوا۟ مِنْهَا مِنْ غَمٍّ أُعِيدُوا۟ فِيهَا وَذُوقُوا۟ عَذَابَ ٱلْحَرِيقِ ﴿٢٢﴾}\\
\textamh{23.\  } & \mytextarabic{إِنَّ ٱللَّهَ يُدْخِلُ ٱلَّذِينَ ءَامَنُوا۟ وَعَمِلُوا۟ ٱلصَّـٰلِحَـٰتِ جَنَّـٰتٍۢ تَجْرِى مِن تَحْتِهَا ٱلْأَنْهَـٰرُ يُحَلَّوْنَ فِيهَا مِنْ أَسَاوِرَ مِن ذَهَبٍۢ وَلُؤْلُؤًۭا ۖ وَلِبَاسُهُمْ فِيهَا حَرِيرٌۭ ﴿٢٣﴾}\\
\textamh{24.\  } & \mytextarabic{وَهُدُوٓا۟ إِلَى ٱلطَّيِّبِ مِنَ ٱلْقَوْلِ وَهُدُوٓا۟ إِلَىٰ صِرَٰطِ ٱلْحَمِيدِ ﴿٢٤﴾}\\
\textamh{25.\  } & \mytextarabic{إِنَّ ٱلَّذِينَ كَفَرُوا۟ وَيَصُدُّونَ عَن سَبِيلِ ٱللَّهِ وَٱلْمَسْجِدِ ٱلْحَرَامِ ٱلَّذِى جَعَلْنَـٰهُ لِلنَّاسِ سَوَآءً ٱلْعَـٰكِفُ فِيهِ وَٱلْبَادِ ۚ وَمَن يُرِدْ فِيهِ بِإِلْحَادٍۭ بِظُلْمٍۢ نُّذِقْهُ مِنْ عَذَابٍ أَلِيمٍۢ ﴿٢٥﴾}\\
\textamh{26.\  } & \mytextarabic{وَإِذْ بَوَّأْنَا لِإِبْرَٰهِيمَ مَكَانَ ٱلْبَيْتِ أَن لَّا تُشْرِكْ بِى شَيْـًۭٔا وَطَهِّرْ بَيْتِىَ لِلطَّآئِفِينَ وَٱلْقَآئِمِينَ وَٱلرُّكَّعِ ٱلسُّجُودِ ﴿٢٦﴾}\\
\textamh{27.\  } & \mytextarabic{وَأَذِّن فِى ٱلنَّاسِ بِٱلْحَجِّ يَأْتُوكَ رِجَالًۭا وَعَلَىٰ كُلِّ ضَامِرٍۢ يَأْتِينَ مِن كُلِّ فَجٍّ عَمِيقٍۢ ﴿٢٧﴾}\\
\textamh{28.\  } & \mytextarabic{لِّيَشْهَدُوا۟ مَنَـٰفِعَ لَهُمْ وَيَذْكُرُوا۟ ٱسْمَ ٱللَّهِ فِىٓ أَيَّامٍۢ مَّعْلُومَـٰتٍ عَلَىٰ مَا رَزَقَهُم مِّنۢ بَهِيمَةِ ٱلْأَنْعَـٰمِ ۖ فَكُلُوا۟ مِنْهَا وَأَطْعِمُوا۟ ٱلْبَآئِسَ ٱلْفَقِيرَ ﴿٢٨﴾}\\
\textamh{29.\  } & \mytextarabic{ثُمَّ لْيَقْضُوا۟ تَفَثَهُمْ وَلْيُوفُوا۟ نُذُورَهُمْ وَلْيَطَّوَّفُوا۟ بِٱلْبَيْتِ ٱلْعَتِيقِ ﴿٢٩﴾}\\
\textamh{30.\  } & \mytextarabic{ذَٟلِكَ وَمَن يُعَظِّمْ حُرُمَـٰتِ ٱللَّهِ فَهُوَ خَيْرٌۭ لَّهُۥ عِندَ رَبِّهِۦ ۗ وَأُحِلَّتْ لَكُمُ ٱلْأَنْعَـٰمُ إِلَّا مَا يُتْلَىٰ عَلَيْكُمْ ۖ فَٱجْتَنِبُوا۟ ٱلرِّجْسَ مِنَ ٱلْأَوْثَـٰنِ وَٱجْتَنِبُوا۟ قَوْلَ ٱلزُّورِ ﴿٣٠﴾}\\
\textamh{31.\  } & \mytextarabic{حُنَفَآءَ لِلَّهِ غَيْرَ مُشْرِكِينَ بِهِۦ ۚ وَمَن يُشْرِكْ بِٱللَّهِ فَكَأَنَّمَا خَرَّ مِنَ ٱلسَّمَآءِ فَتَخْطَفُهُ ٱلطَّيْرُ أَوْ تَهْوِى بِهِ ٱلرِّيحُ فِى مَكَانٍۢ سَحِيقٍۢ ﴿٣١﴾}\\
\textamh{32.\  } & \mytextarabic{ذَٟلِكَ وَمَن يُعَظِّمْ شَعَـٰٓئِرَ ٱللَّهِ فَإِنَّهَا مِن تَقْوَى ٱلْقُلُوبِ ﴿٣٢﴾}\\
\textamh{33.\  } & \mytextarabic{لَكُمْ فِيهَا مَنَـٰفِعُ إِلَىٰٓ أَجَلٍۢ مُّسَمًّۭى ثُمَّ مَحِلُّهَآ إِلَى ٱلْبَيْتِ ٱلْعَتِيقِ ﴿٣٣﴾}\\
\textamh{34.\  } & \mytextarabic{وَلِكُلِّ أُمَّةٍۢ جَعَلْنَا مَنسَكًۭا لِّيَذْكُرُوا۟ ٱسْمَ ٱللَّهِ عَلَىٰ مَا رَزَقَهُم مِّنۢ بَهِيمَةِ ٱلْأَنْعَـٰمِ ۗ فَإِلَـٰهُكُمْ إِلَـٰهٌۭ وَٟحِدٌۭ فَلَهُۥٓ أَسْلِمُوا۟ ۗ وَبَشِّرِ ٱلْمُخْبِتِينَ ﴿٣٤﴾}\\
\textamh{35.\  } & \mytextarabic{ٱلَّذِينَ إِذَا ذُكِرَ ٱللَّهُ وَجِلَتْ قُلُوبُهُمْ وَٱلصَّـٰبِرِينَ عَلَىٰ مَآ أَصَابَهُمْ وَٱلْمُقِيمِى ٱلصَّلَوٰةِ وَمِمَّا رَزَقْنَـٰهُمْ يُنفِقُونَ ﴿٣٥﴾}\\
\textamh{36.\  } & \mytextarabic{وَٱلْبُدْنَ جَعَلْنَـٰهَا لَكُم مِّن شَعَـٰٓئِرِ ٱللَّهِ لَكُمْ فِيهَا خَيْرٌۭ ۖ فَٱذْكُرُوا۟ ٱسْمَ ٱللَّهِ عَلَيْهَا صَوَآفَّ ۖ فَإِذَا وَجَبَتْ جُنُوبُهَا فَكُلُوا۟ مِنْهَا وَأَطْعِمُوا۟ ٱلْقَانِعَ وَٱلْمُعْتَرَّ ۚ كَذَٟلِكَ سَخَّرْنَـٰهَا لَكُمْ لَعَلَّكُمْ تَشْكُرُونَ ﴿٣٦﴾}\\
\textamh{37.\  } & \mytextarabic{لَن يَنَالَ ٱللَّهَ لُحُومُهَا وَلَا دِمَآؤُهَا وَلَـٰكِن يَنَالُهُ ٱلتَّقْوَىٰ مِنكُمْ ۚ كَذَٟلِكَ سَخَّرَهَا لَكُمْ لِتُكَبِّرُوا۟ ٱللَّهَ عَلَىٰ مَا هَدَىٰكُمْ ۗ وَبَشِّرِ ٱلْمُحْسِنِينَ ﴿٣٧﴾}\\
\textamh{38.\  } & \mytextarabic{۞ إِنَّ ٱللَّهَ يُدَٟفِعُ عَنِ ٱلَّذِينَ ءَامَنُوٓا۟ ۗ إِنَّ ٱللَّهَ لَا يُحِبُّ كُلَّ خَوَّانٍۢ كَفُورٍ ﴿٣٨﴾}\\
\textamh{39.\  } & \mytextarabic{أُذِنَ لِلَّذِينَ يُقَـٰتَلُونَ بِأَنَّهُمْ ظُلِمُوا۟ ۚ وَإِنَّ ٱللَّهَ عَلَىٰ نَصْرِهِمْ لَقَدِيرٌ ﴿٣٩﴾}\\
\textamh{40.\  } & \mytextarabic{ٱلَّذِينَ أُخْرِجُوا۟ مِن دِيَـٰرِهِم بِغَيْرِ حَقٍّ إِلَّآ أَن يَقُولُوا۟ رَبُّنَا ٱللَّهُ ۗ وَلَوْلَا دَفْعُ ٱللَّهِ ٱلنَّاسَ بَعْضَهُم بِبَعْضٍۢ لَّهُدِّمَتْ صَوَٟمِعُ وَبِيَعٌۭ وَصَلَوَٟتٌۭ وَمَسَـٰجِدُ يُذْكَرُ فِيهَا ٱسْمُ ٱللَّهِ كَثِيرًۭا ۗ وَلَيَنصُرَنَّ ٱللَّهُ مَن يَنصُرُهُۥٓ ۗ إِنَّ ٱللَّهَ لَقَوِىٌّ عَزِيزٌ ﴿٤٠﴾}\\
\textamh{41.\  } & \mytextarabic{ٱلَّذِينَ إِن مَّكَّنَّـٰهُمْ فِى ٱلْأَرْضِ أَقَامُوا۟ ٱلصَّلَوٰةَ وَءَاتَوُا۟ ٱلزَّكَوٰةَ وَأَمَرُوا۟ بِٱلْمَعْرُوفِ وَنَهَوْا۟ عَنِ ٱلْمُنكَرِ ۗ وَلِلَّهِ عَـٰقِبَةُ ٱلْأُمُورِ ﴿٤١﴾}\\
\textamh{42.\  } & \mytextarabic{وَإِن يُكَذِّبُوكَ فَقَدْ كَذَّبَتْ قَبْلَهُمْ قَوْمُ نُوحٍۢ وَعَادٌۭ وَثَمُودُ ﴿٤٢﴾}\\
\textamh{43.\  } & \mytextarabic{وَقَوْمُ إِبْرَٰهِيمَ وَقَوْمُ لُوطٍۢ ﴿٤٣﴾}\\
\textamh{44.\  } & \mytextarabic{وَأَصْحَـٰبُ مَدْيَنَ ۖ وَكُذِّبَ مُوسَىٰ فَأَمْلَيْتُ لِلْكَـٰفِرِينَ ثُمَّ أَخَذْتُهُمْ ۖ فَكَيْفَ كَانَ نَكِيرِ ﴿٤٤﴾}\\
\textamh{45.\  } & \mytextarabic{فَكَأَيِّن مِّن قَرْيَةٍ أَهْلَكْنَـٰهَا وَهِىَ ظَالِمَةٌۭ فَهِىَ خَاوِيَةٌ عَلَىٰ عُرُوشِهَا وَبِئْرٍۢ مُّعَطَّلَةٍۢ وَقَصْرٍۢ مَّشِيدٍ ﴿٤٥﴾}\\
\textamh{46.\  } & \mytextarabic{أَفَلَمْ يَسِيرُوا۟ فِى ٱلْأَرْضِ فَتَكُونَ لَهُمْ قُلُوبٌۭ يَعْقِلُونَ بِهَآ أَوْ ءَاذَانٌۭ يَسْمَعُونَ بِهَا ۖ فَإِنَّهَا لَا تَعْمَى ٱلْأَبْصَـٰرُ وَلَـٰكِن تَعْمَى ٱلْقُلُوبُ ٱلَّتِى فِى ٱلصُّدُورِ ﴿٤٦﴾}\\
\textamh{47.\  } & \mytextarabic{وَيَسْتَعْجِلُونَكَ بِٱلْعَذَابِ وَلَن يُخْلِفَ ٱللَّهُ وَعْدَهُۥ ۚ وَإِنَّ يَوْمًا عِندَ رَبِّكَ كَأَلْفِ سَنَةٍۢ مِّمَّا تَعُدُّونَ ﴿٤٧﴾}\\
\textamh{48.\  } & \mytextarabic{وَكَأَيِّن مِّن قَرْيَةٍ أَمْلَيْتُ لَهَا وَهِىَ ظَالِمَةٌۭ ثُمَّ أَخَذْتُهَا وَإِلَىَّ ٱلْمَصِيرُ ﴿٤٨﴾}\\
\textamh{49.\  } & \mytextarabic{قُلْ يَـٰٓأَيُّهَا ٱلنَّاسُ إِنَّمَآ أَنَا۠ لَكُمْ نَذِيرٌۭ مُّبِينٌۭ ﴿٤٩﴾}\\
\textamh{50.\  } & \mytextarabic{فَٱلَّذِينَ ءَامَنُوا۟ وَعَمِلُوا۟ ٱلصَّـٰلِحَـٰتِ لَهُم مَّغْفِرَةٌۭ وَرِزْقٌۭ كَرِيمٌۭ ﴿٥٠﴾}\\
\textamh{51.\  } & \mytextarabic{وَٱلَّذِينَ سَعَوْا۟ فِىٓ ءَايَـٰتِنَا مُعَـٰجِزِينَ أُو۟لَـٰٓئِكَ أَصْحَـٰبُ ٱلْجَحِيمِ ﴿٥١﴾}\\
\textamh{52.\  } & \mytextarabic{وَمَآ أَرْسَلْنَا مِن قَبْلِكَ مِن رَّسُولٍۢ وَلَا نَبِىٍّ إِلَّآ إِذَا تَمَنَّىٰٓ أَلْقَى ٱلشَّيْطَٰنُ فِىٓ أُمْنِيَّتِهِۦ فَيَنسَخُ ٱللَّهُ مَا يُلْقِى ٱلشَّيْطَٰنُ ثُمَّ يُحْكِمُ ٱللَّهُ ءَايَـٰتِهِۦ ۗ وَٱللَّهُ عَلِيمٌ حَكِيمٌۭ ﴿٥٢﴾}\\
\textamh{53.\  } & \mytextarabic{لِّيَجْعَلَ مَا يُلْقِى ٱلشَّيْطَٰنُ فِتْنَةًۭ لِّلَّذِينَ فِى قُلُوبِهِم مَّرَضٌۭ وَٱلْقَاسِيَةِ قُلُوبُهُمْ ۗ وَإِنَّ ٱلظَّـٰلِمِينَ لَفِى شِقَاقٍۭ بَعِيدٍۢ ﴿٥٣﴾}\\
\textamh{54.\  } & \mytextarabic{وَلِيَعْلَمَ ٱلَّذِينَ أُوتُوا۟ ٱلْعِلْمَ أَنَّهُ ٱلْحَقُّ مِن رَّبِّكَ فَيُؤْمِنُوا۟ بِهِۦ فَتُخْبِتَ لَهُۥ قُلُوبُهُمْ ۗ وَإِنَّ ٱللَّهَ لَهَادِ ٱلَّذِينَ ءَامَنُوٓا۟ إِلَىٰ صِرَٰطٍۢ مُّسْتَقِيمٍۢ ﴿٥٤﴾}\\
\textamh{55.\  } & \mytextarabic{وَلَا يَزَالُ ٱلَّذِينَ كَفَرُوا۟ فِى مِرْيَةٍۢ مِّنْهُ حَتَّىٰ تَأْتِيَهُمُ ٱلسَّاعَةُ بَغْتَةً أَوْ يَأْتِيَهُمْ عَذَابُ يَوْمٍ عَقِيمٍ ﴿٥٥﴾}\\
\textamh{56.\  } & \mytextarabic{ٱلْمُلْكُ يَوْمَئِذٍۢ لِّلَّهِ يَحْكُمُ بَيْنَهُمْ ۚ فَٱلَّذِينَ ءَامَنُوا۟ وَعَمِلُوا۟ ٱلصَّـٰلِحَـٰتِ فِى جَنَّـٰتِ ٱلنَّعِيمِ ﴿٥٦﴾}\\
\textamh{57.\  } & \mytextarabic{وَٱلَّذِينَ كَفَرُوا۟ وَكَذَّبُوا۟ بِـَٔايَـٰتِنَا فَأُو۟لَـٰٓئِكَ لَهُمْ عَذَابٌۭ مُّهِينٌۭ ﴿٥٧﴾}\\
\textamh{58.\  } & \mytextarabic{وَٱلَّذِينَ هَاجَرُوا۟ فِى سَبِيلِ ٱللَّهِ ثُمَّ قُتِلُوٓا۟ أَوْ مَاتُوا۟ لَيَرْزُقَنَّهُمُ ٱللَّهُ رِزْقًا حَسَنًۭا ۚ وَإِنَّ ٱللَّهَ لَهُوَ خَيْرُ ٱلرَّٟزِقِينَ ﴿٥٨﴾}\\
\textamh{59.\  } & \mytextarabic{لَيُدْخِلَنَّهُم مُّدْخَلًۭا يَرْضَوْنَهُۥ ۗ وَإِنَّ ٱللَّهَ لَعَلِيمٌ حَلِيمٌۭ ﴿٥٩﴾}\\
\textamh{60.\  } & \mytextarabic{۞ ذَٟلِكَ وَمَنْ عَاقَبَ بِمِثْلِ مَا عُوقِبَ بِهِۦ ثُمَّ بُغِىَ عَلَيْهِ لَيَنصُرَنَّهُ ٱللَّهُ ۗ إِنَّ ٱللَّهَ لَعَفُوٌّ غَفُورٌۭ ﴿٦٠﴾}\\
\textamh{61.\  } & \mytextarabic{ذَٟلِكَ بِأَنَّ ٱللَّهَ يُولِجُ ٱلَّيْلَ فِى ٱلنَّهَارِ وَيُولِجُ ٱلنَّهَارَ فِى ٱلَّيْلِ وَأَنَّ ٱللَّهَ سَمِيعٌۢ بَصِيرٌۭ ﴿٦١﴾}\\
\textamh{62.\  } & \mytextarabic{ذَٟلِكَ بِأَنَّ ٱللَّهَ هُوَ ٱلْحَقُّ وَأَنَّ مَا يَدْعُونَ مِن دُونِهِۦ هُوَ ٱلْبَٰطِلُ وَأَنَّ ٱللَّهَ هُوَ ٱلْعَلِىُّ ٱلْكَبِيرُ ﴿٦٢﴾}\\
\textamh{63.\  } & \mytextarabic{أَلَمْ تَرَ أَنَّ ٱللَّهَ أَنزَلَ مِنَ ٱلسَّمَآءِ مَآءًۭ فَتُصْبِحُ ٱلْأَرْضُ مُخْضَرَّةً ۗ إِنَّ ٱللَّهَ لَطِيفٌ خَبِيرٌۭ ﴿٦٣﴾}\\
\textamh{64.\  } & \mytextarabic{لَّهُۥ مَا فِى ٱلسَّمَـٰوَٟتِ وَمَا فِى ٱلْأَرْضِ ۗ وَإِنَّ ٱللَّهَ لَهُوَ ٱلْغَنِىُّ ٱلْحَمِيدُ ﴿٦٤﴾}\\
\textamh{65.\  } & \mytextarabic{أَلَمْ تَرَ أَنَّ ٱللَّهَ سَخَّرَ لَكُم مَّا فِى ٱلْأَرْضِ وَٱلْفُلْكَ تَجْرِى فِى ٱلْبَحْرِ بِأَمْرِهِۦ وَيُمْسِكُ ٱلسَّمَآءَ أَن تَقَعَ عَلَى ٱلْأَرْضِ إِلَّا بِإِذْنِهِۦٓ ۗ إِنَّ ٱللَّهَ بِٱلنَّاسِ لَرَءُوفٌۭ رَّحِيمٌۭ ﴿٦٥﴾}\\
\textamh{66.\  } & \mytextarabic{وَهُوَ ٱلَّذِىٓ أَحْيَاكُمْ ثُمَّ يُمِيتُكُمْ ثُمَّ يُحْيِيكُمْ ۗ إِنَّ ٱلْإِنسَـٰنَ لَكَفُورٌۭ ﴿٦٦﴾}\\
\textamh{67.\  } & \mytextarabic{لِّكُلِّ أُمَّةٍۢ جَعَلْنَا مَنسَكًا هُمْ نَاسِكُوهُ ۖ فَلَا يُنَـٰزِعُنَّكَ فِى ٱلْأَمْرِ ۚ وَٱدْعُ إِلَىٰ رَبِّكَ ۖ إِنَّكَ لَعَلَىٰ هُدًۭى مُّسْتَقِيمٍۢ ﴿٦٧﴾}\\
\textamh{68.\  } & \mytextarabic{وَإِن جَٰدَلُوكَ فَقُلِ ٱللَّهُ أَعْلَمُ بِمَا تَعْمَلُونَ ﴿٦٨﴾}\\
\textamh{69.\  } & \mytextarabic{ٱللَّهُ يَحْكُمُ بَيْنَكُمْ يَوْمَ ٱلْقِيَـٰمَةِ فِيمَا كُنتُمْ فِيهِ تَخْتَلِفُونَ ﴿٦٩﴾}\\
\textamh{70.\  } & \mytextarabic{أَلَمْ تَعْلَمْ أَنَّ ٱللَّهَ يَعْلَمُ مَا فِى ٱلسَّمَآءِ وَٱلْأَرْضِ ۗ إِنَّ ذَٟلِكَ فِى كِتَـٰبٍ ۚ إِنَّ ذَٟلِكَ عَلَى ٱللَّهِ يَسِيرٌۭ ﴿٧٠﴾}\\
\textamh{71.\  } & \mytextarabic{وَيَعْبُدُونَ مِن دُونِ ٱللَّهِ مَا لَمْ يُنَزِّلْ بِهِۦ سُلْطَٰنًۭا وَمَا لَيْسَ لَهُم بِهِۦ عِلْمٌۭ ۗ وَمَا لِلظَّـٰلِمِينَ مِن نَّصِيرٍۢ ﴿٧١﴾}\\
\textamh{72.\  } & \mytextarabic{وَإِذَا تُتْلَىٰ عَلَيْهِمْ ءَايَـٰتُنَا بَيِّنَـٰتٍۢ تَعْرِفُ فِى وُجُوهِ ٱلَّذِينَ كَفَرُوا۟ ٱلْمُنكَرَ ۖ يَكَادُونَ يَسْطُونَ بِٱلَّذِينَ يَتْلُونَ عَلَيْهِمْ ءَايَـٰتِنَا ۗ قُلْ أَفَأُنَبِّئُكُم بِشَرٍّۢ مِّن ذَٟلِكُمُ ۗ ٱلنَّارُ وَعَدَهَا ٱللَّهُ ٱلَّذِينَ كَفَرُوا۟ ۖ وَبِئْسَ ٱلْمَصِيرُ ﴿٧٢﴾}\\
\textamh{73.\  } & \mytextarabic{يَـٰٓأَيُّهَا ٱلنَّاسُ ضُرِبَ مَثَلٌۭ فَٱسْتَمِعُوا۟ لَهُۥٓ ۚ إِنَّ ٱلَّذِينَ تَدْعُونَ مِن دُونِ ٱللَّهِ لَن يَخْلُقُوا۟ ذُبَابًۭا وَلَوِ ٱجْتَمَعُوا۟ لَهُۥ ۖ وَإِن يَسْلُبْهُمُ ٱلذُّبَابُ شَيْـًۭٔا لَّا يَسْتَنقِذُوهُ مِنْهُ ۚ ضَعُفَ ٱلطَّالِبُ وَٱلْمَطْلُوبُ ﴿٧٣﴾}\\
\textamh{74.\  } & \mytextarabic{مَا قَدَرُوا۟ ٱللَّهَ حَقَّ قَدْرِهِۦٓ ۗ إِنَّ ٱللَّهَ لَقَوِىٌّ عَزِيزٌ ﴿٧٤﴾}\\
\textamh{75.\  } & \mytextarabic{ٱللَّهُ يَصْطَفِى مِنَ ٱلْمَلَـٰٓئِكَةِ رُسُلًۭا وَمِنَ ٱلنَّاسِ ۚ إِنَّ ٱللَّهَ سَمِيعٌۢ بَصِيرٌۭ ﴿٧٥﴾}\\
\textamh{76.\  } & \mytextarabic{يَعْلَمُ مَا بَيْنَ أَيْدِيهِمْ وَمَا خَلْفَهُمْ ۗ وَإِلَى ٱللَّهِ تُرْجَعُ ٱلْأُمُورُ ﴿٧٦﴾}\\
\textamh{77.\  } & \mytextarabic{يَـٰٓأَيُّهَا ٱلَّذِينَ ءَامَنُوا۟ ٱرْكَعُوا۟ وَٱسْجُدُوا۟ وَٱعْبُدُوا۟ رَبَّكُمْ وَٱفْعَلُوا۟ ٱلْخَيْرَ لَعَلَّكُمْ تُفْلِحُونَ ۩ ﴿٧٧﴾}\\
\textamh{78.\  } & \mytextarabic{وَجَٰهِدُوا۟ فِى ٱللَّهِ حَقَّ جِهَادِهِۦ ۚ هُوَ ٱجْتَبَىٰكُمْ وَمَا جَعَلَ عَلَيْكُمْ فِى ٱلدِّينِ مِنْ حَرَجٍۢ ۚ مِّلَّةَ أَبِيكُمْ إِبْرَٰهِيمَ ۚ هُوَ سَمَّىٰكُمُ ٱلْمُسْلِمِينَ مِن قَبْلُ وَفِى هَـٰذَا لِيَكُونَ ٱلرَّسُولُ شَهِيدًا عَلَيْكُمْ وَتَكُونُوا۟ شُهَدَآءَ عَلَى ٱلنَّاسِ ۚ فَأَقِيمُوا۟ ٱلصَّلَوٰةَ وَءَاتُوا۟ ٱلزَّكَوٰةَ وَٱعْتَصِمُوا۟ بِٱللَّهِ هُوَ مَوْلَىٰكُمْ ۖ فَنِعْمَ ٱلْمَوْلَىٰ وَنِعْمَ ٱلنَّصِيرُ ﴿٧٨﴾}\\
\end{longtable}
\clearpage
%% License: BSD style (Berkley) (i.e. Put the Copyright owner's name always)
%% Writer and Copyright (to): Bewketu(Bilal) Tadilo (2016-17)
\begin{center}\section{\LR{\textamhsec{ሱራቱ አልሙኡሚን -}  \textarabic{سوره  المؤمنون}}}\end{center}
\begin{longtable}{%
  @{}
    p{.5\textwidth}
  @{~~~}
    p{.5\textwidth}
    @{}
}
\textamh{ቢስሚላሂ አራህመኒ ራሂይም } &  \mytextarabic{بِسْمِ ٱللَّهِ ٱلرَّحْمَـٰنِ ٱلرَّحِيمِ}\\
\textamh{1.\  } & \mytextarabic{ قَدْ أَفْلَحَ ٱلْمُؤْمِنُونَ ﴿١﴾}\\
\textamh{2.\  } & \mytextarabic{ٱلَّذِينَ هُمْ فِى صَلَاتِهِمْ خَـٰشِعُونَ ﴿٢﴾}\\
\textamh{3.\  } & \mytextarabic{وَٱلَّذِينَ هُمْ عَنِ ٱللَّغْوِ مُعْرِضُونَ ﴿٣﴾}\\
\textamh{4.\  } & \mytextarabic{وَٱلَّذِينَ هُمْ لِلزَّكَوٰةِ فَـٰعِلُونَ ﴿٤﴾}\\
\textamh{5.\  } & \mytextarabic{وَٱلَّذِينَ هُمْ لِفُرُوجِهِمْ حَـٰفِظُونَ ﴿٥﴾}\\
\textamh{6.\  } & \mytextarabic{إِلَّا عَلَىٰٓ أَزْوَٟجِهِمْ أَوْ مَا مَلَكَتْ أَيْمَـٰنُهُمْ فَإِنَّهُمْ غَيْرُ مَلُومِينَ ﴿٦﴾}\\
\textamh{7.\  } & \mytextarabic{فَمَنِ ٱبْتَغَىٰ وَرَآءَ ذَٟلِكَ فَأُو۟لَـٰٓئِكَ هُمُ ٱلْعَادُونَ ﴿٧﴾}\\
\textamh{8.\  } & \mytextarabic{وَٱلَّذِينَ هُمْ لِأَمَـٰنَـٰتِهِمْ وَعَهْدِهِمْ رَٰعُونَ ﴿٨﴾}\\
\textamh{9.\  } & \mytextarabic{وَٱلَّذِينَ هُمْ عَلَىٰ صَلَوَٟتِهِمْ يُحَافِظُونَ ﴿٩﴾}\\
\textamh{10.\  } & \mytextarabic{أُو۟لَـٰٓئِكَ هُمُ ٱلْوَٟرِثُونَ ﴿١٠﴾}\\
\textamh{11.\  } & \mytextarabic{ٱلَّذِينَ يَرِثُونَ ٱلْفِرْدَوْسَ هُمْ فِيهَا خَـٰلِدُونَ ﴿١١﴾}\\
\textamh{12.\  } & \mytextarabic{وَلَقَدْ خَلَقْنَا ٱلْإِنسَـٰنَ مِن سُلَـٰلَةٍۢ مِّن طِينٍۢ ﴿١٢﴾}\\
\textamh{13.\  } & \mytextarabic{ثُمَّ جَعَلْنَـٰهُ نُطْفَةًۭ فِى قَرَارٍۢ مَّكِينٍۢ ﴿١٣﴾}\\
\textamh{14.\  } & \mytextarabic{ثُمَّ خَلَقْنَا ٱلنُّطْفَةَ عَلَقَةًۭ فَخَلَقْنَا ٱلْعَلَقَةَ مُضْغَةًۭ فَخَلَقْنَا ٱلْمُضْغَةَ عِظَـٰمًۭا فَكَسَوْنَا ٱلْعِظَـٰمَ لَحْمًۭا ثُمَّ أَنشَأْنَـٰهُ خَلْقًا ءَاخَرَ ۚ فَتَبَارَكَ ٱللَّهُ أَحْسَنُ ٱلْخَـٰلِقِينَ ﴿١٤﴾}\\
\textamh{15.\  } & \mytextarabic{ثُمَّ إِنَّكُم بَعْدَ ذَٟلِكَ لَمَيِّتُونَ ﴿١٥﴾}\\
\textamh{16.\  } & \mytextarabic{ثُمَّ إِنَّكُمْ يَوْمَ ٱلْقِيَـٰمَةِ تُبْعَثُونَ ﴿١٦﴾}\\
\textamh{17.\  } & \mytextarabic{وَلَقَدْ خَلَقْنَا فَوْقَكُمْ سَبْعَ طَرَآئِقَ وَمَا كُنَّا عَنِ ٱلْخَلْقِ غَٰفِلِينَ ﴿١٧﴾}\\
\textamh{18.\  } & \mytextarabic{وَأَنزَلْنَا مِنَ ٱلسَّمَآءِ مَآءًۢ بِقَدَرٍۢ فَأَسْكَنَّـٰهُ فِى ٱلْأَرْضِ ۖ وَإِنَّا عَلَىٰ ذَهَابٍۭ بِهِۦ لَقَـٰدِرُونَ ﴿١٨﴾}\\
\textamh{19.\  } & \mytextarabic{فَأَنشَأْنَا لَكُم بِهِۦ جَنَّـٰتٍۢ مِّن نَّخِيلٍۢ وَأَعْنَـٰبٍۢ لَّكُمْ فِيهَا فَوَٟكِهُ كَثِيرَةٌۭ وَمِنْهَا تَأْكُلُونَ ﴿١٩﴾}\\
\textamh{20.\  } & \mytextarabic{وَشَجَرَةًۭ تَخْرُجُ مِن طُورِ سَيْنَآءَ تَنۢبُتُ بِٱلدُّهْنِ وَصِبْغٍۢ لِّلْءَاكِلِينَ ﴿٢٠﴾}\\
\textamh{21.\  } & \mytextarabic{وَإِنَّ لَكُمْ فِى ٱلْأَنْعَـٰمِ لَعِبْرَةًۭ ۖ نُّسْقِيكُم مِّمَّا فِى بُطُونِهَا وَلَكُمْ فِيهَا مَنَـٰفِعُ كَثِيرَةٌۭ وَمِنْهَا تَأْكُلُونَ ﴿٢١﴾}\\
\textamh{22.\  } & \mytextarabic{وَعَلَيْهَا وَعَلَى ٱلْفُلْكِ تُحْمَلُونَ ﴿٢٢﴾}\\
\textamh{23.\  } & \mytextarabic{وَلَقَدْ أَرْسَلْنَا نُوحًا إِلَىٰ قَوْمِهِۦ فَقَالَ يَـٰقَوْمِ ٱعْبُدُوا۟ ٱللَّهَ مَا لَكُم مِّنْ إِلَـٰهٍ غَيْرُهُۥٓ ۖ أَفَلَا تَتَّقُونَ ﴿٢٣﴾}\\
\textamh{24.\  } & \mytextarabic{فَقَالَ ٱلْمَلَؤُا۟ ٱلَّذِينَ كَفَرُوا۟ مِن قَوْمِهِۦ مَا هَـٰذَآ إِلَّا بَشَرٌۭ مِّثْلُكُمْ يُرِيدُ أَن يَتَفَضَّلَ عَلَيْكُمْ وَلَوْ شَآءَ ٱللَّهُ لَأَنزَلَ مَلَـٰٓئِكَةًۭ مَّا سَمِعْنَا بِهَـٰذَا فِىٓ ءَابَآئِنَا ٱلْأَوَّلِينَ ﴿٢٤﴾}\\
\textamh{25.\  } & \mytextarabic{إِنْ هُوَ إِلَّا رَجُلٌۢ بِهِۦ جِنَّةٌۭ فَتَرَبَّصُوا۟ بِهِۦ حَتَّىٰ حِينٍۢ ﴿٢٥﴾}\\
\textamh{26.\  } & \mytextarabic{قَالَ رَبِّ ٱنصُرْنِى بِمَا كَذَّبُونِ ﴿٢٦﴾}\\
\textamh{27.\  } & \mytextarabic{فَأَوْحَيْنَآ إِلَيْهِ أَنِ ٱصْنَعِ ٱلْفُلْكَ بِأَعْيُنِنَا وَوَحْيِنَا فَإِذَا جَآءَ أَمْرُنَا وَفَارَ ٱلتَّنُّورُ ۙ فَٱسْلُكْ فِيهَا مِن كُلٍّۢ زَوْجَيْنِ ٱثْنَيْنِ وَأَهْلَكَ إِلَّا مَن سَبَقَ عَلَيْهِ ٱلْقَوْلُ مِنْهُمْ ۖ وَلَا تُخَـٰطِبْنِى فِى ٱلَّذِينَ ظَلَمُوٓا۟ ۖ إِنَّهُم مُّغْرَقُونَ ﴿٢٧﴾}\\
\textamh{28.\  } & \mytextarabic{فَإِذَا ٱسْتَوَيْتَ أَنتَ وَمَن مَّعَكَ عَلَى ٱلْفُلْكِ فَقُلِ ٱلْحَمْدُ لِلَّهِ ٱلَّذِى نَجَّىٰنَا مِنَ ٱلْقَوْمِ ٱلظَّـٰلِمِينَ ﴿٢٨﴾}\\
\textamh{29.\  } & \mytextarabic{وَقُل رَّبِّ أَنزِلْنِى مُنزَلًۭا مُّبَارَكًۭا وَأَنتَ خَيْرُ ٱلْمُنزِلِينَ ﴿٢٩﴾}\\
\textamh{30.\  } & \mytextarabic{إِنَّ فِى ذَٟلِكَ لَءَايَـٰتٍۢ وَإِن كُنَّا لَمُبْتَلِينَ ﴿٣٠﴾}\\
\textamh{31.\  } & \mytextarabic{ثُمَّ أَنشَأْنَا مِنۢ بَعْدِهِمْ قَرْنًا ءَاخَرِينَ ﴿٣١﴾}\\
\textamh{32.\  } & \mytextarabic{فَأَرْسَلْنَا فِيهِمْ رَسُولًۭا مِّنْهُمْ أَنِ ٱعْبُدُوا۟ ٱللَّهَ مَا لَكُم مِّنْ إِلَـٰهٍ غَيْرُهُۥٓ ۖ أَفَلَا تَتَّقُونَ ﴿٣٢﴾}\\
\textamh{33.\  } & \mytextarabic{وَقَالَ ٱلْمَلَأُ مِن قَوْمِهِ ٱلَّذِينَ كَفَرُوا۟ وَكَذَّبُوا۟ بِلِقَآءِ ٱلْءَاخِرَةِ وَأَتْرَفْنَـٰهُمْ فِى ٱلْحَيَوٰةِ ٱلدُّنْيَا مَا هَـٰذَآ إِلَّا بَشَرٌۭ مِّثْلُكُمْ يَأْكُلُ مِمَّا تَأْكُلُونَ مِنْهُ وَيَشْرَبُ مِمَّا تَشْرَبُونَ ﴿٣٣﴾}\\
\textamh{34.\  } & \mytextarabic{وَلَئِنْ أَطَعْتُم بَشَرًۭا مِّثْلَكُمْ إِنَّكُمْ إِذًۭا لَّخَـٰسِرُونَ ﴿٣٤﴾}\\
\textamh{35.\  } & \mytextarabic{أَيَعِدُكُمْ أَنَّكُمْ إِذَا مِتُّمْ وَكُنتُمْ تُرَابًۭا وَعِظَـٰمًا أَنَّكُم مُّخْرَجُونَ ﴿٣٥﴾}\\
\textamh{36.\  } & \mytextarabic{۞ هَيْهَاتَ هَيْهَاتَ لِمَا تُوعَدُونَ ﴿٣٦﴾}\\
\textamh{37.\  } & \mytextarabic{إِنْ هِىَ إِلَّا حَيَاتُنَا ٱلدُّنْيَا نَمُوتُ وَنَحْيَا وَمَا نَحْنُ بِمَبْعُوثِينَ ﴿٣٧﴾}\\
\textamh{38.\  } & \mytextarabic{إِنْ هُوَ إِلَّا رَجُلٌ ٱفْتَرَىٰ عَلَى ٱللَّهِ كَذِبًۭا وَمَا نَحْنُ لَهُۥ بِمُؤْمِنِينَ ﴿٣٨﴾}\\
\textamh{39.\  } & \mytextarabic{قَالَ رَبِّ ٱنصُرْنِى بِمَا كَذَّبُونِ ﴿٣٩﴾}\\
\textamh{40.\  } & \mytextarabic{قَالَ عَمَّا قَلِيلٍۢ لَّيُصْبِحُنَّ نَـٰدِمِينَ ﴿٤٠﴾}\\
\textamh{41.\  } & \mytextarabic{فَأَخَذَتْهُمُ ٱلصَّيْحَةُ بِٱلْحَقِّ فَجَعَلْنَـٰهُمْ غُثَآءًۭ ۚ فَبُعْدًۭا لِّلْقَوْمِ ٱلظَّـٰلِمِينَ ﴿٤١﴾}\\
\textamh{42.\  } & \mytextarabic{ثُمَّ أَنشَأْنَا مِنۢ بَعْدِهِمْ قُرُونًا ءَاخَرِينَ ﴿٤٢﴾}\\
\textamh{43.\  } & \mytextarabic{مَا تَسْبِقُ مِنْ أُمَّةٍ أَجَلَهَا وَمَا يَسْتَـْٔخِرُونَ ﴿٤٣﴾}\\
\textamh{44.\  } & \mytextarabic{ثُمَّ أَرْسَلْنَا رُسُلَنَا تَتْرَا ۖ كُلَّ مَا جَآءَ أُمَّةًۭ رَّسُولُهَا كَذَّبُوهُ ۚ فَأَتْبَعْنَا بَعْضَهُم بَعْضًۭا وَجَعَلْنَـٰهُمْ أَحَادِيثَ ۚ فَبُعْدًۭا لِّقَوْمٍۢ لَّا يُؤْمِنُونَ ﴿٤٤﴾}\\
\textamh{45.\  } & \mytextarabic{ثُمَّ أَرْسَلْنَا مُوسَىٰ وَأَخَاهُ هَـٰرُونَ بِـَٔايَـٰتِنَا وَسُلْطَٰنٍۢ مُّبِينٍ ﴿٤٥﴾}\\
\textamh{46.\  } & \mytextarabic{إِلَىٰ فِرْعَوْنَ وَمَلَإِي۟هِۦ فَٱسْتَكْبَرُوا۟ وَكَانُوا۟ قَوْمًا عَالِينَ ﴿٤٦﴾}\\
\textamh{47.\  } & \mytextarabic{فَقَالُوٓا۟ أَنُؤْمِنُ لِبَشَرَيْنِ مِثْلِنَا وَقَوْمُهُمَا لَنَا عَـٰبِدُونَ ﴿٤٧﴾}\\
\textamh{48.\  } & \mytextarabic{فَكَذَّبُوهُمَا فَكَانُوا۟ مِنَ ٱلْمُهْلَكِينَ ﴿٤٨﴾}\\
\textamh{49.\  } & \mytextarabic{وَلَقَدْ ءَاتَيْنَا مُوسَى ٱلْكِتَـٰبَ لَعَلَّهُمْ يَهْتَدُونَ ﴿٤٩﴾}\\
\textamh{50.\  } & \mytextarabic{وَجَعَلْنَا ٱبْنَ مَرْيَمَ وَأُمَّهُۥٓ ءَايَةًۭ وَءَاوَيْنَـٰهُمَآ إِلَىٰ رَبْوَةٍۢ ذَاتِ قَرَارٍۢ وَمَعِينٍۢ ﴿٥٠﴾}\\
\textamh{51.\  } & \mytextarabic{يَـٰٓأَيُّهَا ٱلرُّسُلُ كُلُوا۟ مِنَ ٱلطَّيِّبَٰتِ وَٱعْمَلُوا۟ صَـٰلِحًا ۖ إِنِّى بِمَا تَعْمَلُونَ عَلِيمٌۭ ﴿٥١﴾}\\
\textamh{52.\  } & \mytextarabic{وَإِنَّ هَـٰذِهِۦٓ أُمَّتُكُمْ أُمَّةًۭ وَٟحِدَةًۭ وَأَنَا۠ رَبُّكُمْ فَٱتَّقُونِ ﴿٥٢﴾}\\
\textamh{53.\  } & \mytextarabic{فَتَقَطَّعُوٓا۟ أَمْرَهُم بَيْنَهُمْ زُبُرًۭا ۖ كُلُّ حِزْبٍۭ بِمَا لَدَيْهِمْ فَرِحُونَ ﴿٥٣﴾}\\
\textamh{54.\  } & \mytextarabic{فَذَرْهُمْ فِى غَمْرَتِهِمْ حَتَّىٰ حِينٍ ﴿٥٤﴾}\\
\textamh{55.\  } & \mytextarabic{أَيَحْسَبُونَ أَنَّمَا نُمِدُّهُم بِهِۦ مِن مَّالٍۢ وَبَنِينَ ﴿٥٥﴾}\\
\textamh{56.\  } & \mytextarabic{نُسَارِعُ لَهُمْ فِى ٱلْخَيْرَٰتِ ۚ بَل لَّا يَشْعُرُونَ ﴿٥٦﴾}\\
\textamh{57.\  } & \mytextarabic{إِنَّ ٱلَّذِينَ هُم مِّنْ خَشْيَةِ رَبِّهِم مُّشْفِقُونَ ﴿٥٧﴾}\\
\textamh{58.\  } & \mytextarabic{وَٱلَّذِينَ هُم بِـَٔايَـٰتِ رَبِّهِمْ يُؤْمِنُونَ ﴿٥٨﴾}\\
\textamh{59.\  } & \mytextarabic{وَٱلَّذِينَ هُم بِرَبِّهِمْ لَا يُشْرِكُونَ ﴿٥٩﴾}\\
\textamh{60.\  } & \mytextarabic{وَٱلَّذِينَ يُؤْتُونَ مَآ ءَاتَوا۟ وَّقُلُوبُهُمْ وَجِلَةٌ أَنَّهُمْ إِلَىٰ رَبِّهِمْ رَٰجِعُونَ ﴿٦٠﴾}\\
\textamh{61.\  } & \mytextarabic{أُو۟لَـٰٓئِكَ يُسَـٰرِعُونَ فِى ٱلْخَيْرَٰتِ وَهُمْ لَهَا سَـٰبِقُونَ ﴿٦١﴾}\\
\textamh{62.\  } & \mytextarabic{وَلَا نُكَلِّفُ نَفْسًا إِلَّا وُسْعَهَا ۖ وَلَدَيْنَا كِتَـٰبٌۭ يَنطِقُ بِٱلْحَقِّ ۚ وَهُمْ لَا يُظْلَمُونَ ﴿٦٢﴾}\\
\textamh{63.\  } & \mytextarabic{بَلْ قُلُوبُهُمْ فِى غَمْرَةٍۢ مِّنْ هَـٰذَا وَلَهُمْ أَعْمَـٰلٌۭ مِّن دُونِ ذَٟلِكَ هُمْ لَهَا عَـٰمِلُونَ ﴿٦٣﴾}\\
\textamh{64.\  } & \mytextarabic{حَتَّىٰٓ إِذَآ أَخَذْنَا مُتْرَفِيهِم بِٱلْعَذَابِ إِذَا هُمْ يَجْـَٔرُونَ ﴿٦٤﴾}\\
\textamh{65.\  } & \mytextarabic{لَا تَجْـَٔرُوا۟ ٱلْيَوْمَ ۖ إِنَّكُم مِّنَّا لَا تُنصَرُونَ ﴿٦٥﴾}\\
\textamh{66.\  } & \mytextarabic{قَدْ كَانَتْ ءَايَـٰتِى تُتْلَىٰ عَلَيْكُمْ فَكُنتُمْ عَلَىٰٓ أَعْقَـٰبِكُمْ تَنكِصُونَ ﴿٦٦﴾}\\
\textamh{67.\  } & \mytextarabic{مُسْتَكْبِرِينَ بِهِۦ سَـٰمِرًۭا تَهْجُرُونَ ﴿٦٧﴾}\\
\textamh{68.\  } & \mytextarabic{أَفَلَمْ يَدَّبَّرُوا۟ ٱلْقَوْلَ أَمْ جَآءَهُم مَّا لَمْ يَأْتِ ءَابَآءَهُمُ ٱلْأَوَّلِينَ ﴿٦٨﴾}\\
\textamh{69.\  } & \mytextarabic{أَمْ لَمْ يَعْرِفُوا۟ رَسُولَهُمْ فَهُمْ لَهُۥ مُنكِرُونَ ﴿٦٩﴾}\\
\textamh{70.\  } & \mytextarabic{أَمْ يَقُولُونَ بِهِۦ جِنَّةٌۢ ۚ بَلْ جَآءَهُم بِٱلْحَقِّ وَأَكْثَرُهُمْ لِلْحَقِّ كَـٰرِهُونَ ﴿٧٠﴾}\\
\textamh{71.\  } & \mytextarabic{وَلَوِ ٱتَّبَعَ ٱلْحَقُّ أَهْوَآءَهُمْ لَفَسَدَتِ ٱلسَّمَـٰوَٟتُ وَٱلْأَرْضُ وَمَن فِيهِنَّ ۚ بَلْ أَتَيْنَـٰهُم بِذِكْرِهِمْ فَهُمْ عَن ذِكْرِهِم مُّعْرِضُونَ ﴿٧١﴾}\\
\textamh{72.\  } & \mytextarabic{أَمْ تَسْـَٔلُهُمْ خَرْجًۭا فَخَرَاجُ رَبِّكَ خَيْرٌۭ ۖ وَهُوَ خَيْرُ ٱلرَّٟزِقِينَ ﴿٧٢﴾}\\
\textamh{73.\  } & \mytextarabic{وَإِنَّكَ لَتَدْعُوهُمْ إِلَىٰ صِرَٰطٍۢ مُّسْتَقِيمٍۢ ﴿٧٣﴾}\\
\textamh{74.\  } & \mytextarabic{وَإِنَّ ٱلَّذِينَ لَا يُؤْمِنُونَ بِٱلْءَاخِرَةِ عَنِ ٱلصِّرَٰطِ لَنَـٰكِبُونَ ﴿٧٤﴾}\\
\textamh{75.\  } & \mytextarabic{۞ وَلَوْ رَحِمْنَـٰهُمْ وَكَشَفْنَا مَا بِهِم مِّن ضُرٍّۢ لَّلَجُّوا۟ فِى طُغْيَـٰنِهِمْ يَعْمَهُونَ ﴿٧٥﴾}\\
\textamh{76.\  } & \mytextarabic{وَلَقَدْ أَخَذْنَـٰهُم بِٱلْعَذَابِ فَمَا ٱسْتَكَانُوا۟ لِرَبِّهِمْ وَمَا يَتَضَرَّعُونَ ﴿٧٦﴾}\\
\textamh{77.\  } & \mytextarabic{حَتَّىٰٓ إِذَا فَتَحْنَا عَلَيْهِم بَابًۭا ذَا عَذَابٍۢ شَدِيدٍ إِذَا هُمْ فِيهِ مُبْلِسُونَ ﴿٧٧﴾}\\
\textamh{78.\  } & \mytextarabic{وَهُوَ ٱلَّذِىٓ أَنشَأَ لَكُمُ ٱلسَّمْعَ وَٱلْأَبْصَـٰرَ وَٱلْأَفْـِٔدَةَ ۚ قَلِيلًۭا مَّا تَشْكُرُونَ ﴿٧٨﴾}\\
\textamh{79.\  } & \mytextarabic{وَهُوَ ٱلَّذِى ذَرَأَكُمْ فِى ٱلْأَرْضِ وَإِلَيْهِ تُحْشَرُونَ ﴿٧٩﴾}\\
\textamh{80.\  } & \mytextarabic{وَهُوَ ٱلَّذِى يُحْىِۦ وَيُمِيتُ وَلَهُ ٱخْتِلَـٰفُ ٱلَّيْلِ وَٱلنَّهَارِ ۚ أَفَلَا تَعْقِلُونَ ﴿٨٠﴾}\\
\textamh{81.\  } & \mytextarabic{بَلْ قَالُوا۟ مِثْلَ مَا قَالَ ٱلْأَوَّلُونَ ﴿٨١﴾}\\
\textamh{82.\  } & \mytextarabic{قَالُوٓا۟ أَءِذَا مِتْنَا وَكُنَّا تُرَابًۭا وَعِظَـٰمًا أَءِنَّا لَمَبْعُوثُونَ ﴿٨٢﴾}\\
\textamh{83.\  } & \mytextarabic{لَقَدْ وُعِدْنَا نَحْنُ وَءَابَآؤُنَا هَـٰذَا مِن قَبْلُ إِنْ هَـٰذَآ إِلَّآ أَسَـٰطِيرُ ٱلْأَوَّلِينَ ﴿٨٣﴾}\\
\textamh{84.\  } & \mytextarabic{قُل لِّمَنِ ٱلْأَرْضُ وَمَن فِيهَآ إِن كُنتُمْ تَعْلَمُونَ ﴿٨٤﴾}\\
\textamh{85.\  } & \mytextarabic{سَيَقُولُونَ لِلَّهِ ۚ قُلْ أَفَلَا تَذَكَّرُونَ ﴿٨٥﴾}\\
\textamh{86.\  } & \mytextarabic{قُلْ مَن رَّبُّ ٱلسَّمَـٰوَٟتِ ٱلسَّبْعِ وَرَبُّ ٱلْعَرْشِ ٱلْعَظِيمِ ﴿٨٦﴾}\\
\textamh{87.\  } & \mytextarabic{سَيَقُولُونَ لِلَّهِ ۚ قُلْ أَفَلَا تَتَّقُونَ ﴿٨٧﴾}\\
\textamh{88.\  } & \mytextarabic{قُلْ مَنۢ بِيَدِهِۦ مَلَكُوتُ كُلِّ شَىْءٍۢ وَهُوَ يُجِيرُ وَلَا يُجَارُ عَلَيْهِ إِن كُنتُمْ تَعْلَمُونَ ﴿٨٨﴾}\\
\textamh{89.\  } & \mytextarabic{سَيَقُولُونَ لِلَّهِ ۚ قُلْ فَأَنَّىٰ تُسْحَرُونَ ﴿٨٩﴾}\\
\textamh{90.\  } & \mytextarabic{بَلْ أَتَيْنَـٰهُم بِٱلْحَقِّ وَإِنَّهُمْ لَكَـٰذِبُونَ ﴿٩٠﴾}\\
\textamh{91.\  } & \mytextarabic{مَا ٱتَّخَذَ ٱللَّهُ مِن وَلَدٍۢ وَمَا كَانَ مَعَهُۥ مِنْ إِلَـٰهٍ ۚ إِذًۭا لَّذَهَبَ كُلُّ إِلَـٰهٍۭ بِمَا خَلَقَ وَلَعَلَا بَعْضُهُمْ عَلَىٰ بَعْضٍۢ ۚ سُبْحَـٰنَ ٱللَّهِ عَمَّا يَصِفُونَ ﴿٩١﴾}\\
\textamh{92.\  } & \mytextarabic{عَـٰلِمِ ٱلْغَيْبِ وَٱلشَّهَـٰدَةِ فَتَعَـٰلَىٰ عَمَّا يُشْرِكُونَ ﴿٩٢﴾}\\
\textamh{93.\  } & \mytextarabic{قُل رَّبِّ إِمَّا تُرِيَنِّى مَا يُوعَدُونَ ﴿٩٣﴾}\\
\textamh{94.\  } & \mytextarabic{رَبِّ فَلَا تَجْعَلْنِى فِى ٱلْقَوْمِ ٱلظَّـٰلِمِينَ ﴿٩٤﴾}\\
\textamh{95.\  } & \mytextarabic{وَإِنَّا عَلَىٰٓ أَن نُّرِيَكَ مَا نَعِدُهُمْ لَقَـٰدِرُونَ ﴿٩٥﴾}\\
\textamh{96.\  } & \mytextarabic{ٱدْفَعْ بِٱلَّتِى هِىَ أَحْسَنُ ٱلسَّيِّئَةَ ۚ نَحْنُ أَعْلَمُ بِمَا يَصِفُونَ ﴿٩٦﴾}\\
\textamh{97.\  } & \mytextarabic{وَقُل رَّبِّ أَعُوذُ بِكَ مِنْ هَمَزَٰتِ ٱلشَّيَـٰطِينِ ﴿٩٧﴾}\\
\textamh{98.\  } & \mytextarabic{وَأَعُوذُ بِكَ رَبِّ أَن يَحْضُرُونِ ﴿٩٨﴾}\\
\textamh{99.\  } & \mytextarabic{حَتَّىٰٓ إِذَا جَآءَ أَحَدَهُمُ ٱلْمَوْتُ قَالَ رَبِّ ٱرْجِعُونِ ﴿٩٩﴾}\\
\textamh{100.\  } & \mytextarabic{لَعَلِّىٓ أَعْمَلُ صَـٰلِحًۭا فِيمَا تَرَكْتُ ۚ كَلَّآ ۚ إِنَّهَا كَلِمَةٌ هُوَ قَآئِلُهَا ۖ وَمِن وَرَآئِهِم بَرْزَخٌ إِلَىٰ يَوْمِ يُبْعَثُونَ ﴿١٠٠﴾}\\
\textamh{101.\  } & \mytextarabic{فَإِذَا نُفِخَ فِى ٱلصُّورِ فَلَآ أَنسَابَ بَيْنَهُمْ يَوْمَئِذٍۢ وَلَا يَتَسَآءَلُونَ ﴿١٠١﴾}\\
\textamh{102.\  } & \mytextarabic{فَمَن ثَقُلَتْ مَوَٟزِينُهُۥ فَأُو۟لَـٰٓئِكَ هُمُ ٱلْمُفْلِحُونَ ﴿١٠٢﴾}\\
\textamh{103.\  } & \mytextarabic{وَمَنْ خَفَّتْ مَوَٟزِينُهُۥ فَأُو۟لَـٰٓئِكَ ٱلَّذِينَ خَسِرُوٓا۟ أَنفُسَهُمْ فِى جَهَنَّمَ خَـٰلِدُونَ ﴿١٠٣﴾}\\
\textamh{104.\  } & \mytextarabic{تَلْفَحُ وُجُوهَهُمُ ٱلنَّارُ وَهُمْ فِيهَا كَـٰلِحُونَ ﴿١٠٤﴾}\\
\textamh{105.\  } & \mytextarabic{أَلَمْ تَكُنْ ءَايَـٰتِى تُتْلَىٰ عَلَيْكُمْ فَكُنتُم بِهَا تُكَذِّبُونَ ﴿١٠٥﴾}\\
\textamh{106.\  } & \mytextarabic{قَالُوا۟ رَبَّنَا غَلَبَتْ عَلَيْنَا شِقْوَتُنَا وَكُنَّا قَوْمًۭا ضَآلِّينَ ﴿١٠٦﴾}\\
\textamh{107.\  } & \mytextarabic{رَبَّنَآ أَخْرِجْنَا مِنْهَا فَإِنْ عُدْنَا فَإِنَّا ظَـٰلِمُونَ ﴿١٠٧﴾}\\
\textamh{108.\  } & \mytextarabic{قَالَ ٱخْسَـُٔوا۟ فِيهَا وَلَا تُكَلِّمُونِ ﴿١٠٨﴾}\\
\textamh{109.\  } & \mytextarabic{إِنَّهُۥ كَانَ فَرِيقٌۭ مِّنْ عِبَادِى يَقُولُونَ رَبَّنَآ ءَامَنَّا فَٱغْفِرْ لَنَا وَٱرْحَمْنَا وَأَنتَ خَيْرُ ٱلرَّٟحِمِينَ ﴿١٠٩﴾}\\
\textamh{110.\  } & \mytextarabic{فَٱتَّخَذْتُمُوهُمْ سِخْرِيًّا حَتَّىٰٓ أَنسَوْكُمْ ذِكْرِى وَكُنتُم مِّنْهُمْ تَضْحَكُونَ ﴿١١٠﴾}\\
\textamh{111.\  } & \mytextarabic{إِنِّى جَزَيْتُهُمُ ٱلْيَوْمَ بِمَا صَبَرُوٓا۟ أَنَّهُمْ هُمُ ٱلْفَآئِزُونَ ﴿١١١﴾}\\
\textamh{112.\  } & \mytextarabic{قَـٰلَ كَمْ لَبِثْتُمْ فِى ٱلْأَرْضِ عَدَدَ سِنِينَ ﴿١١٢﴾}\\
\textamh{113.\  } & \mytextarabic{قَالُوا۟ لَبِثْنَا يَوْمًا أَوْ بَعْضَ يَوْمٍۢ فَسْـَٔلِ ٱلْعَآدِّينَ ﴿١١٣﴾}\\
\textamh{114.\  } & \mytextarabic{قَـٰلَ إِن لَّبِثْتُمْ إِلَّا قَلِيلًۭا ۖ لَّوْ أَنَّكُمْ كُنتُمْ تَعْلَمُونَ ﴿١١٤﴾}\\
\textamh{115.\  } & \mytextarabic{أَفَحَسِبْتُمْ أَنَّمَا خَلَقْنَـٰكُمْ عَبَثًۭا وَأَنَّكُمْ إِلَيْنَا لَا تُرْجَعُونَ ﴿١١٥﴾}\\
\textamh{116.\  } & \mytextarabic{فَتَعَـٰلَى ٱللَّهُ ٱلْمَلِكُ ٱلْحَقُّ ۖ لَآ إِلَـٰهَ إِلَّا هُوَ رَبُّ ٱلْعَرْشِ ٱلْكَرِيمِ ﴿١١٦﴾}\\
\textamh{117.\  } & \mytextarabic{وَمَن يَدْعُ مَعَ ٱللَّهِ إِلَـٰهًا ءَاخَرَ لَا بُرْهَـٰنَ لَهُۥ بِهِۦ فَإِنَّمَا حِسَابُهُۥ عِندَ رَبِّهِۦٓ ۚ إِنَّهُۥ لَا يُفْلِحُ ٱلْكَـٰفِرُونَ ﴿١١٧﴾}\\
\textamh{118.\  } & \mytextarabic{وَقُل رَّبِّ ٱغْفِرْ وَٱرْحَمْ وَأَنتَ خَيْرُ ٱلرَّٟحِمِينَ ﴿١١٨﴾}\\
\end{longtable}
\clearpage
%% License: BSD style (Berkley) (i.e. Put the Copyright owner's name always)
%% Writer and Copyright (to): Bewketu(Bilal) Tadilo (2016-17)
\centering\section{\LR{\textamharic{ሱራቱ አንኑር -}  \RL{سوره  النور}}}
\begin{longtable}{%
  @{}
    p{.5\textwidth}
  @{~~~~~~~~~~~~}
    p{.5\textwidth}
    @{}
}
\nopagebreak
\textamh{ቢስሚላሂ አራህመኒ ራሂይም } &  بِسْمِ ٱللَّهِ ٱلرَّحْمَـٰنِ ٱلرَّحِيمِ\\
\textamh{1.\  } &  سُورَةٌ أَنزَلْنَـٰهَا وَفَرَضْنَـٰهَا وَأَنزَلْنَا فِيهَآ ءَايَـٰتٍۭ بَيِّنَـٰتٍۢ لَّعَلَّكُمْ تَذَكَّرُونَ ﴿١﴾\\
\textamh{2.\  } & ٱلزَّانِيَةُ وَٱلزَّانِى فَٱجْلِدُوا۟ كُلَّ وَٟحِدٍۢ مِّنْهُمَا مِا۟ئَةَ جَلْدَةٍۢ ۖ وَلَا تَأْخُذْكُم بِهِمَا رَأْفَةٌۭ فِى دِينِ ٱللَّهِ إِن كُنتُمْ تُؤْمِنُونَ بِٱللَّهِ وَٱلْيَوْمِ ٱلْءَاخِرِ ۖ وَلْيَشْهَدْ عَذَابَهُمَا طَآئِفَةٌۭ مِّنَ ٱلْمُؤْمِنِينَ ﴿٢﴾\\
\textamh{3.\  } & ٱلزَّانِى لَا يَنكِحُ إِلَّا زَانِيَةً أَوْ مُشْرِكَةًۭ وَٱلزَّانِيَةُ لَا يَنكِحُهَآ إِلَّا زَانٍ أَوْ مُشْرِكٌۭ ۚ وَحُرِّمَ ذَٟلِكَ عَلَى ٱلْمُؤْمِنِينَ ﴿٣﴾\\
\textamh{4.\  } & وَٱلَّذِينَ يَرْمُونَ ٱلْمُحْصَنَـٰتِ ثُمَّ لَمْ يَأْتُوا۟ بِأَرْبَعَةِ شُهَدَآءَ فَٱجْلِدُوهُمْ ثَمَـٰنِينَ جَلْدَةًۭ وَلَا تَقْبَلُوا۟ لَهُمْ شَهَـٰدَةً أَبَدًۭا ۚ وَأُو۟لَـٰٓئِكَ هُمُ ٱلْفَـٰسِقُونَ ﴿٤﴾\\
\textamh{5.\  } & إِلَّا ٱلَّذِينَ تَابُوا۟ مِنۢ بَعْدِ ذَٟلِكَ وَأَصْلَحُوا۟ فَإِنَّ ٱللَّهَ غَفُورٌۭ رَّحِيمٌۭ ﴿٥﴾\\
\textamh{6.\  } & وَٱلَّذِينَ يَرْمُونَ أَزْوَٟجَهُمْ وَلَمْ يَكُن لَّهُمْ شُهَدَآءُ إِلَّآ أَنفُسُهُمْ فَشَهَـٰدَةُ أَحَدِهِمْ أَرْبَعُ شَهَـٰدَٟتٍۭ بِٱللَّهِ ۙ إِنَّهُۥ لَمِنَ ٱلصَّـٰدِقِينَ ﴿٦﴾\\
\textamh{7.\  } & وَٱلْخَـٰمِسَةُ أَنَّ لَعْنَتَ ٱللَّهِ عَلَيْهِ إِن كَانَ مِنَ ٱلْكَـٰذِبِينَ ﴿٧﴾\\
\textamh{8.\  } & وَيَدْرَؤُا۟ عَنْهَا ٱلْعَذَابَ أَن تَشْهَدَ أَرْبَعَ شَهَـٰدَٟتٍۭ بِٱللَّهِ ۙ إِنَّهُۥ لَمِنَ ٱلْكَـٰذِبِينَ ﴿٨﴾\\
\textamh{9.\  } & وَٱلْخَـٰمِسَةَ أَنَّ غَضَبَ ٱللَّهِ عَلَيْهَآ إِن كَانَ مِنَ ٱلصَّـٰدِقِينَ ﴿٩﴾\\
\textamh{10.\  } & وَلَوْلَا فَضْلُ ٱللَّهِ عَلَيْكُمْ وَرَحْمَتُهُۥ وَأَنَّ ٱللَّهَ تَوَّابٌ حَكِيمٌ ﴿١٠﴾\\
\textamh{11.\  } & إِنَّ ٱلَّذِينَ جَآءُو بِٱلْإِفْكِ عُصْبَةٌۭ مِّنكُمْ ۚ لَا تَحْسَبُوهُ شَرًّۭا لَّكُم ۖ بَلْ هُوَ خَيْرٌۭ لَّكُمْ ۚ لِكُلِّ ٱمْرِئٍۢ مِّنْهُم مَّا ٱكْتَسَبَ مِنَ ٱلْإِثْمِ ۚ وَٱلَّذِى تَوَلَّىٰ كِبْرَهُۥ مِنْهُمْ لَهُۥ عَذَابٌ عَظِيمٌۭ ﴿١١﴾\\
\textamh{12.\  } & لَّوْلَآ إِذْ سَمِعْتُمُوهُ ظَنَّ ٱلْمُؤْمِنُونَ وَٱلْمُؤْمِنَـٰتُ بِأَنفُسِهِمْ خَيْرًۭا وَقَالُوا۟ هَـٰذَآ إِفْكٌۭ مُّبِينٌۭ ﴿١٢﴾\\
\textamh{13.\  } & لَّوْلَا جَآءُو عَلَيْهِ بِأَرْبَعَةِ شُهَدَآءَ ۚ فَإِذْ لَمْ يَأْتُوا۟ بِٱلشُّهَدَآءِ فَأُو۟لَـٰٓئِكَ عِندَ ٱللَّهِ هُمُ ٱلْكَـٰذِبُونَ ﴿١٣﴾\\
\textamh{14.\  } & وَلَوْلَا فَضْلُ ٱللَّهِ عَلَيْكُمْ وَرَحْمَتُهُۥ فِى ٱلدُّنْيَا وَٱلْءَاخِرَةِ لَمَسَّكُمْ فِى مَآ أَفَضْتُمْ فِيهِ عَذَابٌ عَظِيمٌ ﴿١٤﴾\\
\textamh{15.\  } & إِذْ تَلَقَّوْنَهُۥ بِأَلْسِنَتِكُمْ وَتَقُولُونَ بِأَفْوَاهِكُم مَّا لَيْسَ لَكُم بِهِۦ عِلْمٌۭ وَتَحْسَبُونَهُۥ هَيِّنًۭا وَهُوَ عِندَ ٱللَّهِ عَظِيمٌۭ ﴿١٥﴾\\
\textamh{16.\  } & وَلَوْلَآ إِذْ سَمِعْتُمُوهُ قُلْتُم مَّا يَكُونُ لَنَآ أَن نَّتَكَلَّمَ بِهَـٰذَا سُبْحَـٰنَكَ هَـٰذَا بُهْتَـٰنٌ عَظِيمٌۭ ﴿١٦﴾\\
\textamh{17.\  } & يَعِظُكُمُ ٱللَّهُ أَن تَعُودُوا۟ لِمِثْلِهِۦٓ أَبَدًا إِن كُنتُم مُّؤْمِنِينَ ﴿١٧﴾\\
\textamh{18.\  } & وَيُبَيِّنُ ٱللَّهُ لَكُمُ ٱلْءَايَـٰتِ ۚ وَٱللَّهُ عَلِيمٌ حَكِيمٌ ﴿١٨﴾\\
\textamh{19.\  } & إِنَّ ٱلَّذِينَ يُحِبُّونَ أَن تَشِيعَ ٱلْفَـٰحِشَةُ فِى ٱلَّذِينَ ءَامَنُوا۟ لَهُمْ عَذَابٌ أَلِيمٌۭ فِى ٱلدُّنْيَا وَٱلْءَاخِرَةِ ۚ وَٱللَّهُ يَعْلَمُ وَأَنتُمْ لَا تَعْلَمُونَ ﴿١٩﴾\\
\textamh{20.\  } & وَلَوْلَا فَضْلُ ٱللَّهِ عَلَيْكُمْ وَرَحْمَتُهُۥ وَأَنَّ ٱللَّهَ رَءُوفٌۭ رَّحِيمٌۭ ﴿٢٠﴾\\
\textamh{21.\  } & ۞ يَـٰٓأَيُّهَا ٱلَّذِينَ ءَامَنُوا۟ لَا تَتَّبِعُوا۟ خُطُوَٟتِ ٱلشَّيْطَٰنِ ۚ وَمَن يَتَّبِعْ خُطُوَٟتِ ٱلشَّيْطَٰنِ فَإِنَّهُۥ يَأْمُرُ بِٱلْفَحْشَآءِ وَٱلْمُنكَرِ ۚ وَلَوْلَا فَضْلُ ٱللَّهِ عَلَيْكُمْ وَرَحْمَتُهُۥ مَا زَكَىٰ مِنكُم مِّنْ أَحَدٍ أَبَدًۭا وَلَـٰكِنَّ ٱللَّهَ يُزَكِّى مَن يَشَآءُ ۗ وَٱللَّهُ سَمِيعٌ عَلِيمٌۭ ﴿٢١﴾\\
\textamh{22.\  } & وَلَا يَأْتَلِ أُو۟لُوا۟ ٱلْفَضْلِ مِنكُمْ وَٱلسَّعَةِ أَن يُؤْتُوٓا۟ أُو۟لِى ٱلْقُرْبَىٰ وَٱلْمَسَـٰكِينَ وَٱلْمُهَـٰجِرِينَ فِى سَبِيلِ ٱللَّهِ ۖ وَلْيَعْفُوا۟ وَلْيَصْفَحُوٓا۟ ۗ أَلَا تُحِبُّونَ أَن يَغْفِرَ ٱللَّهُ لَكُمْ ۗ وَٱللَّهُ غَفُورٌۭ رَّحِيمٌ ﴿٢٢﴾\\
\textamh{23.\  } & إِنَّ ٱلَّذِينَ يَرْمُونَ ٱلْمُحْصَنَـٰتِ ٱلْغَٰفِلَـٰتِ ٱلْمُؤْمِنَـٰتِ لُعِنُوا۟ فِى ٱلدُّنْيَا وَٱلْءَاخِرَةِ وَلَهُمْ عَذَابٌ عَظِيمٌۭ ﴿٢٣﴾\\
\textamh{24.\  } & يَوْمَ تَشْهَدُ عَلَيْهِمْ أَلْسِنَتُهُمْ وَأَيْدِيهِمْ وَأَرْجُلُهُم بِمَا كَانُوا۟ يَعْمَلُونَ ﴿٢٤﴾\\
\textamh{25.\  } & يَوْمَئِذٍۢ يُوَفِّيهِمُ ٱللَّهُ دِينَهُمُ ٱلْحَقَّ وَيَعْلَمُونَ أَنَّ ٱللَّهَ هُوَ ٱلْحَقُّ ٱلْمُبِينُ ﴿٢٥﴾\\
\textamh{26.\  } & ٱلْخَبِيثَـٰتُ لِلْخَبِيثِينَ وَٱلْخَبِيثُونَ لِلْخَبِيثَـٰتِ ۖ وَٱلطَّيِّبَٰتُ لِلطَّيِّبِينَ وَٱلطَّيِّبُونَ لِلطَّيِّبَٰتِ ۚ أُو۟لَـٰٓئِكَ مُبَرَّءُونَ مِمَّا يَقُولُونَ ۖ لَهُم مَّغْفِرَةٌۭ وَرِزْقٌۭ كَرِيمٌۭ ﴿٢٦﴾\\
\textamh{27.\  } & يَـٰٓأَيُّهَا ٱلَّذِينَ ءَامَنُوا۟ لَا تَدْخُلُوا۟ بُيُوتًا غَيْرَ بُيُوتِكُمْ حَتَّىٰ تَسْتَأْنِسُوا۟ وَتُسَلِّمُوا۟ عَلَىٰٓ أَهْلِهَا ۚ ذَٟلِكُمْ خَيْرٌۭ لَّكُمْ لَعَلَّكُمْ تَذَكَّرُونَ ﴿٢٧﴾\\
\textamh{28.\  } & فَإِن لَّمْ تَجِدُوا۟ فِيهَآ أَحَدًۭا فَلَا تَدْخُلُوهَا حَتَّىٰ يُؤْذَنَ لَكُمْ ۖ وَإِن قِيلَ لَكُمُ ٱرْجِعُوا۟ فَٱرْجِعُوا۟ ۖ هُوَ أَزْكَىٰ لَكُمْ ۚ وَٱللَّهُ بِمَا تَعْمَلُونَ عَلِيمٌۭ ﴿٢٨﴾\\
\textamh{29.\  } & لَّيْسَ عَلَيْكُمْ جُنَاحٌ أَن تَدْخُلُوا۟ بُيُوتًا غَيْرَ مَسْكُونَةٍۢ فِيهَا مَتَـٰعٌۭ لَّكُمْ ۚ وَٱللَّهُ يَعْلَمُ مَا تُبْدُونَ وَمَا تَكْتُمُونَ ﴿٢٩﴾\\
\textamh{30.\  } & قُل لِّلْمُؤْمِنِينَ يَغُضُّوا۟ مِنْ أَبْصَـٰرِهِمْ وَيَحْفَظُوا۟ فُرُوجَهُمْ ۚ ذَٟلِكَ أَزْكَىٰ لَهُمْ ۗ إِنَّ ٱللَّهَ خَبِيرٌۢ بِمَا يَصْنَعُونَ ﴿٣٠﴾\\
\textamh{31.\  } & وَقُل لِّلْمُؤْمِنَـٰتِ يَغْضُضْنَ مِنْ أَبْصَـٰرِهِنَّ وَيَحْفَظْنَ فُرُوجَهُنَّ وَلَا يُبْدِينَ زِينَتَهُنَّ إِلَّا مَا ظَهَرَ مِنْهَا ۖ وَلْيَضْرِبْنَ بِخُمُرِهِنَّ عَلَىٰ جُيُوبِهِنَّ ۖ وَلَا يُبْدِينَ زِينَتَهُنَّ إِلَّا لِبُعُولَتِهِنَّ أَوْ ءَابَآئِهِنَّ أَوْ ءَابَآءِ بُعُولَتِهِنَّ أَوْ أَبْنَآئِهِنَّ أَوْ أَبْنَآءِ بُعُولَتِهِنَّ أَوْ إِخْوَٟنِهِنَّ أَوْ بَنِىٓ إِخْوَٟنِهِنَّ أَوْ بَنِىٓ أَخَوَٟتِهِنَّ أَوْ نِسَآئِهِنَّ أَوْ مَا مَلَكَتْ أَيْمَـٰنُهُنَّ أَوِ ٱلتَّٰبِعِينَ غَيْرِ أُو۟لِى ٱلْإِرْبَةِ مِنَ ٱلرِّجَالِ أَوِ ٱلطِّفْلِ ٱلَّذِينَ لَمْ يَظْهَرُوا۟ عَلَىٰ عَوْرَٰتِ ٱلنِّسَآءِ ۖ وَلَا يَضْرِبْنَ بِأَرْجُلِهِنَّ لِيُعْلَمَ مَا يُخْفِينَ مِن زِينَتِهِنَّ ۚ وَتُوبُوٓا۟ إِلَى ٱللَّهِ جَمِيعًا أَيُّهَ ٱلْمُؤْمِنُونَ لَعَلَّكُمْ تُفْلِحُونَ ﴿٣١﴾\\
\textamh{32.\  } & وَأَنكِحُوا۟ ٱلْأَيَـٰمَىٰ مِنكُمْ وَٱلصَّـٰلِحِينَ مِنْ عِبَادِكُمْ وَإِمَآئِكُمْ ۚ إِن يَكُونُوا۟ فُقَرَآءَ يُغْنِهِمُ ٱللَّهُ مِن فَضْلِهِۦ ۗ وَٱللَّهُ وَٟسِعٌ عَلِيمٌۭ ﴿٣٢﴾\\
\textamh{33.\  } & وَلْيَسْتَعْفِفِ ٱلَّذِينَ لَا يَجِدُونَ نِكَاحًا حَتَّىٰ يُغْنِيَهُمُ ٱللَّهُ مِن فَضْلِهِۦ ۗ وَٱلَّذِينَ يَبْتَغُونَ ٱلْكِتَـٰبَ مِمَّا مَلَكَتْ أَيْمَـٰنُكُمْ فَكَاتِبُوهُمْ إِنْ عَلِمْتُمْ فِيهِمْ خَيْرًۭا ۖ وَءَاتُوهُم مِّن مَّالِ ٱللَّهِ ٱلَّذِىٓ ءَاتَىٰكُمْ ۚ وَلَا تُكْرِهُوا۟ فَتَيَـٰتِكُمْ عَلَى ٱلْبِغَآءِ إِنْ أَرَدْنَ تَحَصُّنًۭا لِّتَبْتَغُوا۟ عَرَضَ ٱلْحَيَوٰةِ ٱلدُّنْيَا ۚ وَمَن يُكْرِههُّنَّ فَإِنَّ ٱللَّهَ مِنۢ بَعْدِ إِكْرَٰهِهِنَّ غَفُورٌۭ رَّحِيمٌۭ ﴿٣٣﴾\\
\textamh{34.\  } & وَلَقَدْ أَنزَلْنَآ إِلَيْكُمْ ءَايَـٰتٍۢ مُّبَيِّنَـٰتٍۢ وَمَثَلًۭا مِّنَ ٱلَّذِينَ خَلَوْا۟ مِن قَبْلِكُمْ وَمَوْعِظَةًۭ لِّلْمُتَّقِينَ ﴿٣٤﴾\\
\textamh{35.\  } & ۞ ٱللَّهُ نُورُ ٱلسَّمَـٰوَٟتِ وَٱلْأَرْضِ ۚ مَثَلُ نُورِهِۦ كَمِشْكَوٰةٍۢ فِيهَا مِصْبَاحٌ ۖ ٱلْمِصْبَاحُ فِى زُجَاجَةٍ ۖ ٱلزُّجَاجَةُ كَأَنَّهَا كَوْكَبٌۭ دُرِّىٌّۭ يُوقَدُ مِن شَجَرَةٍۢ مُّبَٰرَكَةٍۢ زَيْتُونَةٍۢ لَّا شَرْقِيَّةٍۢ وَلَا غَرْبِيَّةٍۢ يَكَادُ زَيْتُهَا يُضِىٓءُ وَلَوْ لَمْ تَمْسَسْهُ نَارٌۭ ۚ نُّورٌ عَلَىٰ نُورٍۢ ۗ يَهْدِى ٱللَّهُ لِنُورِهِۦ مَن يَشَآءُ ۚ وَيَضْرِبُ ٱللَّهُ ٱلْأَمْثَـٰلَ لِلنَّاسِ ۗ وَٱللَّهُ بِكُلِّ شَىْءٍ عَلِيمٌۭ ﴿٣٥﴾\\
\textamh{36.\  } & فِى بُيُوتٍ أَذِنَ ٱللَّهُ أَن تُرْفَعَ وَيُذْكَرَ فِيهَا ٱسْمُهُۥ يُسَبِّحُ لَهُۥ فِيهَا بِٱلْغُدُوِّ وَٱلْءَاصَالِ ﴿٣٦﴾\\
\textamh{37.\  } & رِجَالٌۭ لَّا تُلْهِيهِمْ تِجَٰرَةٌۭ وَلَا بَيْعٌ عَن ذِكْرِ ٱللَّهِ وَإِقَامِ ٱلصَّلَوٰةِ وَإِيتَآءِ ٱلزَّكَوٰةِ ۙ يَخَافُونَ يَوْمًۭا تَتَقَلَّبُ فِيهِ ٱلْقُلُوبُ وَٱلْأَبْصَـٰرُ ﴿٣٧﴾\\
\textamh{38.\  } & لِيَجْزِيَهُمُ ٱللَّهُ أَحْسَنَ مَا عَمِلُوا۟ وَيَزِيدَهُم مِّن فَضْلِهِۦ ۗ وَٱللَّهُ يَرْزُقُ مَن يَشَآءُ بِغَيْرِ حِسَابٍۢ ﴿٣٨﴾\\
\textamh{39.\  } & وَٱلَّذِينَ كَفَرُوٓا۟ أَعْمَـٰلُهُمْ كَسَرَابٍۭ بِقِيعَةٍۢ يَحْسَبُهُ ٱلظَّمْـَٔانُ مَآءً حَتَّىٰٓ إِذَا جَآءَهُۥ لَمْ يَجِدْهُ شَيْـًۭٔا وَوَجَدَ ٱللَّهَ عِندَهُۥ فَوَفَّىٰهُ حِسَابَهُۥ ۗ وَٱللَّهُ سَرِيعُ ٱلْحِسَابِ ﴿٣٩﴾\\
\textamh{40.\  } & أَوْ كَظُلُمَـٰتٍۢ فِى بَحْرٍۢ لُّجِّىٍّۢ يَغْشَىٰهُ مَوْجٌۭ مِّن فَوْقِهِۦ مَوْجٌۭ مِّن فَوْقِهِۦ سَحَابٌۭ ۚ ظُلُمَـٰتٌۢ بَعْضُهَا فَوْقَ بَعْضٍ إِذَآ أَخْرَجَ يَدَهُۥ لَمْ يَكَدْ يَرَىٰهَا ۗ وَمَن لَّمْ يَجْعَلِ ٱللَّهُ لَهُۥ نُورًۭا فَمَا لَهُۥ مِن نُّورٍ ﴿٤٠﴾\\
\textamh{41.\  } & أَلَمْ تَرَ أَنَّ ٱللَّهَ يُسَبِّحُ لَهُۥ مَن فِى ٱلسَّمَـٰوَٟتِ وَٱلْأَرْضِ وَٱلطَّيْرُ صَـٰٓفَّٰتٍۢ ۖ كُلٌّۭ قَدْ عَلِمَ صَلَاتَهُۥ وَتَسْبِيحَهُۥ ۗ وَٱللَّهُ عَلِيمٌۢ بِمَا يَفْعَلُونَ ﴿٤١﴾\\
\textamh{42.\  } & وَلِلَّهِ مُلْكُ ٱلسَّمَـٰوَٟتِ وَٱلْأَرْضِ ۖ وَإِلَى ٱللَّهِ ٱلْمَصِيرُ ﴿٤٢﴾\\
\textamh{43.\  } & أَلَمْ تَرَ أَنَّ ٱللَّهَ يُزْجِى سَحَابًۭا ثُمَّ يُؤَلِّفُ بَيْنَهُۥ ثُمَّ يَجْعَلُهُۥ رُكَامًۭا فَتَرَى ٱلْوَدْقَ يَخْرُجُ مِنْ خِلَـٰلِهِۦ وَيُنَزِّلُ مِنَ ٱلسَّمَآءِ مِن جِبَالٍۢ فِيهَا مِنۢ بَرَدٍۢ فَيُصِيبُ بِهِۦ مَن يَشَآءُ وَيَصْرِفُهُۥ عَن مَّن يَشَآءُ ۖ يَكَادُ سَنَا بَرْقِهِۦ يَذْهَبُ بِٱلْأَبْصَـٰرِ ﴿٤٣﴾\\
\textamh{44.\  } & يُقَلِّبُ ٱللَّهُ ٱلَّيْلَ وَٱلنَّهَارَ ۚ إِنَّ فِى ذَٟلِكَ لَعِبْرَةًۭ لِّأُو۟لِى ٱلْأَبْصَـٰرِ ﴿٤٤﴾\\
\textamh{45.\  } & وَٱللَّهُ خَلَقَ كُلَّ دَآبَّةٍۢ مِّن مَّآءٍۢ ۖ فَمِنْهُم مَّن يَمْشِى عَلَىٰ بَطْنِهِۦ وَمِنْهُم مَّن يَمْشِى عَلَىٰ رِجْلَيْنِ وَمِنْهُم مَّن يَمْشِى عَلَىٰٓ أَرْبَعٍۢ ۚ يَخْلُقُ ٱللَّهُ مَا يَشَآءُ ۚ إِنَّ ٱللَّهَ عَلَىٰ كُلِّ شَىْءٍۢ قَدِيرٌۭ ﴿٤٥﴾\\
\textamh{46.\  } & لَّقَدْ أَنزَلْنَآ ءَايَـٰتٍۢ مُّبَيِّنَـٰتٍۢ ۚ وَٱللَّهُ يَهْدِى مَن يَشَآءُ إِلَىٰ صِرَٰطٍۢ مُّسْتَقِيمٍۢ ﴿٤٦﴾\\
\textamh{47.\  } & وَيَقُولُونَ ءَامَنَّا بِٱللَّهِ وَبِٱلرَّسُولِ وَأَطَعْنَا ثُمَّ يَتَوَلَّىٰ فَرِيقٌۭ مِّنْهُم مِّنۢ بَعْدِ ذَٟلِكَ ۚ وَمَآ أُو۟لَـٰٓئِكَ بِٱلْمُؤْمِنِينَ ﴿٤٧﴾\\
\textamh{48.\  } & وَإِذَا دُعُوٓا۟ إِلَى ٱللَّهِ وَرَسُولِهِۦ لِيَحْكُمَ بَيْنَهُمْ إِذَا فَرِيقٌۭ مِّنْهُم مُّعْرِضُونَ ﴿٤٨﴾\\
\textamh{49.\  } & وَإِن يَكُن لَّهُمُ ٱلْحَقُّ يَأْتُوٓا۟ إِلَيْهِ مُذْعِنِينَ ﴿٤٩﴾\\
\textamh{50.\  } & أَفِى قُلُوبِهِم مَّرَضٌ أَمِ ٱرْتَابُوٓا۟ أَمْ يَخَافُونَ أَن يَحِيفَ ٱللَّهُ عَلَيْهِمْ وَرَسُولُهُۥ ۚ بَلْ أُو۟لَـٰٓئِكَ هُمُ ٱلظَّـٰلِمُونَ ﴿٥٠﴾\\
\textamh{51.\  } & إِنَّمَا كَانَ قَوْلَ ٱلْمُؤْمِنِينَ إِذَا دُعُوٓا۟ إِلَى ٱللَّهِ وَرَسُولِهِۦ لِيَحْكُمَ بَيْنَهُمْ أَن يَقُولُوا۟ سَمِعْنَا وَأَطَعْنَا ۚ وَأُو۟لَـٰٓئِكَ هُمُ ٱلْمُفْلِحُونَ ﴿٥١﴾\\
\textamh{52.\  } & وَمَن يُطِعِ ٱللَّهَ وَرَسُولَهُۥ وَيَخْشَ ٱللَّهَ وَيَتَّقْهِ فَأُو۟لَـٰٓئِكَ هُمُ ٱلْفَآئِزُونَ ﴿٥٢﴾\\
\textamh{53.\  } & ۞ وَأَقْسَمُوا۟ بِٱللَّهِ جَهْدَ أَيْمَـٰنِهِمْ لَئِنْ أَمَرْتَهُمْ لَيَخْرُجُنَّ ۖ قُل لَّا تُقْسِمُوا۟ ۖ طَاعَةٌۭ مَّعْرُوفَةٌ ۚ إِنَّ ٱللَّهَ خَبِيرٌۢ بِمَا تَعْمَلُونَ ﴿٥٣﴾\\
\textamh{54.\  } & قُلْ أَطِيعُوا۟ ٱللَّهَ وَأَطِيعُوا۟ ٱلرَّسُولَ ۖ فَإِن تَوَلَّوْا۟ فَإِنَّمَا عَلَيْهِ مَا حُمِّلَ وَعَلَيْكُم مَّا حُمِّلْتُمْ ۖ وَإِن تُطِيعُوهُ تَهْتَدُوا۟ ۚ وَمَا عَلَى ٱلرَّسُولِ إِلَّا ٱلْبَلَـٰغُ ٱلْمُبِينُ ﴿٥٤﴾\\
\textamh{55.\  } & وَعَدَ ٱللَّهُ ٱلَّذِينَ ءَامَنُوا۟ مِنكُمْ وَعَمِلُوا۟ ٱلصَّـٰلِحَـٰتِ لَيَسْتَخْلِفَنَّهُمْ فِى ٱلْأَرْضِ كَمَا ٱسْتَخْلَفَ ٱلَّذِينَ مِن قَبْلِهِمْ وَلَيُمَكِّنَنَّ لَهُمْ دِينَهُمُ ٱلَّذِى ٱرْتَضَىٰ لَهُمْ وَلَيُبَدِّلَنَّهُم مِّنۢ بَعْدِ خَوْفِهِمْ أَمْنًۭا ۚ يَعْبُدُونَنِى لَا يُشْرِكُونَ بِى شَيْـًۭٔا ۚ وَمَن كَفَرَ بَعْدَ ذَٟلِكَ فَأُو۟لَـٰٓئِكَ هُمُ ٱلْفَـٰسِقُونَ ﴿٥٥﴾\\
\textamh{56.\  } & وَأَقِيمُوا۟ ٱلصَّلَوٰةَ وَءَاتُوا۟ ٱلزَّكَوٰةَ وَأَطِيعُوا۟ ٱلرَّسُولَ لَعَلَّكُمْ تُرْحَمُونَ ﴿٥٦﴾\\
\textamh{57.\  } & لَا تَحْسَبَنَّ ٱلَّذِينَ كَفَرُوا۟ مُعْجِزِينَ فِى ٱلْأَرْضِ ۚ وَمَأْوَىٰهُمُ ٱلنَّارُ ۖ وَلَبِئْسَ ٱلْمَصِيرُ ﴿٥٧﴾\\
\textamh{58.\  } & يَـٰٓأَيُّهَا ٱلَّذِينَ ءَامَنُوا۟ لِيَسْتَـْٔذِنكُمُ ٱلَّذِينَ مَلَكَتْ أَيْمَـٰنُكُمْ وَٱلَّذِينَ لَمْ يَبْلُغُوا۟ ٱلْحُلُمَ مِنكُمْ ثَلَـٰثَ مَرَّٟتٍۢ ۚ مِّن قَبْلِ صَلَوٰةِ ٱلْفَجْرِ وَحِينَ تَضَعُونَ ثِيَابَكُم مِّنَ ٱلظَّهِيرَةِ وَمِنۢ بَعْدِ صَلَوٰةِ ٱلْعِشَآءِ ۚ ثَلَـٰثُ عَوْرَٰتٍۢ لَّكُمْ ۚ لَيْسَ عَلَيْكُمْ وَلَا عَلَيْهِمْ جُنَاحٌۢ بَعْدَهُنَّ ۚ طَوَّٰفُونَ عَلَيْكُم بَعْضُكُمْ عَلَىٰ بَعْضٍۢ ۚ كَذَٟلِكَ يُبَيِّنُ ٱللَّهُ لَكُمُ ٱلْءَايَـٰتِ ۗ وَٱللَّهُ عَلِيمٌ حَكِيمٌۭ ﴿٥٨﴾\\
\textamh{59.\  } & وَإِذَا بَلَغَ ٱلْأَطْفَـٰلُ مِنكُمُ ٱلْحُلُمَ فَلْيَسْتَـْٔذِنُوا۟ كَمَا ٱسْتَـْٔذَنَ ٱلَّذِينَ مِن قَبْلِهِمْ ۚ كَذَٟلِكَ يُبَيِّنُ ٱللَّهُ لَكُمْ ءَايَـٰتِهِۦ ۗ وَٱللَّهُ عَلِيمٌ حَكِيمٌۭ ﴿٥٩﴾\\
\textamh{60.\  } & وَٱلْقَوَٟعِدُ مِنَ ٱلنِّسَآءِ ٱلَّٰتِى لَا يَرْجُونَ نِكَاحًۭا فَلَيْسَ عَلَيْهِنَّ جُنَاحٌ أَن يَضَعْنَ ثِيَابَهُنَّ غَيْرَ مُتَبَرِّجَٰتٍۭ بِزِينَةٍۢ ۖ وَأَن يَسْتَعْفِفْنَ خَيْرٌۭ لَّهُنَّ ۗ وَٱللَّهُ سَمِيعٌ عَلِيمٌۭ ﴿٦٠﴾\\
\textamh{61.\  } & لَّيْسَ عَلَى ٱلْأَعْمَىٰ حَرَجٌۭ وَلَا عَلَى ٱلْأَعْرَجِ حَرَجٌۭ وَلَا عَلَى ٱلْمَرِيضِ حَرَجٌۭ وَلَا عَلَىٰٓ أَنفُسِكُمْ أَن تَأْكُلُوا۟ مِنۢ بُيُوتِكُمْ أَوْ بُيُوتِ ءَابَآئِكُمْ أَوْ بُيُوتِ أُمَّهَـٰتِكُمْ أَوْ بُيُوتِ إِخْوَٟنِكُمْ أَوْ بُيُوتِ أَخَوَٟتِكُمْ أَوْ بُيُوتِ أَعْمَـٰمِكُمْ أَوْ بُيُوتِ عَمَّٰتِكُمْ أَوْ بُيُوتِ أَخْوَٟلِكُمْ أَوْ بُيُوتِ خَـٰلَـٰتِكُمْ أَوْ مَا مَلَكْتُم مَّفَاتِحَهُۥٓ أَوْ صَدِيقِكُمْ ۚ لَيْسَ عَلَيْكُمْ جُنَاحٌ أَن تَأْكُلُوا۟ جَمِيعًا أَوْ أَشْتَاتًۭا ۚ فَإِذَا دَخَلْتُم بُيُوتًۭا فَسَلِّمُوا۟ عَلَىٰٓ أَنفُسِكُمْ تَحِيَّةًۭ مِّنْ عِندِ ٱللَّهِ مُبَٰرَكَةًۭ طَيِّبَةًۭ ۚ كَذَٟلِكَ يُبَيِّنُ ٱللَّهُ لَكُمُ ٱلْءَايَـٰتِ لَعَلَّكُمْ تَعْقِلُونَ ﴿٦١﴾\\
\textamh{62.\  } & إِنَّمَا ٱلْمُؤْمِنُونَ ٱلَّذِينَ ءَامَنُوا۟ بِٱللَّهِ وَرَسُولِهِۦ وَإِذَا كَانُوا۟ مَعَهُۥ عَلَىٰٓ أَمْرٍۢ جَامِعٍۢ لَّمْ يَذْهَبُوا۟ حَتَّىٰ يَسْتَـْٔذِنُوهُ ۚ إِنَّ ٱلَّذِينَ يَسْتَـْٔذِنُونَكَ أُو۟لَـٰٓئِكَ ٱلَّذِينَ يُؤْمِنُونَ بِٱللَّهِ وَرَسُولِهِۦ ۚ فَإِذَا ٱسْتَـْٔذَنُوكَ لِبَعْضِ شَأْنِهِمْ فَأْذَن لِّمَن شِئْتَ مِنْهُمْ وَٱسْتَغْفِرْ لَهُمُ ٱللَّهَ ۚ إِنَّ ٱللَّهَ غَفُورٌۭ رَّحِيمٌۭ ﴿٦٢﴾\\
\textamh{63.\  } & لَّا تَجْعَلُوا۟ دُعَآءَ ٱلرَّسُولِ بَيْنَكُمْ كَدُعَآءِ بَعْضِكُم بَعْضًۭا ۚ قَدْ يَعْلَمُ ٱللَّهُ ٱلَّذِينَ يَتَسَلَّلُونَ مِنكُمْ لِوَاذًۭا ۚ فَلْيَحْذَرِ ٱلَّذِينَ يُخَالِفُونَ عَنْ أَمْرِهِۦٓ أَن تُصِيبَهُمْ فِتْنَةٌ أَوْ يُصِيبَهُمْ عَذَابٌ أَلِيمٌ ﴿٦٣﴾\\
\textamh{64.\  } & أَلَآ إِنَّ لِلَّهِ مَا فِى ٱلسَّمَـٰوَٟتِ وَٱلْأَرْضِ ۖ قَدْ يَعْلَمُ مَآ أَنتُمْ عَلَيْهِ وَيَوْمَ يُرْجَعُونَ إِلَيْهِ فَيُنَبِّئُهُم بِمَا عَمِلُوا۟ ۗ وَٱللَّهُ بِكُلِّ شَىْءٍ عَلِيمٌۢ ﴿٦٤﴾\\
\end{longtable}
\clearpage
%% License: BSD style (Berkley) (i.e. Put the Copyright owner's name always)
%% Writer and Copyright (to): Bewketu(Bilal) Tadilo (2016-17)
\centering\section{\LR{\textamharic{ሱራቱ አልፉርቃን -}  \RL{سوره  الفرقان}}}
\begin{longtable}{%
  @{}
    p{.5\textwidth}
  @{~~~~~~~~~~~~}
    p{.5\textwidth}
    @{}
}
\nopagebreak
\textamh{ቢስሚላሂ አራህመኒ ራሂይም } &  بِسْمِ ٱللَّهِ ٱلرَّحْمَـٰنِ ٱلرَّحِيمِ\\
\textamh{1.\  } &  تَبَارَكَ ٱلَّذِى نَزَّلَ ٱلْفُرْقَانَ عَلَىٰ عَبْدِهِۦ لِيَكُونَ لِلْعَـٰلَمِينَ نَذِيرًا ﴿١﴾\\
\textamh{2.\  } & ٱلَّذِى لَهُۥ مُلْكُ ٱلسَّمَـٰوَٟتِ وَٱلْأَرْضِ وَلَمْ يَتَّخِذْ وَلَدًۭا وَلَمْ يَكُن لَّهُۥ شَرِيكٌۭ فِى ٱلْمُلْكِ وَخَلَقَ كُلَّ شَىْءٍۢ فَقَدَّرَهُۥ تَقْدِيرًۭا ﴿٢﴾\\
\textamh{3.\  } & وَٱتَّخَذُوا۟ مِن دُونِهِۦٓ ءَالِهَةًۭ لَّا يَخْلُقُونَ شَيْـًۭٔا وَهُمْ يُخْلَقُونَ وَلَا يَمْلِكُونَ لِأَنفُسِهِمْ ضَرًّۭا وَلَا نَفْعًۭا وَلَا يَمْلِكُونَ مَوْتًۭا وَلَا حَيَوٰةًۭ وَلَا نُشُورًۭا ﴿٣﴾\\
\textamh{4.\  } & وَقَالَ ٱلَّذِينَ كَفَرُوٓا۟ إِنْ هَـٰذَآ إِلَّآ إِفْكٌ ٱفْتَرَىٰهُ وَأَعَانَهُۥ عَلَيْهِ قَوْمٌ ءَاخَرُونَ ۖ فَقَدْ جَآءُو ظُلْمًۭا وَزُورًۭا ﴿٤﴾\\
\textamh{5.\  } & وَقَالُوٓا۟ أَسَـٰطِيرُ ٱلْأَوَّلِينَ ٱكْتَتَبَهَا فَهِىَ تُمْلَىٰ عَلَيْهِ بُكْرَةًۭ وَأَصِيلًۭا ﴿٥﴾\\
\textamh{6.\  } & قُلْ أَنزَلَهُ ٱلَّذِى يَعْلَمُ ٱلسِّرَّ فِى ٱلسَّمَـٰوَٟتِ وَٱلْأَرْضِ ۚ إِنَّهُۥ كَانَ غَفُورًۭا رَّحِيمًۭا ﴿٦﴾\\
\textamh{7.\  } & وَقَالُوا۟ مَالِ هَـٰذَا ٱلرَّسُولِ يَأْكُلُ ٱلطَّعَامَ وَيَمْشِى فِى ٱلْأَسْوَاقِ ۙ لَوْلَآ أُنزِلَ إِلَيْهِ مَلَكٌۭ فَيَكُونَ مَعَهُۥ نَذِيرًا ﴿٧﴾\\
\textamh{8.\  } & أَوْ يُلْقَىٰٓ إِلَيْهِ كَنزٌ أَوْ تَكُونُ لَهُۥ جَنَّةٌۭ يَأْكُلُ مِنْهَا ۚ وَقَالَ ٱلظَّـٰلِمُونَ إِن تَتَّبِعُونَ إِلَّا رَجُلًۭا مَّسْحُورًا ﴿٨﴾\\
\textamh{9.\  } & ٱنظُرْ كَيْفَ ضَرَبُوا۟ لَكَ ٱلْأَمْثَـٰلَ فَضَلُّوا۟ فَلَا يَسْتَطِيعُونَ سَبِيلًۭا ﴿٩﴾\\
\textamh{10.\  } & تَبَارَكَ ٱلَّذِىٓ إِن شَآءَ جَعَلَ لَكَ خَيْرًۭا مِّن ذَٟلِكَ جَنَّـٰتٍۢ تَجْرِى مِن تَحْتِهَا ٱلْأَنْهَـٰرُ وَيَجْعَل لَّكَ قُصُورًۢا ﴿١٠﴾\\
\textamh{11.\  } & بَلْ كَذَّبُوا۟ بِٱلسَّاعَةِ ۖ وَأَعْتَدْنَا لِمَن كَذَّبَ بِٱلسَّاعَةِ سَعِيرًا ﴿١١﴾\\
\textamh{12.\  } & إِذَا رَأَتْهُم مِّن مَّكَانٍۭ بَعِيدٍۢ سَمِعُوا۟ لَهَا تَغَيُّظًۭا وَزَفِيرًۭا ﴿١٢﴾\\
\textamh{13.\  } & وَإِذَآ أُلْقُوا۟ مِنْهَا مَكَانًۭا ضَيِّقًۭا مُّقَرَّنِينَ دَعَوْا۟ هُنَالِكَ ثُبُورًۭا ﴿١٣﴾\\
\textamh{14.\  } & لَّا تَدْعُوا۟ ٱلْيَوْمَ ثُبُورًۭا وَٟحِدًۭا وَٱدْعُوا۟ ثُبُورًۭا كَثِيرًۭا ﴿١٤﴾\\
\textamh{15.\  } & قُلْ أَذَٟلِكَ خَيْرٌ أَمْ جَنَّةُ ٱلْخُلْدِ ٱلَّتِى وُعِدَ ٱلْمُتَّقُونَ ۚ كَانَتْ لَهُمْ جَزَآءًۭ وَمَصِيرًۭا ﴿١٥﴾\\
\textamh{16.\  } & لَّهُمْ فِيهَا مَا يَشَآءُونَ خَـٰلِدِينَ ۚ كَانَ عَلَىٰ رَبِّكَ وَعْدًۭا مَّسْـُٔولًۭا ﴿١٦﴾\\
\textamh{17.\  } & وَيَوْمَ يَحْشُرُهُمْ وَمَا يَعْبُدُونَ مِن دُونِ ٱللَّهِ فَيَقُولُ ءَأَنتُمْ أَضْلَلْتُمْ عِبَادِى هَـٰٓؤُلَآءِ أَمْ هُمْ ضَلُّوا۟ ٱلسَّبِيلَ ﴿١٧﴾\\
\textamh{18.\  } & قَالُوا۟ سُبْحَـٰنَكَ مَا كَانَ يَنۢبَغِى لَنَآ أَن نَّتَّخِذَ مِن دُونِكَ مِنْ أَوْلِيَآءَ وَلَـٰكِن مَّتَّعْتَهُمْ وَءَابَآءَهُمْ حَتَّىٰ نَسُوا۟ ٱلذِّكْرَ وَكَانُوا۟ قَوْمًۢا بُورًۭا ﴿١٨﴾\\
\textamh{19.\  } & فَقَدْ كَذَّبُوكُم بِمَا تَقُولُونَ فَمَا تَسْتَطِيعُونَ صَرْفًۭا وَلَا نَصْرًۭا ۚ وَمَن يَظْلِم مِّنكُمْ نُذِقْهُ عَذَابًۭا كَبِيرًۭا ﴿١٩﴾\\
\textamh{20.\  } & وَمَآ أَرْسَلْنَا قَبْلَكَ مِنَ ٱلْمُرْسَلِينَ إِلَّآ إِنَّهُمْ لَيَأْكُلُونَ ٱلطَّعَامَ وَيَمْشُونَ فِى ٱلْأَسْوَاقِ ۗ وَجَعَلْنَا بَعْضَكُمْ لِبَعْضٍۢ فِتْنَةً أَتَصْبِرُونَ ۗ وَكَانَ رَبُّكَ بَصِيرًۭا ﴿٢٠﴾\\
\textamh{21.\  } & ۞ وَقَالَ ٱلَّذِينَ لَا يَرْجُونَ لِقَآءَنَا لَوْلَآ أُنزِلَ عَلَيْنَا ٱلْمَلَـٰٓئِكَةُ أَوْ نَرَىٰ رَبَّنَا ۗ لَقَدِ ٱسْتَكْبَرُوا۟ فِىٓ أَنفُسِهِمْ وَعَتَوْ عُتُوًّۭا كَبِيرًۭا ﴿٢١﴾\\
\textamh{22.\  } & يَوْمَ يَرَوْنَ ٱلْمَلَـٰٓئِكَةَ لَا بُشْرَىٰ يَوْمَئِذٍۢ لِّلْمُجْرِمِينَ وَيَقُولُونَ حِجْرًۭا مَّحْجُورًۭا ﴿٢٢﴾\\
\textamh{23.\  } & وَقَدِمْنَآ إِلَىٰ مَا عَمِلُوا۟ مِنْ عَمَلٍۢ فَجَعَلْنَـٰهُ هَبَآءًۭ مَّنثُورًا ﴿٢٣﴾\\
\textamh{24.\  } & أَصْحَـٰبُ ٱلْجَنَّةِ يَوْمَئِذٍ خَيْرٌۭ مُّسْتَقَرًّۭا وَأَحْسَنُ مَقِيلًۭا ﴿٢٤﴾\\
\textamh{25.\  } & وَيَوْمَ تَشَقَّقُ ٱلسَّمَآءُ بِٱلْغَمَـٰمِ وَنُزِّلَ ٱلْمَلَـٰٓئِكَةُ تَنزِيلًا ﴿٢٥﴾\\
\textamh{26.\  } & ٱلْمُلْكُ يَوْمَئِذٍ ٱلْحَقُّ لِلرَّحْمَـٰنِ ۚ وَكَانَ يَوْمًا عَلَى ٱلْكَـٰفِرِينَ عَسِيرًۭا ﴿٢٦﴾\\
\textamh{27.\  } & وَيَوْمَ يَعَضُّ ٱلظَّالِمُ عَلَىٰ يَدَيْهِ يَقُولُ يَـٰلَيْتَنِى ٱتَّخَذْتُ مَعَ ٱلرَّسُولِ سَبِيلًۭا ﴿٢٧﴾\\
\textamh{28.\  } & يَـٰوَيْلَتَىٰ لَيْتَنِى لَمْ أَتَّخِذْ فُلَانًا خَلِيلًۭا ﴿٢٨﴾\\
\textamh{29.\  } & لَّقَدْ أَضَلَّنِى عَنِ ٱلذِّكْرِ بَعْدَ إِذْ جَآءَنِى ۗ وَكَانَ ٱلشَّيْطَٰنُ لِلْإِنسَـٰنِ خَذُولًۭا ﴿٢٩﴾\\
\textamh{30.\  } & وَقَالَ ٱلرَّسُولُ يَـٰرَبِّ إِنَّ قَوْمِى ٱتَّخَذُوا۟ هَـٰذَا ٱلْقُرْءَانَ مَهْجُورًۭا ﴿٣٠﴾\\
\textamh{31.\  } & وَكَذَٟلِكَ جَعَلْنَا لِكُلِّ نَبِىٍّ عَدُوًّۭا مِّنَ ٱلْمُجْرِمِينَ ۗ وَكَفَىٰ بِرَبِّكَ هَادِيًۭا وَنَصِيرًۭا ﴿٣١﴾\\
\textamh{32.\  } & وَقَالَ ٱلَّذِينَ كَفَرُوا۟ لَوْلَا نُزِّلَ عَلَيْهِ ٱلْقُرْءَانُ جُمْلَةًۭ وَٟحِدَةًۭ ۚ كَذَٟلِكَ لِنُثَبِّتَ بِهِۦ فُؤَادَكَ ۖ وَرَتَّلْنَـٰهُ تَرْتِيلًۭا ﴿٣٢﴾\\
\textamh{33.\  } & وَلَا يَأْتُونَكَ بِمَثَلٍ إِلَّا جِئْنَـٰكَ بِٱلْحَقِّ وَأَحْسَنَ تَفْسِيرًا ﴿٣٣﴾\\
\textamh{34.\  } & ٱلَّذِينَ يُحْشَرُونَ عَلَىٰ وُجُوهِهِمْ إِلَىٰ جَهَنَّمَ أُو۟لَـٰٓئِكَ شَرٌّۭ مَّكَانًۭا وَأَضَلُّ سَبِيلًۭا ﴿٣٤﴾\\
\textamh{35.\  } & وَلَقَدْ ءَاتَيْنَا مُوسَى ٱلْكِتَـٰبَ وَجَعَلْنَا مَعَهُۥٓ أَخَاهُ هَـٰرُونَ وَزِيرًۭا ﴿٣٥﴾\\
\textamh{36.\  } & فَقُلْنَا ٱذْهَبَآ إِلَى ٱلْقَوْمِ ٱلَّذِينَ كَذَّبُوا۟ بِـَٔايَـٰتِنَا فَدَمَّرْنَـٰهُمْ تَدْمِيرًۭا ﴿٣٦﴾\\
\textamh{37.\  } & وَقَوْمَ نُوحٍۢ لَّمَّا كَذَّبُوا۟ ٱلرُّسُلَ أَغْرَقْنَـٰهُمْ وَجَعَلْنَـٰهُمْ لِلنَّاسِ ءَايَةًۭ ۖ وَأَعْتَدْنَا لِلظَّـٰلِمِينَ عَذَابًا أَلِيمًۭا ﴿٣٧﴾\\
\textamh{38.\  } & وَعَادًۭا وَثَمُودَا۟ وَأَصْحَـٰبَ ٱلرَّسِّ وَقُرُونًۢا بَيْنَ ذَٟلِكَ كَثِيرًۭا ﴿٣٨﴾\\
\textamh{39.\  } & وَكُلًّۭا ضَرَبْنَا لَهُ ٱلْأَمْثَـٰلَ ۖ وَكُلًّۭا تَبَّرْنَا تَتْبِيرًۭا ﴿٣٩﴾\\
\textamh{40.\  } & وَلَقَدْ أَتَوْا۟ عَلَى ٱلْقَرْيَةِ ٱلَّتِىٓ أُمْطِرَتْ مَطَرَ ٱلسَّوْءِ ۚ أَفَلَمْ يَكُونُوا۟ يَرَوْنَهَا ۚ بَلْ كَانُوا۟ لَا يَرْجُونَ نُشُورًۭا ﴿٤٠﴾\\
\textamh{41.\  } & وَإِذَا رَأَوْكَ إِن يَتَّخِذُونَكَ إِلَّا هُزُوًا أَهَـٰذَا ٱلَّذِى بَعَثَ ٱللَّهُ رَسُولًا ﴿٤١﴾\\
\textamh{42.\  } & إِن كَادَ لَيُضِلُّنَا عَنْ ءَالِهَتِنَا لَوْلَآ أَن صَبَرْنَا عَلَيْهَا ۚ وَسَوْفَ يَعْلَمُونَ حِينَ يَرَوْنَ ٱلْعَذَابَ مَنْ أَضَلُّ سَبِيلًا ﴿٤٢﴾\\
\textamh{43.\  } & أَرَءَيْتَ مَنِ ٱتَّخَذَ إِلَـٰهَهُۥ هَوَىٰهُ أَفَأَنتَ تَكُونُ عَلَيْهِ وَكِيلًا ﴿٤٣﴾\\
\textamh{44.\  } & أَمْ تَحْسَبُ أَنَّ أَكْثَرَهُمْ يَسْمَعُونَ أَوْ يَعْقِلُونَ ۚ إِنْ هُمْ إِلَّا كَٱلْأَنْعَـٰمِ ۖ بَلْ هُمْ أَضَلُّ سَبِيلًا ﴿٤٤﴾\\
\textamh{45.\  } & أَلَمْ تَرَ إِلَىٰ رَبِّكَ كَيْفَ مَدَّ ٱلظِّلَّ وَلَوْ شَآءَ لَجَعَلَهُۥ سَاكِنًۭا ثُمَّ جَعَلْنَا ٱلشَّمْسَ عَلَيْهِ دَلِيلًۭا ﴿٤٥﴾\\
\textamh{46.\  } & ثُمَّ قَبَضْنَـٰهُ إِلَيْنَا قَبْضًۭا يَسِيرًۭا ﴿٤٦﴾\\
\textamh{47.\  } & وَهُوَ ٱلَّذِى جَعَلَ لَكُمُ ٱلَّيْلَ لِبَاسًۭا وَٱلنَّوْمَ سُبَاتًۭا وَجَعَلَ ٱلنَّهَارَ نُشُورًۭا ﴿٤٧﴾\\
\textamh{48.\  } & وَهُوَ ٱلَّذِىٓ أَرْسَلَ ٱلرِّيَـٰحَ بُشْرًۢا بَيْنَ يَدَىْ رَحْمَتِهِۦ ۚ وَأَنزَلْنَا مِنَ ٱلسَّمَآءِ مَآءًۭ طَهُورًۭا ﴿٤٨﴾\\
\textamh{49.\  } & لِّنُحْۦِىَ بِهِۦ بَلْدَةًۭ مَّيْتًۭا وَنُسْقِيَهُۥ مِمَّا خَلَقْنَآ أَنْعَـٰمًۭا وَأَنَاسِىَّ كَثِيرًۭا ﴿٤٩﴾\\
\textamh{50.\  } & وَلَقَدْ صَرَّفْنَـٰهُ بَيْنَهُمْ لِيَذَّكَّرُوا۟ فَأَبَىٰٓ أَكْثَرُ ٱلنَّاسِ إِلَّا كُفُورًۭا ﴿٥٠﴾\\
\textamh{51.\  } & وَلَوْ شِئْنَا لَبَعَثْنَا فِى كُلِّ قَرْيَةٍۢ نَّذِيرًۭا ﴿٥١﴾\\
\textamh{52.\  } & فَلَا تُطِعِ ٱلْكَـٰفِرِينَ وَجَٰهِدْهُم بِهِۦ جِهَادًۭا كَبِيرًۭا ﴿٥٢﴾\\
\textamh{53.\  } & ۞ وَهُوَ ٱلَّذِى مَرَجَ ٱلْبَحْرَيْنِ هَـٰذَا عَذْبٌۭ فُرَاتٌۭ وَهَـٰذَا مِلْحٌ أُجَاجٌۭ وَجَعَلَ بَيْنَهُمَا بَرْزَخًۭا وَحِجْرًۭا مَّحْجُورًۭا ﴿٥٣﴾\\
\textamh{54.\  } & وَهُوَ ٱلَّذِى خَلَقَ مِنَ ٱلْمَآءِ بَشَرًۭا فَجَعَلَهُۥ نَسَبًۭا وَصِهْرًۭا ۗ وَكَانَ رَبُّكَ قَدِيرًۭا ﴿٥٤﴾\\
\textamh{55.\  } & وَيَعْبُدُونَ مِن دُونِ ٱللَّهِ مَا لَا يَنفَعُهُمْ وَلَا يَضُرُّهُمْ ۗ وَكَانَ ٱلْكَافِرُ عَلَىٰ رَبِّهِۦ ظَهِيرًۭا ﴿٥٥﴾\\
\textamh{56.\  } & وَمَآ أَرْسَلْنَـٰكَ إِلَّا مُبَشِّرًۭا وَنَذِيرًۭا ﴿٥٦﴾\\
\textamh{57.\  } & قُلْ مَآ أَسْـَٔلُكُمْ عَلَيْهِ مِنْ أَجْرٍ إِلَّا مَن شَآءَ أَن يَتَّخِذَ إِلَىٰ رَبِّهِۦ سَبِيلًۭا ﴿٥٧﴾\\
\textamh{58.\  } & وَتَوَكَّلْ عَلَى ٱلْحَىِّ ٱلَّذِى لَا يَمُوتُ وَسَبِّحْ بِحَمْدِهِۦ ۚ وَكَفَىٰ بِهِۦ بِذُنُوبِ عِبَادِهِۦ خَبِيرًا ﴿٥٨﴾\\
\textamh{59.\  } & ٱلَّذِى خَلَقَ ٱلسَّمَـٰوَٟتِ وَٱلْأَرْضَ وَمَا بَيْنَهُمَا فِى سِتَّةِ أَيَّامٍۢ ثُمَّ ٱسْتَوَىٰ عَلَى ٱلْعَرْشِ ۚ ٱلرَّحْمَـٰنُ فَسْـَٔلْ بِهِۦ خَبِيرًۭا ﴿٥٩﴾\\
\textamh{60.\  } & وَإِذَا قِيلَ لَهُمُ ٱسْجُدُوا۟ لِلرَّحْمَـٰنِ قَالُوا۟ وَمَا ٱلرَّحْمَـٰنُ أَنَسْجُدُ لِمَا تَأْمُرُنَا وَزَادَهُمْ نُفُورًۭا ۩ ﴿٦٠﴾\\
\textamh{61.\  } & تَبَارَكَ ٱلَّذِى جَعَلَ فِى ٱلسَّمَآءِ بُرُوجًۭا وَجَعَلَ فِيهَا سِرَٰجًۭا وَقَمَرًۭا مُّنِيرًۭا ﴿٦١﴾\\
\textamh{62.\  } & وَهُوَ ٱلَّذِى جَعَلَ ٱلَّيْلَ وَٱلنَّهَارَ خِلْفَةًۭ لِّمَنْ أَرَادَ أَن يَذَّكَّرَ أَوْ أَرَادَ شُكُورًۭا ﴿٦٢﴾\\
\textamh{63.\  } & وَعِبَادُ ٱلرَّحْمَـٰنِ ٱلَّذِينَ يَمْشُونَ عَلَى ٱلْأَرْضِ هَوْنًۭا وَإِذَا خَاطَبَهُمُ ٱلْجَٰهِلُونَ قَالُوا۟ سَلَـٰمًۭا ﴿٦٣﴾\\
\textamh{64.\  } & وَٱلَّذِينَ يَبِيتُونَ لِرَبِّهِمْ سُجَّدًۭا وَقِيَـٰمًۭا ﴿٦٤﴾\\
\textamh{65.\  } & وَٱلَّذِينَ يَقُولُونَ رَبَّنَا ٱصْرِفْ عَنَّا عَذَابَ جَهَنَّمَ ۖ إِنَّ عَذَابَهَا كَانَ غَرَامًا ﴿٦٥﴾\\
\textamh{66.\  } & إِنَّهَا سَآءَتْ مُسْتَقَرًّۭا وَمُقَامًۭا ﴿٦٦﴾\\
\textamh{67.\  } & وَٱلَّذِينَ إِذَآ أَنفَقُوا۟ لَمْ يُسْرِفُوا۟ وَلَمْ يَقْتُرُوا۟ وَكَانَ بَيْنَ ذَٟلِكَ قَوَامًۭا ﴿٦٧﴾\\
\textamh{68.\  } & وَٱلَّذِينَ لَا يَدْعُونَ مَعَ ٱللَّهِ إِلَـٰهًا ءَاخَرَ وَلَا يَقْتُلُونَ ٱلنَّفْسَ ٱلَّتِى حَرَّمَ ٱللَّهُ إِلَّا بِٱلْحَقِّ وَلَا يَزْنُونَ ۚ وَمَن يَفْعَلْ ذَٟلِكَ يَلْقَ أَثَامًۭا ﴿٦٨﴾\\
\textamh{69.\  } & يُضَٰعَفْ لَهُ ٱلْعَذَابُ يَوْمَ ٱلْقِيَـٰمَةِ وَيَخْلُدْ فِيهِۦ مُهَانًا ﴿٦٩﴾\\
\textamh{70.\  } & إِلَّا مَن تَابَ وَءَامَنَ وَعَمِلَ عَمَلًۭا صَـٰلِحًۭا فَأُو۟لَـٰٓئِكَ يُبَدِّلُ ٱللَّهُ سَيِّـَٔاتِهِمْ حَسَنَـٰتٍۢ ۗ وَكَانَ ٱللَّهُ غَفُورًۭا رَّحِيمًۭا ﴿٧٠﴾\\
\textamh{71.\  } & وَمَن تَابَ وَعَمِلَ صَـٰلِحًۭا فَإِنَّهُۥ يَتُوبُ إِلَى ٱللَّهِ مَتَابًۭا ﴿٧١﴾\\
\textamh{72.\  } & وَٱلَّذِينَ لَا يَشْهَدُونَ ٱلزُّورَ وَإِذَا مَرُّوا۟ بِٱللَّغْوِ مَرُّوا۟ كِرَامًۭا ﴿٧٢﴾\\
\textamh{73.\  } & وَٱلَّذِينَ إِذَا ذُكِّرُوا۟ بِـَٔايَـٰتِ رَبِّهِمْ لَمْ يَخِرُّوا۟ عَلَيْهَا صُمًّۭا وَعُمْيَانًۭا ﴿٧٣﴾\\
\textamh{74.\  } & وَٱلَّذِينَ يَقُولُونَ رَبَّنَا هَبْ لَنَا مِنْ أَزْوَٟجِنَا وَذُرِّيَّٰتِنَا قُرَّةَ أَعْيُنٍۢ وَٱجْعَلْنَا لِلْمُتَّقِينَ إِمَامًا ﴿٧٤﴾\\
\textamh{75.\  } & أُو۟لَـٰٓئِكَ يُجْزَوْنَ ٱلْغُرْفَةَ بِمَا صَبَرُوا۟ وَيُلَقَّوْنَ فِيهَا تَحِيَّةًۭ وَسَلَـٰمًا ﴿٧٥﴾\\
\textamh{76.\  } & خَـٰلِدِينَ فِيهَا ۚ حَسُنَتْ مُسْتَقَرًّۭا وَمُقَامًۭا ﴿٧٦﴾\\
\textamh{77.\  } & قُلْ مَا يَعْبَؤُا۟ بِكُمْ رَبِّى لَوْلَا دُعَآؤُكُمْ ۖ فَقَدْ كَذَّبْتُمْ فَسَوْفَ يَكُونُ لِزَامًۢا ﴿٧٧﴾\\
\end{longtable}
\clearpage
%% License: BSD style (Berkley) (i.e. Put the Copyright owner's name always)
%% Writer and Copyright (to): Bewketu(Bilal) Tadilo (2016-17)
\centering\section{\LR{\textamharic{ሱራቱ አሹኣራኣ -}  \RL{سوره  الشعراء}}}
\begin{longtable}{%
  @{}
    p{.5\textwidth}
  @{~~~~~~~~~~~~}
    p{.5\textwidth}
    @{}
}
\nopagebreak
\textamh{ቢስሚላሂ አራህመኒ ራሂይም } &  بِسْمِ ٱللَّهِ ٱلرَّحْمَـٰنِ ٱلرَّحِيمِ\\
\textamh{1.\  } &  طسٓمٓ ﴿١﴾\\
\textamh{2.\  } & تِلْكَ ءَايَـٰتُ ٱلْكِتَـٰبِ ٱلْمُبِينِ ﴿٢﴾\\
\textamh{3.\  } & لَعَلَّكَ بَٰخِعٌۭ نَّفْسَكَ أَلَّا يَكُونُوا۟ مُؤْمِنِينَ ﴿٣﴾\\
\textamh{4.\  } & إِن نَّشَأْ نُنَزِّلْ عَلَيْهِم مِّنَ ٱلسَّمَآءِ ءَايَةًۭ فَظَلَّتْ أَعْنَـٰقُهُمْ لَهَا خَـٰضِعِينَ ﴿٤﴾\\
\textamh{5.\  } & وَمَا يَأْتِيهِم مِّن ذِكْرٍۢ مِّنَ ٱلرَّحْمَـٰنِ مُحْدَثٍ إِلَّا كَانُوا۟ عَنْهُ مُعْرِضِينَ ﴿٥﴾\\
\textamh{6.\  } & فَقَدْ كَذَّبُوا۟ فَسَيَأْتِيهِمْ أَنۢبَٰٓؤُا۟ مَا كَانُوا۟ بِهِۦ يَسْتَهْزِءُونَ ﴿٦﴾\\
\textamh{7.\  } & أَوَلَمْ يَرَوْا۟ إِلَى ٱلْأَرْضِ كَمْ أَنۢبَتْنَا فِيهَا مِن كُلِّ زَوْجٍۢ كَرِيمٍ ﴿٧﴾\\
\textamh{8.\  } & إِنَّ فِى ذَٟلِكَ لَءَايَةًۭ ۖ وَمَا كَانَ أَكْثَرُهُم مُّؤْمِنِينَ ﴿٨﴾\\
\textamh{9.\  } & وَإِنَّ رَبَّكَ لَهُوَ ٱلْعَزِيزُ ٱلرَّحِيمُ ﴿٩﴾\\
\textamh{10.\  } & وَإِذْ نَادَىٰ رَبُّكَ مُوسَىٰٓ أَنِ ٱئْتِ ٱلْقَوْمَ ٱلظَّـٰلِمِينَ ﴿١٠﴾\\
\textamh{11.\  } & قَوْمَ فِرْعَوْنَ ۚ أَلَا يَتَّقُونَ ﴿١١﴾\\
\textamh{12.\  } & قَالَ رَبِّ إِنِّىٓ أَخَافُ أَن يُكَذِّبُونِ ﴿١٢﴾\\
\textamh{13.\  } & وَيَضِيقُ صَدْرِى وَلَا يَنطَلِقُ لِسَانِى فَأَرْسِلْ إِلَىٰ هَـٰرُونَ ﴿١٣﴾\\
\textamh{14.\  } & وَلَهُمْ عَلَىَّ ذَنۢبٌۭ فَأَخَافُ أَن يَقْتُلُونِ ﴿١٤﴾\\
\textamh{15.\  } & قَالَ كَلَّا ۖ فَٱذْهَبَا بِـَٔايَـٰتِنَآ ۖ إِنَّا مَعَكُم مُّسْتَمِعُونَ ﴿١٥﴾\\
\textamh{16.\  } & فَأْتِيَا فِرْعَوْنَ فَقُولَآ إِنَّا رَسُولُ رَبِّ ٱلْعَـٰلَمِينَ ﴿١٦﴾\\
\textamh{17.\  } & أَنْ أَرْسِلْ مَعَنَا بَنِىٓ إِسْرَٰٓءِيلَ ﴿١٧﴾\\
\textamh{18.\  } & قَالَ أَلَمْ نُرَبِّكَ فِينَا وَلِيدًۭا وَلَبِثْتَ فِينَا مِنْ عُمُرِكَ سِنِينَ ﴿١٨﴾\\
\textamh{19.\  } & وَفَعَلْتَ فَعْلَتَكَ ٱلَّتِى فَعَلْتَ وَأَنتَ مِنَ ٱلْكَـٰفِرِينَ ﴿١٩﴾\\
\textamh{20.\  } & قَالَ فَعَلْتُهَآ إِذًۭا وَأَنَا۠ مِنَ ٱلضَّآلِّينَ ﴿٢٠﴾\\
\textamh{21.\  } & فَفَرَرْتُ مِنكُمْ لَمَّا خِفْتُكُمْ فَوَهَبَ لِى رَبِّى حُكْمًۭا وَجَعَلَنِى مِنَ ٱلْمُرْسَلِينَ ﴿٢١﴾\\
\textamh{22.\  } & وَتِلْكَ نِعْمَةٌۭ تَمُنُّهَا عَلَىَّ أَنْ عَبَّدتَّ بَنِىٓ إِسْرَٰٓءِيلَ ﴿٢٢﴾\\
\textamh{23.\  } & قَالَ فِرْعَوْنُ وَمَا رَبُّ ٱلْعَـٰلَمِينَ ﴿٢٣﴾\\
\textamh{24.\  } & قَالَ رَبُّ ٱلسَّمَـٰوَٟتِ وَٱلْأَرْضِ وَمَا بَيْنَهُمَآ ۖ إِن كُنتُم مُّوقِنِينَ ﴿٢٤﴾\\
\textamh{25.\  } & قَالَ لِمَنْ حَوْلَهُۥٓ أَلَا تَسْتَمِعُونَ ﴿٢٥﴾\\
\textamh{26.\  } & قَالَ رَبُّكُمْ وَرَبُّ ءَابَآئِكُمُ ٱلْأَوَّلِينَ ﴿٢٦﴾\\
\textamh{27.\  } & قَالَ إِنَّ رَسُولَكُمُ ٱلَّذِىٓ أُرْسِلَ إِلَيْكُمْ لَمَجْنُونٌۭ ﴿٢٧﴾\\
\textamh{28.\  } & قَالَ رَبُّ ٱلْمَشْرِقِ وَٱلْمَغْرِبِ وَمَا بَيْنَهُمَآ ۖ إِن كُنتُمْ تَعْقِلُونَ ﴿٢٨﴾\\
\textamh{29.\  } & قَالَ لَئِنِ ٱتَّخَذْتَ إِلَـٰهًا غَيْرِى لَأَجْعَلَنَّكَ مِنَ ٱلْمَسْجُونِينَ ﴿٢٩﴾\\
\textamh{30.\  } & قَالَ أَوَلَوْ جِئْتُكَ بِشَىْءٍۢ مُّبِينٍۢ ﴿٣٠﴾\\
\textamh{31.\  } & قَالَ فَأْتِ بِهِۦٓ إِن كُنتَ مِنَ ٱلصَّـٰدِقِينَ ﴿٣١﴾\\
\textamh{32.\  } & فَأَلْقَىٰ عَصَاهُ فَإِذَا هِىَ ثُعْبَانٌۭ مُّبِينٌۭ ﴿٣٢﴾\\
\textamh{33.\  } & وَنَزَعَ يَدَهُۥ فَإِذَا هِىَ بَيْضَآءُ لِلنَّـٰظِرِينَ ﴿٣٣﴾\\
\textamh{34.\  } & قَالَ لِلْمَلَإِ حَوْلَهُۥٓ إِنَّ هَـٰذَا لَسَـٰحِرٌ عَلِيمٌۭ ﴿٣٤﴾\\
\textamh{35.\  } & يُرِيدُ أَن يُخْرِجَكُم مِّنْ أَرْضِكُم بِسِحْرِهِۦ فَمَاذَا تَأْمُرُونَ ﴿٣٥﴾\\
\textamh{36.\  } & قَالُوٓا۟ أَرْجِهْ وَأَخَاهُ وَٱبْعَثْ فِى ٱلْمَدَآئِنِ حَـٰشِرِينَ ﴿٣٦﴾\\
\textamh{37.\  } & يَأْتُوكَ بِكُلِّ سَحَّارٍ عَلِيمٍۢ ﴿٣٧﴾\\
\textamh{38.\  } & فَجُمِعَ ٱلسَّحَرَةُ لِمِيقَـٰتِ يَوْمٍۢ مَّعْلُومٍۢ ﴿٣٨﴾\\
\textamh{39.\  } & وَقِيلَ لِلنَّاسِ هَلْ أَنتُم مُّجْتَمِعُونَ ﴿٣٩﴾\\
\textamh{40.\  } & لَعَلَّنَا نَتَّبِعُ ٱلسَّحَرَةَ إِن كَانُوا۟ هُمُ ٱلْغَٰلِبِينَ ﴿٤٠﴾\\
\textamh{41.\  } & فَلَمَّا جَآءَ ٱلسَّحَرَةُ قَالُوا۟ لِفِرْعَوْنَ أَئِنَّ لَنَا لَأَجْرًا إِن كُنَّا نَحْنُ ٱلْغَٰلِبِينَ ﴿٤١﴾\\
\textamh{42.\  } & قَالَ نَعَمْ وَإِنَّكُمْ إِذًۭا لَّمِنَ ٱلْمُقَرَّبِينَ ﴿٤٢﴾\\
\textamh{43.\  } & قَالَ لَهُم مُّوسَىٰٓ أَلْقُوا۟ مَآ أَنتُم مُّلْقُونَ ﴿٤٣﴾\\
\textamh{44.\  } & فَأَلْقَوْا۟ حِبَالَهُمْ وَعِصِيَّهُمْ وَقَالُوا۟ بِعِزَّةِ فِرْعَوْنَ إِنَّا لَنَحْنُ ٱلْغَٰلِبُونَ ﴿٤٤﴾\\
\textamh{45.\  } & فَأَلْقَىٰ مُوسَىٰ عَصَاهُ فَإِذَا هِىَ تَلْقَفُ مَا يَأْفِكُونَ ﴿٤٥﴾\\
\textamh{46.\  } & فَأُلْقِىَ ٱلسَّحَرَةُ سَـٰجِدِينَ ﴿٤٦﴾\\
\textamh{47.\  } & قَالُوٓا۟ ءَامَنَّا بِرَبِّ ٱلْعَـٰلَمِينَ ﴿٤٧﴾\\
\textamh{48.\  } & رَبِّ مُوسَىٰ وَهَـٰرُونَ ﴿٤٨﴾\\
\textamh{49.\  } & قَالَ ءَامَنتُمْ لَهُۥ قَبْلَ أَنْ ءَاذَنَ لَكُمْ ۖ إِنَّهُۥ لَكَبِيرُكُمُ ٱلَّذِى عَلَّمَكُمُ ٱلسِّحْرَ فَلَسَوْفَ تَعْلَمُونَ ۚ لَأُقَطِّعَنَّ أَيْدِيَكُمْ وَأَرْجُلَكُم مِّنْ خِلَـٰفٍۢ وَلَأُصَلِّبَنَّكُمْ أَجْمَعِينَ ﴿٤٩﴾\\
\textamh{50.\  } & قَالُوا۟ لَا ضَيْرَ ۖ إِنَّآ إِلَىٰ رَبِّنَا مُنقَلِبُونَ ﴿٥٠﴾\\
\textamh{51.\  } & إِنَّا نَطْمَعُ أَن يَغْفِرَ لَنَا رَبُّنَا خَطَٰيَـٰنَآ أَن كُنَّآ أَوَّلَ ٱلْمُؤْمِنِينَ ﴿٥١﴾\\
\textamh{52.\  } & ۞ وَأَوْحَيْنَآ إِلَىٰ مُوسَىٰٓ أَنْ أَسْرِ بِعِبَادِىٓ إِنَّكُم مُّتَّبَعُونَ ﴿٥٢﴾\\
\textamh{53.\  } & فَأَرْسَلَ فِرْعَوْنُ فِى ٱلْمَدَآئِنِ حَـٰشِرِينَ ﴿٥٣﴾\\
\textamh{54.\  } & إِنَّ هَـٰٓؤُلَآءِ لَشِرْذِمَةٌۭ قَلِيلُونَ ﴿٥٤﴾\\
\textamh{55.\  } & وَإِنَّهُمْ لَنَا لَغَآئِظُونَ ﴿٥٥﴾\\
\textamh{56.\  } & وَإِنَّا لَجَمِيعٌ حَـٰذِرُونَ ﴿٥٦﴾\\
\textamh{57.\  } & فَأَخْرَجْنَـٰهُم مِّن جَنَّـٰتٍۢ وَعُيُونٍۢ ﴿٥٧﴾\\
\textamh{58.\  } & وَكُنُوزٍۢ وَمَقَامٍۢ كَرِيمٍۢ ﴿٥٨﴾\\
\textamh{59.\  } & كَذَٟلِكَ وَأَوْرَثْنَـٰهَا بَنِىٓ إِسْرَٰٓءِيلَ ﴿٥٩﴾\\
\textamh{60.\  } & فَأَتْبَعُوهُم مُّشْرِقِينَ ﴿٦٠﴾\\
\textamh{61.\  } & فَلَمَّا تَرَٰٓءَا ٱلْجَمْعَانِ قَالَ أَصْحَـٰبُ مُوسَىٰٓ إِنَّا لَمُدْرَكُونَ ﴿٦١﴾\\
\textamh{62.\  } & قَالَ كَلَّآ ۖ إِنَّ مَعِىَ رَبِّى سَيَهْدِينِ ﴿٦٢﴾\\
\textamh{63.\  } & فَأَوْحَيْنَآ إِلَىٰ مُوسَىٰٓ أَنِ ٱضْرِب بِّعَصَاكَ ٱلْبَحْرَ ۖ فَٱنفَلَقَ فَكَانَ كُلُّ فِرْقٍۢ كَٱلطَّوْدِ ٱلْعَظِيمِ ﴿٦٣﴾\\
\textamh{64.\  } & وَأَزْلَفْنَا ثَمَّ ٱلْءَاخَرِينَ ﴿٦٤﴾\\
\textamh{65.\  } & وَأَنجَيْنَا مُوسَىٰ وَمَن مَّعَهُۥٓ أَجْمَعِينَ ﴿٦٥﴾\\
\textamh{66.\  } & ثُمَّ أَغْرَقْنَا ٱلْءَاخَرِينَ ﴿٦٦﴾\\
\textamh{67.\  } & إِنَّ فِى ذَٟلِكَ لَءَايَةًۭ ۖ وَمَا كَانَ أَكْثَرُهُم مُّؤْمِنِينَ ﴿٦٧﴾\\
\textamh{68.\  } & وَإِنَّ رَبَّكَ لَهُوَ ٱلْعَزِيزُ ٱلرَّحِيمُ ﴿٦٨﴾\\
\textamh{69.\  } & وَٱتْلُ عَلَيْهِمْ نَبَأَ إِبْرَٰهِيمَ ﴿٦٩﴾\\
\textamh{70.\  } & إِذْ قَالَ لِأَبِيهِ وَقَوْمِهِۦ مَا تَعْبُدُونَ ﴿٧٠﴾\\
\textamh{71.\  } & قَالُوا۟ نَعْبُدُ أَصْنَامًۭا فَنَظَلُّ لَهَا عَـٰكِفِينَ ﴿٧١﴾\\
\textamh{72.\  } & قَالَ هَلْ يَسْمَعُونَكُمْ إِذْ تَدْعُونَ ﴿٧٢﴾\\
\textamh{73.\  } & أَوْ يَنفَعُونَكُمْ أَوْ يَضُرُّونَ ﴿٧٣﴾\\
\textamh{74.\  } & قَالُوا۟ بَلْ وَجَدْنَآ ءَابَآءَنَا كَذَٟلِكَ يَفْعَلُونَ ﴿٧٤﴾\\
\textamh{75.\  } & قَالَ أَفَرَءَيْتُم مَّا كُنتُمْ تَعْبُدُونَ ﴿٧٥﴾\\
\textamh{76.\  } & أَنتُمْ وَءَابَآؤُكُمُ ٱلْأَقْدَمُونَ ﴿٧٦﴾\\
\textamh{77.\  } & فَإِنَّهُمْ عَدُوٌّۭ لِّىٓ إِلَّا رَبَّ ٱلْعَـٰلَمِينَ ﴿٧٧﴾\\
\textamh{78.\  } & ٱلَّذِى خَلَقَنِى فَهُوَ يَهْدِينِ ﴿٧٨﴾\\
\textamh{79.\  } & وَٱلَّذِى هُوَ يُطْعِمُنِى وَيَسْقِينِ ﴿٧٩﴾\\
\textamh{80.\  } & وَإِذَا مَرِضْتُ فَهُوَ يَشْفِينِ ﴿٨٠﴾\\
\textamh{81.\  } & وَٱلَّذِى يُمِيتُنِى ثُمَّ يُحْيِينِ ﴿٨١﴾\\
\textamh{82.\  } & وَٱلَّذِىٓ أَطْمَعُ أَن يَغْفِرَ لِى خَطِيٓـَٔتِى يَوْمَ ٱلدِّينِ ﴿٨٢﴾\\
\textamh{83.\  } & رَبِّ هَبْ لِى حُكْمًۭا وَأَلْحِقْنِى بِٱلصَّـٰلِحِينَ ﴿٨٣﴾\\
\textamh{84.\  } & وَٱجْعَل لِّى لِسَانَ صِدْقٍۢ فِى ٱلْءَاخِرِينَ ﴿٨٤﴾\\
\textamh{85.\  } & وَٱجْعَلْنِى مِن وَرَثَةِ جَنَّةِ ٱلنَّعِيمِ ﴿٨٥﴾\\
\textamh{86.\  } & وَٱغْفِرْ لِأَبِىٓ إِنَّهُۥ كَانَ مِنَ ٱلضَّآلِّينَ ﴿٨٦﴾\\
\textamh{87.\  } & وَلَا تُخْزِنِى يَوْمَ يُبْعَثُونَ ﴿٨٧﴾\\
\textamh{88.\  } & يَوْمَ لَا يَنفَعُ مَالٌۭ وَلَا بَنُونَ ﴿٨٨﴾\\
\textamh{89.\  } & إِلَّا مَنْ أَتَى ٱللَّهَ بِقَلْبٍۢ سَلِيمٍۢ ﴿٨٩﴾\\
\textamh{90.\  } & وَأُزْلِفَتِ ٱلْجَنَّةُ لِلْمُتَّقِينَ ﴿٩٠﴾\\
\textamh{91.\  } & وَبُرِّزَتِ ٱلْجَحِيمُ لِلْغَاوِينَ ﴿٩١﴾\\
\textamh{92.\  } & وَقِيلَ لَهُمْ أَيْنَ مَا كُنتُمْ تَعْبُدُونَ ﴿٩٢﴾\\
\textamh{93.\  } & مِن دُونِ ٱللَّهِ هَلْ يَنصُرُونَكُمْ أَوْ يَنتَصِرُونَ ﴿٩٣﴾\\
\textamh{94.\  } & فَكُبْكِبُوا۟ فِيهَا هُمْ وَٱلْغَاوُۥنَ ﴿٩٤﴾\\
\textamh{95.\  } & وَجُنُودُ إِبْلِيسَ أَجْمَعُونَ ﴿٩٥﴾\\
\textamh{96.\  } & قَالُوا۟ وَهُمْ فِيهَا يَخْتَصِمُونَ ﴿٩٦﴾\\
\textamh{97.\  } & تَٱللَّهِ إِن كُنَّا لَفِى ضَلَـٰلٍۢ مُّبِينٍ ﴿٩٧﴾\\
\textamh{98.\  } & إِذْ نُسَوِّيكُم بِرَبِّ ٱلْعَـٰلَمِينَ ﴿٩٨﴾\\
\textamh{99.\  } & وَمَآ أَضَلَّنَآ إِلَّا ٱلْمُجْرِمُونَ ﴿٩٩﴾\\
\textamh{100.\  } & فَمَا لَنَا مِن شَـٰفِعِينَ ﴿١٠٠﴾\\
\textamh{101.\  } & وَلَا صَدِيقٍ حَمِيمٍۢ ﴿١٠١﴾\\
\textamh{102.\  } & فَلَوْ أَنَّ لَنَا كَرَّةًۭ فَنَكُونَ مِنَ ٱلْمُؤْمِنِينَ ﴿١٠٢﴾\\
\textamh{103.\  } & إِنَّ فِى ذَٟلِكَ لَءَايَةًۭ ۖ وَمَا كَانَ أَكْثَرُهُم مُّؤْمِنِينَ ﴿١٠٣﴾\\
\textamh{104.\  } & وَإِنَّ رَبَّكَ لَهُوَ ٱلْعَزِيزُ ٱلرَّحِيمُ ﴿١٠٤﴾\\
\textamh{105.\  } & كَذَّبَتْ قَوْمُ نُوحٍ ٱلْمُرْسَلِينَ ﴿١٠٥﴾\\
\textamh{106.\  } & إِذْ قَالَ لَهُمْ أَخُوهُمْ نُوحٌ أَلَا تَتَّقُونَ ﴿١٠٦﴾\\
\textamh{107.\  } & إِنِّى لَكُمْ رَسُولٌ أَمِينٌۭ ﴿١٠٧﴾\\
\textamh{108.\  } & فَٱتَّقُوا۟ ٱللَّهَ وَأَطِيعُونِ ﴿١٠٨﴾\\
\textamh{109.\  } & وَمَآ أَسْـَٔلُكُمْ عَلَيْهِ مِنْ أَجْرٍ ۖ إِنْ أَجْرِىَ إِلَّا عَلَىٰ رَبِّ ٱلْعَـٰلَمِينَ ﴿١٠٩﴾\\
\textamh{110.\  } & فَٱتَّقُوا۟ ٱللَّهَ وَأَطِيعُونِ ﴿١١٠﴾\\
\textamh{111.\  } & ۞ قَالُوٓا۟ أَنُؤْمِنُ لَكَ وَٱتَّبَعَكَ ٱلْأَرْذَلُونَ ﴿١١١﴾\\
\textamh{112.\  } & قَالَ وَمَا عِلْمِى بِمَا كَانُوا۟ يَعْمَلُونَ ﴿١١٢﴾\\
\textamh{113.\  } & إِنْ حِسَابُهُمْ إِلَّا عَلَىٰ رَبِّى ۖ لَوْ تَشْعُرُونَ ﴿١١٣﴾\\
\textamh{114.\  } & وَمَآ أَنَا۠ بِطَارِدِ ٱلْمُؤْمِنِينَ ﴿١١٤﴾\\
\textamh{115.\  } & إِنْ أَنَا۠ إِلَّا نَذِيرٌۭ مُّبِينٌۭ ﴿١١٥﴾\\
\textamh{116.\  } & قَالُوا۟ لَئِن لَّمْ تَنتَهِ يَـٰنُوحُ لَتَكُونَنَّ مِنَ ٱلْمَرْجُومِينَ ﴿١١٦﴾\\
\textamh{117.\  } & قَالَ رَبِّ إِنَّ قَوْمِى كَذَّبُونِ ﴿١١٧﴾\\
\textamh{118.\  } & فَٱفْتَحْ بَيْنِى وَبَيْنَهُمْ فَتْحًۭا وَنَجِّنِى وَمَن مَّعِىَ مِنَ ٱلْمُؤْمِنِينَ ﴿١١٨﴾\\
\textamh{119.\  } & فَأَنجَيْنَـٰهُ وَمَن مَّعَهُۥ فِى ٱلْفُلْكِ ٱلْمَشْحُونِ ﴿١١٩﴾\\
\textamh{120.\  } & ثُمَّ أَغْرَقْنَا بَعْدُ ٱلْبَاقِينَ ﴿١٢٠﴾\\
\textamh{121.\  } & إِنَّ فِى ذَٟلِكَ لَءَايَةًۭ ۖ وَمَا كَانَ أَكْثَرُهُم مُّؤْمِنِينَ ﴿١٢١﴾\\
\textamh{122.\  } & وَإِنَّ رَبَّكَ لَهُوَ ٱلْعَزِيزُ ٱلرَّحِيمُ ﴿١٢٢﴾\\
\textamh{123.\  } & كَذَّبَتْ عَادٌ ٱلْمُرْسَلِينَ ﴿١٢٣﴾\\
\textamh{124.\  } & إِذْ قَالَ لَهُمْ أَخُوهُمْ هُودٌ أَلَا تَتَّقُونَ ﴿١٢٤﴾\\
\textamh{125.\  } & إِنِّى لَكُمْ رَسُولٌ أَمِينٌۭ ﴿١٢٥﴾\\
\textamh{126.\  } & فَٱتَّقُوا۟ ٱللَّهَ وَأَطِيعُونِ ﴿١٢٦﴾\\
\textamh{127.\  } & وَمَآ أَسْـَٔلُكُمْ عَلَيْهِ مِنْ أَجْرٍ ۖ إِنْ أَجْرِىَ إِلَّا عَلَىٰ رَبِّ ٱلْعَـٰلَمِينَ ﴿١٢٧﴾\\
\textamh{128.\  } & أَتَبْنُونَ بِكُلِّ رِيعٍ ءَايَةًۭ تَعْبَثُونَ ﴿١٢٨﴾\\
\textamh{129.\  } & وَتَتَّخِذُونَ مَصَانِعَ لَعَلَّكُمْ تَخْلُدُونَ ﴿١٢٩﴾\\
\textamh{130.\  } & وَإِذَا بَطَشْتُم بَطَشْتُمْ جَبَّارِينَ ﴿١٣٠﴾\\
\textamh{131.\  } & فَٱتَّقُوا۟ ٱللَّهَ وَأَطِيعُونِ ﴿١٣١﴾\\
\textamh{132.\  } & وَٱتَّقُوا۟ ٱلَّذِىٓ أَمَدَّكُم بِمَا تَعْلَمُونَ ﴿١٣٢﴾\\
\textamh{133.\  } & أَمَدَّكُم بِأَنْعَـٰمٍۢ وَبَنِينَ ﴿١٣٣﴾\\
\textamh{134.\  } & وَجَنَّـٰتٍۢ وَعُيُونٍ ﴿١٣٤﴾\\
\textamh{135.\  } & إِنِّىٓ أَخَافُ عَلَيْكُمْ عَذَابَ يَوْمٍ عَظِيمٍۢ ﴿١٣٥﴾\\
\textamh{136.\  } & قَالُوا۟ سَوَآءٌ عَلَيْنَآ أَوَعَظْتَ أَمْ لَمْ تَكُن مِّنَ ٱلْوَٟعِظِينَ ﴿١٣٦﴾\\
\textamh{137.\  } & إِنْ هَـٰذَآ إِلَّا خُلُقُ ٱلْأَوَّلِينَ ﴿١٣٧﴾\\
\textamh{138.\  } & وَمَا نَحْنُ بِمُعَذَّبِينَ ﴿١٣٨﴾\\
\textamh{139.\  } & فَكَذَّبُوهُ فَأَهْلَكْنَـٰهُمْ ۗ إِنَّ فِى ذَٟلِكَ لَءَايَةًۭ ۖ وَمَا كَانَ أَكْثَرُهُم مُّؤْمِنِينَ ﴿١٣٩﴾\\
\textamh{140.\  } & وَإِنَّ رَبَّكَ لَهُوَ ٱلْعَزِيزُ ٱلرَّحِيمُ ﴿١٤٠﴾\\
\textamh{141.\  } & كَذَّبَتْ ثَمُودُ ٱلْمُرْسَلِينَ ﴿١٤١﴾\\
\textamh{142.\  } & إِذْ قَالَ لَهُمْ أَخُوهُمْ صَـٰلِحٌ أَلَا تَتَّقُونَ ﴿١٤٢﴾\\
\textamh{143.\  } & إِنِّى لَكُمْ رَسُولٌ أَمِينٌۭ ﴿١٤٣﴾\\
\textamh{144.\  } & فَٱتَّقُوا۟ ٱللَّهَ وَأَطِيعُونِ ﴿١٤٤﴾\\
\textamh{145.\  } & وَمَآ أَسْـَٔلُكُمْ عَلَيْهِ مِنْ أَجْرٍ ۖ إِنْ أَجْرِىَ إِلَّا عَلَىٰ رَبِّ ٱلْعَـٰلَمِينَ ﴿١٤٥﴾\\
\textamh{146.\  } & أَتُتْرَكُونَ فِى مَا هَـٰهُنَآ ءَامِنِينَ ﴿١٤٦﴾\\
\textamh{147.\  } & فِى جَنَّـٰتٍۢ وَعُيُونٍۢ ﴿١٤٧﴾\\
\textamh{148.\  } & وَزُرُوعٍۢ وَنَخْلٍۢ طَلْعُهَا هَضِيمٌۭ ﴿١٤٨﴾\\
\textamh{149.\  } & وَتَنْحِتُونَ مِنَ ٱلْجِبَالِ بُيُوتًۭا فَـٰرِهِينَ ﴿١٤٩﴾\\
\textamh{150.\  } & فَٱتَّقُوا۟ ٱللَّهَ وَأَطِيعُونِ ﴿١٥٠﴾\\
\textamh{151.\  } & وَلَا تُطِيعُوٓا۟ أَمْرَ ٱلْمُسْرِفِينَ ﴿١٥١﴾\\
\textamh{152.\  } & ٱلَّذِينَ يُفْسِدُونَ فِى ٱلْأَرْضِ وَلَا يُصْلِحُونَ ﴿١٥٢﴾\\
\textamh{153.\  } & قَالُوٓا۟ إِنَّمَآ أَنتَ مِنَ ٱلْمُسَحَّرِينَ ﴿١٥٣﴾\\
\textamh{154.\  } & مَآ أَنتَ إِلَّا بَشَرٌۭ مِّثْلُنَا فَأْتِ بِـَٔايَةٍ إِن كُنتَ مِنَ ٱلصَّـٰدِقِينَ ﴿١٥٤﴾\\
\textamh{155.\  } & قَالَ هَـٰذِهِۦ نَاقَةٌۭ لَّهَا شِرْبٌۭ وَلَكُمْ شِرْبُ يَوْمٍۢ مَّعْلُومٍۢ ﴿١٥٥﴾\\
\textamh{156.\  } & وَلَا تَمَسُّوهَا بِسُوٓءٍۢ فَيَأْخُذَكُمْ عَذَابُ يَوْمٍ عَظِيمٍۢ ﴿١٥٦﴾\\
\textamh{157.\  } & فَعَقَرُوهَا فَأَصْبَحُوا۟ نَـٰدِمِينَ ﴿١٥٧﴾\\
\textamh{158.\  } & فَأَخَذَهُمُ ٱلْعَذَابُ ۗ إِنَّ فِى ذَٟلِكَ لَءَايَةًۭ ۖ وَمَا كَانَ أَكْثَرُهُم مُّؤْمِنِينَ ﴿١٥٨﴾\\
\textamh{159.\  } & وَإِنَّ رَبَّكَ لَهُوَ ٱلْعَزِيزُ ٱلرَّحِيمُ ﴿١٥٩﴾\\
\textamh{160.\  } & كَذَّبَتْ قَوْمُ لُوطٍ ٱلْمُرْسَلِينَ ﴿١٦٠﴾\\
\textamh{161.\  } & إِذْ قَالَ لَهُمْ أَخُوهُمْ لُوطٌ أَلَا تَتَّقُونَ ﴿١٦١﴾\\
\textamh{162.\  } & إِنِّى لَكُمْ رَسُولٌ أَمِينٌۭ ﴿١٦٢﴾\\
\textamh{163.\  } & فَٱتَّقُوا۟ ٱللَّهَ وَأَطِيعُونِ ﴿١٦٣﴾\\
\textamh{164.\  } & وَمَآ أَسْـَٔلُكُمْ عَلَيْهِ مِنْ أَجْرٍ ۖ إِنْ أَجْرِىَ إِلَّا عَلَىٰ رَبِّ ٱلْعَـٰلَمِينَ ﴿١٦٤﴾\\
\textamh{165.\  } & أَتَأْتُونَ ٱلذُّكْرَانَ مِنَ ٱلْعَـٰلَمِينَ ﴿١٦٥﴾\\
\textamh{166.\  } & وَتَذَرُونَ مَا خَلَقَ لَكُمْ رَبُّكُم مِّنْ أَزْوَٟجِكُم ۚ بَلْ أَنتُمْ قَوْمٌ عَادُونَ ﴿١٦٦﴾\\
\textamh{167.\  } & قَالُوا۟ لَئِن لَّمْ تَنتَهِ يَـٰلُوطُ لَتَكُونَنَّ مِنَ ٱلْمُخْرَجِينَ ﴿١٦٧﴾\\
\textamh{168.\  } & قَالَ إِنِّى لِعَمَلِكُم مِّنَ ٱلْقَالِينَ ﴿١٦٨﴾\\
\textamh{169.\  } & رَبِّ نَجِّنِى وَأَهْلِى مِمَّا يَعْمَلُونَ ﴿١٦٩﴾\\
\textamh{170.\  } & فَنَجَّيْنَـٰهُ وَأَهْلَهُۥٓ أَجْمَعِينَ ﴿١٧٠﴾\\
\textamh{171.\  } & إِلَّا عَجُوزًۭا فِى ٱلْغَٰبِرِينَ ﴿١٧١﴾\\
\textamh{172.\  } & ثُمَّ دَمَّرْنَا ٱلْءَاخَرِينَ ﴿١٧٢﴾\\
\textamh{173.\  } & وَأَمْطَرْنَا عَلَيْهِم مَّطَرًۭا ۖ فَسَآءَ مَطَرُ ٱلْمُنذَرِينَ ﴿١٧٣﴾\\
\textamh{174.\  } & إِنَّ فِى ذَٟلِكَ لَءَايَةًۭ ۖ وَمَا كَانَ أَكْثَرُهُم مُّؤْمِنِينَ ﴿١٧٤﴾\\
\textamh{175.\  } & وَإِنَّ رَبَّكَ لَهُوَ ٱلْعَزِيزُ ٱلرَّحِيمُ ﴿١٧٥﴾\\
\textamh{176.\  } & كَذَّبَ أَصْحَـٰبُ لْـَٔيْكَةِ ٱلْمُرْسَلِينَ ﴿١٧٦﴾\\
\textamh{177.\  } & إِذْ قَالَ لَهُمْ شُعَيْبٌ أَلَا تَتَّقُونَ ﴿١٧٧﴾\\
\textamh{178.\  } & إِنِّى لَكُمْ رَسُولٌ أَمِينٌۭ ﴿١٧٨﴾\\
\textamh{179.\  } & فَٱتَّقُوا۟ ٱللَّهَ وَأَطِيعُونِ ﴿١٧٩﴾\\
\textamh{180.\  } & وَمَآ أَسْـَٔلُكُمْ عَلَيْهِ مِنْ أَجْرٍ ۖ إِنْ أَجْرِىَ إِلَّا عَلَىٰ رَبِّ ٱلْعَـٰلَمِينَ ﴿١٨٠﴾\\
\textamh{181.\  } & ۞ أَوْفُوا۟ ٱلْكَيْلَ وَلَا تَكُونُوا۟ مِنَ ٱلْمُخْسِرِينَ ﴿١٨١﴾\\
\textamh{182.\  } & وَزِنُوا۟ بِٱلْقِسْطَاسِ ٱلْمُسْتَقِيمِ ﴿١٨٢﴾\\
\textamh{183.\  } & وَلَا تَبْخَسُوا۟ ٱلنَّاسَ أَشْيَآءَهُمْ وَلَا تَعْثَوْا۟ فِى ٱلْأَرْضِ مُفْسِدِينَ ﴿١٨٣﴾\\
\textamh{184.\  } & وَٱتَّقُوا۟ ٱلَّذِى خَلَقَكُمْ وَٱلْجِبِلَّةَ ٱلْأَوَّلِينَ ﴿١٨٤﴾\\
\textamh{185.\  } & قَالُوٓا۟ إِنَّمَآ أَنتَ مِنَ ٱلْمُسَحَّرِينَ ﴿١٨٥﴾\\
\textamh{186.\  } & وَمَآ أَنتَ إِلَّا بَشَرٌۭ مِّثْلُنَا وَإِن نَّظُنُّكَ لَمِنَ ٱلْكَـٰذِبِينَ ﴿١٨٦﴾\\
\textamh{187.\  } & فَأَسْقِطْ عَلَيْنَا كِسَفًۭا مِّنَ ٱلسَّمَآءِ إِن كُنتَ مِنَ ٱلصَّـٰدِقِينَ ﴿١٨٧﴾\\
\textamh{188.\  } & قَالَ رَبِّىٓ أَعْلَمُ بِمَا تَعْمَلُونَ ﴿١٨٨﴾\\
\textamh{189.\  } & فَكَذَّبُوهُ فَأَخَذَهُمْ عَذَابُ يَوْمِ ٱلظُّلَّةِ ۚ إِنَّهُۥ كَانَ عَذَابَ يَوْمٍ عَظِيمٍ ﴿١٨٩﴾\\
\textamh{190.\  } & إِنَّ فِى ذَٟلِكَ لَءَايَةًۭ ۖ وَمَا كَانَ أَكْثَرُهُم مُّؤْمِنِينَ ﴿١٩٠﴾\\
\textamh{191.\  } & وَإِنَّ رَبَّكَ لَهُوَ ٱلْعَزِيزُ ٱلرَّحِيمُ ﴿١٩١﴾\\
\textamh{192.\  } & وَإِنَّهُۥ لَتَنزِيلُ رَبِّ ٱلْعَـٰلَمِينَ ﴿١٩٢﴾\\
\textamh{193.\  } & نَزَلَ بِهِ ٱلرُّوحُ ٱلْأَمِينُ ﴿١٩٣﴾\\
\textamh{194.\  } & عَلَىٰ قَلْبِكَ لِتَكُونَ مِنَ ٱلْمُنذِرِينَ ﴿١٩٤﴾\\
\textamh{195.\  } & بِلِسَانٍ عَرَبِىٍّۢ مُّبِينٍۢ ﴿١٩٥﴾\\
\textamh{196.\  } & وَإِنَّهُۥ لَفِى زُبُرِ ٱلْأَوَّلِينَ ﴿١٩٦﴾\\
\textamh{197.\  } & أَوَلَمْ يَكُن لَّهُمْ ءَايَةً أَن يَعْلَمَهُۥ عُلَمَـٰٓؤُا۟ بَنِىٓ إِسْرَٰٓءِيلَ ﴿١٩٧﴾\\
\textamh{198.\  } & وَلَوْ نَزَّلْنَـٰهُ عَلَىٰ بَعْضِ ٱلْأَعْجَمِينَ ﴿١٩٨﴾\\
\textamh{199.\  } & فَقَرَأَهُۥ عَلَيْهِم مَّا كَانُوا۟ بِهِۦ مُؤْمِنِينَ ﴿١٩٩﴾\\
\textamh{200.\  } & كَذَٟلِكَ سَلَكْنَـٰهُ فِى قُلُوبِ ٱلْمُجْرِمِينَ ﴿٢٠٠﴾\\
\textamh{201.\  } & لَا يُؤْمِنُونَ بِهِۦ حَتَّىٰ يَرَوُا۟ ٱلْعَذَابَ ٱلْأَلِيمَ ﴿٢٠١﴾\\
\textamh{202.\  } & فَيَأْتِيَهُم بَغْتَةًۭ وَهُمْ لَا يَشْعُرُونَ ﴿٢٠٢﴾\\
\textamh{203.\  } & فَيَقُولُوا۟ هَلْ نَحْنُ مُنظَرُونَ ﴿٢٠٣﴾\\
\textamh{204.\  } & أَفَبِعَذَابِنَا يَسْتَعْجِلُونَ ﴿٢٠٤﴾\\
\textamh{205.\  } & أَفَرَءَيْتَ إِن مَّتَّعْنَـٰهُمْ سِنِينَ ﴿٢٠٥﴾\\
\textamh{206.\  } & ثُمَّ جَآءَهُم مَّا كَانُوا۟ يُوعَدُونَ ﴿٢٠٦﴾\\
\textamh{207.\  } & مَآ أَغْنَىٰ عَنْهُم مَّا كَانُوا۟ يُمَتَّعُونَ ﴿٢٠٧﴾\\
\textamh{208.\  } & وَمَآ أَهْلَكْنَا مِن قَرْيَةٍ إِلَّا لَهَا مُنذِرُونَ ﴿٢٠٨﴾\\
\textamh{209.\  } & ذِكْرَىٰ وَمَا كُنَّا ظَـٰلِمِينَ ﴿٢٠٩﴾\\
\textamh{210.\  } & وَمَا تَنَزَّلَتْ بِهِ ٱلشَّيَـٰطِينُ ﴿٢١٠﴾\\
\textamh{211.\  } & وَمَا يَنۢبَغِى لَهُمْ وَمَا يَسْتَطِيعُونَ ﴿٢١١﴾\\
\textamh{212.\  } & إِنَّهُمْ عَنِ ٱلسَّمْعِ لَمَعْزُولُونَ ﴿٢١٢﴾\\
\textamh{213.\  } & فَلَا تَدْعُ مَعَ ٱللَّهِ إِلَـٰهًا ءَاخَرَ فَتَكُونَ مِنَ ٱلْمُعَذَّبِينَ ﴿٢١٣﴾\\
\textamh{214.\  } & وَأَنذِرْ عَشِيرَتَكَ ٱلْأَقْرَبِينَ ﴿٢١٤﴾\\
\textamh{215.\  } & وَٱخْفِضْ جَنَاحَكَ لِمَنِ ٱتَّبَعَكَ مِنَ ٱلْمُؤْمِنِينَ ﴿٢١٥﴾\\
\textamh{216.\  } & فَإِنْ عَصَوْكَ فَقُلْ إِنِّى بَرِىٓءٌۭ مِّمَّا تَعْمَلُونَ ﴿٢١٦﴾\\
\textamh{217.\  } & وَتَوَكَّلْ عَلَى ٱلْعَزِيزِ ٱلرَّحِيمِ ﴿٢١٧﴾\\
\textamh{218.\  } & ٱلَّذِى يَرَىٰكَ حِينَ تَقُومُ ﴿٢١٨﴾\\
\textamh{219.\  } & وَتَقَلُّبَكَ فِى ٱلسَّٰجِدِينَ ﴿٢١٩﴾\\
\textamh{220.\  } & إِنَّهُۥ هُوَ ٱلسَّمِيعُ ٱلْعَلِيمُ ﴿٢٢٠﴾\\
\textamh{221.\  } & هَلْ أُنَبِّئُكُمْ عَلَىٰ مَن تَنَزَّلُ ٱلشَّيَـٰطِينُ ﴿٢٢١﴾\\
\textamh{222.\  } & تَنَزَّلُ عَلَىٰ كُلِّ أَفَّاكٍ أَثِيمٍۢ ﴿٢٢٢﴾\\
\textamh{223.\  } & يُلْقُونَ ٱلسَّمْعَ وَأَكْثَرُهُمْ كَـٰذِبُونَ ﴿٢٢٣﴾\\
\textamh{224.\  } & وَٱلشُّعَرَآءُ يَتَّبِعُهُمُ ٱلْغَاوُۥنَ ﴿٢٢٤﴾\\
\textamh{225.\  } & أَلَمْ تَرَ أَنَّهُمْ فِى كُلِّ وَادٍۢ يَهِيمُونَ ﴿٢٢٥﴾\\
\textamh{226.\  } & وَأَنَّهُمْ يَقُولُونَ مَا لَا يَفْعَلُونَ ﴿٢٢٦﴾\\
\textamh{227.\  } & إِلَّا ٱلَّذِينَ ءَامَنُوا۟ وَعَمِلُوا۟ ٱلصَّـٰلِحَـٰتِ وَذَكَرُوا۟ ٱللَّهَ كَثِيرًۭا وَٱنتَصَرُوا۟ مِنۢ بَعْدِ مَا ظُلِمُوا۟ ۗ وَسَيَعْلَمُ ٱلَّذِينَ ظَلَمُوٓا۟ أَىَّ مُنقَلَبٍۢ يَنقَلِبُونَ ﴿٢٢٧﴾\\
\end{longtable}
\clearpage
%% License: BSD style (Berkley) (i.e. Put the Copyright owner's name always)
%% Writer and Copyright (to): Bewketu(Bilal) Tadilo (2016-17)
\begin{center}\section{\LR{\textamhsec{ሱራቱ አንነምል -}  \textarabic{سوره  النمل}}}\end{center}
\begin{longtable}{%
  @{}
    p{.5\textwidth}
  @{~~~}
    p{.5\textwidth}
    @{}
}
\textamh{ቢስሚላሂ አራህመኒ ራሂይም } &  \mytextarabic{بِسْمِ ٱللَّهِ ٱلرَّحْمَـٰنِ ٱلرَّحِيمِ}\\
\textamh{1.\  } & \mytextarabic{ طسٓ ۚ تِلْكَ ءَايَـٰتُ ٱلْقُرْءَانِ وَكِتَابٍۢ مُّبِينٍ ﴿١﴾}\\
\textamh{2.\  } & \mytextarabic{هُدًۭى وَبُشْرَىٰ لِلْمُؤْمِنِينَ ﴿٢﴾}\\
\textamh{3.\  } & \mytextarabic{ٱلَّذِينَ يُقِيمُونَ ٱلصَّلَوٰةَ وَيُؤْتُونَ ٱلزَّكَوٰةَ وَهُم بِٱلْءَاخِرَةِ هُمْ يُوقِنُونَ ﴿٣﴾}\\
\textamh{4.\  } & \mytextarabic{إِنَّ ٱلَّذِينَ لَا يُؤْمِنُونَ بِٱلْءَاخِرَةِ زَيَّنَّا لَهُمْ أَعْمَـٰلَهُمْ فَهُمْ يَعْمَهُونَ ﴿٤﴾}\\
\textamh{5.\  } & \mytextarabic{أُو۟لَـٰٓئِكَ ٱلَّذِينَ لَهُمْ سُوٓءُ ٱلْعَذَابِ وَهُمْ فِى ٱلْءَاخِرَةِ هُمُ ٱلْأَخْسَرُونَ ﴿٥﴾}\\
\textamh{6.\  } & \mytextarabic{وَإِنَّكَ لَتُلَقَّى ٱلْقُرْءَانَ مِن لَّدُنْ حَكِيمٍ عَلِيمٍ ﴿٦﴾}\\
\textamh{7.\  } & \mytextarabic{إِذْ قَالَ مُوسَىٰ لِأَهْلِهِۦٓ إِنِّىٓ ءَانَسْتُ نَارًۭا سَـَٔاتِيكُم مِّنْهَا بِخَبَرٍ أَوْ ءَاتِيكُم بِشِهَابٍۢ قَبَسٍۢ لَّعَلَّكُمْ تَصْطَلُونَ ﴿٧﴾}\\
\textamh{8.\  } & \mytextarabic{فَلَمَّا جَآءَهَا نُودِىَ أَنۢ بُورِكَ مَن فِى ٱلنَّارِ وَمَنْ حَوْلَهَا وَسُبْحَـٰنَ ٱللَّهِ رَبِّ ٱلْعَـٰلَمِينَ ﴿٨﴾}\\
\textamh{9.\  } & \mytextarabic{يَـٰمُوسَىٰٓ إِنَّهُۥٓ أَنَا ٱللَّهُ ٱلْعَزِيزُ ٱلْحَكِيمُ ﴿٩﴾}\\
\textamh{10.\  } & \mytextarabic{وَأَلْقِ عَصَاكَ ۚ فَلَمَّا رَءَاهَا تَهْتَزُّ كَأَنَّهَا جَآنٌّۭ وَلَّىٰ مُدْبِرًۭا وَلَمْ يُعَقِّبْ ۚ يَـٰمُوسَىٰ لَا تَخَفْ إِنِّى لَا يَخَافُ لَدَىَّ ٱلْمُرْسَلُونَ ﴿١٠﴾}\\
\textamh{11.\  } & \mytextarabic{إِلَّا مَن ظَلَمَ ثُمَّ بَدَّلَ حُسْنًۢا بَعْدَ سُوٓءٍۢ فَإِنِّى غَفُورٌۭ رَّحِيمٌۭ ﴿١١﴾}\\
\textamh{12.\  } & \mytextarabic{وَأَدْخِلْ يَدَكَ فِى جَيْبِكَ تَخْرُجْ بَيْضَآءَ مِنْ غَيْرِ سُوٓءٍۢ ۖ فِى تِسْعِ ءَايَـٰتٍ إِلَىٰ فِرْعَوْنَ وَقَوْمِهِۦٓ ۚ إِنَّهُمْ كَانُوا۟ قَوْمًۭا فَـٰسِقِينَ ﴿١٢﴾}\\
\textamh{13.\  } & \mytextarabic{فَلَمَّا جَآءَتْهُمْ ءَايَـٰتُنَا مُبْصِرَةًۭ قَالُوا۟ هَـٰذَا سِحْرٌۭ مُّبِينٌۭ ﴿١٣﴾}\\
\textamh{14.\  } & \mytextarabic{وَجَحَدُوا۟ بِهَا وَٱسْتَيْقَنَتْهَآ أَنفُسُهُمْ ظُلْمًۭا وَعُلُوًّۭا ۚ فَٱنظُرْ كَيْفَ كَانَ عَـٰقِبَةُ ٱلْمُفْسِدِينَ ﴿١٤﴾}\\
\textamh{15.\  } & \mytextarabic{وَلَقَدْ ءَاتَيْنَا دَاوُۥدَ وَسُلَيْمَـٰنَ عِلْمًۭا ۖ وَقَالَا ٱلْحَمْدُ لِلَّهِ ٱلَّذِى فَضَّلَنَا عَلَىٰ كَثِيرٍۢ مِّنْ عِبَادِهِ ٱلْمُؤْمِنِينَ ﴿١٥﴾}\\
\textamh{16.\  } & \mytextarabic{وَوَرِثَ سُلَيْمَـٰنُ دَاوُۥدَ ۖ وَقَالَ يَـٰٓأَيُّهَا ٱلنَّاسُ عُلِّمْنَا مَنطِقَ ٱلطَّيْرِ وَأُوتِينَا مِن كُلِّ شَىْءٍ ۖ إِنَّ هَـٰذَا لَهُوَ ٱلْفَضْلُ ٱلْمُبِينُ ﴿١٦﴾}\\
\textamh{17.\  } & \mytextarabic{وَحُشِرَ لِسُلَيْمَـٰنَ جُنُودُهُۥ مِنَ ٱلْجِنِّ وَٱلْإِنسِ وَٱلطَّيْرِ فَهُمْ يُوزَعُونَ ﴿١٧﴾}\\
\textamh{18.\  } & \mytextarabic{حَتَّىٰٓ إِذَآ أَتَوْا۟ عَلَىٰ وَادِ ٱلنَّمْلِ قَالَتْ نَمْلَةٌۭ يَـٰٓأَيُّهَا ٱلنَّمْلُ ٱدْخُلُوا۟ مَسَـٰكِنَكُمْ لَا يَحْطِمَنَّكُمْ سُلَيْمَـٰنُ وَجُنُودُهُۥ وَهُمْ لَا يَشْعُرُونَ ﴿١٨﴾}\\
\textamh{19.\  } & \mytextarabic{فَتَبَسَّمَ ضَاحِكًۭا مِّن قَوْلِهَا وَقَالَ رَبِّ أَوْزِعْنِىٓ أَنْ أَشْكُرَ نِعْمَتَكَ ٱلَّتِىٓ أَنْعَمْتَ عَلَىَّ وَعَلَىٰ وَٟلِدَىَّ وَأَنْ أَعْمَلَ صَـٰلِحًۭا تَرْضَىٰهُ وَأَدْخِلْنِى بِرَحْمَتِكَ فِى عِبَادِكَ ٱلصَّـٰلِحِينَ ﴿١٩﴾}\\
\textamh{20.\  } & \mytextarabic{وَتَفَقَّدَ ٱلطَّيْرَ فَقَالَ مَا لِىَ لَآ أَرَى ٱلْهُدْهُدَ أَمْ كَانَ مِنَ ٱلْغَآئِبِينَ ﴿٢٠﴾}\\
\textamh{21.\  } & \mytextarabic{لَأُعَذِّبَنَّهُۥ عَذَابًۭا شَدِيدًا أَوْ لَأَا۟ذْبَحَنَّهُۥٓ أَوْ لَيَأْتِيَنِّى بِسُلْطَٰنٍۢ مُّبِينٍۢ ﴿٢١﴾}\\
\textamh{22.\  } & \mytextarabic{فَمَكَثَ غَيْرَ بَعِيدٍۢ فَقَالَ أَحَطتُ بِمَا لَمْ تُحِطْ بِهِۦ وَجِئْتُكَ مِن سَبَإٍۭ بِنَبَإٍۢ يَقِينٍ ﴿٢٢﴾}\\
\textamh{23.\  } & \mytextarabic{إِنِّى وَجَدتُّ ٱمْرَأَةًۭ تَمْلِكُهُمْ وَأُوتِيَتْ مِن كُلِّ شَىْءٍۢ وَلَهَا عَرْشٌ عَظِيمٌۭ ﴿٢٣﴾}\\
\textamh{24.\  } & \mytextarabic{وَجَدتُّهَا وَقَوْمَهَا يَسْجُدُونَ لِلشَّمْسِ مِن دُونِ ٱللَّهِ وَزَيَّنَ لَهُمُ ٱلشَّيْطَٰنُ أَعْمَـٰلَهُمْ فَصَدَّهُمْ عَنِ ٱلسَّبِيلِ فَهُمْ لَا يَهْتَدُونَ ﴿٢٤﴾}\\
\textamh{25.\  } & \mytextarabic{أَلَّا يَسْجُدُوا۟ لِلَّهِ ٱلَّذِى يُخْرِجُ ٱلْخَبْءَ فِى ٱلسَّمَـٰوَٟتِ وَٱلْأَرْضِ وَيَعْلَمُ مَا تُخْفُونَ وَمَا تُعْلِنُونَ ﴿٢٥﴾}\\
\textamh{26.\  } & \mytextarabic{ٱللَّهُ لَآ إِلَـٰهَ إِلَّا هُوَ رَبُّ ٱلْعَرْشِ ٱلْعَظِيمِ ۩ ﴿٢٦﴾}\\
\textamh{27.\  } & \mytextarabic{۞ قَالَ سَنَنظُرُ أَصَدَقْتَ أَمْ كُنتَ مِنَ ٱلْكَـٰذِبِينَ ﴿٢٧﴾}\\
\textamh{28.\  } & \mytextarabic{ٱذْهَب بِّكِتَـٰبِى هَـٰذَا فَأَلْقِهْ إِلَيْهِمْ ثُمَّ تَوَلَّ عَنْهُمْ فَٱنظُرْ مَاذَا يَرْجِعُونَ ﴿٢٨﴾}\\
\textamh{29.\  } & \mytextarabic{قَالَتْ يَـٰٓأَيُّهَا ٱلْمَلَؤُا۟ إِنِّىٓ أُلْقِىَ إِلَىَّ كِتَـٰبٌۭ كَرِيمٌ ﴿٢٩﴾}\\
\textamh{30.\  } & \mytextarabic{إِنَّهُۥ مِن سُلَيْمَـٰنَ وَإِنَّهُۥ  ﴿٣٠﴾}\\
\textamh{31.\  } & \mytextarabic{أَلَّا تَعْلُوا۟ عَلَىَّ وَأْتُونِى مُسْلِمِينَ ﴿٣١﴾}\\
\textamh{32.\  } & \mytextarabic{قَالَتْ يَـٰٓأَيُّهَا ٱلْمَلَؤُا۟ أَفْتُونِى فِىٓ أَمْرِى مَا كُنتُ قَاطِعَةً أَمْرًا حَتَّىٰ تَشْهَدُونِ ﴿٣٢﴾}\\
\textamh{33.\  } & \mytextarabic{قَالُوا۟ نَحْنُ أُو۟لُوا۟ قُوَّةٍۢ وَأُو۟لُوا۟ بَأْسٍۢ شَدِيدٍۢ وَٱلْأَمْرُ إِلَيْكِ فَٱنظُرِى مَاذَا تَأْمُرِينَ ﴿٣٣﴾}\\
\textamh{34.\  } & \mytextarabic{قَالَتْ إِنَّ ٱلْمُلُوكَ إِذَا دَخَلُوا۟ قَرْيَةً أَفْسَدُوهَا وَجَعَلُوٓا۟ أَعِزَّةَ أَهْلِهَآ أَذِلَّةًۭ ۖ وَكَذَٟلِكَ يَفْعَلُونَ ﴿٣٤﴾}\\
\textamh{35.\  } & \mytextarabic{وَإِنِّى مُرْسِلَةٌ إِلَيْهِم بِهَدِيَّةٍۢ فَنَاظِرَةٌۢ بِمَ يَرْجِعُ ٱلْمُرْسَلُونَ ﴿٣٥﴾}\\
\textamh{36.\  } & \mytextarabic{فَلَمَّا جَآءَ سُلَيْمَـٰنَ قَالَ أَتُمِدُّونَنِ بِمَالٍۢ فَمَآ ءَاتَىٰنِۦَ ٱللَّهُ خَيْرٌۭ مِّمَّآ ءَاتَىٰكُم بَلْ أَنتُم بِهَدِيَّتِكُمْ تَفْرَحُونَ ﴿٣٦﴾}\\
\textamh{37.\  } & \mytextarabic{ٱرْجِعْ إِلَيْهِمْ فَلَنَأْتِيَنَّهُم بِجُنُودٍۢ لَّا قِبَلَ لَهُم بِهَا وَلَنُخْرِجَنَّهُم مِّنْهَآ أَذِلَّةًۭ وَهُمْ صَـٰغِرُونَ ﴿٣٧﴾}\\
\textamh{38.\  } & \mytextarabic{قَالَ يَـٰٓأَيُّهَا ٱلْمَلَؤُا۟ أَيُّكُمْ يَأْتِينِى بِعَرْشِهَا قَبْلَ أَن يَأْتُونِى مُسْلِمِينَ ﴿٣٨﴾}\\
\textamh{39.\  } & \mytextarabic{قَالَ عِفْرِيتٌۭ مِّنَ ٱلْجِنِّ أَنَا۠ ءَاتِيكَ بِهِۦ قَبْلَ أَن تَقُومَ مِن مَّقَامِكَ ۖ وَإِنِّى عَلَيْهِ لَقَوِىٌّ أَمِينٌۭ ﴿٣٩﴾}\\
\textamh{40.\  } & \mytextarabic{قَالَ ٱلَّذِى عِندَهُۥ عِلْمٌۭ مِّنَ ٱلْكِتَـٰبِ أَنَا۠ ءَاتِيكَ بِهِۦ قَبْلَ أَن يَرْتَدَّ إِلَيْكَ طَرْفُكَ ۚ فَلَمَّا رَءَاهُ مُسْتَقِرًّا عِندَهُۥ قَالَ هَـٰذَا مِن فَضْلِ رَبِّى لِيَبْلُوَنِىٓ ءَأَشْكُرُ أَمْ أَكْفُرُ ۖ وَمَن شَكَرَ فَإِنَّمَا يَشْكُرُ لِنَفْسِهِۦ ۖ وَمَن كَفَرَ فَإِنَّ رَبِّى غَنِىٌّۭ كَرِيمٌۭ ﴿٤٠﴾}\\
\textamh{41.\  } & \mytextarabic{قَالَ نَكِّرُوا۟ لَهَا عَرْشَهَا نَنظُرْ أَتَهْتَدِىٓ أَمْ تَكُونُ مِنَ ٱلَّذِينَ لَا يَهْتَدُونَ ﴿٤١﴾}\\
\textamh{42.\  } & \mytextarabic{فَلَمَّا جَآءَتْ قِيلَ أَهَـٰكَذَا عَرْشُكِ ۖ قَالَتْ كَأَنَّهُۥ هُوَ ۚ وَأُوتِينَا ٱلْعِلْمَ مِن قَبْلِهَا وَكُنَّا مُسْلِمِينَ ﴿٤٢﴾}\\
\textamh{43.\  } & \mytextarabic{وَصَدَّهَا مَا كَانَت تَّعْبُدُ مِن دُونِ ٱللَّهِ ۖ إِنَّهَا كَانَتْ مِن قَوْمٍۢ كَـٰفِرِينَ ﴿٤٣﴾}\\
\textamh{44.\  } & \mytextarabic{قِيلَ لَهَا ٱدْخُلِى ٱلصَّرْحَ ۖ فَلَمَّا رَأَتْهُ حَسِبَتْهُ لُجَّةًۭ وَكَشَفَتْ عَن سَاقَيْهَا ۚ قَالَ إِنَّهُۥ صَرْحٌۭ مُّمَرَّدٌۭ مِّن قَوَارِيرَ ۗ قَالَتْ رَبِّ إِنِّى ظَلَمْتُ نَفْسِى وَأَسْلَمْتُ مَعَ سُلَيْمَـٰنَ لِلَّهِ رَبِّ ٱلْعَـٰلَمِينَ ﴿٤٤﴾}\\
\textamh{45.\  } & \mytextarabic{وَلَقَدْ أَرْسَلْنَآ إِلَىٰ ثَمُودَ أَخَاهُمْ صَـٰلِحًا أَنِ ٱعْبُدُوا۟ ٱللَّهَ فَإِذَا هُمْ فَرِيقَانِ يَخْتَصِمُونَ ﴿٤٥﴾}\\
\textamh{46.\  } & \mytextarabic{قَالَ يَـٰقَوْمِ لِمَ تَسْتَعْجِلُونَ بِٱلسَّيِّئَةِ قَبْلَ ٱلْحَسَنَةِ ۖ لَوْلَا تَسْتَغْفِرُونَ ٱللَّهَ لَعَلَّكُمْ تُرْحَمُونَ ﴿٤٦﴾}\\
\textamh{47.\  } & \mytextarabic{قَالُوا۟ ٱطَّيَّرْنَا بِكَ وَبِمَن مَّعَكَ ۚ قَالَ طَٰٓئِرُكُمْ عِندَ ٱللَّهِ ۖ بَلْ أَنتُمْ قَوْمٌۭ تُفْتَنُونَ ﴿٤٧﴾}\\
\textamh{48.\  } & \mytextarabic{وَكَانَ فِى ٱلْمَدِينَةِ تِسْعَةُ رَهْطٍۢ يُفْسِدُونَ فِى ٱلْأَرْضِ وَلَا يُصْلِحُونَ ﴿٤٨﴾}\\
\textamh{49.\  } & \mytextarabic{قَالُوا۟ تَقَاسَمُوا۟ بِٱللَّهِ لَنُبَيِّتَنَّهُۥ وَأَهْلَهُۥ ثُمَّ لَنَقُولَنَّ لِوَلِيِّهِۦ مَا شَهِدْنَا مَهْلِكَ أَهْلِهِۦ وَإِنَّا لَصَـٰدِقُونَ ﴿٤٩﴾}\\
\textamh{50.\  } & \mytextarabic{وَمَكَرُوا۟ مَكْرًۭا وَمَكَرْنَا مَكْرًۭا وَهُمْ لَا يَشْعُرُونَ ﴿٥٠﴾}\\
\textamh{51.\  } & \mytextarabic{فَٱنظُرْ كَيْفَ كَانَ عَـٰقِبَةُ مَكْرِهِمْ أَنَّا دَمَّرْنَـٰهُمْ وَقَوْمَهُمْ أَجْمَعِينَ ﴿٥١﴾}\\
\textamh{52.\  } & \mytextarabic{فَتِلْكَ بُيُوتُهُمْ خَاوِيَةًۢ بِمَا ظَلَمُوٓا۟ ۗ إِنَّ فِى ذَٟلِكَ لَءَايَةًۭ لِّقَوْمٍۢ يَعْلَمُونَ ﴿٥٢﴾}\\
\textamh{53.\  } & \mytextarabic{وَأَنجَيْنَا ٱلَّذِينَ ءَامَنُوا۟ وَكَانُوا۟ يَتَّقُونَ ﴿٥٣﴾}\\
\textamh{54.\  } & \mytextarabic{وَلُوطًا إِذْ قَالَ لِقَوْمِهِۦٓ أَتَأْتُونَ ٱلْفَـٰحِشَةَ وَأَنتُمْ تُبْصِرُونَ ﴿٥٤﴾}\\
\textamh{55.\  } & \mytextarabic{أَئِنَّكُمْ لَتَأْتُونَ ٱلرِّجَالَ شَهْوَةًۭ مِّن دُونِ ٱلنِّسَآءِ ۚ بَلْ أَنتُمْ قَوْمٌۭ تَجْهَلُونَ ﴿٥٥﴾}\\
\textamh{56.\  } & \mytextarabic{۞ فَمَا كَانَ جَوَابَ قَوْمِهِۦٓ إِلَّآ أَن قَالُوٓا۟ أَخْرِجُوٓا۟ ءَالَ لُوطٍۢ مِّن قَرْيَتِكُمْ ۖ إِنَّهُمْ أُنَاسٌۭ يَتَطَهَّرُونَ ﴿٥٦﴾}\\
\textamh{57.\  } & \mytextarabic{فَأَنجَيْنَـٰهُ وَأَهْلَهُۥٓ إِلَّا ٱمْرَأَتَهُۥ قَدَّرْنَـٰهَا مِنَ ٱلْغَٰبِرِينَ ﴿٥٧﴾}\\
\textamh{58.\  } & \mytextarabic{وَأَمْطَرْنَا عَلَيْهِم مَّطَرًۭا ۖ فَسَآءَ مَطَرُ ٱلْمُنذَرِينَ ﴿٥٨﴾}\\
\textamh{59.\  } & \mytextarabic{قُلِ ٱلْحَمْدُ لِلَّهِ وَسَلَـٰمٌ عَلَىٰ عِبَادِهِ ٱلَّذِينَ ٱصْطَفَىٰٓ ۗ ءَآللَّهُ خَيْرٌ أَمَّا يُشْرِكُونَ ﴿٥٩﴾}\\
\textamh{60.\  } & \mytextarabic{أَمَّنْ خَلَقَ ٱلسَّمَـٰوَٟتِ وَٱلْأَرْضَ وَأَنزَلَ لَكُم مِّنَ ٱلسَّمَآءِ مَآءًۭ فَأَنۢبَتْنَا بِهِۦ حَدَآئِقَ ذَاتَ بَهْجَةٍۢ مَّا كَانَ لَكُمْ أَن تُنۢبِتُوا۟ شَجَرَهَآ ۗ أَءِلَـٰهٌۭ مَّعَ ٱللَّهِ ۚ بَلْ هُمْ قَوْمٌۭ يَعْدِلُونَ ﴿٦٠﴾}\\
\textamh{61.\  } & \mytextarabic{أَمَّن جَعَلَ ٱلْأَرْضَ قَرَارًۭا وَجَعَلَ خِلَـٰلَهَآ أَنْهَـٰرًۭا وَجَعَلَ لَهَا رَوَٟسِىَ وَجَعَلَ بَيْنَ ٱلْبَحْرَيْنِ حَاجِزًا ۗ أَءِلَـٰهٌۭ مَّعَ ٱللَّهِ ۚ بَلْ أَكْثَرُهُمْ لَا يَعْلَمُونَ ﴿٦١﴾}\\
\textamh{62.\  } & \mytextarabic{أَمَّن يُجِيبُ ٱلْمُضْطَرَّ إِذَا دَعَاهُ وَيَكْشِفُ ٱلسُّوٓءَ وَيَجْعَلُكُمْ خُلَفَآءَ ٱلْأَرْضِ ۗ أَءِلَـٰهٌۭ مَّعَ ٱللَّهِ ۚ قَلِيلًۭا مَّا تَذَكَّرُونَ ﴿٦٢﴾}\\
\textamh{63.\  } & \mytextarabic{أَمَّن يَهْدِيكُمْ فِى ظُلُمَـٰتِ ٱلْبَرِّ وَٱلْبَحْرِ وَمَن يُرْسِلُ ٱلرِّيَـٰحَ بُشْرًۢا بَيْنَ يَدَىْ رَحْمَتِهِۦٓ ۗ أَءِلَـٰهٌۭ مَّعَ ٱللَّهِ ۚ تَعَـٰلَى ٱللَّهُ عَمَّا يُشْرِكُونَ ﴿٦٣﴾}\\
\textamh{64.\  } & \mytextarabic{أَمَّن يَبْدَؤُا۟ ٱلْخَلْقَ ثُمَّ يُعِيدُهُۥ وَمَن يَرْزُقُكُم مِّنَ ٱلسَّمَآءِ وَٱلْأَرْضِ ۗ أَءِلَـٰهٌۭ مَّعَ ٱللَّهِ ۚ قُلْ هَاتُوا۟ بُرْهَـٰنَكُمْ إِن كُنتُمْ صَـٰدِقِينَ ﴿٦٤﴾}\\
\textamh{65.\  } & \mytextarabic{قُل لَّا يَعْلَمُ مَن فِى ٱلسَّمَـٰوَٟتِ وَٱلْأَرْضِ ٱلْغَيْبَ إِلَّا ٱللَّهُ ۚ وَمَا يَشْعُرُونَ أَيَّانَ يُبْعَثُونَ ﴿٦٥﴾}\\
\textamh{66.\  } & \mytextarabic{بَلِ ٱدَّٰرَكَ عِلْمُهُمْ فِى ٱلْءَاخِرَةِ ۚ بَلْ هُمْ فِى شَكٍّۢ مِّنْهَا ۖ بَلْ هُم مِّنْهَا عَمُونَ ﴿٦٦﴾}\\
\textamh{67.\  } & \mytextarabic{وَقَالَ ٱلَّذِينَ كَفَرُوٓا۟ أَءِذَا كُنَّا تُرَٰبًۭا وَءَابَآؤُنَآ أَئِنَّا لَمُخْرَجُونَ ﴿٦٧﴾}\\
\textamh{68.\  } & \mytextarabic{لَقَدْ وُعِدْنَا هَـٰذَا نَحْنُ وَءَابَآؤُنَا مِن قَبْلُ إِنْ هَـٰذَآ إِلَّآ أَسَـٰطِيرُ ٱلْأَوَّلِينَ ﴿٦٨﴾}\\
\textamh{69.\  } & \mytextarabic{قُلْ سِيرُوا۟ فِى ٱلْأَرْضِ فَٱنظُرُوا۟ كَيْفَ كَانَ عَـٰقِبَةُ ٱلْمُجْرِمِينَ ﴿٦٩﴾}\\
\textamh{70.\  } & \mytextarabic{وَلَا تَحْزَنْ عَلَيْهِمْ وَلَا تَكُن فِى ضَيْقٍۢ مِّمَّا يَمْكُرُونَ ﴿٧٠﴾}\\
\textamh{71.\  } & \mytextarabic{وَيَقُولُونَ مَتَىٰ هَـٰذَا ٱلْوَعْدُ إِن كُنتُمْ صَـٰدِقِينَ ﴿٧١﴾}\\
\textamh{72.\  } & \mytextarabic{قُلْ عَسَىٰٓ أَن يَكُونَ رَدِفَ لَكُم بَعْضُ ٱلَّذِى تَسْتَعْجِلُونَ ﴿٧٢﴾}\\
\textamh{73.\  } & \mytextarabic{وَإِنَّ رَبَّكَ لَذُو فَضْلٍ عَلَى ٱلنَّاسِ وَلَـٰكِنَّ أَكْثَرَهُمْ لَا يَشْكُرُونَ ﴿٧٣﴾}\\
\textamh{74.\  } & \mytextarabic{وَإِنَّ رَبَّكَ لَيَعْلَمُ مَا تُكِنُّ صُدُورُهُمْ وَمَا يُعْلِنُونَ ﴿٧٤﴾}\\
\textamh{75.\  } & \mytextarabic{وَمَا مِنْ غَآئِبَةٍۢ فِى ٱلسَّمَآءِ وَٱلْأَرْضِ إِلَّا فِى كِتَـٰبٍۢ مُّبِينٍ ﴿٧٥﴾}\\
\textamh{76.\  } & \mytextarabic{إِنَّ هَـٰذَا ٱلْقُرْءَانَ يَقُصُّ عَلَىٰ بَنِىٓ إِسْرَٰٓءِيلَ أَكْثَرَ ٱلَّذِى هُمْ فِيهِ يَخْتَلِفُونَ ﴿٧٦﴾}\\
\textamh{77.\  } & \mytextarabic{وَإِنَّهُۥ لَهُدًۭى وَرَحْمَةٌۭ لِّلْمُؤْمِنِينَ ﴿٧٧﴾}\\
\textamh{78.\  } & \mytextarabic{إِنَّ رَبَّكَ يَقْضِى بَيْنَهُم بِحُكْمِهِۦ ۚ وَهُوَ ٱلْعَزِيزُ ٱلْعَلِيمُ ﴿٧٨﴾}\\
\textamh{79.\  } & \mytextarabic{فَتَوَكَّلْ عَلَى ٱللَّهِ ۖ إِنَّكَ عَلَى ٱلْحَقِّ ٱلْمُبِينِ ﴿٧٩﴾}\\
\textamh{80.\  } & \mytextarabic{إِنَّكَ لَا تُسْمِعُ ٱلْمَوْتَىٰ وَلَا تُسْمِعُ ٱلصُّمَّ ٱلدُّعَآءَ إِذَا وَلَّوْا۟ مُدْبِرِينَ ﴿٨٠﴾}\\
\textamh{81.\  } & \mytextarabic{وَمَآ أَنتَ بِهَـٰدِى ٱلْعُمْىِ عَن ضَلَـٰلَتِهِمْ ۖ إِن تُسْمِعُ إِلَّا مَن يُؤْمِنُ بِـَٔايَـٰتِنَا فَهُم مُّسْلِمُونَ ﴿٨١﴾}\\
\textamh{82.\  } & \mytextarabic{۞ وَإِذَا وَقَعَ ٱلْقَوْلُ عَلَيْهِمْ أَخْرَجْنَا لَهُمْ دَآبَّةًۭ مِّنَ ٱلْأَرْضِ تُكَلِّمُهُمْ أَنَّ ٱلنَّاسَ كَانُوا۟ بِـَٔايَـٰتِنَا لَا يُوقِنُونَ ﴿٨٢﴾}\\
\textamh{83.\  } & \mytextarabic{وَيَوْمَ نَحْشُرُ مِن كُلِّ أُمَّةٍۢ فَوْجًۭا مِّمَّن يُكَذِّبُ بِـَٔايَـٰتِنَا فَهُمْ يُوزَعُونَ ﴿٨٣﴾}\\
\textamh{84.\  } & \mytextarabic{حَتَّىٰٓ إِذَا جَآءُو قَالَ أَكَذَّبْتُم بِـَٔايَـٰتِى وَلَمْ تُحِيطُوا۟ بِهَا عِلْمًا أَمَّاذَا كُنتُمْ تَعْمَلُونَ ﴿٨٤﴾}\\
\textamh{85.\  } & \mytextarabic{وَوَقَعَ ٱلْقَوْلُ عَلَيْهِم بِمَا ظَلَمُوا۟ فَهُمْ لَا يَنطِقُونَ ﴿٨٥﴾}\\
\textamh{86.\  } & \mytextarabic{أَلَمْ يَرَوْا۟ أَنَّا جَعَلْنَا ٱلَّيْلَ لِيَسْكُنُوا۟ فِيهِ وَٱلنَّهَارَ مُبْصِرًا ۚ إِنَّ فِى ذَٟلِكَ لَءَايَـٰتٍۢ لِّقَوْمٍۢ يُؤْمِنُونَ ﴿٨٦﴾}\\
\textamh{87.\  } & \mytextarabic{وَيَوْمَ يُنفَخُ فِى ٱلصُّورِ فَفَزِعَ مَن فِى ٱلسَّمَـٰوَٟتِ وَمَن فِى ٱلْأَرْضِ إِلَّا مَن شَآءَ ٱللَّهُ ۚ وَكُلٌّ أَتَوْهُ دَٟخِرِينَ ﴿٨٧﴾}\\
\textamh{88.\  } & \mytextarabic{وَتَرَى ٱلْجِبَالَ تَحْسَبُهَا جَامِدَةًۭ وَهِىَ تَمُرُّ مَرَّ ٱلسَّحَابِ ۚ صُنْعَ ٱللَّهِ ٱلَّذِىٓ أَتْقَنَ كُلَّ شَىْءٍ ۚ إِنَّهُۥ خَبِيرٌۢ بِمَا تَفْعَلُونَ ﴿٨٨﴾}\\
\textamh{89.\  } & \mytextarabic{مَن جَآءَ بِٱلْحَسَنَةِ فَلَهُۥ خَيْرٌۭ مِّنْهَا وَهُم مِّن فَزَعٍۢ يَوْمَئِذٍ ءَامِنُونَ ﴿٨٩﴾}\\
\textamh{90.\  } & \mytextarabic{وَمَن جَآءَ بِٱلسَّيِّئَةِ فَكُبَّتْ وُجُوهُهُمْ فِى ٱلنَّارِ هَلْ تُجْزَوْنَ إِلَّا مَا كُنتُمْ تَعْمَلُونَ ﴿٩٠﴾}\\
\textamh{91.\  } & \mytextarabic{إِنَّمَآ أُمِرْتُ أَنْ أَعْبُدَ رَبَّ هَـٰذِهِ ٱلْبَلْدَةِ ٱلَّذِى حَرَّمَهَا وَلَهُۥ كُلُّ شَىْءٍۢ ۖ وَأُمِرْتُ أَنْ أَكُونَ مِنَ ٱلْمُسْلِمِينَ ﴿٩١﴾}\\
\textamh{92.\  } & \mytextarabic{وَأَنْ أَتْلُوَا۟ ٱلْقُرْءَانَ ۖ فَمَنِ ٱهْتَدَىٰ فَإِنَّمَا يَهْتَدِى لِنَفْسِهِۦ ۖ وَمَن ضَلَّ فَقُلْ إِنَّمَآ أَنَا۠ مِنَ ٱلْمُنذِرِينَ ﴿٩٢﴾}\\
\textamh{93.\  } & \mytextarabic{وَقُلِ ٱلْحَمْدُ لِلَّهِ سَيُرِيكُمْ ءَايَـٰتِهِۦ فَتَعْرِفُونَهَا ۚ وَمَا رَبُّكَ بِغَٰفِلٍ عَمَّا تَعْمَلُونَ ﴿٩٣﴾}\\
\end{longtable}
\clearpage
%% License: BSD style (Berkley) (i.e. Put the Copyright owner's name always)
%% Writer and Copyright (to): Bewketu(Bilal) Tadilo (2016-17)
\begin{center}\section{\LR{\textamhsec{ሱራቱ አልቀሰስ -}  \textarabic{سوره  القصص}}}\end{center}
\begin{longtable}{%
  @{}
    p{.5\textwidth}
  @{~~~}
    p{.5\textwidth}
    @{}
}
\textamh{ቢስሚላሂ አራህመኒ ራሂይም } &  \mytextarabic{بِسْمِ ٱللَّهِ ٱلرَّحْمَـٰنِ ٱلرَّحِيمِ}\\
\textamh{1.\  } & \mytextarabic{ طسٓمٓ ﴿١﴾}\\
\textamh{2.\  } & \mytextarabic{تِلْكَ ءَايَـٰتُ ٱلْكِتَـٰبِ ٱلْمُبِينِ ﴿٢﴾}\\
\textamh{3.\  } & \mytextarabic{نَتْلُوا۟ عَلَيْكَ مِن نَّبَإِ مُوسَىٰ وَفِرْعَوْنَ بِٱلْحَقِّ لِقَوْمٍۢ يُؤْمِنُونَ ﴿٣﴾}\\
\textamh{4.\  } & \mytextarabic{إِنَّ فِرْعَوْنَ عَلَا فِى ٱلْأَرْضِ وَجَعَلَ أَهْلَهَا شِيَعًۭا يَسْتَضْعِفُ طَآئِفَةًۭ مِّنْهُمْ يُذَبِّحُ أَبْنَآءَهُمْ وَيَسْتَحْىِۦ نِسَآءَهُمْ ۚ إِنَّهُۥ كَانَ مِنَ ٱلْمُفْسِدِينَ ﴿٤﴾}\\
\textamh{5.\  } & \mytextarabic{وَنُرِيدُ أَن نَّمُنَّ عَلَى ٱلَّذِينَ ٱسْتُضْعِفُوا۟ فِى ٱلْأَرْضِ وَنَجْعَلَهُمْ أَئِمَّةًۭ وَنَجْعَلَهُمُ ٱلْوَٟرِثِينَ ﴿٥﴾}\\
\textamh{6.\  } & \mytextarabic{وَنُمَكِّنَ لَهُمْ فِى ٱلْأَرْضِ وَنُرِىَ فِرْعَوْنَ وَهَـٰمَـٰنَ وَجُنُودَهُمَا مِنْهُم مَّا كَانُوا۟ يَحْذَرُونَ ﴿٦﴾}\\
\textamh{7.\  } & \mytextarabic{وَأَوْحَيْنَآ إِلَىٰٓ أُمِّ مُوسَىٰٓ أَنْ أَرْضِعِيهِ ۖ فَإِذَا خِفْتِ عَلَيْهِ فَأَلْقِيهِ فِى ٱلْيَمِّ وَلَا تَخَافِى وَلَا تَحْزَنِىٓ ۖ إِنَّا رَآدُّوهُ إِلَيْكِ وَجَاعِلُوهُ مِنَ ٱلْمُرْسَلِينَ ﴿٧﴾}\\
\textamh{8.\  } & \mytextarabic{فَٱلْتَقَطَهُۥٓ ءَالُ فِرْعَوْنَ لِيَكُونَ لَهُمْ عَدُوًّۭا وَحَزَنًا ۗ إِنَّ فِرْعَوْنَ وَهَـٰمَـٰنَ وَجُنُودَهُمَا كَانُوا۟ خَـٰطِـِٔينَ ﴿٨﴾}\\
\textamh{9.\  } & \mytextarabic{وَقَالَتِ ٱمْرَأَتُ فِرْعَوْنَ قُرَّتُ عَيْنٍۢ لِّى وَلَكَ ۖ لَا تَقْتُلُوهُ عَسَىٰٓ أَن يَنفَعَنَآ أَوْ نَتَّخِذَهُۥ وَلَدًۭا وَهُمْ لَا يَشْعُرُونَ ﴿٩﴾}\\
\textamh{10.\  } & \mytextarabic{وَأَصْبَحَ فُؤَادُ أُمِّ مُوسَىٰ فَـٰرِغًا ۖ إِن كَادَتْ لَتُبْدِى بِهِۦ لَوْلَآ أَن رَّبَطْنَا عَلَىٰ قَلْبِهَا لِتَكُونَ مِنَ ٱلْمُؤْمِنِينَ ﴿١٠﴾}\\
\textamh{11.\  } & \mytextarabic{وَقَالَتْ لِأُخْتِهِۦ قُصِّيهِ ۖ فَبَصُرَتْ بِهِۦ عَن جُنُبٍۢ وَهُمْ لَا يَشْعُرُونَ ﴿١١﴾}\\
\textamh{12.\  } & \mytextarabic{۞ وَحَرَّمْنَا عَلَيْهِ ٱلْمَرَاضِعَ مِن قَبْلُ فَقَالَتْ هَلْ أَدُلُّكُمْ عَلَىٰٓ أَهْلِ بَيْتٍۢ يَكْفُلُونَهُۥ لَكُمْ وَهُمْ لَهُۥ نَـٰصِحُونَ ﴿١٢﴾}\\
\textamh{13.\  } & \mytextarabic{فَرَدَدْنَـٰهُ إِلَىٰٓ أُمِّهِۦ كَىْ تَقَرَّ عَيْنُهَا وَلَا تَحْزَنَ وَلِتَعْلَمَ أَنَّ وَعْدَ ٱللَّهِ حَقٌّۭ وَلَـٰكِنَّ أَكْثَرَهُمْ لَا يَعْلَمُونَ ﴿١٣﴾}\\
\textamh{14.\  } & \mytextarabic{وَلَمَّا بَلَغَ أَشُدَّهُۥ وَٱسْتَوَىٰٓ ءَاتَيْنَـٰهُ حُكْمًۭا وَعِلْمًۭا ۚ وَكَذَٟلِكَ نَجْزِى ٱلْمُحْسِنِينَ ﴿١٤﴾}\\
\textamh{15.\  } & \mytextarabic{وَدَخَلَ ٱلْمَدِينَةَ عَلَىٰ حِينِ غَفْلَةٍۢ مِّنْ أَهْلِهَا فَوَجَدَ فِيهَا رَجُلَيْنِ يَقْتَتِلَانِ هَـٰذَا مِن شِيعَتِهِۦ وَهَـٰذَا مِنْ عَدُوِّهِۦ ۖ فَٱسْتَغَٰثَهُ ٱلَّذِى مِن شِيعَتِهِۦ عَلَى ٱلَّذِى مِنْ عَدُوِّهِۦ فَوَكَزَهُۥ مُوسَىٰ فَقَضَىٰ عَلَيْهِ ۖ قَالَ هَـٰذَا مِنْ عَمَلِ ٱلشَّيْطَٰنِ ۖ إِنَّهُۥ عَدُوٌّۭ مُّضِلٌّۭ مُّبِينٌۭ ﴿١٥﴾}\\
\textamh{16.\  } & \mytextarabic{قَالَ رَبِّ إِنِّى ظَلَمْتُ نَفْسِى فَٱغْفِرْ لِى فَغَفَرَ لَهُۥٓ ۚ إِنَّهُۥ هُوَ ٱلْغَفُورُ ٱلرَّحِيمُ ﴿١٦﴾}\\
\textamh{17.\  } & \mytextarabic{قَالَ رَبِّ بِمَآ أَنْعَمْتَ عَلَىَّ فَلَنْ أَكُونَ ظَهِيرًۭا لِّلْمُجْرِمِينَ ﴿١٧﴾}\\
\textamh{18.\  } & \mytextarabic{فَأَصْبَحَ فِى ٱلْمَدِينَةِ خَآئِفًۭا يَتَرَقَّبُ فَإِذَا ٱلَّذِى ٱسْتَنصَرَهُۥ بِٱلْأَمْسِ يَسْتَصْرِخُهُۥ ۚ قَالَ لَهُۥ مُوسَىٰٓ إِنَّكَ لَغَوِىٌّۭ مُّبِينٌۭ ﴿١٨﴾}\\
\textamh{19.\  } & \mytextarabic{فَلَمَّآ أَنْ أَرَادَ أَن يَبْطِشَ بِٱلَّذِى هُوَ عَدُوٌّۭ لَّهُمَا قَالَ يَـٰمُوسَىٰٓ أَتُرِيدُ أَن تَقْتُلَنِى كَمَا قَتَلْتَ نَفْسًۢا بِٱلْأَمْسِ ۖ إِن تُرِيدُ إِلَّآ أَن تَكُونَ جَبَّارًۭا فِى ٱلْأَرْضِ وَمَا تُرِيدُ أَن تَكُونَ مِنَ ٱلْمُصْلِحِينَ ﴿١٩﴾}\\
\textamh{20.\  } & \mytextarabic{وَجَآءَ رَجُلٌۭ مِّنْ أَقْصَا ٱلْمَدِينَةِ يَسْعَىٰ قَالَ يَـٰمُوسَىٰٓ إِنَّ ٱلْمَلَأَ يَأْتَمِرُونَ بِكَ لِيَقْتُلُوكَ فَٱخْرُجْ إِنِّى لَكَ مِنَ ٱلنَّـٰصِحِينَ ﴿٢٠﴾}\\
\textamh{21.\  } & \mytextarabic{فَخَرَجَ مِنْهَا خَآئِفًۭا يَتَرَقَّبُ ۖ قَالَ رَبِّ نَجِّنِى مِنَ ٱلْقَوْمِ ٱلظَّـٰلِمِينَ ﴿٢١﴾}\\
\textamh{22.\  } & \mytextarabic{وَلَمَّا تَوَجَّهَ تِلْقَآءَ مَدْيَنَ قَالَ عَسَىٰ رَبِّىٓ أَن يَهْدِيَنِى سَوَآءَ ٱلسَّبِيلِ ﴿٢٢﴾}\\
\textamh{23.\  } & \mytextarabic{وَلَمَّا وَرَدَ مَآءَ مَدْيَنَ وَجَدَ عَلَيْهِ أُمَّةًۭ مِّنَ ٱلنَّاسِ يَسْقُونَ وَوَجَدَ مِن دُونِهِمُ ٱمْرَأَتَيْنِ تَذُودَانِ ۖ قَالَ مَا خَطْبُكُمَا ۖ قَالَتَا لَا نَسْقِى حَتَّىٰ يُصْدِرَ ٱلرِّعَآءُ ۖ وَأَبُونَا شَيْخٌۭ كَبِيرٌۭ ﴿٢٣﴾}\\
\textamh{24.\  } & \mytextarabic{فَسَقَىٰ لَهُمَا ثُمَّ تَوَلَّىٰٓ إِلَى ٱلظِّلِّ فَقَالَ رَبِّ إِنِّى لِمَآ أَنزَلْتَ إِلَىَّ مِنْ خَيْرٍۢ فَقِيرٌۭ ﴿٢٤﴾}\\
\textamh{25.\  } & \mytextarabic{فَجَآءَتْهُ إِحْدَىٰهُمَا تَمْشِى عَلَى ٱسْتِحْيَآءٍۢ قَالَتْ إِنَّ أَبِى يَدْعُوكَ لِيَجْزِيَكَ أَجْرَ مَا سَقَيْتَ لَنَا ۚ فَلَمَّا جَآءَهُۥ وَقَصَّ عَلَيْهِ ٱلْقَصَصَ قَالَ لَا تَخَفْ ۖ نَجَوْتَ مِنَ ٱلْقَوْمِ ٱلظَّـٰلِمِينَ ﴿٢٥﴾}\\
\textamh{26.\  } & \mytextarabic{قَالَتْ إِحْدَىٰهُمَا يَـٰٓأَبَتِ ٱسْتَـْٔجِرْهُ ۖ إِنَّ خَيْرَ مَنِ ٱسْتَـْٔجَرْتَ ٱلْقَوِىُّ ٱلْأَمِينُ ﴿٢٦﴾}\\
\textamh{27.\  } & \mytextarabic{قَالَ إِنِّىٓ أُرِيدُ أَنْ أُنكِحَكَ إِحْدَى ٱبْنَتَىَّ هَـٰتَيْنِ عَلَىٰٓ أَن تَأْجُرَنِى ثَمَـٰنِىَ حِجَجٍۢ ۖ فَإِنْ أَتْمَمْتَ عَشْرًۭا فَمِنْ عِندِكَ ۖ وَمَآ أُرِيدُ أَنْ أَشُقَّ عَلَيْكَ ۚ سَتَجِدُنِىٓ إِن شَآءَ ٱللَّهُ مِنَ ٱلصَّـٰلِحِينَ ﴿٢٧﴾}\\
\textamh{28.\  } & \mytextarabic{قَالَ ذَٟلِكَ بَيْنِى وَبَيْنَكَ ۖ أَيَّمَا ٱلْأَجَلَيْنِ قَضَيْتُ فَلَا عُدْوَٟنَ عَلَىَّ ۖ وَٱللَّهُ عَلَىٰ مَا نَقُولُ وَكِيلٌۭ ﴿٢٨﴾}\\
\textamh{29.\  } & \mytextarabic{۞ فَلَمَّا قَضَىٰ مُوسَى ٱلْأَجَلَ وَسَارَ بِأَهْلِهِۦٓ ءَانَسَ مِن جَانِبِ ٱلطُّورِ نَارًۭا قَالَ لِأَهْلِهِ ٱمْكُثُوٓا۟ إِنِّىٓ ءَانَسْتُ نَارًۭا لَّعَلِّىٓ ءَاتِيكُم مِّنْهَا بِخَبَرٍ أَوْ جَذْوَةٍۢ مِّنَ ٱلنَّارِ لَعَلَّكُمْ تَصْطَلُونَ ﴿٢٩﴾}\\
\textamh{30.\  } & \mytextarabic{فَلَمَّآ أَتَىٰهَا نُودِىَ مِن شَـٰطِئِ ٱلْوَادِ ٱلْأَيْمَنِ فِى ٱلْبُقْعَةِ ٱلْمُبَٰرَكَةِ مِنَ ٱلشَّجَرَةِ أَن يَـٰمُوسَىٰٓ إِنِّىٓ أَنَا ٱللَّهُ رَبُّ ٱلْعَـٰلَمِينَ ﴿٣٠﴾}\\
\textamh{31.\  } & \mytextarabic{وَأَنْ أَلْقِ عَصَاكَ ۖ فَلَمَّا رَءَاهَا تَهْتَزُّ كَأَنَّهَا جَآنٌّۭ وَلَّىٰ مُدْبِرًۭا وَلَمْ يُعَقِّبْ ۚ يَـٰمُوسَىٰٓ أَقْبِلْ وَلَا تَخَفْ ۖ إِنَّكَ مِنَ ٱلْءَامِنِينَ ﴿٣١﴾}\\
\textamh{32.\  } & \mytextarabic{ٱسْلُكْ يَدَكَ فِى جَيْبِكَ تَخْرُجْ بَيْضَآءَ مِنْ غَيْرِ سُوٓءٍۢ وَٱضْمُمْ إِلَيْكَ جَنَاحَكَ مِنَ ٱلرَّهْبِ ۖ فَذَٟنِكَ بُرْهَـٰنَانِ مِن رَّبِّكَ إِلَىٰ فِرْعَوْنَ وَمَلَإِي۟هِۦٓ ۚ إِنَّهُمْ كَانُوا۟ قَوْمًۭا فَـٰسِقِينَ ﴿٣٢﴾}\\
\textamh{33.\  } & \mytextarabic{قَالَ رَبِّ إِنِّى قَتَلْتُ مِنْهُمْ نَفْسًۭا فَأَخَافُ أَن يَقْتُلُونِ ﴿٣٣﴾}\\
\textamh{34.\  } & \mytextarabic{وَأَخِى هَـٰرُونُ هُوَ أَفْصَحُ مِنِّى لِسَانًۭا فَأَرْسِلْهُ مَعِىَ رِدْءًۭا يُصَدِّقُنِىٓ ۖ إِنِّىٓ أَخَافُ أَن يُكَذِّبُونِ ﴿٣٤﴾}\\
\textamh{35.\  } & \mytextarabic{قَالَ سَنَشُدُّ عَضُدَكَ بِأَخِيكَ وَنَجْعَلُ لَكُمَا سُلْطَٰنًۭا فَلَا يَصِلُونَ إِلَيْكُمَا ۚ بِـَٔايَـٰتِنَآ أَنتُمَا وَمَنِ ٱتَّبَعَكُمَا ٱلْغَٰلِبُونَ ﴿٣٥﴾}\\
\textamh{36.\  } & \mytextarabic{فَلَمَّا جَآءَهُم مُّوسَىٰ بِـَٔايَـٰتِنَا بَيِّنَـٰتٍۢ قَالُوا۟ مَا هَـٰذَآ إِلَّا سِحْرٌۭ مُّفْتَرًۭى وَمَا سَمِعْنَا بِهَـٰذَا فِىٓ ءَابَآئِنَا ٱلْأَوَّلِينَ ﴿٣٦﴾}\\
\textamh{37.\  } & \mytextarabic{وَقَالَ مُوسَىٰ رَبِّىٓ أَعْلَمُ بِمَن جَآءَ بِٱلْهُدَىٰ مِنْ عِندِهِۦ وَمَن تَكُونُ لَهُۥ عَـٰقِبَةُ ٱلدَّارِ ۖ إِنَّهُۥ لَا يُفْلِحُ ٱلظَّـٰلِمُونَ ﴿٣٧﴾}\\
\textamh{38.\  } & \mytextarabic{وَقَالَ فِرْعَوْنُ يَـٰٓأَيُّهَا ٱلْمَلَأُ مَا عَلِمْتُ لَكُم مِّنْ إِلَـٰهٍ غَيْرِى فَأَوْقِدْ لِى يَـٰهَـٰمَـٰنُ عَلَى ٱلطِّينِ فَٱجْعَل لِّى صَرْحًۭا لَّعَلِّىٓ أَطَّلِعُ إِلَىٰٓ إِلَـٰهِ مُوسَىٰ وَإِنِّى لَأَظُنُّهُۥ مِنَ ٱلْكَـٰذِبِينَ ﴿٣٨﴾}\\
\textamh{39.\  } & \mytextarabic{وَٱسْتَكْبَرَ هُوَ وَجُنُودُهُۥ فِى ٱلْأَرْضِ بِغَيْرِ ٱلْحَقِّ وَظَنُّوٓا۟ أَنَّهُمْ إِلَيْنَا لَا يُرْجَعُونَ ﴿٣٩﴾}\\
\textamh{40.\  } & \mytextarabic{فَأَخَذْنَـٰهُ وَجُنُودَهُۥ فَنَبَذْنَـٰهُمْ فِى ٱلْيَمِّ ۖ فَٱنظُرْ كَيْفَ كَانَ عَـٰقِبَةُ ٱلظَّـٰلِمِينَ ﴿٤٠﴾}\\
\textamh{41.\  } & \mytextarabic{وَجَعَلْنَـٰهُمْ أَئِمَّةًۭ يَدْعُونَ إِلَى ٱلنَّارِ ۖ وَيَوْمَ ٱلْقِيَـٰمَةِ لَا يُنصَرُونَ ﴿٤١﴾}\\
\textamh{42.\  } & \mytextarabic{وَأَتْبَعْنَـٰهُمْ فِى هَـٰذِهِ ٱلدُّنْيَا لَعْنَةًۭ ۖ وَيَوْمَ ٱلْقِيَـٰمَةِ هُم مِّنَ ٱلْمَقْبُوحِينَ ﴿٤٢﴾}\\
\textamh{43.\  } & \mytextarabic{وَلَقَدْ ءَاتَيْنَا مُوسَى ٱلْكِتَـٰبَ مِنۢ بَعْدِ مَآ أَهْلَكْنَا ٱلْقُرُونَ ٱلْأُولَىٰ بَصَآئِرَ لِلنَّاسِ وَهُدًۭى وَرَحْمَةًۭ لَّعَلَّهُمْ يَتَذَكَّرُونَ ﴿٤٣﴾}\\
\textamh{44.\  } & \mytextarabic{وَمَا كُنتَ بِجَانِبِ ٱلْغَرْبِىِّ إِذْ قَضَيْنَآ إِلَىٰ مُوسَى ٱلْأَمْرَ وَمَا كُنتَ مِنَ ٱلشَّـٰهِدِينَ ﴿٤٤﴾}\\
\textamh{45.\  } & \mytextarabic{وَلَـٰكِنَّآ أَنشَأْنَا قُرُونًۭا فَتَطَاوَلَ عَلَيْهِمُ ٱلْعُمُرُ ۚ وَمَا كُنتَ ثَاوِيًۭا فِىٓ أَهْلِ مَدْيَنَ تَتْلُوا۟ عَلَيْهِمْ ءَايَـٰتِنَا وَلَـٰكِنَّا كُنَّا مُرْسِلِينَ ﴿٤٥﴾}\\
\textamh{46.\  } & \mytextarabic{وَمَا كُنتَ بِجَانِبِ ٱلطُّورِ إِذْ نَادَيْنَا وَلَـٰكِن رَّحْمَةًۭ مِّن رَّبِّكَ لِتُنذِرَ قَوْمًۭا مَّآ أَتَىٰهُم مِّن نَّذِيرٍۢ مِّن قَبْلِكَ لَعَلَّهُمْ يَتَذَكَّرُونَ ﴿٤٦﴾}\\
\textamh{47.\  } & \mytextarabic{وَلَوْلَآ أَن تُصِيبَهُم مُّصِيبَةٌۢ بِمَا قَدَّمَتْ أَيْدِيهِمْ فَيَقُولُوا۟ رَبَّنَا لَوْلَآ أَرْسَلْتَ إِلَيْنَا رَسُولًۭا فَنَتَّبِعَ ءَايَـٰتِكَ وَنَكُونَ مِنَ ٱلْمُؤْمِنِينَ ﴿٤٧﴾}\\
\textamh{48.\  } & \mytextarabic{فَلَمَّا جَآءَهُمُ ٱلْحَقُّ مِنْ عِندِنَا قَالُوا۟ لَوْلَآ أُوتِىَ مِثْلَ مَآ أُوتِىَ مُوسَىٰٓ ۚ أَوَلَمْ يَكْفُرُوا۟ بِمَآ أُوتِىَ مُوسَىٰ مِن قَبْلُ ۖ قَالُوا۟ سِحْرَانِ تَظَـٰهَرَا وَقَالُوٓا۟ إِنَّا بِكُلٍّۢ كَـٰفِرُونَ ﴿٤٨﴾}\\
\textamh{49.\  } & \mytextarabic{قُلْ فَأْتُوا۟ بِكِتَـٰبٍۢ مِّنْ عِندِ ٱللَّهِ هُوَ أَهْدَىٰ مِنْهُمَآ أَتَّبِعْهُ إِن كُنتُمْ صَـٰدِقِينَ ﴿٤٩﴾}\\
\textamh{50.\  } & \mytextarabic{فَإِن لَّمْ يَسْتَجِيبُوا۟ لَكَ فَٱعْلَمْ أَنَّمَا يَتَّبِعُونَ أَهْوَآءَهُمْ ۚ وَمَنْ أَضَلُّ مِمَّنِ ٱتَّبَعَ هَوَىٰهُ بِغَيْرِ هُدًۭى مِّنَ ٱللَّهِ ۚ إِنَّ ٱللَّهَ لَا يَهْدِى ٱلْقَوْمَ ٱلظَّـٰلِمِينَ ﴿٥٠﴾}\\
\textamh{51.\  } & \mytextarabic{۞ وَلَقَدْ وَصَّلْنَا لَهُمُ ٱلْقَوْلَ لَعَلَّهُمْ يَتَذَكَّرُونَ ﴿٥١﴾}\\
\textamh{52.\  } & \mytextarabic{ٱلَّذِينَ ءَاتَيْنَـٰهُمُ ٱلْكِتَـٰبَ مِن قَبْلِهِۦ هُم بِهِۦ يُؤْمِنُونَ ﴿٥٢﴾}\\
\textamh{53.\  } & \mytextarabic{وَإِذَا يُتْلَىٰ عَلَيْهِمْ قَالُوٓا۟ ءَامَنَّا بِهِۦٓ إِنَّهُ ٱلْحَقُّ مِن رَّبِّنَآ إِنَّا كُنَّا مِن قَبْلِهِۦ مُسْلِمِينَ ﴿٥٣﴾}\\
\textamh{54.\  } & \mytextarabic{أُو۟لَـٰٓئِكَ يُؤْتَوْنَ أَجْرَهُم مَّرَّتَيْنِ بِمَا صَبَرُوا۟ وَيَدْرَءُونَ بِٱلْحَسَنَةِ ٱلسَّيِّئَةَ وَمِمَّا رَزَقْنَـٰهُمْ يُنفِقُونَ ﴿٥٤﴾}\\
\textamh{55.\  } & \mytextarabic{وَإِذَا سَمِعُوا۟ ٱللَّغْوَ أَعْرَضُوا۟ عَنْهُ وَقَالُوا۟ لَنَآ أَعْمَـٰلُنَا وَلَكُمْ أَعْمَـٰلُكُمْ سَلَـٰمٌ عَلَيْكُمْ لَا نَبْتَغِى ٱلْجَٰهِلِينَ ﴿٥٥﴾}\\
\textamh{56.\  } & \mytextarabic{إِنَّكَ لَا تَهْدِى مَنْ أَحْبَبْتَ وَلَـٰكِنَّ ٱللَّهَ يَهْدِى مَن يَشَآءُ ۚ وَهُوَ أَعْلَمُ بِٱلْمُهْتَدِينَ ﴿٥٦﴾}\\
\textamh{57.\  } & \mytextarabic{وَقَالُوٓا۟ إِن نَّتَّبِعِ ٱلْهُدَىٰ مَعَكَ نُتَخَطَّفْ مِنْ أَرْضِنَآ ۚ أَوَلَمْ نُمَكِّن لَّهُمْ حَرَمًا ءَامِنًۭا يُجْبَىٰٓ إِلَيْهِ ثَمَرَٰتُ كُلِّ شَىْءٍۢ رِّزْقًۭا مِّن لَّدُنَّا وَلَـٰكِنَّ أَكْثَرَهُمْ لَا يَعْلَمُونَ ﴿٥٧﴾}\\
\textamh{58.\  } & \mytextarabic{وَكَمْ أَهْلَكْنَا مِن قَرْيَةٍۭ بَطِرَتْ مَعِيشَتَهَا ۖ فَتِلْكَ مَسَـٰكِنُهُمْ لَمْ تُسْكَن مِّنۢ بَعْدِهِمْ إِلَّا قَلِيلًۭا ۖ وَكُنَّا نَحْنُ ٱلْوَٟرِثِينَ ﴿٥٨﴾}\\
\textamh{59.\  } & \mytextarabic{وَمَا كَانَ رَبُّكَ مُهْلِكَ ٱلْقُرَىٰ حَتَّىٰ يَبْعَثَ فِىٓ أُمِّهَا رَسُولًۭا يَتْلُوا۟ عَلَيْهِمْ ءَايَـٰتِنَا ۚ وَمَا كُنَّا مُهْلِكِى ٱلْقُرَىٰٓ إِلَّا وَأَهْلُهَا ظَـٰلِمُونَ ﴿٥٩﴾}\\
\textamh{60.\  } & \mytextarabic{وَمَآ أُوتِيتُم مِّن شَىْءٍۢ فَمَتَـٰعُ ٱلْحَيَوٰةِ ٱلدُّنْيَا وَزِينَتُهَا ۚ وَمَا عِندَ ٱللَّهِ خَيْرٌۭ وَأَبْقَىٰٓ ۚ أَفَلَا تَعْقِلُونَ ﴿٦٠﴾}\\
\textamh{61.\  } & \mytextarabic{أَفَمَن وَعَدْنَـٰهُ وَعْدًا حَسَنًۭا فَهُوَ لَـٰقِيهِ كَمَن مَّتَّعْنَـٰهُ مَتَـٰعَ ٱلْحَيَوٰةِ ٱلدُّنْيَا ثُمَّ هُوَ يَوْمَ ٱلْقِيَـٰمَةِ مِنَ ٱلْمُحْضَرِينَ ﴿٦١﴾}\\
\textamh{62.\  } & \mytextarabic{وَيَوْمَ يُنَادِيهِمْ فَيَقُولُ أَيْنَ شُرَكَآءِىَ ٱلَّذِينَ كُنتُمْ تَزْعُمُونَ ﴿٦٢﴾}\\
\textamh{63.\  } & \mytextarabic{قَالَ ٱلَّذِينَ حَقَّ عَلَيْهِمُ ٱلْقَوْلُ رَبَّنَا هَـٰٓؤُلَآءِ ٱلَّذِينَ أَغْوَيْنَآ أَغْوَيْنَـٰهُمْ كَمَا غَوَيْنَا ۖ تَبَرَّأْنَآ إِلَيْكَ ۖ مَا كَانُوٓا۟ إِيَّانَا يَعْبُدُونَ ﴿٦٣﴾}\\
\textamh{64.\  } & \mytextarabic{وَقِيلَ ٱدْعُوا۟ شُرَكَآءَكُمْ فَدَعَوْهُمْ فَلَمْ يَسْتَجِيبُوا۟ لَهُمْ وَرَأَوُا۟ ٱلْعَذَابَ ۚ لَوْ أَنَّهُمْ كَانُوا۟ يَهْتَدُونَ ﴿٦٤﴾}\\
\textamh{65.\  } & \mytextarabic{وَيَوْمَ يُنَادِيهِمْ فَيَقُولُ مَاذَآ أَجَبْتُمُ ٱلْمُرْسَلِينَ ﴿٦٥﴾}\\
\textamh{66.\  } & \mytextarabic{فَعَمِيَتْ عَلَيْهِمُ ٱلْأَنۢبَآءُ يَوْمَئِذٍۢ فَهُمْ لَا يَتَسَآءَلُونَ ﴿٦٦﴾}\\
\textamh{67.\  } & \mytextarabic{فَأَمَّا مَن تَابَ وَءَامَنَ وَعَمِلَ صَـٰلِحًۭا فَعَسَىٰٓ أَن يَكُونَ مِنَ ٱلْمُفْلِحِينَ ﴿٦٧﴾}\\
\textamh{68.\  } & \mytextarabic{وَرَبُّكَ يَخْلُقُ مَا يَشَآءُ وَيَخْتَارُ ۗ مَا كَانَ لَهُمُ ٱلْخِيَرَةُ ۚ سُبْحَـٰنَ ٱللَّهِ وَتَعَـٰلَىٰ عَمَّا يُشْرِكُونَ ﴿٦٨﴾}\\
\textamh{69.\  } & \mytextarabic{وَرَبُّكَ يَعْلَمُ مَا تُكِنُّ صُدُورُهُمْ وَمَا يُعْلِنُونَ ﴿٦٩﴾}\\
\textamh{70.\  } & \mytextarabic{وَهُوَ ٱللَّهُ لَآ إِلَـٰهَ إِلَّا هُوَ ۖ لَهُ ٱلْحَمْدُ فِى ٱلْأُولَىٰ وَٱلْءَاخِرَةِ ۖ وَلَهُ ٱلْحُكْمُ وَإِلَيْهِ تُرْجَعُونَ ﴿٧٠﴾}\\
\textamh{71.\  } & \mytextarabic{قُلْ أَرَءَيْتُمْ إِن جَعَلَ ٱللَّهُ عَلَيْكُمُ ٱلَّيْلَ سَرْمَدًا إِلَىٰ يَوْمِ ٱلْقِيَـٰمَةِ مَنْ إِلَـٰهٌ غَيْرُ ٱللَّهِ يَأْتِيكُم بِضِيَآءٍ ۖ أَفَلَا تَسْمَعُونَ ﴿٧١﴾}\\
\textamh{72.\  } & \mytextarabic{قُلْ أَرَءَيْتُمْ إِن جَعَلَ ٱللَّهُ عَلَيْكُمُ ٱلنَّهَارَ سَرْمَدًا إِلَىٰ يَوْمِ ٱلْقِيَـٰمَةِ مَنْ إِلَـٰهٌ غَيْرُ ٱللَّهِ يَأْتِيكُم بِلَيْلٍۢ تَسْكُنُونَ فِيهِ ۖ أَفَلَا تُبْصِرُونَ ﴿٧٢﴾}\\
\textamh{73.\  } & \mytextarabic{وَمِن رَّحْمَتِهِۦ جَعَلَ لَكُمُ ٱلَّيْلَ وَٱلنَّهَارَ لِتَسْكُنُوا۟ فِيهِ وَلِتَبْتَغُوا۟ مِن فَضْلِهِۦ وَلَعَلَّكُمْ تَشْكُرُونَ ﴿٧٣﴾}\\
\textamh{74.\  } & \mytextarabic{وَيَوْمَ يُنَادِيهِمْ فَيَقُولُ أَيْنَ شُرَكَآءِىَ ٱلَّذِينَ كُنتُمْ تَزْعُمُونَ ﴿٧٤﴾}\\
\textamh{75.\  } & \mytextarabic{وَنَزَعْنَا مِن كُلِّ أُمَّةٍۢ شَهِيدًۭا فَقُلْنَا هَاتُوا۟ بُرْهَـٰنَكُمْ فَعَلِمُوٓا۟ أَنَّ ٱلْحَقَّ لِلَّهِ وَضَلَّ عَنْهُم مَّا كَانُوا۟ يَفْتَرُونَ ﴿٧٥﴾}\\
\textamh{76.\  } & \mytextarabic{۞ إِنَّ قَـٰرُونَ كَانَ مِن قَوْمِ مُوسَىٰ فَبَغَىٰ عَلَيْهِمْ ۖ وَءَاتَيْنَـٰهُ مِنَ ٱلْكُنُوزِ مَآ إِنَّ مَفَاتِحَهُۥ لَتَنُوٓأُ بِٱلْعُصْبَةِ أُو۟لِى ٱلْقُوَّةِ إِذْ قَالَ لَهُۥ قَوْمُهُۥ لَا تَفْرَحْ ۖ إِنَّ ٱللَّهَ لَا يُحِبُّ ٱلْفَرِحِينَ ﴿٧٦﴾}\\
\textamh{77.\  } & \mytextarabic{وَٱبْتَغِ فِيمَآ ءَاتَىٰكَ ٱللَّهُ ٱلدَّارَ ٱلْءَاخِرَةَ ۖ وَلَا تَنسَ نَصِيبَكَ مِنَ ٱلدُّنْيَا ۖ وَأَحْسِن كَمَآ أَحْسَنَ ٱللَّهُ إِلَيْكَ ۖ وَلَا تَبْغِ ٱلْفَسَادَ فِى ٱلْأَرْضِ ۖ إِنَّ ٱللَّهَ لَا يُحِبُّ ٱلْمُفْسِدِينَ ﴿٧٧﴾}\\
\textamh{78.\  } & \mytextarabic{قَالَ إِنَّمَآ أُوتِيتُهُۥ عَلَىٰ عِلْمٍ عِندِىٓ ۚ أَوَلَمْ يَعْلَمْ أَنَّ ٱللَّهَ قَدْ أَهْلَكَ مِن قَبْلِهِۦ مِنَ ٱلْقُرُونِ مَنْ هُوَ أَشَدُّ مِنْهُ قُوَّةًۭ وَأَكْثَرُ جَمْعًۭا ۚ وَلَا يُسْـَٔلُ عَن ذُنُوبِهِمُ ٱلْمُجْرِمُونَ ﴿٧٨﴾}\\
\textamh{79.\  } & \mytextarabic{فَخَرَجَ عَلَىٰ قَوْمِهِۦ فِى زِينَتِهِۦ ۖ قَالَ ٱلَّذِينَ يُرِيدُونَ ٱلْحَيَوٰةَ ٱلدُّنْيَا يَـٰلَيْتَ لَنَا مِثْلَ مَآ أُوتِىَ قَـٰرُونُ إِنَّهُۥ لَذُو حَظٍّ عَظِيمٍۢ ﴿٧٩﴾}\\
\textamh{80.\  } & \mytextarabic{وَقَالَ ٱلَّذِينَ أُوتُوا۟ ٱلْعِلْمَ وَيْلَكُمْ ثَوَابُ ٱللَّهِ خَيْرٌۭ لِّمَنْ ءَامَنَ وَعَمِلَ صَـٰلِحًۭا وَلَا يُلَقَّىٰهَآ إِلَّا ٱلصَّـٰبِرُونَ ﴿٨٠﴾}\\
\textamh{81.\  } & \mytextarabic{فَخَسَفْنَا بِهِۦ وَبِدَارِهِ ٱلْأَرْضَ فَمَا كَانَ لَهُۥ مِن فِئَةٍۢ يَنصُرُونَهُۥ مِن دُونِ ٱللَّهِ وَمَا كَانَ مِنَ ٱلْمُنتَصِرِينَ ﴿٨١﴾}\\
\textamh{82.\  } & \mytextarabic{وَأَصْبَحَ ٱلَّذِينَ تَمَنَّوْا۟ مَكَانَهُۥ بِٱلْأَمْسِ يَقُولُونَ وَيْكَأَنَّ ٱللَّهَ يَبْسُطُ ٱلرِّزْقَ لِمَن يَشَآءُ مِنْ عِبَادِهِۦ وَيَقْدِرُ ۖ لَوْلَآ أَن مَّنَّ ٱللَّهُ عَلَيْنَا لَخَسَفَ بِنَا ۖ وَيْكَأَنَّهُۥ لَا يُفْلِحُ ٱلْكَـٰفِرُونَ ﴿٨٢﴾}\\
\textamh{83.\  } & \mytextarabic{تِلْكَ ٱلدَّارُ ٱلْءَاخِرَةُ نَجْعَلُهَا لِلَّذِينَ لَا يُرِيدُونَ عُلُوًّۭا فِى ٱلْأَرْضِ وَلَا فَسَادًۭا ۚ وَٱلْعَـٰقِبَةُ لِلْمُتَّقِينَ ﴿٨٣﴾}\\
\textamh{84.\  } & \mytextarabic{مَن جَآءَ بِٱلْحَسَنَةِ فَلَهُۥ خَيْرٌۭ مِّنْهَا ۖ وَمَن جَآءَ بِٱلسَّيِّئَةِ فَلَا يُجْزَى ٱلَّذِينَ عَمِلُوا۟ ٱلسَّيِّـَٔاتِ إِلَّا مَا كَانُوا۟ يَعْمَلُونَ ﴿٨٤﴾}\\
\textamh{85.\  } & \mytextarabic{إِنَّ ٱلَّذِى فَرَضَ عَلَيْكَ ٱلْقُرْءَانَ لَرَآدُّكَ إِلَىٰ مَعَادٍۢ ۚ قُل رَّبِّىٓ أَعْلَمُ مَن جَآءَ بِٱلْهُدَىٰ وَمَنْ هُوَ فِى ضَلَـٰلٍۢ مُّبِينٍۢ ﴿٨٥﴾}\\
\textamh{86.\  } & \mytextarabic{وَمَا كُنتَ تَرْجُوٓا۟ أَن يُلْقَىٰٓ إِلَيْكَ ٱلْكِتَـٰبُ إِلَّا رَحْمَةًۭ مِّن رَّبِّكَ ۖ فَلَا تَكُونَنَّ ظَهِيرًۭا لِّلْكَـٰفِرِينَ ﴿٨٦﴾}\\
\textamh{87.\  } & \mytextarabic{وَلَا يَصُدُّنَّكَ عَنْ ءَايَـٰتِ ٱللَّهِ بَعْدَ إِذْ أُنزِلَتْ إِلَيْكَ ۖ وَٱدْعُ إِلَىٰ رَبِّكَ ۖ وَلَا تَكُونَنَّ مِنَ ٱلْمُشْرِكِينَ ﴿٨٧﴾}\\
\textamh{88.\  } & \mytextarabic{وَلَا تَدْعُ مَعَ ٱللَّهِ إِلَـٰهًا ءَاخَرَ ۘ لَآ إِلَـٰهَ إِلَّا هُوَ ۚ كُلُّ شَىْءٍ هَالِكٌ إِلَّا وَجْهَهُۥ ۚ لَهُ ٱلْحُكْمُ وَإِلَيْهِ تُرْجَعُونَ ﴿٨٨﴾}\\
\end{longtable}
\clearpage
%% License: BSD style (Berkley) (i.e. Put the Copyright owner's name always)
%% Writer and Copyright (to): Bewketu(Bilal) Tadilo (2016-17)
\centering\section{\LR{\textamharic{ሱራቱ አልአንከቡት -}  \RL{سوره  العنكبوت}}}
\begin{longtable}{%
  @{}
    p{.5\textwidth}
  @{~~~~~~~~~~~~~}
    p{.5\textwidth}
    @{}
}
\nopagebreak
\textamh{ቢስሚላሂ አራህመኒ ራሂይም } &  بِسْمِ ٱللَّهِ ٱلرَّحْمَـٰنِ ٱلرَّحِيمِ\\
\textamh{1.\  } &  الٓمٓ ﴿١﴾\\
\textamh{2.\  } & أَحَسِبَ ٱلنَّاسُ أَن يُتْرَكُوٓا۟ أَن يَقُولُوٓا۟ ءَامَنَّا وَهُمْ لَا يُفْتَنُونَ ﴿٢﴾\\
\textamh{3.\  } & وَلَقَدْ فَتَنَّا ٱلَّذِينَ مِن قَبْلِهِمْ ۖ فَلَيَعْلَمَنَّ ٱللَّهُ ٱلَّذِينَ صَدَقُوا۟ وَلَيَعْلَمَنَّ ٱلْكَـٰذِبِينَ ﴿٣﴾\\
\textamh{4.\  } & أَمْ حَسِبَ ٱلَّذِينَ يَعْمَلُونَ ٱلسَّيِّـَٔاتِ أَن يَسْبِقُونَا ۚ سَآءَ مَا يَحْكُمُونَ ﴿٤﴾\\
\textamh{5.\  } & مَن كَانَ يَرْجُوا۟ لِقَآءَ ٱللَّهِ فَإِنَّ أَجَلَ ٱللَّهِ لَءَاتٍۢ ۚ وَهُوَ ٱلسَّمِيعُ ٱلْعَلِيمُ ﴿٥﴾\\
\textamh{6.\  } & وَمَن جَٰهَدَ فَإِنَّمَا يُجَٰهِدُ لِنَفْسِهِۦٓ ۚ إِنَّ ٱللَّهَ لَغَنِىٌّ عَنِ ٱلْعَـٰلَمِينَ ﴿٦﴾\\
\textamh{7.\  } & وَٱلَّذِينَ ءَامَنُوا۟ وَعَمِلُوا۟ ٱلصَّـٰلِحَـٰتِ لَنُكَفِّرَنَّ عَنْهُمْ سَيِّـَٔاتِهِمْ وَلَنَجْزِيَنَّهُمْ أَحْسَنَ ٱلَّذِى كَانُوا۟ يَعْمَلُونَ ﴿٧﴾\\
\textamh{8.\  } & وَوَصَّيْنَا ٱلْإِنسَـٰنَ بِوَٟلِدَيْهِ حُسْنًۭا ۖ وَإِن جَٰهَدَاكَ لِتُشْرِكَ بِى مَا لَيْسَ لَكَ بِهِۦ عِلْمٌۭ فَلَا تُطِعْهُمَآ ۚ إِلَىَّ مَرْجِعُكُمْ فَأُنَبِّئُكُم بِمَا كُنتُمْ تَعْمَلُونَ ﴿٨﴾\\
\textamh{9.\  } & وَٱلَّذِينَ ءَامَنُوا۟ وَعَمِلُوا۟ ٱلصَّـٰلِحَـٰتِ لَنُدْخِلَنَّهُمْ فِى ٱلصَّـٰلِحِينَ ﴿٩﴾\\
\textamh{10.\  } & وَمِنَ ٱلنَّاسِ مَن يَقُولُ ءَامَنَّا بِٱللَّهِ فَإِذَآ أُوذِىَ فِى ٱللَّهِ جَعَلَ فِتْنَةَ ٱلنَّاسِ كَعَذَابِ ٱللَّهِ وَلَئِن جَآءَ نَصْرٌۭ مِّن رَّبِّكَ لَيَقُولُنَّ إِنَّا كُنَّا مَعَكُمْ ۚ أَوَلَيْسَ ٱللَّهُ بِأَعْلَمَ بِمَا فِى صُدُورِ ٱلْعَـٰلَمِينَ ﴿١٠﴾\\
\textamh{11.\  } & وَلَيَعْلَمَنَّ ٱللَّهُ ٱلَّذِينَ ءَامَنُوا۟ وَلَيَعْلَمَنَّ ٱلْمُنَـٰفِقِينَ ﴿١١﴾\\
\textamh{12.\  } & وَقَالَ ٱلَّذِينَ كَفَرُوا۟ لِلَّذِينَ ءَامَنُوا۟ ٱتَّبِعُوا۟ سَبِيلَنَا وَلْنَحْمِلْ خَطَٰيَـٰكُمْ وَمَا هُم بِحَـٰمِلِينَ مِنْ خَطَٰيَـٰهُم مِّن شَىْءٍ ۖ إِنَّهُمْ لَكَـٰذِبُونَ ﴿١٢﴾\\
\textamh{13.\  } & وَلَيَحْمِلُنَّ أَثْقَالَهُمْ وَأَثْقَالًۭا مَّعَ أَثْقَالِهِمْ ۖ وَلَيُسْـَٔلُنَّ يَوْمَ ٱلْقِيَـٰمَةِ عَمَّا كَانُوا۟ يَفْتَرُونَ ﴿١٣﴾\\
\textamh{14.\  } & وَلَقَدْ أَرْسَلْنَا نُوحًا إِلَىٰ قَوْمِهِۦ فَلَبِثَ فِيهِمْ أَلْفَ سَنَةٍ إِلَّا خَمْسِينَ عَامًۭا فَأَخَذَهُمُ ٱلطُّوفَانُ وَهُمْ ظَـٰلِمُونَ ﴿١٤﴾\\
\textamh{15.\  } & فَأَنجَيْنَـٰهُ وَأَصْحَـٰبَ ٱلسَّفِينَةِ وَجَعَلْنَـٰهَآ ءَايَةًۭ لِّلْعَـٰلَمِينَ ﴿١٥﴾\\
\textamh{16.\  } & وَإِبْرَٰهِيمَ إِذْ قَالَ لِقَوْمِهِ ٱعْبُدُوا۟ ٱللَّهَ وَٱتَّقُوهُ ۖ ذَٟلِكُمْ خَيْرٌۭ لَّكُمْ إِن كُنتُمْ تَعْلَمُونَ ﴿١٦﴾\\
\textamh{17.\  } & إِنَّمَا تَعْبُدُونَ مِن دُونِ ٱللَّهِ أَوْثَـٰنًۭا وَتَخْلُقُونَ إِفْكًا ۚ إِنَّ ٱلَّذِينَ تَعْبُدُونَ مِن دُونِ ٱللَّهِ لَا يَمْلِكُونَ لَكُمْ رِزْقًۭا فَٱبْتَغُوا۟ عِندَ ٱللَّهِ ٱلرِّزْقَ وَٱعْبُدُوهُ وَٱشْكُرُوا۟ لَهُۥٓ ۖ إِلَيْهِ تُرْجَعُونَ ﴿١٧﴾\\
\textamh{18.\  } & وَإِن تُكَذِّبُوا۟ فَقَدْ كَذَّبَ أُمَمٌۭ مِّن قَبْلِكُمْ ۖ وَمَا عَلَى ٱلرَّسُولِ إِلَّا ٱلْبَلَـٰغُ ٱلْمُبِينُ ﴿١٨﴾\\
\textamh{19.\  } & أَوَلَمْ يَرَوْا۟ كَيْفَ يُبْدِئُ ٱللَّهُ ٱلْخَلْقَ ثُمَّ يُعِيدُهُۥٓ ۚ إِنَّ ذَٟلِكَ عَلَى ٱللَّهِ يَسِيرٌۭ ﴿١٩﴾\\
\textamh{20.\  } & قُلْ سِيرُوا۟ فِى ٱلْأَرْضِ فَٱنظُرُوا۟ كَيْفَ بَدَأَ ٱلْخَلْقَ ۚ ثُمَّ ٱللَّهُ يُنشِئُ ٱلنَّشْأَةَ ٱلْءَاخِرَةَ ۚ إِنَّ ٱللَّهَ عَلَىٰ كُلِّ شَىْءٍۢ قَدِيرٌۭ ﴿٢٠﴾\\
\textamh{21.\  } & يُعَذِّبُ مَن يَشَآءُ وَيَرْحَمُ مَن يَشَآءُ ۖ وَإِلَيْهِ تُقْلَبُونَ ﴿٢١﴾\\
\textamh{22.\  } & وَمَآ أَنتُم بِمُعْجِزِينَ فِى ٱلْأَرْضِ وَلَا فِى ٱلسَّمَآءِ ۖ وَمَا لَكُم مِّن دُونِ ٱللَّهِ مِن وَلِىٍّۢ وَلَا نَصِيرٍۢ ﴿٢٢﴾\\
\textamh{23.\  } & وَٱلَّذِينَ كَفَرُوا۟ بِـَٔايَـٰتِ ٱللَّهِ وَلِقَآئِهِۦٓ أُو۟لَـٰٓئِكَ يَئِسُوا۟ مِن رَّحْمَتِى وَأُو۟لَـٰٓئِكَ لَهُمْ عَذَابٌ أَلِيمٌۭ ﴿٢٣﴾\\
\textamh{24.\  } & فَمَا كَانَ جَوَابَ قَوْمِهِۦٓ إِلَّآ أَن قَالُوا۟ ٱقْتُلُوهُ أَوْ حَرِّقُوهُ فَأَنجَىٰهُ ٱللَّهُ مِنَ ٱلنَّارِ ۚ إِنَّ فِى ذَٟلِكَ لَءَايَـٰتٍۢ لِّقَوْمٍۢ يُؤْمِنُونَ ﴿٢٤﴾\\
\textamh{25.\  } & وَقَالَ إِنَّمَا ٱتَّخَذْتُم مِّن دُونِ ٱللَّهِ أَوْثَـٰنًۭا مَّوَدَّةَ بَيْنِكُمْ فِى ٱلْحَيَوٰةِ ٱلدُّنْيَا ۖ ثُمَّ يَوْمَ ٱلْقِيَـٰمَةِ يَكْفُرُ بَعْضُكُم بِبَعْضٍۢ وَيَلْعَنُ بَعْضُكُم بَعْضًۭا وَمَأْوَىٰكُمُ ٱلنَّارُ وَمَا لَكُم مِّن نَّـٰصِرِينَ ﴿٢٥﴾\\
\textamh{26.\  } & ۞ فَـَٔامَنَ لَهُۥ لُوطٌۭ ۘ وَقَالَ إِنِّى مُهَاجِرٌ إِلَىٰ رَبِّىٓ ۖ إِنَّهُۥ هُوَ ٱلْعَزِيزُ ٱلْحَكِيمُ ﴿٢٦﴾\\
\textamh{27.\  } & وَوَهَبْنَا لَهُۥٓ إِسْحَـٰقَ وَيَعْقُوبَ وَجَعَلْنَا فِى ذُرِّيَّتِهِ ٱلنُّبُوَّةَ وَٱلْكِتَـٰبَ وَءَاتَيْنَـٰهُ أَجْرَهُۥ فِى ٱلدُّنْيَا ۖ وَإِنَّهُۥ فِى ٱلْءَاخِرَةِ لَمِنَ ٱلصَّـٰلِحِينَ ﴿٢٧﴾\\
\textamh{28.\  } & وَلُوطًا إِذْ قَالَ لِقَوْمِهِۦٓ إِنَّكُمْ لَتَأْتُونَ ٱلْفَـٰحِشَةَ مَا سَبَقَكُم بِهَا مِنْ أَحَدٍۢ مِّنَ ٱلْعَـٰلَمِينَ ﴿٢٨﴾\\
\textamh{29.\  } & أَئِنَّكُمْ لَتَأْتُونَ ٱلرِّجَالَ وَتَقْطَعُونَ ٱلسَّبِيلَ وَتَأْتُونَ فِى نَادِيكُمُ ٱلْمُنكَرَ ۖ فَمَا كَانَ جَوَابَ قَوْمِهِۦٓ إِلَّآ أَن قَالُوا۟ ٱئْتِنَا بِعَذَابِ ٱللَّهِ إِن كُنتَ مِنَ ٱلصَّـٰدِقِينَ ﴿٢٩﴾\\
\textamh{30.\  } & قَالَ رَبِّ ٱنصُرْنِى عَلَى ٱلْقَوْمِ ٱلْمُفْسِدِينَ ﴿٣٠﴾\\
\textamh{31.\  } & وَلَمَّا جَآءَتْ رُسُلُنَآ إِبْرَٰهِيمَ بِٱلْبُشْرَىٰ قَالُوٓا۟ إِنَّا مُهْلِكُوٓا۟ أَهْلِ هَـٰذِهِ ٱلْقَرْيَةِ ۖ إِنَّ أَهْلَهَا كَانُوا۟ ظَـٰلِمِينَ ﴿٣١﴾\\
\textamh{32.\  } & قَالَ إِنَّ فِيهَا لُوطًۭا ۚ قَالُوا۟ نَحْنُ أَعْلَمُ بِمَن فِيهَا ۖ لَنُنَجِّيَنَّهُۥ وَأَهْلَهُۥٓ إِلَّا ٱمْرَأَتَهُۥ كَانَتْ مِنَ ٱلْغَٰبِرِينَ ﴿٣٢﴾\\
\textamh{33.\  } & وَلَمَّآ أَن جَآءَتْ رُسُلُنَا لُوطًۭا سِىٓءَ بِهِمْ وَضَاقَ بِهِمْ ذَرْعًۭا وَقَالُوا۟ لَا تَخَفْ وَلَا تَحْزَنْ ۖ إِنَّا مُنَجُّوكَ وَأَهْلَكَ إِلَّا ٱمْرَأَتَكَ كَانَتْ مِنَ ٱلْغَٰبِرِينَ ﴿٣٣﴾\\
\textamh{34.\  } & إِنَّا مُنزِلُونَ عَلَىٰٓ أَهْلِ هَـٰذِهِ ٱلْقَرْيَةِ رِجْزًۭا مِّنَ ٱلسَّمَآءِ بِمَا كَانُوا۟ يَفْسُقُونَ ﴿٣٤﴾\\
\textamh{35.\  } & وَلَقَد تَّرَكْنَا مِنْهَآ ءَايَةًۢ بَيِّنَةًۭ لِّقَوْمٍۢ يَعْقِلُونَ ﴿٣٥﴾\\
\textamh{36.\  } & وَإِلَىٰ مَدْيَنَ أَخَاهُمْ شُعَيْبًۭا فَقَالَ يَـٰقَوْمِ ٱعْبُدُوا۟ ٱللَّهَ وَٱرْجُوا۟ ٱلْيَوْمَ ٱلْءَاخِرَ وَلَا تَعْثَوْا۟ فِى ٱلْأَرْضِ مُفْسِدِينَ ﴿٣٦﴾\\
\textamh{37.\  } & فَكَذَّبُوهُ فَأَخَذَتْهُمُ ٱلرَّجْفَةُ فَأَصْبَحُوا۟ فِى دَارِهِمْ جَٰثِمِينَ ﴿٣٧﴾\\
\textamh{38.\  } & وَعَادًۭا وَثَمُودَا۟ وَقَد تَّبَيَّنَ لَكُم مِّن مَّسَـٰكِنِهِمْ ۖ وَزَيَّنَ لَهُمُ ٱلشَّيْطَٰنُ أَعْمَـٰلَهُمْ فَصَدَّهُمْ عَنِ ٱلسَّبِيلِ وَكَانُوا۟ مُسْتَبْصِرِينَ ﴿٣٨﴾\\
\textamh{39.\  } & وَقَـٰرُونَ وَفِرْعَوْنَ وَهَـٰمَـٰنَ ۖ وَلَقَدْ جَآءَهُم مُّوسَىٰ بِٱلْبَيِّنَـٰتِ فَٱسْتَكْبَرُوا۟ فِى ٱلْأَرْضِ وَمَا كَانُوا۟ سَـٰبِقِينَ ﴿٣٩﴾\\
\textamh{40.\  } & فَكُلًّا أَخَذْنَا بِذَنۢبِهِۦ ۖ فَمِنْهُم مَّنْ أَرْسَلْنَا عَلَيْهِ حَاصِبًۭا وَمِنْهُم مَّنْ أَخَذَتْهُ ٱلصَّيْحَةُ وَمِنْهُم مَّنْ خَسَفْنَا بِهِ ٱلْأَرْضَ وَمِنْهُم مَّنْ أَغْرَقْنَا ۚ وَمَا كَانَ ٱللَّهُ لِيَظْلِمَهُمْ وَلَـٰكِن كَانُوٓا۟ أَنفُسَهُمْ يَظْلِمُونَ ﴿٤٠﴾\\
\textamh{41.\  } & مَثَلُ ٱلَّذِينَ ٱتَّخَذُوا۟ مِن دُونِ ٱللَّهِ أَوْلِيَآءَ كَمَثَلِ ٱلْعَنكَبُوتِ ٱتَّخَذَتْ بَيْتًۭا ۖ وَإِنَّ أَوْهَنَ ٱلْبُيُوتِ لَبَيْتُ ٱلْعَنكَبُوتِ ۖ لَوْ كَانُوا۟ يَعْلَمُونَ ﴿٤١﴾\\
\textamh{42.\  } & إِنَّ ٱللَّهَ يَعْلَمُ مَا يَدْعُونَ مِن دُونِهِۦ مِن شَىْءٍۢ ۚ وَهُوَ ٱلْعَزِيزُ ٱلْحَكِيمُ ﴿٤٢﴾\\
\textamh{43.\  } & وَتِلْكَ ٱلْأَمْثَـٰلُ نَضْرِبُهَا لِلنَّاسِ ۖ وَمَا يَعْقِلُهَآ إِلَّا ٱلْعَـٰلِمُونَ ﴿٤٣﴾\\
\textamh{44.\  } & خَلَقَ ٱللَّهُ ٱلسَّمَـٰوَٟتِ وَٱلْأَرْضَ بِٱلْحَقِّ ۚ إِنَّ فِى ذَٟلِكَ لَءَايَةًۭ لِّلْمُؤْمِنِينَ ﴿٤٤﴾\\
\textamh{45.\  } & ٱتْلُ مَآ أُوحِىَ إِلَيْكَ مِنَ ٱلْكِتَـٰبِ وَأَقِمِ ٱلصَّلَوٰةَ ۖ إِنَّ ٱلصَّلَوٰةَ تَنْهَىٰ عَنِ ٱلْفَحْشَآءِ وَٱلْمُنكَرِ ۗ وَلَذِكْرُ ٱللَّهِ أَكْبَرُ ۗ وَٱللَّهُ يَعْلَمُ مَا تَصْنَعُونَ ﴿٤٥﴾\\
\textamh{46.\  } & ۞ وَلَا تُجَٰدِلُوٓا۟ أَهْلَ ٱلْكِتَـٰبِ إِلَّا بِٱلَّتِى هِىَ أَحْسَنُ إِلَّا ٱلَّذِينَ ظَلَمُوا۟ مِنْهُمْ ۖ وَقُولُوٓا۟ ءَامَنَّا بِٱلَّذِىٓ أُنزِلَ إِلَيْنَا وَأُنزِلَ إِلَيْكُمْ وَإِلَـٰهُنَا وَإِلَـٰهُكُمْ وَٟحِدٌۭ وَنَحْنُ لَهُۥ مُسْلِمُونَ ﴿٤٦﴾\\
\textamh{47.\  } & وَكَذَٟلِكَ أَنزَلْنَآ إِلَيْكَ ٱلْكِتَـٰبَ ۚ فَٱلَّذِينَ ءَاتَيْنَـٰهُمُ ٱلْكِتَـٰبَ يُؤْمِنُونَ بِهِۦ ۖ وَمِنْ هَـٰٓؤُلَآءِ مَن يُؤْمِنُ بِهِۦ ۚ وَمَا يَجْحَدُ بِـَٔايَـٰتِنَآ إِلَّا ٱلْكَـٰفِرُونَ ﴿٤٧﴾\\
\textamh{48.\  } & وَمَا كُنتَ تَتْلُوا۟ مِن قَبْلِهِۦ مِن كِتَـٰبٍۢ وَلَا تَخُطُّهُۥ بِيَمِينِكَ ۖ إِذًۭا لَّٱرْتَابَ ٱلْمُبْطِلُونَ ﴿٤٨﴾\\
\textamh{49.\  } & بَلْ هُوَ ءَايَـٰتٌۢ بَيِّنَـٰتٌۭ فِى صُدُورِ ٱلَّذِينَ أُوتُوا۟ ٱلْعِلْمَ ۚ وَمَا يَجْحَدُ بِـَٔايَـٰتِنَآ إِلَّا ٱلظَّـٰلِمُونَ ﴿٤٩﴾\\
\textamh{50.\  } & وَقَالُوا۟ لَوْلَآ أُنزِلَ عَلَيْهِ ءَايَـٰتٌۭ مِّن رَّبِّهِۦ ۖ قُلْ إِنَّمَا ٱلْءَايَـٰتُ عِندَ ٱللَّهِ وَإِنَّمَآ أَنَا۠ نَذِيرٌۭ مُّبِينٌ ﴿٥٠﴾\\
\textamh{51.\  } & أَوَلَمْ يَكْفِهِمْ أَنَّآ أَنزَلْنَا عَلَيْكَ ٱلْكِتَـٰبَ يُتْلَىٰ عَلَيْهِمْ ۚ إِنَّ فِى ذَٟلِكَ لَرَحْمَةًۭ وَذِكْرَىٰ لِقَوْمٍۢ يُؤْمِنُونَ ﴿٥١﴾\\
\textamh{52.\  } & قُلْ كَفَىٰ بِٱللَّهِ بَيْنِى وَبَيْنَكُمْ شَهِيدًۭا ۖ يَعْلَمُ مَا فِى ٱلسَّمَـٰوَٟتِ وَٱلْأَرْضِ ۗ وَٱلَّذِينَ ءَامَنُوا۟ بِٱلْبَٰطِلِ وَكَفَرُوا۟ بِٱللَّهِ أُو۟لَـٰٓئِكَ هُمُ ٱلْخَـٰسِرُونَ ﴿٥٢﴾\\
\textamh{53.\  } & وَيَسْتَعْجِلُونَكَ بِٱلْعَذَابِ ۚ وَلَوْلَآ أَجَلٌۭ مُّسَمًّۭى لَّجَآءَهُمُ ٱلْعَذَابُ وَلَيَأْتِيَنَّهُم بَغْتَةًۭ وَهُمْ لَا يَشْعُرُونَ ﴿٥٣﴾\\
\textamh{54.\  } & يَسْتَعْجِلُونَكَ بِٱلْعَذَابِ وَإِنَّ جَهَنَّمَ لَمُحِيطَةٌۢ بِٱلْكَـٰفِرِينَ ﴿٥٤﴾\\
\textamh{55.\  } & يَوْمَ يَغْشَىٰهُمُ ٱلْعَذَابُ مِن فَوْقِهِمْ وَمِن تَحْتِ أَرْجُلِهِمْ وَيَقُولُ ذُوقُوا۟ مَا كُنتُمْ تَعْمَلُونَ ﴿٥٥﴾\\
\textamh{56.\  } & يَـٰعِبَادِىَ ٱلَّذِينَ ءَامَنُوٓا۟ إِنَّ أَرْضِى وَٟسِعَةٌۭ فَإِيَّٰىَ فَٱعْبُدُونِ ﴿٥٦﴾\\
\textamh{57.\  } & كُلُّ نَفْسٍۢ ذَآئِقَةُ ٱلْمَوْتِ ۖ ثُمَّ إِلَيْنَا تُرْجَعُونَ ﴿٥٧﴾\\
\textamh{58.\  } & وَٱلَّذِينَ ءَامَنُوا۟ وَعَمِلُوا۟ ٱلصَّـٰلِحَـٰتِ لَنُبَوِّئَنَّهُم مِّنَ ٱلْجَنَّةِ غُرَفًۭا تَجْرِى مِن تَحْتِهَا ٱلْأَنْهَـٰرُ خَـٰلِدِينَ فِيهَا ۚ نِعْمَ أَجْرُ ٱلْعَـٰمِلِينَ ﴿٥٨﴾\\
\textamh{59.\  } & ٱلَّذِينَ صَبَرُوا۟ وَعَلَىٰ رَبِّهِمْ يَتَوَكَّلُونَ ﴿٥٩﴾\\
\textamh{60.\  } & وَكَأَيِّن مِّن دَآبَّةٍۢ لَّا تَحْمِلُ رِزْقَهَا ٱللَّهُ يَرْزُقُهَا وَإِيَّاكُمْ ۚ وَهُوَ ٱلسَّمِيعُ ٱلْعَلِيمُ ﴿٦٠﴾\\
\textamh{61.\  } & وَلَئِن سَأَلْتَهُم مَّنْ خَلَقَ ٱلسَّمَـٰوَٟتِ وَٱلْأَرْضَ وَسَخَّرَ ٱلشَّمْسَ وَٱلْقَمَرَ لَيَقُولُنَّ ٱللَّهُ ۖ فَأَنَّىٰ يُؤْفَكُونَ ﴿٦١﴾\\
\textamh{62.\  } & ٱللَّهُ يَبْسُطُ ٱلرِّزْقَ لِمَن يَشَآءُ مِنْ عِبَادِهِۦ وَيَقْدِرُ لَهُۥٓ ۚ إِنَّ ٱللَّهَ بِكُلِّ شَىْءٍ عَلِيمٌۭ ﴿٦٢﴾\\
\textamh{63.\  } & وَلَئِن سَأَلْتَهُم مَّن نَّزَّلَ مِنَ ٱلسَّمَآءِ مَآءًۭ فَأَحْيَا بِهِ ٱلْأَرْضَ مِنۢ بَعْدِ مَوْتِهَا لَيَقُولُنَّ ٱللَّهُ ۚ قُلِ ٱلْحَمْدُ لِلَّهِ ۚ بَلْ أَكْثَرُهُمْ لَا يَعْقِلُونَ ﴿٦٣﴾\\
\textamh{64.\  } & وَمَا هَـٰذِهِ ٱلْحَيَوٰةُ ٱلدُّنْيَآ إِلَّا لَهْوٌۭ وَلَعِبٌۭ ۚ وَإِنَّ ٱلدَّارَ ٱلْءَاخِرَةَ لَهِىَ ٱلْحَيَوَانُ ۚ لَوْ كَانُوا۟ يَعْلَمُونَ ﴿٦٤﴾\\
\textamh{65.\  } & فَإِذَا رَكِبُوا۟ فِى ٱلْفُلْكِ دَعَوُا۟ ٱللَّهَ مُخْلِصِينَ لَهُ ٱلدِّينَ فَلَمَّا نَجَّىٰهُمْ إِلَى ٱلْبَرِّ إِذَا هُمْ يُشْرِكُونَ ﴿٦٥﴾\\
\textamh{66.\  } & لِيَكْفُرُوا۟ بِمَآ ءَاتَيْنَـٰهُمْ وَلِيَتَمَتَّعُوا۟ ۖ فَسَوْفَ يَعْلَمُونَ ﴿٦٦﴾\\
\textamh{67.\  } & أَوَلَمْ يَرَوْا۟ أَنَّا جَعَلْنَا حَرَمًا ءَامِنًۭا وَيُتَخَطَّفُ ٱلنَّاسُ مِنْ حَوْلِهِمْ ۚ أَفَبِٱلْبَٰطِلِ يُؤْمِنُونَ وَبِنِعْمَةِ ٱللَّهِ يَكْفُرُونَ ﴿٦٧﴾\\
\textamh{68.\  } & وَمَنْ أَظْلَمُ مِمَّنِ ٱفْتَرَىٰ عَلَى ٱللَّهِ كَذِبًا أَوْ كَذَّبَ بِٱلْحَقِّ لَمَّا جَآءَهُۥٓ ۚ أَلَيْسَ فِى جَهَنَّمَ مَثْوًۭى لِّلْكَـٰفِرِينَ ﴿٦٨﴾\\
\textamh{69.\  } & وَٱلَّذِينَ جَٰهَدُوا۟ فِينَا لَنَهْدِيَنَّهُمْ سُبُلَنَا ۚ وَإِنَّ ٱللَّهَ لَمَعَ ٱلْمُحْسِنِينَ ﴿٦٩﴾\\
\end{longtable}
\clearpage
%% License: BSD style (Berkley) (i.e. Put the Copyright owner's name always)
%% Writer and Copyright (to): Bewketu(Bilal) Tadilo (2016-17)
\centering\section{\LR{\textamharic{ሱራቱ አርሩም -}  \RL{سوره  الروم}}}
\begin{longtable}{%
  @{}
    p{.5\textwidth}
  @{~~~~~~~~~~~~~}
    p{.5\textwidth}
    @{}
}
\nopagebreak
\textamh{\ \ \ \ \ \  ቢስሚላሂ አራህመኒ ራሂይም } &  بِسْمِ ٱللَّهِ ٱلرَّحْمَـٰنِ ٱلرَّحِيمِ\\
\textamh{1.\  } &  الٓمٓ ﴿١﴾\\
\textamh{2.\  } & غُلِبَتِ ٱلرُّومُ ﴿٢﴾\\
\textamh{3.\  } & فِىٓ أَدْنَى ٱلْأَرْضِ وَهُم مِّنۢ بَعْدِ غَلَبِهِمْ سَيَغْلِبُونَ ﴿٣﴾\\
\textamh{4.\  } & فِى بِضْعِ سِنِينَ ۗ لِلَّهِ ٱلْأَمْرُ مِن قَبْلُ وَمِنۢ بَعْدُ ۚ وَيَوْمَئِذٍۢ يَفْرَحُ ٱلْمُؤْمِنُونَ ﴿٤﴾\\
\textamh{5.\  } & بِنَصْرِ ٱللَّهِ ۚ يَنصُرُ مَن يَشَآءُ ۖ وَهُوَ ٱلْعَزِيزُ ٱلرَّحِيمُ ﴿٥﴾\\
\textamh{6.\  } & وَعْدَ ٱللَّهِ ۖ لَا يُخْلِفُ ٱللَّهُ وَعْدَهُۥ وَلَـٰكِنَّ أَكْثَرَ ٱلنَّاسِ لَا يَعْلَمُونَ ﴿٦﴾\\
\textamh{7.\  } & يَعْلَمُونَ ظَـٰهِرًۭا مِّنَ ٱلْحَيَوٰةِ ٱلدُّنْيَا وَهُمْ عَنِ ٱلْءَاخِرَةِ هُمْ غَٰفِلُونَ ﴿٧﴾\\
\textamh{8.\  } & أَوَلَمْ يَتَفَكَّرُوا۟ فِىٓ أَنفُسِهِم ۗ مَّا خَلَقَ ٱللَّهُ ٱلسَّمَـٰوَٟتِ وَٱلْأَرْضَ وَمَا بَيْنَهُمَآ إِلَّا بِٱلْحَقِّ وَأَجَلٍۢ مُّسَمًّۭى ۗ وَإِنَّ كَثِيرًۭا مِّنَ ٱلنَّاسِ بِلِقَآئِ رَبِّهِمْ لَكَـٰفِرُونَ ﴿٨﴾\\
\textamh{9.\  } & أَوَلَمْ يَسِيرُوا۟ فِى ٱلْأَرْضِ فَيَنظُرُوا۟ كَيْفَ كَانَ عَـٰقِبَةُ ٱلَّذِينَ مِن قَبْلِهِمْ ۚ كَانُوٓا۟ أَشَدَّ مِنْهُمْ قُوَّةًۭ وَأَثَارُوا۟ ٱلْأَرْضَ وَعَمَرُوهَآ أَكْثَرَ مِمَّا عَمَرُوهَا وَجَآءَتْهُمْ رُسُلُهُم بِٱلْبَيِّنَـٰتِ ۖ فَمَا كَانَ ٱللَّهُ لِيَظْلِمَهُمْ وَلَـٰكِن كَانُوٓا۟ أَنفُسَهُمْ يَظْلِمُونَ ﴿٩﴾\\
\textamh{10.\  } & ثُمَّ كَانَ عَـٰقِبَةَ ٱلَّذِينَ أَسَـٰٓـُٔوا۟ ٱلسُّوٓأَىٰٓ أَن كَذَّبُوا۟ بِـَٔايَـٰتِ ٱللَّهِ وَكَانُوا۟ بِهَا يَسْتَهْزِءُونَ ﴿١٠﴾\\
\textamh{11.\  } & ٱللَّهُ يَبْدَؤُا۟ ٱلْخَلْقَ ثُمَّ يُعِيدُهُۥ ثُمَّ إِلَيْهِ تُرْجَعُونَ ﴿١١﴾\\
\textamh{12.\  } & وَيَوْمَ تَقُومُ ٱلسَّاعَةُ يُبْلِسُ ٱلْمُجْرِمُونَ ﴿١٢﴾\\
\textamh{13.\  } & وَلَمْ يَكُن لَّهُم مِّن شُرَكَآئِهِمْ شُفَعَـٰٓؤُا۟ وَكَانُوا۟ بِشُرَكَآئِهِمْ كَـٰفِرِينَ ﴿١٣﴾\\
\textamh{14.\  } & وَيَوْمَ تَقُومُ ٱلسَّاعَةُ يَوْمَئِذٍۢ يَتَفَرَّقُونَ ﴿١٤﴾\\
\textamh{15.\  } & فَأَمَّا ٱلَّذِينَ ءَامَنُوا۟ وَعَمِلُوا۟ ٱلصَّـٰلِحَـٰتِ فَهُمْ فِى رَوْضَةٍۢ يُحْبَرُونَ ﴿١٥﴾\\
\textamh{16.\  } & وَأَمَّا ٱلَّذِينَ كَفَرُوا۟ وَكَذَّبُوا۟ بِـَٔايَـٰتِنَا وَلِقَآئِ ٱلْءَاخِرَةِ فَأُو۟لَـٰٓئِكَ فِى ٱلْعَذَابِ مُحْضَرُونَ ﴿١٦﴾\\
\textamh{17.\  } & فَسُبْحَـٰنَ ٱللَّهِ حِينَ تُمْسُونَ وَحِينَ تُصْبِحُونَ ﴿١٧﴾\\
\textamh{18.\  } & وَلَهُ ٱلْحَمْدُ فِى ٱلسَّمَـٰوَٟتِ وَٱلْأَرْضِ وَعَشِيًّۭا وَحِينَ تُظْهِرُونَ ﴿١٨﴾\\
\textamh{19.\  } & يُخْرِجُ ٱلْحَىَّ مِنَ ٱلْمَيِّتِ وَيُخْرِجُ ٱلْمَيِّتَ مِنَ ٱلْحَىِّ وَيُحْىِ ٱلْأَرْضَ بَعْدَ مَوْتِهَا ۚ وَكَذَٟلِكَ تُخْرَجُونَ ﴿١٩﴾\\
\textamh{20.\  } & وَمِنْ ءَايَـٰتِهِۦٓ أَنْ خَلَقَكُم مِّن تُرَابٍۢ ثُمَّ إِذَآ أَنتُم بَشَرٌۭ تَنتَشِرُونَ ﴿٢٠﴾\\
\textamh{21.\  } & وَمِنْ ءَايَـٰتِهِۦٓ أَنْ خَلَقَ لَكُم مِّنْ أَنفُسِكُمْ أَزْوَٟجًۭا لِّتَسْكُنُوٓا۟ إِلَيْهَا وَجَعَلَ بَيْنَكُم مَّوَدَّةًۭ وَرَحْمَةً ۚ إِنَّ فِى ذَٟلِكَ لَءَايَـٰتٍۢ لِّقَوْمٍۢ يَتَفَكَّرُونَ ﴿٢١﴾\\
\textamh{22.\  } & وَمِنْ ءَايَـٰتِهِۦ خَلْقُ ٱلسَّمَـٰوَٟتِ وَٱلْأَرْضِ وَٱخْتِلَـٰفُ أَلْسِنَتِكُمْ وَأَلْوَٟنِكُمْ ۚ إِنَّ فِى ذَٟلِكَ لَءَايَـٰتٍۢ لِّلْعَـٰلِمِينَ ﴿٢٢﴾\\
\textamh{23.\  } & وَمِنْ ءَايَـٰتِهِۦ مَنَامُكُم بِٱلَّيْلِ وَٱلنَّهَارِ وَٱبْتِغَآؤُكُم مِّن فَضْلِهِۦٓ ۚ إِنَّ فِى ذَٟلِكَ لَءَايَـٰتٍۢ لِّقَوْمٍۢ يَسْمَعُونَ ﴿٢٣﴾\\
\textamh{24.\  } & وَمِنْ ءَايَـٰتِهِۦ يُرِيكُمُ ٱلْبَرْقَ خَوْفًۭا وَطَمَعًۭا وَيُنَزِّلُ مِنَ ٱلسَّمَآءِ مَآءًۭ فَيُحْىِۦ بِهِ ٱلْأَرْضَ بَعْدَ مَوْتِهَآ ۚ إِنَّ فِى ذَٟلِكَ لَءَايَـٰتٍۢ لِّقَوْمٍۢ يَعْقِلُونَ ﴿٢٤﴾\\
\textamh{25.\  } & وَمِنْ ءَايَـٰتِهِۦٓ أَن تَقُومَ ٱلسَّمَآءُ وَٱلْأَرْضُ بِأَمْرِهِۦ ۚ ثُمَّ إِذَا دَعَاكُمْ دَعْوَةًۭ مِّنَ ٱلْأَرْضِ إِذَآ أَنتُمْ تَخْرُجُونَ ﴿٢٥﴾\\
\textamh{26.\  } & وَلَهُۥ مَن فِى ٱلسَّمَـٰوَٟتِ وَٱلْأَرْضِ ۖ كُلٌّۭ لَّهُۥ قَـٰنِتُونَ ﴿٢٦﴾\\
\textamh{27.\  } & وَهُوَ ٱلَّذِى يَبْدَؤُا۟ ٱلْخَلْقَ ثُمَّ يُعِيدُهُۥ وَهُوَ أَهْوَنُ عَلَيْهِ ۚ وَلَهُ ٱلْمَثَلُ ٱلْأَعْلَىٰ فِى ٱلسَّمَـٰوَٟتِ وَٱلْأَرْضِ ۚ وَهُوَ ٱلْعَزِيزُ ٱلْحَكِيمُ ﴿٢٧﴾\\
\textamh{28.\  } & ضَرَبَ لَكُم مَّثَلًۭا مِّنْ أَنفُسِكُمْ ۖ هَل لَّكُم مِّن مَّا مَلَكَتْ أَيْمَـٰنُكُم مِّن شُرَكَآءَ فِى مَا رَزَقْنَـٰكُمْ فَأَنتُمْ فِيهِ سَوَآءٌۭ تَخَافُونَهُمْ كَخِيفَتِكُمْ أَنفُسَكُمْ ۚ كَذَٟلِكَ نُفَصِّلُ ٱلْءَايَـٰتِ لِقَوْمٍۢ يَعْقِلُونَ ﴿٢٨﴾\\
\textamh{29.\  } & بَلِ ٱتَّبَعَ ٱلَّذِينَ ظَلَمُوٓا۟ أَهْوَآءَهُم بِغَيْرِ عِلْمٍۢ ۖ فَمَن يَهْدِى مَنْ أَضَلَّ ٱللَّهُ ۖ وَمَا لَهُم مِّن نَّـٰصِرِينَ ﴿٢٩﴾\\
\textamh{30.\  } & فَأَقِمْ وَجْهَكَ لِلدِّينِ حَنِيفًۭا ۚ فِطْرَتَ ٱللَّهِ ٱلَّتِى فَطَرَ ٱلنَّاسَ عَلَيْهَا ۚ لَا تَبْدِيلَ لِخَلْقِ ٱللَّهِ ۚ ذَٟلِكَ ٱلدِّينُ ٱلْقَيِّمُ وَلَـٰكِنَّ أَكْثَرَ ٱلنَّاسِ لَا يَعْلَمُونَ ﴿٣٠﴾\\
\textamh{31.\  } & ۞ مُنِيبِينَ إِلَيْهِ وَٱتَّقُوهُ وَأَقِيمُوا۟ ٱلصَّلَوٰةَ وَلَا تَكُونُوا۟ مِنَ ٱلْمُشْرِكِينَ ﴿٣١﴾\\
\textamh{32.\  } & مِنَ ٱلَّذِينَ فَرَّقُوا۟ دِينَهُمْ وَكَانُوا۟ شِيَعًۭا ۖ كُلُّ حِزْبٍۭ بِمَا لَدَيْهِمْ فَرِحُونَ ﴿٣٢﴾\\
\textamh{33.\  } & وَإِذَا مَسَّ ٱلنَّاسَ ضُرٌّۭ دَعَوْا۟ رَبَّهُم مُّنِيبِينَ إِلَيْهِ ثُمَّ إِذَآ أَذَاقَهُم مِّنْهُ رَحْمَةً إِذَا فَرِيقٌۭ مِّنْهُم بِرَبِّهِمْ يُشْرِكُونَ ﴿٣٣﴾\\
\textamh{34.\  } & لِيَكْفُرُوا۟ بِمَآ ءَاتَيْنَـٰهُمْ ۚ فَتَمَتَّعُوا۟ فَسَوْفَ تَعْلَمُونَ ﴿٣٤﴾\\
\textamh{35.\  } & أَمْ أَنزَلْنَا عَلَيْهِمْ سُلْطَٰنًۭا فَهُوَ يَتَكَلَّمُ بِمَا كَانُوا۟ بِهِۦ يُشْرِكُونَ ﴿٣٥﴾\\
\textamh{36.\  } & وَإِذَآ أَذَقْنَا ٱلنَّاسَ رَحْمَةًۭ فَرِحُوا۟ بِهَا ۖ وَإِن تُصِبْهُمْ سَيِّئَةٌۢ بِمَا قَدَّمَتْ أَيْدِيهِمْ إِذَا هُمْ يَقْنَطُونَ ﴿٣٦﴾\\
\textamh{37.\  } & أَوَلَمْ يَرَوْا۟ أَنَّ ٱللَّهَ يَبْسُطُ ٱلرِّزْقَ لِمَن يَشَآءُ وَيَقْدِرُ ۚ إِنَّ فِى ذَٟلِكَ لَءَايَـٰتٍۢ لِّقَوْمٍۢ يُؤْمِنُونَ ﴿٣٧﴾\\
\textamh{38.\  } & فَـَٔاتِ ذَا ٱلْقُرْبَىٰ حَقَّهُۥ وَٱلْمِسْكِينَ وَٱبْنَ ٱلسَّبِيلِ ۚ ذَٟلِكَ خَيْرٌۭ لِّلَّذِينَ يُرِيدُونَ وَجْهَ ٱللَّهِ ۖ وَأُو۟لَـٰٓئِكَ هُمُ ٱلْمُفْلِحُونَ ﴿٣٨﴾\\
\textamh{39.\  } & وَمَآ ءَاتَيْتُم مِّن رِّبًۭا لِّيَرْبُوَا۟ فِىٓ أَمْوَٟلِ ٱلنَّاسِ فَلَا يَرْبُوا۟ عِندَ ٱللَّهِ ۖ وَمَآ ءَاتَيْتُم مِّن زَكَوٰةٍۢ تُرِيدُونَ وَجْهَ ٱللَّهِ فَأُو۟لَـٰٓئِكَ هُمُ ٱلْمُضْعِفُونَ ﴿٣٩﴾\\
\textamh{40.\  } & ٱللَّهُ ٱلَّذِى خَلَقَكُمْ ثُمَّ رَزَقَكُمْ ثُمَّ يُمِيتُكُمْ ثُمَّ يُحْيِيكُمْ ۖ هَلْ مِن شُرَكَآئِكُم مَّن يَفْعَلُ مِن ذَٟلِكُم مِّن شَىْءٍۢ ۚ سُبْحَـٰنَهُۥ وَتَعَـٰلَىٰ عَمَّا يُشْرِكُونَ ﴿٤٠﴾\\
\textamh{41.\  } & ظَهَرَ ٱلْفَسَادُ فِى ٱلْبَرِّ وَٱلْبَحْرِ بِمَا كَسَبَتْ أَيْدِى ٱلنَّاسِ لِيُذِيقَهُم بَعْضَ ٱلَّذِى عَمِلُوا۟ لَعَلَّهُمْ يَرْجِعُونَ ﴿٤١﴾\\
\textamh{42.\  } & قُلْ سِيرُوا۟ فِى ٱلْأَرْضِ فَٱنظُرُوا۟ كَيْفَ كَانَ عَـٰقِبَةُ ٱلَّذِينَ مِن قَبْلُ ۚ كَانَ أَكْثَرُهُم مُّشْرِكِينَ ﴿٤٢﴾\\
\textamh{43.\  } & فَأَقِمْ وَجْهَكَ لِلدِّينِ ٱلْقَيِّمِ مِن قَبْلِ أَن يَأْتِىَ يَوْمٌۭ لَّا مَرَدَّ لَهُۥ مِنَ ٱللَّهِ ۖ يَوْمَئِذٍۢ يَصَّدَّعُونَ ﴿٤٣﴾\\
\textamh{44.\  } & مَن كَفَرَ فَعَلَيْهِ كُفْرُهُۥ ۖ وَمَنْ عَمِلَ صَـٰلِحًۭا فَلِأَنفُسِهِمْ يَمْهَدُونَ ﴿٤٤﴾\\
\textamh{45.\  } & لِيَجْزِىَ ٱلَّذِينَ ءَامَنُوا۟ وَعَمِلُوا۟ ٱلصَّـٰلِحَـٰتِ مِن فَضْلِهِۦٓ ۚ إِنَّهُۥ لَا يُحِبُّ ٱلْكَـٰفِرِينَ ﴿٤٥﴾\\
\textamh{46.\  } & وَمِنْ ءَايَـٰتِهِۦٓ أَن يُرْسِلَ ٱلرِّيَاحَ مُبَشِّرَٰتٍۢ وَلِيُذِيقَكُم مِّن رَّحْمَتِهِۦ وَلِتَجْرِىَ ٱلْفُلْكُ بِأَمْرِهِۦ وَلِتَبْتَغُوا۟ مِن فَضْلِهِۦ وَلَعَلَّكُمْ تَشْكُرُونَ ﴿٤٦﴾\\
\textamh{47.\  } & وَلَقَدْ أَرْسَلْنَا مِن قَبْلِكَ رُسُلًا إِلَىٰ قَوْمِهِمْ فَجَآءُوهُم بِٱلْبَيِّنَـٰتِ فَٱنتَقَمْنَا مِنَ ٱلَّذِينَ أَجْرَمُوا۟ ۖ وَكَانَ حَقًّا عَلَيْنَا نَصْرُ ٱلْمُؤْمِنِينَ ﴿٤٧﴾\\
\textamh{48.\  } & ٱللَّهُ ٱلَّذِى يُرْسِلُ ٱلرِّيَـٰحَ فَتُثِيرُ سَحَابًۭا فَيَبْسُطُهُۥ فِى ٱلسَّمَآءِ كَيْفَ يَشَآءُ وَيَجْعَلُهُۥ كِسَفًۭا فَتَرَى ٱلْوَدْقَ يَخْرُجُ مِنْ خِلَـٰلِهِۦ ۖ فَإِذَآ أَصَابَ بِهِۦ مَن يَشَآءُ مِنْ عِبَادِهِۦٓ إِذَا هُمْ يَسْتَبْشِرُونَ ﴿٤٨﴾\\
\textamh{49.\  } & وَإِن كَانُوا۟ مِن قَبْلِ أَن يُنَزَّلَ عَلَيْهِم مِّن قَبْلِهِۦ لَمُبْلِسِينَ ﴿٤٩﴾\\
\textamh{50.\  } & فَٱنظُرْ إِلَىٰٓ ءَاثَـٰرِ رَحْمَتِ ٱللَّهِ كَيْفَ يُحْىِ ٱلْأَرْضَ بَعْدَ مَوْتِهَآ ۚ إِنَّ ذَٟلِكَ لَمُحْىِ ٱلْمَوْتَىٰ ۖ وَهُوَ عَلَىٰ كُلِّ شَىْءٍۢ قَدِيرٌۭ ﴿٥٠﴾\\
\textamh{51.\  } & وَلَئِنْ أَرْسَلْنَا رِيحًۭا فَرَأَوْهُ مُصْفَرًّۭا لَّظَلُّوا۟ مِنۢ بَعْدِهِۦ يَكْفُرُونَ ﴿٥١﴾\\
\textamh{52.\  } & فَإِنَّكَ لَا تُسْمِعُ ٱلْمَوْتَىٰ وَلَا تُسْمِعُ ٱلصُّمَّ ٱلدُّعَآءَ إِذَا وَلَّوْا۟ مُدْبِرِينَ ﴿٥٢﴾\\
\textamh{53.\  } & وَمَآ أَنتَ بِهَـٰدِ ٱلْعُمْىِ عَن ضَلَـٰلَتِهِمْ ۖ إِن تُسْمِعُ إِلَّا مَن يُؤْمِنُ بِـَٔايَـٰتِنَا فَهُم مُّسْلِمُونَ ﴿٥٣﴾\\
\textamh{54.\  } & ۞ ٱللَّهُ ٱلَّذِى خَلَقَكُم مِّن ضَعْفٍۢ ثُمَّ جَعَلَ مِنۢ بَعْدِ ضَعْفٍۢ قُوَّةًۭ ثُمَّ جَعَلَ مِنۢ بَعْدِ قُوَّةٍۢ ضَعْفًۭا وَشَيْبَةًۭ ۚ يَخْلُقُ مَا يَشَآءُ ۖ وَهُوَ ٱلْعَلِيمُ ٱلْقَدِيرُ ﴿٥٤﴾\\
\textamh{55.\  } & وَيَوْمَ تَقُومُ ٱلسَّاعَةُ يُقْسِمُ ٱلْمُجْرِمُونَ مَا لَبِثُوا۟ غَيْرَ سَاعَةٍۢ ۚ كَذَٟلِكَ كَانُوا۟ يُؤْفَكُونَ ﴿٥٥﴾\\
\textamh{56.\  } & وَقَالَ ٱلَّذِينَ أُوتُوا۟ ٱلْعِلْمَ وَٱلْإِيمَـٰنَ لَقَدْ لَبِثْتُمْ فِى كِتَـٰبِ ٱللَّهِ إِلَىٰ يَوْمِ ٱلْبَعْثِ ۖ فَهَـٰذَا يَوْمُ ٱلْبَعْثِ وَلَـٰكِنَّكُمْ كُنتُمْ لَا تَعْلَمُونَ ﴿٥٦﴾\\
\textamh{57.\  } & فَيَوْمَئِذٍۢ لَّا يَنفَعُ ٱلَّذِينَ ظَلَمُوا۟ مَعْذِرَتُهُمْ وَلَا هُمْ يُسْتَعْتَبُونَ ﴿٥٧﴾\\
\textamh{58.\  } & وَلَقَدْ ضَرَبْنَا لِلنَّاسِ فِى هَـٰذَا ٱلْقُرْءَانِ مِن كُلِّ مَثَلٍۢ ۚ وَلَئِن جِئْتَهُم بِـَٔايَةٍۢ لَّيَقُولَنَّ ٱلَّذِينَ كَفَرُوٓا۟ إِنْ أَنتُمْ إِلَّا مُبْطِلُونَ ﴿٥٨﴾\\
\textamh{59.\  } & كَذَٟلِكَ يَطْبَعُ ٱللَّهُ عَلَىٰ قُلُوبِ ٱلَّذِينَ لَا يَعْلَمُونَ ﴿٥٩﴾\\
\textamh{60.\  } & فَٱصْبِرْ إِنَّ وَعْدَ ٱللَّهِ حَقٌّۭ ۖ وَلَا يَسْتَخِفَّنَّكَ ٱلَّذِينَ لَا يُوقِنُونَ ﴿٦٠﴾\\
\end{longtable} \newpage

%% License: BSD style (Berkley) (i.e. Put the Copyright owner's name always)
%% Writer and Copyright (to): Bewketu(Bilal) Tadilo (2016-17)
\centering\section{\LR{\textamharic{ሱራቱ ሉቅማን -}  \RL{سوره  لقمان}}}
\begin{longtable}{%
  @{}
    p{.5\textwidth}
  @{~~~~~~~~~~~~~}
    p{.5\textwidth}
    @{}
}
\nopagebreak
\textamh{ቢስሚላሂ አራህመኒ ራሂይም } &  بِسْمِ ٱللَّهِ ٱلرَّحْمَـٰنِ ٱلرَّحِيمِ\\
\textamh{1.\  } &  الٓمٓ ﴿١﴾\\
\textamh{2.\  } & تِلْكَ ءَايَـٰتُ ٱلْكِتَـٰبِ ٱلْحَكِيمِ ﴿٢﴾\\
\textamh{3.\  } & هُدًۭى وَرَحْمَةًۭ لِّلْمُحْسِنِينَ ﴿٣﴾\\
\textamh{4.\  } & ٱلَّذِينَ يُقِيمُونَ ٱلصَّلَوٰةَ وَيُؤْتُونَ ٱلزَّكَوٰةَ وَهُم بِٱلْءَاخِرَةِ هُمْ يُوقِنُونَ ﴿٤﴾\\
\textamh{5.\  } & أُو۟لَـٰٓئِكَ عَلَىٰ هُدًۭى مِّن رَّبِّهِمْ ۖ وَأُو۟لَـٰٓئِكَ هُمُ ٱلْمُفْلِحُونَ ﴿٥﴾\\
\textamh{6.\  } & وَمِنَ ٱلنَّاسِ مَن يَشْتَرِى لَهْوَ ٱلْحَدِيثِ لِيُضِلَّ عَن سَبِيلِ ٱللَّهِ بِغَيْرِ عِلْمٍۢ وَيَتَّخِذَهَا هُزُوًا ۚ أُو۟لَـٰٓئِكَ لَهُمْ عَذَابٌۭ مُّهِينٌۭ ﴿٦﴾\\
\textamh{7.\  } & وَإِذَا تُتْلَىٰ عَلَيْهِ ءَايَـٰتُنَا وَلَّىٰ مُسْتَكْبِرًۭا كَأَن لَّمْ يَسْمَعْهَا كَأَنَّ فِىٓ أُذُنَيْهِ وَقْرًۭا ۖ فَبَشِّرْهُ بِعَذَابٍ أَلِيمٍ ﴿٧﴾\\
\textamh{8.\  } & إِنَّ ٱلَّذِينَ ءَامَنُوا۟ وَعَمِلُوا۟ ٱلصَّـٰلِحَـٰتِ لَهُمْ جَنَّـٰتُ ٱلنَّعِيمِ ﴿٨﴾\\
\textamh{9.\  } & خَـٰلِدِينَ فِيهَا ۖ وَعْدَ ٱللَّهِ حَقًّۭا ۚ وَهُوَ ٱلْعَزِيزُ ٱلْحَكِيمُ ﴿٩﴾\\
\textamh{10.\  } & خَلَقَ ٱلسَّمَـٰوَٟتِ بِغَيْرِ عَمَدٍۢ تَرَوْنَهَا ۖ وَأَلْقَىٰ فِى ٱلْأَرْضِ رَوَٟسِىَ أَن تَمِيدَ بِكُمْ وَبَثَّ فِيهَا مِن كُلِّ دَآبَّةٍۢ ۚ وَأَنزَلْنَا مِنَ ٱلسَّمَآءِ مَآءًۭ فَأَنۢبَتْنَا فِيهَا مِن كُلِّ زَوْجٍۢ كَرِيمٍ ﴿١٠﴾\\
\textamh{11.\  } & هَـٰذَا خَلْقُ ٱللَّهِ فَأَرُونِى مَاذَا خَلَقَ ٱلَّذِينَ مِن دُونِهِۦ ۚ بَلِ ٱلظَّـٰلِمُونَ فِى ضَلَـٰلٍۢ مُّبِينٍۢ ﴿١١﴾\\
\textamh{12.\  } & وَلَقَدْ ءَاتَيْنَا لُقْمَـٰنَ ٱلْحِكْمَةَ أَنِ ٱشْكُرْ لِلَّهِ ۚ وَمَن يَشْكُرْ فَإِنَّمَا يَشْكُرُ لِنَفْسِهِۦ ۖ وَمَن كَفَرَ فَإِنَّ ٱللَّهَ غَنِىٌّ حَمِيدٌۭ ﴿١٢﴾\\
\textamh{13.\  } & وَإِذْ قَالَ لُقْمَـٰنُ لِٱبْنِهِۦ وَهُوَ يَعِظُهُۥ يَـٰبُنَىَّ لَا تُشْرِكْ بِٱللَّهِ ۖ إِنَّ ٱلشِّرْكَ لَظُلْمٌ عَظِيمٌۭ ﴿١٣﴾\\
\textamh{14.\  } & وَوَصَّيْنَا ٱلْإِنسَـٰنَ بِوَٟلِدَيْهِ حَمَلَتْهُ أُمُّهُۥ وَهْنًا عَلَىٰ وَهْنٍۢ وَفِصَـٰلُهُۥ فِى عَامَيْنِ أَنِ ٱشْكُرْ لِى وَلِوَٟلِدَيْكَ إِلَىَّ ٱلْمَصِيرُ ﴿١٤﴾\\
\textamh{15.\  } & وَإِن جَٰهَدَاكَ عَلَىٰٓ أَن تُشْرِكَ بِى مَا لَيْسَ لَكَ بِهِۦ عِلْمٌۭ فَلَا تُطِعْهُمَا ۖ وَصَاحِبْهُمَا فِى ٱلدُّنْيَا مَعْرُوفًۭا ۖ وَٱتَّبِعْ سَبِيلَ مَنْ أَنَابَ إِلَىَّ ۚ ثُمَّ إِلَىَّ مَرْجِعُكُمْ فَأُنَبِّئُكُم بِمَا كُنتُمْ تَعْمَلُونَ ﴿١٥﴾\\
\textamh{16.\  } & يَـٰبُنَىَّ إِنَّهَآ إِن تَكُ مِثْقَالَ حَبَّةٍۢ مِّنْ خَرْدَلٍۢ فَتَكُن فِى صَخْرَةٍ أَوْ فِى ٱلسَّمَـٰوَٟتِ أَوْ فِى ٱلْأَرْضِ يَأْتِ بِهَا ٱللَّهُ ۚ إِنَّ ٱللَّهَ لَطِيفٌ خَبِيرٌۭ ﴿١٦﴾\\
\textamh{17.\  } & يَـٰبُنَىَّ أَقِمِ ٱلصَّلَوٰةَ وَأْمُرْ بِٱلْمَعْرُوفِ وَٱنْهَ عَنِ ٱلْمُنكَرِ وَٱصْبِرْ عَلَىٰ مَآ أَصَابَكَ ۖ إِنَّ ذَٟلِكَ مِنْ عَزْمِ ٱلْأُمُورِ ﴿١٧﴾\\
\textamh{18.\  } & وَلَا تُصَعِّرْ خَدَّكَ لِلنَّاسِ وَلَا تَمْشِ فِى ٱلْأَرْضِ مَرَحًا ۖ إِنَّ ٱللَّهَ لَا يُحِبُّ كُلَّ مُخْتَالٍۢ فَخُورٍۢ ﴿١٨﴾\\
\textamh{19.\  } & وَٱقْصِدْ فِى مَشْيِكَ وَٱغْضُضْ مِن صَوْتِكَ ۚ إِنَّ أَنكَرَ ٱلْأَصْوَٟتِ لَصَوْتُ ٱلْحَمِيرِ ﴿١٩﴾\\
\textamh{20.\  } & أَلَمْ تَرَوْا۟ أَنَّ ٱللَّهَ سَخَّرَ لَكُم مَّا فِى ٱلسَّمَـٰوَٟتِ وَمَا فِى ٱلْأَرْضِ وَأَسْبَغَ عَلَيْكُمْ نِعَمَهُۥ ظَـٰهِرَةًۭ وَبَاطِنَةًۭ ۗ وَمِنَ ٱلنَّاسِ مَن يُجَٰدِلُ فِى ٱللَّهِ بِغَيْرِ عِلْمٍۢ وَلَا هُدًۭى وَلَا كِتَـٰبٍۢ مُّنِيرٍۢ ﴿٢٠﴾\\
\textamh{21.\  } & وَإِذَا قِيلَ لَهُمُ ٱتَّبِعُوا۟ مَآ أَنزَلَ ٱللَّهُ قَالُوا۟ بَلْ نَتَّبِعُ مَا وَجَدْنَا عَلَيْهِ ءَابَآءَنَآ ۚ أَوَلَوْ كَانَ ٱلشَّيْطَٰنُ يَدْعُوهُمْ إِلَىٰ عَذَابِ ٱلسَّعِيرِ ﴿٢١﴾\\
\textamh{22.\  } & ۞ وَمَن يُسْلِمْ وَجْهَهُۥٓ إِلَى ٱللَّهِ وَهُوَ مُحْسِنٌۭ فَقَدِ ٱسْتَمْسَكَ بِٱلْعُرْوَةِ ٱلْوُثْقَىٰ ۗ وَإِلَى ٱللَّهِ عَـٰقِبَةُ ٱلْأُمُورِ ﴿٢٢﴾\\
\textamh{23.\  } & وَمَن كَفَرَ فَلَا يَحْزُنكَ كُفْرُهُۥٓ ۚ إِلَيْنَا مَرْجِعُهُمْ فَنُنَبِّئُهُم بِمَا عَمِلُوٓا۟ ۚ إِنَّ ٱللَّهَ عَلِيمٌۢ بِذَاتِ ٱلصُّدُورِ ﴿٢٣﴾\\
\textamh{24.\  } & نُمَتِّعُهُمْ قَلِيلًۭا ثُمَّ نَضْطَرُّهُمْ إِلَىٰ عَذَابٍ غَلِيظٍۢ ﴿٢٤﴾\\
\textamh{25.\  } & وَلَئِن سَأَلْتَهُم مَّنْ خَلَقَ ٱلسَّمَـٰوَٟتِ وَٱلْأَرْضَ لَيَقُولُنَّ ٱللَّهُ ۚ قُلِ ٱلْحَمْدُ لِلَّهِ ۚ بَلْ أَكْثَرُهُمْ لَا يَعْلَمُونَ ﴿٢٥﴾\\
\textamh{26.\  } & لِلَّهِ مَا فِى ٱلسَّمَـٰوَٟتِ وَٱلْأَرْضِ ۚ إِنَّ ٱللَّهَ هُوَ ٱلْغَنِىُّ ٱلْحَمِيدُ ﴿٢٦﴾\\
\textamh{27.\  } & وَلَوْ أَنَّمَا فِى ٱلْأَرْضِ مِن شَجَرَةٍ أَقْلَـٰمٌۭ وَٱلْبَحْرُ يَمُدُّهُۥ مِنۢ بَعْدِهِۦ سَبْعَةُ أَبْحُرٍۢ مَّا نَفِدَتْ كَلِمَـٰتُ ٱللَّهِ ۗ إِنَّ ٱللَّهَ عَزِيزٌ حَكِيمٌۭ ﴿٢٧﴾\\
\textamh{28.\  } & مَّا خَلْقُكُمْ وَلَا بَعْثُكُمْ إِلَّا كَنَفْسٍۢ وَٟحِدَةٍ ۗ إِنَّ ٱللَّهَ سَمِيعٌۢ بَصِيرٌ ﴿٢٨﴾\\
\textamh{29.\  } & أَلَمْ تَرَ أَنَّ ٱللَّهَ يُولِجُ ٱلَّيْلَ فِى ٱلنَّهَارِ وَيُولِجُ ٱلنَّهَارَ فِى ٱلَّيْلِ وَسَخَّرَ ٱلشَّمْسَ وَٱلْقَمَرَ كُلٌّۭ يَجْرِىٓ إِلَىٰٓ أَجَلٍۢ مُّسَمًّۭى وَأَنَّ ٱللَّهَ بِمَا تَعْمَلُونَ خَبِيرٌۭ ﴿٢٩﴾\\
\textamh{30.\  } & ذَٟلِكَ بِأَنَّ ٱللَّهَ هُوَ ٱلْحَقُّ وَأَنَّ مَا يَدْعُونَ مِن دُونِهِ ٱلْبَٰطِلُ وَأَنَّ ٱللَّهَ هُوَ ٱلْعَلِىُّ ٱلْكَبِيرُ ﴿٣٠﴾\\
\textamh{31.\  } & أَلَمْ تَرَ أَنَّ ٱلْفُلْكَ تَجْرِى فِى ٱلْبَحْرِ بِنِعْمَتِ ٱللَّهِ لِيُرِيَكُم مِّنْ ءَايَـٰتِهِۦٓ ۚ إِنَّ فِى ذَٟلِكَ لَءَايَـٰتٍۢ لِّكُلِّ صَبَّارٍۢ شَكُورٍۢ ﴿٣١﴾\\
\textamh{32.\  } & وَإِذَا غَشِيَهُم مَّوْجٌۭ كَٱلظُّلَلِ دَعَوُا۟ ٱللَّهَ مُخْلِصِينَ لَهُ ٱلدِّينَ فَلَمَّا نَجَّىٰهُمْ إِلَى ٱلْبَرِّ فَمِنْهُم مُّقْتَصِدٌۭ ۚ وَمَا يَجْحَدُ بِـَٔايَـٰتِنَآ إِلَّا كُلُّ خَتَّارٍۢ كَفُورٍۢ ﴿٣٢﴾\\
\textamh{33.\  } & يَـٰٓأَيُّهَا ٱلنَّاسُ ٱتَّقُوا۟ رَبَّكُمْ وَٱخْشَوْا۟ يَوْمًۭا لَّا يَجْزِى وَالِدٌ عَن وَلَدِهِۦ وَلَا مَوْلُودٌ هُوَ جَازٍ عَن وَالِدِهِۦ شَيْـًٔا ۚ إِنَّ وَعْدَ ٱللَّهِ حَقٌّۭ ۖ فَلَا تَغُرَّنَّكُمُ ٱلْحَيَوٰةُ ٱلدُّنْيَا وَلَا يَغُرَّنَّكُم بِٱللَّهِ ٱلْغَرُورُ ﴿٣٣﴾\\
\textamh{34.\  } & إِنَّ ٱللَّهَ عِندَهُۥ عِلْمُ ٱلسَّاعَةِ وَيُنَزِّلُ ٱلْغَيْثَ وَيَعْلَمُ مَا فِى ٱلْأَرْحَامِ ۖ وَمَا تَدْرِى نَفْسٌۭ مَّاذَا تَكْسِبُ غَدًۭا ۖ وَمَا تَدْرِى نَفْسٌۢ بِأَىِّ أَرْضٍۢ تَمُوتُ ۚ إِنَّ ٱللَّهَ عَلِيمٌ خَبِيرٌۢ ﴿٣٤﴾\\
\end{longtable}
\clearpage
%% License: BSD style (Berkley) (i.e. Put the Copyright owner's name always)
%% Writer and Copyright (to): Bewketu(Bilal) Tadilo (2016-17)
\centering\section{\LR{\textamharic{ሱራቱ አስሰጅደ -}  \RL{سوره  السجدة}}}
\begin{longtable}{%
  @{}
    p{.5\textwidth}
  @{~~~~~~~~~~~~~}
    p{.5\textwidth}
    @{}
}
\nopagebreak
\textamh{\ \ \ \ \ \  ቢስሚላሂ አራህመኒ ራሂይም } &  بِسْمِ ٱللَّهِ ٱلرَّحْمَـٰنِ ٱلرَّحِيمِ\\
\textamh{1.\  } &  الٓمٓ ﴿١﴾\\
\textamh{2.\  } & تَنزِيلُ ٱلْكِتَـٰبِ لَا رَيْبَ فِيهِ مِن رَّبِّ ٱلْعَـٰلَمِينَ ﴿٢﴾\\
\textamh{3.\  } & أَمْ يَقُولُونَ ٱفْتَرَىٰهُ ۚ بَلْ هُوَ ٱلْحَقُّ مِن رَّبِّكَ لِتُنذِرَ قَوْمًۭا مَّآ أَتَىٰهُم مِّن نَّذِيرٍۢ مِّن قَبْلِكَ لَعَلَّهُمْ يَهْتَدُونَ ﴿٣﴾\\
\textamh{4.\  } & ٱللَّهُ ٱلَّذِى خَلَقَ ٱلسَّمَـٰوَٟتِ وَٱلْأَرْضَ وَمَا بَيْنَهُمَا فِى سِتَّةِ أَيَّامٍۢ ثُمَّ ٱسْتَوَىٰ عَلَى ٱلْعَرْشِ ۖ مَا لَكُم مِّن دُونِهِۦ مِن وَلِىٍّۢ وَلَا شَفِيعٍ ۚ أَفَلَا تَتَذَكَّرُونَ ﴿٤﴾\\
\textamh{5.\  } & يُدَبِّرُ ٱلْأَمْرَ مِنَ ٱلسَّمَآءِ إِلَى ٱلْأَرْضِ ثُمَّ يَعْرُجُ إِلَيْهِ فِى يَوْمٍۢ كَانَ مِقْدَارُهُۥٓ أَلْفَ سَنَةٍۢ مِّمَّا تَعُدُّونَ ﴿٥﴾\\
\textamh{6.\  } & ذَٟلِكَ عَـٰلِمُ ٱلْغَيْبِ وَٱلشَّهَـٰدَةِ ٱلْعَزِيزُ ٱلرَّحِيمُ ﴿٦﴾\\
\textamh{7.\  } & ٱلَّذِىٓ أَحْسَنَ كُلَّ شَىْءٍ خَلَقَهُۥ ۖ وَبَدَأَ خَلْقَ ٱلْإِنسَـٰنِ مِن طِينٍۢ ﴿٧﴾\\
\textamh{8.\  } & ثُمَّ جَعَلَ نَسْلَهُۥ مِن سُلَـٰلَةٍۢ مِّن مَّآءٍۢ مَّهِينٍۢ ﴿٨﴾\\
\textamh{9.\  } & ثُمَّ سَوَّىٰهُ وَنَفَخَ فِيهِ مِن رُّوحِهِۦ ۖ وَجَعَلَ لَكُمُ ٱلسَّمْعَ وَٱلْأَبْصَـٰرَ وَٱلْأَفْـِٔدَةَ ۚ قَلِيلًۭا مَّا تَشْكُرُونَ ﴿٩﴾\\
\textamh{10.\  } & وَقَالُوٓا۟ أَءِذَا ضَلَلْنَا فِى ٱلْأَرْضِ أَءِنَّا لَفِى خَلْقٍۢ جَدِيدٍۭ ۚ بَلْ هُم بِلِقَآءِ رَبِّهِمْ كَـٰفِرُونَ ﴿١٠﴾\\
\textamh{11.\  } & ۞ قُلْ يَتَوَفَّىٰكُم مَّلَكُ ٱلْمَوْتِ ٱلَّذِى وُكِّلَ بِكُمْ ثُمَّ إِلَىٰ رَبِّكُمْ تُرْجَعُونَ ﴿١١﴾\\
\textamh{12.\  } & وَلَوْ تَرَىٰٓ إِذِ ٱلْمُجْرِمُونَ نَاكِسُوا۟ رُءُوسِهِمْ عِندَ رَبِّهِمْ رَبَّنَآ أَبْصَرْنَا وَسَمِعْنَا فَٱرْجِعْنَا نَعْمَلْ صَـٰلِحًا إِنَّا مُوقِنُونَ ﴿١٢﴾\\
\textamh{13.\  } & وَلَوْ شِئْنَا لَءَاتَيْنَا كُلَّ نَفْسٍ هُدَىٰهَا وَلَـٰكِنْ حَقَّ ٱلْقَوْلُ مِنِّى لَأَمْلَأَنَّ جَهَنَّمَ مِنَ ٱلْجِنَّةِ وَٱلنَّاسِ أَجْمَعِينَ ﴿١٣﴾\\
\textamh{14.\  } & فَذُوقُوا۟ بِمَا نَسِيتُمْ لِقَآءَ يَوْمِكُمْ هَـٰذَآ إِنَّا نَسِينَـٰكُمْ ۖ وَذُوقُوا۟ عَذَابَ ٱلْخُلْدِ بِمَا كُنتُمْ تَعْمَلُونَ ﴿١٤﴾\\
\textamh{15.\  } & إِنَّمَا يُؤْمِنُ بِـَٔايَـٰتِنَا ٱلَّذِينَ إِذَا ذُكِّرُوا۟ بِهَا خَرُّوا۟ سُجَّدًۭا وَسَبَّحُوا۟ بِحَمْدِ رَبِّهِمْ وَهُمْ لَا يَسْتَكْبِرُونَ ۩ ﴿١٥﴾\\
\textamh{16.\  } & تَتَجَافَىٰ جُنُوبُهُمْ عَنِ ٱلْمَضَاجِعِ يَدْعُونَ رَبَّهُمْ خَوْفًۭا وَطَمَعًۭا وَمِمَّا رَزَقْنَـٰهُمْ يُنفِقُونَ ﴿١٦﴾\\
\textamh{17.\  } & فَلَا تَعْلَمُ نَفْسٌۭ مَّآ أُخْفِىَ لَهُم مِّن قُرَّةِ أَعْيُنٍۢ جَزَآءًۢ بِمَا كَانُوا۟ يَعْمَلُونَ ﴿١٧﴾\\
\textamh{18.\  } & أَفَمَن كَانَ مُؤْمِنًۭا كَمَن كَانَ فَاسِقًۭا ۚ لَّا يَسْتَوُۥنَ ﴿١٨﴾\\
\textamh{19.\  } & أَمَّا ٱلَّذِينَ ءَامَنُوا۟ وَعَمِلُوا۟ ٱلصَّـٰلِحَـٰتِ فَلَهُمْ جَنَّـٰتُ ٱلْمَأْوَىٰ نُزُلًۢا بِمَا كَانُوا۟ يَعْمَلُونَ ﴿١٩﴾\\
\textamh{20.\  } & وَأَمَّا ٱلَّذِينَ فَسَقُوا۟ فَمَأْوَىٰهُمُ ٱلنَّارُ ۖ كُلَّمَآ أَرَادُوٓا۟ أَن يَخْرُجُوا۟ مِنْهَآ أُعِيدُوا۟ فِيهَا وَقِيلَ لَهُمْ ذُوقُوا۟ عَذَابَ ٱلنَّارِ ٱلَّذِى كُنتُم بِهِۦ تُكَذِّبُونَ ﴿٢٠﴾\\
\textamh{21.\  } & وَلَنُذِيقَنَّهُم مِّنَ ٱلْعَذَابِ ٱلْأَدْنَىٰ دُونَ ٱلْعَذَابِ ٱلْأَكْبَرِ لَعَلَّهُمْ يَرْجِعُونَ ﴿٢١﴾\\
\textamh{22.\  } & وَمَنْ أَظْلَمُ مِمَّن ذُكِّرَ بِـَٔايَـٰتِ رَبِّهِۦ ثُمَّ أَعْرَضَ عَنْهَآ ۚ إِنَّا مِنَ ٱلْمُجْرِمِينَ مُنتَقِمُونَ ﴿٢٢﴾\\
\textamh{23.\  } & وَلَقَدْ ءَاتَيْنَا مُوسَى ٱلْكِتَـٰبَ فَلَا تَكُن فِى مِرْيَةٍۢ مِّن لِّقَآئِهِۦ ۖ وَجَعَلْنَـٰهُ هُدًۭى لِّبَنِىٓ إِسْرَٰٓءِيلَ ﴿٢٣﴾\\
\textamh{24.\  } & وَجَعَلْنَا مِنْهُمْ أَئِمَّةًۭ يَهْدُونَ بِأَمْرِنَا لَمَّا صَبَرُوا۟ ۖ وَكَانُوا۟ بِـَٔايَـٰتِنَا يُوقِنُونَ ﴿٢٤﴾\\
\textamh{25.\  } & إِنَّ رَبَّكَ هُوَ يَفْصِلُ بَيْنَهُمْ يَوْمَ ٱلْقِيَـٰمَةِ فِيمَا كَانُوا۟ فِيهِ يَخْتَلِفُونَ ﴿٢٥﴾\\
\textamh{26.\  } & أَوَلَمْ يَهْدِ لَهُمْ كَمْ أَهْلَكْنَا مِن قَبْلِهِم مِّنَ ٱلْقُرُونِ يَمْشُونَ فِى مَسَـٰكِنِهِمْ ۚ إِنَّ فِى ذَٟلِكَ لَءَايَـٰتٍ ۖ أَفَلَا يَسْمَعُونَ ﴿٢٦﴾\\
\textamh{27.\  } & أَوَلَمْ يَرَوْا۟ أَنَّا نَسُوقُ ٱلْمَآءَ إِلَى ٱلْأَرْضِ ٱلْجُرُزِ فَنُخْرِجُ بِهِۦ زَرْعًۭا تَأْكُلُ مِنْهُ أَنْعَـٰمُهُمْ وَأَنفُسُهُمْ ۖ أَفَلَا يُبْصِرُونَ ﴿٢٧﴾\\
\textamh{28.\  } & وَيَقُولُونَ مَتَىٰ هَـٰذَا ٱلْفَتْحُ إِن كُنتُمْ صَـٰدِقِينَ ﴿٢٨﴾\\
\textamh{29.\  } & قُلْ يَوْمَ ٱلْفَتْحِ لَا يَنفَعُ ٱلَّذِينَ كَفَرُوٓا۟ إِيمَـٰنُهُمْ وَلَا هُمْ يُنظَرُونَ ﴿٢٩﴾\\
\textamh{30.\  } & فَأَعْرِضْ عَنْهُمْ وَٱنتَظِرْ إِنَّهُم مُّنتَظِرُونَ ﴿٣٠﴾\\
\end{longtable} \newpage

%% License: BSD style (Berkley) (i.e. Put the Copyright owner's name always)
%% Writer and Copyright (to): Bewketu(Bilal) Tadilo (2016-17)
\begin{center}\section{\LR{\textamhsec{ሱራቱ አልአህዛብ -}  \textarabic{سوره  الأحزاب}}}\end{center}
\begin{longtable}{%
  @{}
    p{.5\textwidth}
  @{~~~}
    p{.5\textwidth}
    @{}
}
\textamh{ቢስሚላሂ አራህመኒ ራሂይም } &  \mytextarabic{بِسْمِ ٱللَّهِ ٱلرَّحْمَـٰنِ ٱلرَّحِيمِ}\\
\textamh{1.\  } & \mytextarabic{ يَـٰٓأَيُّهَا ٱلنَّبِىُّ ٱتَّقِ ٱللَّهَ وَلَا تُطِعِ ٱلْكَـٰفِرِينَ وَٱلْمُنَـٰفِقِينَ ۗ إِنَّ ٱللَّهَ كَانَ عَلِيمًا حَكِيمًۭا ﴿١﴾}\\
\textamh{2.\  } & \mytextarabic{وَٱتَّبِعْ مَا يُوحَىٰٓ إِلَيْكَ مِن رَّبِّكَ ۚ إِنَّ ٱللَّهَ كَانَ بِمَا تَعْمَلُونَ خَبِيرًۭا ﴿٢﴾}\\
\textamh{3.\  } & \mytextarabic{وَتَوَكَّلْ عَلَى ٱللَّهِ ۚ وَكَفَىٰ بِٱللَّهِ وَكِيلًۭا ﴿٣﴾}\\
\textamh{4.\  } & \mytextarabic{مَّا جَعَلَ ٱللَّهُ لِرَجُلٍۢ مِّن قَلْبَيْنِ فِى جَوْفِهِۦ ۚ وَمَا جَعَلَ أَزْوَٟجَكُمُ ٱلَّٰٓـِٔى تُظَـٰهِرُونَ مِنْهُنَّ أُمَّهَـٰتِكُمْ ۚ وَمَا جَعَلَ أَدْعِيَآءَكُمْ أَبْنَآءَكُمْ ۚ ذَٟلِكُمْ قَوْلُكُم بِأَفْوَٟهِكُمْ ۖ وَٱللَّهُ يَقُولُ ٱلْحَقَّ وَهُوَ يَهْدِى ٱلسَّبِيلَ ﴿٤﴾}\\
\textamh{5.\  } & \mytextarabic{ٱدْعُوهُمْ لِءَابَآئِهِمْ هُوَ أَقْسَطُ عِندَ ٱللَّهِ ۚ فَإِن لَّمْ تَعْلَمُوٓا۟ ءَابَآءَهُمْ فَإِخْوَٟنُكُمْ فِى ٱلدِّينِ وَمَوَٟلِيكُمْ ۚ وَلَيْسَ عَلَيْكُمْ جُنَاحٌۭ فِيمَآ أَخْطَأْتُم بِهِۦ وَلَـٰكِن مَّا تَعَمَّدَتْ قُلُوبُكُمْ ۚ وَكَانَ ٱللَّهُ غَفُورًۭا رَّحِيمًا ﴿٥﴾}\\
\textamh{6.\  } & \mytextarabic{ٱلنَّبِىُّ أَوْلَىٰ بِٱلْمُؤْمِنِينَ مِنْ أَنفُسِهِمْ ۖ وَأَزْوَٟجُهُۥٓ أُمَّهَـٰتُهُمْ ۗ وَأُو۟لُوا۟ ٱلْأَرْحَامِ بَعْضُهُمْ أَوْلَىٰ بِبَعْضٍۢ فِى كِتَـٰبِ ٱللَّهِ مِنَ ٱلْمُؤْمِنِينَ وَٱلْمُهَـٰجِرِينَ إِلَّآ أَن تَفْعَلُوٓا۟ إِلَىٰٓ أَوْلِيَآئِكُم مَّعْرُوفًۭا ۚ كَانَ ذَٟلِكَ فِى ٱلْكِتَـٰبِ مَسْطُورًۭا ﴿٦﴾}\\
\textamh{7.\  } & \mytextarabic{وَإِذْ أَخَذْنَا مِنَ ٱلنَّبِيِّۦنَ مِيثَـٰقَهُمْ وَمِنكَ وَمِن نُّوحٍۢ وَإِبْرَٰهِيمَ وَمُوسَىٰ وَعِيسَى ٱبْنِ مَرْيَمَ ۖ وَأَخَذْنَا مِنْهُم مِّيثَـٰقًا غَلِيظًۭا ﴿٧﴾}\\
\textamh{8.\  } & \mytextarabic{لِّيَسْـَٔلَ ٱلصَّـٰدِقِينَ عَن صِدْقِهِمْ ۚ وَأَعَدَّ لِلْكَـٰفِرِينَ عَذَابًا أَلِيمًۭا ﴿٨﴾}\\
\textamh{9.\  } & \mytextarabic{يَـٰٓأَيُّهَا ٱلَّذِينَ ءَامَنُوا۟ ٱذْكُرُوا۟ نِعْمَةَ ٱللَّهِ عَلَيْكُمْ إِذْ جَآءَتْكُمْ جُنُودٌۭ فَأَرْسَلْنَا عَلَيْهِمْ رِيحًۭا وَجُنُودًۭا لَّمْ تَرَوْهَا ۚ وَكَانَ ٱللَّهُ بِمَا تَعْمَلُونَ بَصِيرًا ﴿٩﴾}\\
\textamh{10.\  } & \mytextarabic{إِذْ جَآءُوكُم مِّن فَوْقِكُمْ وَمِنْ أَسْفَلَ مِنكُمْ وَإِذْ زَاغَتِ ٱلْأَبْصَـٰرُ وَبَلَغَتِ ٱلْقُلُوبُ ٱلْحَنَاجِرَ وَتَظُنُّونَ بِٱللَّهِ ٱلظُّنُونَا۠ ﴿١٠﴾}\\
\textamh{11.\  } & \mytextarabic{هُنَالِكَ ٱبْتُلِىَ ٱلْمُؤْمِنُونَ وَزُلْزِلُوا۟ زِلْزَالًۭا شَدِيدًۭا ﴿١١﴾}\\
\textamh{12.\  } & \mytextarabic{وَإِذْ يَقُولُ ٱلْمُنَـٰفِقُونَ وَٱلَّذِينَ فِى قُلُوبِهِم مَّرَضٌۭ مَّا وَعَدَنَا ٱللَّهُ وَرَسُولُهُۥٓ إِلَّا غُرُورًۭا ﴿١٢﴾}\\
\textamh{13.\  } & \mytextarabic{وَإِذْ قَالَت طَّآئِفَةٌۭ مِّنْهُمْ يَـٰٓأَهْلَ يَثْرِبَ لَا مُقَامَ لَكُمْ فَٱرْجِعُوا۟ ۚ وَيَسْتَـْٔذِنُ فَرِيقٌۭ مِّنْهُمُ ٱلنَّبِىَّ يَقُولُونَ إِنَّ بُيُوتَنَا عَوْرَةٌۭ وَمَا هِىَ بِعَوْرَةٍ ۖ إِن يُرِيدُونَ إِلَّا فِرَارًۭا ﴿١٣﴾}\\
\textamh{14.\  } & \mytextarabic{وَلَوْ دُخِلَتْ عَلَيْهِم مِّنْ أَقْطَارِهَا ثُمَّ سُئِلُوا۟ ٱلْفِتْنَةَ لَءَاتَوْهَا وَمَا تَلَبَّثُوا۟ بِهَآ إِلَّا يَسِيرًۭا ﴿١٤﴾}\\
\textamh{15.\  } & \mytextarabic{وَلَقَدْ كَانُوا۟ عَـٰهَدُوا۟ ٱللَّهَ مِن قَبْلُ لَا يُوَلُّونَ ٱلْأَدْبَٰرَ ۚ وَكَانَ عَهْدُ ٱللَّهِ مَسْـُٔولًۭا ﴿١٥﴾}\\
\textamh{16.\  } & \mytextarabic{قُل لَّن يَنفَعَكُمُ ٱلْفِرَارُ إِن فَرَرْتُم مِّنَ ٱلْمَوْتِ أَوِ ٱلْقَتْلِ وَإِذًۭا لَّا تُمَتَّعُونَ إِلَّا قَلِيلًۭا ﴿١٦﴾}\\
\textamh{17.\  } & \mytextarabic{قُلْ مَن ذَا ٱلَّذِى يَعْصِمُكُم مِّنَ ٱللَّهِ إِنْ أَرَادَ بِكُمْ سُوٓءًا أَوْ أَرَادَ بِكُمْ رَحْمَةًۭ ۚ وَلَا يَجِدُونَ لَهُم مِّن دُونِ ٱللَّهِ وَلِيًّۭا وَلَا نَصِيرًۭا ﴿١٧﴾}\\
\textamh{18.\  } & \mytextarabic{۞ قَدْ يَعْلَمُ ٱللَّهُ ٱلْمُعَوِّقِينَ مِنكُمْ وَٱلْقَآئِلِينَ لِإِخْوَٟنِهِمْ هَلُمَّ إِلَيْنَا ۖ وَلَا يَأْتُونَ ٱلْبَأْسَ إِلَّا قَلِيلًا ﴿١٨﴾}\\
\textamh{19.\  } & \mytextarabic{أَشِحَّةً عَلَيْكُمْ ۖ فَإِذَا جَآءَ ٱلْخَوْفُ رَأَيْتَهُمْ يَنظُرُونَ إِلَيْكَ تَدُورُ أَعْيُنُهُمْ كَٱلَّذِى يُغْشَىٰ عَلَيْهِ مِنَ ٱلْمَوْتِ ۖ فَإِذَا ذَهَبَ ٱلْخَوْفُ سَلَقُوكُم بِأَلْسِنَةٍ حِدَادٍ أَشِحَّةً عَلَى ٱلْخَيْرِ ۚ أُو۟لَـٰٓئِكَ لَمْ يُؤْمِنُوا۟ فَأَحْبَطَ ٱللَّهُ أَعْمَـٰلَهُمْ ۚ وَكَانَ ذَٟلِكَ عَلَى ٱللَّهِ يَسِيرًۭا ﴿١٩﴾}\\
\textamh{20.\  } & \mytextarabic{يَحْسَبُونَ ٱلْأَحْزَابَ لَمْ يَذْهَبُوا۟ ۖ وَإِن يَأْتِ ٱلْأَحْزَابُ يَوَدُّوا۟ لَوْ أَنَّهُم بَادُونَ فِى ٱلْأَعْرَابِ يَسْـَٔلُونَ عَنْ أَنۢبَآئِكُمْ ۖ وَلَوْ كَانُوا۟ فِيكُم مَّا قَـٰتَلُوٓا۟ إِلَّا قَلِيلًۭا ﴿٢٠﴾}\\
\textamh{21.\  } & \mytextarabic{لَّقَدْ كَانَ لَكُمْ فِى رَسُولِ ٱللَّهِ أُسْوَةٌ حَسَنَةٌۭ لِّمَن كَانَ يَرْجُوا۟ ٱللَّهَ وَٱلْيَوْمَ ٱلْءَاخِرَ وَذَكَرَ ٱللَّهَ كَثِيرًۭا ﴿٢١﴾}\\
\textamh{22.\  } & \mytextarabic{وَلَمَّا رَءَا ٱلْمُؤْمِنُونَ ٱلْأَحْزَابَ قَالُوا۟ هَـٰذَا مَا وَعَدَنَا ٱللَّهُ وَرَسُولُهُۥ وَصَدَقَ ٱللَّهُ وَرَسُولُهُۥ ۚ وَمَا زَادَهُمْ إِلَّآ إِيمَـٰنًۭا وَتَسْلِيمًۭا ﴿٢٢﴾}\\
\textamh{23.\  } & \mytextarabic{مِّنَ ٱلْمُؤْمِنِينَ رِجَالٌۭ صَدَقُوا۟ مَا عَـٰهَدُوا۟ ٱللَّهَ عَلَيْهِ ۖ فَمِنْهُم مَّن قَضَىٰ نَحْبَهُۥ وَمِنْهُم مَّن يَنتَظِرُ ۖ وَمَا بَدَّلُوا۟ تَبْدِيلًۭا ﴿٢٣﴾}\\
\textamh{24.\  } & \mytextarabic{لِّيَجْزِىَ ٱللَّهُ ٱلصَّـٰدِقِينَ بِصِدْقِهِمْ وَيُعَذِّبَ ٱلْمُنَـٰفِقِينَ إِن شَآءَ أَوْ يَتُوبَ عَلَيْهِمْ ۚ إِنَّ ٱللَّهَ كَانَ غَفُورًۭا رَّحِيمًۭا ﴿٢٤﴾}\\
\textamh{25.\  } & \mytextarabic{وَرَدَّ ٱللَّهُ ٱلَّذِينَ كَفَرُوا۟ بِغَيْظِهِمْ لَمْ يَنَالُوا۟ خَيْرًۭا ۚ وَكَفَى ٱللَّهُ ٱلْمُؤْمِنِينَ ٱلْقِتَالَ ۚ وَكَانَ ٱللَّهُ قَوِيًّا عَزِيزًۭا ﴿٢٥﴾}\\
\textamh{26.\  } & \mytextarabic{وَأَنزَلَ ٱلَّذِينَ ظَـٰهَرُوهُم مِّنْ أَهْلِ ٱلْكِتَـٰبِ مِن صَيَاصِيهِمْ وَقَذَفَ فِى قُلُوبِهِمُ ٱلرُّعْبَ فَرِيقًۭا تَقْتُلُونَ وَتَأْسِرُونَ فَرِيقًۭا ﴿٢٦﴾}\\
\textamh{27.\  } & \mytextarabic{وَأَوْرَثَكُمْ أَرْضَهُمْ وَدِيَـٰرَهُمْ وَأَمْوَٟلَهُمْ وَأَرْضًۭا لَّمْ تَطَـُٔوهَا ۚ وَكَانَ ٱللَّهُ عَلَىٰ كُلِّ شَىْءٍۢ قَدِيرًۭا ﴿٢٧﴾}\\
\textamh{28.\  } & \mytextarabic{يَـٰٓأَيُّهَا ٱلنَّبِىُّ قُل لِّأَزْوَٟجِكَ إِن كُنتُنَّ تُرِدْنَ ٱلْحَيَوٰةَ ٱلدُّنْيَا وَزِينَتَهَا فَتَعَالَيْنَ أُمَتِّعْكُنَّ وَأُسَرِّحْكُنَّ سَرَاحًۭا جَمِيلًۭا ﴿٢٨﴾}\\
\textamh{29.\  } & \mytextarabic{وَإِن كُنتُنَّ تُرِدْنَ ٱللَّهَ وَرَسُولَهُۥ وَٱلدَّارَ ٱلْءَاخِرَةَ فَإِنَّ ٱللَّهَ أَعَدَّ لِلْمُحْسِنَـٰتِ مِنكُنَّ أَجْرًا عَظِيمًۭا ﴿٢٩﴾}\\
\textamh{30.\  } & \mytextarabic{يَـٰنِسَآءَ ٱلنَّبِىِّ مَن يَأْتِ مِنكُنَّ بِفَـٰحِشَةٍۢ مُّبَيِّنَةٍۢ يُضَٰعَفْ لَهَا ٱلْعَذَابُ ضِعْفَيْنِ ۚ وَكَانَ ذَٟلِكَ عَلَى ٱللَّهِ يَسِيرًۭا ﴿٣٠﴾}\\
\textamh{31.\  } & \mytextarabic{۞ وَمَن يَقْنُتْ مِنكُنَّ لِلَّهِ وَرَسُولِهِۦ وَتَعْمَلْ صَـٰلِحًۭا نُّؤْتِهَآ أَجْرَهَا مَرَّتَيْنِ وَأَعْتَدْنَا لَهَا رِزْقًۭا كَرِيمًۭا ﴿٣١﴾}\\
\textamh{32.\  } & \mytextarabic{يَـٰنِسَآءَ ٱلنَّبِىِّ لَسْتُنَّ كَأَحَدٍۢ مِّنَ ٱلنِّسَآءِ ۚ إِنِ ٱتَّقَيْتُنَّ فَلَا تَخْضَعْنَ بِٱلْقَوْلِ فَيَطْمَعَ ٱلَّذِى فِى قَلْبِهِۦ مَرَضٌۭ وَقُلْنَ قَوْلًۭا مَّعْرُوفًۭا ﴿٣٢﴾}\\
\textamh{33.\  } & \mytextarabic{وَقَرْنَ فِى بُيُوتِكُنَّ وَلَا تَبَرَّجْنَ تَبَرُّجَ ٱلْجَٰهِلِيَّةِ ٱلْأُولَىٰ ۖ وَأَقِمْنَ ٱلصَّلَوٰةَ وَءَاتِينَ ٱلزَّكَوٰةَ وَأَطِعْنَ ٱللَّهَ وَرَسُولَهُۥٓ ۚ إِنَّمَا يُرِيدُ ٱللَّهُ لِيُذْهِبَ عَنكُمُ ٱلرِّجْسَ أَهْلَ ٱلْبَيْتِ وَيُطَهِّرَكُمْ تَطْهِيرًۭا ﴿٣٣﴾}\\
\textamh{34.\  } & \mytextarabic{وَٱذْكُرْنَ مَا يُتْلَىٰ فِى بُيُوتِكُنَّ مِنْ ءَايَـٰتِ ٱللَّهِ وَٱلْحِكْمَةِ ۚ إِنَّ ٱللَّهَ كَانَ لَطِيفًا خَبِيرًا ﴿٣٤﴾}\\
\textamh{35.\  } & \mytextarabic{إِنَّ ٱلْمُسْلِمِينَ وَٱلْمُسْلِمَـٰتِ وَٱلْمُؤْمِنِينَ وَٱلْمُؤْمِنَـٰتِ وَٱلْقَـٰنِتِينَ وَٱلْقَـٰنِتَـٰتِ وَٱلصَّـٰدِقِينَ وَٱلصَّـٰدِقَـٰتِ وَٱلصَّـٰبِرِينَ وَٱلصَّـٰبِرَٰتِ وَٱلْخَـٰشِعِينَ وَٱلْخَـٰشِعَـٰتِ وَٱلْمُتَصَدِّقِينَ وَٱلْمُتَصَدِّقَـٰتِ وَٱلصَّـٰٓئِمِينَ وَٱلصَّـٰٓئِمَـٰتِ وَٱلْحَـٰفِظِينَ فُرُوجَهُمْ وَٱلْحَـٰفِظَـٰتِ وَٱلذَّٰكِرِينَ ٱللَّهَ كَثِيرًۭا وَٱلذَّٰكِرَٰتِ أَعَدَّ ٱللَّهُ لَهُم مَّغْفِرَةًۭ وَأَجْرًا عَظِيمًۭا ﴿٣٥﴾}\\
\textamh{36.\  } & \mytextarabic{وَمَا كَانَ لِمُؤْمِنٍۢ وَلَا مُؤْمِنَةٍ إِذَا قَضَى ٱللَّهُ وَرَسُولُهُۥٓ أَمْرًا أَن يَكُونَ لَهُمُ ٱلْخِيَرَةُ مِنْ أَمْرِهِمْ ۗ وَمَن يَعْصِ ٱللَّهَ وَرَسُولَهُۥ فَقَدْ ضَلَّ ضَلَـٰلًۭا مُّبِينًۭا ﴿٣٦﴾}\\
\textamh{37.\  } & \mytextarabic{وَإِذْ تَقُولُ لِلَّذِىٓ أَنْعَمَ ٱللَّهُ عَلَيْهِ وَأَنْعَمْتَ عَلَيْهِ أَمْسِكْ عَلَيْكَ زَوْجَكَ وَٱتَّقِ ٱللَّهَ وَتُخْفِى فِى نَفْسِكَ مَا ٱللَّهُ مُبْدِيهِ وَتَخْشَى ٱلنَّاسَ وَٱللَّهُ أَحَقُّ أَن تَخْشَىٰهُ ۖ فَلَمَّا قَضَىٰ زَيْدٌۭ مِّنْهَا وَطَرًۭا زَوَّجْنَـٰكَهَا لِكَىْ لَا يَكُونَ عَلَى ٱلْمُؤْمِنِينَ حَرَجٌۭ فِىٓ أَزْوَٟجِ أَدْعِيَآئِهِمْ إِذَا قَضَوْا۟ مِنْهُنَّ وَطَرًۭا ۚ وَكَانَ أَمْرُ ٱللَّهِ مَفْعُولًۭا ﴿٣٧﴾}\\
\textamh{38.\  } & \mytextarabic{مَّا كَانَ عَلَى ٱلنَّبِىِّ مِنْ حَرَجٍۢ فِيمَا فَرَضَ ٱللَّهُ لَهُۥ ۖ سُنَّةَ ٱللَّهِ فِى ٱلَّذِينَ خَلَوْا۟ مِن قَبْلُ ۚ وَكَانَ أَمْرُ ٱللَّهِ قَدَرًۭا مَّقْدُورًا ﴿٣٨﴾}\\
\textamh{39.\  } & \mytextarabic{ٱلَّذِينَ يُبَلِّغُونَ رِسَـٰلَـٰتِ ٱللَّهِ وَيَخْشَوْنَهُۥ وَلَا يَخْشَوْنَ أَحَدًا إِلَّا ٱللَّهَ ۗ وَكَفَىٰ بِٱللَّهِ حَسِيبًۭا ﴿٣٩﴾}\\
\textamh{40.\  } & \mytextarabic{مَّا كَانَ مُحَمَّدٌ أَبَآ أَحَدٍۢ مِّن رِّجَالِكُمْ وَلَـٰكِن رَّسُولَ ٱللَّهِ وَخَاتَمَ ٱلنَّبِيِّۦنَ ۗ وَكَانَ ٱللَّهُ بِكُلِّ شَىْءٍ عَلِيمًۭا ﴿٤٠﴾}\\
\textamh{41.\  } & \mytextarabic{يَـٰٓأَيُّهَا ٱلَّذِينَ ءَامَنُوا۟ ٱذْكُرُوا۟ ٱللَّهَ ذِكْرًۭا كَثِيرًۭا ﴿٤١﴾}\\
\textamh{42.\  } & \mytextarabic{وَسَبِّحُوهُ بُكْرَةًۭ وَأَصِيلًا ﴿٤٢﴾}\\
\textamh{43.\  } & \mytextarabic{هُوَ ٱلَّذِى يُصَلِّى عَلَيْكُمْ وَمَلَـٰٓئِكَتُهُۥ لِيُخْرِجَكُم مِّنَ ٱلظُّلُمَـٰتِ إِلَى ٱلنُّورِ ۚ وَكَانَ بِٱلْمُؤْمِنِينَ رَحِيمًۭا ﴿٤٣﴾}\\
\textamh{44.\  } & \mytextarabic{تَحِيَّتُهُمْ يَوْمَ يَلْقَوْنَهُۥ سَلَـٰمٌۭ ۚ وَأَعَدَّ لَهُمْ أَجْرًۭا كَرِيمًۭا ﴿٤٤﴾}\\
\textamh{45.\  } & \mytextarabic{يَـٰٓأَيُّهَا ٱلنَّبِىُّ إِنَّآ أَرْسَلْنَـٰكَ شَـٰهِدًۭا وَمُبَشِّرًۭا وَنَذِيرًۭا ﴿٤٥﴾}\\
\textamh{46.\  } & \mytextarabic{وَدَاعِيًا إِلَى ٱللَّهِ بِإِذْنِهِۦ وَسِرَاجًۭا مُّنِيرًۭا ﴿٤٦﴾}\\
\textamh{47.\  } & \mytextarabic{وَبَشِّرِ ٱلْمُؤْمِنِينَ بِأَنَّ لَهُم مِّنَ ٱللَّهِ فَضْلًۭا كَبِيرًۭا ﴿٤٧﴾}\\
\textamh{48.\  } & \mytextarabic{وَلَا تُطِعِ ٱلْكَـٰفِرِينَ وَٱلْمُنَـٰفِقِينَ وَدَعْ أَذَىٰهُمْ وَتَوَكَّلْ عَلَى ٱللَّهِ ۚ وَكَفَىٰ بِٱللَّهِ وَكِيلًۭا ﴿٤٨﴾}\\
\textamh{49.\  } & \mytextarabic{يَـٰٓأَيُّهَا ٱلَّذِينَ ءَامَنُوٓا۟ إِذَا نَكَحْتُمُ ٱلْمُؤْمِنَـٰتِ ثُمَّ طَلَّقْتُمُوهُنَّ مِن قَبْلِ أَن تَمَسُّوهُنَّ فَمَا لَكُمْ عَلَيْهِنَّ مِنْ عِدَّةٍۢ تَعْتَدُّونَهَا ۖ فَمَتِّعُوهُنَّ وَسَرِّحُوهُنَّ سَرَاحًۭا جَمِيلًۭا ﴿٤٩﴾}\\
\textamh{50.\  } & \mytextarabic{يَـٰٓأَيُّهَا ٱلنَّبِىُّ إِنَّآ أَحْلَلْنَا لَكَ أَزْوَٟجَكَ ٱلَّٰتِىٓ ءَاتَيْتَ أُجُورَهُنَّ وَمَا مَلَكَتْ يَمِينُكَ مِمَّآ أَفَآءَ ٱللَّهُ عَلَيْكَ وَبَنَاتِ عَمِّكَ وَبَنَاتِ عَمَّٰتِكَ وَبَنَاتِ خَالِكَ وَبَنَاتِ خَـٰلَـٰتِكَ ٱلَّٰتِى هَاجَرْنَ مَعَكَ وَٱمْرَأَةًۭ مُّؤْمِنَةً إِن وَهَبَتْ نَفْسَهَا لِلنَّبِىِّ إِنْ أَرَادَ ٱلنَّبِىُّ أَن يَسْتَنكِحَهَا خَالِصَةًۭ لَّكَ مِن دُونِ ٱلْمُؤْمِنِينَ ۗ قَدْ عَلِمْنَا مَا فَرَضْنَا عَلَيْهِمْ فِىٓ أَزْوَٟجِهِمْ وَمَا مَلَكَتْ أَيْمَـٰنُهُمْ لِكَيْلَا يَكُونَ عَلَيْكَ حَرَجٌۭ ۗ وَكَانَ ٱللَّهُ غَفُورًۭا رَّحِيمًۭا ﴿٥٠﴾}\\
\textamh{51.\  } & \mytextarabic{۞ تُرْجِى مَن تَشَآءُ مِنْهُنَّ وَتُـْٔوِىٓ إِلَيْكَ مَن تَشَآءُ ۖ وَمَنِ ٱبْتَغَيْتَ مِمَّنْ عَزَلْتَ فَلَا جُنَاحَ عَلَيْكَ ۚ ذَٟلِكَ أَدْنَىٰٓ أَن تَقَرَّ أَعْيُنُهُنَّ وَلَا يَحْزَنَّ وَيَرْضَيْنَ بِمَآ ءَاتَيْتَهُنَّ كُلُّهُنَّ ۚ وَٱللَّهُ يَعْلَمُ مَا فِى قُلُوبِكُمْ ۚ وَكَانَ ٱللَّهُ عَلِيمًا حَلِيمًۭا ﴿٥١﴾}\\
\textamh{52.\  } & \mytextarabic{لَّا يَحِلُّ لَكَ ٱلنِّسَآءُ مِنۢ بَعْدُ وَلَآ أَن تَبَدَّلَ بِهِنَّ مِنْ أَزْوَٟجٍۢ وَلَوْ أَعْجَبَكَ حُسْنُهُنَّ إِلَّا مَا مَلَكَتْ يَمِينُكَ ۗ وَكَانَ ٱللَّهُ عَلَىٰ كُلِّ شَىْءٍۢ رَّقِيبًۭا ﴿٥٢﴾}\\
\textamh{53.\  } & \mytextarabic{يَـٰٓأَيُّهَا ٱلَّذِينَ ءَامَنُوا۟ لَا تَدْخُلُوا۟ بُيُوتَ ٱلنَّبِىِّ إِلَّآ أَن يُؤْذَنَ لَكُمْ إِلَىٰ طَعَامٍ غَيْرَ نَـٰظِرِينَ إِنَىٰهُ وَلَـٰكِنْ إِذَا دُعِيتُمْ فَٱدْخُلُوا۟ فَإِذَا طَعِمْتُمْ فَٱنتَشِرُوا۟ وَلَا مُسْتَـْٔنِسِينَ لِحَدِيثٍ ۚ إِنَّ ذَٟلِكُمْ كَانَ يُؤْذِى ٱلنَّبِىَّ فَيَسْتَحْىِۦ مِنكُمْ ۖ وَٱللَّهُ لَا يَسْتَحْىِۦ مِنَ ٱلْحَقِّ ۚ وَإِذَا سَأَلْتُمُوهُنَّ مَتَـٰعًۭا فَسْـَٔلُوهُنَّ مِن وَرَآءِ حِجَابٍۢ ۚ ذَٟلِكُمْ أَطْهَرُ لِقُلُوبِكُمْ وَقُلُوبِهِنَّ ۚ وَمَا كَانَ لَكُمْ أَن تُؤْذُوا۟ رَسُولَ ٱللَّهِ وَلَآ أَن تَنكِحُوٓا۟ أَزْوَٟجَهُۥ مِنۢ بَعْدِهِۦٓ أَبَدًا ۚ إِنَّ ذَٟلِكُمْ كَانَ عِندَ ٱللَّهِ عَظِيمًا ﴿٥٣﴾}\\
\textamh{54.\  } & \mytextarabic{إِن تُبْدُوا۟ شَيْـًٔا أَوْ تُخْفُوهُ فَإِنَّ ٱللَّهَ كَانَ بِكُلِّ شَىْءٍ عَلِيمًۭا ﴿٥٤﴾}\\
\textamh{55.\  } & \mytextarabic{لَّا جُنَاحَ عَلَيْهِنَّ فِىٓ ءَابَآئِهِنَّ وَلَآ أَبْنَآئِهِنَّ وَلَآ إِخْوَٟنِهِنَّ وَلَآ أَبْنَآءِ إِخْوَٟنِهِنَّ وَلَآ أَبْنَآءِ أَخَوَٟتِهِنَّ وَلَا نِسَآئِهِنَّ وَلَا مَا مَلَكَتْ أَيْمَـٰنُهُنَّ ۗ وَٱتَّقِينَ ٱللَّهَ ۚ إِنَّ ٱللَّهَ كَانَ عَلَىٰ كُلِّ شَىْءٍۢ شَهِيدًا ﴿٥٥﴾}\\
\textamh{56.\  } & \mytextarabic{إِنَّ ٱللَّهَ وَمَلَـٰٓئِكَتَهُۥ يُصَلُّونَ عَلَى ٱلنَّبِىِّ ۚ يَـٰٓأَيُّهَا ٱلَّذِينَ ءَامَنُوا۟ صَلُّوا۟ عَلَيْهِ وَسَلِّمُوا۟ تَسْلِيمًا ﴿٥٦﴾}\\
\textamh{57.\  } & \mytextarabic{إِنَّ ٱلَّذِينَ يُؤْذُونَ ٱللَّهَ وَرَسُولَهُۥ لَعَنَهُمُ ٱللَّهُ فِى ٱلدُّنْيَا وَٱلْءَاخِرَةِ وَأَعَدَّ لَهُمْ عَذَابًۭا مُّهِينًۭا ﴿٥٧﴾}\\
\textamh{58.\  } & \mytextarabic{وَٱلَّذِينَ يُؤْذُونَ ٱلْمُؤْمِنِينَ وَٱلْمُؤْمِنَـٰتِ بِغَيْرِ مَا ٱكْتَسَبُوا۟ فَقَدِ ٱحْتَمَلُوا۟ بُهْتَـٰنًۭا وَإِثْمًۭا مُّبِينًۭا ﴿٥٨﴾}\\
\textamh{59.\  } & \mytextarabic{يَـٰٓأَيُّهَا ٱلنَّبِىُّ قُل لِّأَزْوَٟجِكَ وَبَنَاتِكَ وَنِسَآءِ ٱلْمُؤْمِنِينَ يُدْنِينَ عَلَيْهِنَّ مِن جَلَـٰبِيبِهِنَّ ۚ ذَٟلِكَ أَدْنَىٰٓ أَن يُعْرَفْنَ فَلَا يُؤْذَيْنَ ۗ وَكَانَ ٱللَّهُ غَفُورًۭا رَّحِيمًۭا ﴿٥٩﴾}\\
\textamh{60.\  } & \mytextarabic{۞ لَّئِن لَّمْ يَنتَهِ ٱلْمُنَـٰفِقُونَ وَٱلَّذِينَ فِى قُلُوبِهِم مَّرَضٌۭ وَٱلْمُرْجِفُونَ فِى ٱلْمَدِينَةِ لَنُغْرِيَنَّكَ بِهِمْ ثُمَّ لَا يُجَاوِرُونَكَ فِيهَآ إِلَّا قَلِيلًۭا ﴿٦٠﴾}\\
\textamh{61.\  } & \mytextarabic{مَّلْعُونِينَ ۖ أَيْنَمَا ثُقِفُوٓا۟ أُخِذُوا۟ وَقُتِّلُوا۟ تَقْتِيلًۭا ﴿٦١﴾}\\
\textamh{62.\  } & \mytextarabic{سُنَّةَ ٱللَّهِ فِى ٱلَّذِينَ خَلَوْا۟ مِن قَبْلُ ۖ وَلَن تَجِدَ لِسُنَّةِ ٱللَّهِ تَبْدِيلًۭا ﴿٦٢﴾}\\
\textamh{63.\  } & \mytextarabic{يَسْـَٔلُكَ ٱلنَّاسُ عَنِ ٱلسَّاعَةِ ۖ قُلْ إِنَّمَا عِلْمُهَا عِندَ ٱللَّهِ ۚ وَمَا يُدْرِيكَ لَعَلَّ ٱلسَّاعَةَ تَكُونُ قَرِيبًا ﴿٦٣﴾}\\
\textamh{64.\  } & \mytextarabic{إِنَّ ٱللَّهَ لَعَنَ ٱلْكَـٰفِرِينَ وَأَعَدَّ لَهُمْ سَعِيرًا ﴿٦٤﴾}\\
\textamh{65.\  } & \mytextarabic{خَـٰلِدِينَ فِيهَآ أَبَدًۭا ۖ لَّا يَجِدُونَ وَلِيًّۭا وَلَا نَصِيرًۭا ﴿٦٥﴾}\\
\textamh{66.\  } & \mytextarabic{يَوْمَ تُقَلَّبُ وُجُوهُهُمْ فِى ٱلنَّارِ يَقُولُونَ يَـٰلَيْتَنَآ أَطَعْنَا ٱللَّهَ وَأَطَعْنَا ٱلرَّسُولَا۠ ﴿٦٦﴾}\\
\textamh{67.\  } & \mytextarabic{وَقَالُوا۟ رَبَّنَآ إِنَّآ أَطَعْنَا سَادَتَنَا وَكُبَرَآءَنَا فَأَضَلُّونَا ٱلسَّبِيلَا۠ ﴿٦٧﴾}\\
\textamh{68.\  } & \mytextarabic{رَبَّنَآ ءَاتِهِمْ ضِعْفَيْنِ مِنَ ٱلْعَذَابِ وَٱلْعَنْهُمْ لَعْنًۭا كَبِيرًۭا ﴿٦٨﴾}\\
\textamh{69.\  } & \mytextarabic{يَـٰٓأَيُّهَا ٱلَّذِينَ ءَامَنُوا۟ لَا تَكُونُوا۟ كَٱلَّذِينَ ءَاذَوْا۟ مُوسَىٰ فَبَرَّأَهُ ٱللَّهُ مِمَّا قَالُوا۟ ۚ وَكَانَ عِندَ ٱللَّهِ وَجِيهًۭا ﴿٦٩﴾}\\
\textamh{70.\  } & \mytextarabic{يَـٰٓأَيُّهَا ٱلَّذِينَ ءَامَنُوا۟ ٱتَّقُوا۟ ٱللَّهَ وَقُولُوا۟ قَوْلًۭا سَدِيدًۭا ﴿٧٠﴾}\\
\textamh{71.\  } & \mytextarabic{يُصْلِحْ لَكُمْ أَعْمَـٰلَكُمْ وَيَغْفِرْ لَكُمْ ذُنُوبَكُمْ ۗ وَمَن يُطِعِ ٱللَّهَ وَرَسُولَهُۥ فَقَدْ فَازَ فَوْزًا عَظِيمًا ﴿٧١﴾}\\
\textamh{72.\  } & \mytextarabic{إِنَّا عَرَضْنَا ٱلْأَمَانَةَ عَلَى ٱلسَّمَـٰوَٟتِ وَٱلْأَرْضِ وَٱلْجِبَالِ فَأَبَيْنَ أَن يَحْمِلْنَهَا وَأَشْفَقْنَ مِنْهَا وَحَمَلَهَا ٱلْإِنسَـٰنُ ۖ إِنَّهُۥ كَانَ ظَلُومًۭا جَهُولًۭا ﴿٧٢﴾}\\
\textamh{73.\  } & \mytextarabic{لِّيُعَذِّبَ ٱللَّهُ ٱلْمُنَـٰفِقِينَ وَٱلْمُنَـٰفِقَـٰتِ وَٱلْمُشْرِكِينَ وَٱلْمُشْرِكَـٰتِ وَيَتُوبَ ٱللَّهُ عَلَى ٱلْمُؤْمِنِينَ وَٱلْمُؤْمِنَـٰتِ ۗ وَكَانَ ٱللَّهُ غَفُورًۭا رَّحِيمًۢا ﴿٧٣﴾}\\
\end{longtable}
\clearpage
%% License: BSD style (Berkley) (i.e. Put the Copyright owner's name always)
%% Writer and Copyright (to): Bewketu(Bilal) Tadilo (2016-17)
\begin{center}\section{\LR{\textamhsec{ሱራቱ ሳባ -}  \textarabic{سوره  سبإ}}}\end{center}
\begin{longtable}{%
  @{}
    p{.5\textwidth}
  @{~~~}
    p{.5\textwidth}
    @{}
}
\textamh{ቢስሚላሂ አራህመኒ ራሂይም } &  \mytextarabic{بِسْمِ ٱللَّهِ ٱلرَّحْمَـٰنِ ٱلرَّحِيمِ}\\
\textamh{1.\  } & \mytextarabic{ ٱلْحَمْدُ لِلَّهِ ٱلَّذِى لَهُۥ مَا فِى ٱلسَّمَـٰوَٟتِ وَمَا فِى ٱلْأَرْضِ وَلَهُ ٱلْحَمْدُ فِى ٱلْءَاخِرَةِ ۚ وَهُوَ ٱلْحَكِيمُ ٱلْخَبِيرُ ﴿١﴾}\\
\textamh{2.\  } & \mytextarabic{يَعْلَمُ مَا يَلِجُ فِى ٱلْأَرْضِ وَمَا يَخْرُجُ مِنْهَا وَمَا يَنزِلُ مِنَ ٱلسَّمَآءِ وَمَا يَعْرُجُ فِيهَا ۚ وَهُوَ ٱلرَّحِيمُ ٱلْغَفُورُ ﴿٢﴾}\\
\textamh{3.\  } & \mytextarabic{وَقَالَ ٱلَّذِينَ كَفَرُوا۟ لَا تَأْتِينَا ٱلسَّاعَةُ ۖ قُلْ بَلَىٰ وَرَبِّى لَتَأْتِيَنَّكُمْ عَـٰلِمِ ٱلْغَيْبِ ۖ لَا يَعْزُبُ عَنْهُ مِثْقَالُ ذَرَّةٍۢ فِى ٱلسَّمَـٰوَٟتِ وَلَا فِى ٱلْأَرْضِ وَلَآ أَصْغَرُ مِن ذَٟلِكَ وَلَآ أَكْبَرُ إِلَّا فِى كِتَـٰبٍۢ مُّبِينٍۢ ﴿٣﴾}\\
\textamh{4.\  } & \mytextarabic{لِّيَجْزِىَ ٱلَّذِينَ ءَامَنُوا۟ وَعَمِلُوا۟ ٱلصَّـٰلِحَـٰتِ ۚ أُو۟لَـٰٓئِكَ لَهُم مَّغْفِرَةٌۭ وَرِزْقٌۭ كَرِيمٌۭ ﴿٤﴾}\\
\textamh{5.\  } & \mytextarabic{وَٱلَّذِينَ سَعَوْ فِىٓ ءَايَـٰتِنَا مُعَـٰجِزِينَ أُو۟لَـٰٓئِكَ لَهُمْ عَذَابٌۭ مِّن رِّجْزٍ أَلِيمٌۭ ﴿٥﴾}\\
\textamh{6.\  } & \mytextarabic{وَيَرَى ٱلَّذِينَ أُوتُوا۟ ٱلْعِلْمَ ٱلَّذِىٓ أُنزِلَ إِلَيْكَ مِن رَّبِّكَ هُوَ ٱلْحَقَّ وَيَهْدِىٓ إِلَىٰ صِرَٰطِ ٱلْعَزِيزِ ٱلْحَمِيدِ ﴿٦﴾}\\
\textamh{7.\  } & \mytextarabic{وَقَالَ ٱلَّذِينَ كَفَرُوا۟ هَلْ نَدُلُّكُمْ عَلَىٰ رَجُلٍۢ يُنَبِّئُكُمْ إِذَا مُزِّقْتُمْ كُلَّ مُمَزَّقٍ إِنَّكُمْ لَفِى خَلْقٍۢ جَدِيدٍ ﴿٧﴾}\\
\textamh{8.\  } & \mytextarabic{أَفْتَرَىٰ عَلَى ٱللَّهِ كَذِبًا أَم بِهِۦ جِنَّةٌۢ ۗ بَلِ ٱلَّذِينَ لَا يُؤْمِنُونَ بِٱلْءَاخِرَةِ فِى ٱلْعَذَابِ وَٱلضَّلَـٰلِ ٱلْبَعِيدِ ﴿٨﴾}\\
\textamh{9.\  } & \mytextarabic{أَفَلَمْ يَرَوْا۟ إِلَىٰ مَا بَيْنَ أَيْدِيهِمْ وَمَا خَلْفَهُم مِّنَ ٱلسَّمَآءِ وَٱلْأَرْضِ ۚ إِن نَّشَأْ نَخْسِفْ بِهِمُ ٱلْأَرْضَ أَوْ نُسْقِطْ عَلَيْهِمْ كِسَفًۭا مِّنَ ٱلسَّمَآءِ ۚ إِنَّ فِى ذَٟلِكَ لَءَايَةًۭ لِّكُلِّ عَبْدٍۢ مُّنِيبٍۢ ﴿٩﴾}\\
\textamh{10.\  } & \mytextarabic{۞ وَلَقَدْ ءَاتَيْنَا دَاوُۥدَ مِنَّا فَضْلًۭا ۖ يَـٰجِبَالُ أَوِّبِى مَعَهُۥ وَٱلطَّيْرَ ۖ وَأَلَنَّا لَهُ ٱلْحَدِيدَ ﴿١٠﴾}\\
\textamh{11.\  } & \mytextarabic{أَنِ ٱعْمَلْ سَـٰبِغَٰتٍۢ وَقَدِّرْ فِى ٱلسَّرْدِ ۖ وَٱعْمَلُوا۟ صَـٰلِحًا ۖ إِنِّى بِمَا تَعْمَلُونَ بَصِيرٌۭ ﴿١١﴾}\\
\textamh{12.\  } & \mytextarabic{وَلِسُلَيْمَـٰنَ ٱلرِّيحَ غُدُوُّهَا شَهْرٌۭ وَرَوَاحُهَا شَهْرٌۭ ۖ وَأَسَلْنَا لَهُۥ عَيْنَ ٱلْقِطْرِ ۖ وَمِنَ ٱلْجِنِّ مَن يَعْمَلُ بَيْنَ يَدَيْهِ بِإِذْنِ رَبِّهِۦ ۖ وَمَن يَزِغْ مِنْهُمْ عَنْ أَمْرِنَا نُذِقْهُ مِنْ عَذَابِ ٱلسَّعِيرِ ﴿١٢﴾}\\
\textamh{13.\  } & \mytextarabic{يَعْمَلُونَ لَهُۥ مَا يَشَآءُ مِن مَّحَـٰرِيبَ وَتَمَـٰثِيلَ وَجِفَانٍۢ كَٱلْجَوَابِ وَقُدُورٍۢ رَّاسِيَـٰتٍ ۚ ٱعْمَلُوٓا۟ ءَالَ دَاوُۥدَ شُكْرًۭا ۚ وَقَلِيلٌۭ مِّنْ عِبَادِىَ ٱلشَّكُورُ ﴿١٣﴾}\\
\textamh{14.\  } & \mytextarabic{فَلَمَّا قَضَيْنَا عَلَيْهِ ٱلْمَوْتَ مَا دَلَّهُمْ عَلَىٰ مَوْتِهِۦٓ إِلَّا دَآبَّةُ ٱلْأَرْضِ تَأْكُلُ مِنسَأَتَهُۥ ۖ فَلَمَّا خَرَّ تَبَيَّنَتِ ٱلْجِنُّ أَن لَّوْ كَانُوا۟ يَعْلَمُونَ ٱلْغَيْبَ مَا لَبِثُوا۟ فِى ٱلْعَذَابِ ٱلْمُهِينِ ﴿١٤﴾}\\
\textamh{15.\  } & \mytextarabic{لَقَدْ كَانَ لِسَبَإٍۢ فِى مَسْكَنِهِمْ ءَايَةٌۭ ۖ جَنَّتَانِ عَن يَمِينٍۢ وَشِمَالٍۢ ۖ كُلُوا۟ مِن رِّزْقِ رَبِّكُمْ وَٱشْكُرُوا۟ لَهُۥ ۚ بَلْدَةٌۭ طَيِّبَةٌۭ وَرَبٌّ غَفُورٌۭ ﴿١٥﴾}\\
\textamh{16.\  } & \mytextarabic{فَأَعْرَضُوا۟ فَأَرْسَلْنَا عَلَيْهِمْ سَيْلَ ٱلْعَرِمِ وَبَدَّلْنَـٰهُم بِجَنَّتَيْهِمْ جَنَّتَيْنِ ذَوَاتَىْ أُكُلٍ خَمْطٍۢ وَأَثْلٍۢ وَشَىْءٍۢ مِّن سِدْرٍۢ قَلِيلٍۢ ﴿١٦﴾}\\
\textamh{17.\  } & \mytextarabic{ذَٟلِكَ جَزَيْنَـٰهُم بِمَا كَفَرُوا۟ ۖ وَهَلْ نُجَٰزِىٓ إِلَّا ٱلْكَفُورَ ﴿١٧﴾}\\
\textamh{18.\  } & \mytextarabic{وَجَعَلْنَا بَيْنَهُمْ وَبَيْنَ ٱلْقُرَى ٱلَّتِى بَٰرَكْنَا فِيهَا قُرًۭى ظَـٰهِرَةًۭ وَقَدَّرْنَا فِيهَا ٱلسَّيْرَ ۖ سِيرُوا۟ فِيهَا لَيَالِىَ وَأَيَّامًا ءَامِنِينَ ﴿١٨﴾}\\
\textamh{19.\  } & \mytextarabic{فَقَالُوا۟ رَبَّنَا بَٰعِدْ بَيْنَ أَسْفَارِنَا وَظَلَمُوٓا۟ أَنفُسَهُمْ فَجَعَلْنَـٰهُمْ أَحَادِيثَ وَمَزَّقْنَـٰهُمْ كُلَّ مُمَزَّقٍ ۚ إِنَّ فِى ذَٟلِكَ لَءَايَـٰتٍۢ لِّكُلِّ صَبَّارٍۢ شَكُورٍۢ ﴿١٩﴾}\\
\textamh{20.\  } & \mytextarabic{وَلَقَدْ صَدَّقَ عَلَيْهِمْ إِبْلِيسُ ظَنَّهُۥ فَٱتَّبَعُوهُ إِلَّا فَرِيقًۭا مِّنَ ٱلْمُؤْمِنِينَ ﴿٢٠﴾}\\
\textamh{21.\  } & \mytextarabic{وَمَا كَانَ لَهُۥ عَلَيْهِم مِّن سُلْطَٰنٍ إِلَّا لِنَعْلَمَ مَن يُؤْمِنُ بِٱلْءَاخِرَةِ مِمَّنْ هُوَ مِنْهَا فِى شَكٍّۢ ۗ وَرَبُّكَ عَلَىٰ كُلِّ شَىْءٍ حَفِيظٌۭ ﴿٢١﴾}\\
\textamh{22.\  } & \mytextarabic{قُلِ ٱدْعُوا۟ ٱلَّذِينَ زَعَمْتُم مِّن دُونِ ٱللَّهِ ۖ لَا يَمْلِكُونَ مِثْقَالَ ذَرَّةٍۢ فِى ٱلسَّمَـٰوَٟتِ وَلَا فِى ٱلْأَرْضِ وَمَا لَهُمْ فِيهِمَا مِن شِرْكٍۢ وَمَا لَهُۥ مِنْهُم مِّن ظَهِيرٍۢ ﴿٢٢﴾}\\
\textamh{23.\  } & \mytextarabic{وَلَا تَنفَعُ ٱلشَّفَـٰعَةُ عِندَهُۥٓ إِلَّا لِمَنْ أَذِنَ لَهُۥ ۚ حَتَّىٰٓ إِذَا فُزِّعَ عَن قُلُوبِهِمْ قَالُوا۟ مَاذَا قَالَ رَبُّكُمْ ۖ قَالُوا۟ ٱلْحَقَّ ۖ وَهُوَ ٱلْعَلِىُّ ٱلْكَبِيرُ ﴿٢٣﴾}\\
\textamh{24.\  } & \mytextarabic{۞ قُلْ مَن يَرْزُقُكُم مِّنَ ٱلسَّمَـٰوَٟتِ وَٱلْأَرْضِ ۖ قُلِ ٱللَّهُ ۖ وَإِنَّآ أَوْ إِيَّاكُمْ لَعَلَىٰ هُدًى أَوْ فِى ضَلَـٰلٍۢ مُّبِينٍۢ ﴿٢٤﴾}\\
\textamh{25.\  } & \mytextarabic{قُل لَّا تُسْـَٔلُونَ عَمَّآ أَجْرَمْنَا وَلَا نُسْـَٔلُ عَمَّا تَعْمَلُونَ ﴿٢٥﴾}\\
\textamh{26.\  } & \mytextarabic{قُلْ يَجْمَعُ بَيْنَنَا رَبُّنَا ثُمَّ يَفْتَحُ بَيْنَنَا بِٱلْحَقِّ وَهُوَ ٱلْفَتَّاحُ ٱلْعَلِيمُ ﴿٢٦﴾}\\
\textamh{27.\  } & \mytextarabic{قُلْ أَرُونِىَ ٱلَّذِينَ أَلْحَقْتُم بِهِۦ شُرَكَآءَ ۖ كَلَّا ۚ بَلْ هُوَ ٱللَّهُ ٱلْعَزِيزُ ٱلْحَكِيمُ ﴿٢٧﴾}\\
\textamh{28.\  } & \mytextarabic{وَمَآ أَرْسَلْنَـٰكَ إِلَّا كَآفَّةًۭ لِّلنَّاسِ بَشِيرًۭا وَنَذِيرًۭا وَلَـٰكِنَّ أَكْثَرَ ٱلنَّاسِ لَا يَعْلَمُونَ ﴿٢٨﴾}\\
\textamh{29.\  } & \mytextarabic{وَيَقُولُونَ مَتَىٰ هَـٰذَا ٱلْوَعْدُ إِن كُنتُمْ صَـٰدِقِينَ ﴿٢٩﴾}\\
\textamh{30.\  } & \mytextarabic{قُل لَّكُم مِّيعَادُ يَوْمٍۢ لَّا تَسْتَـْٔخِرُونَ عَنْهُ سَاعَةًۭ وَلَا تَسْتَقْدِمُونَ ﴿٣٠﴾}\\
\textamh{31.\  } & \mytextarabic{وَقَالَ ٱلَّذِينَ كَفَرُوا۟ لَن نُّؤْمِنَ بِهَـٰذَا ٱلْقُرْءَانِ وَلَا بِٱلَّذِى بَيْنَ يَدَيْهِ ۗ وَلَوْ تَرَىٰٓ إِذِ ٱلظَّـٰلِمُونَ مَوْقُوفُونَ عِندَ رَبِّهِمْ يَرْجِعُ بَعْضُهُمْ إِلَىٰ بَعْضٍ ٱلْقَوْلَ يَقُولُ ٱلَّذِينَ ٱسْتُضْعِفُوا۟ لِلَّذِينَ ٱسْتَكْبَرُوا۟ لَوْلَآ أَنتُمْ لَكُنَّا مُؤْمِنِينَ ﴿٣١﴾}\\
\textamh{32.\  } & \mytextarabic{قَالَ ٱلَّذِينَ ٱسْتَكْبَرُوا۟ لِلَّذِينَ ٱسْتُضْعِفُوٓا۟ أَنَحْنُ صَدَدْنَـٰكُمْ عَنِ ٱلْهُدَىٰ بَعْدَ إِذْ جَآءَكُم ۖ بَلْ كُنتُم مُّجْرِمِينَ ﴿٣٢﴾}\\
\textamh{33.\  } & \mytextarabic{وَقَالَ ٱلَّذِينَ ٱسْتُضْعِفُوا۟ لِلَّذِينَ ٱسْتَكْبَرُوا۟ بَلْ مَكْرُ ٱلَّيْلِ وَٱلنَّهَارِ إِذْ تَأْمُرُونَنَآ أَن نَّكْفُرَ بِٱللَّهِ وَنَجْعَلَ لَهُۥٓ أَندَادًۭا ۚ وَأَسَرُّوا۟ ٱلنَّدَامَةَ لَمَّا رَأَوُا۟ ٱلْعَذَابَ وَجَعَلْنَا ٱلْأَغْلَـٰلَ فِىٓ أَعْنَاقِ ٱلَّذِينَ كَفَرُوا۟ ۚ هَلْ يُجْزَوْنَ إِلَّا مَا كَانُوا۟ يَعْمَلُونَ ﴿٣٣﴾}\\
\textamh{34.\  } & \mytextarabic{وَمَآ أَرْسَلْنَا فِى قَرْيَةٍۢ مِّن نَّذِيرٍ إِلَّا قَالَ مُتْرَفُوهَآ إِنَّا بِمَآ أُرْسِلْتُم بِهِۦ كَـٰفِرُونَ ﴿٣٤﴾}\\
\textamh{35.\  } & \mytextarabic{وَقَالُوا۟ نَحْنُ أَكْثَرُ أَمْوَٟلًۭا وَأَوْلَـٰدًۭا وَمَا نَحْنُ بِمُعَذَّبِينَ ﴿٣٥﴾}\\
\textamh{36.\  } & \mytextarabic{قُلْ إِنَّ رَبِّى يَبْسُطُ ٱلرِّزْقَ لِمَن يَشَآءُ وَيَقْدِرُ وَلَـٰكِنَّ أَكْثَرَ ٱلنَّاسِ لَا يَعْلَمُونَ ﴿٣٦﴾}\\
\textamh{37.\  } & \mytextarabic{وَمَآ أَمْوَٟلُكُمْ وَلَآ أَوْلَـٰدُكُم بِٱلَّتِى تُقَرِّبُكُمْ عِندَنَا زُلْفَىٰٓ إِلَّا مَنْ ءَامَنَ وَعَمِلَ صَـٰلِحًۭا فَأُو۟لَـٰٓئِكَ لَهُمْ جَزَآءُ ٱلضِّعْفِ بِمَا عَمِلُوا۟ وَهُمْ فِى ٱلْغُرُفَـٰتِ ءَامِنُونَ ﴿٣٧﴾}\\
\textamh{38.\  } & \mytextarabic{وَٱلَّذِينَ يَسْعَوْنَ فِىٓ ءَايَـٰتِنَا مُعَـٰجِزِينَ أُو۟لَـٰٓئِكَ فِى ٱلْعَذَابِ مُحْضَرُونَ ﴿٣٨﴾}\\
\textamh{39.\  } & \mytextarabic{قُلْ إِنَّ رَبِّى يَبْسُطُ ٱلرِّزْقَ لِمَن يَشَآءُ مِنْ عِبَادِهِۦ وَيَقْدِرُ لَهُۥ ۚ وَمَآ أَنفَقْتُم مِّن شَىْءٍۢ فَهُوَ يُخْلِفُهُۥ ۖ وَهُوَ خَيْرُ ٱلرَّٟزِقِينَ ﴿٣٩﴾}\\
\textamh{40.\  } & \mytextarabic{وَيَوْمَ يَحْشُرُهُمْ جَمِيعًۭا ثُمَّ يَقُولُ لِلْمَلَـٰٓئِكَةِ أَهَـٰٓؤُلَآءِ إِيَّاكُمْ كَانُوا۟ يَعْبُدُونَ ﴿٤٠﴾}\\
\textamh{41.\  } & \mytextarabic{قَالُوا۟ سُبْحَـٰنَكَ أَنتَ وَلِيُّنَا مِن دُونِهِم ۖ بَلْ كَانُوا۟ يَعْبُدُونَ ٱلْجِنَّ ۖ أَكْثَرُهُم بِهِم مُّؤْمِنُونَ ﴿٤١﴾}\\
\textamh{42.\  } & \mytextarabic{فَٱلْيَوْمَ لَا يَمْلِكُ بَعْضُكُمْ لِبَعْضٍۢ نَّفْعًۭا وَلَا ضَرًّۭا وَنَقُولُ لِلَّذِينَ ظَلَمُوا۟ ذُوقُوا۟ عَذَابَ ٱلنَّارِ ٱلَّتِى كُنتُم بِهَا تُكَذِّبُونَ ﴿٤٢﴾}\\
\textamh{43.\  } & \mytextarabic{وَإِذَا تُتْلَىٰ عَلَيْهِمْ ءَايَـٰتُنَا بَيِّنَـٰتٍۢ قَالُوا۟ مَا هَـٰذَآ إِلَّا رَجُلٌۭ يُرِيدُ أَن يَصُدَّكُمْ عَمَّا كَانَ يَعْبُدُ ءَابَآؤُكُمْ وَقَالُوا۟ مَا هَـٰذَآ إِلَّآ إِفْكٌۭ مُّفْتَرًۭى ۚ وَقَالَ ٱلَّذِينَ كَفَرُوا۟ لِلْحَقِّ لَمَّا جَآءَهُمْ إِنْ هَـٰذَآ إِلَّا سِحْرٌۭ مُّبِينٌۭ ﴿٤٣﴾}\\
\textamh{44.\  } & \mytextarabic{وَمَآ ءَاتَيْنَـٰهُم مِّن كُتُبٍۢ يَدْرُسُونَهَا ۖ وَمَآ أَرْسَلْنَآ إِلَيْهِمْ قَبْلَكَ مِن نَّذِيرٍۢ ﴿٤٤﴾}\\
\textamh{45.\  } & \mytextarabic{وَكَذَّبَ ٱلَّذِينَ مِن قَبْلِهِمْ وَمَا بَلَغُوا۟ مِعْشَارَ مَآ ءَاتَيْنَـٰهُمْ فَكَذَّبُوا۟ رُسُلِى ۖ فَكَيْفَ كَانَ نَكِيرِ ﴿٤٥﴾}\\
\textamh{46.\  } & \mytextarabic{۞ قُلْ إِنَّمَآ أَعِظُكُم بِوَٟحِدَةٍ ۖ أَن تَقُومُوا۟ لِلَّهِ مَثْنَىٰ وَفُرَٰدَىٰ ثُمَّ تَتَفَكَّرُوا۟ ۚ مَا بِصَاحِبِكُم مِّن جِنَّةٍ ۚ إِنْ هُوَ إِلَّا نَذِيرٌۭ لَّكُم بَيْنَ يَدَىْ عَذَابٍۢ شَدِيدٍۢ ﴿٤٦﴾}\\
\textamh{47.\  } & \mytextarabic{قُلْ مَا سَأَلْتُكُم مِّنْ أَجْرٍۢ فَهُوَ لَكُمْ ۖ إِنْ أَجْرِىَ إِلَّا عَلَى ٱللَّهِ ۖ وَهُوَ عَلَىٰ كُلِّ شَىْءٍۢ شَهِيدٌۭ ﴿٤٧﴾}\\
\textamh{48.\  } & \mytextarabic{قُلْ إِنَّ رَبِّى يَقْذِفُ بِٱلْحَقِّ عَلَّٰمُ ٱلْغُيُوبِ ﴿٤٨﴾}\\
\textamh{49.\  } & \mytextarabic{قُلْ جَآءَ ٱلْحَقُّ وَمَا يُبْدِئُ ٱلْبَٰطِلُ وَمَا يُعِيدُ ﴿٤٩﴾}\\
\textamh{50.\  } & \mytextarabic{قُلْ إِن ضَلَلْتُ فَإِنَّمَآ أَضِلُّ عَلَىٰ نَفْسِى ۖ وَإِنِ ٱهْتَدَيْتُ فَبِمَا يُوحِىٓ إِلَىَّ رَبِّىٓ ۚ إِنَّهُۥ سَمِيعٌۭ قَرِيبٌۭ ﴿٥٠﴾}\\
\textamh{51.\  } & \mytextarabic{وَلَوْ تَرَىٰٓ إِذْ فَزِعُوا۟ فَلَا فَوْتَ وَأُخِذُوا۟ مِن مَّكَانٍۢ قَرِيبٍۢ ﴿٥١﴾}\\
\textamh{52.\  } & \mytextarabic{وَقَالُوٓا۟ ءَامَنَّا بِهِۦ وَأَنَّىٰ لَهُمُ ٱلتَّنَاوُشُ مِن مَّكَانٍۭ بَعِيدٍۢ ﴿٥٢﴾}\\
\textamh{53.\  } & \mytextarabic{وَقَدْ كَفَرُوا۟ بِهِۦ مِن قَبْلُ ۖ وَيَقْذِفُونَ بِٱلْغَيْبِ مِن مَّكَانٍۭ بَعِيدٍۢ ﴿٥٣﴾}\\
\textamh{54.\  } & \mytextarabic{وَحِيلَ بَيْنَهُمْ وَبَيْنَ مَا يَشْتَهُونَ كَمَا فُعِلَ بِأَشْيَاعِهِم مِّن قَبْلُ ۚ إِنَّهُمْ كَانُوا۟ فِى شَكٍّۢ مُّرِيبٍۭ ﴿٥٤﴾}\\
\end{longtable}
\clearpage
%% License: BSD style (Berkley) (i.e. Put the Copyright owner's name always)
%% Writer and Copyright (to): Bewketu(Bilal) Tadilo (2016-17)
\centering\section{\LR{\textamharic{ሱራቱ ፋጢር -}  \RL{سوره  فاطر}}}
\begin{longtable}{%
  @{}
    p{.5\textwidth}
  @{~~~~~~~~~~~~}
    p{.5\textwidth}
    @{}
}
\nopagebreak
\textamh{ቢስሚላሂ አራህመኒ ራሂይም } &  بِسْمِ ٱللَّهِ ٱلرَّحْمَـٰنِ ٱلرَّحِيمِ\\
\textamh{1.\  } &  ٱلْحَمْدُ لِلَّهِ فَاطِرِ ٱلسَّمَـٰوَٟتِ وَٱلْأَرْضِ جَاعِلِ ٱلْمَلَـٰٓئِكَةِ رُسُلًا أُو۟لِىٓ أَجْنِحَةٍۢ مَّثْنَىٰ وَثُلَـٰثَ وَرُبَٰعَ ۚ يَزِيدُ فِى ٱلْخَلْقِ مَا يَشَآءُ ۚ إِنَّ ٱللَّهَ عَلَىٰ كُلِّ شَىْءٍۢ قَدِيرٌۭ ﴿١﴾\\
\textamh{2.\  } & مَّا يَفْتَحِ ٱللَّهُ لِلنَّاسِ مِن رَّحْمَةٍۢ فَلَا مُمْسِكَ لَهَا ۖ وَمَا يُمْسِكْ فَلَا مُرْسِلَ لَهُۥ مِنۢ بَعْدِهِۦ ۚ وَهُوَ ٱلْعَزِيزُ ٱلْحَكِيمُ ﴿٢﴾\\
\textamh{3.\  } & يَـٰٓأَيُّهَا ٱلنَّاسُ ٱذْكُرُوا۟ نِعْمَتَ ٱللَّهِ عَلَيْكُمْ ۚ هَلْ مِنْ خَـٰلِقٍ غَيْرُ ٱللَّهِ يَرْزُقُكُم مِّنَ ٱلسَّمَآءِ وَٱلْأَرْضِ ۚ لَآ إِلَـٰهَ إِلَّا هُوَ ۖ فَأَنَّىٰ تُؤْفَكُونَ ﴿٣﴾\\
\textamh{4.\  } & وَإِن يُكَذِّبُوكَ فَقَدْ كُذِّبَتْ رُسُلٌۭ مِّن قَبْلِكَ ۚ وَإِلَى ٱللَّهِ تُرْجَعُ ٱلْأُمُورُ ﴿٤﴾\\
\textamh{5.\  } & يَـٰٓأَيُّهَا ٱلنَّاسُ إِنَّ وَعْدَ ٱللَّهِ حَقٌّۭ ۖ فَلَا تَغُرَّنَّكُمُ ٱلْحَيَوٰةُ ٱلدُّنْيَا ۖ وَلَا يَغُرَّنَّكُم بِٱللَّهِ ٱلْغَرُورُ ﴿٥﴾\\
\textamh{6.\  } & إِنَّ ٱلشَّيْطَٰنَ لَكُمْ عَدُوٌّۭ فَٱتَّخِذُوهُ عَدُوًّا ۚ إِنَّمَا يَدْعُوا۟ حِزْبَهُۥ لِيَكُونُوا۟ مِنْ أَصْحَـٰبِ ٱلسَّعِيرِ ﴿٦﴾\\
\textamh{7.\  } & ٱلَّذِينَ كَفَرُوا۟ لَهُمْ عَذَابٌۭ شَدِيدٌۭ ۖ وَٱلَّذِينَ ءَامَنُوا۟ وَعَمِلُوا۟ ٱلصَّـٰلِحَـٰتِ لَهُم مَّغْفِرَةٌۭ وَأَجْرٌۭ كَبِيرٌ ﴿٧﴾\\
\textamh{8.\  } & أَفَمَن زُيِّنَ لَهُۥ سُوٓءُ عَمَلِهِۦ فَرَءَاهُ حَسَنًۭا ۖ فَإِنَّ ٱللَّهَ يُضِلُّ مَن يَشَآءُ وَيَهْدِى مَن يَشَآءُ ۖ فَلَا تَذْهَبْ نَفْسُكَ عَلَيْهِمْ حَسَرَٰتٍ ۚ إِنَّ ٱللَّهَ عَلِيمٌۢ بِمَا يَصْنَعُونَ ﴿٨﴾\\
\textamh{9.\  } & وَٱللَّهُ ٱلَّذِىٓ أَرْسَلَ ٱلرِّيَـٰحَ فَتُثِيرُ سَحَابًۭا فَسُقْنَـٰهُ إِلَىٰ بَلَدٍۢ مَّيِّتٍۢ فَأَحْيَيْنَا بِهِ ٱلْأَرْضَ بَعْدَ مَوْتِهَا ۚ كَذَٟلِكَ ٱلنُّشُورُ ﴿٩﴾\\
\textamh{10.\  } & مَن كَانَ يُرِيدُ ٱلْعِزَّةَ فَلِلَّهِ ٱلْعِزَّةُ جَمِيعًا ۚ إِلَيْهِ يَصْعَدُ ٱلْكَلِمُ ٱلطَّيِّبُ وَٱلْعَمَلُ ٱلصَّـٰلِحُ يَرْفَعُهُۥ ۚ وَٱلَّذِينَ يَمْكُرُونَ ٱلسَّيِّـَٔاتِ لَهُمْ عَذَابٌۭ شَدِيدٌۭ ۖ وَمَكْرُ أُو۟لَـٰٓئِكَ هُوَ يَبُورُ ﴿١٠﴾\\
\textamh{11.\  } & وَٱللَّهُ خَلَقَكُم مِّن تُرَابٍۢ ثُمَّ مِن نُّطْفَةٍۢ ثُمَّ جَعَلَكُمْ أَزْوَٟجًۭا ۚ وَمَا تَحْمِلُ مِنْ أُنثَىٰ وَلَا تَضَعُ إِلَّا بِعِلْمِهِۦ ۚ وَمَا يُعَمَّرُ مِن مُّعَمَّرٍۢ وَلَا يُنقَصُ مِنْ عُمُرِهِۦٓ إِلَّا فِى كِتَـٰبٍ ۚ إِنَّ ذَٟلِكَ عَلَى ٱللَّهِ يَسِيرٌۭ ﴿١١﴾\\
\textamh{12.\  } & وَمَا يَسْتَوِى ٱلْبَحْرَانِ هَـٰذَا عَذْبٌۭ فُرَاتٌۭ سَآئِغٌۭ شَرَابُهُۥ وَهَـٰذَا مِلْحٌ أُجَاجٌۭ ۖ وَمِن كُلٍّۢ تَأْكُلُونَ لَحْمًۭا طَرِيًّۭا وَتَسْتَخْرِجُونَ حِلْيَةًۭ تَلْبَسُونَهَا ۖ وَتَرَى ٱلْفُلْكَ فِيهِ مَوَاخِرَ لِتَبْتَغُوا۟ مِن فَضْلِهِۦ وَلَعَلَّكُمْ تَشْكُرُونَ ﴿١٢﴾\\
\textamh{13.\  } & يُولِجُ ٱلَّيْلَ فِى ٱلنَّهَارِ وَيُولِجُ ٱلنَّهَارَ فِى ٱلَّيْلِ وَسَخَّرَ ٱلشَّمْسَ وَٱلْقَمَرَ كُلٌّۭ يَجْرِى لِأَجَلٍۢ مُّسَمًّۭى ۚ ذَٟلِكُمُ ٱللَّهُ رَبُّكُمْ لَهُ ٱلْمُلْكُ ۚ وَٱلَّذِينَ تَدْعُونَ مِن دُونِهِۦ مَا يَمْلِكُونَ مِن قِطْمِيرٍ ﴿١٣﴾\\
\textamh{14.\  } & إِن تَدْعُوهُمْ لَا يَسْمَعُوا۟ دُعَآءَكُمْ وَلَوْ سَمِعُوا۟ مَا ٱسْتَجَابُوا۟ لَكُمْ ۖ وَيَوْمَ ٱلْقِيَـٰمَةِ يَكْفُرُونَ بِشِرْكِكُمْ ۚ وَلَا يُنَبِّئُكَ مِثْلُ خَبِيرٍۢ ﴿١٤﴾\\
\textamh{15.\  } & ۞ يَـٰٓأَيُّهَا ٱلنَّاسُ أَنتُمُ ٱلْفُقَرَآءُ إِلَى ٱللَّهِ ۖ وَٱللَّهُ هُوَ ٱلْغَنِىُّ ٱلْحَمِيدُ ﴿١٥﴾\\
\textamh{16.\  } & إِن يَشَأْ يُذْهِبْكُمْ وَيَأْتِ بِخَلْقٍۢ جَدِيدٍۢ ﴿١٦﴾\\
\textamh{17.\  } & وَمَا ذَٟلِكَ عَلَى ٱللَّهِ بِعَزِيزٍۢ ﴿١٧﴾\\
\textamh{18.\  } & وَلَا تَزِرُ وَازِرَةٌۭ وِزْرَ أُخْرَىٰ ۚ وَإِن تَدْعُ مُثْقَلَةٌ إِلَىٰ حِمْلِهَا لَا يُحْمَلْ مِنْهُ شَىْءٌۭ وَلَوْ كَانَ ذَا قُرْبَىٰٓ ۗ إِنَّمَا تُنذِرُ ٱلَّذِينَ يَخْشَوْنَ رَبَّهُم بِٱلْغَيْبِ وَأَقَامُوا۟ ٱلصَّلَوٰةَ ۚ وَمَن تَزَكَّىٰ فَإِنَّمَا يَتَزَكَّىٰ لِنَفْسِهِۦ ۚ وَإِلَى ٱللَّهِ ٱلْمَصِيرُ ﴿١٨﴾\\
\textamh{19.\  } & وَمَا يَسْتَوِى ٱلْأَعْمَىٰ وَٱلْبَصِيرُ ﴿١٩﴾\\
\textamh{20.\  } & وَلَا ٱلظُّلُمَـٰتُ وَلَا ٱلنُّورُ ﴿٢٠﴾\\
\textamh{21.\  } & وَلَا ٱلظِّلُّ وَلَا ٱلْحَرُورُ ﴿٢١﴾\\
\textamh{22.\  } & وَمَا يَسْتَوِى ٱلْأَحْيَآءُ وَلَا ٱلْأَمْوَٟتُ ۚ إِنَّ ٱللَّهَ يُسْمِعُ مَن يَشَآءُ ۖ وَمَآ أَنتَ بِمُسْمِعٍۢ مَّن فِى ٱلْقُبُورِ ﴿٢٢﴾\\
\textamh{23.\  } & إِنْ أَنتَ إِلَّا نَذِيرٌ ﴿٢٣﴾\\
\textamh{24.\  } & إِنَّآ أَرْسَلْنَـٰكَ بِٱلْحَقِّ بَشِيرًۭا وَنَذِيرًۭا ۚ وَإِن مِّنْ أُمَّةٍ إِلَّا خَلَا فِيهَا نَذِيرٌۭ ﴿٢٤﴾\\
\textamh{25.\  } & وَإِن يُكَذِّبُوكَ فَقَدْ كَذَّبَ ٱلَّذِينَ مِن قَبْلِهِمْ جَآءَتْهُمْ رُسُلُهُم بِٱلْبَيِّنَـٰتِ وَبِٱلزُّبُرِ وَبِٱلْكِتَـٰبِ ٱلْمُنِيرِ ﴿٢٥﴾\\
\textamh{26.\  } & ثُمَّ أَخَذْتُ ٱلَّذِينَ كَفَرُوا۟ ۖ فَكَيْفَ كَانَ نَكِيرِ ﴿٢٦﴾\\
\textamh{27.\  } & أَلَمْ تَرَ أَنَّ ٱللَّهَ أَنزَلَ مِنَ ٱلسَّمَآءِ مَآءًۭ فَأَخْرَجْنَا بِهِۦ ثَمَرَٰتٍۢ مُّخْتَلِفًا أَلْوَٟنُهَا ۚ وَمِنَ ٱلْجِبَالِ جُدَدٌۢ بِيضٌۭ وَحُمْرٌۭ مُّخْتَلِفٌ أَلْوَٟنُهَا وَغَرَابِيبُ سُودٌۭ ﴿٢٧﴾\\
\textamh{28.\  } & وَمِنَ ٱلنَّاسِ وَٱلدَّوَآبِّ وَٱلْأَنْعَـٰمِ مُخْتَلِفٌ أَلْوَٟنُهُۥ كَذَٟلِكَ ۗ إِنَّمَا يَخْشَى ٱللَّهَ مِنْ عِبَادِهِ ٱلْعُلَمَـٰٓؤُا۟ ۗ إِنَّ ٱللَّهَ عَزِيزٌ غَفُورٌ ﴿٢٨﴾\\
\textamh{29.\  } & إِنَّ ٱلَّذِينَ يَتْلُونَ كِتَـٰبَ ٱللَّهِ وَأَقَامُوا۟ ٱلصَّلَوٰةَ وَأَنفَقُوا۟ مِمَّا رَزَقْنَـٰهُمْ سِرًّۭا وَعَلَانِيَةًۭ يَرْجُونَ تِجَٰرَةًۭ لَّن تَبُورَ ﴿٢٩﴾\\
\textamh{30.\  } & لِيُوَفِّيَهُمْ أُجُورَهُمْ وَيَزِيدَهُم مِّن فَضْلِهِۦٓ ۚ إِنَّهُۥ غَفُورٌۭ شَكُورٌۭ ﴿٣٠﴾\\
\textamh{31.\  } & وَٱلَّذِىٓ أَوْحَيْنَآ إِلَيْكَ مِنَ ٱلْكِتَـٰبِ هُوَ ٱلْحَقُّ مُصَدِّقًۭا لِّمَا بَيْنَ يَدَيْهِ ۗ إِنَّ ٱللَّهَ بِعِبَادِهِۦ لَخَبِيرٌۢ بَصِيرٌۭ ﴿٣١﴾\\
\textamh{32.\  } & ثُمَّ أَوْرَثْنَا ٱلْكِتَـٰبَ ٱلَّذِينَ ٱصْطَفَيْنَا مِنْ عِبَادِنَا ۖ فَمِنْهُمْ ظَالِمٌۭ لِّنَفْسِهِۦ وَمِنْهُم مُّقْتَصِدٌۭ وَمِنْهُمْ سَابِقٌۢ بِٱلْخَيْرَٰتِ بِإِذْنِ ٱللَّهِ ۚ ذَٟلِكَ هُوَ ٱلْفَضْلُ ٱلْكَبِيرُ ﴿٣٢﴾\\
\textamh{33.\  } & جَنَّـٰتُ عَدْنٍۢ يَدْخُلُونَهَا يُحَلَّوْنَ فِيهَا مِنْ أَسَاوِرَ مِن ذَهَبٍۢ وَلُؤْلُؤًۭا ۖ وَلِبَاسُهُمْ فِيهَا حَرِيرٌۭ ﴿٣٣﴾\\
\textamh{34.\  } & وَقَالُوا۟ ٱلْحَمْدُ لِلَّهِ ٱلَّذِىٓ أَذْهَبَ عَنَّا ٱلْحَزَنَ ۖ إِنَّ رَبَّنَا لَغَفُورٌۭ شَكُورٌ ﴿٣٤﴾\\
\textamh{35.\  } & ٱلَّذِىٓ أَحَلَّنَا دَارَ ٱلْمُقَامَةِ مِن فَضْلِهِۦ لَا يَمَسُّنَا فِيهَا نَصَبٌۭ وَلَا يَمَسُّنَا فِيهَا لُغُوبٌۭ ﴿٣٥﴾\\
\textamh{36.\  } & وَٱلَّذِينَ كَفَرُوا۟ لَهُمْ نَارُ جَهَنَّمَ لَا يُقْضَىٰ عَلَيْهِمْ فَيَمُوتُوا۟ وَلَا يُخَفَّفُ عَنْهُم مِّنْ عَذَابِهَا ۚ كَذَٟلِكَ نَجْزِى كُلَّ كَفُورٍۢ ﴿٣٦﴾\\
\textamh{37.\  } & وَهُمْ يَصْطَرِخُونَ فِيهَا رَبَّنَآ أَخْرِجْنَا نَعْمَلْ صَـٰلِحًا غَيْرَ ٱلَّذِى كُنَّا نَعْمَلُ ۚ أَوَلَمْ نُعَمِّرْكُم مَّا يَتَذَكَّرُ فِيهِ مَن تَذَكَّرَ وَجَآءَكُمُ ٱلنَّذِيرُ ۖ فَذُوقُوا۟ فَمَا لِلظَّـٰلِمِينَ مِن نَّصِيرٍ ﴿٣٧﴾\\
\textamh{38.\  } & إِنَّ ٱللَّهَ عَـٰلِمُ غَيْبِ ٱلسَّمَـٰوَٟتِ وَٱلْأَرْضِ ۚ إِنَّهُۥ عَلِيمٌۢ بِذَاتِ ٱلصُّدُورِ ﴿٣٨﴾\\
\textamh{39.\  } & هُوَ ٱلَّذِى جَعَلَكُمْ خَلَـٰٓئِفَ فِى ٱلْأَرْضِ ۚ فَمَن كَفَرَ فَعَلَيْهِ كُفْرُهُۥ ۖ وَلَا يَزِيدُ ٱلْكَـٰفِرِينَ كُفْرُهُمْ عِندَ رَبِّهِمْ إِلَّا مَقْتًۭا ۖ وَلَا يَزِيدُ ٱلْكَـٰفِرِينَ كُفْرُهُمْ إِلَّا خَسَارًۭا ﴿٣٩﴾\\
\textamh{40.\  } & قُلْ أَرَءَيْتُمْ شُرَكَآءَكُمُ ٱلَّذِينَ تَدْعُونَ مِن دُونِ ٱللَّهِ أَرُونِى مَاذَا خَلَقُوا۟ مِنَ ٱلْأَرْضِ أَمْ لَهُمْ شِرْكٌۭ فِى ٱلسَّمَـٰوَٟتِ أَمْ ءَاتَيْنَـٰهُمْ كِتَـٰبًۭا فَهُمْ عَلَىٰ بَيِّنَتٍۢ مِّنْهُ ۚ بَلْ إِن يَعِدُ ٱلظَّـٰلِمُونَ بَعْضُهُم بَعْضًا إِلَّا غُرُورًا ﴿٤٠﴾\\
\textamh{41.\  } & ۞ إِنَّ ٱللَّهَ يُمْسِكُ ٱلسَّمَـٰوَٟتِ وَٱلْأَرْضَ أَن تَزُولَا ۚ وَلَئِن زَالَتَآ إِنْ أَمْسَكَهُمَا مِنْ أَحَدٍۢ مِّنۢ بَعْدِهِۦٓ ۚ إِنَّهُۥ كَانَ حَلِيمًا غَفُورًۭا ﴿٤١﴾\\
\textamh{42.\  } & وَأَقْسَمُوا۟ بِٱللَّهِ جَهْدَ أَيْمَـٰنِهِمْ لَئِن جَآءَهُمْ نَذِيرٌۭ لَّيَكُونُنَّ أَهْدَىٰ مِنْ إِحْدَى ٱلْأُمَمِ ۖ فَلَمَّا جَآءَهُمْ نَذِيرٌۭ مَّا زَادَهُمْ إِلَّا نُفُورًا ﴿٤٢﴾\\
\textamh{43.\  } & ٱسْتِكْبَارًۭا فِى ٱلْأَرْضِ وَمَكْرَ ٱلسَّيِّئِ ۚ وَلَا يَحِيقُ ٱلْمَكْرُ ٱلسَّيِّئُ إِلَّا بِأَهْلِهِۦ ۚ فَهَلْ يَنظُرُونَ إِلَّا سُنَّتَ ٱلْأَوَّلِينَ ۚ فَلَن تَجِدَ لِسُنَّتِ ٱللَّهِ تَبْدِيلًۭا ۖ وَلَن تَجِدَ لِسُنَّتِ ٱللَّهِ تَحْوِيلًا ﴿٤٣﴾\\
\textamh{44.\  } & أَوَلَمْ يَسِيرُوا۟ فِى ٱلْأَرْضِ فَيَنظُرُوا۟ كَيْفَ كَانَ عَـٰقِبَةُ ٱلَّذِينَ مِن قَبْلِهِمْ وَكَانُوٓا۟ أَشَدَّ مِنْهُمْ قُوَّةًۭ ۚ وَمَا كَانَ ٱللَّهُ لِيُعْجِزَهُۥ مِن شَىْءٍۢ فِى ٱلسَّمَـٰوَٟتِ وَلَا فِى ٱلْأَرْضِ ۚ إِنَّهُۥ كَانَ عَلِيمًۭا قَدِيرًۭا ﴿٤٤﴾\\
\textamh{45.\  } & وَلَوْ يُؤَاخِذُ ٱللَّهُ ٱلنَّاسَ بِمَا كَسَبُوا۟ مَا تَرَكَ عَلَىٰ ظَهْرِهَا مِن دَآبَّةٍۢ وَلَـٰكِن يُؤَخِّرُهُمْ إِلَىٰٓ أَجَلٍۢ مُّسَمًّۭى ۖ فَإِذَا جَآءَ أَجَلُهُمْ فَإِنَّ ٱللَّهَ كَانَ بِعِبَادِهِۦ بَصِيرًۢا ﴿٤٥﴾\\
\end{longtable}
\clearpage
%% License: BSD style (Berkley) (i.e. Put the Copyright owner's name always)
%% Writer and Copyright (to): Bewketu(Bilal) Tadilo (2016-17)
\centering\section{\LR{\textamharic{ሱራቱ ያሲን -}  \RL{سوره  يس}}}
\begin{longtable}{%
  @{}
    p{.5\textwidth}
  @{~~~~~~~~~~~~~}
    p{.5\textwidth}
    @{}
}
\nopagebreak
\textamh{\ \ \ \ \ \  ቢስሚላሂ አራህመኒ ራሂይም } &  بِسْمِ ٱللَّهِ ٱلرَّحْمَـٰنِ ٱلرَّحِيمِ\\
\textamh{1.\  } &  يسٓ ﴿١﴾\\
\textamh{2.\  } & وَٱلْقُرْءَانِ ٱلْحَكِيمِ ﴿٢﴾\\
\textamh{3.\  } & إِنَّكَ لَمِنَ ٱلْمُرْسَلِينَ ﴿٣﴾\\
\textamh{4.\  } & عَلَىٰ صِرَٰطٍۢ مُّسْتَقِيمٍۢ ﴿٤﴾\\
\textamh{5.\  } & تَنزِيلَ ٱلْعَزِيزِ ٱلرَّحِيمِ ﴿٥﴾\\
\textamh{6.\  } & لِتُنذِرَ قَوْمًۭا مَّآ أُنذِرَ ءَابَآؤُهُمْ فَهُمْ غَٰفِلُونَ ﴿٦﴾\\
\textamh{7.\  } & لَقَدْ حَقَّ ٱلْقَوْلُ عَلَىٰٓ أَكْثَرِهِمْ فَهُمْ لَا يُؤْمِنُونَ ﴿٧﴾\\
\textamh{8.\  } & إِنَّا جَعَلْنَا فِىٓ أَعْنَـٰقِهِمْ أَغْلَـٰلًۭا فَهِىَ إِلَى ٱلْأَذْقَانِ فَهُم مُّقْمَحُونَ ﴿٨﴾\\
\textamh{9.\  } & وَجَعَلْنَا مِنۢ بَيْنِ أَيْدِيهِمْ سَدًّۭا وَمِنْ خَلْفِهِمْ سَدًّۭا فَأَغْشَيْنَـٰهُمْ فَهُمْ لَا يُبْصِرُونَ ﴿٩﴾\\
\textamh{10.\  } & وَسَوَآءٌ عَلَيْهِمْ ءَأَنذَرْتَهُمْ أَمْ لَمْ تُنذِرْهُمْ لَا يُؤْمِنُونَ ﴿١٠﴾\\
\textamh{11.\  } & إِنَّمَا تُنذِرُ مَنِ ٱتَّبَعَ ٱلذِّكْرَ وَخَشِىَ ٱلرَّحْمَـٰنَ بِٱلْغَيْبِ ۖ فَبَشِّرْهُ بِمَغْفِرَةٍۢ وَأَجْرٍۢ كَرِيمٍ ﴿١١﴾\\
\textamh{12.\  } & إِنَّا نَحْنُ نُحْىِ ٱلْمَوْتَىٰ وَنَكْتُبُ مَا قَدَّمُوا۟ وَءَاثَـٰرَهُمْ ۚ وَكُلَّ شَىْءٍ أَحْصَيْنَـٰهُ فِىٓ إِمَامٍۢ مُّبِينٍۢ ﴿١٢﴾\\
\textamh{13.\  } & وَٱضْرِبْ لَهُم مَّثَلًا أَصْحَـٰبَ ٱلْقَرْيَةِ إِذْ جَآءَهَا ٱلْمُرْسَلُونَ ﴿١٣﴾\\
\textamh{14.\  } & إِذْ أَرْسَلْنَآ إِلَيْهِمُ ٱثْنَيْنِ فَكَذَّبُوهُمَا فَعَزَّزْنَا بِثَالِثٍۢ فَقَالُوٓا۟ إِنَّآ إِلَيْكُم مُّرْسَلُونَ ﴿١٤﴾\\
\textamh{15.\  } & قَالُوا۟ مَآ أَنتُمْ إِلَّا بَشَرٌۭ مِّثْلُنَا وَمَآ أَنزَلَ ٱلرَّحْمَـٰنُ مِن شَىْءٍ إِنْ أَنتُمْ إِلَّا تَكْذِبُونَ ﴿١٥﴾\\
\textamh{16.\  } & قَالُوا۟ رَبُّنَا يَعْلَمُ إِنَّآ إِلَيْكُمْ لَمُرْسَلُونَ ﴿١٦﴾\\
\textamh{17.\  } & وَمَا عَلَيْنَآ إِلَّا ٱلْبَلَـٰغُ ٱلْمُبِينُ ﴿١٧﴾\\
\textamh{18.\  } & قَالُوٓا۟ إِنَّا تَطَيَّرْنَا بِكُمْ ۖ لَئِن لَّمْ تَنتَهُوا۟ لَنَرْجُمَنَّكُمْ وَلَيَمَسَّنَّكُم مِّنَّا عَذَابٌ أَلِيمٌۭ ﴿١٨﴾\\
\textamh{19.\  } & قَالُوا۟ طَٰٓئِرُكُم مَّعَكُمْ ۚ أَئِن ذُكِّرْتُم ۚ بَلْ أَنتُمْ قَوْمٌۭ مُّسْرِفُونَ ﴿١٩﴾\\
\textamh{20.\  } & وَجَآءَ مِنْ أَقْصَا ٱلْمَدِينَةِ رَجُلٌۭ يَسْعَىٰ قَالَ يَـٰقَوْمِ ٱتَّبِعُوا۟ ٱلْمُرْسَلِينَ ﴿٢٠﴾\\
\textamh{21.\  } & ٱتَّبِعُوا۟ مَن لَّا يَسْـَٔلُكُمْ أَجْرًۭا وَهُم مُّهْتَدُونَ ﴿٢١﴾\\
\textamh{22.\  } & وَمَا لِىَ لَآ أَعْبُدُ ٱلَّذِى فَطَرَنِى وَإِلَيْهِ تُرْجَعُونَ ﴿٢٢﴾\\
\textamh{23.\  } & ءَأَتَّخِذُ مِن دُونِهِۦٓ ءَالِهَةً إِن يُرِدْنِ ٱلرَّحْمَـٰنُ بِضُرٍّۢ لَّا تُغْنِ عَنِّى شَفَـٰعَتُهُمْ شَيْـًۭٔا وَلَا يُنقِذُونِ ﴿٢٣﴾\\
\textamh{24.\  } & إِنِّىٓ إِذًۭا لَّفِى ضَلَـٰلٍۢ مُّبِينٍ ﴿٢٤﴾\\
\textamh{25.\  } & إِنِّىٓ ءَامَنتُ بِرَبِّكُمْ فَٱسْمَعُونِ ﴿٢٥﴾\\
\textamh{26.\  } & قِيلَ ٱدْخُلِ ٱلْجَنَّةَ ۖ قَالَ يَـٰلَيْتَ قَوْمِى يَعْلَمُونَ ﴿٢٦﴾\\
\textamh{27.\  } & بِمَا غَفَرَ لِى رَبِّى وَجَعَلَنِى مِنَ ٱلْمُكْرَمِينَ ﴿٢٧﴾\\
\textamh{28.\  } & ۞ وَمَآ أَنزَلْنَا عَلَىٰ قَوْمِهِۦ مِنۢ بَعْدِهِۦ مِن جُندٍۢ مِّنَ ٱلسَّمَآءِ وَمَا كُنَّا مُنزِلِينَ ﴿٢٨﴾\\
\textamh{29.\  } & إِن كَانَتْ إِلَّا صَيْحَةًۭ وَٟحِدَةًۭ فَإِذَا هُمْ خَـٰمِدُونَ ﴿٢٩﴾\\
\textamh{30.\  } & يَـٰحَسْرَةً عَلَى ٱلْعِبَادِ ۚ مَا يَأْتِيهِم مِّن رَّسُولٍ إِلَّا كَانُوا۟ بِهِۦ يَسْتَهْزِءُونَ ﴿٣٠﴾\\
\textamh{31.\  } & أَلَمْ يَرَوْا۟ كَمْ أَهْلَكْنَا قَبْلَهُم مِّنَ ٱلْقُرُونِ أَنَّهُمْ إِلَيْهِمْ لَا يَرْجِعُونَ ﴿٣١﴾\\
\textamh{32.\  } & وَإِن كُلٌّۭ لَّمَّا جَمِيعٌۭ لَّدَيْنَا مُحْضَرُونَ ﴿٣٢﴾\\
\textamh{33.\  } & وَءَايَةٌۭ لَّهُمُ ٱلْأَرْضُ ٱلْمَيْتَةُ أَحْيَيْنَـٰهَا وَأَخْرَجْنَا مِنْهَا حَبًّۭا فَمِنْهُ يَأْكُلُونَ ﴿٣٣﴾\\
\textamh{34.\  } & وَجَعَلْنَا فِيهَا جَنَّـٰتٍۢ مِّن نَّخِيلٍۢ وَأَعْنَـٰبٍۢ وَفَجَّرْنَا فِيهَا مِنَ ٱلْعُيُونِ ﴿٣٤﴾\\
\textamh{35.\  } & لِيَأْكُلُوا۟ مِن ثَمَرِهِۦ وَمَا عَمِلَتْهُ أَيْدِيهِمْ ۖ أَفَلَا يَشْكُرُونَ ﴿٣٥﴾\\
\textamh{36.\  } & سُبْحَـٰنَ ٱلَّذِى خَلَقَ ٱلْأَزْوَٟجَ كُلَّهَا مِمَّا تُنۢبِتُ ٱلْأَرْضُ وَمِنْ أَنفُسِهِمْ وَمِمَّا لَا يَعْلَمُونَ ﴿٣٦﴾\\
\textamh{37.\  } & وَءَايَةٌۭ لَّهُمُ ٱلَّيْلُ نَسْلَخُ مِنْهُ ٱلنَّهَارَ فَإِذَا هُم مُّظْلِمُونَ ﴿٣٧﴾\\
\textamh{38.\  } & وَٱلشَّمْسُ تَجْرِى لِمُسْتَقَرٍّۢ لَّهَا ۚ ذَٟلِكَ تَقْدِيرُ ٱلْعَزِيزِ ٱلْعَلِيمِ ﴿٣٨﴾\\
\textamh{39.\  } & وَٱلْقَمَرَ قَدَّرْنَـٰهُ مَنَازِلَ حَتَّىٰ عَادَ كَٱلْعُرْجُونِ ٱلْقَدِيمِ ﴿٣٩﴾\\
\textamh{40.\  } & لَا ٱلشَّمْسُ يَنۢبَغِى لَهَآ أَن تُدْرِكَ ٱلْقَمَرَ وَلَا ٱلَّيْلُ سَابِقُ ٱلنَّهَارِ ۚ وَكُلٌّۭ فِى فَلَكٍۢ يَسْبَحُونَ ﴿٤٠﴾\\
\textamh{41.\  } & وَءَايَةٌۭ لَّهُمْ أَنَّا حَمَلْنَا ذُرِّيَّتَهُمْ فِى ٱلْفُلْكِ ٱلْمَشْحُونِ ﴿٤١﴾\\
\textamh{42.\  } & وَخَلَقْنَا لَهُم مِّن مِّثْلِهِۦ مَا يَرْكَبُونَ ﴿٤٢﴾\\
\textamh{43.\  } & وَإِن نَّشَأْ نُغْرِقْهُمْ فَلَا صَرِيخَ لَهُمْ وَلَا هُمْ يُنقَذُونَ ﴿٤٣﴾\\
\textamh{44.\  } & إِلَّا رَحْمَةًۭ مِّنَّا وَمَتَـٰعًا إِلَىٰ حِينٍۢ ﴿٤٤﴾\\
\textamh{45.\  } & وَإِذَا قِيلَ لَهُمُ ٱتَّقُوا۟ مَا بَيْنَ أَيْدِيكُمْ وَمَا خَلْفَكُمْ لَعَلَّكُمْ تُرْحَمُونَ ﴿٤٥﴾\\
\textamh{46.\  } & وَمَا تَأْتِيهِم مِّنْ ءَايَةٍۢ مِّنْ ءَايَـٰتِ رَبِّهِمْ إِلَّا كَانُوا۟ عَنْهَا مُعْرِضِينَ ﴿٤٦﴾\\
\textamh{47.\  } & وَإِذَا قِيلَ لَهُمْ أَنفِقُوا۟ مِمَّا رَزَقَكُمُ ٱللَّهُ قَالَ ٱلَّذِينَ كَفَرُوا۟ لِلَّذِينَ ءَامَنُوٓا۟ أَنُطْعِمُ مَن لَّوْ يَشَآءُ ٱللَّهُ أَطْعَمَهُۥٓ إِنْ أَنتُمْ إِلَّا فِى ضَلَـٰلٍۢ مُّبِينٍۢ ﴿٤٧﴾\\
\textamh{48.\  } & وَيَقُولُونَ مَتَىٰ هَـٰذَا ٱلْوَعْدُ إِن كُنتُمْ صَـٰدِقِينَ ﴿٤٨﴾\\
\textamh{49.\  } & مَا يَنظُرُونَ إِلَّا صَيْحَةًۭ وَٟحِدَةًۭ تَأْخُذُهُمْ وَهُمْ يَخِصِّمُونَ ﴿٤٩﴾\\
\textamh{50.\  } & فَلَا يَسْتَطِيعُونَ تَوْصِيَةًۭ وَلَآ إِلَىٰٓ أَهْلِهِمْ يَرْجِعُونَ ﴿٥٠﴾\\
\textamh{51.\  } & وَنُفِخَ فِى ٱلصُّورِ فَإِذَا هُم مِّنَ ٱلْأَجْدَاثِ إِلَىٰ رَبِّهِمْ يَنسِلُونَ ﴿٥١﴾\\
\textamh{52.\  } & قَالُوا۟ يَـٰوَيْلَنَا مَنۢ بَعَثَنَا مِن مَّرْقَدِنَا ۜ ۗ هَـٰذَا مَا وَعَدَ ٱلرَّحْمَـٰنُ وَصَدَقَ ٱلْمُرْسَلُونَ ﴿٥٢﴾\\
\textamh{53.\  } & إِن كَانَتْ إِلَّا صَيْحَةًۭ وَٟحِدَةًۭ فَإِذَا هُمْ جَمِيعٌۭ لَّدَيْنَا مُحْضَرُونَ ﴿٥٣﴾\\
\textamh{54.\  } & فَٱلْيَوْمَ لَا تُظْلَمُ نَفْسٌۭ شَيْـًۭٔا وَلَا تُجْزَوْنَ إِلَّا مَا كُنتُمْ تَعْمَلُونَ ﴿٥٤﴾\\
\textamh{55.\  } & إِنَّ أَصْحَـٰبَ ٱلْجَنَّةِ ٱلْيَوْمَ فِى شُغُلٍۢ فَـٰكِهُونَ ﴿٥٥﴾\\
\textamh{56.\  } & هُمْ وَأَزْوَٟجُهُمْ فِى ظِلَـٰلٍ عَلَى ٱلْأَرَآئِكِ مُتَّكِـُٔونَ ﴿٥٦﴾\\
\textamh{57.\  } & لَهُمْ فِيهَا فَـٰكِهَةٌۭ وَلَهُم مَّا يَدَّعُونَ ﴿٥٧﴾\\
\textamh{58.\  } & سَلَـٰمٌۭ قَوْلًۭا مِّن رَّبٍّۢ رَّحِيمٍۢ ﴿٥٨﴾\\
\textamh{59.\  } & وَٱمْتَـٰزُوا۟ ٱلْيَوْمَ أَيُّهَا ٱلْمُجْرِمُونَ ﴿٥٩﴾\\
\textamh{60.\  } & ۞ أَلَمْ أَعْهَدْ إِلَيْكُمْ يَـٰبَنِىٓ ءَادَمَ أَن لَّا تَعْبُدُوا۟ ٱلشَّيْطَٰنَ ۖ إِنَّهُۥ لَكُمْ عَدُوٌّۭ مُّبِينٌۭ ﴿٦٠﴾\\
\textamh{61.\  } & وَأَنِ ٱعْبُدُونِى ۚ هَـٰذَا صِرَٰطٌۭ مُّسْتَقِيمٌۭ ﴿٦١﴾\\
\textamh{62.\  } & وَلَقَدْ أَضَلَّ مِنكُمْ جِبِلًّۭا كَثِيرًا ۖ أَفَلَمْ تَكُونُوا۟ تَعْقِلُونَ ﴿٦٢﴾\\
\textamh{63.\  } & هَـٰذِهِۦ جَهَنَّمُ ٱلَّتِى كُنتُمْ تُوعَدُونَ ﴿٦٣﴾\\
\textamh{64.\  } & ٱصْلَوْهَا ٱلْيَوْمَ بِمَا كُنتُمْ تَكْفُرُونَ ﴿٦٤﴾\\
\textamh{65.\  } & ٱلْيَوْمَ نَخْتِمُ عَلَىٰٓ أَفْوَٟهِهِمْ وَتُكَلِّمُنَآ أَيْدِيهِمْ وَتَشْهَدُ أَرْجُلُهُم بِمَا كَانُوا۟ يَكْسِبُونَ ﴿٦٥﴾\\
\textamh{66.\  } & وَلَوْ نَشَآءُ لَطَمَسْنَا عَلَىٰٓ أَعْيُنِهِمْ فَٱسْتَبَقُوا۟ ٱلصِّرَٰطَ فَأَنَّىٰ يُبْصِرُونَ ﴿٦٦﴾\\
\textamh{67.\  } & وَلَوْ نَشَآءُ لَمَسَخْنَـٰهُمْ عَلَىٰ مَكَانَتِهِمْ فَمَا ٱسْتَطَٰعُوا۟ مُضِيًّۭا وَلَا يَرْجِعُونَ ﴿٦٧﴾\\
\textamh{68.\  } & وَمَن نُّعَمِّرْهُ نُنَكِّسْهُ فِى ٱلْخَلْقِ ۖ أَفَلَا يَعْقِلُونَ ﴿٦٨﴾\\
\textamh{69.\  } & وَمَا عَلَّمْنَـٰهُ ٱلشِّعْرَ وَمَا يَنۢبَغِى لَهُۥٓ ۚ إِنْ هُوَ إِلَّا ذِكْرٌۭ وَقُرْءَانٌۭ مُّبِينٌۭ ﴿٦٩﴾\\
\textamh{70.\  } & لِّيُنذِرَ مَن كَانَ حَيًّۭا وَيَحِقَّ ٱلْقَوْلُ عَلَى ٱلْكَـٰفِرِينَ ﴿٧٠﴾\\
\textamh{71.\  } & أَوَلَمْ يَرَوْا۟ أَنَّا خَلَقْنَا لَهُم مِّمَّا عَمِلَتْ أَيْدِينَآ أَنْعَـٰمًۭا فَهُمْ لَهَا مَـٰلِكُونَ ﴿٧١﴾\\
\textamh{72.\  } & وَذَلَّلْنَـٰهَا لَهُمْ فَمِنْهَا رَكُوبُهُمْ وَمِنْهَا يَأْكُلُونَ ﴿٧٢﴾\\
\textamh{73.\  } & وَلَهُمْ فِيهَا مَنَـٰفِعُ وَمَشَارِبُ ۖ أَفَلَا يَشْكُرُونَ ﴿٧٣﴾\\
\textamh{74.\  } & وَٱتَّخَذُوا۟ مِن دُونِ ٱللَّهِ ءَالِهَةًۭ لَّعَلَّهُمْ يُنصَرُونَ ﴿٧٤﴾\\
\textamh{75.\  } & لَا يَسْتَطِيعُونَ نَصْرَهُمْ وَهُمْ لَهُمْ جُندٌۭ مُّحْضَرُونَ ﴿٧٥﴾\\
\textamh{76.\  } & فَلَا يَحْزُنكَ قَوْلُهُمْ ۘ إِنَّا نَعْلَمُ مَا يُسِرُّونَ وَمَا يُعْلِنُونَ ﴿٧٦﴾\\
\textamh{77.\  } & أَوَلَمْ يَرَ ٱلْإِنسَـٰنُ أَنَّا خَلَقْنَـٰهُ مِن نُّطْفَةٍۢ فَإِذَا هُوَ خَصِيمٌۭ مُّبِينٌۭ ﴿٧٧﴾\\
\textamh{78.\  } & وَضَرَبَ لَنَا مَثَلًۭا وَنَسِىَ خَلْقَهُۥ ۖ قَالَ مَن يُحْىِ ٱلْعِظَـٰمَ وَهِىَ رَمِيمٌۭ ﴿٧٨﴾\\
\textamh{79.\  } & قُلْ يُحْيِيهَا ٱلَّذِىٓ أَنشَأَهَآ أَوَّلَ مَرَّةٍۢ ۖ وَهُوَ بِكُلِّ خَلْقٍ عَلِيمٌ ﴿٧٩﴾\\
\textamh{80.\  } & ٱلَّذِى جَعَلَ لَكُم مِّنَ ٱلشَّجَرِ ٱلْأَخْضَرِ نَارًۭا فَإِذَآ أَنتُم مِّنْهُ تُوقِدُونَ ﴿٨٠﴾\\
\textamh{81.\  } & أَوَلَيْسَ ٱلَّذِى خَلَقَ ٱلسَّمَـٰوَٟتِ وَٱلْأَرْضَ بِقَـٰدِرٍ عَلَىٰٓ أَن يَخْلُقَ مِثْلَهُم ۚ بَلَىٰ وَهُوَ ٱلْخَلَّٰقُ ٱلْعَلِيمُ ﴿٨١﴾\\
\textamh{82.\  } & إِنَّمَآ أَمْرُهُۥٓ إِذَآ أَرَادَ شَيْـًٔا أَن يَقُولَ لَهُۥ كُن فَيَكُونُ ﴿٨٢﴾\\
\textamh{83.\  } & فَسُبْحَـٰنَ ٱلَّذِى بِيَدِهِۦ مَلَكُوتُ كُلِّ شَىْءٍۢ وَإِلَيْهِ تُرْجَعُونَ ﴿٨٣﴾\\
\end{longtable} \newpage

%% License: BSD style (Berkley) (i.e. Put the Copyright owner's name always)
%% Writer and Copyright (to): Bewketu(Bilal) Tadilo (2016-17)
\centering\section{\LR{\textamharic{ሱራቱ አስሳፋት -}  \RL{سوره  الصافات}}}
\begin{longtable}{%
  @{}
    p{.5\textwidth}
  @{~~~~~~~~~~~~}
    p{.5\textwidth}
    @{}
}
\nopagebreak
\textamh{ቢስሚላሂ አራህመኒ ራሂይም } &  بِسْمِ ٱللَّهِ ٱلرَّحْمَـٰنِ ٱلرَّحِيمِ\\
\textamh{1.\  } &  وَٱلصَّـٰٓفَّٰتِ صَفًّۭا ﴿١﴾\\
\textamh{2.\  } & فَٱلزَّٰجِرَٰتِ زَجْرًۭا ﴿٢﴾\\
\textamh{3.\  } & فَٱلتَّٰلِيَـٰتِ ذِكْرًا ﴿٣﴾\\
\textamh{4.\  } & إِنَّ إِلَـٰهَكُمْ لَوَٟحِدٌۭ ﴿٤﴾\\
\textamh{5.\  } & رَّبُّ ٱلسَّمَـٰوَٟتِ وَٱلْأَرْضِ وَمَا بَيْنَهُمَا وَرَبُّ ٱلْمَشَـٰرِقِ ﴿٥﴾\\
\textamh{6.\  } & إِنَّا زَيَّنَّا ٱلسَّمَآءَ ٱلدُّنْيَا بِزِينَةٍ ٱلْكَوَاكِبِ ﴿٦﴾\\
\textamh{7.\  } & وَحِفْظًۭا مِّن كُلِّ شَيْطَٰنٍۢ مَّارِدٍۢ ﴿٧﴾\\
\textamh{8.\  } & لَّا يَسَّمَّعُونَ إِلَى ٱلْمَلَإِ ٱلْأَعْلَىٰ وَيُقْذَفُونَ مِن كُلِّ جَانِبٍۢ ﴿٨﴾\\
\textamh{9.\  } & دُحُورًۭا ۖ وَلَهُمْ عَذَابٌۭ وَاصِبٌ ﴿٩﴾\\
\textamh{10.\  } & إِلَّا مَنْ خَطِفَ ٱلْخَطْفَةَ فَأَتْبَعَهُۥ شِهَابٌۭ ثَاقِبٌۭ ﴿١٠﴾\\
\textamh{11.\  } & فَٱسْتَفْتِهِمْ أَهُمْ أَشَدُّ خَلْقًا أَم مَّنْ خَلَقْنَآ ۚ إِنَّا خَلَقْنَـٰهُم مِّن طِينٍۢ لَّازِبٍۭ ﴿١١﴾\\
\textamh{12.\  } & بَلْ عَجِبْتَ وَيَسْخَرُونَ ﴿١٢﴾\\
\textamh{13.\  } & وَإِذَا ذُكِّرُوا۟ لَا يَذْكُرُونَ ﴿١٣﴾\\
\textamh{14.\  } & وَإِذَا رَأَوْا۟ ءَايَةًۭ يَسْتَسْخِرُونَ ﴿١٤﴾\\
\textamh{15.\  } & وَقَالُوٓا۟ إِنْ هَـٰذَآ إِلَّا سِحْرٌۭ مُّبِينٌ ﴿١٥﴾\\
\textamh{16.\  } & أَءِذَا مِتْنَا وَكُنَّا تُرَابًۭا وَعِظَـٰمًا أَءِنَّا لَمَبْعُوثُونَ ﴿١٦﴾\\
\textamh{17.\  } & أَوَءَابَآؤُنَا ٱلْأَوَّلُونَ ﴿١٧﴾\\
\textamh{18.\  } & قُلْ نَعَمْ وَأَنتُمْ دَٟخِرُونَ ﴿١٨﴾\\
\textamh{19.\  } & فَإِنَّمَا هِىَ زَجْرَةٌۭ وَٟحِدَةٌۭ فَإِذَا هُمْ يَنظُرُونَ ﴿١٩﴾\\
\textamh{20.\  } & وَقَالُوا۟ يَـٰوَيْلَنَا هَـٰذَا يَوْمُ ٱلدِّينِ ﴿٢٠﴾\\
\textamh{21.\  } & هَـٰذَا يَوْمُ ٱلْفَصْلِ ٱلَّذِى كُنتُم بِهِۦ تُكَذِّبُونَ ﴿٢١﴾\\
\textamh{22.\  } & ۞ ٱحْشُرُوا۟ ٱلَّذِينَ ظَلَمُوا۟ وَأَزْوَٟجَهُمْ وَمَا كَانُوا۟ يَعْبُدُونَ ﴿٢٢﴾\\
\textamh{23.\  } & مِن دُونِ ٱللَّهِ فَٱهْدُوهُمْ إِلَىٰ صِرَٰطِ ٱلْجَحِيمِ ﴿٢٣﴾\\
\textamh{24.\  } & وَقِفُوهُمْ ۖ إِنَّهُم مَّسْـُٔولُونَ ﴿٢٤﴾\\
\textamh{25.\  } & مَا لَكُمْ لَا تَنَاصَرُونَ ﴿٢٥﴾\\
\textamh{26.\  } & بَلْ هُمُ ٱلْيَوْمَ مُسْتَسْلِمُونَ ﴿٢٦﴾\\
\textamh{27.\  } & وَأَقْبَلَ بَعْضُهُمْ عَلَىٰ بَعْضٍۢ يَتَسَآءَلُونَ ﴿٢٧﴾\\
\textamh{28.\  } & قَالُوٓا۟ إِنَّكُمْ كُنتُمْ تَأْتُونَنَا عَنِ ٱلْيَمِينِ ﴿٢٨﴾\\
\textamh{29.\  } & قَالُوا۟ بَل لَّمْ تَكُونُوا۟ مُؤْمِنِينَ ﴿٢٩﴾\\
\textamh{30.\  } & وَمَا كَانَ لَنَا عَلَيْكُم مِّن سُلْطَٰنٍۭ ۖ بَلْ كُنتُمْ قَوْمًۭا طَٰغِينَ ﴿٣٠﴾\\
\textamh{31.\  } & فَحَقَّ عَلَيْنَا قَوْلُ رَبِّنَآ ۖ إِنَّا لَذَآئِقُونَ ﴿٣١﴾\\
\textamh{32.\  } & فَأَغْوَيْنَـٰكُمْ إِنَّا كُنَّا غَٰوِينَ ﴿٣٢﴾\\
\textamh{33.\  } & فَإِنَّهُمْ يَوْمَئِذٍۢ فِى ٱلْعَذَابِ مُشْتَرِكُونَ ﴿٣٣﴾\\
\textamh{34.\  } & إِنَّا كَذَٟلِكَ نَفْعَلُ بِٱلْمُجْرِمِينَ ﴿٣٤﴾\\
\textamh{35.\  } & إِنَّهُمْ كَانُوٓا۟ إِذَا قِيلَ لَهُمْ لَآ إِلَـٰهَ إِلَّا ٱللَّهُ يَسْتَكْبِرُونَ ﴿٣٥﴾\\
\textamh{36.\  } & وَيَقُولُونَ أَئِنَّا لَتَارِكُوٓا۟ ءَالِهَتِنَا لِشَاعِرٍۢ مَّجْنُونٍۭ ﴿٣٦﴾\\
\textamh{37.\  } & بَلْ جَآءَ بِٱلْحَقِّ وَصَدَّقَ ٱلْمُرْسَلِينَ ﴿٣٧﴾\\
\textamh{38.\  } & إِنَّكُمْ لَذَآئِقُوا۟ ٱلْعَذَابِ ٱلْأَلِيمِ ﴿٣٨﴾\\
\textamh{39.\  } & وَمَا تُجْزَوْنَ إِلَّا مَا كُنتُمْ تَعْمَلُونَ ﴿٣٩﴾\\
\textamh{40.\  } & إِلَّا عِبَادَ ٱللَّهِ ٱلْمُخْلَصِينَ ﴿٤٠﴾\\
\textamh{41.\  } & أُو۟لَـٰٓئِكَ لَهُمْ رِزْقٌۭ مَّعْلُومٌۭ ﴿٤١﴾\\
\textamh{42.\  } & فَوَٟكِهُ ۖ وَهُم مُّكْرَمُونَ ﴿٤٢﴾\\
\textamh{43.\  } & فِى جَنَّـٰتِ ٱلنَّعِيمِ ﴿٤٣﴾\\
\textamh{44.\  } & عَلَىٰ سُرُرٍۢ مُّتَقَـٰبِلِينَ ﴿٤٤﴾\\
\textamh{45.\  } & يُطَافُ عَلَيْهِم بِكَأْسٍۢ مِّن مَّعِينٍۭ ﴿٤٥﴾\\
\textamh{46.\  } & بَيْضَآءَ لَذَّةٍۢ لِّلشَّـٰرِبِينَ ﴿٤٦﴾\\
\textamh{47.\  } & لَا فِيهَا غَوْلٌۭ وَلَا هُمْ عَنْهَا يُنزَفُونَ ﴿٤٧﴾\\
\textamh{48.\  } & وَعِندَهُمْ قَـٰصِرَٰتُ ٱلطَّرْفِ عِينٌۭ ﴿٤٨﴾\\
\textamh{49.\  } & كَأَنَّهُنَّ بَيْضٌۭ مَّكْنُونٌۭ ﴿٤٩﴾\\
\textamh{50.\  } & فَأَقْبَلَ بَعْضُهُمْ عَلَىٰ بَعْضٍۢ يَتَسَآءَلُونَ ﴿٥٠﴾\\
\textamh{51.\  } & قَالَ قَآئِلٌۭ مِّنْهُمْ إِنِّى كَانَ لِى قَرِينٌۭ ﴿٥١﴾\\
\textamh{52.\  } & يَقُولُ أَءِنَّكَ لَمِنَ ٱلْمُصَدِّقِينَ ﴿٥٢﴾\\
\textamh{53.\  } & أَءِذَا مِتْنَا وَكُنَّا تُرَابًۭا وَعِظَـٰمًا أَءِنَّا لَمَدِينُونَ ﴿٥٣﴾\\
\textamh{54.\  } & قَالَ هَلْ أَنتُم مُّطَّلِعُونَ ﴿٥٤﴾\\
\textamh{55.\  } & فَٱطَّلَعَ فَرَءَاهُ فِى سَوَآءِ ٱلْجَحِيمِ ﴿٥٥﴾\\
\textamh{56.\  } & قَالَ تَٱللَّهِ إِن كِدتَّ لَتُرْدِينِ ﴿٥٦﴾\\
\textamh{57.\  } & وَلَوْلَا نِعْمَةُ رَبِّى لَكُنتُ مِنَ ٱلْمُحْضَرِينَ ﴿٥٧﴾\\
\textamh{58.\  } & أَفَمَا نَحْنُ بِمَيِّتِينَ ﴿٥٨﴾\\
\textamh{59.\  } & إِلَّا مَوْتَتَنَا ٱلْأُولَىٰ وَمَا نَحْنُ بِمُعَذَّبِينَ ﴿٥٩﴾\\
\textamh{60.\  } & إِنَّ هَـٰذَا لَهُوَ ٱلْفَوْزُ ٱلْعَظِيمُ ﴿٦٠﴾\\
\textamh{61.\  } & لِمِثْلِ هَـٰذَا فَلْيَعْمَلِ ٱلْعَـٰمِلُونَ ﴿٦١﴾\\
\textamh{62.\  } & أَذَٟلِكَ خَيْرٌۭ نُّزُلًا أَمْ شَجَرَةُ ٱلزَّقُّومِ ﴿٦٢﴾\\
\textamh{63.\  } & إِنَّا جَعَلْنَـٰهَا فِتْنَةًۭ لِّلظَّـٰلِمِينَ ﴿٦٣﴾\\
\textamh{64.\  } & إِنَّهَا شَجَرَةٌۭ تَخْرُجُ فِىٓ أَصْلِ ٱلْجَحِيمِ ﴿٦٤﴾\\
\textamh{65.\  } & طَلْعُهَا كَأَنَّهُۥ رُءُوسُ ٱلشَّيَـٰطِينِ ﴿٦٥﴾\\
\textamh{66.\  } & فَإِنَّهُمْ لَءَاكِلُونَ مِنْهَا فَمَالِـُٔونَ مِنْهَا ٱلْبُطُونَ ﴿٦٦﴾\\
\textamh{67.\  } & ثُمَّ إِنَّ لَهُمْ عَلَيْهَا لَشَوْبًۭا مِّنْ حَمِيمٍۢ ﴿٦٧﴾\\
\textamh{68.\  } & ثُمَّ إِنَّ مَرْجِعَهُمْ لَإِلَى ٱلْجَحِيمِ ﴿٦٨﴾\\
\textamh{69.\  } & إِنَّهُمْ أَلْفَوْا۟ ءَابَآءَهُمْ ضَآلِّينَ ﴿٦٩﴾\\
\textamh{70.\  } & فَهُمْ عَلَىٰٓ ءَاثَـٰرِهِمْ يُهْرَعُونَ ﴿٧٠﴾\\
\textamh{71.\  } & وَلَقَدْ ضَلَّ قَبْلَهُمْ أَكْثَرُ ٱلْأَوَّلِينَ ﴿٧١﴾\\
\textamh{72.\  } & وَلَقَدْ أَرْسَلْنَا فِيهِم مُّنذِرِينَ ﴿٧٢﴾\\
\textamh{73.\  } & فَٱنظُرْ كَيْفَ كَانَ عَـٰقِبَةُ ٱلْمُنذَرِينَ ﴿٧٣﴾\\
\textamh{74.\  } & إِلَّا عِبَادَ ٱللَّهِ ٱلْمُخْلَصِينَ ﴿٧٤﴾\\
\textamh{75.\  } & وَلَقَدْ نَادَىٰنَا نُوحٌۭ فَلَنِعْمَ ٱلْمُجِيبُونَ ﴿٧٥﴾\\
\textamh{76.\  } & وَنَجَّيْنَـٰهُ وَأَهْلَهُۥ مِنَ ٱلْكَرْبِ ٱلْعَظِيمِ ﴿٧٦﴾\\
\textamh{77.\  } & وَجَعَلْنَا ذُرِّيَّتَهُۥ هُمُ ٱلْبَاقِينَ ﴿٧٧﴾\\
\textamh{78.\  } & وَتَرَكْنَا عَلَيْهِ فِى ٱلْءَاخِرِينَ ﴿٧٨﴾\\
\textamh{79.\  } & سَلَـٰمٌ عَلَىٰ نُوحٍۢ فِى ٱلْعَـٰلَمِينَ ﴿٧٩﴾\\
\textamh{80.\  } & إِنَّا كَذَٟلِكَ نَجْزِى ٱلْمُحْسِنِينَ ﴿٨٠﴾\\
\textamh{81.\  } & إِنَّهُۥ مِنْ عِبَادِنَا ٱلْمُؤْمِنِينَ ﴿٨١﴾\\
\textamh{82.\  } & ثُمَّ أَغْرَقْنَا ٱلْءَاخَرِينَ ﴿٨٢﴾\\
\textamh{83.\  } & ۞ وَإِنَّ مِن شِيعَتِهِۦ لَإِبْرَٰهِيمَ ﴿٨٣﴾\\
\textamh{84.\  } & إِذْ جَآءَ رَبَّهُۥ بِقَلْبٍۢ سَلِيمٍ ﴿٨٤﴾\\
\textamh{85.\  } & إِذْ قَالَ لِأَبِيهِ وَقَوْمِهِۦ مَاذَا تَعْبُدُونَ ﴿٨٥﴾\\
\textamh{86.\  } & أَئِفْكًا ءَالِهَةًۭ دُونَ ٱللَّهِ تُرِيدُونَ ﴿٨٦﴾\\
\textamh{87.\  } & فَمَا ظَنُّكُم بِرَبِّ ٱلْعَـٰلَمِينَ ﴿٨٧﴾\\
\textamh{88.\  } & فَنَظَرَ نَظْرَةًۭ فِى ٱلنُّجُومِ ﴿٨٨﴾\\
\textamh{89.\  } & فَقَالَ إِنِّى سَقِيمٌۭ ﴿٨٩﴾\\
\textamh{90.\  } & فَتَوَلَّوْا۟ عَنْهُ مُدْبِرِينَ ﴿٩٠﴾\\
\textamh{91.\  } & فَرَاغَ إِلَىٰٓ ءَالِهَتِهِمْ فَقَالَ أَلَا تَأْكُلُونَ ﴿٩١﴾\\
\textamh{92.\  } & مَا لَكُمْ لَا تَنطِقُونَ ﴿٩٢﴾\\
\textamh{93.\  } & فَرَاغَ عَلَيْهِمْ ضَرْبًۢا بِٱلْيَمِينِ ﴿٩٣﴾\\
\textamh{94.\  } & فَأَقْبَلُوٓا۟ إِلَيْهِ يَزِفُّونَ ﴿٩٤﴾\\
\textamh{95.\  } & قَالَ أَتَعْبُدُونَ مَا تَنْحِتُونَ ﴿٩٥﴾\\
\textamh{96.\  } & وَٱللَّهُ خَلَقَكُمْ وَمَا تَعْمَلُونَ ﴿٩٦﴾\\
\textamh{97.\  } & قَالُوا۟ ٱبْنُوا۟ لَهُۥ بُنْيَـٰنًۭا فَأَلْقُوهُ فِى ٱلْجَحِيمِ ﴿٩٧﴾\\
\textamh{98.\  } & فَأَرَادُوا۟ بِهِۦ كَيْدًۭا فَجَعَلْنَـٰهُمُ ٱلْأَسْفَلِينَ ﴿٩٨﴾\\
\textamh{99.\  } & وَقَالَ إِنِّى ذَاهِبٌ إِلَىٰ رَبِّى سَيَهْدِينِ ﴿٩٩﴾\\
\textamh{100.\  } & رَبِّ هَبْ لِى مِنَ ٱلصَّـٰلِحِينَ ﴿١٠٠﴾\\
\textamh{101.\  } & فَبَشَّرْنَـٰهُ بِغُلَـٰمٍ حَلِيمٍۢ ﴿١٠١﴾\\
\textamh{102.\  } & فَلَمَّا بَلَغَ مَعَهُ ٱلسَّعْىَ قَالَ يَـٰبُنَىَّ إِنِّىٓ أَرَىٰ فِى ٱلْمَنَامِ أَنِّىٓ أَذْبَحُكَ فَٱنظُرْ مَاذَا تَرَىٰ ۚ قَالَ يَـٰٓأَبَتِ ٱفْعَلْ مَا تُؤْمَرُ ۖ سَتَجِدُنِىٓ إِن شَآءَ ٱللَّهُ مِنَ ٱلصَّـٰبِرِينَ ﴿١٠٢﴾\\
\textamh{103.\  } & فَلَمَّآ أَسْلَمَا وَتَلَّهُۥ لِلْجَبِينِ ﴿١٠٣﴾\\
\textamh{104.\  } & وَنَـٰدَيْنَـٰهُ أَن يَـٰٓإِبْرَٰهِيمُ ﴿١٠٤﴾\\
\textamh{105.\  } & قَدْ صَدَّقْتَ ٱلرُّءْيَآ ۚ إِنَّا كَذَٟلِكَ نَجْزِى ٱلْمُحْسِنِينَ ﴿١٠٥﴾\\
\textamh{106.\  } & إِنَّ هَـٰذَا لَهُوَ ٱلْبَلَـٰٓؤُا۟ ٱلْمُبِينُ ﴿١٠٦﴾\\
\textamh{107.\  } & وَفَدَيْنَـٰهُ بِذِبْحٍ عَظِيمٍۢ ﴿١٠٧﴾\\
\textamh{108.\  } & وَتَرَكْنَا عَلَيْهِ فِى ٱلْءَاخِرِينَ ﴿١٠٨﴾\\
\textamh{109.\  } & سَلَـٰمٌ عَلَىٰٓ إِبْرَٰهِيمَ ﴿١٠٩﴾\\
\textamh{110.\  } & كَذَٟلِكَ نَجْزِى ٱلْمُحْسِنِينَ ﴿١١٠﴾\\
\textamh{111.\  } & إِنَّهُۥ مِنْ عِبَادِنَا ٱلْمُؤْمِنِينَ ﴿١١١﴾\\
\textamh{112.\  } & وَبَشَّرْنَـٰهُ بِإِسْحَـٰقَ نَبِيًّۭا مِّنَ ٱلصَّـٰلِحِينَ ﴿١١٢﴾\\
\textamh{113.\  } & وَبَٰرَكْنَا عَلَيْهِ وَعَلَىٰٓ إِسْحَـٰقَ ۚ وَمِن ذُرِّيَّتِهِمَا مُحْسِنٌۭ وَظَالِمٌۭ لِّنَفْسِهِۦ مُبِينٌۭ ﴿١١٣﴾\\
\textamh{114.\  } & وَلَقَدْ مَنَنَّا عَلَىٰ مُوسَىٰ وَهَـٰرُونَ ﴿١١٤﴾\\
\textamh{115.\  } & وَنَجَّيْنَـٰهُمَا وَقَوْمَهُمَا مِنَ ٱلْكَرْبِ ٱلْعَظِيمِ ﴿١١٥﴾\\
\textamh{116.\  } & وَنَصَرْنَـٰهُمْ فَكَانُوا۟ هُمُ ٱلْغَٰلِبِينَ ﴿١١٦﴾\\
\textamh{117.\  } & وَءَاتَيْنَـٰهُمَا ٱلْكِتَـٰبَ ٱلْمُسْتَبِينَ ﴿١١٧﴾\\
\textamh{118.\  } & وَهَدَيْنَـٰهُمَا ٱلصِّرَٰطَ ٱلْمُسْتَقِيمَ ﴿١١٨﴾\\
\textamh{119.\  } & وَتَرَكْنَا عَلَيْهِمَا فِى ٱلْءَاخِرِينَ ﴿١١٩﴾\\
\textamh{120.\  } & سَلَـٰمٌ عَلَىٰ مُوسَىٰ وَهَـٰرُونَ ﴿١٢٠﴾\\
\textamh{121.\  } & إِنَّا كَذَٟلِكَ نَجْزِى ٱلْمُحْسِنِينَ ﴿١٢١﴾\\
\textamh{122.\  } & إِنَّهُمَا مِنْ عِبَادِنَا ٱلْمُؤْمِنِينَ ﴿١٢٢﴾\\
\textamh{123.\  } & وَإِنَّ إِلْيَاسَ لَمِنَ ٱلْمُرْسَلِينَ ﴿١٢٣﴾\\
\textamh{124.\  } & إِذْ قَالَ لِقَوْمِهِۦٓ أَلَا تَتَّقُونَ ﴿١٢٤﴾\\
\textamh{125.\  } & أَتَدْعُونَ بَعْلًۭا وَتَذَرُونَ أَحْسَنَ ٱلْخَـٰلِقِينَ ﴿١٢٥﴾\\
\textamh{126.\  } & ٱللَّهَ رَبَّكُمْ وَرَبَّ ءَابَآئِكُمُ ٱلْأَوَّلِينَ ﴿١٢٦﴾\\
\textamh{127.\  } & فَكَذَّبُوهُ فَإِنَّهُمْ لَمُحْضَرُونَ ﴿١٢٧﴾\\
\textamh{128.\  } & إِلَّا عِبَادَ ٱللَّهِ ٱلْمُخْلَصِينَ ﴿١٢٨﴾\\
\textamh{129.\  } & وَتَرَكْنَا عَلَيْهِ فِى ٱلْءَاخِرِينَ ﴿١٢٩﴾\\
\textamh{130.\  } & سَلَـٰمٌ عَلَىٰٓ إِلْ يَاسِينَ ﴿١٣٠﴾\\
\textamh{131.\  } & إِنَّا كَذَٟلِكَ نَجْزِى ٱلْمُحْسِنِينَ ﴿١٣١﴾\\
\textamh{132.\  } & إِنَّهُۥ مِنْ عِبَادِنَا ٱلْمُؤْمِنِينَ ﴿١٣٢﴾\\
\textamh{133.\  } & وَإِنَّ لُوطًۭا لَّمِنَ ٱلْمُرْسَلِينَ ﴿١٣٣﴾\\
\textamh{134.\  } & إِذْ نَجَّيْنَـٰهُ وَأَهْلَهُۥٓ أَجْمَعِينَ ﴿١٣٤﴾\\
\textamh{135.\  } & إِلَّا عَجُوزًۭا فِى ٱلْغَٰبِرِينَ ﴿١٣٥﴾\\
\textamh{136.\  } & ثُمَّ دَمَّرْنَا ٱلْءَاخَرِينَ ﴿١٣٦﴾\\
\textamh{137.\  } & وَإِنَّكُمْ لَتَمُرُّونَ عَلَيْهِم مُّصْبِحِينَ ﴿١٣٧﴾\\
\textamh{138.\  } & وَبِٱلَّيْلِ ۗ أَفَلَا تَعْقِلُونَ ﴿١٣٨﴾\\
\textamh{139.\  } & وَإِنَّ يُونُسَ لَمِنَ ٱلْمُرْسَلِينَ ﴿١٣٩﴾\\
\textamh{140.\  } & إِذْ أَبَقَ إِلَى ٱلْفُلْكِ ٱلْمَشْحُونِ ﴿١٤٠﴾\\
\textamh{141.\  } & فَسَاهَمَ فَكَانَ مِنَ ٱلْمُدْحَضِينَ ﴿١٤١﴾\\
\textamh{142.\  } & فَٱلْتَقَمَهُ ٱلْحُوتُ وَهُوَ مُلِيمٌۭ ﴿١٤٢﴾\\
\textamh{143.\  } & فَلَوْلَآ أَنَّهُۥ كَانَ مِنَ ٱلْمُسَبِّحِينَ ﴿١٤٣﴾\\
\textamh{144.\  } & لَلَبِثَ فِى بَطْنِهِۦٓ إِلَىٰ يَوْمِ يُبْعَثُونَ ﴿١٤٤﴾\\
\textamh{145.\  } & ۞ فَنَبَذْنَـٰهُ بِٱلْعَرَآءِ وَهُوَ سَقِيمٌۭ ﴿١٤٥﴾\\
\textamh{146.\  } & وَأَنۢبَتْنَا عَلَيْهِ شَجَرَةًۭ مِّن يَقْطِينٍۢ ﴿١٤٦﴾\\
\textamh{147.\  } & وَأَرْسَلْنَـٰهُ إِلَىٰ مِا۟ئَةِ أَلْفٍ أَوْ يَزِيدُونَ ﴿١٤٧﴾\\
\textamh{148.\  } & فَـَٔامَنُوا۟ فَمَتَّعْنَـٰهُمْ إِلَىٰ حِينٍۢ ﴿١٤٨﴾\\
\textamh{149.\  } & فَٱسْتَفْتِهِمْ أَلِرَبِّكَ ٱلْبَنَاتُ وَلَهُمُ ٱلْبَنُونَ ﴿١٤٩﴾\\
\textamh{150.\  } & أَمْ خَلَقْنَا ٱلْمَلَـٰٓئِكَةَ إِنَـٰثًۭا وَهُمْ شَـٰهِدُونَ ﴿١٥٠﴾\\
\textamh{151.\  } & أَلَآ إِنَّهُم مِّنْ إِفْكِهِمْ لَيَقُولُونَ ﴿١٥١﴾\\
\textamh{152.\  } & وَلَدَ ٱللَّهُ وَإِنَّهُمْ لَكَـٰذِبُونَ ﴿١٥٢﴾\\
\textamh{153.\  } & أَصْطَفَى ٱلْبَنَاتِ عَلَى ٱلْبَنِينَ ﴿١٥٣﴾\\
\textamh{154.\  } & مَا لَكُمْ كَيْفَ تَحْكُمُونَ ﴿١٥٤﴾\\
\textamh{155.\  } & أَفَلَا تَذَكَّرُونَ ﴿١٥٥﴾\\
\textamh{156.\  } & أَمْ لَكُمْ سُلْطَٰنٌۭ مُّبِينٌۭ ﴿١٥٦﴾\\
\textamh{157.\  } & فَأْتُوا۟ بِكِتَـٰبِكُمْ إِن كُنتُمْ صَـٰدِقِينَ ﴿١٥٧﴾\\
\textamh{158.\  } & وَجَعَلُوا۟ بَيْنَهُۥ وَبَيْنَ ٱلْجِنَّةِ نَسَبًۭا ۚ وَلَقَدْ عَلِمَتِ ٱلْجِنَّةُ إِنَّهُمْ لَمُحْضَرُونَ ﴿١٥٨﴾\\
\textamh{159.\  } & سُبْحَـٰنَ ٱللَّهِ عَمَّا يَصِفُونَ ﴿١٥٩﴾\\
\textamh{160.\  } & إِلَّا عِبَادَ ٱللَّهِ ٱلْمُخْلَصِينَ ﴿١٦٠﴾\\
\textamh{161.\  } & فَإِنَّكُمْ وَمَا تَعْبُدُونَ ﴿١٦١﴾\\
\textamh{162.\  } & مَآ أَنتُمْ عَلَيْهِ بِفَـٰتِنِينَ ﴿١٦٢﴾\\
\textamh{163.\  } & إِلَّا مَنْ هُوَ صَالِ ٱلْجَحِيمِ ﴿١٦٣﴾\\
\textamh{164.\  } & وَمَا مِنَّآ إِلَّا لَهُۥ مَقَامٌۭ مَّعْلُومٌۭ ﴿١٦٤﴾\\
\textamh{165.\  } & وَإِنَّا لَنَحْنُ ٱلصَّآفُّونَ ﴿١٦٥﴾\\
\textamh{166.\  } & وَإِنَّا لَنَحْنُ ٱلْمُسَبِّحُونَ ﴿١٦٦﴾\\
\textamh{167.\  } & وَإِن كَانُوا۟ لَيَقُولُونَ ﴿١٦٧﴾\\
\textamh{168.\  } & لَوْ أَنَّ عِندَنَا ذِكْرًۭا مِّنَ ٱلْأَوَّلِينَ ﴿١٦٨﴾\\
\textamh{169.\  } & لَكُنَّا عِبَادَ ٱللَّهِ ٱلْمُخْلَصِينَ ﴿١٦٩﴾\\
\textamh{170.\  } & فَكَفَرُوا۟ بِهِۦ ۖ فَسَوْفَ يَعْلَمُونَ ﴿١٧٠﴾\\
\textamh{171.\  } & وَلَقَدْ سَبَقَتْ كَلِمَتُنَا لِعِبَادِنَا ٱلْمُرْسَلِينَ ﴿١٧١﴾\\
\textamh{172.\  } & إِنَّهُمْ لَهُمُ ٱلْمَنصُورُونَ ﴿١٧٢﴾\\
\textamh{173.\  } & وَإِنَّ جُندَنَا لَهُمُ ٱلْغَٰلِبُونَ ﴿١٧٣﴾\\
\textamh{174.\  } & فَتَوَلَّ عَنْهُمْ حَتَّىٰ حِينٍۢ ﴿١٧٤﴾\\
\textamh{175.\  } & وَأَبْصِرْهُمْ فَسَوْفَ يُبْصِرُونَ ﴿١٧٥﴾\\
\textamh{176.\  } & أَفَبِعَذَابِنَا يَسْتَعْجِلُونَ ﴿١٧٦﴾\\
\textamh{177.\  } & فَإِذَا نَزَلَ بِسَاحَتِهِمْ فَسَآءَ صَبَاحُ ٱلْمُنذَرِينَ ﴿١٧٧﴾\\
\textamh{178.\  } & وَتَوَلَّ عَنْهُمْ حَتَّىٰ حِينٍۢ ﴿١٧٨﴾\\
\textamh{179.\  } & وَأَبْصِرْ فَسَوْفَ يُبْصِرُونَ ﴿١٧٩﴾\\
\textamh{180.\  } & سُبْحَـٰنَ رَبِّكَ رَبِّ ٱلْعِزَّةِ عَمَّا يَصِفُونَ ﴿١٨٠﴾\\
\textamh{181.\  } & وَسَلَـٰمٌ عَلَى ٱلْمُرْسَلِينَ ﴿١٨١﴾\\
\textamh{182.\  } & وَٱلْحَمْدُ لِلَّهِ رَبِّ ٱلْعَـٰلَمِينَ ﴿١٨٢﴾\\
\end{longtable}
\clearpage
%% License: BSD style (Berkley) (i.e. Put the Copyright owner's name always)
%% Writer and Copyright (to): Bewketu(Bilal) Tadilo (2016-17)
\centering\section{\LR{\textamharic{ሱራቱ ሷድ -}  \RL{سوره  ص}}}
\begin{longtable}{%
  @{}
    p{.5\textwidth}
  @{~~~~~~~~~~~~~}
    p{.5\textwidth}
    @{}
}
\nopagebreak
\textamh{ቢስሚላሂ አራህመኒ ራሂይም } &  بِسْمِ ٱللَّهِ ٱلرَّحْمَـٰنِ ٱلرَّحِيمِ\\
\textamh{1.\  } &  صٓ ۚ وَٱلْقُرْءَانِ ذِى ٱلذِّكْرِ ﴿١﴾\\
\textamh{2.\  } & بَلِ ٱلَّذِينَ كَفَرُوا۟ فِى عِزَّةٍۢ وَشِقَاقٍۢ ﴿٢﴾\\
\textamh{3.\  } & كَمْ أَهْلَكْنَا مِن قَبْلِهِم مِّن قَرْنٍۢ فَنَادَوا۟ وَّلَاتَ حِينَ مَنَاصٍۢ ﴿٣﴾\\
\textamh{4.\  } & وَعَجِبُوٓا۟ أَن جَآءَهُم مُّنذِرٌۭ مِّنْهُمْ ۖ وَقَالَ ٱلْكَـٰفِرُونَ هَـٰذَا سَـٰحِرٌۭ كَذَّابٌ ﴿٤﴾\\
\textamh{5.\  } & أَجَعَلَ ٱلْءَالِهَةَ إِلَـٰهًۭا وَٟحِدًا ۖ إِنَّ هَـٰذَا لَشَىْءٌ عُجَابٌۭ ﴿٥﴾\\
\textamh{6.\  } & وَٱنطَلَقَ ٱلْمَلَأُ مِنْهُمْ أَنِ ٱمْشُوا۟ وَٱصْبِرُوا۟ عَلَىٰٓ ءَالِهَتِكُمْ ۖ إِنَّ هَـٰذَا لَشَىْءٌۭ يُرَادُ ﴿٦﴾\\
\textamh{7.\  } & مَا سَمِعْنَا بِهَـٰذَا فِى ٱلْمِلَّةِ ٱلْءَاخِرَةِ إِنْ هَـٰذَآ إِلَّا ٱخْتِلَـٰقٌ ﴿٧﴾\\
\textamh{8.\  } & أَءُنزِلَ عَلَيْهِ ٱلذِّكْرُ مِنۢ بَيْنِنَا ۚ بَلْ هُمْ فِى شَكٍّۢ مِّن ذِكْرِى ۖ بَل لَّمَّا يَذُوقُوا۟ عَذَابِ ﴿٨﴾\\
\textamh{9.\  } & أَمْ عِندَهُمْ خَزَآئِنُ رَحْمَةِ رَبِّكَ ٱلْعَزِيزِ ٱلْوَهَّابِ ﴿٩﴾\\
\textamh{10.\  } & أَمْ لَهُم مُّلْكُ ٱلسَّمَـٰوَٟتِ وَٱلْأَرْضِ وَمَا بَيْنَهُمَا ۖ فَلْيَرْتَقُوا۟ فِى ٱلْأَسْبَٰبِ ﴿١٠﴾\\
\textamh{11.\  } & جُندٌۭ مَّا هُنَالِكَ مَهْزُومٌۭ مِّنَ ٱلْأَحْزَابِ ﴿١١﴾\\
\textamh{12.\  } & كَذَّبَتْ قَبْلَهُمْ قَوْمُ نُوحٍۢ وَعَادٌۭ وَفِرْعَوْنُ ذُو ٱلْأَوْتَادِ ﴿١٢﴾\\
\textamh{13.\  } & وَثَمُودُ وَقَوْمُ لُوطٍۢ وَأَصْحَـٰبُ لْـَٔيْكَةِ ۚ أُو۟لَـٰٓئِكَ ٱلْأَحْزَابُ ﴿١٣﴾\\
\textamh{14.\  } & إِن كُلٌّ إِلَّا كَذَّبَ ٱلرُّسُلَ فَحَقَّ عِقَابِ ﴿١٤﴾\\
\textamh{15.\  } & وَمَا يَنظُرُ هَـٰٓؤُلَآءِ إِلَّا صَيْحَةًۭ وَٟحِدَةًۭ مَّا لَهَا مِن فَوَاقٍۢ ﴿١٥﴾\\
\textamh{16.\  } & وَقَالُوا۟ رَبَّنَا عَجِّل لَّنَا قِطَّنَا قَبْلَ يَوْمِ ٱلْحِسَابِ ﴿١٦﴾\\
\textamh{17.\  } & ٱصْبِرْ عَلَىٰ مَا يَقُولُونَ وَٱذْكُرْ عَبْدَنَا دَاوُۥدَ ذَا ٱلْأَيْدِ ۖ إِنَّهُۥٓ أَوَّابٌ ﴿١٧﴾\\
\textamh{18.\  } & إِنَّا سَخَّرْنَا ٱلْجِبَالَ مَعَهُۥ يُسَبِّحْنَ بِٱلْعَشِىِّ وَٱلْإِشْرَاقِ ﴿١٨﴾\\
\textamh{19.\  } & وَٱلطَّيْرَ مَحْشُورَةًۭ ۖ كُلٌّۭ لَّهُۥٓ أَوَّابٌۭ ﴿١٩﴾\\
\textamh{20.\  } & وَشَدَدْنَا مُلْكَهُۥ وَءَاتَيْنَـٰهُ ٱلْحِكْمَةَ وَفَصْلَ ٱلْخِطَابِ ﴿٢٠﴾\\
\textamh{21.\  } & ۞ وَهَلْ أَتَىٰكَ نَبَؤُا۟ ٱلْخَصْمِ إِذْ تَسَوَّرُوا۟ ٱلْمِحْرَابَ ﴿٢١﴾\\
\textamh{22.\  } & إِذْ دَخَلُوا۟ عَلَىٰ دَاوُۥدَ فَفَزِعَ مِنْهُمْ ۖ قَالُوا۟ لَا تَخَفْ ۖ خَصْمَانِ بَغَىٰ بَعْضُنَا عَلَىٰ بَعْضٍۢ فَٱحْكُم بَيْنَنَا بِٱلْحَقِّ وَلَا تُشْطِطْ وَٱهْدِنَآ إِلَىٰ سَوَآءِ ٱلصِّرَٰطِ ﴿٢٢﴾\\
\textamh{23.\  } & إِنَّ هَـٰذَآ أَخِى لَهُۥ تِسْعٌۭ وَتِسْعُونَ نَعْجَةًۭ وَلِىَ نَعْجَةٌۭ وَٟحِدَةٌۭ فَقَالَ أَكْفِلْنِيهَا وَعَزَّنِى فِى ٱلْخِطَابِ ﴿٢٣﴾\\
\textamh{24.\  } & قَالَ لَقَدْ ظَلَمَكَ بِسُؤَالِ نَعْجَتِكَ إِلَىٰ نِعَاجِهِۦ ۖ وَإِنَّ كَثِيرًۭا مِّنَ ٱلْخُلَطَآءِ لَيَبْغِى بَعْضُهُمْ عَلَىٰ بَعْضٍ إِلَّا ٱلَّذِينَ ءَامَنُوا۟ وَعَمِلُوا۟ ٱلصَّـٰلِحَـٰتِ وَقَلِيلٌۭ مَّا هُمْ ۗ وَظَنَّ دَاوُۥدُ أَنَّمَا فَتَنَّـٰهُ فَٱسْتَغْفَرَ رَبَّهُۥ وَخَرَّ رَاكِعًۭا وَأَنَابَ ۩ ﴿٢٤﴾\\
\textamh{25.\  } & فَغَفَرْنَا لَهُۥ ذَٟلِكَ ۖ وَإِنَّ لَهُۥ عِندَنَا لَزُلْفَىٰ وَحُسْنَ مَـَٔابٍۢ ﴿٢٥﴾\\
\textamh{26.\  } & يَـٰدَاوُۥدُ إِنَّا جَعَلْنَـٰكَ خَلِيفَةًۭ فِى ٱلْأَرْضِ فَٱحْكُم بَيْنَ ٱلنَّاسِ بِٱلْحَقِّ وَلَا تَتَّبِعِ ٱلْهَوَىٰ فَيُضِلَّكَ عَن سَبِيلِ ٱللَّهِ ۚ إِنَّ ٱلَّذِينَ يَضِلُّونَ عَن سَبِيلِ ٱللَّهِ لَهُمْ عَذَابٌۭ شَدِيدٌۢ بِمَا نَسُوا۟ يَوْمَ ٱلْحِسَابِ ﴿٢٦﴾\\
\textamh{27.\  } & وَمَا خَلَقْنَا ٱلسَّمَآءَ وَٱلْأَرْضَ وَمَا بَيْنَهُمَا بَٰطِلًۭا ۚ ذَٟلِكَ ظَنُّ ٱلَّذِينَ كَفَرُوا۟ ۚ فَوَيْلٌۭ لِّلَّذِينَ كَفَرُوا۟ مِنَ ٱلنَّارِ ﴿٢٧﴾\\
\textamh{28.\  } & أَمْ نَجْعَلُ ٱلَّذِينَ ءَامَنُوا۟ وَعَمِلُوا۟ ٱلصَّـٰلِحَـٰتِ كَٱلْمُفْسِدِينَ فِى ٱلْأَرْضِ أَمْ نَجْعَلُ ٱلْمُتَّقِينَ كَٱلْفُجَّارِ ﴿٢٨﴾\\
\textamh{29.\  } & كِتَـٰبٌ أَنزَلْنَـٰهُ إِلَيْكَ مُبَٰرَكٌۭ لِّيَدَّبَّرُوٓا۟ ءَايَـٰتِهِۦ وَلِيَتَذَكَّرَ أُو۟لُوا۟ ٱلْأَلْبَٰبِ ﴿٢٩﴾\\
\textamh{30.\  } & وَوَهَبْنَا لِدَاوُۥدَ سُلَيْمَـٰنَ ۚ نِعْمَ ٱلْعَبْدُ ۖ إِنَّهُۥٓ أَوَّابٌ ﴿٣٠﴾\\
\textamh{31.\  } & إِذْ عُرِضَ عَلَيْهِ بِٱلْعَشِىِّ ٱلصَّـٰفِنَـٰتُ ٱلْجِيَادُ ﴿٣١﴾\\
\textamh{32.\  } & فَقَالَ إِنِّىٓ أَحْبَبْتُ حُبَّ ٱلْخَيْرِ عَن ذِكْرِ رَبِّى حَتَّىٰ تَوَارَتْ بِٱلْحِجَابِ ﴿٣٢﴾\\
\textamh{33.\  } & رُدُّوهَا عَلَىَّ ۖ فَطَفِقَ مَسْحًۢا بِٱلسُّوقِ وَٱلْأَعْنَاقِ ﴿٣٣﴾\\
\textamh{34.\  } & وَلَقَدْ فَتَنَّا سُلَيْمَـٰنَ وَأَلْقَيْنَا عَلَىٰ كُرْسِيِّهِۦ جَسَدًۭا ثُمَّ أَنَابَ ﴿٣٤﴾\\
\textamh{35.\  } & قَالَ رَبِّ ٱغْفِرْ لِى وَهَبْ لِى مُلْكًۭا لَّا يَنۢبَغِى لِأَحَدٍۢ مِّنۢ بَعْدِىٓ ۖ إِنَّكَ أَنتَ ٱلْوَهَّابُ ﴿٣٥﴾\\
\textamh{36.\  } & فَسَخَّرْنَا لَهُ ٱلرِّيحَ تَجْرِى بِأَمْرِهِۦ رُخَآءً حَيْثُ أَصَابَ ﴿٣٦﴾\\
\textamh{37.\  } & وَٱلشَّيَـٰطِينَ كُلَّ بَنَّآءٍۢ وَغَوَّاصٍۢ ﴿٣٧﴾\\
\textamh{38.\  } & وَءَاخَرِينَ مُقَرَّنِينَ فِى ٱلْأَصْفَادِ ﴿٣٨﴾\\
\textamh{39.\  } & هَـٰذَا عَطَآؤُنَا فَٱمْنُنْ أَوْ أَمْسِكْ بِغَيْرِ حِسَابٍۢ ﴿٣٩﴾\\
\textamh{40.\  } & وَإِنَّ لَهُۥ عِندَنَا لَزُلْفَىٰ وَحُسْنَ مَـَٔابٍۢ ﴿٤٠﴾\\
\textamh{41.\  } & وَٱذْكُرْ عَبْدَنَآ أَيُّوبَ إِذْ نَادَىٰ رَبَّهُۥٓ أَنِّى مَسَّنِىَ ٱلشَّيْطَٰنُ بِنُصْبٍۢ وَعَذَابٍ ﴿٤١﴾\\
\textamh{42.\  } & ٱرْكُضْ بِرِجْلِكَ ۖ هَـٰذَا مُغْتَسَلٌۢ بَارِدٌۭ وَشَرَابٌۭ ﴿٤٢﴾\\
\textamh{43.\  } & وَوَهَبْنَا لَهُۥٓ أَهْلَهُۥ وَمِثْلَهُم مَّعَهُمْ رَحْمَةًۭ مِّنَّا وَذِكْرَىٰ لِأُو۟لِى ٱلْأَلْبَٰبِ ﴿٤٣﴾\\
\textamh{44.\  } & وَخُذْ بِيَدِكَ ضِغْثًۭا فَٱضْرِب بِّهِۦ وَلَا تَحْنَثْ ۗ إِنَّا وَجَدْنَـٰهُ صَابِرًۭا ۚ نِّعْمَ ٱلْعَبْدُ ۖ إِنَّهُۥٓ أَوَّابٌۭ ﴿٤٤﴾\\
\textamh{45.\  } & وَٱذْكُرْ عِبَٰدَنَآ إِبْرَٰهِيمَ وَإِسْحَـٰقَ وَيَعْقُوبَ أُو۟لِى ٱلْأَيْدِى وَٱلْأَبْصَـٰرِ ﴿٤٥﴾\\
\textamh{46.\  } & إِنَّآ أَخْلَصْنَـٰهُم بِخَالِصَةٍۢ ذِكْرَى ٱلدَّارِ ﴿٤٦﴾\\
\textamh{47.\  } & وَإِنَّهُمْ عِندَنَا لَمِنَ ٱلْمُصْطَفَيْنَ ٱلْأَخْيَارِ ﴿٤٧﴾\\
\textamh{48.\  } & وَٱذْكُرْ إِسْمَـٰعِيلَ وَٱلْيَسَعَ وَذَا ٱلْكِفْلِ ۖ وَكُلٌّۭ مِّنَ ٱلْأَخْيَارِ ﴿٤٨﴾\\
\textamh{49.\  } & هَـٰذَا ذِكْرٌۭ ۚ وَإِنَّ لِلْمُتَّقِينَ لَحُسْنَ مَـَٔابٍۢ ﴿٤٩﴾\\
\textamh{50.\  } & جَنَّـٰتِ عَدْنٍۢ مُّفَتَّحَةًۭ لَّهُمُ ٱلْأَبْوَٟبُ ﴿٥٠﴾\\
\textamh{51.\  } & مُتَّكِـِٔينَ فِيهَا يَدْعُونَ فِيهَا بِفَـٰكِهَةٍۢ كَثِيرَةٍۢ وَشَرَابٍۢ ﴿٥١﴾\\
\textamh{52.\  } & ۞ وَعِندَهُمْ قَـٰصِرَٰتُ ٱلطَّرْفِ أَتْرَابٌ ﴿٥٢﴾\\
\textamh{53.\  } & هَـٰذَا مَا تُوعَدُونَ لِيَوْمِ ٱلْحِسَابِ ﴿٥٣﴾\\
\textamh{54.\  } & إِنَّ هَـٰذَا لَرِزْقُنَا مَا لَهُۥ مِن نَّفَادٍ ﴿٥٤﴾\\
\textamh{55.\  } & هَـٰذَا ۚ وَإِنَّ لِلطَّٰغِينَ لَشَرَّ مَـَٔابٍۢ ﴿٥٥﴾\\
\textamh{56.\  } & جَهَنَّمَ يَصْلَوْنَهَا فَبِئْسَ ٱلْمِهَادُ ﴿٥٦﴾\\
\textamh{57.\  } & هَـٰذَا فَلْيَذُوقُوهُ حَمِيمٌۭ وَغَسَّاقٌۭ ﴿٥٧﴾\\
\textamh{58.\  } & وَءَاخَرُ مِن شَكْلِهِۦٓ أَزْوَٟجٌ ﴿٥٨﴾\\
\textamh{59.\  } & هَـٰذَا فَوْجٌۭ مُّقْتَحِمٌۭ مَّعَكُمْ ۖ لَا مَرْحَبًۢا بِهِمْ ۚ إِنَّهُمْ صَالُوا۟ ٱلنَّارِ ﴿٥٩﴾\\
\textamh{60.\  } & قَالُوا۟ بَلْ أَنتُمْ لَا مَرْحَبًۢا بِكُمْ ۖ أَنتُمْ قَدَّمْتُمُوهُ لَنَا ۖ فَبِئْسَ ٱلْقَرَارُ ﴿٦٠﴾\\
\textamh{61.\  } & قَالُوا۟ رَبَّنَا مَن قَدَّمَ لَنَا هَـٰذَا فَزِدْهُ عَذَابًۭا ضِعْفًۭا فِى ٱلنَّارِ ﴿٦١﴾\\
\textamh{62.\  } & وَقَالُوا۟ مَا لَنَا لَا نَرَىٰ رِجَالًۭا كُنَّا نَعُدُّهُم مِّنَ ٱلْأَشْرَارِ ﴿٦٢﴾\\
\textamh{63.\  } & أَتَّخَذْنَـٰهُمْ سِخْرِيًّا أَمْ زَاغَتْ عَنْهُمُ ٱلْأَبْصَـٰرُ ﴿٦٣﴾\\
\textamh{64.\  } & إِنَّ ذَٟلِكَ لَحَقٌّۭ تَخَاصُمُ أَهْلِ ٱلنَّارِ ﴿٦٤﴾\\
\textamh{65.\  } & قُلْ إِنَّمَآ أَنَا۠ مُنذِرٌۭ ۖ وَمَا مِنْ إِلَـٰهٍ إِلَّا ٱللَّهُ ٱلْوَٟحِدُ ٱلْقَهَّارُ ﴿٦٥﴾\\
\textamh{66.\  } & رَبُّ ٱلسَّمَـٰوَٟتِ وَٱلْأَرْضِ وَمَا بَيْنَهُمَا ٱلْعَزِيزُ ٱلْغَفَّٰرُ ﴿٦٦﴾\\
\textamh{67.\  } & قُلْ هُوَ نَبَؤٌا۟ عَظِيمٌ ﴿٦٧﴾\\
\textamh{68.\  } & أَنتُمْ عَنْهُ مُعْرِضُونَ ﴿٦٨﴾\\
\textamh{69.\  } & مَا كَانَ لِىَ مِنْ عِلْمٍۭ بِٱلْمَلَإِ ٱلْأَعْلَىٰٓ إِذْ يَخْتَصِمُونَ ﴿٦٩﴾\\
\textamh{70.\  } & إِن يُوحَىٰٓ إِلَىَّ إِلَّآ أَنَّمَآ أَنَا۠ نَذِيرٌۭ مُّبِينٌ ﴿٧٠﴾\\
\textamh{71.\  } & إِذْ قَالَ رَبُّكَ لِلْمَلَـٰٓئِكَةِ إِنِّى خَـٰلِقٌۢ بَشَرًۭا مِّن طِينٍۢ ﴿٧١﴾\\
\textamh{72.\  } & فَإِذَا سَوَّيْتُهُۥ وَنَفَخْتُ فِيهِ مِن رُّوحِى فَقَعُوا۟ لَهُۥ سَـٰجِدِينَ ﴿٧٢﴾\\
\textamh{73.\  } & فَسَجَدَ ٱلْمَلَـٰٓئِكَةُ كُلُّهُمْ أَجْمَعُونَ ﴿٧٣﴾\\
\textamh{74.\  } & إِلَّآ إِبْلِيسَ ٱسْتَكْبَرَ وَكَانَ مِنَ ٱلْكَـٰفِرِينَ ﴿٧٤﴾\\
\textamh{75.\  } & قَالَ يَـٰٓإِبْلِيسُ مَا مَنَعَكَ أَن تَسْجُدَ لِمَا خَلَقْتُ بِيَدَىَّ ۖ أَسْتَكْبَرْتَ أَمْ كُنتَ مِنَ ٱلْعَالِينَ ﴿٧٥﴾\\
\textamh{76.\  } & قَالَ أَنَا۠ خَيْرٌۭ مِّنْهُ ۖ خَلَقْتَنِى مِن نَّارٍۢ وَخَلَقْتَهُۥ مِن طِينٍۢ ﴿٧٦﴾\\
\textamh{77.\  } & قَالَ فَٱخْرُجْ مِنْهَا فَإِنَّكَ رَجِيمٌۭ ﴿٧٧﴾\\
\textamh{78.\  } & وَإِنَّ عَلَيْكَ لَعْنَتِىٓ إِلَىٰ يَوْمِ ٱلدِّينِ ﴿٧٨﴾\\
\textamh{79.\  } & قَالَ رَبِّ فَأَنظِرْنِىٓ إِلَىٰ يَوْمِ يُبْعَثُونَ ﴿٧٩﴾\\
\textamh{80.\  } & قَالَ فَإِنَّكَ مِنَ ٱلْمُنظَرِينَ ﴿٨٠﴾\\
\textamh{81.\  } & إِلَىٰ يَوْمِ ٱلْوَقْتِ ٱلْمَعْلُومِ ﴿٨١﴾\\
\textamh{82.\  } & قَالَ فَبِعِزَّتِكَ لَأُغْوِيَنَّهُمْ أَجْمَعِينَ ﴿٨٢﴾\\
\textamh{83.\  } & إِلَّا عِبَادَكَ مِنْهُمُ ٱلْمُخْلَصِينَ ﴿٨٣﴾\\
\textamh{84.\  } & قَالَ فَٱلْحَقُّ وَٱلْحَقَّ أَقُولُ ﴿٨٤﴾\\
\textamh{85.\  } & لَأَمْلَأَنَّ جَهَنَّمَ مِنكَ وَمِمَّن تَبِعَكَ مِنْهُمْ أَجْمَعِينَ ﴿٨٥﴾\\
\textamh{86.\  } & قُلْ مَآ أَسْـَٔلُكُمْ عَلَيْهِ مِنْ أَجْرٍۢ وَمَآ أَنَا۠ مِنَ ٱلْمُتَكَلِّفِينَ ﴿٨٦﴾\\
\textamh{87.\  } & إِنْ هُوَ إِلَّا ذِكْرٌۭ لِّلْعَـٰلَمِينَ ﴿٨٧﴾\\
\textamh{88.\  } & وَلَتَعْلَمُنَّ نَبَأَهُۥ بَعْدَ حِينٍۭ ﴿٨٨﴾\\
\end{longtable}
\clearpage
%% License: BSD style (Berkley) (i.e. Put the Copyright owner's name always)
%% Writer and Copyright (to): Bewketu(Bilal) Tadilo (2016-17)
\centering\section{\LR{\textamharic{ሱራቱ አልዙመር -}  \RL{سوره  الزمر}}}
\begin{longtable}{%
  @{}
    p{.5\textwidth}
  @{~~~~~~~~~~~~}
    p{.5\textwidth}
    @{}
}
\nopagebreak
\textamh{ቢስሚላሂ አራህመኒ ራሂይም } &  بِسْمِ ٱللَّهِ ٱلرَّحْمَـٰنِ ٱلرَّحِيمِ\\
\textamh{1.\  } &  تَنزِيلُ ٱلْكِتَـٰبِ مِنَ ٱللَّهِ ٱلْعَزِيزِ ٱلْحَكِيمِ ﴿١﴾\\
\textamh{2.\  } & إِنَّآ أَنزَلْنَآ إِلَيْكَ ٱلْكِتَـٰبَ بِٱلْحَقِّ فَٱعْبُدِ ٱللَّهَ مُخْلِصًۭا لَّهُ ٱلدِّينَ ﴿٢﴾\\
\textamh{3.\  } & أَلَا لِلَّهِ ٱلدِّينُ ٱلْخَالِصُ ۚ وَٱلَّذِينَ ٱتَّخَذُوا۟ مِن دُونِهِۦٓ أَوْلِيَآءَ مَا نَعْبُدُهُمْ إِلَّا لِيُقَرِّبُونَآ إِلَى ٱللَّهِ زُلْفَىٰٓ إِنَّ ٱللَّهَ يَحْكُمُ بَيْنَهُمْ فِى مَا هُمْ فِيهِ يَخْتَلِفُونَ ۗ إِنَّ ٱللَّهَ لَا يَهْدِى مَنْ هُوَ كَـٰذِبٌۭ كَفَّارٌۭ ﴿٣﴾\\
\textamh{4.\  } & لَّوْ أَرَادَ ٱللَّهُ أَن يَتَّخِذَ وَلَدًۭا لَّٱصْطَفَىٰ مِمَّا يَخْلُقُ مَا يَشَآءُ ۚ سُبْحَـٰنَهُۥ ۖ هُوَ ٱللَّهُ ٱلْوَٟحِدُ ٱلْقَهَّارُ ﴿٤﴾\\
\textamh{5.\  } & خَلَقَ ٱلسَّمَـٰوَٟتِ وَٱلْأَرْضَ بِٱلْحَقِّ ۖ يُكَوِّرُ ٱلَّيْلَ عَلَى ٱلنَّهَارِ وَيُكَوِّرُ ٱلنَّهَارَ عَلَى ٱلَّيْلِ ۖ وَسَخَّرَ ٱلشَّمْسَ وَٱلْقَمَرَ ۖ كُلٌّۭ يَجْرِى لِأَجَلٍۢ مُّسَمًّى ۗ أَلَا هُوَ ٱلْعَزِيزُ ٱلْغَفَّٰرُ ﴿٥﴾\\
\textamh{6.\  } & خَلَقَكُم مِّن نَّفْسٍۢ وَٟحِدَةٍۢ ثُمَّ جَعَلَ مِنْهَا زَوْجَهَا وَأَنزَلَ لَكُم مِّنَ ٱلْأَنْعَـٰمِ ثَمَـٰنِيَةَ أَزْوَٟجٍۢ ۚ يَخْلُقُكُمْ فِى بُطُونِ أُمَّهَـٰتِكُمْ خَلْقًۭا مِّنۢ بَعْدِ خَلْقٍۢ فِى ظُلُمَـٰتٍۢ ثَلَـٰثٍۢ ۚ ذَٟلِكُمُ ٱللَّهُ رَبُّكُمْ لَهُ ٱلْمُلْكُ ۖ لَآ إِلَـٰهَ إِلَّا هُوَ ۖ فَأَنَّىٰ تُصْرَفُونَ ﴿٦﴾\\
\textamh{7.\  } & إِن تَكْفُرُوا۟ فَإِنَّ ٱللَّهَ غَنِىٌّ عَنكُمْ ۖ وَلَا يَرْضَىٰ لِعِبَادِهِ ٱلْكُفْرَ ۖ وَإِن تَشْكُرُوا۟ يَرْضَهُ لَكُمْ ۗ وَلَا تَزِرُ وَازِرَةٌۭ وِزْرَ أُخْرَىٰ ۗ ثُمَّ إِلَىٰ رَبِّكُم مَّرْجِعُكُمْ فَيُنَبِّئُكُم بِمَا كُنتُمْ تَعْمَلُونَ ۚ إِنَّهُۥ عَلِيمٌۢ بِذَاتِ ٱلصُّدُورِ ﴿٧﴾\\
\textamh{8.\  } & ۞ وَإِذَا مَسَّ ٱلْإِنسَـٰنَ ضُرٌّۭ دَعَا رَبَّهُۥ مُنِيبًا إِلَيْهِ ثُمَّ إِذَا خَوَّلَهُۥ نِعْمَةًۭ مِّنْهُ نَسِىَ مَا كَانَ يَدْعُوٓا۟ إِلَيْهِ مِن قَبْلُ وَجَعَلَ لِلَّهِ أَندَادًۭا لِّيُضِلَّ عَن سَبِيلِهِۦ ۚ قُلْ تَمَتَّعْ بِكُفْرِكَ قَلِيلًا ۖ إِنَّكَ مِنْ أَصْحَـٰبِ ٱلنَّارِ ﴿٨﴾\\
\textamh{9.\  } & أَمَّنْ هُوَ قَـٰنِتٌ ءَانَآءَ ٱلَّيْلِ سَاجِدًۭا وَقَآئِمًۭا يَحْذَرُ ٱلْءَاخِرَةَ وَيَرْجُوا۟ رَحْمَةَ رَبِّهِۦ ۗ قُلْ هَلْ يَسْتَوِى ٱلَّذِينَ يَعْلَمُونَ وَٱلَّذِينَ لَا يَعْلَمُونَ ۗ إِنَّمَا يَتَذَكَّرُ أُو۟لُوا۟ ٱلْأَلْبَٰبِ ﴿٩﴾\\
\textamh{10.\  } & قُلْ يَـٰعِبَادِ ٱلَّذِينَ ءَامَنُوا۟ ٱتَّقُوا۟ رَبَّكُمْ ۚ لِلَّذِينَ أَحْسَنُوا۟ فِى هَـٰذِهِ ٱلدُّنْيَا حَسَنَةٌۭ ۗ وَأَرْضُ ٱللَّهِ وَٟسِعَةٌ ۗ إِنَّمَا يُوَفَّى ٱلصَّـٰبِرُونَ أَجْرَهُم بِغَيْرِ حِسَابٍۢ ﴿١٠﴾\\
\textamh{11.\  } & قُلْ إِنِّىٓ أُمِرْتُ أَنْ أَعْبُدَ ٱللَّهَ مُخْلِصًۭا لَّهُ ٱلدِّينَ ﴿١١﴾\\
\textamh{12.\  } & وَأُمِرْتُ لِأَنْ أَكُونَ أَوَّلَ ٱلْمُسْلِمِينَ ﴿١٢﴾\\
\textamh{13.\  } & قُلْ إِنِّىٓ أَخَافُ إِنْ عَصَيْتُ رَبِّى عَذَابَ يَوْمٍ عَظِيمٍۢ ﴿١٣﴾\\
\textamh{14.\  } & قُلِ ٱللَّهَ أَعْبُدُ مُخْلِصًۭا لَّهُۥ دِينِى ﴿١٤﴾\\
\textamh{15.\  } & فَٱعْبُدُوا۟ مَا شِئْتُم مِّن دُونِهِۦ ۗ قُلْ إِنَّ ٱلْخَـٰسِرِينَ ٱلَّذِينَ خَسِرُوٓا۟ أَنفُسَهُمْ وَأَهْلِيهِمْ يَوْمَ ٱلْقِيَـٰمَةِ ۗ أَلَا ذَٟلِكَ هُوَ ٱلْخُسْرَانُ ٱلْمُبِينُ ﴿١٥﴾\\
\textamh{16.\  } & لَهُم مِّن فَوْقِهِمْ ظُلَلٌۭ مِّنَ ٱلنَّارِ وَمِن تَحْتِهِمْ ظُلَلٌۭ ۚ ذَٟلِكَ يُخَوِّفُ ٱللَّهُ بِهِۦ عِبَادَهُۥ ۚ يَـٰعِبَادِ فَٱتَّقُونِ ﴿١٦﴾\\
\textamh{17.\  } & وَٱلَّذِينَ ٱجْتَنَبُوا۟ ٱلطَّٰغُوتَ أَن يَعْبُدُوهَا وَأَنَابُوٓا۟ إِلَى ٱللَّهِ لَهُمُ ٱلْبُشْرَىٰ ۚ فَبَشِّرْ عِبَادِ ﴿١٧﴾\\
\textamh{18.\  } & ٱلَّذِينَ يَسْتَمِعُونَ ٱلْقَوْلَ فَيَتَّبِعُونَ أَحْسَنَهُۥٓ ۚ أُو۟لَـٰٓئِكَ ٱلَّذِينَ هَدَىٰهُمُ ٱللَّهُ ۖ وَأُو۟لَـٰٓئِكَ هُمْ أُو۟لُوا۟ ٱلْأَلْبَٰبِ ﴿١٨﴾\\
\textamh{19.\  } & أَفَمَنْ حَقَّ عَلَيْهِ كَلِمَةُ ٱلْعَذَابِ أَفَأَنتَ تُنقِذُ مَن فِى ٱلنَّارِ ﴿١٩﴾\\
\textamh{20.\  } & لَـٰكِنِ ٱلَّذِينَ ٱتَّقَوْا۟ رَبَّهُمْ لَهُمْ غُرَفٌۭ مِّن فَوْقِهَا غُرَفٌۭ مَّبْنِيَّةٌۭ تَجْرِى مِن تَحْتِهَا ٱلْأَنْهَـٰرُ ۖ وَعْدَ ٱللَّهِ ۖ لَا يُخْلِفُ ٱللَّهُ ٱلْمِيعَادَ ﴿٢٠﴾\\
\textamh{21.\  } & أَلَمْ تَرَ أَنَّ ٱللَّهَ أَنزَلَ مِنَ ٱلسَّمَآءِ مَآءًۭ فَسَلَكَهُۥ يَنَـٰبِيعَ فِى ٱلْأَرْضِ ثُمَّ يُخْرِجُ بِهِۦ زَرْعًۭا مُّخْتَلِفًا أَلْوَٟنُهُۥ ثُمَّ يَهِيجُ فَتَرَىٰهُ مُصْفَرًّۭا ثُمَّ يَجْعَلُهُۥ حُطَٰمًا ۚ إِنَّ فِى ذَٟلِكَ لَذِكْرَىٰ لِأُو۟لِى ٱلْأَلْبَٰبِ ﴿٢١﴾\\
\textamh{22.\  } & أَفَمَن شَرَحَ ٱللَّهُ صَدْرَهُۥ لِلْإِسْلَـٰمِ فَهُوَ عَلَىٰ نُورٍۢ مِّن رَّبِّهِۦ ۚ فَوَيْلٌۭ لِّلْقَـٰسِيَةِ قُلُوبُهُم مِّن ذِكْرِ ٱللَّهِ ۚ أُو۟لَـٰٓئِكَ فِى ضَلَـٰلٍۢ مُّبِينٍ ﴿٢٢﴾\\
\textamh{23.\  } & ٱللَّهُ نَزَّلَ أَحْسَنَ ٱلْحَدِيثِ كِتَـٰبًۭا مُّتَشَـٰبِهًۭا مَّثَانِىَ تَقْشَعِرُّ مِنْهُ جُلُودُ ٱلَّذِينَ يَخْشَوْنَ رَبَّهُمْ ثُمَّ تَلِينُ جُلُودُهُمْ وَقُلُوبُهُمْ إِلَىٰ ذِكْرِ ٱللَّهِ ۚ ذَٟلِكَ هُدَى ٱللَّهِ يَهْدِى بِهِۦ مَن يَشَآءُ ۚ وَمَن يُضْلِلِ ٱللَّهُ فَمَا لَهُۥ مِنْ هَادٍ ﴿٢٣﴾\\
\textamh{24.\  } & أَفَمَن يَتَّقِى بِوَجْهِهِۦ سُوٓءَ ٱلْعَذَابِ يَوْمَ ٱلْقِيَـٰمَةِ ۚ وَقِيلَ لِلظَّـٰلِمِينَ ذُوقُوا۟ مَا كُنتُمْ تَكْسِبُونَ ﴿٢٤﴾\\
\textamh{25.\  } & كَذَّبَ ٱلَّذِينَ مِن قَبْلِهِمْ فَأَتَىٰهُمُ ٱلْعَذَابُ مِنْ حَيْثُ لَا يَشْعُرُونَ ﴿٢٥﴾\\
\textamh{26.\  } & فَأَذَاقَهُمُ ٱللَّهُ ٱلْخِزْىَ فِى ٱلْحَيَوٰةِ ٱلدُّنْيَا ۖ وَلَعَذَابُ ٱلْءَاخِرَةِ أَكْبَرُ ۚ لَوْ كَانُوا۟ يَعْلَمُونَ ﴿٢٦﴾\\
\textamh{27.\  } & وَلَقَدْ ضَرَبْنَا لِلنَّاسِ فِى هَـٰذَا ٱلْقُرْءَانِ مِن كُلِّ مَثَلٍۢ لَّعَلَّهُمْ يَتَذَكَّرُونَ ﴿٢٧﴾\\
\textamh{28.\  } & قُرْءَانًا عَرَبِيًّا غَيْرَ ذِى عِوَجٍۢ لَّعَلَّهُمْ يَتَّقُونَ ﴿٢٨﴾\\
\textamh{29.\  } & ضَرَبَ ٱللَّهُ مَثَلًۭا رَّجُلًۭا فِيهِ شُرَكَآءُ مُتَشَـٰكِسُونَ وَرَجُلًۭا سَلَمًۭا لِّرَجُلٍ هَلْ يَسْتَوِيَانِ مَثَلًا ۚ ٱلْحَمْدُ لِلَّهِ ۚ بَلْ أَكْثَرُهُمْ لَا يَعْلَمُونَ ﴿٢٩﴾\\
\textamh{30.\  } & إِنَّكَ مَيِّتٌۭ وَإِنَّهُم مَّيِّتُونَ ﴿٣٠﴾\\
\textamh{31.\  } & ثُمَّ إِنَّكُمْ يَوْمَ ٱلْقِيَـٰمَةِ عِندَ رَبِّكُمْ تَخْتَصِمُونَ ﴿٣١﴾\\
\textamh{32.\  } & ۞ فَمَنْ أَظْلَمُ مِمَّن كَذَبَ عَلَى ٱللَّهِ وَكَذَّبَ بِٱلصِّدْقِ إِذْ جَآءَهُۥٓ ۚ أَلَيْسَ فِى جَهَنَّمَ مَثْوًۭى لِّلْكَـٰفِرِينَ ﴿٣٢﴾\\
\textamh{33.\  } & وَٱلَّذِى جَآءَ بِٱلصِّدْقِ وَصَدَّقَ بِهِۦٓ ۙ أُو۟لَـٰٓئِكَ هُمُ ٱلْمُتَّقُونَ ﴿٣٣﴾\\
\textamh{34.\  } & لَهُم مَّا يَشَآءُونَ عِندَ رَبِّهِمْ ۚ ذَٟلِكَ جَزَآءُ ٱلْمُحْسِنِينَ ﴿٣٤﴾\\
\textamh{35.\  } & لِيُكَفِّرَ ٱللَّهُ عَنْهُمْ أَسْوَأَ ٱلَّذِى عَمِلُوا۟ وَيَجْزِيَهُمْ أَجْرَهُم بِأَحْسَنِ ٱلَّذِى كَانُوا۟ يَعْمَلُونَ ﴿٣٥﴾\\
\textamh{36.\  } & أَلَيْسَ ٱللَّهُ بِكَافٍ عَبْدَهُۥ ۖ وَيُخَوِّفُونَكَ بِٱلَّذِينَ مِن دُونِهِۦ ۚ وَمَن يُضْلِلِ ٱللَّهُ فَمَا لَهُۥ مِنْ هَادٍۢ ﴿٣٦﴾\\
\textamh{37.\  } & وَمَن يَهْدِ ٱللَّهُ فَمَا لَهُۥ مِن مُّضِلٍّ ۗ أَلَيْسَ ٱللَّهُ بِعَزِيزٍۢ ذِى ٱنتِقَامٍۢ ﴿٣٧﴾\\
\textamh{38.\  } & وَلَئِن سَأَلْتَهُم مَّنْ خَلَقَ ٱلسَّمَـٰوَٟتِ وَٱلْأَرْضَ لَيَقُولُنَّ ٱللَّهُ ۚ قُلْ أَفَرَءَيْتُم مَّا تَدْعُونَ مِن دُونِ ٱللَّهِ إِنْ أَرَادَنِىَ ٱللَّهُ بِضُرٍّ هَلْ هُنَّ كَـٰشِفَـٰتُ ضُرِّهِۦٓ أَوْ أَرَادَنِى بِرَحْمَةٍ هَلْ هُنَّ مُمْسِكَـٰتُ رَحْمَتِهِۦ ۚ قُلْ حَسْبِىَ ٱللَّهُ ۖ عَلَيْهِ يَتَوَكَّلُ ٱلْمُتَوَكِّلُونَ ﴿٣٨﴾\\
\textamh{39.\  } & قُلْ يَـٰقَوْمِ ٱعْمَلُوا۟ عَلَىٰ مَكَانَتِكُمْ إِنِّى عَـٰمِلٌۭ ۖ فَسَوْفَ تَعْلَمُونَ ﴿٣٩﴾\\
\textamh{40.\  } & مَن يَأْتِيهِ عَذَابٌۭ يُخْزِيهِ وَيَحِلُّ عَلَيْهِ عَذَابٌۭ مُّقِيمٌ ﴿٤٠﴾\\
\textamh{41.\  } & إِنَّآ أَنزَلْنَا عَلَيْكَ ٱلْكِتَـٰبَ لِلنَّاسِ بِٱلْحَقِّ ۖ فَمَنِ ٱهْتَدَىٰ فَلِنَفْسِهِۦ ۖ وَمَن ضَلَّ فَإِنَّمَا يَضِلُّ عَلَيْهَا ۖ وَمَآ أَنتَ عَلَيْهِم بِوَكِيلٍ ﴿٤١﴾\\
\textamh{42.\  } & ٱللَّهُ يَتَوَفَّى ٱلْأَنفُسَ حِينَ مَوْتِهَا وَٱلَّتِى لَمْ تَمُتْ فِى مَنَامِهَا ۖ فَيُمْسِكُ ٱلَّتِى قَضَىٰ عَلَيْهَا ٱلْمَوْتَ وَيُرْسِلُ ٱلْأُخْرَىٰٓ إِلَىٰٓ أَجَلٍۢ مُّسَمًّى ۚ إِنَّ فِى ذَٟلِكَ لَءَايَـٰتٍۢ لِّقَوْمٍۢ يَتَفَكَّرُونَ ﴿٤٢﴾\\
\textamh{43.\  } & أَمِ ٱتَّخَذُوا۟ مِن دُونِ ٱللَّهِ شُفَعَآءَ ۚ قُلْ أَوَلَوْ كَانُوا۟ لَا يَمْلِكُونَ شَيْـًۭٔا وَلَا يَعْقِلُونَ ﴿٤٣﴾\\
\textamh{44.\  } & قُل لِّلَّهِ ٱلشَّفَـٰعَةُ جَمِيعًۭا ۖ لَّهُۥ مُلْكُ ٱلسَّمَـٰوَٟتِ وَٱلْأَرْضِ ۖ ثُمَّ إِلَيْهِ تُرْجَعُونَ ﴿٤٤﴾\\
\textamh{45.\  } & وَإِذَا ذُكِرَ ٱللَّهُ وَحْدَهُ ٱشْمَأَزَّتْ قُلُوبُ ٱلَّذِينَ لَا يُؤْمِنُونَ بِٱلْءَاخِرَةِ ۖ وَإِذَا ذُكِرَ ٱلَّذِينَ مِن دُونِهِۦٓ إِذَا هُمْ يَسْتَبْشِرُونَ ﴿٤٥﴾\\
\textamh{46.\  } & قُلِ ٱللَّهُمَّ فَاطِرَ ٱلسَّمَـٰوَٟتِ وَٱلْأَرْضِ عَـٰلِمَ ٱلْغَيْبِ وَٱلشَّهَـٰدَةِ أَنتَ تَحْكُمُ بَيْنَ عِبَادِكَ فِى مَا كَانُوا۟ فِيهِ يَخْتَلِفُونَ ﴿٤٦﴾\\
\textamh{47.\  } & وَلَوْ أَنَّ لِلَّذِينَ ظَلَمُوا۟ مَا فِى ٱلْأَرْضِ جَمِيعًۭا وَمِثْلَهُۥ مَعَهُۥ لَٱفْتَدَوْا۟ بِهِۦ مِن سُوٓءِ ٱلْعَذَابِ يَوْمَ ٱلْقِيَـٰمَةِ ۚ وَبَدَا لَهُم مِّنَ ٱللَّهِ مَا لَمْ يَكُونُوا۟ يَحْتَسِبُونَ ﴿٤٧﴾\\
\textamh{48.\  } & وَبَدَا لَهُمْ سَيِّـَٔاتُ مَا كَسَبُوا۟ وَحَاقَ بِهِم مَّا كَانُوا۟ بِهِۦ يَسْتَهْزِءُونَ ﴿٤٨﴾\\
\textamh{49.\  } & فَإِذَا مَسَّ ٱلْإِنسَـٰنَ ضُرٌّۭ دَعَانَا ثُمَّ إِذَا خَوَّلْنَـٰهُ نِعْمَةًۭ مِّنَّا قَالَ إِنَّمَآ أُوتِيتُهُۥ عَلَىٰ عِلْمٍۭ ۚ بَلْ هِىَ فِتْنَةٌۭ وَلَـٰكِنَّ أَكْثَرَهُمْ لَا يَعْلَمُونَ ﴿٤٩﴾\\
\textamh{50.\  } & قَدْ قَالَهَا ٱلَّذِينَ مِن قَبْلِهِمْ فَمَآ أَغْنَىٰ عَنْهُم مَّا كَانُوا۟ يَكْسِبُونَ ﴿٥٠﴾\\
\textamh{51.\  } & فَأَصَابَهُمْ سَيِّـَٔاتُ مَا كَسَبُوا۟ ۚ وَٱلَّذِينَ ظَلَمُوا۟ مِنْ هَـٰٓؤُلَآءِ سَيُصِيبُهُمْ سَيِّـَٔاتُ مَا كَسَبُوا۟ وَمَا هُم بِمُعْجِزِينَ ﴿٥١﴾\\
\textamh{52.\  } & أَوَلَمْ يَعْلَمُوٓا۟ أَنَّ ٱللَّهَ يَبْسُطُ ٱلرِّزْقَ لِمَن يَشَآءُ وَيَقْدِرُ ۚ إِنَّ فِى ذَٟلِكَ لَءَايَـٰتٍۢ لِّقَوْمٍۢ يُؤْمِنُونَ ﴿٥٢﴾\\
\textamh{53.\  } & ۞ قُلْ يَـٰعِبَادِىَ ٱلَّذِينَ أَسْرَفُوا۟ عَلَىٰٓ أَنفُسِهِمْ لَا تَقْنَطُوا۟ مِن رَّحْمَةِ ٱللَّهِ ۚ إِنَّ ٱللَّهَ يَغْفِرُ ٱلذُّنُوبَ جَمِيعًا ۚ إِنَّهُۥ هُوَ ٱلْغَفُورُ ٱلرَّحِيمُ ﴿٥٣﴾\\
\textamh{54.\  } & وَأَنِيبُوٓا۟ إِلَىٰ رَبِّكُمْ وَأَسْلِمُوا۟ لَهُۥ مِن قَبْلِ أَن يَأْتِيَكُمُ ٱلْعَذَابُ ثُمَّ لَا تُنصَرُونَ ﴿٥٤﴾\\
\textamh{55.\  } & وَٱتَّبِعُوٓا۟ أَحْسَنَ مَآ أُنزِلَ إِلَيْكُم مِّن رَّبِّكُم مِّن قَبْلِ أَن يَأْتِيَكُمُ ٱلْعَذَابُ بَغْتَةًۭ وَأَنتُمْ لَا تَشْعُرُونَ ﴿٥٥﴾\\
\textamh{56.\  } & أَن تَقُولَ نَفْسٌۭ يَـٰحَسْرَتَىٰ عَلَىٰ مَا فَرَّطتُ فِى جَنۢبِ ٱللَّهِ وَإِن كُنتُ لَمِنَ ٱلسَّٰخِرِينَ ﴿٥٦﴾\\
\textamh{57.\  } & أَوْ تَقُولَ لَوْ أَنَّ ٱللَّهَ هَدَىٰنِى لَكُنتُ مِنَ ٱلْمُتَّقِينَ ﴿٥٧﴾\\
\textamh{58.\  } & أَوْ تَقُولَ حِينَ تَرَى ٱلْعَذَابَ لَوْ أَنَّ لِى كَرَّةًۭ فَأَكُونَ مِنَ ٱلْمُحْسِنِينَ ﴿٥٨﴾\\
\textamh{59.\  } & بَلَىٰ قَدْ جَآءَتْكَ ءَايَـٰتِى فَكَذَّبْتَ بِهَا وَٱسْتَكْبَرْتَ وَكُنتَ مِنَ ٱلْكَـٰفِرِينَ ﴿٥٩﴾\\
\textamh{60.\  } & وَيَوْمَ ٱلْقِيَـٰمَةِ تَرَى ٱلَّذِينَ كَذَبُوا۟ عَلَى ٱللَّهِ وُجُوهُهُم مُّسْوَدَّةٌ ۚ أَلَيْسَ فِى جَهَنَّمَ مَثْوًۭى لِّلْمُتَكَبِّرِينَ ﴿٦٠﴾\\
\textamh{61.\  } & وَيُنَجِّى ٱللَّهُ ٱلَّذِينَ ٱتَّقَوْا۟ بِمَفَازَتِهِمْ لَا يَمَسُّهُمُ ٱلسُّوٓءُ وَلَا هُمْ يَحْزَنُونَ ﴿٦١﴾\\
\textamh{62.\  } & ٱللَّهُ خَـٰلِقُ كُلِّ شَىْءٍۢ ۖ وَهُوَ عَلَىٰ كُلِّ شَىْءٍۢ وَكِيلٌۭ ﴿٦٢﴾\\
\textamh{63.\  } & لَّهُۥ مَقَالِيدُ ٱلسَّمَـٰوَٟتِ وَٱلْأَرْضِ ۗ وَٱلَّذِينَ كَفَرُوا۟ بِـَٔايَـٰتِ ٱللَّهِ أُو۟لَـٰٓئِكَ هُمُ ٱلْخَـٰسِرُونَ ﴿٦٣﴾\\
\textamh{64.\  } & قُلْ أَفَغَيْرَ ٱللَّهِ تَأْمُرُوٓنِّىٓ أَعْبُدُ أَيُّهَا ٱلْجَٰهِلُونَ ﴿٦٤﴾\\
\textamh{65.\  } & وَلَقَدْ أُوحِىَ إِلَيْكَ وَإِلَى ٱلَّذِينَ مِن قَبْلِكَ لَئِنْ أَشْرَكْتَ لَيَحْبَطَنَّ عَمَلُكَ وَلَتَكُونَنَّ مِنَ ٱلْخَـٰسِرِينَ ﴿٦٥﴾\\
\textamh{66.\  } & بَلِ ٱللَّهَ فَٱعْبُدْ وَكُن مِّنَ ٱلشَّـٰكِرِينَ ﴿٦٦﴾\\
\textamh{67.\  } & وَمَا قَدَرُوا۟ ٱللَّهَ حَقَّ قَدْرِهِۦ وَٱلْأَرْضُ جَمِيعًۭا قَبْضَتُهُۥ يَوْمَ ٱلْقِيَـٰمَةِ وَٱلسَّمَـٰوَٟتُ مَطْوِيَّٰتٌۢ بِيَمِينِهِۦ ۚ سُبْحَـٰنَهُۥ وَتَعَـٰلَىٰ عَمَّا يُشْرِكُونَ ﴿٦٧﴾\\
\textamh{68.\  } & وَنُفِخَ فِى ٱلصُّورِ فَصَعِقَ مَن فِى ٱلسَّمَـٰوَٟتِ وَمَن فِى ٱلْأَرْضِ إِلَّا مَن شَآءَ ٱللَّهُ ۖ ثُمَّ نُفِخَ فِيهِ أُخْرَىٰ فَإِذَا هُمْ قِيَامٌۭ يَنظُرُونَ ﴿٦٨﴾\\
\textamh{69.\  } & وَأَشْرَقَتِ ٱلْأَرْضُ بِنُورِ رَبِّهَا وَوُضِعَ ٱلْكِتَـٰبُ وَجِا۟ىٓءَ بِٱلنَّبِيِّۦنَ وَٱلشُّهَدَآءِ وَقُضِىَ بَيْنَهُم بِٱلْحَقِّ وَهُمْ لَا يُظْلَمُونَ ﴿٦٩﴾\\
\textamh{70.\  } & وَوُفِّيَتْ كُلُّ نَفْسٍۢ مَّا عَمِلَتْ وَهُوَ أَعْلَمُ بِمَا يَفْعَلُونَ ﴿٧٠﴾\\
\textamh{71.\  } & وَسِيقَ ٱلَّذِينَ كَفَرُوٓا۟ إِلَىٰ جَهَنَّمَ زُمَرًا ۖ حَتَّىٰٓ إِذَا جَآءُوهَا فُتِحَتْ أَبْوَٟبُهَا وَقَالَ لَهُمْ خَزَنَتُهَآ أَلَمْ يَأْتِكُمْ رُسُلٌۭ مِّنكُمْ يَتْلُونَ عَلَيْكُمْ ءَايَـٰتِ رَبِّكُمْ وَيُنذِرُونَكُمْ لِقَآءَ يَوْمِكُمْ هَـٰذَا ۚ قَالُوا۟ بَلَىٰ وَلَـٰكِنْ حَقَّتْ كَلِمَةُ ٱلْعَذَابِ عَلَى ٱلْكَـٰفِرِينَ ﴿٧١﴾\\
\textamh{72.\  } & قِيلَ ٱدْخُلُوٓا۟ أَبْوَٟبَ جَهَنَّمَ خَـٰلِدِينَ فِيهَا ۖ فَبِئْسَ مَثْوَى ٱلْمُتَكَبِّرِينَ ﴿٧٢﴾\\
\textamh{73.\  } & وَسِيقَ ٱلَّذِينَ ٱتَّقَوْا۟ رَبَّهُمْ إِلَى ٱلْجَنَّةِ زُمَرًا ۖ حَتَّىٰٓ إِذَا جَآءُوهَا وَفُتِحَتْ أَبْوَٟبُهَا وَقَالَ لَهُمْ خَزَنَتُهَا سَلَـٰمٌ عَلَيْكُمْ طِبْتُمْ فَٱدْخُلُوهَا خَـٰلِدِينَ ﴿٧٣﴾\\
\textamh{74.\  } & وَقَالُوا۟ ٱلْحَمْدُ لِلَّهِ ٱلَّذِى صَدَقَنَا وَعْدَهُۥ وَأَوْرَثَنَا ٱلْأَرْضَ نَتَبَوَّأُ مِنَ ٱلْجَنَّةِ حَيْثُ نَشَآءُ ۖ فَنِعْمَ أَجْرُ ٱلْعَـٰمِلِينَ ﴿٧٤﴾\\
\textamh{75.\  } & وَتَرَى ٱلْمَلَـٰٓئِكَةَ حَآفِّينَ مِنْ حَوْلِ ٱلْعَرْشِ يُسَبِّحُونَ بِحَمْدِ رَبِّهِمْ ۖ وَقُضِىَ بَيْنَهُم بِٱلْحَقِّ وَقِيلَ ٱلْحَمْدُ لِلَّهِ رَبِّ ٱلْعَـٰلَمِينَ ﴿٧٥﴾\\
\end{longtable}
\clearpage
%% License: BSD style (Berkley) (i.e. Put the Copyright owner's name always)
%% Writer and Copyright (to): Bewketu(Bilal) Tadilo (2016-17)
\centering\section{\LR{\textamharic{ሱራቱ ጋፊር -}  \RL{سوره  غافر}}}
\begin{longtable}{%
  @{}
    p{.5\textwidth}
  @{~~~~~~~~~~~~~}
    p{.5\textwidth}
    @{}
}
\nopagebreak
\textamh{\ \ \ \ \ \  ቢስሚላሂ አራህመኒ ራሂይም } &  بِسْمِ ٱللَّهِ ٱلرَّحْمَـٰنِ ٱلرَّحِيمِ\\
\textamh{1.\  } &  حمٓ ﴿١﴾\\
\textamh{2.\  } & تَنزِيلُ ٱلْكِتَـٰبِ مِنَ ٱللَّهِ ٱلْعَزِيزِ ٱلْعَلِيمِ ﴿٢﴾\\
\textamh{3.\  } & غَافِرِ ٱلذَّنۢبِ وَقَابِلِ ٱلتَّوْبِ شَدِيدِ ٱلْعِقَابِ ذِى ٱلطَّوْلِ ۖ لَآ إِلَـٰهَ إِلَّا هُوَ ۖ إِلَيْهِ ٱلْمَصِيرُ ﴿٣﴾\\
\textamh{4.\  } & مَا يُجَٰدِلُ فِىٓ ءَايَـٰتِ ٱللَّهِ إِلَّا ٱلَّذِينَ كَفَرُوا۟ فَلَا يَغْرُرْكَ تَقَلُّبُهُمْ فِى ٱلْبِلَـٰدِ ﴿٤﴾\\
\textamh{5.\  } & كَذَّبَتْ قَبْلَهُمْ قَوْمُ نُوحٍۢ وَٱلْأَحْزَابُ مِنۢ بَعْدِهِمْ ۖ وَهَمَّتْ كُلُّ أُمَّةٍۭ بِرَسُولِهِمْ لِيَأْخُذُوهُ ۖ وَجَٰدَلُوا۟ بِٱلْبَٰطِلِ لِيُدْحِضُوا۟ بِهِ ٱلْحَقَّ فَأَخَذْتُهُمْ ۖ فَكَيْفَ كَانَ عِقَابِ ﴿٥﴾\\
\textamh{6.\  } & وَكَذَٟلِكَ حَقَّتْ كَلِمَتُ رَبِّكَ عَلَى ٱلَّذِينَ كَفَرُوٓا۟ أَنَّهُمْ أَصْحَـٰبُ ٱلنَّارِ ﴿٦﴾\\
\textamh{7.\  } & ٱلَّذِينَ يَحْمِلُونَ ٱلْعَرْشَ وَمَنْ حَوْلَهُۥ يُسَبِّحُونَ بِحَمْدِ رَبِّهِمْ وَيُؤْمِنُونَ بِهِۦ وَيَسْتَغْفِرُونَ لِلَّذِينَ ءَامَنُوا۟ رَبَّنَا وَسِعْتَ كُلَّ شَىْءٍۢ رَّحْمَةًۭ وَعِلْمًۭا فَٱغْفِرْ لِلَّذِينَ تَابُوا۟ وَٱتَّبَعُوا۟ سَبِيلَكَ وَقِهِمْ عَذَابَ ٱلْجَحِيمِ ﴿٧﴾\\
\textamh{8.\  } & رَبَّنَا وَأَدْخِلْهُمْ جَنَّـٰتِ عَدْنٍ ٱلَّتِى وَعَدتَّهُمْ وَمَن صَلَحَ مِنْ ءَابَآئِهِمْ وَأَزْوَٟجِهِمْ وَذُرِّيَّٰتِهِمْ ۚ إِنَّكَ أَنتَ ٱلْعَزِيزُ ٱلْحَكِيمُ ﴿٨﴾\\
\textamh{9.\  } & وَقِهِمُ ٱلسَّيِّـَٔاتِ ۚ وَمَن تَقِ ٱلسَّيِّـَٔاتِ يَوْمَئِذٍۢ فَقَدْ رَحِمْتَهُۥ ۚ وَذَٟلِكَ هُوَ ٱلْفَوْزُ ٱلْعَظِيمُ ﴿٩﴾\\
\textamh{10.\  } & إِنَّ ٱلَّذِينَ كَفَرُوا۟ يُنَادَوْنَ لَمَقْتُ ٱللَّهِ أَكْبَرُ مِن مَّقْتِكُمْ أَنفُسَكُمْ إِذْ تُدْعَوْنَ إِلَى ٱلْإِيمَـٰنِ فَتَكْفُرُونَ ﴿١٠﴾\\
\textamh{11.\  } & قَالُوا۟ رَبَّنَآ أَمَتَّنَا ٱثْنَتَيْنِ وَأَحْيَيْتَنَا ٱثْنَتَيْنِ فَٱعْتَرَفْنَا بِذُنُوبِنَا فَهَلْ إِلَىٰ خُرُوجٍۢ مِّن سَبِيلٍۢ ﴿١١﴾\\
\textamh{12.\  } & ذَٟلِكُم بِأَنَّهُۥٓ إِذَا دُعِىَ ٱللَّهُ وَحْدَهُۥ كَفَرْتُمْ ۖ وَإِن يُشْرَكْ بِهِۦ تُؤْمِنُوا۟ ۚ فَٱلْحُكْمُ لِلَّهِ ٱلْعَلِىِّ ٱلْكَبِيرِ ﴿١٢﴾\\
\textamh{13.\  } & هُوَ ٱلَّذِى يُرِيكُمْ ءَايَـٰتِهِۦ وَيُنَزِّلُ لَكُم مِّنَ ٱلسَّمَآءِ رِزْقًۭا ۚ وَمَا يَتَذَكَّرُ إِلَّا مَن يُنِيبُ ﴿١٣﴾\\
\textamh{14.\  } & فَٱدْعُوا۟ ٱللَّهَ مُخْلِصِينَ لَهُ ٱلدِّينَ وَلَوْ كَرِهَ ٱلْكَـٰفِرُونَ ﴿١٤﴾\\
\textamh{15.\  } & رَفِيعُ ٱلدَّرَجَٰتِ ذُو ٱلْعَرْشِ يُلْقِى ٱلرُّوحَ مِنْ أَمْرِهِۦ عَلَىٰ مَن يَشَآءُ مِنْ عِبَادِهِۦ لِيُنذِرَ يَوْمَ ٱلتَّلَاقِ ﴿١٥﴾\\
\textamh{16.\  } & يَوْمَ هُم بَٰرِزُونَ ۖ لَا يَخْفَىٰ عَلَى ٱللَّهِ مِنْهُمْ شَىْءٌۭ ۚ لِّمَنِ ٱلْمُلْكُ ٱلْيَوْمَ ۖ لِلَّهِ ٱلْوَٟحِدِ ٱلْقَهَّارِ ﴿١٦﴾\\
\textamh{17.\  } & ٱلْيَوْمَ تُجْزَىٰ كُلُّ نَفْسٍۭ بِمَا كَسَبَتْ ۚ لَا ظُلْمَ ٱلْيَوْمَ ۚ إِنَّ ٱللَّهَ سَرِيعُ ٱلْحِسَابِ ﴿١٧﴾\\
\textamh{18.\  } & وَأَنذِرْهُمْ يَوْمَ ٱلْءَازِفَةِ إِذِ ٱلْقُلُوبُ لَدَى ٱلْحَنَاجِرِ كَـٰظِمِينَ ۚ مَا لِلظَّـٰلِمِينَ مِنْ حَمِيمٍۢ وَلَا شَفِيعٍۢ يُطَاعُ ﴿١٨﴾\\
\textamh{19.\  } & يَعْلَمُ خَآئِنَةَ ٱلْأَعْيُنِ وَمَا تُخْفِى ٱلصُّدُورُ ﴿١٩﴾\\
\textamh{20.\  } & وَٱللَّهُ يَقْضِى بِٱلْحَقِّ ۖ وَٱلَّذِينَ يَدْعُونَ مِن دُونِهِۦ لَا يَقْضُونَ بِشَىْءٍ ۗ إِنَّ ٱللَّهَ هُوَ ٱلسَّمِيعُ ٱلْبَصِيرُ ﴿٢٠﴾\\
\textamh{21.\  } & ۞ أَوَلَمْ يَسِيرُوا۟ فِى ٱلْأَرْضِ فَيَنظُرُوا۟ كَيْفَ كَانَ عَـٰقِبَةُ ٱلَّذِينَ كَانُوا۟ مِن قَبْلِهِمْ ۚ كَانُوا۟ هُمْ أَشَدَّ مِنْهُمْ قُوَّةًۭ وَءَاثَارًۭا فِى ٱلْأَرْضِ فَأَخَذَهُمُ ٱللَّهُ بِذُنُوبِهِمْ وَمَا كَانَ لَهُم مِّنَ ٱللَّهِ مِن وَاقٍۢ ﴿٢١﴾\\
\textamh{22.\  } & ذَٟلِكَ بِأَنَّهُمْ كَانَت تَّأْتِيهِمْ رُسُلُهُم بِٱلْبَيِّنَـٰتِ فَكَفَرُوا۟ فَأَخَذَهُمُ ٱللَّهُ ۚ إِنَّهُۥ قَوِىٌّۭ شَدِيدُ ٱلْعِقَابِ ﴿٢٢﴾\\
\textamh{23.\  } & وَلَقَدْ أَرْسَلْنَا مُوسَىٰ بِـَٔايَـٰتِنَا وَسُلْطَٰنٍۢ مُّبِينٍ ﴿٢٣﴾\\
\textamh{24.\  } & إِلَىٰ فِرْعَوْنَ وَهَـٰمَـٰنَ وَقَـٰرُونَ فَقَالُوا۟ سَـٰحِرٌۭ كَذَّابٌۭ ﴿٢٤﴾\\
\textamh{25.\  } & فَلَمَّا جَآءَهُم بِٱلْحَقِّ مِنْ عِندِنَا قَالُوا۟ ٱقْتُلُوٓا۟ أَبْنَآءَ ٱلَّذِينَ ءَامَنُوا۟ مَعَهُۥ وَٱسْتَحْيُوا۟ نِسَآءَهُمْ ۚ وَمَا كَيْدُ ٱلْكَـٰفِرِينَ إِلَّا فِى ضَلَـٰلٍۢ ﴿٢٥﴾\\
\textamh{26.\  } & وَقَالَ فِرْعَوْنُ ذَرُونِىٓ أَقْتُلْ مُوسَىٰ وَلْيَدْعُ رَبَّهُۥٓ ۖ إِنِّىٓ أَخَافُ أَن يُبَدِّلَ دِينَكُمْ أَوْ أَن يُظْهِرَ فِى ٱلْأَرْضِ ٱلْفَسَادَ ﴿٢٦﴾\\
\textamh{27.\  } & وَقَالَ مُوسَىٰٓ إِنِّى عُذْتُ بِرَبِّى وَرَبِّكُم مِّن كُلِّ مُتَكَبِّرٍۢ لَّا يُؤْمِنُ بِيَوْمِ ٱلْحِسَابِ ﴿٢٧﴾\\
\textamh{28.\  } & وَقَالَ رَجُلٌۭ مُّؤْمِنٌۭ مِّنْ ءَالِ فِرْعَوْنَ يَكْتُمُ إِيمَـٰنَهُۥٓ أَتَقْتُلُونَ رَجُلًا أَن يَقُولَ رَبِّىَ ٱللَّهُ وَقَدْ جَآءَكُم بِٱلْبَيِّنَـٰتِ مِن رَّبِّكُمْ ۖ وَإِن يَكُ كَـٰذِبًۭا فَعَلَيْهِ كَذِبُهُۥ ۖ وَإِن يَكُ صَادِقًۭا يُصِبْكُم بَعْضُ ٱلَّذِى يَعِدُكُمْ ۖ إِنَّ ٱللَّهَ لَا يَهْدِى مَنْ هُوَ مُسْرِفٌۭ كَذَّابٌۭ ﴿٢٨﴾\\
\textamh{29.\  } & يَـٰقَوْمِ لَكُمُ ٱلْمُلْكُ ٱلْيَوْمَ ظَـٰهِرِينَ فِى ٱلْأَرْضِ فَمَن يَنصُرُنَا مِنۢ بَأْسِ ٱللَّهِ إِن جَآءَنَا ۚ قَالَ فِرْعَوْنُ مَآ أُرِيكُمْ إِلَّا مَآ أَرَىٰ وَمَآ أَهْدِيكُمْ إِلَّا سَبِيلَ ٱلرَّشَادِ ﴿٢٩﴾\\
\textamh{30.\  } & وَقَالَ ٱلَّذِىٓ ءَامَنَ يَـٰقَوْمِ إِنِّىٓ أَخَافُ عَلَيْكُم مِّثْلَ يَوْمِ ٱلْأَحْزَابِ ﴿٣٠﴾\\
\textamh{31.\  } & مِثْلَ دَأْبِ قَوْمِ نُوحٍۢ وَعَادٍۢ وَثَمُودَ وَٱلَّذِينَ مِنۢ بَعْدِهِمْ ۚ وَمَا ٱللَّهُ يُرِيدُ ظُلْمًۭا لِّلْعِبَادِ ﴿٣١﴾\\
\textamh{32.\  } & وَيَـٰقَوْمِ إِنِّىٓ أَخَافُ عَلَيْكُمْ يَوْمَ ٱلتَّنَادِ ﴿٣٢﴾\\
\textamh{33.\  } & يَوْمَ تُوَلُّونَ مُدْبِرِينَ مَا لَكُم مِّنَ ٱللَّهِ مِنْ عَاصِمٍۢ ۗ وَمَن يُضْلِلِ ٱللَّهُ فَمَا لَهُۥ مِنْ هَادٍۢ ﴿٣٣﴾\\
\textamh{34.\  } & وَلَقَدْ جَآءَكُمْ يُوسُفُ مِن قَبْلُ بِٱلْبَيِّنَـٰتِ فَمَا زِلْتُمْ فِى شَكٍّۢ مِّمَّا جَآءَكُم بِهِۦ ۖ حَتَّىٰٓ إِذَا هَلَكَ قُلْتُمْ لَن يَبْعَثَ ٱللَّهُ مِنۢ بَعْدِهِۦ رَسُولًۭا ۚ كَذَٟلِكَ يُضِلُّ ٱللَّهُ مَنْ هُوَ مُسْرِفٌۭ مُّرْتَابٌ ﴿٣٤﴾\\
\textamh{35.\  } & ٱلَّذِينَ يُجَٰدِلُونَ فِىٓ ءَايَـٰتِ ٱللَّهِ بِغَيْرِ سُلْطَٰنٍ أَتَىٰهُمْ ۖ كَبُرَ مَقْتًا عِندَ ٱللَّهِ وَعِندَ ٱلَّذِينَ ءَامَنُوا۟ ۚ كَذَٟلِكَ يَطْبَعُ ٱللَّهُ عَلَىٰ كُلِّ قَلْبِ مُتَكَبِّرٍۢ جَبَّارٍۢ ﴿٣٥﴾\\
\textamh{36.\  } & وَقَالَ فِرْعَوْنُ يَـٰهَـٰمَـٰنُ ٱبْنِ لِى صَرْحًۭا لَّعَلِّىٓ أَبْلُغُ ٱلْأَسْبَٰبَ ﴿٣٦﴾\\
\textamh{37.\  } & أَسْبَٰبَ ٱلسَّمَـٰوَٟتِ فَأَطَّلِعَ إِلَىٰٓ إِلَـٰهِ مُوسَىٰ وَإِنِّى لَأَظُنُّهُۥ كَـٰذِبًۭا ۚ وَكَذَٟلِكَ زُيِّنَ لِفِرْعَوْنَ سُوٓءُ عَمَلِهِۦ وَصُدَّ عَنِ ٱلسَّبِيلِ ۚ وَمَا كَيْدُ فِرْعَوْنَ إِلَّا فِى تَبَابٍۢ ﴿٣٧﴾\\
\textamh{38.\  } & وَقَالَ ٱلَّذِىٓ ءَامَنَ يَـٰقَوْمِ ٱتَّبِعُونِ أَهْدِكُمْ سَبِيلَ ٱلرَّشَادِ ﴿٣٨﴾\\
\textamh{39.\  } & يَـٰقَوْمِ إِنَّمَا هَـٰذِهِ ٱلْحَيَوٰةُ ٱلدُّنْيَا مَتَـٰعٌۭ وَإِنَّ ٱلْءَاخِرَةَ هِىَ دَارُ ٱلْقَرَارِ ﴿٣٩﴾\\
\textamh{40.\  } & مَنْ عَمِلَ سَيِّئَةًۭ فَلَا يُجْزَىٰٓ إِلَّا مِثْلَهَا ۖ وَمَنْ عَمِلَ صَـٰلِحًۭا مِّن ذَكَرٍ أَوْ أُنثَىٰ وَهُوَ مُؤْمِنٌۭ فَأُو۟لَـٰٓئِكَ يَدْخُلُونَ ٱلْجَنَّةَ يُرْزَقُونَ فِيهَا بِغَيْرِ حِسَابٍۢ ﴿٤٠﴾\\
\textamh{41.\  } & ۞ وَيَـٰقَوْمِ مَا لِىٓ أَدْعُوكُمْ إِلَى ٱلنَّجَوٰةِ وَتَدْعُونَنِىٓ إِلَى ٱلنَّارِ ﴿٤١﴾\\
\textamh{42.\  } & تَدْعُونَنِى لِأَكْفُرَ بِٱللَّهِ وَأُشْرِكَ بِهِۦ مَا لَيْسَ لِى بِهِۦ عِلْمٌۭ وَأَنَا۠ أَدْعُوكُمْ إِلَى ٱلْعَزِيزِ ٱلْغَفَّٰرِ ﴿٤٢﴾\\
\textamh{43.\  } & لَا جَرَمَ أَنَّمَا تَدْعُونَنِىٓ إِلَيْهِ لَيْسَ لَهُۥ دَعْوَةٌۭ فِى ٱلدُّنْيَا وَلَا فِى ٱلْءَاخِرَةِ وَأَنَّ مَرَدَّنَآ إِلَى ٱللَّهِ وَأَنَّ ٱلْمُسْرِفِينَ هُمْ أَصْحَـٰبُ ٱلنَّارِ ﴿٤٣﴾\\
\textamh{44.\  } & فَسَتَذْكُرُونَ مَآ أَقُولُ لَكُمْ ۚ وَأُفَوِّضُ أَمْرِىٓ إِلَى ٱللَّهِ ۚ إِنَّ ٱللَّهَ بَصِيرٌۢ بِٱلْعِبَادِ ﴿٤٤﴾\\
\textamh{45.\  } & فَوَقَىٰهُ ٱللَّهُ سَيِّـَٔاتِ مَا مَكَرُوا۟ ۖ وَحَاقَ بِـَٔالِ فِرْعَوْنَ سُوٓءُ ٱلْعَذَابِ ﴿٤٥﴾\\
\textamh{46.\  } & ٱلنَّارُ يُعْرَضُونَ عَلَيْهَا غُدُوًّۭا وَعَشِيًّۭا ۖ وَيَوْمَ تَقُومُ ٱلسَّاعَةُ أَدْخِلُوٓا۟ ءَالَ فِرْعَوْنَ أَشَدَّ ٱلْعَذَابِ ﴿٤٦﴾\\
\textamh{47.\  } & وَإِذْ يَتَحَآجُّونَ فِى ٱلنَّارِ فَيَقُولُ ٱلضُّعَفَـٰٓؤُا۟ لِلَّذِينَ ٱسْتَكْبَرُوٓا۟ إِنَّا كُنَّا لَكُمْ تَبَعًۭا فَهَلْ أَنتُم مُّغْنُونَ عَنَّا نَصِيبًۭا مِّنَ ٱلنَّارِ ﴿٤٧﴾\\
\textamh{48.\  } & قَالَ ٱلَّذِينَ ٱسْتَكْبَرُوٓا۟ إِنَّا كُلٌّۭ فِيهَآ إِنَّ ٱللَّهَ قَدْ حَكَمَ بَيْنَ ٱلْعِبَادِ ﴿٤٨﴾\\
\textamh{49.\  } & وَقَالَ ٱلَّذِينَ فِى ٱلنَّارِ لِخَزَنَةِ جَهَنَّمَ ٱدْعُوا۟ رَبَّكُمْ يُخَفِّفْ عَنَّا يَوْمًۭا مِّنَ ٱلْعَذَابِ ﴿٤٩﴾\\
\textamh{50.\  } & قَالُوٓا۟ أَوَلَمْ تَكُ تَأْتِيكُمْ رُسُلُكُم بِٱلْبَيِّنَـٰتِ ۖ قَالُوا۟ بَلَىٰ ۚ قَالُوا۟ فَٱدْعُوا۟ ۗ وَمَا دُعَـٰٓؤُا۟ ٱلْكَـٰفِرِينَ إِلَّا فِى ضَلَـٰلٍ ﴿٥٠﴾\\
\textamh{51.\  } & إِنَّا لَنَنصُرُ رُسُلَنَا وَٱلَّذِينَ ءَامَنُوا۟ فِى ٱلْحَيَوٰةِ ٱلدُّنْيَا وَيَوْمَ يَقُومُ ٱلْأَشْهَـٰدُ ﴿٥١﴾\\
\textamh{52.\  } & يَوْمَ لَا يَنفَعُ ٱلظَّـٰلِمِينَ مَعْذِرَتُهُمْ ۖ وَلَهُمُ ٱللَّعْنَةُ وَلَهُمْ سُوٓءُ ٱلدَّارِ ﴿٥٢﴾\\
\textamh{53.\  } & وَلَقَدْ ءَاتَيْنَا مُوسَى ٱلْهُدَىٰ وَأَوْرَثْنَا بَنِىٓ إِسْرَٰٓءِيلَ ٱلْكِتَـٰبَ ﴿٥٣﴾\\
\textamh{54.\  } & هُدًۭى وَذِكْرَىٰ لِأُو۟لِى ٱلْأَلْبَٰبِ ﴿٥٤﴾\\
\textamh{55.\  } & فَٱصْبِرْ إِنَّ وَعْدَ ٱللَّهِ حَقٌّۭ وَٱسْتَغْفِرْ لِذَنۢبِكَ وَسَبِّحْ بِحَمْدِ رَبِّكَ بِٱلْعَشِىِّ وَٱلْإِبْكَـٰرِ ﴿٥٥﴾\\
\textamh{56.\  } & إِنَّ ٱلَّذِينَ يُجَٰدِلُونَ فِىٓ ءَايَـٰتِ ٱللَّهِ بِغَيْرِ سُلْطَٰنٍ أَتَىٰهُمْ ۙ إِن فِى صُدُورِهِمْ إِلَّا كِبْرٌۭ مَّا هُم بِبَٰلِغِيهِ ۚ فَٱسْتَعِذْ بِٱللَّهِ ۖ إِنَّهُۥ هُوَ ٱلسَّمِيعُ ٱلْبَصِيرُ ﴿٥٦﴾\\
\textamh{57.\  } & لَخَلْقُ ٱلسَّمَـٰوَٟتِ وَٱلْأَرْضِ أَكْبَرُ مِنْ خَلْقِ ٱلنَّاسِ وَلَـٰكِنَّ أَكْثَرَ ٱلنَّاسِ لَا يَعْلَمُونَ ﴿٥٧﴾\\
\textamh{58.\  } & وَمَا يَسْتَوِى ٱلْأَعْمَىٰ وَٱلْبَصِيرُ وَٱلَّذِينَ ءَامَنُوا۟ وَعَمِلُوا۟ ٱلصَّـٰلِحَـٰتِ وَلَا ٱلْمُسِىٓءُ ۚ قَلِيلًۭا مَّا تَتَذَكَّرُونَ ﴿٥٨﴾\\
\textamh{59.\  } & إِنَّ ٱلسَّاعَةَ لَءَاتِيَةٌۭ لَّا رَيْبَ فِيهَا وَلَـٰكِنَّ أَكْثَرَ ٱلنَّاسِ لَا يُؤْمِنُونَ ﴿٥٩﴾\\
\textamh{60.\  } & وَقَالَ رَبُّكُمُ ٱدْعُونِىٓ أَسْتَجِبْ لَكُمْ ۚ إِنَّ ٱلَّذِينَ يَسْتَكْبِرُونَ عَنْ عِبَادَتِى سَيَدْخُلُونَ جَهَنَّمَ دَاخِرِينَ ﴿٦٠﴾\\
\textamh{61.\  } & ٱللَّهُ ٱلَّذِى جَعَلَ لَكُمُ ٱلَّيْلَ لِتَسْكُنُوا۟ فِيهِ وَٱلنَّهَارَ مُبْصِرًا ۚ إِنَّ ٱللَّهَ لَذُو فَضْلٍ عَلَى ٱلنَّاسِ وَلَـٰكِنَّ أَكْثَرَ ٱلنَّاسِ لَا يَشْكُرُونَ ﴿٦١﴾\\
\textamh{62.\  } & ذَٟلِكُمُ ٱللَّهُ رَبُّكُمْ خَـٰلِقُ كُلِّ شَىْءٍۢ لَّآ إِلَـٰهَ إِلَّا هُوَ ۖ فَأَنَّىٰ تُؤْفَكُونَ ﴿٦٢﴾\\
\textamh{63.\  } & كَذَٟلِكَ يُؤْفَكُ ٱلَّذِينَ كَانُوا۟ بِـَٔايَـٰتِ ٱللَّهِ يَجْحَدُونَ ﴿٦٣﴾\\
\textamh{64.\  } & ٱللَّهُ ٱلَّذِى جَعَلَ لَكُمُ ٱلْأَرْضَ قَرَارًۭا وَٱلسَّمَآءَ بِنَآءًۭ وَصَوَّرَكُمْ فَأَحْسَنَ صُوَرَكُمْ وَرَزَقَكُم مِّنَ ٱلطَّيِّبَٰتِ ۚ ذَٟلِكُمُ ٱللَّهُ رَبُّكُمْ ۖ فَتَبَارَكَ ٱللَّهُ رَبُّ ٱلْعَـٰلَمِينَ ﴿٦٤﴾\\
\textamh{65.\  } & هُوَ ٱلْحَىُّ لَآ إِلَـٰهَ إِلَّا هُوَ فَٱدْعُوهُ مُخْلِصِينَ لَهُ ٱلدِّينَ ۗ ٱلْحَمْدُ لِلَّهِ رَبِّ ٱلْعَـٰلَمِينَ ﴿٦٥﴾\\
\textamh{66.\  } & ۞ قُلْ إِنِّى نُهِيتُ أَنْ أَعْبُدَ ٱلَّذِينَ تَدْعُونَ مِن دُونِ ٱللَّهِ لَمَّا جَآءَنِىَ ٱلْبَيِّنَـٰتُ مِن رَّبِّى وَأُمِرْتُ أَنْ أُسْلِمَ لِرَبِّ ٱلْعَـٰلَمِينَ ﴿٦٦﴾\\
\textamh{67.\  } & هُوَ ٱلَّذِى خَلَقَكُم مِّن تُرَابٍۢ ثُمَّ مِن نُّطْفَةٍۢ ثُمَّ مِنْ عَلَقَةٍۢ ثُمَّ يُخْرِجُكُمْ طِفْلًۭا ثُمَّ لِتَبْلُغُوٓا۟ أَشُدَّكُمْ ثُمَّ لِتَكُونُوا۟ شُيُوخًۭا ۚ وَمِنكُم مَّن يُتَوَفَّىٰ مِن قَبْلُ ۖ وَلِتَبْلُغُوٓا۟ أَجَلًۭا مُّسَمًّۭى وَلَعَلَّكُمْ تَعْقِلُونَ ﴿٦٧﴾\\
\textamh{68.\  } & هُوَ ٱلَّذِى يُحْىِۦ وَيُمِيتُ ۖ فَإِذَا قَضَىٰٓ أَمْرًۭا فَإِنَّمَا يَقُولُ لَهُۥ كُن فَيَكُونُ ﴿٦٨﴾\\
\textamh{69.\  } & أَلَمْ تَرَ إِلَى ٱلَّذِينَ يُجَٰدِلُونَ فِىٓ ءَايَـٰتِ ٱللَّهِ أَنَّىٰ يُصْرَفُونَ ﴿٦٩﴾\\
\textamh{70.\  } & ٱلَّذِينَ كَذَّبُوا۟ بِٱلْكِتَـٰبِ وَبِمَآ أَرْسَلْنَا بِهِۦ رُسُلَنَا ۖ فَسَوْفَ يَعْلَمُونَ ﴿٧٠﴾\\
\textamh{71.\  } & إِذِ ٱلْأَغْلَـٰلُ فِىٓ أَعْنَـٰقِهِمْ وَٱلسَّلَـٰسِلُ يُسْحَبُونَ ﴿٧١﴾\\
\textamh{72.\  } & فِى ٱلْحَمِيمِ ثُمَّ فِى ٱلنَّارِ يُسْجَرُونَ ﴿٧٢﴾\\
\textamh{73.\  } & ثُمَّ قِيلَ لَهُمْ أَيْنَ مَا كُنتُمْ تُشْرِكُونَ ﴿٧٣﴾\\
\textamh{74.\  } & مِن دُونِ ٱللَّهِ ۖ قَالُوا۟ ضَلُّوا۟ عَنَّا بَل لَّمْ نَكُن نَّدْعُوا۟ مِن قَبْلُ شَيْـًۭٔا ۚ كَذَٟلِكَ يُضِلُّ ٱللَّهُ ٱلْكَـٰفِرِينَ ﴿٧٤﴾\\
\textamh{75.\  } & ذَٟلِكُم بِمَا كُنتُمْ تَفْرَحُونَ فِى ٱلْأَرْضِ بِغَيْرِ ٱلْحَقِّ وَبِمَا كُنتُمْ تَمْرَحُونَ ﴿٧٥﴾\\
\textamh{76.\  } & ٱدْخُلُوٓا۟ أَبْوَٟبَ جَهَنَّمَ خَـٰلِدِينَ فِيهَا ۖ فَبِئْسَ مَثْوَى ٱلْمُتَكَبِّرِينَ ﴿٧٦﴾\\
\textamh{77.\  } & فَٱصْبِرْ إِنَّ وَعْدَ ٱللَّهِ حَقٌّۭ ۚ فَإِمَّا نُرِيَنَّكَ بَعْضَ ٱلَّذِى نَعِدُهُمْ أَوْ نَتَوَفَّيَنَّكَ فَإِلَيْنَا يُرْجَعُونَ ﴿٧٧﴾\\
\textamh{78.\  } & وَلَقَدْ أَرْسَلْنَا رُسُلًۭا مِّن قَبْلِكَ مِنْهُم مَّن قَصَصْنَا عَلَيْكَ وَمِنْهُم مَّن لَّمْ نَقْصُصْ عَلَيْكَ ۗ وَمَا كَانَ لِرَسُولٍ أَن يَأْتِىَ بِـَٔايَةٍ إِلَّا بِإِذْنِ ٱللَّهِ ۚ فَإِذَا جَآءَ أَمْرُ ٱللَّهِ قُضِىَ بِٱلْحَقِّ وَخَسِرَ هُنَالِكَ ٱلْمُبْطِلُونَ ﴿٧٨﴾\\
\textamh{79.\  } & ٱللَّهُ ٱلَّذِى جَعَلَ لَكُمُ ٱلْأَنْعَـٰمَ لِتَرْكَبُوا۟ مِنْهَا وَمِنْهَا تَأْكُلُونَ ﴿٧٩﴾\\
\textamh{80.\  } & وَلَكُمْ فِيهَا مَنَـٰفِعُ وَلِتَبْلُغُوا۟ عَلَيْهَا حَاجَةًۭ فِى صُدُورِكُمْ وَعَلَيْهَا وَعَلَى ٱلْفُلْكِ تُحْمَلُونَ ﴿٨٠﴾\\
\textamh{81.\  } & وَيُرِيكُمْ ءَايَـٰتِهِۦ فَأَىَّ ءَايَـٰتِ ٱللَّهِ تُنكِرُونَ ﴿٨١﴾\\
\textamh{82.\  } & أَفَلَمْ يَسِيرُوا۟ فِى ٱلْأَرْضِ فَيَنظُرُوا۟ كَيْفَ كَانَ عَـٰقِبَةُ ٱلَّذِينَ مِن قَبْلِهِمْ ۚ كَانُوٓا۟ أَكْثَرَ مِنْهُمْ وَأَشَدَّ قُوَّةًۭ وَءَاثَارًۭا فِى ٱلْأَرْضِ فَمَآ أَغْنَىٰ عَنْهُم مَّا كَانُوا۟ يَكْسِبُونَ ﴿٨٢﴾\\
\textamh{83.\  } & فَلَمَّا جَآءَتْهُمْ رُسُلُهُم بِٱلْبَيِّنَـٰتِ فَرِحُوا۟ بِمَا عِندَهُم مِّنَ ٱلْعِلْمِ وَحَاقَ بِهِم مَّا كَانُوا۟ بِهِۦ يَسْتَهْزِءُونَ ﴿٨٣﴾\\
\textamh{84.\  } & فَلَمَّا رَأَوْا۟ بَأْسَنَا قَالُوٓا۟ ءَامَنَّا بِٱللَّهِ وَحْدَهُۥ وَكَفَرْنَا بِمَا كُنَّا بِهِۦ مُشْرِكِينَ ﴿٨٤﴾\\
\textamh{85.\  } & فَلَمْ يَكُ يَنفَعُهُمْ إِيمَـٰنُهُمْ لَمَّا رَأَوْا۟ بَأْسَنَا ۖ سُنَّتَ ٱللَّهِ ٱلَّتِى قَدْ خَلَتْ فِى عِبَادِهِۦ ۖ وَخَسِرَ هُنَالِكَ ٱلْكَـٰفِرُونَ ﴿٨٥﴾\\
\end{longtable} \newpage

%% License: BSD style (Berkley) (i.e. Put the Copyright owner's name always)
%% Writer and Copyright (to): Bewketu(Bilal) Tadilo (2016-17)
\centering\section{\LR{\textamharic{ሱራቱ ፉሲላት -}  \RL{سوره  فصلت}}}
\begin{longtable}{%
  @{}
    p{.5\textwidth}
  @{~~~~~~~~~~~~~}
    p{.5\textwidth}
    @{}
}
\nopagebreak
\textamh{ቢስሚላሂ አራህመኒ ራሂይም } &  بِسْمِ ٱللَّهِ ٱلرَّحْمَـٰنِ ٱلرَّحِيمِ\\
\textamh{1.\  } &  حمٓ ﴿١﴾\\
\textamh{2.\  } & تَنزِيلٌۭ مِّنَ ٱلرَّحْمَـٰنِ ٱلرَّحِيمِ ﴿٢﴾\\
\textamh{3.\  } & كِتَـٰبٌۭ فُصِّلَتْ ءَايَـٰتُهُۥ قُرْءَانًا عَرَبِيًّۭا لِّقَوْمٍۢ يَعْلَمُونَ ﴿٣﴾\\
\textamh{4.\  } & بَشِيرًۭا وَنَذِيرًۭا فَأَعْرَضَ أَكْثَرُهُمْ فَهُمْ لَا يَسْمَعُونَ ﴿٤﴾\\
\textamh{5.\  } & وَقَالُوا۟ قُلُوبُنَا فِىٓ أَكِنَّةٍۢ مِّمَّا تَدْعُونَآ إِلَيْهِ وَفِىٓ ءَاذَانِنَا وَقْرٌۭ وَمِنۢ بَيْنِنَا وَبَيْنِكَ حِجَابٌۭ فَٱعْمَلْ إِنَّنَا عَـٰمِلُونَ ﴿٥﴾\\
\textamh{6.\  } & قُلْ إِنَّمَآ أَنَا۠ بَشَرٌۭ مِّثْلُكُمْ يُوحَىٰٓ إِلَىَّ أَنَّمَآ إِلَـٰهُكُمْ إِلَـٰهٌۭ وَٟحِدٌۭ فَٱسْتَقِيمُوٓا۟ إِلَيْهِ وَٱسْتَغْفِرُوهُ ۗ وَوَيْلٌۭ لِّلْمُشْرِكِينَ ﴿٦﴾\\
\textamh{7.\  } & ٱلَّذِينَ لَا يُؤْتُونَ ٱلزَّكَوٰةَ وَهُم بِٱلْءَاخِرَةِ هُمْ كَـٰفِرُونَ ﴿٧﴾\\
\textamh{8.\  } & إِنَّ ٱلَّذِينَ ءَامَنُوا۟ وَعَمِلُوا۟ ٱلصَّـٰلِحَـٰتِ لَهُمْ أَجْرٌ غَيْرُ مَمْنُونٍۢ ﴿٨﴾\\
\textamh{9.\  } & ۞ قُلْ أَئِنَّكُمْ لَتَكْفُرُونَ بِٱلَّذِى خَلَقَ ٱلْأَرْضَ فِى يَوْمَيْنِ وَتَجْعَلُونَ لَهُۥٓ أَندَادًۭا ۚ ذَٟلِكَ رَبُّ ٱلْعَـٰلَمِينَ ﴿٩﴾\\
\textamh{10.\  } & وَجَعَلَ فِيهَا رَوَٟسِىَ مِن فَوْقِهَا وَبَٰرَكَ فِيهَا وَقَدَّرَ فِيهَآ أَقْوَٟتَهَا فِىٓ أَرْبَعَةِ أَيَّامٍۢ سَوَآءًۭ لِّلسَّآئِلِينَ ﴿١٠﴾\\
\textamh{11.\  } & ثُمَّ ٱسْتَوَىٰٓ إِلَى ٱلسَّمَآءِ وَهِىَ دُخَانٌۭ فَقَالَ لَهَا وَلِلْأَرْضِ ٱئْتِيَا طَوْعًا أَوْ كَرْهًۭا قَالَتَآ أَتَيْنَا طَآئِعِينَ ﴿١١﴾\\
\textamh{12.\  } & فَقَضَىٰهُنَّ سَبْعَ سَمَـٰوَاتٍۢ فِى يَوْمَيْنِ وَأَوْحَىٰ فِى كُلِّ سَمَآءٍ أَمْرَهَا ۚ وَزَيَّنَّا ٱلسَّمَآءَ ٱلدُّنْيَا بِمَصَـٰبِيحَ وَحِفْظًۭا ۚ ذَٟلِكَ تَقْدِيرُ ٱلْعَزِيزِ ٱلْعَلِيمِ ﴿١٢﴾\\
\textamh{13.\  } & فَإِنْ أَعْرَضُوا۟ فَقُلْ أَنذَرْتُكُمْ صَـٰعِقَةًۭ مِّثْلَ صَـٰعِقَةِ عَادٍۢ وَثَمُودَ ﴿١٣﴾\\
\textamh{14.\  } & إِذْ جَآءَتْهُمُ ٱلرُّسُلُ مِنۢ بَيْنِ أَيْدِيهِمْ وَمِنْ خَلْفِهِمْ أَلَّا تَعْبُدُوٓا۟ إِلَّا ٱللَّهَ ۖ قَالُوا۟ لَوْ شَآءَ رَبُّنَا لَأَنزَلَ مَلَـٰٓئِكَةًۭ فَإِنَّا بِمَآ أُرْسِلْتُم بِهِۦ كَـٰفِرُونَ ﴿١٤﴾\\
\textamh{15.\  } & فَأَمَّا عَادٌۭ فَٱسْتَكْبَرُوا۟ فِى ٱلْأَرْضِ بِغَيْرِ ٱلْحَقِّ وَقَالُوا۟ مَنْ أَشَدُّ مِنَّا قُوَّةً ۖ أَوَلَمْ يَرَوْا۟ أَنَّ ٱللَّهَ ٱلَّذِى خَلَقَهُمْ هُوَ أَشَدُّ مِنْهُمْ قُوَّةًۭ ۖ وَكَانُوا۟ بِـَٔايَـٰتِنَا يَجْحَدُونَ ﴿١٥﴾\\
\textamh{16.\  } & فَأَرْسَلْنَا عَلَيْهِمْ رِيحًۭا صَرْصَرًۭا فِىٓ أَيَّامٍۢ نَّحِسَاتٍۢ لِّنُذِيقَهُمْ عَذَابَ ٱلْخِزْىِ فِى ٱلْحَيَوٰةِ ٱلدُّنْيَا ۖ وَلَعَذَابُ ٱلْءَاخِرَةِ أَخْزَىٰ ۖ وَهُمْ لَا يُنصَرُونَ ﴿١٦﴾\\
\textamh{17.\  } & وَأَمَّا ثَمُودُ فَهَدَيْنَـٰهُمْ فَٱسْتَحَبُّوا۟ ٱلْعَمَىٰ عَلَى ٱلْهُدَىٰ فَأَخَذَتْهُمْ صَـٰعِقَةُ ٱلْعَذَابِ ٱلْهُونِ بِمَا كَانُوا۟ يَكْسِبُونَ ﴿١٧﴾\\
\textamh{18.\  } & وَنَجَّيْنَا ٱلَّذِينَ ءَامَنُوا۟ وَكَانُوا۟ يَتَّقُونَ ﴿١٨﴾\\
\textamh{19.\  } & وَيَوْمَ يُحْشَرُ أَعْدَآءُ ٱللَّهِ إِلَى ٱلنَّارِ فَهُمْ يُوزَعُونَ ﴿١٩﴾\\
\textamh{20.\  } & حَتَّىٰٓ إِذَا مَا جَآءُوهَا شَهِدَ عَلَيْهِمْ سَمْعُهُمْ وَأَبْصَـٰرُهُمْ وَجُلُودُهُم بِمَا كَانُوا۟ يَعْمَلُونَ ﴿٢٠﴾\\
\textamh{21.\  } & وَقَالُوا۟ لِجُلُودِهِمْ لِمَ شَهِدتُّمْ عَلَيْنَا ۖ قَالُوٓا۟ أَنطَقَنَا ٱللَّهُ ٱلَّذِىٓ أَنطَقَ كُلَّ شَىْءٍۢ وَهُوَ خَلَقَكُمْ أَوَّلَ مَرَّةٍۢ وَإِلَيْهِ تُرْجَعُونَ ﴿٢١﴾\\
\textamh{22.\  } & وَمَا كُنتُمْ تَسْتَتِرُونَ أَن يَشْهَدَ عَلَيْكُمْ سَمْعُكُمْ وَلَآ أَبْصَـٰرُكُمْ وَلَا جُلُودُكُمْ وَلَـٰكِن ظَنَنتُمْ أَنَّ ٱللَّهَ لَا يَعْلَمُ كَثِيرًۭا مِّمَّا تَعْمَلُونَ ﴿٢٢﴾\\
\textamh{23.\  } & وَذَٟلِكُمْ ظَنُّكُمُ ٱلَّذِى ظَنَنتُم بِرَبِّكُمْ أَرْدَىٰكُمْ فَأَصْبَحْتُم مِّنَ ٱلْخَـٰسِرِينَ ﴿٢٣﴾\\
\textamh{24.\  } & فَإِن يَصْبِرُوا۟ فَٱلنَّارُ مَثْوًۭى لَّهُمْ ۖ وَإِن يَسْتَعْتِبُوا۟ فَمَا هُم مِّنَ ٱلْمُعْتَبِينَ ﴿٢٤﴾\\
\textamh{25.\  } & ۞ وَقَيَّضْنَا لَهُمْ قُرَنَآءَ فَزَيَّنُوا۟ لَهُم مَّا بَيْنَ أَيْدِيهِمْ وَمَا خَلْفَهُمْ وَحَقَّ عَلَيْهِمُ ٱلْقَوْلُ فِىٓ أُمَمٍۢ قَدْ خَلَتْ مِن قَبْلِهِم مِّنَ ٱلْجِنِّ وَٱلْإِنسِ ۖ إِنَّهُمْ كَانُوا۟ خَـٰسِرِينَ ﴿٢٥﴾\\
\textamh{26.\  } & وَقَالَ ٱلَّذِينَ كَفَرُوا۟ لَا تَسْمَعُوا۟ لِهَـٰذَا ٱلْقُرْءَانِ وَٱلْغَوْا۟ فِيهِ لَعَلَّكُمْ تَغْلِبُونَ ﴿٢٦﴾\\
\textamh{27.\  } & فَلَنُذِيقَنَّ ٱلَّذِينَ كَفَرُوا۟ عَذَابًۭا شَدِيدًۭا وَلَنَجْزِيَنَّهُمْ أَسْوَأَ ٱلَّذِى كَانُوا۟ يَعْمَلُونَ ﴿٢٧﴾\\
\textamh{28.\  } & ذَٟلِكَ جَزَآءُ أَعْدَآءِ ٱللَّهِ ٱلنَّارُ ۖ لَهُمْ فِيهَا دَارُ ٱلْخُلْدِ ۖ جَزَآءًۢ بِمَا كَانُوا۟ بِـَٔايَـٰتِنَا يَجْحَدُونَ ﴿٢٨﴾\\
\textamh{29.\  } & وَقَالَ ٱلَّذِينَ كَفَرُوا۟ رَبَّنَآ أَرِنَا ٱلَّذَيْنِ أَضَلَّانَا مِنَ ٱلْجِنِّ وَٱلْإِنسِ نَجْعَلْهُمَا تَحْتَ أَقْدَامِنَا لِيَكُونَا مِنَ ٱلْأَسْفَلِينَ ﴿٢٩﴾\\
\textamh{30.\  } & إِنَّ ٱلَّذِينَ قَالُوا۟ رَبُّنَا ٱللَّهُ ثُمَّ ٱسْتَقَـٰمُوا۟ تَتَنَزَّلُ عَلَيْهِمُ ٱلْمَلَـٰٓئِكَةُ أَلَّا تَخَافُوا۟ وَلَا تَحْزَنُوا۟ وَأَبْشِرُوا۟ بِٱلْجَنَّةِ ٱلَّتِى كُنتُمْ تُوعَدُونَ ﴿٣٠﴾\\
\textamh{31.\  } & نَحْنُ أَوْلِيَآؤُكُمْ فِى ٱلْحَيَوٰةِ ٱلدُّنْيَا وَفِى ٱلْءَاخِرَةِ ۖ وَلَكُمْ فِيهَا مَا تَشْتَهِىٓ أَنفُسُكُمْ وَلَكُمْ فِيهَا مَا تَدَّعُونَ ﴿٣١﴾\\
\textamh{32.\  } & نُزُلًۭا مِّنْ غَفُورٍۢ رَّحِيمٍۢ ﴿٣٢﴾\\
\textamh{33.\  } & وَمَنْ أَحْسَنُ قَوْلًۭا مِّمَّن دَعَآ إِلَى ٱللَّهِ وَعَمِلَ صَـٰلِحًۭا وَقَالَ إِنَّنِى مِنَ ٱلْمُسْلِمِينَ ﴿٣٣﴾\\
\textamh{34.\  } & وَلَا تَسْتَوِى ٱلْحَسَنَةُ وَلَا ٱلسَّيِّئَةُ ۚ ٱدْفَعْ بِٱلَّتِى هِىَ أَحْسَنُ فَإِذَا ٱلَّذِى بَيْنَكَ وَبَيْنَهُۥ عَدَٟوَةٌۭ كَأَنَّهُۥ وَلِىٌّ حَمِيمٌۭ ﴿٣٤﴾\\
\textamh{35.\  } & وَمَا يُلَقَّىٰهَآ إِلَّا ٱلَّذِينَ صَبَرُوا۟ وَمَا يُلَقَّىٰهَآ إِلَّا ذُو حَظٍّ عَظِيمٍۢ ﴿٣٥﴾\\
\textamh{36.\  } & وَإِمَّا يَنزَغَنَّكَ مِنَ ٱلشَّيْطَٰنِ نَزْغٌۭ فَٱسْتَعِذْ بِٱللَّهِ ۖ إِنَّهُۥ هُوَ ٱلسَّمِيعُ ٱلْعَلِيمُ ﴿٣٦﴾\\
\textamh{37.\  } & وَمِنْ ءَايَـٰتِهِ ٱلَّيْلُ وَٱلنَّهَارُ وَٱلشَّمْسُ وَٱلْقَمَرُ ۚ لَا تَسْجُدُوا۟ لِلشَّمْسِ وَلَا لِلْقَمَرِ وَٱسْجُدُوا۟ لِلَّهِ ٱلَّذِى خَلَقَهُنَّ إِن كُنتُمْ إِيَّاهُ تَعْبُدُونَ ﴿٣٧﴾\\
\textamh{38.\  } & فَإِنِ ٱسْتَكْبَرُوا۟ فَٱلَّذِينَ عِندَ رَبِّكَ يُسَبِّحُونَ لَهُۥ بِٱلَّيْلِ وَٱلنَّهَارِ وَهُمْ لَا يَسْـَٔمُونَ ۩ ﴿٣٨﴾\\
\textamh{39.\  } & وَمِنْ ءَايَـٰتِهِۦٓ أَنَّكَ تَرَى ٱلْأَرْضَ خَـٰشِعَةًۭ فَإِذَآ أَنزَلْنَا عَلَيْهَا ٱلْمَآءَ ٱهْتَزَّتْ وَرَبَتْ ۚ إِنَّ ٱلَّذِىٓ أَحْيَاهَا لَمُحْىِ ٱلْمَوْتَىٰٓ ۚ إِنَّهُۥ عَلَىٰ كُلِّ شَىْءٍۢ قَدِيرٌ ﴿٣٩﴾\\
\textamh{40.\  } & إِنَّ ٱلَّذِينَ يُلْحِدُونَ فِىٓ ءَايَـٰتِنَا لَا يَخْفَوْنَ عَلَيْنَآ ۗ أَفَمَن يُلْقَىٰ فِى ٱلنَّارِ خَيْرٌ أَم مَّن يَأْتِىٓ ءَامِنًۭا يَوْمَ ٱلْقِيَـٰمَةِ ۚ ٱعْمَلُوا۟ مَا شِئْتُمْ ۖ إِنَّهُۥ بِمَا تَعْمَلُونَ بَصِيرٌ ﴿٤٠﴾\\
\textamh{41.\  } & إِنَّ ٱلَّذِينَ كَفَرُوا۟ بِٱلذِّكْرِ لَمَّا جَآءَهُمْ ۖ وَإِنَّهُۥ لَكِتَـٰبٌ عَزِيزٌۭ ﴿٤١﴾\\
\textamh{42.\  } & لَّا يَأْتِيهِ ٱلْبَٰطِلُ مِنۢ بَيْنِ يَدَيْهِ وَلَا مِنْ خَلْفِهِۦ ۖ تَنزِيلٌۭ مِّنْ حَكِيمٍ حَمِيدٍۢ ﴿٤٢﴾\\
\textamh{43.\  } & مَّا يُقَالُ لَكَ إِلَّا مَا قَدْ قِيلَ لِلرُّسُلِ مِن قَبْلِكَ ۚ إِنَّ رَبَّكَ لَذُو مَغْفِرَةٍۢ وَذُو عِقَابٍ أَلِيمٍۢ ﴿٤٣﴾\\
\textamh{44.\  } & وَلَوْ جَعَلْنَـٰهُ قُرْءَانًا أَعْجَمِيًّۭا لَّقَالُوا۟ لَوْلَا فُصِّلَتْ ءَايَـٰتُهُۥٓ ۖ ءَا۬عْجَمِىٌّۭ وَعَرَبِىٌّۭ ۗ قُلْ هُوَ لِلَّذِينَ ءَامَنُوا۟ هُدًۭى وَشِفَآءٌۭ ۖ وَٱلَّذِينَ لَا يُؤْمِنُونَ فِىٓ ءَاذَانِهِمْ وَقْرٌۭ وَهُوَ عَلَيْهِمْ عَمًى ۚ أُو۟لَـٰٓئِكَ يُنَادَوْنَ مِن مَّكَانٍۭ بَعِيدٍۢ ﴿٤٤﴾\\
\textamh{45.\  } & وَلَقَدْ ءَاتَيْنَا مُوسَى ٱلْكِتَـٰبَ فَٱخْتُلِفَ فِيهِ ۗ وَلَوْلَا كَلِمَةٌۭ سَبَقَتْ مِن رَّبِّكَ لَقُضِىَ بَيْنَهُمْ ۚ وَإِنَّهُمْ لَفِى شَكٍّۢ مِّنْهُ مُرِيبٍۢ ﴿٤٥﴾\\
\textamh{46.\  } & مَّنْ عَمِلَ صَـٰلِحًۭا فَلِنَفْسِهِۦ ۖ وَمَنْ أَسَآءَ فَعَلَيْهَا ۗ وَمَا رَبُّكَ بِظَلَّٰمٍۢ لِّلْعَبِيدِ ﴿٤٦﴾\\
\textamh{47.\  } & ۞ إِلَيْهِ يُرَدُّ عِلْمُ ٱلسَّاعَةِ ۚ وَمَا تَخْرُجُ مِن ثَمَرَٰتٍۢ مِّنْ أَكْمَامِهَا وَمَا تَحْمِلُ مِنْ أُنثَىٰ وَلَا تَضَعُ إِلَّا بِعِلْمِهِۦ ۚ وَيَوْمَ يُنَادِيهِمْ أَيْنَ شُرَكَآءِى قَالُوٓا۟ ءَاذَنَّـٰكَ مَا مِنَّا مِن شَهِيدٍۢ ﴿٤٧﴾\\
\textamh{48.\  } & وَضَلَّ عَنْهُم مَّا كَانُوا۟ يَدْعُونَ مِن قَبْلُ ۖ وَظَنُّوا۟ مَا لَهُم مِّن مَّحِيصٍۢ ﴿٤٨﴾\\
\textamh{49.\  } & لَّا يَسْـَٔمُ ٱلْإِنسَـٰنُ مِن دُعَآءِ ٱلْخَيْرِ وَإِن مَّسَّهُ ٱلشَّرُّ فَيَـُٔوسٌۭ قَنُوطٌۭ ﴿٤٩﴾\\
\textamh{50.\  } & وَلَئِنْ أَذَقْنَـٰهُ رَحْمَةًۭ مِّنَّا مِنۢ بَعْدِ ضَرَّآءَ مَسَّتْهُ لَيَقُولَنَّ هَـٰذَا لِى وَمَآ أَظُنُّ ٱلسَّاعَةَ قَآئِمَةًۭ وَلَئِن رُّجِعْتُ إِلَىٰ رَبِّىٓ إِنَّ لِى عِندَهُۥ لَلْحُسْنَىٰ ۚ فَلَنُنَبِّئَنَّ ٱلَّذِينَ كَفَرُوا۟ بِمَا عَمِلُوا۟ وَلَنُذِيقَنَّهُم مِّنْ عَذَابٍ غَلِيظٍۢ ﴿٥٠﴾\\
\textamh{51.\  } & وَإِذَآ أَنْعَمْنَا عَلَى ٱلْإِنسَـٰنِ أَعْرَضَ وَنَـَٔا بِجَانِبِهِۦ وَإِذَا مَسَّهُ ٱلشَّرُّ فَذُو دُعَآءٍ عَرِيضٍۢ ﴿٥١﴾\\
\textamh{52.\  } & قُلْ أَرَءَيْتُمْ إِن كَانَ مِنْ عِندِ ٱللَّهِ ثُمَّ كَفَرْتُم بِهِۦ مَنْ أَضَلُّ مِمَّنْ هُوَ فِى شِقَاقٍۭ بَعِيدٍۢ ﴿٥٢﴾\\
\textamh{53.\  } & سَنُرِيهِمْ ءَايَـٰتِنَا فِى ٱلْءَافَاقِ وَفِىٓ أَنفُسِهِمْ حَتَّىٰ يَتَبَيَّنَ لَهُمْ أَنَّهُ ٱلْحَقُّ ۗ أَوَلَمْ يَكْفِ بِرَبِّكَ أَنَّهُۥ عَلَىٰ كُلِّ شَىْءٍۢ شَهِيدٌ ﴿٥٣﴾\\
\textamh{54.\  } & أَلَآ إِنَّهُمْ فِى مِرْيَةٍۢ مِّن لِّقَآءِ رَبِّهِمْ ۗ أَلَآ إِنَّهُۥ بِكُلِّ شَىْءٍۢ مُّحِيطٌۢ ﴿٥٤﴾\\
\end{longtable}
\clearpage
%% License: BSD style (Berkley) (i.e. Put the Copyright owner's name always)
%% Writer and Copyright (to): Bewketu(Bilal) Tadilo (2016-17)
\centering\section{\LR{\textamharic{ሱራቱ አሽሹራ -}  \RL{سوره  الشورى}}}
\begin{longtable}{%
  @{}
    p{.5\textwidth}
  @{~~~~~~~~~~~~~}
    p{.5\textwidth}
    @{}
}
\nopagebreak
\textamh{\ \ \ \ \ \  ቢስሚላሂ አራህመኒ ራሂይም } &  بِسْمِ ٱللَّهِ ٱلرَّحْمَـٰنِ ٱلرَّحِيمِ\\
\textamh{1.\  } &  حمٓ ﴿١﴾\\
\textamh{2.\  } & عٓسٓقٓ ﴿٢﴾\\
\textamh{3.\  } & كَذَٟلِكَ يُوحِىٓ إِلَيْكَ وَإِلَى ٱلَّذِينَ مِن قَبْلِكَ ٱللَّهُ ٱلْعَزِيزُ ٱلْحَكِيمُ ﴿٣﴾\\
\textamh{4.\  } & لَهُۥ مَا فِى ٱلسَّمَـٰوَٟتِ وَمَا فِى ٱلْأَرْضِ ۖ وَهُوَ ٱلْعَلِىُّ ٱلْعَظِيمُ ﴿٤﴾\\
\textamh{5.\  } & تَكَادُ ٱلسَّمَـٰوَٟتُ يَتَفَطَّرْنَ مِن فَوْقِهِنَّ ۚ وَٱلْمَلَـٰٓئِكَةُ يُسَبِّحُونَ بِحَمْدِ رَبِّهِمْ وَيَسْتَغْفِرُونَ لِمَن فِى ٱلْأَرْضِ ۗ أَلَآ إِنَّ ٱللَّهَ هُوَ ٱلْغَفُورُ ٱلرَّحِيمُ ﴿٥﴾\\
\textamh{6.\  } & وَٱلَّذِينَ ٱتَّخَذُوا۟ مِن دُونِهِۦٓ أَوْلِيَآءَ ٱللَّهُ حَفِيظٌ عَلَيْهِمْ وَمَآ أَنتَ عَلَيْهِم بِوَكِيلٍۢ ﴿٦﴾\\
\textamh{7.\  } & وَكَذَٟلِكَ أَوْحَيْنَآ إِلَيْكَ قُرْءَانًا عَرَبِيًّۭا لِّتُنذِرَ أُمَّ ٱلْقُرَىٰ وَمَنْ حَوْلَهَا وَتُنذِرَ يَوْمَ ٱلْجَمْعِ لَا رَيْبَ فِيهِ ۚ فَرِيقٌۭ فِى ٱلْجَنَّةِ وَفَرِيقٌۭ فِى ٱلسَّعِيرِ ﴿٧﴾\\
\textamh{8.\  } & وَلَوْ شَآءَ ٱللَّهُ لَجَعَلَهُمْ أُمَّةًۭ وَٟحِدَةًۭ وَلَـٰكِن يُدْخِلُ مَن يَشَآءُ فِى رَحْمَتِهِۦ ۚ وَٱلظَّـٰلِمُونَ مَا لَهُم مِّن وَلِىٍّۢ وَلَا نَصِيرٍ ﴿٨﴾\\
\textamh{9.\  } & أَمِ ٱتَّخَذُوا۟ مِن دُونِهِۦٓ أَوْلِيَآءَ ۖ فَٱللَّهُ هُوَ ٱلْوَلِىُّ وَهُوَ يُحْىِ ٱلْمَوْتَىٰ وَهُوَ عَلَىٰ كُلِّ شَىْءٍۢ قَدِيرٌۭ ﴿٩﴾\\
\textamh{10.\  } & وَمَا ٱخْتَلَفْتُمْ فِيهِ مِن شَىْءٍۢ فَحُكْمُهُۥٓ إِلَى ٱللَّهِ ۚ ذَٟلِكُمُ ٱللَّهُ رَبِّى عَلَيْهِ تَوَكَّلْتُ وَإِلَيْهِ أُنِيبُ ﴿١٠﴾\\
\textamh{11.\  } & فَاطِرُ ٱلسَّمَـٰوَٟتِ وَٱلْأَرْضِ ۚ جَعَلَ لَكُم مِّنْ أَنفُسِكُمْ أَزْوَٟجًۭا وَمِنَ ٱلْأَنْعَـٰمِ أَزْوَٟجًۭا ۖ يَذْرَؤُكُمْ فِيهِ ۚ لَيْسَ كَمِثْلِهِۦ شَىْءٌۭ ۖ وَهُوَ ٱلسَّمِيعُ ٱلْبَصِيرُ ﴿١١﴾\\
\textamh{12.\  } & لَهُۥ مَقَالِيدُ ٱلسَّمَـٰوَٟتِ وَٱلْأَرْضِ ۖ يَبْسُطُ ٱلرِّزْقَ لِمَن يَشَآءُ وَيَقْدِرُ ۚ إِنَّهُۥ بِكُلِّ شَىْءٍ عَلِيمٌۭ ﴿١٢﴾\\
\textamh{13.\  } & ۞ شَرَعَ لَكُم مِّنَ ٱلدِّينِ مَا وَصَّىٰ بِهِۦ نُوحًۭا وَٱلَّذِىٓ أَوْحَيْنَآ إِلَيْكَ وَمَا وَصَّيْنَا بِهِۦٓ إِبْرَٰهِيمَ وَمُوسَىٰ وَعِيسَىٰٓ ۖ أَنْ أَقِيمُوا۟ ٱلدِّينَ وَلَا تَتَفَرَّقُوا۟ فِيهِ ۚ كَبُرَ عَلَى ٱلْمُشْرِكِينَ مَا تَدْعُوهُمْ إِلَيْهِ ۚ ٱللَّهُ يَجْتَبِىٓ إِلَيْهِ مَن يَشَآءُ وَيَهْدِىٓ إِلَيْهِ مَن يُنِيبُ ﴿١٣﴾\\
\textamh{14.\  } & وَمَا تَفَرَّقُوٓا۟ إِلَّا مِنۢ بَعْدِ مَا جَآءَهُمُ ٱلْعِلْمُ بَغْيًۢا بَيْنَهُمْ ۚ وَلَوْلَا كَلِمَةٌۭ سَبَقَتْ مِن رَّبِّكَ إِلَىٰٓ أَجَلٍۢ مُّسَمًّۭى لَّقُضِىَ بَيْنَهُمْ ۚ وَإِنَّ ٱلَّذِينَ أُورِثُوا۟ ٱلْكِتَـٰبَ مِنۢ بَعْدِهِمْ لَفِى شَكٍّۢ مِّنْهُ مُرِيبٍۢ ﴿١٤﴾\\
\textamh{15.\  } & فَلِذَٟلِكَ فَٱدْعُ ۖ وَٱسْتَقِمْ كَمَآ أُمِرْتَ ۖ وَلَا تَتَّبِعْ أَهْوَآءَهُمْ ۖ وَقُلْ ءَامَنتُ بِمَآ أَنزَلَ ٱللَّهُ مِن كِتَـٰبٍۢ ۖ وَأُمِرْتُ لِأَعْدِلَ بَيْنَكُمُ ۖ ٱللَّهُ رَبُّنَا وَرَبُّكُمْ ۖ لَنَآ أَعْمَـٰلُنَا وَلَكُمْ أَعْمَـٰلُكُمْ ۖ لَا حُجَّةَ بَيْنَنَا وَبَيْنَكُمُ ۖ ٱللَّهُ يَجْمَعُ بَيْنَنَا ۖ وَإِلَيْهِ ٱلْمَصِيرُ ﴿١٥﴾\\
\textamh{16.\  } & وَٱلَّذِينَ يُحَآجُّونَ فِى ٱللَّهِ مِنۢ بَعْدِ مَا ٱسْتُجِيبَ لَهُۥ حُجَّتُهُمْ دَاحِضَةٌ عِندَ رَبِّهِمْ وَعَلَيْهِمْ غَضَبٌۭ وَلَهُمْ عَذَابٌۭ شَدِيدٌ ﴿١٦﴾\\
\textamh{17.\  } & ٱللَّهُ ٱلَّذِىٓ أَنزَلَ ٱلْكِتَـٰبَ بِٱلْحَقِّ وَٱلْمِيزَانَ ۗ وَمَا يُدْرِيكَ لَعَلَّ ٱلسَّاعَةَ قَرِيبٌۭ ﴿١٧﴾\\
\textamh{18.\  } & يَسْتَعْجِلُ بِهَا ٱلَّذِينَ لَا يُؤْمِنُونَ بِهَا ۖ وَٱلَّذِينَ ءَامَنُوا۟ مُشْفِقُونَ مِنْهَا وَيَعْلَمُونَ أَنَّهَا ٱلْحَقُّ ۗ أَلَآ إِنَّ ٱلَّذِينَ يُمَارُونَ فِى ٱلسَّاعَةِ لَفِى ضَلَـٰلٍۭ بَعِيدٍ ﴿١٨﴾\\
\textamh{19.\  } & ٱللَّهُ لَطِيفٌۢ بِعِبَادِهِۦ يَرْزُقُ مَن يَشَآءُ ۖ وَهُوَ ٱلْقَوِىُّ ٱلْعَزِيزُ ﴿١٩﴾\\
\textamh{20.\  } & مَن كَانَ يُرِيدُ حَرْثَ ٱلْءَاخِرَةِ نَزِدْ لَهُۥ فِى حَرْثِهِۦ ۖ وَمَن كَانَ يُرِيدُ حَرْثَ ٱلدُّنْيَا نُؤْتِهِۦ مِنْهَا وَمَا لَهُۥ فِى ٱلْءَاخِرَةِ مِن نَّصِيبٍ ﴿٢٠﴾\\
\textamh{21.\  } & أَمْ لَهُمْ شُرَكَـٰٓؤُا۟ شَرَعُوا۟ لَهُم مِّنَ ٱلدِّينِ مَا لَمْ يَأْذَنۢ بِهِ ٱللَّهُ ۚ وَلَوْلَا كَلِمَةُ ٱلْفَصْلِ لَقُضِىَ بَيْنَهُمْ ۗ وَإِنَّ ٱلظَّـٰلِمِينَ لَهُمْ عَذَابٌ أَلِيمٌۭ ﴿٢١﴾\\
\textamh{22.\  } & تَرَى ٱلظَّـٰلِمِينَ مُشْفِقِينَ مِمَّا كَسَبُوا۟ وَهُوَ وَاقِعٌۢ بِهِمْ ۗ وَٱلَّذِينَ ءَامَنُوا۟ وَعَمِلُوا۟ ٱلصَّـٰلِحَـٰتِ فِى رَوْضَاتِ ٱلْجَنَّاتِ ۖ لَهُم مَّا يَشَآءُونَ عِندَ رَبِّهِمْ ۚ ذَٟلِكَ هُوَ ٱلْفَضْلُ ٱلْكَبِيرُ ﴿٢٢﴾\\
\textamh{23.\  } & ذَٟلِكَ ٱلَّذِى يُبَشِّرُ ٱللَّهُ عِبَادَهُ ٱلَّذِينَ ءَامَنُوا۟ وَعَمِلُوا۟ ٱلصَّـٰلِحَـٰتِ ۗ قُل لَّآ أَسْـَٔلُكُمْ عَلَيْهِ أَجْرًا إِلَّا ٱلْمَوَدَّةَ فِى ٱلْقُرْبَىٰ ۗ وَمَن يَقْتَرِفْ حَسَنَةًۭ نَّزِدْ لَهُۥ فِيهَا حُسْنًا ۚ إِنَّ ٱللَّهَ غَفُورٌۭ شَكُورٌ ﴿٢٣﴾\\
\textamh{24.\  } & أَمْ يَقُولُونَ ٱفْتَرَىٰ عَلَى ٱللَّهِ كَذِبًۭا ۖ فَإِن يَشَإِ ٱللَّهُ يَخْتِمْ عَلَىٰ قَلْبِكَ ۗ وَيَمْحُ ٱللَّهُ ٱلْبَٰطِلَ وَيُحِقُّ ٱلْحَقَّ بِكَلِمَـٰتِهِۦٓ ۚ إِنَّهُۥ عَلِيمٌۢ بِذَاتِ ٱلصُّدُورِ ﴿٢٤﴾\\
\textamh{25.\  } & وَهُوَ ٱلَّذِى يَقْبَلُ ٱلتَّوْبَةَ عَنْ عِبَادِهِۦ وَيَعْفُوا۟ عَنِ ٱلسَّيِّـَٔاتِ وَيَعْلَمُ مَا تَفْعَلُونَ ﴿٢٥﴾\\
\textamh{26.\  } & وَيَسْتَجِيبُ ٱلَّذِينَ ءَامَنُوا۟ وَعَمِلُوا۟ ٱلصَّـٰلِحَـٰتِ وَيَزِيدُهُم مِّن فَضْلِهِۦ ۚ وَٱلْكَـٰفِرُونَ لَهُمْ عَذَابٌۭ شَدِيدٌۭ ﴿٢٦﴾\\
\textamh{27.\  } & ۞ وَلَوْ بَسَطَ ٱللَّهُ ٱلرِّزْقَ لِعِبَادِهِۦ لَبَغَوْا۟ فِى ٱلْأَرْضِ وَلَـٰكِن يُنَزِّلُ بِقَدَرٍۢ مَّا يَشَآءُ ۚ إِنَّهُۥ بِعِبَادِهِۦ خَبِيرٌۢ بَصِيرٌۭ ﴿٢٧﴾\\
\textamh{28.\  } & وَهُوَ ٱلَّذِى يُنَزِّلُ ٱلْغَيْثَ مِنۢ بَعْدِ مَا قَنَطُوا۟ وَيَنشُرُ رَحْمَتَهُۥ ۚ وَهُوَ ٱلْوَلِىُّ ٱلْحَمِيدُ ﴿٢٨﴾\\
\textamh{29.\  } & وَمِنْ ءَايَـٰتِهِۦ خَلْقُ ٱلسَّمَـٰوَٟتِ وَٱلْأَرْضِ وَمَا بَثَّ فِيهِمَا مِن دَآبَّةٍۢ ۚ وَهُوَ عَلَىٰ جَمْعِهِمْ إِذَا يَشَآءُ قَدِيرٌۭ ﴿٢٩﴾\\
\textamh{30.\  } & وَمَآ أَصَـٰبَكُم مِّن مُّصِيبَةٍۢ فَبِمَا كَسَبَتْ أَيْدِيكُمْ وَيَعْفُوا۟ عَن كَثِيرٍۢ ﴿٣٠﴾\\
\textamh{31.\  } & وَمَآ أَنتُم بِمُعْجِزِينَ فِى ٱلْأَرْضِ ۖ وَمَا لَكُم مِّن دُونِ ٱللَّهِ مِن وَلِىٍّۢ وَلَا نَصِيرٍۢ ﴿٣١﴾\\
\textamh{32.\  } & وَمِنْ ءَايَـٰتِهِ ٱلْجَوَارِ فِى ٱلْبَحْرِ كَٱلْأَعْلَـٰمِ ﴿٣٢﴾\\
\textamh{33.\  } & إِن يَشَأْ يُسْكِنِ ٱلرِّيحَ فَيَظْلَلْنَ رَوَاكِدَ عَلَىٰ ظَهْرِهِۦٓ ۚ إِنَّ فِى ذَٟلِكَ لَءَايَـٰتٍۢ لِّكُلِّ صَبَّارٍۢ شَكُورٍ ﴿٣٣﴾\\
\textamh{34.\  } & أَوْ يُوبِقْهُنَّ بِمَا كَسَبُوا۟ وَيَعْفُ عَن كَثِيرٍۢ ﴿٣٤﴾\\
\textamh{35.\  } & وَيَعْلَمَ ٱلَّذِينَ يُجَٰدِلُونَ فِىٓ ءَايَـٰتِنَا مَا لَهُم مِّن مَّحِيصٍۢ ﴿٣٥﴾\\
\textamh{36.\  } & فَمَآ أُوتِيتُم مِّن شَىْءٍۢ فَمَتَـٰعُ ٱلْحَيَوٰةِ ٱلدُّنْيَا ۖ وَمَا عِندَ ٱللَّهِ خَيْرٌۭ وَأَبْقَىٰ لِلَّذِينَ ءَامَنُوا۟ وَعَلَىٰ رَبِّهِمْ يَتَوَكَّلُونَ ﴿٣٦﴾\\
\textamh{37.\  } & وَٱلَّذِينَ يَجْتَنِبُونَ كَبَٰٓئِرَ ٱلْإِثْمِ وَٱلْفَوَٟحِشَ وَإِذَا مَا غَضِبُوا۟ هُمْ يَغْفِرُونَ ﴿٣٧﴾\\
\textamh{38.\  } & وَٱلَّذِينَ ٱسْتَجَابُوا۟ لِرَبِّهِمْ وَأَقَامُوا۟ ٱلصَّلَوٰةَ وَأَمْرُهُمْ شُورَىٰ بَيْنَهُمْ وَمِمَّا رَزَقْنَـٰهُمْ يُنفِقُونَ ﴿٣٨﴾\\
\textamh{39.\  } & وَٱلَّذِينَ إِذَآ أَصَابَهُمُ ٱلْبَغْىُ هُمْ يَنتَصِرُونَ ﴿٣٩﴾\\
\textamh{40.\  } & وَجَزَٰٓؤُا۟ سَيِّئَةٍۢ سَيِّئَةٌۭ مِّثْلُهَا ۖ فَمَنْ عَفَا وَأَصْلَحَ فَأَجْرُهُۥ عَلَى ٱللَّهِ ۚ إِنَّهُۥ لَا يُحِبُّ ٱلظَّـٰلِمِينَ ﴿٤٠﴾\\
\textamh{41.\  } & وَلَمَنِ ٱنتَصَرَ بَعْدَ ظُلْمِهِۦ فَأُو۟لَـٰٓئِكَ مَا عَلَيْهِم مِّن سَبِيلٍ ﴿٤١﴾\\
\textamh{42.\  } & إِنَّمَا ٱلسَّبِيلُ عَلَى ٱلَّذِينَ يَظْلِمُونَ ٱلنَّاسَ وَيَبْغُونَ فِى ٱلْأَرْضِ بِغَيْرِ ٱلْحَقِّ ۚ أُو۟لَـٰٓئِكَ لَهُمْ عَذَابٌ أَلِيمٌۭ ﴿٤٢﴾\\
\textamh{43.\  } & وَلَمَن صَبَرَ وَغَفَرَ إِنَّ ذَٟلِكَ لَمِنْ عَزْمِ ٱلْأُمُورِ ﴿٤٣﴾\\
\textamh{44.\  } & وَمَن يُضْلِلِ ٱللَّهُ فَمَا لَهُۥ مِن وَلِىٍّۢ مِّنۢ بَعْدِهِۦ ۗ وَتَرَى ٱلظَّـٰلِمِينَ لَمَّا رَأَوُا۟ ٱلْعَذَابَ يَقُولُونَ هَلْ إِلَىٰ مَرَدٍّۢ مِّن سَبِيلٍۢ ﴿٤٤﴾\\
\textamh{45.\  } & وَتَرَىٰهُمْ يُعْرَضُونَ عَلَيْهَا خَـٰشِعِينَ مِنَ ٱلذُّلِّ يَنظُرُونَ مِن طَرْفٍ خَفِىٍّۢ ۗ وَقَالَ ٱلَّذِينَ ءَامَنُوٓا۟ إِنَّ ٱلْخَـٰسِرِينَ ٱلَّذِينَ خَسِرُوٓا۟ أَنفُسَهُمْ وَأَهْلِيهِمْ يَوْمَ ٱلْقِيَـٰمَةِ ۗ أَلَآ إِنَّ ٱلظَّـٰلِمِينَ فِى عَذَابٍۢ مُّقِيمٍۢ ﴿٤٥﴾\\
\textamh{46.\  } & وَمَا كَانَ لَهُم مِّنْ أَوْلِيَآءَ يَنصُرُونَهُم مِّن دُونِ ٱللَّهِ ۗ وَمَن يُضْلِلِ ٱللَّهُ فَمَا لَهُۥ مِن سَبِيلٍ ﴿٤٦﴾\\
\textamh{47.\  } & ٱسْتَجِيبُوا۟ لِرَبِّكُم مِّن قَبْلِ أَن يَأْتِىَ يَوْمٌۭ لَّا مَرَدَّ لَهُۥ مِنَ ٱللَّهِ ۚ مَا لَكُم مِّن مَّلْجَإٍۢ يَوْمَئِذٍۢ وَمَا لَكُم مِّن نَّكِيرٍۢ ﴿٤٧﴾\\
\textamh{48.\  } & فَإِنْ أَعْرَضُوا۟ فَمَآ أَرْسَلْنَـٰكَ عَلَيْهِمْ حَفِيظًا ۖ إِنْ عَلَيْكَ إِلَّا ٱلْبَلَـٰغُ ۗ وَإِنَّآ إِذَآ أَذَقْنَا ٱلْإِنسَـٰنَ مِنَّا رَحْمَةًۭ فَرِحَ بِهَا ۖ وَإِن تُصِبْهُمْ سَيِّئَةٌۢ بِمَا قَدَّمَتْ أَيْدِيهِمْ فَإِنَّ ٱلْإِنسَـٰنَ كَفُورٌۭ ﴿٤٨﴾\\
\textamh{49.\  } & لِّلَّهِ مُلْكُ ٱلسَّمَـٰوَٟتِ وَٱلْأَرْضِ ۚ يَخْلُقُ مَا يَشَآءُ ۚ يَهَبُ لِمَن يَشَآءُ إِنَـٰثًۭا وَيَهَبُ لِمَن يَشَآءُ ٱلذُّكُورَ ﴿٤٩﴾\\
\textamh{50.\  } & أَوْ يُزَوِّجُهُمْ ذُكْرَانًۭا وَإِنَـٰثًۭا ۖ وَيَجْعَلُ مَن يَشَآءُ عَقِيمًا ۚ إِنَّهُۥ عَلِيمٌۭ قَدِيرٌۭ ﴿٥٠﴾\\
\textamh{51.\  } & ۞ وَمَا كَانَ لِبَشَرٍ أَن يُكَلِّمَهُ ٱللَّهُ إِلَّا وَحْيًا أَوْ مِن وَرَآئِ حِجَابٍ أَوْ يُرْسِلَ رَسُولًۭا فَيُوحِىَ بِإِذْنِهِۦ مَا يَشَآءُ ۚ إِنَّهُۥ عَلِىٌّ حَكِيمٌۭ ﴿٥١﴾\\
\textamh{52.\  } & وَكَذَٟلِكَ أَوْحَيْنَآ إِلَيْكَ رُوحًۭا مِّنْ أَمْرِنَا ۚ مَا كُنتَ تَدْرِى مَا ٱلْكِتَـٰبُ وَلَا ٱلْإِيمَـٰنُ وَلَـٰكِن جَعَلْنَـٰهُ نُورًۭا نَّهْدِى بِهِۦ مَن نَّشَآءُ مِنْ عِبَادِنَا ۚ وَإِنَّكَ لَتَهْدِىٓ إِلَىٰ صِرَٰطٍۢ مُّسْتَقِيمٍۢ ﴿٥٢﴾\\
\textamh{53.\  } & صِرَٰطِ ٱللَّهِ ٱلَّذِى لَهُۥ مَا فِى ٱلسَّمَـٰوَٟتِ وَمَا فِى ٱلْأَرْضِ ۗ أَلَآ إِلَى ٱللَّهِ تَصِيرُ ٱلْأُمُورُ ﴿٥٣﴾\\
\end{longtable} \newpage

%% License: BSD style (Berkley) (i.e. Put the Copyright owner's name always)
%% Writer and Copyright (to): Bewketu(Bilal) Tadilo (2016-17)
\begin{center}\section{\LR{\textamhsec{ሱራቱ አልዙኽሩፍ -}  \textarabic{سوره  الزخرف}}}\end{center}
\begin{longtable}{%
  @{}
    p{.5\textwidth}
  @{~~~}
    p{.5\textwidth}
    @{}
}
\textamh{ቢስሚላሂ አራህመኒ ራሂይም } &  \mytextarabic{بِسْمِ ٱللَّهِ ٱلرَّحْمَـٰنِ ٱلرَّحِيمِ}\\
\textamh{1.\  } & \mytextarabic{ حمٓ ﴿١﴾}\\
\textamh{2.\  } & \mytextarabic{وَٱلْكِتَـٰبِ ٱلْمُبِينِ ﴿٢﴾}\\
\textamh{3.\  } & \mytextarabic{إِنَّا جَعَلْنَـٰهُ قُرْءَٰنًا عَرَبِيًّۭا لَّعَلَّكُمْ تَعْقِلُونَ ﴿٣﴾}\\
\textamh{4.\  } & \mytextarabic{وَإِنَّهُۥ فِىٓ أُمِّ ٱلْكِتَـٰبِ لَدَيْنَا لَعَلِىٌّ حَكِيمٌ ﴿٤﴾}\\
\textamh{5.\  } & \mytextarabic{أَفَنَضْرِبُ عَنكُمُ ٱلذِّكْرَ صَفْحًا أَن كُنتُمْ قَوْمًۭا مُّسْرِفِينَ ﴿٥﴾}\\
\textamh{6.\  } & \mytextarabic{وَكَمْ أَرْسَلْنَا مِن نَّبِىٍّۢ فِى ٱلْأَوَّلِينَ ﴿٦﴾}\\
\textamh{7.\  } & \mytextarabic{وَمَا يَأْتِيهِم مِّن نَّبِىٍّ إِلَّا كَانُوا۟ بِهِۦ يَسْتَهْزِءُونَ ﴿٧﴾}\\
\textamh{8.\  } & \mytextarabic{فَأَهْلَكْنَآ أَشَدَّ مِنْهُم بَطْشًۭا وَمَضَىٰ مَثَلُ ٱلْأَوَّلِينَ ﴿٨﴾}\\
\textamh{9.\  } & \mytextarabic{وَلَئِن سَأَلْتَهُم مَّنْ خَلَقَ ٱلسَّمَـٰوَٟتِ وَٱلْأَرْضَ لَيَقُولُنَّ خَلَقَهُنَّ ٱلْعَزِيزُ ٱلْعَلِيمُ ﴿٩﴾}\\
\textamh{10.\  } & \mytextarabic{ٱلَّذِى جَعَلَ لَكُمُ ٱلْأَرْضَ مَهْدًۭا وَجَعَلَ لَكُمْ فِيهَا سُبُلًۭا لَّعَلَّكُمْ تَهْتَدُونَ ﴿١٠﴾}\\
\textamh{11.\  } & \mytextarabic{وَٱلَّذِى نَزَّلَ مِنَ ٱلسَّمَآءِ مَآءًۢ بِقَدَرٍۢ فَأَنشَرْنَا بِهِۦ بَلْدَةًۭ مَّيْتًۭا ۚ كَذَٟلِكَ تُخْرَجُونَ ﴿١١﴾}\\
\textamh{12.\  } & \mytextarabic{وَٱلَّذِى خَلَقَ ٱلْأَزْوَٟجَ كُلَّهَا وَجَعَلَ لَكُم مِّنَ ٱلْفُلْكِ وَٱلْأَنْعَـٰمِ مَا تَرْكَبُونَ ﴿١٢﴾}\\
\textamh{13.\  } & \mytextarabic{لِتَسْتَوُۥا۟ عَلَىٰ ظُهُورِهِۦ ثُمَّ تَذْكُرُوا۟ نِعْمَةَ رَبِّكُمْ إِذَا ٱسْتَوَيْتُمْ عَلَيْهِ وَتَقُولُوا۟ سُبْحَـٰنَ ٱلَّذِى سَخَّرَ لَنَا هَـٰذَا وَمَا كُنَّا لَهُۥ مُقْرِنِينَ ﴿١٣﴾}\\
\textamh{14.\  } & \mytextarabic{وَإِنَّآ إِلَىٰ رَبِّنَا لَمُنقَلِبُونَ ﴿١٤﴾}\\
\textamh{15.\  } & \mytextarabic{وَجَعَلُوا۟ لَهُۥ مِنْ عِبَادِهِۦ جُزْءًا ۚ إِنَّ ٱلْإِنسَـٰنَ لَكَفُورٌۭ مُّبِينٌ ﴿١٥﴾}\\
\textamh{16.\  } & \mytextarabic{أَمِ ٱتَّخَذَ مِمَّا يَخْلُقُ بَنَاتٍۢ وَأَصْفَىٰكُم بِٱلْبَنِينَ ﴿١٦﴾}\\
\textamh{17.\  } & \mytextarabic{وَإِذَا بُشِّرَ أَحَدُهُم بِمَا ضَرَبَ لِلرَّحْمَـٰنِ مَثَلًۭا ظَلَّ وَجْهُهُۥ مُسْوَدًّۭا وَهُوَ كَظِيمٌ ﴿١٧﴾}\\
\textamh{18.\  } & \mytextarabic{أَوَمَن يُنَشَّؤُا۟ فِى ٱلْحِلْيَةِ وَهُوَ فِى ٱلْخِصَامِ غَيْرُ مُبِينٍۢ ﴿١٨﴾}\\
\textamh{19.\  } & \mytextarabic{وَجَعَلُوا۟ ٱلْمَلَـٰٓئِكَةَ ٱلَّذِينَ هُمْ عِبَٰدُ ٱلرَّحْمَـٰنِ إِنَـٰثًا ۚ أَشَهِدُوا۟ خَلْقَهُمْ ۚ سَتُكْتَبُ شَهَـٰدَتُهُمْ وَيُسْـَٔلُونَ ﴿١٩﴾}\\
\textamh{20.\  } & \mytextarabic{وَقَالُوا۟ لَوْ شَآءَ ٱلرَّحْمَـٰنُ مَا عَبَدْنَـٰهُم ۗ مَّا لَهُم بِذَٟلِكَ مِنْ عِلْمٍ ۖ إِنْ هُمْ إِلَّا يَخْرُصُونَ ﴿٢٠﴾}\\
\textamh{21.\  } & \mytextarabic{أَمْ ءَاتَيْنَـٰهُمْ كِتَـٰبًۭا مِّن قَبْلِهِۦ فَهُم بِهِۦ مُسْتَمْسِكُونَ ﴿٢١﴾}\\
\textamh{22.\  } & \mytextarabic{بَلْ قَالُوٓا۟ إِنَّا وَجَدْنَآ ءَابَآءَنَا عَلَىٰٓ أُمَّةٍۢ وَإِنَّا عَلَىٰٓ ءَاثَـٰرِهِم مُّهْتَدُونَ ﴿٢٢﴾}\\
\textamh{23.\  } & \mytextarabic{وَكَذَٟلِكَ مَآ أَرْسَلْنَا مِن قَبْلِكَ فِى قَرْيَةٍۢ مِّن نَّذِيرٍ إِلَّا قَالَ مُتْرَفُوهَآ إِنَّا وَجَدْنَآ ءَابَآءَنَا عَلَىٰٓ أُمَّةٍۢ وَإِنَّا عَلَىٰٓ ءَاثَـٰرِهِم مُّقْتَدُونَ ﴿٢٣﴾}\\
\textamh{24.\  } & \mytextarabic{۞ قَـٰلَ أَوَلَوْ جِئْتُكُم بِأَهْدَىٰ مِمَّا وَجَدتُّمْ عَلَيْهِ ءَابَآءَكُمْ ۖ قَالُوٓا۟ إِنَّا بِمَآ أُرْسِلْتُم بِهِۦ كَـٰفِرُونَ ﴿٢٤﴾}\\
\textamh{25.\  } & \mytextarabic{فَٱنتَقَمْنَا مِنْهُمْ ۖ فَٱنظُرْ كَيْفَ كَانَ عَـٰقِبَةُ ٱلْمُكَذِّبِينَ ﴿٢٥﴾}\\
\textamh{26.\  } & \mytextarabic{وَإِذْ قَالَ إِبْرَٰهِيمُ لِأَبِيهِ وَقَوْمِهِۦٓ إِنَّنِى بَرَآءٌۭ مِّمَّا تَعْبُدُونَ ﴿٢٦﴾}\\
\textamh{27.\  } & \mytextarabic{إِلَّا ٱلَّذِى فَطَرَنِى فَإِنَّهُۥ سَيَهْدِينِ ﴿٢٧﴾}\\
\textamh{28.\  } & \mytextarabic{وَجَعَلَهَا كَلِمَةًۢ بَاقِيَةًۭ فِى عَقِبِهِۦ لَعَلَّهُمْ يَرْجِعُونَ ﴿٢٨﴾}\\
\textamh{29.\  } & \mytextarabic{بَلْ مَتَّعْتُ هَـٰٓؤُلَآءِ وَءَابَآءَهُمْ حَتَّىٰ جَآءَهُمُ ٱلْحَقُّ وَرَسُولٌۭ مُّبِينٌۭ ﴿٢٩﴾}\\
\textamh{30.\  } & \mytextarabic{وَلَمَّا جَآءَهُمُ ٱلْحَقُّ قَالُوا۟ هَـٰذَا سِحْرٌۭ وَإِنَّا بِهِۦ كَـٰفِرُونَ ﴿٣٠﴾}\\
\textamh{31.\  } & \mytextarabic{وَقَالُوا۟ لَوْلَا نُزِّلَ هَـٰذَا ٱلْقُرْءَانُ عَلَىٰ رَجُلٍۢ مِّنَ ٱلْقَرْيَتَيْنِ عَظِيمٍ ﴿٣١﴾}\\
\textamh{32.\  } & \mytextarabic{أَهُمْ يَقْسِمُونَ رَحْمَتَ رَبِّكَ ۚ نَحْنُ قَسَمْنَا بَيْنَهُم مَّعِيشَتَهُمْ فِى ٱلْحَيَوٰةِ ٱلدُّنْيَا ۚ وَرَفَعْنَا بَعْضَهُمْ فَوْقَ بَعْضٍۢ دَرَجَٰتٍۢ لِّيَتَّخِذَ بَعْضُهُم بَعْضًۭا سُخْرِيًّۭا ۗ وَرَحْمَتُ رَبِّكَ خَيْرٌۭ مِّمَّا يَجْمَعُونَ ﴿٣٢﴾}\\
\textamh{33.\  } & \mytextarabic{وَلَوْلَآ أَن يَكُونَ ٱلنَّاسُ أُمَّةًۭ وَٟحِدَةًۭ لَّجَعَلْنَا لِمَن يَكْفُرُ بِٱلرَّحْمَـٰنِ لِبُيُوتِهِمْ سُقُفًۭا مِّن فِضَّةٍۢ وَمَعَارِجَ عَلَيْهَا يَظْهَرُونَ ﴿٣٣﴾}\\
\textamh{34.\  } & \mytextarabic{وَلِبُيُوتِهِمْ أَبْوَٟبًۭا وَسُرُرًا عَلَيْهَا يَتَّكِـُٔونَ ﴿٣٤﴾}\\
\textamh{35.\  } & \mytextarabic{وَزُخْرُفًۭا ۚ وَإِن كُلُّ ذَٟلِكَ لَمَّا مَتَـٰعُ ٱلْحَيَوٰةِ ٱلدُّنْيَا ۚ وَٱلْءَاخِرَةُ عِندَ رَبِّكَ لِلْمُتَّقِينَ ﴿٣٥﴾}\\
\textamh{36.\  } & \mytextarabic{وَمَن يَعْشُ عَن ذِكْرِ ٱلرَّحْمَـٰنِ نُقَيِّضْ لَهُۥ شَيْطَٰنًۭا فَهُوَ لَهُۥ قَرِينٌۭ ﴿٣٦﴾}\\
\textamh{37.\  } & \mytextarabic{وَإِنَّهُمْ لَيَصُدُّونَهُمْ عَنِ ٱلسَّبِيلِ وَيَحْسَبُونَ أَنَّهُم مُّهْتَدُونَ ﴿٣٧﴾}\\
\textamh{38.\  } & \mytextarabic{حَتَّىٰٓ إِذَا جَآءَنَا قَالَ يَـٰلَيْتَ بَيْنِى وَبَيْنَكَ بُعْدَ ٱلْمَشْرِقَيْنِ فَبِئْسَ ٱلْقَرِينُ ﴿٣٨﴾}\\
\textamh{39.\  } & \mytextarabic{وَلَن يَنفَعَكُمُ ٱلْيَوْمَ إِذ ظَّلَمْتُمْ أَنَّكُمْ فِى ٱلْعَذَابِ مُشْتَرِكُونَ ﴿٣٩﴾}\\
\textamh{40.\  } & \mytextarabic{أَفَأَنتَ تُسْمِعُ ٱلصُّمَّ أَوْ تَهْدِى ٱلْعُمْىَ وَمَن كَانَ فِى ضَلَـٰلٍۢ مُّبِينٍۢ ﴿٤٠﴾}\\
\textamh{41.\  } & \mytextarabic{فَإِمَّا نَذْهَبَنَّ بِكَ فَإِنَّا مِنْهُم مُّنتَقِمُونَ ﴿٤١﴾}\\
\textamh{42.\  } & \mytextarabic{أَوْ نُرِيَنَّكَ ٱلَّذِى وَعَدْنَـٰهُمْ فَإِنَّا عَلَيْهِم مُّقْتَدِرُونَ ﴿٤٢﴾}\\
\textamh{43.\  } & \mytextarabic{فَٱسْتَمْسِكْ بِٱلَّذِىٓ أُوحِىَ إِلَيْكَ ۖ إِنَّكَ عَلَىٰ صِرَٰطٍۢ مُّسْتَقِيمٍۢ ﴿٤٣﴾}\\
\textamh{44.\  } & \mytextarabic{وَإِنَّهُۥ لَذِكْرٌۭ لَّكَ وَلِقَوْمِكَ ۖ وَسَوْفَ تُسْـَٔلُونَ ﴿٤٤﴾}\\
\textamh{45.\  } & \mytextarabic{وَسْـَٔلْ مَنْ أَرْسَلْنَا مِن قَبْلِكَ مِن رُّسُلِنَآ أَجَعَلْنَا مِن دُونِ ٱلرَّحْمَـٰنِ ءَالِهَةًۭ يُعْبَدُونَ ﴿٤٥﴾}\\
\textamh{46.\  } & \mytextarabic{وَلَقَدْ أَرْسَلْنَا مُوسَىٰ بِـَٔايَـٰتِنَآ إِلَىٰ فِرْعَوْنَ وَمَلَإِي۟هِۦ فَقَالَ إِنِّى رَسُولُ رَبِّ ٱلْعَـٰلَمِينَ ﴿٤٦﴾}\\
\textamh{47.\  } & \mytextarabic{فَلَمَّا جَآءَهُم بِـَٔايَـٰتِنَآ إِذَا هُم مِّنْهَا يَضْحَكُونَ ﴿٤٧﴾}\\
\textamh{48.\  } & \mytextarabic{وَمَا نُرِيهِم مِّنْ ءَايَةٍ إِلَّا هِىَ أَكْبَرُ مِنْ أُخْتِهَا ۖ وَأَخَذْنَـٰهُم بِٱلْعَذَابِ لَعَلَّهُمْ يَرْجِعُونَ ﴿٤٨﴾}\\
\textamh{49.\  } & \mytextarabic{وَقَالُوا۟ يَـٰٓأَيُّهَ ٱلسَّاحِرُ ٱدْعُ لَنَا رَبَّكَ بِمَا عَهِدَ عِندَكَ إِنَّنَا لَمُهْتَدُونَ ﴿٤٩﴾}\\
\textamh{50.\  } & \mytextarabic{فَلَمَّا كَشَفْنَا عَنْهُمُ ٱلْعَذَابَ إِذَا هُمْ يَنكُثُونَ ﴿٥٠﴾}\\
\textamh{51.\  } & \mytextarabic{وَنَادَىٰ فِرْعَوْنُ فِى قَوْمِهِۦ قَالَ يَـٰقَوْمِ أَلَيْسَ لِى مُلْكُ مِصْرَ وَهَـٰذِهِ ٱلْأَنْهَـٰرُ تَجْرِى مِن تَحْتِىٓ ۖ أَفَلَا تُبْصِرُونَ ﴿٥١﴾}\\
\textamh{52.\  } & \mytextarabic{أَمْ أَنَا۠ خَيْرٌۭ مِّنْ هَـٰذَا ٱلَّذِى هُوَ مَهِينٌۭ وَلَا يَكَادُ يُبِينُ ﴿٥٢﴾}\\
\textamh{53.\  } & \mytextarabic{فَلَوْلَآ أُلْقِىَ عَلَيْهِ أَسْوِرَةٌۭ مِّن ذَهَبٍ أَوْ جَآءَ مَعَهُ ٱلْمَلَـٰٓئِكَةُ مُقْتَرِنِينَ ﴿٥٣﴾}\\
\textamh{54.\  } & \mytextarabic{فَٱسْتَخَفَّ قَوْمَهُۥ فَأَطَاعُوهُ ۚ إِنَّهُمْ كَانُوا۟ قَوْمًۭا فَـٰسِقِينَ ﴿٥٤﴾}\\
\textamh{55.\  } & \mytextarabic{فَلَمَّآ ءَاسَفُونَا ٱنتَقَمْنَا مِنْهُمْ فَأَغْرَقْنَـٰهُمْ أَجْمَعِينَ ﴿٥٥﴾}\\
\textamh{56.\  } & \mytextarabic{فَجَعَلْنَـٰهُمْ سَلَفًۭا وَمَثَلًۭا لِّلْءَاخِرِينَ ﴿٥٦﴾}\\
\textamh{57.\  } & \mytextarabic{۞ وَلَمَّا ضُرِبَ ٱبْنُ مَرْيَمَ مَثَلًا إِذَا قَوْمُكَ مِنْهُ يَصِدُّونَ ﴿٥٧﴾}\\
\textamh{58.\  } & \mytextarabic{وَقَالُوٓا۟ ءَأَٰلِهَتُنَا خَيْرٌ أَمْ هُوَ ۚ مَا ضَرَبُوهُ لَكَ إِلَّا جَدَلًۢا ۚ بَلْ هُمْ قَوْمٌ خَصِمُونَ ﴿٥٨﴾}\\
\textamh{59.\  } & \mytextarabic{إِنْ هُوَ إِلَّا عَبْدٌ أَنْعَمْنَا عَلَيْهِ وَجَعَلْنَـٰهُ مَثَلًۭا لِّبَنِىٓ إِسْرَٰٓءِيلَ ﴿٥٩﴾}\\
\textamh{60.\  } & \mytextarabic{وَلَوْ نَشَآءُ لَجَعَلْنَا مِنكُم مَّلَـٰٓئِكَةًۭ فِى ٱلْأَرْضِ يَخْلُفُونَ ﴿٦٠﴾}\\
\textamh{61.\  } & \mytextarabic{وَإِنَّهُۥ لَعِلْمٌۭ لِّلسَّاعَةِ فَلَا تَمْتَرُنَّ بِهَا وَٱتَّبِعُونِ ۚ هَـٰذَا صِرَٰطٌۭ مُّسْتَقِيمٌۭ ﴿٦١﴾}\\
\textamh{62.\  } & \mytextarabic{وَلَا يَصُدَّنَّكُمُ ٱلشَّيْطَٰنُ ۖ إِنَّهُۥ لَكُمْ عَدُوٌّۭ مُّبِينٌۭ ﴿٦٢﴾}\\
\textamh{63.\  } & \mytextarabic{وَلَمَّا جَآءَ عِيسَىٰ بِٱلْبَيِّنَـٰتِ قَالَ قَدْ جِئْتُكُم بِٱلْحِكْمَةِ وَلِأُبَيِّنَ لَكُم بَعْضَ ٱلَّذِى تَخْتَلِفُونَ فِيهِ ۖ فَٱتَّقُوا۟ ٱللَّهَ وَأَطِيعُونِ ﴿٦٣﴾}\\
\textamh{64.\  } & \mytextarabic{إِنَّ ٱللَّهَ هُوَ رَبِّى وَرَبُّكُمْ فَٱعْبُدُوهُ ۚ هَـٰذَا صِرَٰطٌۭ مُّسْتَقِيمٌۭ ﴿٦٤﴾}\\
\textamh{65.\  } & \mytextarabic{فَٱخْتَلَفَ ٱلْأَحْزَابُ مِنۢ بَيْنِهِمْ ۖ فَوَيْلٌۭ لِّلَّذِينَ ظَلَمُوا۟ مِنْ عَذَابِ يَوْمٍ أَلِيمٍ ﴿٦٥﴾}\\
\textamh{66.\  } & \mytextarabic{هَلْ يَنظُرُونَ إِلَّا ٱلسَّاعَةَ أَن تَأْتِيَهُم بَغْتَةًۭ وَهُمْ لَا يَشْعُرُونَ ﴿٦٦﴾}\\
\textamh{67.\  } & \mytextarabic{ٱلْأَخِلَّآءُ يَوْمَئِذٍۭ بَعْضُهُمْ لِبَعْضٍ عَدُوٌّ إِلَّا ٱلْمُتَّقِينَ ﴿٦٧﴾}\\
\textamh{68.\  } & \mytextarabic{يَـٰعِبَادِ لَا خَوْفٌ عَلَيْكُمُ ٱلْيَوْمَ وَلَآ أَنتُمْ تَحْزَنُونَ ﴿٦٨﴾}\\
\textamh{69.\  } & \mytextarabic{ٱلَّذِينَ ءَامَنُوا۟ بِـَٔايَـٰتِنَا وَكَانُوا۟ مُسْلِمِينَ ﴿٦٩﴾}\\
\textamh{70.\  } & \mytextarabic{ٱدْخُلُوا۟ ٱلْجَنَّةَ أَنتُمْ وَأَزْوَٟجُكُمْ تُحْبَرُونَ ﴿٧٠﴾}\\
\textamh{71.\  } & \mytextarabic{يُطَافُ عَلَيْهِم بِصِحَافٍۢ مِّن ذَهَبٍۢ وَأَكْوَابٍۢ ۖ وَفِيهَا مَا تَشْتَهِيهِ ٱلْأَنفُسُ وَتَلَذُّ ٱلْأَعْيُنُ ۖ وَأَنتُمْ فِيهَا خَـٰلِدُونَ ﴿٧١﴾}\\
\textamh{72.\  } & \mytextarabic{وَتِلْكَ ٱلْجَنَّةُ ٱلَّتِىٓ أُورِثْتُمُوهَا بِمَا كُنتُمْ تَعْمَلُونَ ﴿٧٢﴾}\\
\textamh{73.\  } & \mytextarabic{لَكُمْ فِيهَا فَـٰكِهَةٌۭ كَثِيرَةٌۭ مِّنْهَا تَأْكُلُونَ ﴿٧٣﴾}\\
\textamh{74.\  } & \mytextarabic{إِنَّ ٱلْمُجْرِمِينَ فِى عَذَابِ جَهَنَّمَ خَـٰلِدُونَ ﴿٧٤﴾}\\
\textamh{75.\  } & \mytextarabic{لَا يُفَتَّرُ عَنْهُمْ وَهُمْ فِيهِ مُبْلِسُونَ ﴿٧٥﴾}\\
\textamh{76.\  } & \mytextarabic{وَمَا ظَلَمْنَـٰهُمْ وَلَـٰكِن كَانُوا۟ هُمُ ٱلظَّـٰلِمِينَ ﴿٧٦﴾}\\
\textamh{77.\  } & \mytextarabic{وَنَادَوْا۟ يَـٰمَـٰلِكُ لِيَقْضِ عَلَيْنَا رَبُّكَ ۖ قَالَ إِنَّكُم مَّٰكِثُونَ ﴿٧٧﴾}\\
\textamh{78.\  } & \mytextarabic{لَقَدْ جِئْنَـٰكُم بِٱلْحَقِّ وَلَـٰكِنَّ أَكْثَرَكُمْ لِلْحَقِّ كَـٰرِهُونَ ﴿٧٨﴾}\\
\textamh{79.\  } & \mytextarabic{أَمْ أَبْرَمُوٓا۟ أَمْرًۭا فَإِنَّا مُبْرِمُونَ ﴿٧٩﴾}\\
\textamh{80.\  } & \mytextarabic{أَمْ يَحْسَبُونَ أَنَّا لَا نَسْمَعُ سِرَّهُمْ وَنَجْوَىٰهُم ۚ بَلَىٰ وَرُسُلُنَا لَدَيْهِمْ يَكْتُبُونَ ﴿٨٠﴾}\\
\textamh{81.\  } & \mytextarabic{قُلْ إِن كَانَ لِلرَّحْمَـٰنِ وَلَدٌۭ فَأَنَا۠ أَوَّلُ ٱلْعَـٰبِدِينَ ﴿٨١﴾}\\
\textamh{82.\  } & \mytextarabic{سُبْحَـٰنَ رَبِّ ٱلسَّمَـٰوَٟتِ وَٱلْأَرْضِ رَبِّ ٱلْعَرْشِ عَمَّا يَصِفُونَ ﴿٨٢﴾}\\
\textamh{83.\  } & \mytextarabic{فَذَرْهُمْ يَخُوضُوا۟ وَيَلْعَبُوا۟ حَتَّىٰ يُلَـٰقُوا۟ يَوْمَهُمُ ٱلَّذِى يُوعَدُونَ ﴿٨٣﴾}\\
\textamh{84.\  } & \mytextarabic{وَهُوَ ٱلَّذِى فِى ٱلسَّمَآءِ إِلَـٰهٌۭ وَفِى ٱلْأَرْضِ إِلَـٰهٌۭ ۚ وَهُوَ ٱلْحَكِيمُ ٱلْعَلِيمُ ﴿٨٤﴾}\\
\textamh{85.\  } & \mytextarabic{وَتَبَارَكَ ٱلَّذِى لَهُۥ مُلْكُ ٱلسَّمَـٰوَٟتِ وَٱلْأَرْضِ وَمَا بَيْنَهُمَا وَعِندَهُۥ عِلْمُ ٱلسَّاعَةِ وَإِلَيْهِ تُرْجَعُونَ ﴿٨٥﴾}\\
\textamh{86.\  } & \mytextarabic{وَلَا يَمْلِكُ ٱلَّذِينَ يَدْعُونَ مِن دُونِهِ ٱلشَّفَـٰعَةَ إِلَّا مَن شَهِدَ بِٱلْحَقِّ وَهُمْ يَعْلَمُونَ ﴿٨٦﴾}\\
\textamh{87.\  } & \mytextarabic{وَلَئِن سَأَلْتَهُم مَّنْ خَلَقَهُمْ لَيَقُولُنَّ ٱللَّهُ ۖ فَأَنَّىٰ يُؤْفَكُونَ ﴿٨٧﴾}\\
\textamh{88.\  } & \mytextarabic{وَقِيلِهِۦ يَـٰرَبِّ إِنَّ هَـٰٓؤُلَآءِ قَوْمٌۭ لَّا يُؤْمِنُونَ ﴿٨٨﴾}\\
\textamh{89.\  } & \mytextarabic{فَٱصْفَحْ عَنْهُمْ وَقُلْ سَلَـٰمٌۭ ۚ فَسَوْفَ يَعْلَمُونَ ﴿٨٩﴾}\\
\end{longtable}
\clearpage
%% License: BSD style (Berkley) (i.e. Put the Copyright owner's name always)
%% Writer and Copyright (to): Bewketu(Bilal) Tadilo (2016-17)
\centering\section{\LR{\textamharic{ሱራቱ አልዱኻን -}  \RL{سوره  الدخان}}}
\begin{longtable}{%
  @{}
    p{.5\textwidth}
  @{~~~~~~~~~~~~~}
    p{.5\textwidth}
    @{}
}
\nopagebreak
\textamh{\ \ \ \ \ \  ቢስሚላሂ አራህመኒ ራሂይም } &  بِسْمِ ٱللَّهِ ٱلرَّحْمَـٰنِ ٱلرَّحِيمِ\\
\textamh{1.\  } &  حمٓ ﴿١﴾\\
\textamh{2.\  } & وَٱلْكِتَـٰبِ ٱلْمُبِينِ ﴿٢﴾\\
\textamh{3.\  } & إِنَّآ أَنزَلْنَـٰهُ فِى لَيْلَةٍۢ مُّبَٰرَكَةٍ ۚ إِنَّا كُنَّا مُنذِرِينَ ﴿٣﴾\\
\textamh{4.\  } & فِيهَا يُفْرَقُ كُلُّ أَمْرٍ حَكِيمٍ ﴿٤﴾\\
\textamh{5.\  } & أَمْرًۭا مِّنْ عِندِنَآ ۚ إِنَّا كُنَّا مُرْسِلِينَ ﴿٥﴾\\
\textamh{6.\  } & رَحْمَةًۭ مِّن رَّبِّكَ ۚ إِنَّهُۥ هُوَ ٱلسَّمِيعُ ٱلْعَلِيمُ ﴿٦﴾\\
\textamh{7.\  } & رَبِّ ٱلسَّمَـٰوَٟتِ وَٱلْأَرْضِ وَمَا بَيْنَهُمَآ ۖ إِن كُنتُم مُّوقِنِينَ ﴿٧﴾\\
\textamh{8.\  } & لَآ إِلَـٰهَ إِلَّا هُوَ يُحْىِۦ وَيُمِيتُ ۖ رَبُّكُمْ وَرَبُّ ءَابَآئِكُمُ ٱلْأَوَّلِينَ ﴿٨﴾\\
\textamh{9.\  } & بَلْ هُمْ فِى شَكٍّۢ يَلْعَبُونَ ﴿٩﴾\\
\textamh{10.\  } & فَٱرْتَقِبْ يَوْمَ تَأْتِى ٱلسَّمَآءُ بِدُخَانٍۢ مُّبِينٍۢ ﴿١٠﴾\\
\textamh{11.\  } & يَغْشَى ٱلنَّاسَ ۖ هَـٰذَا عَذَابٌ أَلِيمٌۭ ﴿١١﴾\\
\textamh{12.\  } & رَّبَّنَا ٱكْشِفْ عَنَّا ٱلْعَذَابَ إِنَّا مُؤْمِنُونَ ﴿١٢﴾\\
\textamh{13.\  } & أَنَّىٰ لَهُمُ ٱلذِّكْرَىٰ وَقَدْ جَآءَهُمْ رَسُولٌۭ مُّبِينٌۭ ﴿١٣﴾\\
\textamh{14.\  } & ثُمَّ تَوَلَّوْا۟ عَنْهُ وَقَالُوا۟ مُعَلَّمٌۭ مَّجْنُونٌ ﴿١٤﴾\\
\textamh{15.\  } & إِنَّا كَاشِفُوا۟ ٱلْعَذَابِ قَلِيلًا ۚ إِنَّكُمْ عَآئِدُونَ ﴿١٥﴾\\
\textamh{16.\  } & يَوْمَ نَبْطِشُ ٱلْبَطْشَةَ ٱلْكُبْرَىٰٓ إِنَّا مُنتَقِمُونَ ﴿١٦﴾\\
\textamh{17.\  } & ۞ وَلَقَدْ فَتَنَّا قَبْلَهُمْ قَوْمَ فِرْعَوْنَ وَجَآءَهُمْ رَسُولٌۭ كَرِيمٌ ﴿١٧﴾\\
\textamh{18.\  } & أَنْ أَدُّوٓا۟ إِلَىَّ عِبَادَ ٱللَّهِ ۖ إِنِّى لَكُمْ رَسُولٌ أَمِينٌۭ ﴿١٨﴾\\
\textamh{19.\  } & وَأَن لَّا تَعْلُوا۟ عَلَى ٱللَّهِ ۖ إِنِّىٓ ءَاتِيكُم بِسُلْطَٰنٍۢ مُّبِينٍۢ ﴿١٩﴾\\
\textamh{20.\  } & وَإِنِّى عُذْتُ بِرَبِّى وَرَبِّكُمْ أَن تَرْجُمُونِ ﴿٢٠﴾\\
\textamh{21.\  } & وَإِن لَّمْ تُؤْمِنُوا۟ لِى فَٱعْتَزِلُونِ ﴿٢١﴾\\
\textamh{22.\  } & فَدَعَا رَبَّهُۥٓ أَنَّ هَـٰٓؤُلَآءِ قَوْمٌۭ مُّجْرِمُونَ ﴿٢٢﴾\\
\textamh{23.\  } & فَأَسْرِ بِعِبَادِى لَيْلًا إِنَّكُم مُّتَّبَعُونَ ﴿٢٣﴾\\
\textamh{24.\  } & وَٱتْرُكِ ٱلْبَحْرَ رَهْوًا ۖ إِنَّهُمْ جُندٌۭ مُّغْرَقُونَ ﴿٢٤﴾\\
\textamh{25.\  } & كَمْ تَرَكُوا۟ مِن جَنَّـٰتٍۢ وَعُيُونٍۢ ﴿٢٥﴾\\
\textamh{26.\  } & وَزُرُوعٍۢ وَمَقَامٍۢ كَرِيمٍۢ ﴿٢٦﴾\\
\textamh{27.\  } & وَنَعْمَةٍۢ كَانُوا۟ فِيهَا فَـٰكِهِينَ ﴿٢٧﴾\\
\textamh{28.\  } & كَذَٟلِكَ ۖ وَأَوْرَثْنَـٰهَا قَوْمًا ءَاخَرِينَ ﴿٢٨﴾\\
\textamh{29.\  } & فَمَا بَكَتْ عَلَيْهِمُ ٱلسَّمَآءُ وَٱلْأَرْضُ وَمَا كَانُوا۟ مُنظَرِينَ ﴿٢٩﴾\\
\textamh{30.\  } & وَلَقَدْ نَجَّيْنَا بَنِىٓ إِسْرَٰٓءِيلَ مِنَ ٱلْعَذَابِ ٱلْمُهِينِ ﴿٣٠﴾\\
\textamh{31.\  } & مِن فِرْعَوْنَ ۚ إِنَّهُۥ كَانَ عَالِيًۭا مِّنَ ٱلْمُسْرِفِينَ ﴿٣١﴾\\
\textamh{32.\  } & وَلَقَدِ ٱخْتَرْنَـٰهُمْ عَلَىٰ عِلْمٍ عَلَى ٱلْعَـٰلَمِينَ ﴿٣٢﴾\\
\textamh{33.\  } & وَءَاتَيْنَـٰهُم مِّنَ ٱلْءَايَـٰتِ مَا فِيهِ بَلَـٰٓؤٌۭا۟ مُّبِينٌ ﴿٣٣﴾\\
\textamh{34.\  } & إِنَّ هَـٰٓؤُلَآءِ لَيَقُولُونَ ﴿٣٤﴾\\
\textamh{35.\  } & إِنْ هِىَ إِلَّا مَوْتَتُنَا ٱلْأُولَىٰ وَمَا نَحْنُ بِمُنشَرِينَ ﴿٣٥﴾\\
\textamh{36.\  } & فَأْتُوا۟ بِـَٔابَآئِنَآ إِن كُنتُمْ صَـٰدِقِينَ ﴿٣٦﴾\\
\textamh{37.\  } & أَهُمْ خَيْرٌ أَمْ قَوْمُ تُبَّعٍۢ وَٱلَّذِينَ مِن قَبْلِهِمْ ۚ أَهْلَكْنَـٰهُمْ ۖ إِنَّهُمْ كَانُوا۟ مُجْرِمِينَ ﴿٣٧﴾\\
\textamh{38.\  } & وَمَا خَلَقْنَا ٱلسَّمَـٰوَٟتِ وَٱلْأَرْضَ وَمَا بَيْنَهُمَا لَـٰعِبِينَ ﴿٣٨﴾\\
\textamh{39.\  } & مَا خَلَقْنَـٰهُمَآ إِلَّا بِٱلْحَقِّ وَلَـٰكِنَّ أَكْثَرَهُمْ لَا يَعْلَمُونَ ﴿٣٩﴾\\
\textamh{40.\  } & إِنَّ يَوْمَ ٱلْفَصْلِ مِيقَـٰتُهُمْ أَجْمَعِينَ ﴿٤٠﴾\\
\textamh{41.\  } & يَوْمَ لَا يُغْنِى مَوْلًى عَن مَّوْلًۭى شَيْـًۭٔا وَلَا هُمْ يُنصَرُونَ ﴿٤١﴾\\
\textamh{42.\  } & إِلَّا مَن رَّحِمَ ٱللَّهُ ۚ إِنَّهُۥ هُوَ ٱلْعَزِيزُ ٱلرَّحِيمُ ﴿٤٢﴾\\
\textamh{43.\  } & إِنَّ شَجَرَتَ ٱلزَّقُّومِ ﴿٤٣﴾\\
\textamh{44.\  } & طَعَامُ ٱلْأَثِيمِ ﴿٤٤﴾\\
\textamh{45.\  } & كَٱلْمُهْلِ يَغْلِى فِى ٱلْبُطُونِ ﴿٤٥﴾\\
\textamh{46.\  } & كَغَلْىِ ٱلْحَمِيمِ ﴿٤٦﴾\\
\textamh{47.\  } & خُذُوهُ فَٱعْتِلُوهُ إِلَىٰ سَوَآءِ ٱلْجَحِيمِ ﴿٤٧﴾\\
\textamh{48.\  } & ثُمَّ صُبُّوا۟ فَوْقَ رَأْسِهِۦ مِنْ عَذَابِ ٱلْحَمِيمِ ﴿٤٨﴾\\
\textamh{49.\  } & ذُقْ إِنَّكَ أَنتَ ٱلْعَزِيزُ ٱلْكَرِيمُ ﴿٤٩﴾\\
\textamh{50.\  } & إِنَّ هَـٰذَا مَا كُنتُم بِهِۦ تَمْتَرُونَ ﴿٥٠﴾\\
\textamh{51.\  } & إِنَّ ٱلْمُتَّقِينَ فِى مَقَامٍ أَمِينٍۢ ﴿٥١﴾\\
\textamh{52.\  } & فِى جَنَّـٰتٍۢ وَعُيُونٍۢ ﴿٥٢﴾\\
\textamh{53.\  } & يَلْبَسُونَ مِن سُندُسٍۢ وَإِسْتَبْرَقٍۢ مُّتَقَـٰبِلِينَ ﴿٥٣﴾\\
\textamh{54.\  } & كَذَٟلِكَ وَزَوَّجْنَـٰهُم بِحُورٍ عِينٍۢ ﴿٥٤﴾\\
\textamh{55.\  } & يَدْعُونَ فِيهَا بِكُلِّ فَـٰكِهَةٍ ءَامِنِينَ ﴿٥٥﴾\\
\textamh{56.\  } & لَا يَذُوقُونَ فِيهَا ٱلْمَوْتَ إِلَّا ٱلْمَوْتَةَ ٱلْأُولَىٰ ۖ وَوَقَىٰهُمْ عَذَابَ ٱلْجَحِيمِ ﴿٥٦﴾\\
\textamh{57.\  } & فَضْلًۭا مِّن رَّبِّكَ ۚ ذَٟلِكَ هُوَ ٱلْفَوْزُ ٱلْعَظِيمُ ﴿٥٧﴾\\
\textamh{58.\  } & فَإِنَّمَا يَسَّرْنَـٰهُ بِلِسَانِكَ لَعَلَّهُمْ يَتَذَكَّرُونَ ﴿٥٨﴾\\
\textamh{59.\  } & فَٱرْتَقِبْ إِنَّهُم مُّرْتَقِبُونَ ﴿٥٩﴾\\
\end{longtable} \newpage

%% License: BSD style (Berkley) (i.e. Put the Copyright owner's name always)
%% Writer and Copyright (to): Bewketu(Bilal) Tadilo (2016-17)
\centering\section{\LR{\textamharic{ሱራቱ አልጃቲያት -}  \RL{سوره  الجاثية}}}
\begin{longtable}{%
  @{}
    p{.5\textwidth}
  @{~~~~~~~~~~~~~}
    p{.5\textwidth}
    @{}
}
\nopagebreak
\textamh{\ \ \ \ \ \  ቢስሚላሂ አራህመኒ ራሂይም } &  بِسْمِ ٱللَّهِ ٱلرَّحْمَـٰنِ ٱلرَّحِيمِ\\
\textamh{1.\  } &  حمٓ ﴿١﴾\\
\textamh{2.\  } & تَنزِيلُ ٱلْكِتَـٰبِ مِنَ ٱللَّهِ ٱلْعَزِيزِ ٱلْحَكِيمِ ﴿٢﴾\\
\textamh{3.\  } & إِنَّ فِى ٱلسَّمَـٰوَٟتِ وَٱلْأَرْضِ لَءَايَـٰتٍۢ لِّلْمُؤْمِنِينَ ﴿٣﴾\\
\textamh{4.\  } & وَفِى خَلْقِكُمْ وَمَا يَبُثُّ مِن دَآبَّةٍ ءَايَـٰتٌۭ لِّقَوْمٍۢ يُوقِنُونَ ﴿٤﴾\\
\textamh{5.\  } & وَٱخْتِلَـٰفِ ٱلَّيْلِ وَٱلنَّهَارِ وَمَآ أَنزَلَ ٱللَّهُ مِنَ ٱلسَّمَآءِ مِن رِّزْقٍۢ فَأَحْيَا بِهِ ٱلْأَرْضَ بَعْدَ مَوْتِهَا وَتَصْرِيفِ ٱلرِّيَـٰحِ ءَايَـٰتٌۭ لِّقَوْمٍۢ يَعْقِلُونَ ﴿٥﴾\\
\textamh{6.\  } & تِلْكَ ءَايَـٰتُ ٱللَّهِ نَتْلُوهَا عَلَيْكَ بِٱلْحَقِّ ۖ فَبِأَىِّ حَدِيثٍۭ بَعْدَ ٱللَّهِ وَءَايَـٰتِهِۦ يُؤْمِنُونَ ﴿٦﴾\\
\textamh{7.\  } & وَيْلٌۭ لِّكُلِّ أَفَّاكٍ أَثِيمٍۢ ﴿٧﴾\\
\textamh{8.\  } & يَسْمَعُ ءَايَـٰتِ ٱللَّهِ تُتْلَىٰ عَلَيْهِ ثُمَّ يُصِرُّ مُسْتَكْبِرًۭا كَأَن لَّمْ يَسْمَعْهَا ۖ فَبَشِّرْهُ بِعَذَابٍ أَلِيمٍۢ ﴿٨﴾\\
\textamh{9.\  } & وَإِذَا عَلِمَ مِنْ ءَايَـٰتِنَا شَيْـًٔا ٱتَّخَذَهَا هُزُوًا ۚ أُو۟لَـٰٓئِكَ لَهُمْ عَذَابٌۭ مُّهِينٌۭ ﴿٩﴾\\
\textamh{10.\  } & مِّن وَرَآئِهِمْ جَهَنَّمُ ۖ وَلَا يُغْنِى عَنْهُم مَّا كَسَبُوا۟ شَيْـًۭٔا وَلَا مَا ٱتَّخَذُوا۟ مِن دُونِ ٱللَّهِ أَوْلِيَآءَ ۖ وَلَهُمْ عَذَابٌ عَظِيمٌ ﴿١٠﴾\\
\textamh{11.\  } & هَـٰذَا هُدًۭى ۖ وَٱلَّذِينَ كَفَرُوا۟ بِـَٔايَـٰتِ رَبِّهِمْ لَهُمْ عَذَابٌۭ مِّن رِّجْزٍ أَلِيمٌ ﴿١١﴾\\
\textamh{12.\  } & ۞ ٱللَّهُ ٱلَّذِى سَخَّرَ لَكُمُ ٱلْبَحْرَ لِتَجْرِىَ ٱلْفُلْكُ فِيهِ بِأَمْرِهِۦ وَلِتَبْتَغُوا۟ مِن فَضْلِهِۦ وَلَعَلَّكُمْ تَشْكُرُونَ ﴿١٢﴾\\
\textamh{13.\  } & وَسَخَّرَ لَكُم مَّا فِى ٱلسَّمَـٰوَٟتِ وَمَا فِى ٱلْأَرْضِ جَمِيعًۭا مِّنْهُ ۚ إِنَّ فِى ذَٟلِكَ لَءَايَـٰتٍۢ لِّقَوْمٍۢ يَتَفَكَّرُونَ ﴿١٣﴾\\
\textamh{14.\  } & قُل لِّلَّذِينَ ءَامَنُوا۟ يَغْفِرُوا۟ لِلَّذِينَ لَا يَرْجُونَ أَيَّامَ ٱللَّهِ لِيَجْزِىَ قَوْمًۢا بِمَا كَانُوا۟ يَكْسِبُونَ ﴿١٤﴾\\
\textamh{15.\  } & مَنْ عَمِلَ صَـٰلِحًۭا فَلِنَفْسِهِۦ ۖ وَمَنْ أَسَآءَ فَعَلَيْهَا ۖ ثُمَّ إِلَىٰ رَبِّكُمْ تُرْجَعُونَ ﴿١٥﴾\\
\textamh{16.\  } & وَلَقَدْ ءَاتَيْنَا بَنِىٓ إِسْرَٰٓءِيلَ ٱلْكِتَـٰبَ وَٱلْحُكْمَ وَٱلنُّبُوَّةَ وَرَزَقْنَـٰهُم مِّنَ ٱلطَّيِّبَٰتِ وَفَضَّلْنَـٰهُمْ عَلَى ٱلْعَـٰلَمِينَ ﴿١٦﴾\\
\textamh{17.\  } & وَءَاتَيْنَـٰهُم بَيِّنَـٰتٍۢ مِّنَ ٱلْأَمْرِ ۖ فَمَا ٱخْتَلَفُوٓا۟ إِلَّا مِنۢ بَعْدِ مَا جَآءَهُمُ ٱلْعِلْمُ بَغْيًۢا بَيْنَهُمْ ۚ إِنَّ رَبَّكَ يَقْضِى بَيْنَهُمْ يَوْمَ ٱلْقِيَـٰمَةِ فِيمَا كَانُوا۟ فِيهِ يَخْتَلِفُونَ ﴿١٧﴾\\
\textamh{18.\  } & ثُمَّ جَعَلْنَـٰكَ عَلَىٰ شَرِيعَةٍۢ مِّنَ ٱلْأَمْرِ فَٱتَّبِعْهَا وَلَا تَتَّبِعْ أَهْوَآءَ ٱلَّذِينَ لَا يَعْلَمُونَ ﴿١٨﴾\\
\textamh{19.\  } & إِنَّهُمْ لَن يُغْنُوا۟ عَنكَ مِنَ ٱللَّهِ شَيْـًۭٔا ۚ وَإِنَّ ٱلظَّـٰلِمِينَ بَعْضُهُمْ أَوْلِيَآءُ بَعْضٍۢ ۖ وَٱللَّهُ وَلِىُّ ٱلْمُتَّقِينَ ﴿١٩﴾\\
\textamh{20.\  } & هَـٰذَا بَصَـٰٓئِرُ لِلنَّاسِ وَهُدًۭى وَرَحْمَةٌۭ لِّقَوْمٍۢ يُوقِنُونَ ﴿٢٠﴾\\
\textamh{21.\  } & أَمْ حَسِبَ ٱلَّذِينَ ٱجْتَرَحُوا۟ ٱلسَّيِّـَٔاتِ أَن نَّجْعَلَهُمْ كَٱلَّذِينَ ءَامَنُوا۟ وَعَمِلُوا۟ ٱلصَّـٰلِحَـٰتِ سَوَآءًۭ مَّحْيَاهُمْ وَمَمَاتُهُمْ ۚ سَآءَ مَا يَحْكُمُونَ ﴿٢١﴾\\
\textamh{22.\  } & وَخَلَقَ ٱللَّهُ ٱلسَّمَـٰوَٟتِ وَٱلْأَرْضَ بِٱلْحَقِّ وَلِتُجْزَىٰ كُلُّ نَفْسٍۭ بِمَا كَسَبَتْ وَهُمْ لَا يُظْلَمُونَ ﴿٢٢﴾\\
\textamh{23.\  } & أَفَرَءَيْتَ مَنِ ٱتَّخَذَ إِلَـٰهَهُۥ هَوَىٰهُ وَأَضَلَّهُ ٱللَّهُ عَلَىٰ عِلْمٍۢ وَخَتَمَ عَلَىٰ سَمْعِهِۦ وَقَلْبِهِۦ وَجَعَلَ عَلَىٰ بَصَرِهِۦ غِشَـٰوَةًۭ فَمَن يَهْدِيهِ مِنۢ بَعْدِ ٱللَّهِ ۚ أَفَلَا تَذَكَّرُونَ ﴿٢٣﴾\\
\textamh{24.\  } & وَقَالُوا۟ مَا هِىَ إِلَّا حَيَاتُنَا ٱلدُّنْيَا نَمُوتُ وَنَحْيَا وَمَا يُهْلِكُنَآ إِلَّا ٱلدَّهْرُ ۚ وَمَا لَهُم بِذَٟلِكَ مِنْ عِلْمٍ ۖ إِنْ هُمْ إِلَّا يَظُنُّونَ ﴿٢٤﴾\\
\textamh{25.\  } & وَإِذَا تُتْلَىٰ عَلَيْهِمْ ءَايَـٰتُنَا بَيِّنَـٰتٍۢ مَّا كَانَ حُجَّتَهُمْ إِلَّآ أَن قَالُوا۟ ٱئْتُوا۟ بِـَٔابَآئِنَآ إِن كُنتُمْ صَـٰدِقِينَ ﴿٢٥﴾\\
\textamh{26.\  } & قُلِ ٱللَّهُ يُحْيِيكُمْ ثُمَّ يُمِيتُكُمْ ثُمَّ يَجْمَعُكُمْ إِلَىٰ يَوْمِ ٱلْقِيَـٰمَةِ لَا رَيْبَ فِيهِ وَلَـٰكِنَّ أَكْثَرَ ٱلنَّاسِ لَا يَعْلَمُونَ ﴿٢٦﴾\\
\textamh{27.\  } & وَلِلَّهِ مُلْكُ ٱلسَّمَـٰوَٟتِ وَٱلْأَرْضِ ۚ وَيَوْمَ تَقُومُ ٱلسَّاعَةُ يَوْمَئِذٍۢ يَخْسَرُ ٱلْمُبْطِلُونَ ﴿٢٧﴾\\
\textamh{28.\  } & وَتَرَىٰ كُلَّ أُمَّةٍۢ جَاثِيَةًۭ ۚ كُلُّ أُمَّةٍۢ تُدْعَىٰٓ إِلَىٰ كِتَـٰبِهَا ٱلْيَوْمَ تُجْزَوْنَ مَا كُنتُمْ تَعْمَلُونَ ﴿٢٨﴾\\
\textamh{29.\  } & هَـٰذَا كِتَـٰبُنَا يَنطِقُ عَلَيْكُم بِٱلْحَقِّ ۚ إِنَّا كُنَّا نَسْتَنسِخُ مَا كُنتُمْ تَعْمَلُونَ ﴿٢٩﴾\\
\textamh{30.\  } & فَأَمَّا ٱلَّذِينَ ءَامَنُوا۟ وَعَمِلُوا۟ ٱلصَّـٰلِحَـٰتِ فَيُدْخِلُهُمْ رَبُّهُمْ فِى رَحْمَتِهِۦ ۚ ذَٟلِكَ هُوَ ٱلْفَوْزُ ٱلْمُبِينُ ﴿٣٠﴾\\
\textamh{31.\  } & وَأَمَّا ٱلَّذِينَ كَفَرُوٓا۟ أَفَلَمْ تَكُنْ ءَايَـٰتِى تُتْلَىٰ عَلَيْكُمْ فَٱسْتَكْبَرْتُمْ وَكُنتُمْ قَوْمًۭا مُّجْرِمِينَ ﴿٣١﴾\\
\textamh{32.\  } & وَإِذَا قِيلَ إِنَّ وَعْدَ ٱللَّهِ حَقٌّۭ وَٱلسَّاعَةُ لَا رَيْبَ فِيهَا قُلْتُم مَّا نَدْرِى مَا ٱلسَّاعَةُ إِن نَّظُنُّ إِلَّا ظَنًّۭا وَمَا نَحْنُ بِمُسْتَيْقِنِينَ ﴿٣٢﴾\\
\textamh{33.\  } & وَبَدَا لَهُمْ سَيِّـَٔاتُ مَا عَمِلُوا۟ وَحَاقَ بِهِم مَّا كَانُوا۟ بِهِۦ يَسْتَهْزِءُونَ ﴿٣٣﴾\\
\textamh{34.\  } & وَقِيلَ ٱلْيَوْمَ نَنسَىٰكُمْ كَمَا نَسِيتُمْ لِقَآءَ يَوْمِكُمْ هَـٰذَا وَمَأْوَىٰكُمُ ٱلنَّارُ وَمَا لَكُم مِّن نَّـٰصِرِينَ ﴿٣٤﴾\\
\textamh{35.\  } & ذَٟلِكُم بِأَنَّكُمُ ٱتَّخَذْتُمْ ءَايَـٰتِ ٱللَّهِ هُزُوًۭا وَغَرَّتْكُمُ ٱلْحَيَوٰةُ ٱلدُّنْيَا ۚ فَٱلْيَوْمَ لَا يُخْرَجُونَ مِنْهَا وَلَا هُمْ يُسْتَعْتَبُونَ ﴿٣٥﴾\\
\textamh{36.\  } & فَلِلَّهِ ٱلْحَمْدُ رَبِّ ٱلسَّمَـٰوَٟتِ وَرَبِّ ٱلْأَرْضِ رَبِّ ٱلْعَـٰلَمِينَ ﴿٣٦﴾\\
\textamh{37.\  } & وَلَهُ ٱلْكِبْرِيَآءُ فِى ٱلسَّمَـٰوَٟتِ وَٱلْأَرْضِ ۖ وَهُوَ ٱلْعَزِيزُ ٱلْحَكِيمُ ﴿٣٧﴾\\
\end{longtable} \newpage

%% License: BSD style (Berkley) (i.e. Put the Copyright owner's name always)
%% Writer and Copyright (to): Bewketu(Bilal) Tadilo (2016-17)
\begin{center}\section{\LR{\textamhsec{ሱራቱ አልኣህቃፍ -}  \textarabic{سوره  الأحقاف}}}\end{center}
\begin{longtable}{%
  @{}
    p{.5\textwidth}
  @{~~~}
    p{.5\textwidth}
    @{}
}
\textamh{ቢስሚላሂ አራህመኒ ራሂይም } &  \mytextarabic{بِسْمِ ٱللَّهِ ٱلرَّحْمَـٰنِ ٱلرَّحِيمِ}\\
\textamh{1.\  } & \mytextarabic{ حمٓ ﴿١﴾}\\
\textamh{2.\  } & \mytextarabic{تَنزِيلُ ٱلْكِتَـٰبِ مِنَ ٱللَّهِ ٱلْعَزِيزِ ٱلْحَكِيمِ ﴿٢﴾}\\
\textamh{3.\  } & \mytextarabic{مَا خَلَقْنَا ٱلسَّمَـٰوَٟتِ وَٱلْأَرْضَ وَمَا بَيْنَهُمَآ إِلَّا بِٱلْحَقِّ وَأَجَلٍۢ مُّسَمًّۭى ۚ وَٱلَّذِينَ كَفَرُوا۟ عَمَّآ أُنذِرُوا۟ مُعْرِضُونَ ﴿٣﴾}\\
\textamh{4.\  } & \mytextarabic{قُلْ أَرَءَيْتُم مَّا تَدْعُونَ مِن دُونِ ٱللَّهِ أَرُونِى مَاذَا خَلَقُوا۟ مِنَ ٱلْأَرْضِ أَمْ لَهُمْ شِرْكٌۭ فِى ٱلسَّمَـٰوَٟتِ ۖ ٱئْتُونِى بِكِتَـٰبٍۢ مِّن قَبْلِ هَـٰذَآ أَوْ أَثَـٰرَةٍۢ مِّنْ عِلْمٍ إِن كُنتُمْ صَـٰدِقِينَ ﴿٤﴾}\\
\textamh{5.\  } & \mytextarabic{وَمَنْ أَضَلُّ مِمَّن يَدْعُوا۟ مِن دُونِ ٱللَّهِ مَن لَّا يَسْتَجِيبُ لَهُۥٓ إِلَىٰ يَوْمِ ٱلْقِيَـٰمَةِ وَهُمْ عَن دُعَآئِهِمْ غَٰفِلُونَ ﴿٥﴾}\\
\textamh{6.\  } & \mytextarabic{وَإِذَا حُشِرَ ٱلنَّاسُ كَانُوا۟ لَهُمْ أَعْدَآءًۭ وَكَانُوا۟ بِعِبَادَتِهِمْ كَـٰفِرِينَ ﴿٦﴾}\\
\textamh{7.\  } & \mytextarabic{وَإِذَا تُتْلَىٰ عَلَيْهِمْ ءَايَـٰتُنَا بَيِّنَـٰتٍۢ قَالَ ٱلَّذِينَ كَفَرُوا۟ لِلْحَقِّ لَمَّا جَآءَهُمْ هَـٰذَا سِحْرٌۭ مُّبِينٌ ﴿٧﴾}\\
\textamh{8.\  } & \mytextarabic{أَمْ يَقُولُونَ ٱفْتَرَىٰهُ ۖ قُلْ إِنِ ٱفْتَرَيْتُهُۥ فَلَا تَمْلِكُونَ لِى مِنَ ٱللَّهِ شَيْـًٔا ۖ هُوَ أَعْلَمُ بِمَا تُفِيضُونَ فِيهِ ۖ كَفَىٰ بِهِۦ شَهِيدًۢا بَيْنِى وَبَيْنَكُمْ ۖ وَهُوَ ٱلْغَفُورُ ٱلرَّحِيمُ ﴿٨﴾}\\
\textamh{9.\  } & \mytextarabic{قُلْ مَا كُنتُ بِدْعًۭا مِّنَ ٱلرُّسُلِ وَمَآ أَدْرِى مَا يُفْعَلُ بِى وَلَا بِكُمْ ۖ إِنْ أَتَّبِعُ إِلَّا مَا يُوحَىٰٓ إِلَىَّ وَمَآ أَنَا۠ إِلَّا نَذِيرٌۭ مُّبِينٌۭ ﴿٩﴾}\\
\textamh{10.\  } & \mytextarabic{قُلْ أَرَءَيْتُمْ إِن كَانَ مِنْ عِندِ ٱللَّهِ وَكَفَرْتُم بِهِۦ وَشَهِدَ شَاهِدٌۭ مِّنۢ بَنِىٓ إِسْرَٰٓءِيلَ عَلَىٰ مِثْلِهِۦ فَـَٔامَنَ وَٱسْتَكْبَرْتُمْ ۖ إِنَّ ٱللَّهَ لَا يَهْدِى ٱلْقَوْمَ ٱلظَّـٰلِمِينَ ﴿١٠﴾}\\
\textamh{11.\  } & \mytextarabic{وَقَالَ ٱلَّذِينَ كَفَرُوا۟ لِلَّذِينَ ءَامَنُوا۟ لَوْ كَانَ خَيْرًۭا مَّا سَبَقُونَآ إِلَيْهِ ۚ وَإِذْ لَمْ يَهْتَدُوا۟ بِهِۦ فَسَيَقُولُونَ هَـٰذَآ إِفْكٌۭ قَدِيمٌۭ ﴿١١﴾}\\
\textamh{12.\  } & \mytextarabic{وَمِن قَبْلِهِۦ كِتَـٰبُ مُوسَىٰٓ إِمَامًۭا وَرَحْمَةًۭ ۚ وَهَـٰذَا كِتَـٰبٌۭ مُّصَدِّقٌۭ لِّسَانًا عَرَبِيًّۭا لِّيُنذِرَ ٱلَّذِينَ ظَلَمُوا۟ وَبُشْرَىٰ لِلْمُحْسِنِينَ ﴿١٢﴾}\\
\textamh{13.\  } & \mytextarabic{إِنَّ ٱلَّذِينَ قَالُوا۟ رَبُّنَا ٱللَّهُ ثُمَّ ٱسْتَقَـٰمُوا۟ فَلَا خَوْفٌ عَلَيْهِمْ وَلَا هُمْ يَحْزَنُونَ ﴿١٣﴾}\\
\textamh{14.\  } & \mytextarabic{أُو۟لَـٰٓئِكَ أَصْحَـٰبُ ٱلْجَنَّةِ خَـٰلِدِينَ فِيهَا جَزَآءًۢ بِمَا كَانُوا۟ يَعْمَلُونَ ﴿١٤﴾}\\
\textamh{15.\  } & \mytextarabic{وَوَصَّيْنَا ٱلْإِنسَـٰنَ بِوَٟلِدَيْهِ إِحْسَـٰنًا ۖ حَمَلَتْهُ أُمُّهُۥ كُرْهًۭا وَوَضَعَتْهُ كُرْهًۭا ۖ وَحَمْلُهُۥ وَفِصَـٰلُهُۥ ثَلَـٰثُونَ شَهْرًا ۚ حَتَّىٰٓ إِذَا بَلَغَ أَشُدَّهُۥ وَبَلَغَ أَرْبَعِينَ سَنَةًۭ قَالَ رَبِّ أَوْزِعْنِىٓ أَنْ أَشْكُرَ نِعْمَتَكَ ٱلَّتِىٓ أَنْعَمْتَ عَلَىَّ وَعَلَىٰ وَٟلِدَىَّ وَأَنْ أَعْمَلَ صَـٰلِحًۭا تَرْضَىٰهُ وَأَصْلِحْ لِى فِى ذُرِّيَّتِىٓ ۖ إِنِّى تُبْتُ إِلَيْكَ وَإِنِّى مِنَ ٱلْمُسْلِمِينَ ﴿١٥﴾}\\
\textamh{16.\  } & \mytextarabic{أُو۟لَـٰٓئِكَ ٱلَّذِينَ نَتَقَبَّلُ عَنْهُمْ أَحْسَنَ مَا عَمِلُوا۟ وَنَتَجَاوَزُ عَن سَيِّـَٔاتِهِمْ فِىٓ أَصْحَـٰبِ ٱلْجَنَّةِ ۖ وَعْدَ ٱلصِّدْقِ ٱلَّذِى كَانُوا۟ يُوعَدُونَ ﴿١٦﴾}\\
\textamh{17.\  } & \mytextarabic{وَٱلَّذِى قَالَ لِوَٟلِدَيْهِ أُفٍّۢ لَّكُمَآ أَتَعِدَانِنِىٓ أَنْ أُخْرَجَ وَقَدْ خَلَتِ ٱلْقُرُونُ مِن قَبْلِى وَهُمَا يَسْتَغِيثَانِ ٱللَّهَ وَيْلَكَ ءَامِنْ إِنَّ وَعْدَ ٱللَّهِ حَقٌّۭ فَيَقُولُ مَا هَـٰذَآ إِلَّآ أَسَـٰطِيرُ ٱلْأَوَّلِينَ ﴿١٧﴾}\\
\textamh{18.\  } & \mytextarabic{أُو۟لَـٰٓئِكَ ٱلَّذِينَ حَقَّ عَلَيْهِمُ ٱلْقَوْلُ فِىٓ أُمَمٍۢ قَدْ خَلَتْ مِن قَبْلِهِم مِّنَ ٱلْجِنِّ وَٱلْإِنسِ ۖ إِنَّهُمْ كَانُوا۟ خَـٰسِرِينَ ﴿١٨﴾}\\
\textamh{19.\  } & \mytextarabic{وَلِكُلٍّۢ دَرَجَٰتٌۭ مِّمَّا عَمِلُوا۟ ۖ وَلِيُوَفِّيَهُمْ أَعْمَـٰلَهُمْ وَهُمْ لَا يُظْلَمُونَ ﴿١٩﴾}\\
\textamh{20.\  } & \mytextarabic{وَيَوْمَ يُعْرَضُ ٱلَّذِينَ كَفَرُوا۟ عَلَى ٱلنَّارِ أَذْهَبْتُمْ طَيِّبَٰتِكُمْ فِى حَيَاتِكُمُ ٱلدُّنْيَا وَٱسْتَمْتَعْتُم بِهَا فَٱلْيَوْمَ تُجْزَوْنَ عَذَابَ ٱلْهُونِ بِمَا كُنتُمْ تَسْتَكْبِرُونَ فِى ٱلْأَرْضِ بِغَيْرِ ٱلْحَقِّ وَبِمَا كُنتُمْ تَفْسُقُونَ ﴿٢٠﴾}\\
\textamh{21.\  } & \mytextarabic{۞ وَٱذْكُرْ أَخَا عَادٍ إِذْ أَنذَرَ قَوْمَهُۥ بِٱلْأَحْقَافِ وَقَدْ خَلَتِ ٱلنُّذُرُ مِنۢ بَيْنِ يَدَيْهِ وَمِنْ خَلْفِهِۦٓ أَلَّا تَعْبُدُوٓا۟ إِلَّا ٱللَّهَ إِنِّىٓ أَخَافُ عَلَيْكُمْ عَذَابَ يَوْمٍ عَظِيمٍۢ ﴿٢١﴾}\\
\textamh{22.\  } & \mytextarabic{قَالُوٓا۟ أَجِئْتَنَا لِتَأْفِكَنَا عَنْ ءَالِهَتِنَا فَأْتِنَا بِمَا تَعِدُنَآ إِن كُنتَ مِنَ ٱلصَّـٰدِقِينَ ﴿٢٢﴾}\\
\textamh{23.\  } & \mytextarabic{قَالَ إِنَّمَا ٱلْعِلْمُ عِندَ ٱللَّهِ وَأُبَلِّغُكُم مَّآ أُرْسِلْتُ بِهِۦ وَلَـٰكِنِّىٓ أَرَىٰكُمْ قَوْمًۭا تَجْهَلُونَ ﴿٢٣﴾}\\
\textamh{24.\  } & \mytextarabic{فَلَمَّا رَأَوْهُ عَارِضًۭا مُّسْتَقْبِلَ أَوْدِيَتِهِمْ قَالُوا۟ هَـٰذَا عَارِضٌۭ مُّمْطِرُنَا ۚ بَلْ هُوَ مَا ٱسْتَعْجَلْتُم بِهِۦ ۖ رِيحٌۭ فِيهَا عَذَابٌ أَلِيمٌۭ ﴿٢٤﴾}\\
\textamh{25.\  } & \mytextarabic{تُدَمِّرُ كُلَّ شَىْءٍۭ بِأَمْرِ رَبِّهَا فَأَصْبَحُوا۟ لَا يُرَىٰٓ إِلَّا مَسَـٰكِنُهُمْ ۚ كَذَٟلِكَ نَجْزِى ٱلْقَوْمَ ٱلْمُجْرِمِينَ ﴿٢٥﴾}\\
\textamh{26.\  } & \mytextarabic{وَلَقَدْ مَكَّنَّـٰهُمْ فِيمَآ إِن مَّكَّنَّـٰكُمْ فِيهِ وَجَعَلْنَا لَهُمْ سَمْعًۭا وَأَبْصَـٰرًۭا وَأَفْـِٔدَةًۭ فَمَآ أَغْنَىٰ عَنْهُمْ سَمْعُهُمْ وَلَآ أَبْصَـٰرُهُمْ وَلَآ أَفْـِٔدَتُهُم مِّن شَىْءٍ إِذْ كَانُوا۟ يَجْحَدُونَ بِـَٔايَـٰتِ ٱللَّهِ وَحَاقَ بِهِم مَّا كَانُوا۟ بِهِۦ يَسْتَهْزِءُونَ ﴿٢٦﴾}\\
\textamh{27.\  } & \mytextarabic{وَلَقَدْ أَهْلَكْنَا مَا حَوْلَكُم مِّنَ ٱلْقُرَىٰ وَصَرَّفْنَا ٱلْءَايَـٰتِ لَعَلَّهُمْ يَرْجِعُونَ ﴿٢٧﴾}\\
\textamh{28.\  } & \mytextarabic{فَلَوْلَا نَصَرَهُمُ ٱلَّذِينَ ٱتَّخَذُوا۟ مِن دُونِ ٱللَّهِ قُرْبَانًا ءَالِهَةًۢ ۖ بَلْ ضَلُّوا۟ عَنْهُمْ ۚ وَذَٟلِكَ إِفْكُهُمْ وَمَا كَانُوا۟ يَفْتَرُونَ ﴿٢٨﴾}\\
\textamh{29.\  } & \mytextarabic{وَإِذْ صَرَفْنَآ إِلَيْكَ نَفَرًۭا مِّنَ ٱلْجِنِّ يَسْتَمِعُونَ ٱلْقُرْءَانَ فَلَمَّا حَضَرُوهُ قَالُوٓا۟ أَنصِتُوا۟ ۖ فَلَمَّا قُضِىَ وَلَّوْا۟ إِلَىٰ قَوْمِهِم مُّنذِرِينَ ﴿٢٩﴾}\\
\textamh{30.\  } & \mytextarabic{قَالُوا۟ يَـٰقَوْمَنَآ إِنَّا سَمِعْنَا كِتَـٰبًا أُنزِلَ مِنۢ بَعْدِ مُوسَىٰ مُصَدِّقًۭا لِّمَا بَيْنَ يَدَيْهِ يَهْدِىٓ إِلَى ٱلْحَقِّ وَإِلَىٰ طَرِيقٍۢ مُّسْتَقِيمٍۢ ﴿٣٠﴾}\\
\textamh{31.\  } & \mytextarabic{يَـٰقَوْمَنَآ أَجِيبُوا۟ دَاعِىَ ٱللَّهِ وَءَامِنُوا۟ بِهِۦ يَغْفِرْ لَكُم مِّن ذُنُوبِكُمْ وَيُجِرْكُم مِّنْ عَذَابٍ أَلِيمٍۢ ﴿٣١﴾}\\
\textamh{32.\  } & \mytextarabic{وَمَن لَّا يُجِبْ دَاعِىَ ٱللَّهِ فَلَيْسَ بِمُعْجِزٍۢ فِى ٱلْأَرْضِ وَلَيْسَ لَهُۥ مِن دُونِهِۦٓ أَوْلِيَآءُ ۚ أُو۟لَـٰٓئِكَ فِى ضَلَـٰلٍۢ مُّبِينٍ ﴿٣٢﴾}\\
\textamh{33.\  } & \mytextarabic{أَوَلَمْ يَرَوْا۟ أَنَّ ٱللَّهَ ٱلَّذِى خَلَقَ ٱلسَّمَـٰوَٟتِ وَٱلْأَرْضَ وَلَمْ يَعْىَ بِخَلْقِهِنَّ بِقَـٰدِرٍ عَلَىٰٓ أَن يُحْۦِىَ ٱلْمَوْتَىٰ ۚ بَلَىٰٓ إِنَّهُۥ عَلَىٰ كُلِّ شَىْءٍۢ قَدِيرٌۭ ﴿٣٣﴾}\\
\textamh{34.\  } & \mytextarabic{وَيَوْمَ يُعْرَضُ ٱلَّذِينَ كَفَرُوا۟ عَلَى ٱلنَّارِ أَلَيْسَ هَـٰذَا بِٱلْحَقِّ ۖ قَالُوا۟ بَلَىٰ وَرَبِّنَا ۚ قَالَ فَذُوقُوا۟ ٱلْعَذَابَ بِمَا كُنتُمْ تَكْفُرُونَ ﴿٣٤﴾}\\
\textamh{35.\  } & \mytextarabic{فَٱصْبِرْ كَمَا صَبَرَ أُو۟لُوا۟ ٱلْعَزْمِ مِنَ ٱلرُّسُلِ وَلَا تَسْتَعْجِل لَّهُمْ ۚ كَأَنَّهُمْ يَوْمَ يَرَوْنَ مَا يُوعَدُونَ لَمْ يَلْبَثُوٓا۟ إِلَّا سَاعَةًۭ مِّن نَّهَارٍۭ ۚ بَلَـٰغٌۭ ۚ فَهَلْ يُهْلَكُ إِلَّا ٱلْقَوْمُ ٱلْفَـٰسِقُونَ ﴿٣٥﴾}\\
\end{longtable}
\clearpage
%% License: BSD style (Berkley) (i.e. Put the Copyright owner's name always)
%% Writer and Copyright (to): Bewketu(Bilal) Tadilo (2016-17)
\begin{center}\section{\LR{\textamhsec{ሱራቱ ሙሐመድ -}  \textarabic{سوره  محمد}}}\end{center}
\begin{longtable}{%
  @{}
    p{.5\textwidth}
  @{~~~}
    p{.5\textwidth}
    @{}
}
\textamh{ቢስሚላሂ አራህመኒ ራሂይም } &  \mytextarabic{بِسْمِ ٱللَّهِ ٱلرَّحْمَـٰنِ ٱلرَّحِيمِ}\\
\textamh{1.\  } & \mytextarabic{ ٱلَّذِينَ كَفَرُوا۟ وَصَدُّوا۟ عَن سَبِيلِ ٱللَّهِ أَضَلَّ أَعْمَـٰلَهُمْ ﴿١﴾}\\
\textamh{2.\  } & \mytextarabic{وَٱلَّذِينَ ءَامَنُوا۟ وَعَمِلُوا۟ ٱلصَّـٰلِحَـٰتِ وَءَامَنُوا۟ بِمَا نُزِّلَ عَلَىٰ مُحَمَّدٍۢ وَهُوَ ٱلْحَقُّ مِن رَّبِّهِمْ ۙ كَفَّرَ عَنْهُمْ سَيِّـَٔاتِهِمْ وَأَصْلَحَ بَالَهُمْ ﴿٢﴾}\\
\textamh{3.\  } & \mytextarabic{ذَٟلِكَ بِأَنَّ ٱلَّذِينَ كَفَرُوا۟ ٱتَّبَعُوا۟ ٱلْبَٰطِلَ وَأَنَّ ٱلَّذِينَ ءَامَنُوا۟ ٱتَّبَعُوا۟ ٱلْحَقَّ مِن رَّبِّهِمْ ۚ كَذَٟلِكَ يَضْرِبُ ٱللَّهُ لِلنَّاسِ أَمْثَـٰلَهُمْ ﴿٣﴾}\\
\textamh{4.\  } & \mytextarabic{فَإِذَا لَقِيتُمُ ٱلَّذِينَ كَفَرُوا۟ فَضَرْبَ ٱلرِّقَابِ حَتَّىٰٓ إِذَآ أَثْخَنتُمُوهُمْ فَشُدُّوا۟ ٱلْوَثَاقَ فَإِمَّا مَنًّۢا بَعْدُ وَإِمَّا فِدَآءً حَتَّىٰ تَضَعَ ٱلْحَرْبُ أَوْزَارَهَا ۚ ذَٟلِكَ وَلَوْ يَشَآءُ ٱللَّهُ لَٱنتَصَرَ مِنْهُمْ وَلَـٰكِن لِّيَبْلُوَا۟ بَعْضَكُم بِبَعْضٍۢ ۗ وَٱلَّذِينَ قُتِلُوا۟ فِى سَبِيلِ ٱللَّهِ فَلَن يُضِلَّ أَعْمَـٰلَهُمْ ﴿٤﴾}\\
\textamh{5.\  } & \mytextarabic{سَيَهْدِيهِمْ وَيُصْلِحُ بَالَهُمْ ﴿٥﴾}\\
\textamh{6.\  } & \mytextarabic{وَيُدْخِلُهُمُ ٱلْجَنَّةَ عَرَّفَهَا لَهُمْ ﴿٦﴾}\\
\textamh{7.\  } & \mytextarabic{يَـٰٓأَيُّهَا ٱلَّذِينَ ءَامَنُوٓا۟ إِن تَنصُرُوا۟ ٱللَّهَ يَنصُرْكُمْ وَيُثَبِّتْ أَقْدَامَكُمْ ﴿٧﴾}\\
\textamh{8.\  } & \mytextarabic{وَٱلَّذِينَ كَفَرُوا۟ فَتَعْسًۭا لَّهُمْ وَأَضَلَّ أَعْمَـٰلَهُمْ ﴿٨﴾}\\
\textamh{9.\  } & \mytextarabic{ذَٟلِكَ بِأَنَّهُمْ كَرِهُوا۟ مَآ أَنزَلَ ٱللَّهُ فَأَحْبَطَ أَعْمَـٰلَهُمْ ﴿٩﴾}\\
\textamh{10.\  } & \mytextarabic{۞ أَفَلَمْ يَسِيرُوا۟ فِى ٱلْأَرْضِ فَيَنظُرُوا۟ كَيْفَ كَانَ عَـٰقِبَةُ ٱلَّذِينَ مِن قَبْلِهِمْ ۚ دَمَّرَ ٱللَّهُ عَلَيْهِمْ ۖ وَلِلْكَـٰفِرِينَ أَمْثَـٰلُهَا ﴿١٠﴾}\\
\textamh{11.\  } & \mytextarabic{ذَٟلِكَ بِأَنَّ ٱللَّهَ مَوْلَى ٱلَّذِينَ ءَامَنُوا۟ وَأَنَّ ٱلْكَـٰفِرِينَ لَا مَوْلَىٰ لَهُمْ ﴿١١﴾}\\
\textamh{12.\  } & \mytextarabic{إِنَّ ٱللَّهَ يُدْخِلُ ٱلَّذِينَ ءَامَنُوا۟ وَعَمِلُوا۟ ٱلصَّـٰلِحَـٰتِ جَنَّـٰتٍۢ تَجْرِى مِن تَحْتِهَا ٱلْأَنْهَـٰرُ ۖ وَٱلَّذِينَ كَفَرُوا۟ يَتَمَتَّعُونَ وَيَأْكُلُونَ كَمَا تَأْكُلُ ٱلْأَنْعَـٰمُ وَٱلنَّارُ مَثْوًۭى لَّهُمْ ﴿١٢﴾}\\
\textamh{13.\  } & \mytextarabic{وَكَأَيِّن مِّن قَرْيَةٍ هِىَ أَشَدُّ قُوَّةًۭ مِّن قَرْيَتِكَ ٱلَّتِىٓ أَخْرَجَتْكَ أَهْلَكْنَـٰهُمْ فَلَا نَاصِرَ لَهُمْ ﴿١٣﴾}\\
\textamh{14.\  } & \mytextarabic{أَفَمَن كَانَ عَلَىٰ بَيِّنَةٍۢ مِّن رَّبِّهِۦ كَمَن زُيِّنَ لَهُۥ سُوٓءُ عَمَلِهِۦ وَٱتَّبَعُوٓا۟ أَهْوَآءَهُم ﴿١٤﴾}\\
\textamh{15.\  } & \mytextarabic{مَّثَلُ ٱلْجَنَّةِ ٱلَّتِى وُعِدَ ٱلْمُتَّقُونَ ۖ فِيهَآ أَنْهَـٰرٌۭ مِّن مَّآءٍ غَيْرِ ءَاسِنٍۢ وَأَنْهَـٰرٌۭ مِّن لَّبَنٍۢ لَّمْ يَتَغَيَّرْ طَعْمُهُۥ وَأَنْهَـٰرٌۭ مِّنْ خَمْرٍۢ لَّذَّةٍۢ لِّلشَّـٰرِبِينَ وَأَنْهَـٰرٌۭ مِّنْ عَسَلٍۢ مُّصَفًّۭى ۖ وَلَهُمْ فِيهَا مِن كُلِّ ٱلثَّمَرَٰتِ وَمَغْفِرَةٌۭ مِّن رَّبِّهِمْ ۖ كَمَنْ هُوَ خَـٰلِدٌۭ فِى ٱلنَّارِ وَسُقُوا۟ مَآءً حَمِيمًۭا فَقَطَّعَ أَمْعَآءَهُمْ ﴿١٥﴾}\\
\textamh{16.\  } & \mytextarabic{وَمِنْهُم مَّن يَسْتَمِعُ إِلَيْكَ حَتَّىٰٓ إِذَا خَرَجُوا۟ مِنْ عِندِكَ قَالُوا۟ لِلَّذِينَ أُوتُوا۟ ٱلْعِلْمَ مَاذَا قَالَ ءَانِفًا ۚ أُو۟لَـٰٓئِكَ ٱلَّذِينَ طَبَعَ ٱللَّهُ عَلَىٰ قُلُوبِهِمْ وَٱتَّبَعُوٓا۟ أَهْوَآءَهُمْ ﴿١٦﴾}\\
\textamh{17.\  } & \mytextarabic{وَٱلَّذِينَ ٱهْتَدَوْا۟ زَادَهُمْ هُدًۭى وَءَاتَىٰهُمْ تَقْوَىٰهُمْ ﴿١٧﴾}\\
\textamh{18.\  } & \mytextarabic{فَهَلْ يَنظُرُونَ إِلَّا ٱلسَّاعَةَ أَن تَأْتِيَهُم بَغْتَةًۭ ۖ فَقَدْ جَآءَ أَشْرَاطُهَا ۚ فَأَنَّىٰ لَهُمْ إِذَا جَآءَتْهُمْ ذِكْرَىٰهُمْ ﴿١٨﴾}\\
\textamh{19.\  } & \mytextarabic{فَٱعْلَمْ أَنَّهُۥ لَآ إِلَـٰهَ إِلَّا ٱللَّهُ وَٱسْتَغْفِرْ لِذَنۢبِكَ وَلِلْمُؤْمِنِينَ وَٱلْمُؤْمِنَـٰتِ ۗ وَٱللَّهُ يَعْلَمُ مُتَقَلَّبَكُمْ وَمَثْوَىٰكُمْ ﴿١٩﴾}\\
\textamh{20.\  } & \mytextarabic{وَيَقُولُ ٱلَّذِينَ ءَامَنُوا۟ لَوْلَا نُزِّلَتْ سُورَةٌۭ ۖ فَإِذَآ أُنزِلَتْ سُورَةٌۭ مُّحْكَمَةٌۭ وَذُكِرَ فِيهَا ٱلْقِتَالُ ۙ رَأَيْتَ ٱلَّذِينَ فِى قُلُوبِهِم مَّرَضٌۭ يَنظُرُونَ إِلَيْكَ نَظَرَ ٱلْمَغْشِىِّ عَلَيْهِ مِنَ ٱلْمَوْتِ ۖ فَأَوْلَىٰ لَهُمْ ﴿٢٠﴾}\\
\textamh{21.\  } & \mytextarabic{طَاعَةٌۭ وَقَوْلٌۭ مَّعْرُوفٌۭ ۚ فَإِذَا عَزَمَ ٱلْأَمْرُ فَلَوْ صَدَقُوا۟ ٱللَّهَ لَكَانَ خَيْرًۭا لَّهُمْ ﴿٢١﴾}\\
\textamh{22.\  } & \mytextarabic{فَهَلْ عَسَيْتُمْ إِن تَوَلَّيْتُمْ أَن تُفْسِدُوا۟ فِى ٱلْأَرْضِ وَتُقَطِّعُوٓا۟ أَرْحَامَكُمْ ﴿٢٢﴾}\\
\textamh{23.\  } & \mytextarabic{أُو۟لَـٰٓئِكَ ٱلَّذِينَ لَعَنَهُمُ ٱللَّهُ فَأَصَمَّهُمْ وَأَعْمَىٰٓ أَبْصَـٰرَهُمْ ﴿٢٣﴾}\\
\textamh{24.\  } & \mytextarabic{أَفَلَا يَتَدَبَّرُونَ ٱلْقُرْءَانَ أَمْ عَلَىٰ قُلُوبٍ أَقْفَالُهَآ ﴿٢٤﴾}\\
\textamh{25.\  } & \mytextarabic{إِنَّ ٱلَّذِينَ ٱرْتَدُّوا۟ عَلَىٰٓ أَدْبَٰرِهِم مِّنۢ بَعْدِ مَا تَبَيَّنَ لَهُمُ ٱلْهُدَى ۙ ٱلشَّيْطَٰنُ سَوَّلَ لَهُمْ وَأَمْلَىٰ لَهُمْ ﴿٢٥﴾}\\
\textamh{26.\  } & \mytextarabic{ذَٟلِكَ بِأَنَّهُمْ قَالُوا۟ لِلَّذِينَ كَرِهُوا۟ مَا نَزَّلَ ٱللَّهُ سَنُطِيعُكُمْ فِى بَعْضِ ٱلْأَمْرِ ۖ وَٱللَّهُ يَعْلَمُ إِسْرَارَهُمْ ﴿٢٦﴾}\\
\textamh{27.\  } & \mytextarabic{فَكَيْفَ إِذَا تَوَفَّتْهُمُ ٱلْمَلَـٰٓئِكَةُ يَضْرِبُونَ وُجُوهَهُمْ وَأَدْبَٰرَهُمْ ﴿٢٧﴾}\\
\textamh{28.\  } & \mytextarabic{ذَٟلِكَ بِأَنَّهُمُ ٱتَّبَعُوا۟ مَآ أَسْخَطَ ٱللَّهَ وَكَرِهُوا۟ رِضْوَٟنَهُۥ فَأَحْبَطَ أَعْمَـٰلَهُمْ ﴿٢٨﴾}\\
\textamh{29.\  } & \mytextarabic{أَمْ حَسِبَ ٱلَّذِينَ فِى قُلُوبِهِم مَّرَضٌ أَن لَّن يُخْرِجَ ٱللَّهُ أَضْغَٰنَهُمْ ﴿٢٩﴾}\\
\textamh{30.\  } & \mytextarabic{وَلَوْ نَشَآءُ لَأَرَيْنَـٰكَهُمْ فَلَعَرَفْتَهُم بِسِيمَـٰهُمْ ۚ وَلَتَعْرِفَنَّهُمْ فِى لَحْنِ ٱلْقَوْلِ ۚ وَٱللَّهُ يَعْلَمُ أَعْمَـٰلَكُمْ ﴿٣٠﴾}\\
\textamh{31.\  } & \mytextarabic{وَلَنَبْلُوَنَّكُمْ حَتَّىٰ نَعْلَمَ ٱلْمُجَٰهِدِينَ مِنكُمْ وَٱلصَّـٰبِرِينَ وَنَبْلُوَا۟ أَخْبَارَكُمْ ﴿٣١﴾}\\
\textamh{32.\  } & \mytextarabic{إِنَّ ٱلَّذِينَ كَفَرُوا۟ وَصَدُّوا۟ عَن سَبِيلِ ٱللَّهِ وَشَآقُّوا۟ ٱلرَّسُولَ مِنۢ بَعْدِ مَا تَبَيَّنَ لَهُمُ ٱلْهُدَىٰ لَن يَضُرُّوا۟ ٱللَّهَ شَيْـًۭٔا وَسَيُحْبِطُ أَعْمَـٰلَهُمْ ﴿٣٢﴾}\\
\textamh{33.\  } & \mytextarabic{۞ يَـٰٓأَيُّهَا ٱلَّذِينَ ءَامَنُوٓا۟ أَطِيعُوا۟ ٱللَّهَ وَأَطِيعُوا۟ ٱلرَّسُولَ وَلَا تُبْطِلُوٓا۟ أَعْمَـٰلَكُمْ ﴿٣٣﴾}\\
\textamh{34.\  } & \mytextarabic{إِنَّ ٱلَّذِينَ كَفَرُوا۟ وَصَدُّوا۟ عَن سَبِيلِ ٱللَّهِ ثُمَّ مَاتُوا۟ وَهُمْ كُفَّارٌۭ فَلَن يَغْفِرَ ٱللَّهُ لَهُمْ ﴿٣٤﴾}\\
\textamh{35.\  } & \mytextarabic{فَلَا تَهِنُوا۟ وَتَدْعُوٓا۟ إِلَى ٱلسَّلْمِ وَأَنتُمُ ٱلْأَعْلَوْنَ وَٱللَّهُ مَعَكُمْ وَلَن يَتِرَكُمْ أَعْمَـٰلَكُمْ ﴿٣٥﴾}\\
\textamh{36.\  } & \mytextarabic{إِنَّمَا ٱلْحَيَوٰةُ ٱلدُّنْيَا لَعِبٌۭ وَلَهْوٌۭ ۚ وَإِن تُؤْمِنُوا۟ وَتَتَّقُوا۟ يُؤْتِكُمْ أُجُورَكُمْ وَلَا يَسْـَٔلْكُمْ أَمْوَٟلَكُمْ ﴿٣٦﴾}\\
\textamh{37.\  } & \mytextarabic{إِن يَسْـَٔلْكُمُوهَا فَيُحْفِكُمْ تَبْخَلُوا۟ وَيُخْرِجْ أَضْغَٰنَكُمْ ﴿٣٧﴾}\\
\textamh{38.\  } & \mytextarabic{هَـٰٓأَنتُمْ هَـٰٓؤُلَآءِ تُدْعَوْنَ لِتُنفِقُوا۟ فِى سَبِيلِ ٱللَّهِ فَمِنكُم مَّن يَبْخَلُ ۖ وَمَن يَبْخَلْ فَإِنَّمَا يَبْخَلُ عَن نَّفْسِهِۦ ۚ وَٱللَّهُ ٱلْغَنِىُّ وَأَنتُمُ ٱلْفُقَرَآءُ ۚ وَإِن تَتَوَلَّوْا۟ يَسْتَبْدِلْ قَوْمًا غَيْرَكُمْ ثُمَّ لَا يَكُونُوٓا۟ أَمْثَـٰلَكُم ﴿٣٨﴾}\\
\end{longtable}
\clearpage
%% License: BSD style (Berkley) (i.e. Put the Copyright owner's name always)
%% Writer and Copyright (to): Bewketu(Bilal) Tadilo (2016-17)
\centering\section{\LR{\textamharic{ሱራቱ አልፈትህ -}  \RL{سوره  الفتح}}}
\begin{longtable}{%
  @{}
    p{.5\textwidth}
  @{~~~~~~~~~~~~}
    p{.5\textwidth}
    @{}
}
\nopagebreak
\textamh{ቢስሚላሂ አራህመኒ ራሂይም } &  بِسْمِ ٱللَّهِ ٱلرَّحْمَـٰنِ ٱلرَّحِيمِ\\
\textamh{1.\  } &  إِنَّا فَتَحْنَا لَكَ فَتْحًۭا مُّبِينًۭا ﴿١﴾\\
\textamh{2.\  } & لِّيَغْفِرَ لَكَ ٱللَّهُ مَا تَقَدَّمَ مِن ذَنۢبِكَ وَمَا تَأَخَّرَ وَيُتِمَّ نِعْمَتَهُۥ عَلَيْكَ وَيَهْدِيَكَ صِرَٰطًۭا مُّسْتَقِيمًۭا ﴿٢﴾\\
\textamh{3.\  } & وَيَنصُرَكَ ٱللَّهُ نَصْرًا عَزِيزًا ﴿٣﴾\\
\textamh{4.\  } & هُوَ ٱلَّذِىٓ أَنزَلَ ٱلسَّكِينَةَ فِى قُلُوبِ ٱلْمُؤْمِنِينَ لِيَزْدَادُوٓا۟ إِيمَـٰنًۭا مَّعَ إِيمَـٰنِهِمْ ۗ وَلِلَّهِ جُنُودُ ٱلسَّمَـٰوَٟتِ وَٱلْأَرْضِ ۚ وَكَانَ ٱللَّهُ عَلِيمًا حَكِيمًۭا ﴿٤﴾\\
\textamh{5.\  } & لِّيُدْخِلَ ٱلْمُؤْمِنِينَ وَٱلْمُؤْمِنَـٰتِ جَنَّـٰتٍۢ تَجْرِى مِن تَحْتِهَا ٱلْأَنْهَـٰرُ خَـٰلِدِينَ فِيهَا وَيُكَفِّرَ عَنْهُمْ سَيِّـَٔاتِهِمْ ۚ وَكَانَ ذَٟلِكَ عِندَ ٱللَّهِ فَوْزًا عَظِيمًۭا ﴿٥﴾\\
\textamh{6.\  } & وَيُعَذِّبَ ٱلْمُنَـٰفِقِينَ وَٱلْمُنَـٰفِقَـٰتِ وَٱلْمُشْرِكِينَ وَٱلْمُشْرِكَـٰتِ ٱلظَّآنِّينَ بِٱللَّهِ ظَنَّ ٱلسَّوْءِ ۚ عَلَيْهِمْ دَآئِرَةُ ٱلسَّوْءِ ۖ وَغَضِبَ ٱللَّهُ عَلَيْهِمْ وَلَعَنَهُمْ وَأَعَدَّ لَهُمْ جَهَنَّمَ ۖ وَسَآءَتْ مَصِيرًۭا ﴿٦﴾\\
\textamh{7.\  } & وَلِلَّهِ جُنُودُ ٱلسَّمَـٰوَٟتِ وَٱلْأَرْضِ ۚ وَكَانَ ٱللَّهُ عَزِيزًا حَكِيمًا ﴿٧﴾\\
\textamh{8.\  } & إِنَّآ أَرْسَلْنَـٰكَ شَـٰهِدًۭا وَمُبَشِّرًۭا وَنَذِيرًۭا ﴿٨﴾\\
\textamh{9.\  } & لِّتُؤْمِنُوا۟ بِٱللَّهِ وَرَسُولِهِۦ وَتُعَزِّرُوهُ وَتُوَقِّرُوهُ وَتُسَبِّحُوهُ بُكْرَةًۭ وَأَصِيلًا ﴿٩﴾\\
\textamh{10.\  } & إِنَّ ٱلَّذِينَ يُبَايِعُونَكَ إِنَّمَا يُبَايِعُونَ ٱللَّهَ يَدُ ٱللَّهِ فَوْقَ أَيْدِيهِمْ ۚ فَمَن نَّكَثَ فَإِنَّمَا يَنكُثُ عَلَىٰ نَفْسِهِۦ ۖ وَمَنْ أَوْفَىٰ بِمَا عَـٰهَدَ عَلَيْهُ ٱللَّهَ فَسَيُؤْتِيهِ أَجْرًا عَظِيمًۭا ﴿١٠﴾\\
\textamh{11.\  } & سَيَقُولُ لَكَ ٱلْمُخَلَّفُونَ مِنَ ٱلْأَعْرَابِ شَغَلَتْنَآ أَمْوَٟلُنَا وَأَهْلُونَا فَٱسْتَغْفِرْ لَنَا ۚ يَقُولُونَ بِأَلْسِنَتِهِم مَّا لَيْسَ فِى قُلُوبِهِمْ ۚ قُلْ فَمَن يَمْلِكُ لَكُم مِّنَ ٱللَّهِ شَيْـًٔا إِنْ أَرَادَ بِكُمْ ضَرًّا أَوْ أَرَادَ بِكُمْ نَفْعًۢا ۚ بَلْ كَانَ ٱللَّهُ بِمَا تَعْمَلُونَ خَبِيرًۢا ﴿١١﴾\\
\textamh{12.\  } & بَلْ ظَنَنتُمْ أَن لَّن يَنقَلِبَ ٱلرَّسُولُ وَٱلْمُؤْمِنُونَ إِلَىٰٓ أَهْلِيهِمْ أَبَدًۭا وَزُيِّنَ ذَٟلِكَ فِى قُلُوبِكُمْ وَظَنَنتُمْ ظَنَّ ٱلسَّوْءِ وَكُنتُمْ قَوْمًۢا بُورًۭا ﴿١٢﴾\\
\textamh{13.\  } & وَمَن لَّمْ يُؤْمِنۢ بِٱللَّهِ وَرَسُولِهِۦ فَإِنَّآ أَعْتَدْنَا لِلْكَـٰفِرِينَ سَعِيرًۭا ﴿١٣﴾\\
\textamh{14.\  } & وَلِلَّهِ مُلْكُ ٱلسَّمَـٰوَٟتِ وَٱلْأَرْضِ ۚ يَغْفِرُ لِمَن يَشَآءُ وَيُعَذِّبُ مَن يَشَآءُ ۚ وَكَانَ ٱللَّهُ غَفُورًۭا رَّحِيمًۭا ﴿١٤﴾\\
\textamh{15.\  } & سَيَقُولُ ٱلْمُخَلَّفُونَ إِذَا ٱنطَلَقْتُمْ إِلَىٰ مَغَانِمَ لِتَأْخُذُوهَا ذَرُونَا نَتَّبِعْكُمْ ۖ يُرِيدُونَ أَن يُبَدِّلُوا۟ كَلَـٰمَ ٱللَّهِ ۚ قُل لَّن تَتَّبِعُونَا كَذَٟلِكُمْ قَالَ ٱللَّهُ مِن قَبْلُ ۖ فَسَيَقُولُونَ بَلْ تَحْسُدُونَنَا ۚ بَلْ كَانُوا۟ لَا يَفْقَهُونَ إِلَّا قَلِيلًۭا ﴿١٥﴾\\
\textamh{16.\  } & قُل لِّلْمُخَلَّفِينَ مِنَ ٱلْأَعْرَابِ سَتُدْعَوْنَ إِلَىٰ قَوْمٍ أُو۟لِى بَأْسٍۢ شَدِيدٍۢ تُقَـٰتِلُونَهُمْ أَوْ يُسْلِمُونَ ۖ فَإِن تُطِيعُوا۟ يُؤْتِكُمُ ٱللَّهُ أَجْرًا حَسَنًۭا ۖ وَإِن تَتَوَلَّوْا۟ كَمَا تَوَلَّيْتُم مِّن قَبْلُ يُعَذِّبْكُمْ عَذَابًا أَلِيمًۭا ﴿١٦﴾\\
\textamh{17.\  } & لَّيْسَ عَلَى ٱلْأَعْمَىٰ حَرَجٌۭ وَلَا عَلَى ٱلْأَعْرَجِ حَرَجٌۭ وَلَا عَلَى ٱلْمَرِيضِ حَرَجٌۭ ۗ وَمَن يُطِعِ ٱللَّهَ وَرَسُولَهُۥ يُدْخِلْهُ جَنَّـٰتٍۢ تَجْرِى مِن تَحْتِهَا ٱلْأَنْهَـٰرُ ۖ وَمَن يَتَوَلَّ يُعَذِّبْهُ عَذَابًا أَلِيمًۭا ﴿١٧﴾\\
\textamh{18.\  } & ۞ لَّقَدْ رَضِىَ ٱللَّهُ عَنِ ٱلْمُؤْمِنِينَ إِذْ يُبَايِعُونَكَ تَحْتَ ٱلشَّجَرَةِ فَعَلِمَ مَا فِى قُلُوبِهِمْ فَأَنزَلَ ٱلسَّكِينَةَ عَلَيْهِمْ وَأَثَـٰبَهُمْ فَتْحًۭا قَرِيبًۭا ﴿١٨﴾\\
\textamh{19.\  } & وَمَغَانِمَ كَثِيرَةًۭ يَأْخُذُونَهَا ۗ وَكَانَ ٱللَّهُ عَزِيزًا حَكِيمًۭا ﴿١٩﴾\\
\textamh{20.\  } & وَعَدَكُمُ ٱللَّهُ مَغَانِمَ كَثِيرَةًۭ تَأْخُذُونَهَا فَعَجَّلَ لَكُمْ هَـٰذِهِۦ وَكَفَّ أَيْدِىَ ٱلنَّاسِ عَنكُمْ وَلِتَكُونَ ءَايَةًۭ لِّلْمُؤْمِنِينَ وَيَهْدِيَكُمْ صِرَٰطًۭا مُّسْتَقِيمًۭا ﴿٢٠﴾\\
\textamh{21.\  } & وَأُخْرَىٰ لَمْ تَقْدِرُوا۟ عَلَيْهَا قَدْ أَحَاطَ ٱللَّهُ بِهَا ۚ وَكَانَ ٱللَّهُ عَلَىٰ كُلِّ شَىْءٍۢ قَدِيرًۭا ﴿٢١﴾\\
\textamh{22.\  } & وَلَوْ قَـٰتَلَكُمُ ٱلَّذِينَ كَفَرُوا۟ لَوَلَّوُا۟ ٱلْأَدْبَٰرَ ثُمَّ لَا يَجِدُونَ وَلِيًّۭا وَلَا نَصِيرًۭا ﴿٢٢﴾\\
\textamh{23.\  } & سُنَّةَ ٱللَّهِ ٱلَّتِى قَدْ خَلَتْ مِن قَبْلُ ۖ وَلَن تَجِدَ لِسُنَّةِ ٱللَّهِ تَبْدِيلًۭا ﴿٢٣﴾\\
\textamh{24.\  } & وَهُوَ ٱلَّذِى كَفَّ أَيْدِيَهُمْ عَنكُمْ وَأَيْدِيَكُمْ عَنْهُم بِبَطْنِ مَكَّةَ مِنۢ بَعْدِ أَنْ أَظْفَرَكُمْ عَلَيْهِمْ ۚ وَكَانَ ٱللَّهُ بِمَا تَعْمَلُونَ بَصِيرًا ﴿٢٤﴾\\
\textamh{25.\  } & هُمُ ٱلَّذِينَ كَفَرُوا۟ وَصَدُّوكُمْ عَنِ ٱلْمَسْجِدِ ٱلْحَرَامِ وَٱلْهَدْىَ مَعْكُوفًا أَن يَبْلُغَ مَحِلَّهُۥ ۚ وَلَوْلَا رِجَالٌۭ مُّؤْمِنُونَ وَنِسَآءٌۭ مُّؤْمِنَـٰتٌۭ لَّمْ تَعْلَمُوهُمْ أَن تَطَـُٔوهُمْ فَتُصِيبَكُم مِّنْهُم مَّعَرَّةٌۢ بِغَيْرِ عِلْمٍۢ ۖ لِّيُدْخِلَ ٱللَّهُ فِى رَحْمَتِهِۦ مَن يَشَآءُ ۚ لَوْ تَزَيَّلُوا۟ لَعَذَّبْنَا ٱلَّذِينَ كَفَرُوا۟ مِنْهُمْ عَذَابًا أَلِيمًا ﴿٢٥﴾\\
\textamh{26.\  } & إِذْ جَعَلَ ٱلَّذِينَ كَفَرُوا۟ فِى قُلُوبِهِمُ ٱلْحَمِيَّةَ حَمِيَّةَ ٱلْجَٰهِلِيَّةِ فَأَنزَلَ ٱللَّهُ سَكِينَتَهُۥ عَلَىٰ رَسُولِهِۦ وَعَلَى ٱلْمُؤْمِنِينَ وَأَلْزَمَهُمْ كَلِمَةَ ٱلتَّقْوَىٰ وَكَانُوٓا۟ أَحَقَّ بِهَا وَأَهْلَهَا ۚ وَكَانَ ٱللَّهُ بِكُلِّ شَىْءٍ عَلِيمًۭا ﴿٢٦﴾\\
\textamh{27.\  } & لَّقَدْ صَدَقَ ٱللَّهُ رَسُولَهُ ٱلرُّءْيَا بِٱلْحَقِّ ۖ لَتَدْخُلُنَّ ٱلْمَسْجِدَ ٱلْحَرَامَ إِن شَآءَ ٱللَّهُ ءَامِنِينَ مُحَلِّقِينَ رُءُوسَكُمْ وَمُقَصِّرِينَ لَا تَخَافُونَ ۖ فَعَلِمَ مَا لَمْ تَعْلَمُوا۟ فَجَعَلَ مِن دُونِ ذَٟلِكَ فَتْحًۭا قَرِيبًا ﴿٢٧﴾\\
\textamh{28.\  } & هُوَ ٱلَّذِىٓ أَرْسَلَ رَسُولَهُۥ بِٱلْهُدَىٰ وَدِينِ ٱلْحَقِّ لِيُظْهِرَهُۥ عَلَى ٱلدِّينِ كُلِّهِۦ ۚ وَكَفَىٰ بِٱللَّهِ شَهِيدًۭا ﴿٢٨﴾\\
\textamh{29.\  } & مُّحَمَّدٌۭ رَّسُولُ ٱللَّهِ ۚ وَٱلَّذِينَ مَعَهُۥٓ أَشِدَّآءُ عَلَى ٱلْكُفَّارِ رُحَمَآءُ بَيْنَهُمْ ۖ تَرَىٰهُمْ رُكَّعًۭا سُجَّدًۭا يَبْتَغُونَ فَضْلًۭا مِّنَ ٱللَّهِ وَرِضْوَٟنًۭا ۖ سِيمَاهُمْ فِى وُجُوهِهِم مِّنْ أَثَرِ ٱلسُّجُودِ ۚ ذَٟلِكَ مَثَلُهُمْ فِى ٱلتَّوْرَىٰةِ ۚ وَمَثَلُهُمْ فِى ٱلْإِنجِيلِ كَزَرْعٍ أَخْرَجَ شَطْـَٔهُۥ فَـَٔازَرَهُۥ فَٱسْتَغْلَظَ فَٱسْتَوَىٰ عَلَىٰ سُوقِهِۦ يُعْجِبُ ٱلزُّرَّاعَ لِيَغِيظَ بِهِمُ ٱلْكُفَّارَ ۗ وَعَدَ ٱللَّهُ ٱلَّذِينَ ءَامَنُوا۟ وَعَمِلُوا۟ ٱلصَّـٰلِحَـٰتِ مِنْهُم مَّغْفِرَةًۭ وَأَجْرًا عَظِيمًۢا ﴿٢٩﴾\\
\end{longtable}
\clearpage
%% License: BSD style (Berkley) (i.e. Put the Copyright owner's name always)
%% Writer and Copyright (to): Bewketu(Bilal) Tadilo (2016-17)
\centering\section{\LR{\textamharic{ሱራቱ አልሁጁራት -}  \RL{سوره  الحجرات}}}
\begin{longtable}{%
  @{}
    p{.5\textwidth}
  @{~~~~~~~~~~~~}
    p{.5\textwidth}
    @{}
}
\nopagebreak
\textamh{ቢስሚላሂ አራህመኒ ራሂይም } &  بِسْمِ ٱللَّهِ ٱلرَّحْمَـٰنِ ٱلرَّحِيمِ\\
\textamh{1.\  } &  يَـٰٓأَيُّهَا ٱلَّذِينَ ءَامَنُوا۟ لَا تُقَدِّمُوا۟ بَيْنَ يَدَىِ ٱللَّهِ وَرَسُولِهِۦ ۖ وَٱتَّقُوا۟ ٱللَّهَ ۚ إِنَّ ٱللَّهَ سَمِيعٌ عَلِيمٌۭ ﴿١﴾\\
\textamh{2.\  } & يَـٰٓأَيُّهَا ٱلَّذِينَ ءَامَنُوا۟ لَا تَرْفَعُوٓا۟ أَصْوَٟتَكُمْ فَوْقَ صَوْتِ ٱلنَّبِىِّ وَلَا تَجْهَرُوا۟ لَهُۥ بِٱلْقَوْلِ كَجَهْرِ بَعْضِكُمْ لِبَعْضٍ أَن تَحْبَطَ أَعْمَـٰلُكُمْ وَأَنتُمْ لَا تَشْعُرُونَ ﴿٢﴾\\
\textamh{3.\  } & إِنَّ ٱلَّذِينَ يَغُضُّونَ أَصْوَٟتَهُمْ عِندَ رَسُولِ ٱللَّهِ أُو۟لَـٰٓئِكَ ٱلَّذِينَ ٱمْتَحَنَ ٱللَّهُ قُلُوبَهُمْ لِلتَّقْوَىٰ ۚ لَهُم مَّغْفِرَةٌۭ وَأَجْرٌ عَظِيمٌ ﴿٣﴾\\
\textamh{4.\  } & إِنَّ ٱلَّذِينَ يُنَادُونَكَ مِن وَرَآءِ ٱلْحُجُرَٰتِ أَكْثَرُهُمْ لَا يَعْقِلُونَ ﴿٤﴾\\
\textamh{5.\  } & وَلَوْ أَنَّهُمْ صَبَرُوا۟ حَتَّىٰ تَخْرُجَ إِلَيْهِمْ لَكَانَ خَيْرًۭا لَّهُمْ ۚ وَٱللَّهُ غَفُورٌۭ رَّحِيمٌۭ ﴿٥﴾\\
\textamh{6.\  } & يَـٰٓأَيُّهَا ٱلَّذِينَ ءَامَنُوٓا۟ إِن جَآءَكُمْ فَاسِقٌۢ بِنَبَإٍۢ فَتَبَيَّنُوٓا۟ أَن تُصِيبُوا۟ قَوْمًۢا بِجَهَـٰلَةٍۢ فَتُصْبِحُوا۟ عَلَىٰ مَا فَعَلْتُمْ نَـٰدِمِينَ ﴿٦﴾\\
\textamh{7.\  } & وَٱعْلَمُوٓا۟ أَنَّ فِيكُمْ رَسُولَ ٱللَّهِ ۚ لَوْ يُطِيعُكُمْ فِى كَثِيرٍۢ مِّنَ ٱلْأَمْرِ لَعَنِتُّمْ وَلَـٰكِنَّ ٱللَّهَ حَبَّبَ إِلَيْكُمُ ٱلْإِيمَـٰنَ وَزَيَّنَهُۥ فِى قُلُوبِكُمْ وَكَرَّهَ إِلَيْكُمُ ٱلْكُفْرَ وَٱلْفُسُوقَ وَٱلْعِصْيَانَ ۚ أُو۟لَـٰٓئِكَ هُمُ ٱلرَّٟشِدُونَ ﴿٧﴾\\
\textamh{8.\  } & فَضْلًۭا مِّنَ ٱللَّهِ وَنِعْمَةًۭ ۚ وَٱللَّهُ عَلِيمٌ حَكِيمٌۭ ﴿٨﴾\\
\textamh{9.\  } & وَإِن طَآئِفَتَانِ مِنَ ٱلْمُؤْمِنِينَ ٱقْتَتَلُوا۟ فَأَصْلِحُوا۟ بَيْنَهُمَا ۖ فَإِنۢ بَغَتْ إِحْدَىٰهُمَا عَلَى ٱلْأُخْرَىٰ فَقَـٰتِلُوا۟ ٱلَّتِى تَبْغِى حَتَّىٰ تَفِىٓءَ إِلَىٰٓ أَمْرِ ٱللَّهِ ۚ فَإِن فَآءَتْ فَأَصْلِحُوا۟ بَيْنَهُمَا بِٱلْعَدْلِ وَأَقْسِطُوٓا۟ ۖ إِنَّ ٱللَّهَ يُحِبُّ ٱلْمُقْسِطِينَ ﴿٩﴾\\
\textamh{10.\  } & إِنَّمَا ٱلْمُؤْمِنُونَ إِخْوَةٌۭ فَأَصْلِحُوا۟ بَيْنَ أَخَوَيْكُمْ ۚ وَٱتَّقُوا۟ ٱللَّهَ لَعَلَّكُمْ تُرْحَمُونَ ﴿١٠﴾\\
\textamh{11.\  } & يَـٰٓأَيُّهَا ٱلَّذِينَ ءَامَنُوا۟ لَا يَسْخَرْ قَوْمٌۭ مِّن قَوْمٍ عَسَىٰٓ أَن يَكُونُوا۟ خَيْرًۭا مِّنْهُمْ وَلَا نِسَآءٌۭ مِّن نِّسَآءٍ عَسَىٰٓ أَن يَكُنَّ خَيْرًۭا مِّنْهُنَّ ۖ وَلَا تَلْمِزُوٓا۟ أَنفُسَكُمْ وَلَا تَنَابَزُوا۟ بِٱلْأَلْقَـٰبِ ۖ بِئْسَ ٱلِٱسْمُ ٱلْفُسُوقُ بَعْدَ ٱلْإِيمَـٰنِ ۚ وَمَن لَّمْ يَتُبْ فَأُو۟لَـٰٓئِكَ هُمُ ٱلظَّـٰلِمُونَ ﴿١١﴾\\
\textamh{12.\  } & يَـٰٓأَيُّهَا ٱلَّذِينَ ءَامَنُوا۟ ٱجْتَنِبُوا۟ كَثِيرًۭا مِّنَ ٱلظَّنِّ إِنَّ بَعْضَ ٱلظَّنِّ إِثْمٌۭ ۖ وَلَا تَجَسَّسُوا۟ وَلَا يَغْتَب بَّعْضُكُم بَعْضًا ۚ أَيُحِبُّ أَحَدُكُمْ أَن يَأْكُلَ لَحْمَ أَخِيهِ مَيْتًۭا فَكَرِهْتُمُوهُ ۚ وَٱتَّقُوا۟ ٱللَّهَ ۚ إِنَّ ٱللَّهَ تَوَّابٌۭ رَّحِيمٌۭ ﴿١٢﴾\\
\textamh{13.\  } & يَـٰٓأَيُّهَا ٱلنَّاسُ إِنَّا خَلَقْنَـٰكُم مِّن ذَكَرٍۢ وَأُنثَىٰ وَجَعَلْنَـٰكُمْ شُعُوبًۭا وَقَبَآئِلَ لِتَعَارَفُوٓا۟ ۚ إِنَّ أَكْرَمَكُمْ عِندَ ٱللَّهِ أَتْقَىٰكُمْ ۚ إِنَّ ٱللَّهَ عَلِيمٌ خَبِيرٌۭ ﴿١٣﴾\\
\textamh{14.\  } & ۞ قَالَتِ ٱلْأَعْرَابُ ءَامَنَّا ۖ قُل لَّمْ تُؤْمِنُوا۟ وَلَـٰكِن قُولُوٓا۟ أَسْلَمْنَا وَلَمَّا يَدْخُلِ ٱلْإِيمَـٰنُ فِى قُلُوبِكُمْ ۖ وَإِن تُطِيعُوا۟ ٱللَّهَ وَرَسُولَهُۥ لَا يَلِتْكُم مِّنْ أَعْمَـٰلِكُمْ شَيْـًٔا ۚ إِنَّ ٱللَّهَ غَفُورٌۭ رَّحِيمٌ ﴿١٤﴾\\
\textamh{15.\  } & إِنَّمَا ٱلْمُؤْمِنُونَ ٱلَّذِينَ ءَامَنُوا۟ بِٱللَّهِ وَرَسُولِهِۦ ثُمَّ لَمْ يَرْتَابُوا۟ وَجَٰهَدُوا۟ بِأَمْوَٟلِهِمْ وَأَنفُسِهِمْ فِى سَبِيلِ ٱللَّهِ ۚ أُو۟لَـٰٓئِكَ هُمُ ٱلصَّـٰدِقُونَ ﴿١٥﴾\\
\textamh{16.\  } & قُلْ أَتُعَلِّمُونَ ٱللَّهَ بِدِينِكُمْ وَٱللَّهُ يَعْلَمُ مَا فِى ٱلسَّمَـٰوَٟتِ وَمَا فِى ٱلْأَرْضِ ۚ وَٱللَّهُ بِكُلِّ شَىْءٍ عَلِيمٌۭ ﴿١٦﴾\\
\textamh{17.\  } & يَمُنُّونَ عَلَيْكَ أَنْ أَسْلَمُوا۟ ۖ قُل لَّا تَمُنُّوا۟ عَلَىَّ إِسْلَـٰمَكُم ۖ بَلِ ٱللَّهُ يَمُنُّ عَلَيْكُمْ أَنْ هَدَىٰكُمْ لِلْإِيمَـٰنِ إِن كُنتُمْ صَـٰدِقِينَ ﴿١٧﴾\\
\textamh{18.\  } & إِنَّ ٱللَّهَ يَعْلَمُ غَيْبَ ٱلسَّمَـٰوَٟتِ وَٱلْأَرْضِ ۚ وَٱللَّهُ بَصِيرٌۢ بِمَا تَعْمَلُونَ ﴿١٨﴾\\
\end{longtable}
\clearpage
%% License: BSD style (Berkley) (i.e. Put the Copyright owner's name always)
%% Writer and Copyright (to): Bewketu(Bilal) Tadilo (2016-17)
\centering\section{\LR{\textamharic{ሱራቱ ቃፍ -}  \RL{سوره  ق}}}
\begin{longtable}{%
  @{}
    p{.5\textwidth}
  @{~~~~~~~~~~~~}
    p{.5\textwidth}
    @{}
}
\nopagebreak
\textamh{ቢስሚላሂ አራህመኒ ራሂይም } &  بِسْمِ ٱللَّهِ ٱلرَّحْمَـٰنِ ٱلرَّحِيمِ\\
\textamh{1.\  } &  قٓ ۚ وَٱلْقُرْءَانِ ٱلْمَجِيدِ ﴿١﴾\\
\textamh{2.\  } & بَلْ عَجِبُوٓا۟ أَن جَآءَهُم مُّنذِرٌۭ مِّنْهُمْ فَقَالَ ٱلْكَـٰفِرُونَ هَـٰذَا شَىْءٌ عَجِيبٌ ﴿٢﴾\\
\textamh{3.\  } & أَءِذَا مِتْنَا وَكُنَّا تُرَابًۭا ۖ ذَٟلِكَ رَجْعٌۢ بَعِيدٌۭ ﴿٣﴾\\
\textamh{4.\  } & قَدْ عَلِمْنَا مَا تَنقُصُ ٱلْأَرْضُ مِنْهُمْ ۖ وَعِندَنَا كِتَـٰبٌ حَفِيظٌۢ ﴿٤﴾\\
\textamh{5.\  } & بَلْ كَذَّبُوا۟ بِٱلْحَقِّ لَمَّا جَآءَهُمْ فَهُمْ فِىٓ أَمْرٍۢ مَّرِيجٍ ﴿٥﴾\\
\textamh{6.\  } & أَفَلَمْ يَنظُرُوٓا۟ إِلَى ٱلسَّمَآءِ فَوْقَهُمْ كَيْفَ بَنَيْنَـٰهَا وَزَيَّنَّـٰهَا وَمَا لَهَا مِن فُرُوجٍۢ ﴿٦﴾\\
\textamh{7.\  } & وَٱلْأَرْضَ مَدَدْنَـٰهَا وَأَلْقَيْنَا فِيهَا رَوَٟسِىَ وَأَنۢبَتْنَا فِيهَا مِن كُلِّ زَوْجٍۭ بَهِيجٍۢ ﴿٧﴾\\
\textamh{8.\  } & تَبْصِرَةًۭ وَذِكْرَىٰ لِكُلِّ عَبْدٍۢ مُّنِيبٍۢ ﴿٨﴾\\
\textamh{9.\  } & وَنَزَّلْنَا مِنَ ٱلسَّمَآءِ مَآءًۭ مُّبَٰرَكًۭا فَأَنۢبَتْنَا بِهِۦ جَنَّـٰتٍۢ وَحَبَّ ٱلْحَصِيدِ ﴿٩﴾\\
\textamh{10.\  } & وَٱلنَّخْلَ بَاسِقَـٰتٍۢ لَّهَا طَلْعٌۭ نَّضِيدٌۭ ﴿١٠﴾\\
\textamh{11.\  } & رِّزْقًۭا لِّلْعِبَادِ ۖ وَأَحْيَيْنَا بِهِۦ بَلْدَةًۭ مَّيْتًۭا ۚ كَذَٟلِكَ ٱلْخُرُوجُ ﴿١١﴾\\
\textamh{12.\  } & كَذَّبَتْ قَبْلَهُمْ قَوْمُ نُوحٍۢ وَأَصْحَـٰبُ ٱلرَّسِّ وَثَمُودُ ﴿١٢﴾\\
\textamh{13.\  } & وَعَادٌۭ وَفِرْعَوْنُ وَإِخْوَٟنُ لُوطٍۢ ﴿١٣﴾\\
\textamh{14.\  } & وَأَصْحَـٰبُ ٱلْأَيْكَةِ وَقَوْمُ تُبَّعٍۢ ۚ كُلٌّۭ كَذَّبَ ٱلرُّسُلَ فَحَقَّ وَعِيدِ ﴿١٤﴾\\
\textamh{15.\  } & أَفَعَيِينَا بِٱلْخَلْقِ ٱلْأَوَّلِ ۚ بَلْ هُمْ فِى لَبْسٍۢ مِّنْ خَلْقٍۢ جَدِيدٍۢ ﴿١٥﴾\\
\textamh{16.\  } & وَلَقَدْ خَلَقْنَا ٱلْإِنسَـٰنَ وَنَعْلَمُ مَا تُوَسْوِسُ بِهِۦ نَفْسُهُۥ ۖ وَنَحْنُ أَقْرَبُ إِلَيْهِ مِنْ حَبْلِ ٱلْوَرِيدِ ﴿١٦﴾\\
\textamh{17.\  } & إِذْ يَتَلَقَّى ٱلْمُتَلَقِّيَانِ عَنِ ٱلْيَمِينِ وَعَنِ ٱلشِّمَالِ قَعِيدٌۭ ﴿١٧﴾\\
\textamh{18.\  } & مَّا يَلْفِظُ مِن قَوْلٍ إِلَّا لَدَيْهِ رَقِيبٌ عَتِيدٌۭ ﴿١٨﴾\\
\textamh{19.\  } & وَجَآءَتْ سَكْرَةُ ٱلْمَوْتِ بِٱلْحَقِّ ۖ ذَٟلِكَ مَا كُنتَ مِنْهُ تَحِيدُ ﴿١٩﴾\\
\textamh{20.\  } & وَنُفِخَ فِى ٱلصُّورِ ۚ ذَٟلِكَ يَوْمُ ٱلْوَعِيدِ ﴿٢٠﴾\\
\textamh{21.\  } & وَجَآءَتْ كُلُّ نَفْسٍۢ مَّعَهَا سَآئِقٌۭ وَشَهِيدٌۭ ﴿٢١﴾\\
\textamh{22.\  } & لَّقَدْ كُنتَ فِى غَفْلَةٍۢ مِّنْ هَـٰذَا فَكَشَفْنَا عَنكَ غِطَآءَكَ فَبَصَرُكَ ٱلْيَوْمَ حَدِيدٌۭ ﴿٢٢﴾\\
\textamh{23.\  } & وَقَالَ قَرِينُهُۥ هَـٰذَا مَا لَدَىَّ عَتِيدٌ ﴿٢٣﴾\\
\textamh{24.\  } & أَلْقِيَا فِى جَهَنَّمَ كُلَّ كَفَّارٍ عَنِيدٍۢ ﴿٢٤﴾\\
\textamh{25.\  } & مَّنَّاعٍۢ لِّلْخَيْرِ مُعْتَدٍۢ مُّرِيبٍ ﴿٢٥﴾\\
\textamh{26.\  } & ٱلَّذِى جَعَلَ مَعَ ٱللَّهِ إِلَـٰهًا ءَاخَرَ فَأَلْقِيَاهُ فِى ٱلْعَذَابِ ٱلشَّدِيدِ ﴿٢٦﴾\\
\textamh{27.\  } & ۞ قَالَ قَرِينُهُۥ رَبَّنَا مَآ أَطْغَيْتُهُۥ وَلَـٰكِن كَانَ فِى ضَلَـٰلٍۭ بَعِيدٍۢ ﴿٢٧﴾\\
\textamh{28.\  } & قَالَ لَا تَخْتَصِمُوا۟ لَدَىَّ وَقَدْ قَدَّمْتُ إِلَيْكُم بِٱلْوَعِيدِ ﴿٢٨﴾\\
\textamh{29.\  } & مَا يُبَدَّلُ ٱلْقَوْلُ لَدَىَّ وَمَآ أَنَا۠ بِظَلَّٰمٍۢ لِّلْعَبِيدِ ﴿٢٩﴾\\
\textamh{30.\  } & يَوْمَ نَقُولُ لِجَهَنَّمَ هَلِ ٱمْتَلَأْتِ وَتَقُولُ هَلْ مِن مَّزِيدٍۢ ﴿٣٠﴾\\
\textamh{31.\  } & وَأُزْلِفَتِ ٱلْجَنَّةُ لِلْمُتَّقِينَ غَيْرَ بَعِيدٍ ﴿٣١﴾\\
\textamh{32.\  } & هَـٰذَا مَا تُوعَدُونَ لِكُلِّ أَوَّابٍ حَفِيظٍۢ ﴿٣٢﴾\\
\textamh{33.\  } & مَّنْ خَشِىَ ٱلرَّحْمَـٰنَ بِٱلْغَيْبِ وَجَآءَ بِقَلْبٍۢ مُّنِيبٍ ﴿٣٣﴾\\
\textamh{34.\  } & ٱدْخُلُوهَا بِسَلَـٰمٍۢ ۖ ذَٟلِكَ يَوْمُ ٱلْخُلُودِ ﴿٣٤﴾\\
\textamh{35.\  } & لَهُم مَّا يَشَآءُونَ فِيهَا وَلَدَيْنَا مَزِيدٌۭ ﴿٣٥﴾\\
\textamh{36.\  } & وَكَمْ أَهْلَكْنَا قَبْلَهُم مِّن قَرْنٍ هُمْ أَشَدُّ مِنْهُم بَطْشًۭا فَنَقَّبُوا۟ فِى ٱلْبِلَـٰدِ هَلْ مِن مَّحِيصٍ ﴿٣٦﴾\\
\textamh{37.\  } & إِنَّ فِى ذَٟلِكَ لَذِكْرَىٰ لِمَن كَانَ لَهُۥ قَلْبٌ أَوْ أَلْقَى ٱلسَّمْعَ وَهُوَ شَهِيدٌۭ ﴿٣٧﴾\\
\textamh{38.\  } & وَلَقَدْ خَلَقْنَا ٱلسَّمَـٰوَٟتِ وَٱلْأَرْضَ وَمَا بَيْنَهُمَا فِى سِتَّةِ أَيَّامٍۢ وَمَا مَسَّنَا مِن لُّغُوبٍۢ ﴿٣٨﴾\\
\textamh{39.\  } & فَٱصْبِرْ عَلَىٰ مَا يَقُولُونَ وَسَبِّحْ بِحَمْدِ رَبِّكَ قَبْلَ طُلُوعِ ٱلشَّمْسِ وَقَبْلَ ٱلْغُرُوبِ ﴿٣٩﴾\\
\textamh{40.\  } & وَمِنَ ٱلَّيْلِ فَسَبِّحْهُ وَأَدْبَٰرَ ٱلسُّجُودِ ﴿٤٠﴾\\
\textamh{41.\  } & وَٱسْتَمِعْ يَوْمَ يُنَادِ ٱلْمُنَادِ مِن مَّكَانٍۢ قَرِيبٍۢ ﴿٤١﴾\\
\textamh{42.\  } & يَوْمَ يَسْمَعُونَ ٱلصَّيْحَةَ بِٱلْحَقِّ ۚ ذَٟلِكَ يَوْمُ ٱلْخُرُوجِ ﴿٤٢﴾\\
\textamh{43.\  } & إِنَّا نَحْنُ نُحْىِۦ وَنُمِيتُ وَإِلَيْنَا ٱلْمَصِيرُ ﴿٤٣﴾\\
\textamh{44.\  } & يَوْمَ تَشَقَّقُ ٱلْأَرْضُ عَنْهُمْ سِرَاعًۭا ۚ ذَٟلِكَ حَشْرٌ عَلَيْنَا يَسِيرٌۭ ﴿٤٤﴾\\
\textamh{45.\  } & نَّحْنُ أَعْلَمُ بِمَا يَقُولُونَ ۖ وَمَآ أَنتَ عَلَيْهِم بِجَبَّارٍۢ ۖ فَذَكِّرْ بِٱلْقُرْءَانِ مَن يَخَافُ وَعِيدِ ﴿٤٥﴾\\
\end{longtable}
\clearpage
%% License: BSD style (Berkley) (i.e. Put the Copyright owner's name always)
%% Writer and Copyright (to): Bewketu(Bilal) Tadilo (2016-17)
\begin{center}\section{\LR{\textamhsec{ሱራቱ አልዛረያት -}  \textarabic{سوره  الذاريات}}}\end{center}
\begin{longtable}{%
  @{}
    p{.5\textwidth}
  @{~~~}
    p{.5\textwidth}
    @{}
}
\textamh{ቢስሚላሂ አራህመኒ ራሂይም } &  \mytextarabic{بِسْمِ ٱللَّهِ ٱلرَّحْمَـٰنِ ٱلرَّحِيمِ}\\
\textamh{1.\  } & \mytextarabic{ وَٱلذَّٰرِيَـٰتِ ذَرْوًۭا ﴿١﴾}\\
\textamh{2.\  } & \mytextarabic{فَٱلْحَـٰمِلَـٰتِ وِقْرًۭا ﴿٢﴾}\\
\textamh{3.\  } & \mytextarabic{فَٱلْجَٰرِيَـٰتِ يُسْرًۭا ﴿٣﴾}\\
\textamh{4.\  } & \mytextarabic{فَٱلْمُقَسِّمَـٰتِ أَمْرًا ﴿٤﴾}\\
\textamh{5.\  } & \mytextarabic{إِنَّمَا تُوعَدُونَ لَصَادِقٌۭ ﴿٥﴾}\\
\textamh{6.\  } & \mytextarabic{وَإِنَّ ٱلدِّينَ لَوَٟقِعٌۭ ﴿٦﴾}\\
\textamh{7.\  } & \mytextarabic{وَٱلسَّمَآءِ ذَاتِ ٱلْحُبُكِ ﴿٧﴾}\\
\textamh{8.\  } & \mytextarabic{إِنَّكُمْ لَفِى قَوْلٍۢ مُّخْتَلِفٍۢ ﴿٨﴾}\\
\textamh{9.\  } & \mytextarabic{يُؤْفَكُ عَنْهُ مَنْ أُفِكَ ﴿٩﴾}\\
\textamh{10.\  } & \mytextarabic{قُتِلَ ٱلْخَرَّٟصُونَ ﴿١٠﴾}\\
\textamh{11.\  } & \mytextarabic{ٱلَّذِينَ هُمْ فِى غَمْرَةٍۢ سَاهُونَ ﴿١١﴾}\\
\textamh{12.\  } & \mytextarabic{يَسْـَٔلُونَ أَيَّانَ يَوْمُ ٱلدِّينِ ﴿١٢﴾}\\
\textamh{13.\  } & \mytextarabic{يَوْمَ هُمْ عَلَى ٱلنَّارِ يُفْتَنُونَ ﴿١٣﴾}\\
\textamh{14.\  } & \mytextarabic{ذُوقُوا۟ فِتْنَتَكُمْ هَـٰذَا ٱلَّذِى كُنتُم بِهِۦ تَسْتَعْجِلُونَ ﴿١٤﴾}\\
\textamh{15.\  } & \mytextarabic{إِنَّ ٱلْمُتَّقِينَ فِى جَنَّـٰتٍۢ وَعُيُونٍ ﴿١٥﴾}\\
\textamh{16.\  } & \mytextarabic{ءَاخِذِينَ مَآ ءَاتَىٰهُمْ رَبُّهُمْ ۚ إِنَّهُمْ كَانُوا۟ قَبْلَ ذَٟلِكَ مُحْسِنِينَ ﴿١٦﴾}\\
\textamh{17.\  } & \mytextarabic{كَانُوا۟ قَلِيلًۭا مِّنَ ٱلَّيْلِ مَا يَهْجَعُونَ ﴿١٧﴾}\\
\textamh{18.\  } & \mytextarabic{وَبِٱلْأَسْحَارِ هُمْ يَسْتَغْفِرُونَ ﴿١٨﴾}\\
\textamh{19.\  } & \mytextarabic{وَفِىٓ أَمْوَٟلِهِمْ حَقٌّۭ لِّلسَّآئِلِ وَٱلْمَحْرُومِ ﴿١٩﴾}\\
\textamh{20.\  } & \mytextarabic{وَفِى ٱلْأَرْضِ ءَايَـٰتٌۭ لِّلْمُوقِنِينَ ﴿٢٠﴾}\\
\textamh{21.\  } & \mytextarabic{وَفِىٓ أَنفُسِكُمْ ۚ أَفَلَا تُبْصِرُونَ ﴿٢١﴾}\\
\textamh{22.\  } & \mytextarabic{وَفِى ٱلسَّمَآءِ رِزْقُكُمْ وَمَا تُوعَدُونَ ﴿٢٢﴾}\\
\textamh{23.\  } & \mytextarabic{فَوَرَبِّ ٱلسَّمَآءِ وَٱلْأَرْضِ إِنَّهُۥ لَحَقٌّۭ مِّثْلَ مَآ أَنَّكُمْ تَنطِقُونَ ﴿٢٣﴾}\\
\textamh{24.\  } & \mytextarabic{هَلْ أَتَىٰكَ حَدِيثُ ضَيْفِ إِبْرَٰهِيمَ ٱلْمُكْرَمِينَ ﴿٢٤﴾}\\
\textamh{25.\  } & \mytextarabic{إِذْ دَخَلُوا۟ عَلَيْهِ فَقَالُوا۟ سَلَـٰمًۭا ۖ قَالَ سَلَـٰمٌۭ قَوْمٌۭ مُّنكَرُونَ ﴿٢٥﴾}\\
\textamh{26.\  } & \mytextarabic{فَرَاغَ إِلَىٰٓ أَهْلِهِۦ فَجَآءَ بِعِجْلٍۢ سَمِينٍۢ ﴿٢٦﴾}\\
\textamh{27.\  } & \mytextarabic{فَقَرَّبَهُۥٓ إِلَيْهِمْ قَالَ أَلَا تَأْكُلُونَ ﴿٢٧﴾}\\
\textamh{28.\  } & \mytextarabic{فَأَوْجَسَ مِنْهُمْ خِيفَةًۭ ۖ قَالُوا۟ لَا تَخَفْ ۖ وَبَشَّرُوهُ بِغُلَـٰمٍ عَلِيمٍۢ ﴿٢٨﴾}\\
\textamh{29.\  } & \mytextarabic{فَأَقْبَلَتِ ٱمْرَأَتُهُۥ فِى صَرَّةٍۢ فَصَكَّتْ وَجْهَهَا وَقَالَتْ عَجُوزٌ عَقِيمٌۭ ﴿٢٩﴾}\\
\textamh{30.\  } & \mytextarabic{قَالُوا۟ كَذَٟلِكِ قَالَ رَبُّكِ ۖ إِنَّهُۥ هُوَ ٱلْحَكِيمُ ٱلْعَلِيمُ ﴿٣٠﴾}\\
\textamh{31.\  } & \mytextarabic{۞ قَالَ فَمَا خَطْبُكُمْ أَيُّهَا ٱلْمُرْسَلُونَ ﴿٣١﴾}\\
\textamh{32.\  } & \mytextarabic{قَالُوٓا۟ إِنَّآ أُرْسِلْنَآ إِلَىٰ قَوْمٍۢ مُّجْرِمِينَ ﴿٣٢﴾}\\
\textamh{33.\  } & \mytextarabic{لِنُرْسِلَ عَلَيْهِمْ حِجَارَةًۭ مِّن طِينٍۢ ﴿٣٣﴾}\\
\textamh{34.\  } & \mytextarabic{مُّسَوَّمَةً عِندَ رَبِّكَ لِلْمُسْرِفِينَ ﴿٣٤﴾}\\
\textamh{35.\  } & \mytextarabic{فَأَخْرَجْنَا مَن كَانَ فِيهَا مِنَ ٱلْمُؤْمِنِينَ ﴿٣٥﴾}\\
\textamh{36.\  } & \mytextarabic{فَمَا وَجَدْنَا فِيهَا غَيْرَ بَيْتٍۢ مِّنَ ٱلْمُسْلِمِينَ ﴿٣٦﴾}\\
\textamh{37.\  } & \mytextarabic{وَتَرَكْنَا فِيهَآ ءَايَةًۭ لِّلَّذِينَ يَخَافُونَ ٱلْعَذَابَ ٱلْأَلِيمَ ﴿٣٧﴾}\\
\textamh{38.\  } & \mytextarabic{وَفِى مُوسَىٰٓ إِذْ أَرْسَلْنَـٰهُ إِلَىٰ فِرْعَوْنَ بِسُلْطَٰنٍۢ مُّبِينٍۢ ﴿٣٨﴾}\\
\textamh{39.\  } & \mytextarabic{فَتَوَلَّىٰ بِرُكْنِهِۦ وَقَالَ سَـٰحِرٌ أَوْ مَجْنُونٌۭ ﴿٣٩﴾}\\
\textamh{40.\  } & \mytextarabic{فَأَخَذْنَـٰهُ وَجُنُودَهُۥ فَنَبَذْنَـٰهُمْ فِى ٱلْيَمِّ وَهُوَ مُلِيمٌۭ ﴿٤٠﴾}\\
\textamh{41.\  } & \mytextarabic{وَفِى عَادٍ إِذْ أَرْسَلْنَا عَلَيْهِمُ ٱلرِّيحَ ٱلْعَقِيمَ ﴿٤١﴾}\\
\textamh{42.\  } & \mytextarabic{مَا تَذَرُ مِن شَىْءٍ أَتَتْ عَلَيْهِ إِلَّا جَعَلَتْهُ كَٱلرَّمِيمِ ﴿٤٢﴾}\\
\textamh{43.\  } & \mytextarabic{وَفِى ثَمُودَ إِذْ قِيلَ لَهُمْ تَمَتَّعُوا۟ حَتَّىٰ حِينٍۢ ﴿٤٣﴾}\\
\textamh{44.\  } & \mytextarabic{فَعَتَوْا۟ عَنْ أَمْرِ رَبِّهِمْ فَأَخَذَتْهُمُ ٱلصَّـٰعِقَةُ وَهُمْ يَنظُرُونَ ﴿٤٤﴾}\\
\textamh{45.\  } & \mytextarabic{فَمَا ٱسْتَطَٰعُوا۟ مِن قِيَامٍۢ وَمَا كَانُوا۟ مُنتَصِرِينَ ﴿٤٥﴾}\\
\textamh{46.\  } & \mytextarabic{وَقَوْمَ نُوحٍۢ مِّن قَبْلُ ۖ إِنَّهُمْ كَانُوا۟ قَوْمًۭا فَـٰسِقِينَ ﴿٤٦﴾}\\
\textamh{47.\  } & \mytextarabic{وَٱلسَّمَآءَ بَنَيْنَـٰهَا بِأَيْي۟دٍۢ وَإِنَّا لَمُوسِعُونَ ﴿٤٧﴾}\\
\textamh{48.\  } & \mytextarabic{وَٱلْأَرْضَ فَرَشْنَـٰهَا فَنِعْمَ ٱلْمَـٰهِدُونَ ﴿٤٨﴾}\\
\textamh{49.\  } & \mytextarabic{وَمِن كُلِّ شَىْءٍ خَلَقْنَا زَوْجَيْنِ لَعَلَّكُمْ تَذَكَّرُونَ ﴿٤٩﴾}\\
\textamh{50.\  } & \mytextarabic{فَفِرُّوٓا۟ إِلَى ٱللَّهِ ۖ إِنِّى لَكُم مِّنْهُ نَذِيرٌۭ مُّبِينٌۭ ﴿٥٠﴾}\\
\textamh{51.\  } & \mytextarabic{وَلَا تَجْعَلُوا۟ مَعَ ٱللَّهِ إِلَـٰهًا ءَاخَرَ ۖ إِنِّى لَكُم مِّنْهُ نَذِيرٌۭ مُّبِينٌۭ ﴿٥١﴾}\\
\textamh{52.\  } & \mytextarabic{كَذَٟلِكَ مَآ أَتَى ٱلَّذِينَ مِن قَبْلِهِم مِّن رَّسُولٍ إِلَّا قَالُوا۟ سَاحِرٌ أَوْ مَجْنُونٌ ﴿٥٢﴾}\\
\textamh{53.\  } & \mytextarabic{أَتَوَاصَوْا۟ بِهِۦ ۚ بَلْ هُمْ قَوْمٌۭ طَاغُونَ ﴿٥٣﴾}\\
\textamh{54.\  } & \mytextarabic{فَتَوَلَّ عَنْهُمْ فَمَآ أَنتَ بِمَلُومٍۢ ﴿٥٤﴾}\\
\textamh{55.\  } & \mytextarabic{وَذَكِّرْ فَإِنَّ ٱلذِّكْرَىٰ تَنفَعُ ٱلْمُؤْمِنِينَ ﴿٥٥﴾}\\
\textamh{56.\  } & \mytextarabic{وَمَا خَلَقْتُ ٱلْجِنَّ وَٱلْإِنسَ إِلَّا لِيَعْبُدُونِ ﴿٥٦﴾}\\
\textamh{57.\  } & \mytextarabic{مَآ أُرِيدُ مِنْهُم مِّن رِّزْقٍۢ وَمَآ أُرِيدُ أَن يُطْعِمُونِ ﴿٥٧﴾}\\
\textamh{58.\  } & \mytextarabic{إِنَّ ٱللَّهَ هُوَ ٱلرَّزَّاقُ ذُو ٱلْقُوَّةِ ٱلْمَتِينُ ﴿٥٨﴾}\\
\textamh{59.\  } & \mytextarabic{فَإِنَّ لِلَّذِينَ ظَلَمُوا۟ ذَنُوبًۭا مِّثْلَ ذَنُوبِ أَصْحَـٰبِهِمْ فَلَا يَسْتَعْجِلُونِ ﴿٥٩﴾}\\
\textamh{60.\  } & \mytextarabic{فَوَيْلٌۭ لِّلَّذِينَ كَفَرُوا۟ مِن يَوْمِهِمُ ٱلَّذِى يُوعَدُونَ ﴿٦٠﴾}\\
\end{longtable}
\clearpage
%% License: BSD style (Berkley) (i.e. Put the Copyright owner's name always)
%% Writer and Copyright (to): Bewketu(Bilal) Tadilo (2016-17)
\centering\section{\LR{\textamharic{ሱራቱ አጥጡር -}  \RL{سوره  الطور}}}
\begin{longtable}{%
  @{}
    p{.5\textwidth}
  @{~~~~~~~~~~~~~}
    p{.5\textwidth}
    @{}
}
\nopagebreak
\textamh{ቢስሚላሂ አራህመኒ ራሂይም } &  بِسْمِ ٱللَّهِ ٱلرَّحْمَـٰنِ ٱلرَّحِيمِ\\
\textamh{1.\  } &  وَٱلطُّورِ ﴿١﴾\\
\textamh{2.\  } & وَكِتَـٰبٍۢ مَّسْطُورٍۢ ﴿٢﴾\\
\textamh{3.\  } & فِى رَقٍّۢ مَّنشُورٍۢ ﴿٣﴾\\
\textamh{4.\  } & وَٱلْبَيْتِ ٱلْمَعْمُورِ ﴿٤﴾\\
\textamh{5.\  } & وَٱلسَّقْفِ ٱلْمَرْفُوعِ ﴿٥﴾\\
\textamh{6.\  } & وَٱلْبَحْرِ ٱلْمَسْجُورِ ﴿٦﴾\\
\textamh{7.\  } & إِنَّ عَذَابَ رَبِّكَ لَوَٟقِعٌۭ ﴿٧﴾\\
\textamh{8.\  } & مَّا لَهُۥ مِن دَافِعٍۢ ﴿٨﴾\\
\textamh{9.\  } & يَوْمَ تَمُورُ ٱلسَّمَآءُ مَوْرًۭا ﴿٩﴾\\
\textamh{10.\  } & وَتَسِيرُ ٱلْجِبَالُ سَيْرًۭا ﴿١٠﴾\\
\textamh{11.\  } & فَوَيْلٌۭ يَوْمَئِذٍۢ لِّلْمُكَذِّبِينَ ﴿١١﴾\\
\textamh{12.\  } & ٱلَّذِينَ هُمْ فِى خَوْضٍۢ يَلْعَبُونَ ﴿١٢﴾\\
\textamh{13.\  } & يَوْمَ يُدَعُّونَ إِلَىٰ نَارِ جَهَنَّمَ دَعًّا ﴿١٣﴾\\
\textamh{14.\  } & هَـٰذِهِ ٱلنَّارُ ٱلَّتِى كُنتُم بِهَا تُكَذِّبُونَ ﴿١٤﴾\\
\textamh{15.\  } & أَفَسِحْرٌ هَـٰذَآ أَمْ أَنتُمْ لَا تُبْصِرُونَ ﴿١٥﴾\\
\textamh{16.\  } & ٱصْلَوْهَا فَٱصْبِرُوٓا۟ أَوْ لَا تَصْبِرُوا۟ سَوَآءٌ عَلَيْكُمْ ۖ إِنَّمَا تُجْزَوْنَ مَا كُنتُمْ تَعْمَلُونَ ﴿١٦﴾\\
\textamh{17.\  } & إِنَّ ٱلْمُتَّقِينَ فِى جَنَّـٰتٍۢ وَنَعِيمٍۢ ﴿١٧﴾\\
\textamh{18.\  } & فَـٰكِهِينَ بِمَآ ءَاتَىٰهُمْ رَبُّهُمْ وَوَقَىٰهُمْ رَبُّهُمْ عَذَابَ ٱلْجَحِيمِ ﴿١٨﴾\\
\textamh{19.\  } & كُلُوا۟ وَٱشْرَبُوا۟ هَنِيٓـًٔۢا بِمَا كُنتُمْ تَعْمَلُونَ ﴿١٩﴾\\
\textamh{20.\  } & مُتَّكِـِٔينَ عَلَىٰ سُرُرٍۢ مَّصْفُوفَةٍۢ ۖ وَزَوَّجْنَـٰهُم بِحُورٍ عِينٍۢ ﴿٢٠﴾\\
\textamh{21.\  } & وَٱلَّذِينَ ءَامَنُوا۟ وَٱتَّبَعَتْهُمْ ذُرِّيَّتُهُم بِإِيمَـٰنٍ أَلْحَقْنَا بِهِمْ ذُرِّيَّتَهُمْ وَمَآ أَلَتْنَـٰهُم مِّنْ عَمَلِهِم مِّن شَىْءٍۢ ۚ كُلُّ ٱمْرِئٍۭ بِمَا كَسَبَ رَهِينٌۭ ﴿٢١﴾\\
\textamh{22.\  } & وَأَمْدَدْنَـٰهُم بِفَـٰكِهَةٍۢ وَلَحْمٍۢ مِّمَّا يَشْتَهُونَ ﴿٢٢﴾\\
\textamh{23.\  } & يَتَنَـٰزَعُونَ فِيهَا كَأْسًۭا لَّا لَغْوٌۭ فِيهَا وَلَا تَأْثِيمٌۭ ﴿٢٣﴾\\
\textamh{24.\  } & ۞ وَيَطُوفُ عَلَيْهِمْ غِلْمَانٌۭ لَّهُمْ كَأَنَّهُمْ لُؤْلُؤٌۭ مَّكْنُونٌۭ ﴿٢٤﴾\\
\textamh{25.\  } & وَأَقْبَلَ بَعْضُهُمْ عَلَىٰ بَعْضٍۢ يَتَسَآءَلُونَ ﴿٢٥﴾\\
\textamh{26.\  } & قَالُوٓا۟ إِنَّا كُنَّا قَبْلُ فِىٓ أَهْلِنَا مُشْفِقِينَ ﴿٢٦﴾\\
\textamh{27.\  } & فَمَنَّ ٱللَّهُ عَلَيْنَا وَوَقَىٰنَا عَذَابَ ٱلسَّمُومِ ﴿٢٧﴾\\
\textamh{28.\  } & إِنَّا كُنَّا مِن قَبْلُ نَدْعُوهُ ۖ إِنَّهُۥ هُوَ ٱلْبَرُّ ٱلرَّحِيمُ ﴿٢٨﴾\\
\textamh{29.\  } & فَذَكِّرْ فَمَآ أَنتَ بِنِعْمَتِ رَبِّكَ بِكَاهِنٍۢ وَلَا مَجْنُونٍ ﴿٢٩﴾\\
\textamh{30.\  } & أَمْ يَقُولُونَ شَاعِرٌۭ نَّتَرَبَّصُ بِهِۦ رَيْبَ ٱلْمَنُونِ ﴿٣٠﴾\\
\textamh{31.\  } & قُلْ تَرَبَّصُوا۟ فَإِنِّى مَعَكُم مِّنَ ٱلْمُتَرَبِّصِينَ ﴿٣١﴾\\
\textamh{32.\  } & أَمْ تَأْمُرُهُمْ أَحْلَـٰمُهُم بِهَـٰذَآ ۚ أَمْ هُمْ قَوْمٌۭ طَاغُونَ ﴿٣٢﴾\\
\textamh{33.\  } & أَمْ يَقُولُونَ تَقَوَّلَهُۥ ۚ بَل لَّا يُؤْمِنُونَ ﴿٣٣﴾\\
\textamh{34.\  } & فَلْيَأْتُوا۟ بِحَدِيثٍۢ مِّثْلِهِۦٓ إِن كَانُوا۟ صَـٰدِقِينَ ﴿٣٤﴾\\
\textamh{35.\  } & أَمْ خُلِقُوا۟ مِنْ غَيْرِ شَىْءٍ أَمْ هُمُ ٱلْخَـٰلِقُونَ ﴿٣٥﴾\\
\textamh{36.\  } & أَمْ خَلَقُوا۟ ٱلسَّمَـٰوَٟتِ وَٱلْأَرْضَ ۚ بَل لَّا يُوقِنُونَ ﴿٣٦﴾\\
\textamh{37.\  } & أَمْ عِندَهُمْ خَزَآئِنُ رَبِّكَ أَمْ هُمُ ٱلْمُصَۣيْطِرُونَ ﴿٣٧﴾\\
\textamh{38.\  } & أَمْ لَهُمْ سُلَّمٌۭ يَسْتَمِعُونَ فِيهِ ۖ فَلْيَأْتِ مُسْتَمِعُهُم بِسُلْطَٰنٍۢ مُّبِينٍ ﴿٣٨﴾\\
\textamh{39.\  } & أَمْ لَهُ ٱلْبَنَـٰتُ وَلَكُمُ ٱلْبَنُونَ ﴿٣٩﴾\\
\textamh{40.\  } & أَمْ تَسْـَٔلُهُمْ أَجْرًۭا فَهُم مِّن مَّغْرَمٍۢ مُّثْقَلُونَ ﴿٤٠﴾\\
\textamh{41.\  } & أَمْ عِندَهُمُ ٱلْغَيْبُ فَهُمْ يَكْتُبُونَ ﴿٤١﴾\\
\textamh{42.\  } & أَمْ يُرِيدُونَ كَيْدًۭا ۖ فَٱلَّذِينَ كَفَرُوا۟ هُمُ ٱلْمَكِيدُونَ ﴿٤٢﴾\\
\textamh{43.\  } & أَمْ لَهُمْ إِلَـٰهٌ غَيْرُ ٱللَّهِ ۚ سُبْحَـٰنَ ٱللَّهِ عَمَّا يُشْرِكُونَ ﴿٤٣﴾\\
\textamh{44.\  } & وَإِن يَرَوْا۟ كِسْفًۭا مِّنَ ٱلسَّمَآءِ سَاقِطًۭا يَقُولُوا۟ سَحَابٌۭ مَّرْكُومٌۭ ﴿٤٤﴾\\
\textamh{45.\  } & فَذَرْهُمْ حَتَّىٰ يُلَـٰقُوا۟ يَوْمَهُمُ ٱلَّذِى فِيهِ يُصْعَقُونَ ﴿٤٥﴾\\
\textamh{46.\  } & يَوْمَ لَا يُغْنِى عَنْهُمْ كَيْدُهُمْ شَيْـًۭٔا وَلَا هُمْ يُنصَرُونَ ﴿٤٦﴾\\
\textamh{47.\  } & وَإِنَّ لِلَّذِينَ ظَلَمُوا۟ عَذَابًۭا دُونَ ذَٟلِكَ وَلَـٰكِنَّ أَكْثَرَهُمْ لَا يَعْلَمُونَ ﴿٤٧﴾\\
\textamh{48.\  } & وَٱصْبِرْ لِحُكْمِ رَبِّكَ فَإِنَّكَ بِأَعْيُنِنَا ۖ وَسَبِّحْ بِحَمْدِ رَبِّكَ حِينَ تَقُومُ ﴿٤٨﴾\\
\textamh{49.\  } & وَمِنَ ٱلَّيْلِ فَسَبِّحْهُ وَإِدْبَٰرَ ٱلنُّجُومِ ﴿٤٩﴾\\
\end{longtable}
\clearpage
%% License: BSD style (Berkley) (i.e. Put the Copyright owner's name always)
%% Writer and Copyright (to): Bewketu(Bilal) Tadilo (2016-17)
\centering\section{\LR{\textamharic{ሱራቱ አዝዙኽሩፍ -}  \RL{سوره  النجم}}}
\begin{longtable}{%
  @{}
    p{.5\textwidth}
  @{~~~~~~~~~~~~~}
    p{.5\textwidth}
    @{}
}
\nopagebreak
\textamh{\ \ \ \ \ \  ቢስሚላሂ አራህመኒ ራሂይም } &  بِسْمِ ٱللَّهِ ٱلرَّحْمَـٰنِ ٱلرَّحِيمِ\\
\textamh{1.\  } &  وَٱلنَّجْمِ إِذَا هَوَىٰ ﴿١﴾\\
\textamh{2.\  } & مَا ضَلَّ صَاحِبُكُمْ وَمَا غَوَىٰ ﴿٢﴾\\
\textamh{3.\  } & وَمَا يَنطِقُ عَنِ ٱلْهَوَىٰٓ ﴿٣﴾\\
\textamh{4.\  } & إِنْ هُوَ إِلَّا وَحْىٌۭ يُوحَىٰ ﴿٤﴾\\
\textamh{5.\  } & عَلَّمَهُۥ شَدِيدُ ٱلْقُوَىٰ ﴿٥﴾\\
\textamh{6.\  } & ذُو مِرَّةٍۢ فَٱسْتَوَىٰ ﴿٦﴾\\
\textamh{7.\  } & وَهُوَ بِٱلْأُفُقِ ٱلْأَعْلَىٰ ﴿٧﴾\\
\textamh{8.\  } & ثُمَّ دَنَا فَتَدَلَّىٰ ﴿٨﴾\\
\textamh{9.\  } & فَكَانَ قَابَ قَوْسَيْنِ أَوْ أَدْنَىٰ ﴿٩﴾\\
\textamh{10.\  } & فَأَوْحَىٰٓ إِلَىٰ عَبْدِهِۦ مَآ أَوْحَىٰ ﴿١٠﴾\\
\textamh{11.\  } & مَا كَذَبَ ٱلْفُؤَادُ مَا رَأَىٰٓ ﴿١١﴾\\
\textamh{12.\  } & أَفَتُمَـٰرُونَهُۥ عَلَىٰ مَا يَرَىٰ ﴿١٢﴾\\
\textamh{13.\  } & وَلَقَدْ رَءَاهُ نَزْلَةً أُخْرَىٰ ﴿١٣﴾\\
\textamh{14.\  } & عِندَ سِدْرَةِ ٱلْمُنتَهَىٰ ﴿١٤﴾\\
\textamh{15.\  } & عِندَهَا جَنَّةُ ٱلْمَأْوَىٰٓ ﴿١٥﴾\\
\textamh{16.\  } & إِذْ يَغْشَى ٱلسِّدْرَةَ مَا يَغْشَىٰ ﴿١٦﴾\\
\textamh{17.\  } & مَا زَاغَ ٱلْبَصَرُ وَمَا طَغَىٰ ﴿١٧﴾\\
\textamh{18.\  } & لَقَدْ رَأَىٰ مِنْ ءَايَـٰتِ رَبِّهِ ٱلْكُبْرَىٰٓ ﴿١٨﴾\\
\textamh{19.\  } & أَفَرَءَيْتُمُ ٱللَّٰتَ وَٱلْعُزَّىٰ ﴿١٩﴾\\
\textamh{20.\  } & وَمَنَوٰةَ ٱلثَّالِثَةَ ٱلْأُخْرَىٰٓ ﴿٢٠﴾\\
\textamh{21.\  } & أَلَكُمُ ٱلذَّكَرُ وَلَهُ ٱلْأُنثَىٰ ﴿٢١﴾\\
\textamh{22.\  } & تِلْكَ إِذًۭا قِسْمَةٌۭ ضِيزَىٰٓ ﴿٢٢﴾\\
\textamh{23.\  } & إِنْ هِىَ إِلَّآ أَسْمَآءٌۭ سَمَّيْتُمُوهَآ أَنتُمْ وَءَابَآؤُكُم مَّآ أَنزَلَ ٱللَّهُ بِهَا مِن سُلْطَٰنٍ ۚ إِن يَتَّبِعُونَ إِلَّا ٱلظَّنَّ وَمَا تَهْوَى ٱلْأَنفُسُ ۖ وَلَقَدْ جَآءَهُم مِّن رَّبِّهِمُ ٱلْهُدَىٰٓ ﴿٢٣﴾\\
\textamh{24.\  } & أَمْ لِلْإِنسَـٰنِ مَا تَمَنَّىٰ ﴿٢٤﴾\\
\textamh{25.\  } & فَلِلَّهِ ٱلْءَاخِرَةُ وَٱلْأُولَىٰ ﴿٢٥﴾\\
\textamh{26.\  } & ۞ وَكَم مِّن مَّلَكٍۢ فِى ٱلسَّمَـٰوَٟتِ لَا تُغْنِى شَفَـٰعَتُهُمْ شَيْـًٔا إِلَّا مِنۢ بَعْدِ أَن يَأْذَنَ ٱللَّهُ لِمَن يَشَآءُ وَيَرْضَىٰٓ ﴿٢٦﴾\\
\textamh{27.\  } & إِنَّ ٱلَّذِينَ لَا يُؤْمِنُونَ بِٱلْءَاخِرَةِ لَيُسَمُّونَ ٱلْمَلَـٰٓئِكَةَ تَسْمِيَةَ ٱلْأُنثَىٰ ﴿٢٧﴾\\
\textamh{28.\  } & وَمَا لَهُم بِهِۦ مِنْ عِلْمٍ ۖ إِن يَتَّبِعُونَ إِلَّا ٱلظَّنَّ ۖ وَإِنَّ ٱلظَّنَّ لَا يُغْنِى مِنَ ٱلْحَقِّ شَيْـًۭٔا ﴿٢٨﴾\\
\textamh{29.\  } & فَأَعْرِضْ عَن مَّن تَوَلَّىٰ عَن ذِكْرِنَا وَلَمْ يُرِدْ إِلَّا ٱلْحَيَوٰةَ ٱلدُّنْيَا ﴿٢٩﴾\\
\textamh{30.\  } & ذَٟلِكَ مَبْلَغُهُم مِّنَ ٱلْعِلْمِ ۚ إِنَّ رَبَّكَ هُوَ أَعْلَمُ بِمَن ضَلَّ عَن سَبِيلِهِۦ وَهُوَ أَعْلَمُ بِمَنِ ٱهْتَدَىٰ ﴿٣٠﴾\\
\textamh{31.\  } & وَلِلَّهِ مَا فِى ٱلسَّمَـٰوَٟتِ وَمَا فِى ٱلْأَرْضِ لِيَجْزِىَ ٱلَّذِينَ أَسَـٰٓـُٔوا۟ بِمَا عَمِلُوا۟ وَيَجْزِىَ ٱلَّذِينَ أَحْسَنُوا۟ بِٱلْحُسْنَى ﴿٣١﴾\\
\textamh{32.\  } & ٱلَّذِينَ يَجْتَنِبُونَ كَبَٰٓئِرَ ٱلْإِثْمِ وَٱلْفَوَٟحِشَ إِلَّا ٱللَّمَمَ ۚ إِنَّ رَبَّكَ وَٟسِعُ ٱلْمَغْفِرَةِ ۚ هُوَ أَعْلَمُ بِكُمْ إِذْ أَنشَأَكُم مِّنَ ٱلْأَرْضِ وَإِذْ أَنتُمْ أَجِنَّةٌۭ فِى بُطُونِ أُمَّهَـٰتِكُمْ ۖ فَلَا تُزَكُّوٓا۟ أَنفُسَكُمْ ۖ هُوَ أَعْلَمُ بِمَنِ ٱتَّقَىٰٓ ﴿٣٢﴾\\
\textamh{33.\  } & أَفَرَءَيْتَ ٱلَّذِى تَوَلَّىٰ ﴿٣٣﴾\\
\textamh{34.\  } & وَأَعْطَىٰ قَلِيلًۭا وَأَكْدَىٰٓ ﴿٣٤﴾\\
\textamh{35.\  } & أَعِندَهُۥ عِلْمُ ٱلْغَيْبِ فَهُوَ يَرَىٰٓ ﴿٣٥﴾\\
\textamh{36.\  } & أَمْ لَمْ يُنَبَّأْ بِمَا فِى صُحُفِ مُوسَىٰ ﴿٣٦﴾\\
\textamh{37.\  } & وَإِبْرَٰهِيمَ ٱلَّذِى وَفَّىٰٓ ﴿٣٧﴾\\
\textamh{38.\  } & أَلَّا تَزِرُ وَازِرَةٌۭ وِزْرَ أُخْرَىٰ ﴿٣٨﴾\\
\textamh{39.\  } & وَأَن لَّيْسَ لِلْإِنسَـٰنِ إِلَّا مَا سَعَىٰ ﴿٣٩﴾\\
\textamh{40.\  } & وَأَنَّ سَعْيَهُۥ سَوْفَ يُرَىٰ ﴿٤٠﴾\\
\textamh{41.\  } & ثُمَّ يُجْزَىٰهُ ٱلْجَزَآءَ ٱلْأَوْفَىٰ ﴿٤١﴾\\
\textamh{42.\  } & وَأَنَّ إِلَىٰ رَبِّكَ ٱلْمُنتَهَىٰ ﴿٤٢﴾\\
\textamh{43.\  } & وَأَنَّهُۥ هُوَ أَضْحَكَ وَأَبْكَىٰ ﴿٤٣﴾\\
\textamh{44.\  } & وَأَنَّهُۥ هُوَ أَمَاتَ وَأَحْيَا ﴿٤٤﴾\\
\textamh{45.\  } & وَأَنَّهُۥ خَلَقَ ٱلزَّوْجَيْنِ ٱلذَّكَرَ وَٱلْأُنثَىٰ ﴿٤٥﴾\\
\textamh{46.\  } & مِن نُّطْفَةٍ إِذَا تُمْنَىٰ ﴿٤٦﴾\\
\textamh{47.\  } & وَأَنَّ عَلَيْهِ ٱلنَّشْأَةَ ٱلْأُخْرَىٰ ﴿٤٧﴾\\
\textamh{48.\  } & وَأَنَّهُۥ هُوَ أَغْنَىٰ وَأَقْنَىٰ ﴿٤٨﴾\\
\textamh{49.\  } & وَأَنَّهُۥ هُوَ رَبُّ ٱلشِّعْرَىٰ ﴿٤٩﴾\\
\textamh{50.\  } & وَأَنَّهُۥٓ أَهْلَكَ عَادًا ٱلْأُولَىٰ ﴿٥٠﴾\\
\textamh{51.\  } & وَثَمُودَا۟ فَمَآ أَبْقَىٰ ﴿٥١﴾\\
\textamh{52.\  } & وَقَوْمَ نُوحٍۢ مِّن قَبْلُ ۖ إِنَّهُمْ كَانُوا۟ هُمْ أَظْلَمَ وَأَطْغَىٰ ﴿٥٢﴾\\
\textamh{53.\  } & وَٱلْمُؤْتَفِكَةَ أَهْوَىٰ ﴿٥٣﴾\\
\textamh{54.\  } & فَغَشَّىٰهَا مَا غَشَّىٰ ﴿٥٤﴾\\
\textamh{55.\  } & فَبِأَىِّ ءَالَآءِ رَبِّكَ تَتَمَارَىٰ ﴿٥٥﴾\\
\textamh{56.\  } & هَـٰذَا نَذِيرٌۭ مِّنَ ٱلنُّذُرِ ٱلْأُولَىٰٓ ﴿٥٦﴾\\
\textamh{57.\  } & أَزِفَتِ ٱلْءَازِفَةُ ﴿٥٧﴾\\
\textamh{58.\  } & لَيْسَ لَهَا مِن دُونِ ٱللَّهِ كَاشِفَةٌ ﴿٥٨﴾\\
\textamh{59.\  } & أَفَمِنْ هَـٰذَا ٱلْحَدِيثِ تَعْجَبُونَ ﴿٥٩﴾\\
\textamh{60.\  } & وَتَضْحَكُونَ وَلَا تَبْكُونَ ﴿٦٠﴾\\
\textamh{61.\  } & وَأَنتُمْ سَـٰمِدُونَ ﴿٦١﴾\\
\textamh{62.\  } & فَٱسْجُدُوا۟ لِلَّهِ وَٱعْبُدُوا۟ ۩ ﴿٦٢﴾\\
\end{longtable} \newpage

%% License: BSD style (Berkley) (i.e. Put the Copyright owner's name always)
%% Writer and Copyright (to): Bewketu(Bilal) Tadilo (2016-17)
\centering\section{\LR{\textamharic{ሱራቱ አልቀመር -}  \RL{سوره  القمر}}}
\begin{longtable}{%
  @{}
    p{.5\textwidth}
  @{~~~~~~~~~~~~}
    p{.5\textwidth}
    @{}
}
\nopagebreak
\textamh{ቢስሚላሂ አራህመኒ ራሂይም } &  بِسْمِ ٱللَّهِ ٱلرَّحْمَـٰنِ ٱلرَّحِيمِ\\
\textamh{1.\  } &  ٱقْتَرَبَتِ ٱلسَّاعَةُ وَٱنشَقَّ ٱلْقَمَرُ ﴿١﴾\\
\textamh{2.\  } & وَإِن يَرَوْا۟ ءَايَةًۭ يُعْرِضُوا۟ وَيَقُولُوا۟ سِحْرٌۭ مُّسْتَمِرٌّۭ ﴿٢﴾\\
\textamh{3.\  } & وَكَذَّبُوا۟ وَٱتَّبَعُوٓا۟ أَهْوَآءَهُمْ ۚ وَكُلُّ أَمْرٍۢ مُّسْتَقِرٌّۭ ﴿٣﴾\\
\textamh{4.\  } & وَلَقَدْ جَآءَهُم مِّنَ ٱلْأَنۢبَآءِ مَا فِيهِ مُزْدَجَرٌ ﴿٤﴾\\
\textamh{5.\  } & حِكْمَةٌۢ بَٰلِغَةٌۭ ۖ فَمَا تُغْنِ ٱلنُّذُرُ ﴿٥﴾\\
\textamh{6.\  } & فَتَوَلَّ عَنْهُمْ ۘ يَوْمَ يَدْعُ ٱلدَّاعِ إِلَىٰ شَىْءٍۢ نُّكُرٍ ﴿٦﴾\\
\textamh{7.\  } & خُشَّعًا أَبْصَـٰرُهُمْ يَخْرُجُونَ مِنَ ٱلْأَجْدَاثِ كَأَنَّهُمْ جَرَادٌۭ مُّنتَشِرٌۭ ﴿٧﴾\\
\textamh{8.\  } & مُّهْطِعِينَ إِلَى ٱلدَّاعِ ۖ يَقُولُ ٱلْكَـٰفِرُونَ هَـٰذَا يَوْمٌ عَسِرٌۭ ﴿٨﴾\\
\textamh{9.\  } & ۞ كَذَّبَتْ قَبْلَهُمْ قَوْمُ نُوحٍۢ فَكَذَّبُوا۟ عَبْدَنَا وَقَالُوا۟ مَجْنُونٌۭ وَٱزْدُجِرَ ﴿٩﴾\\
\textamh{10.\  } & فَدَعَا رَبَّهُۥٓ أَنِّى مَغْلُوبٌۭ فَٱنتَصِرْ ﴿١٠﴾\\
\textamh{11.\  } & فَفَتَحْنَآ أَبْوَٟبَ ٱلسَّمَآءِ بِمَآءٍۢ مُّنْهَمِرٍۢ ﴿١١﴾\\
\textamh{12.\  } & وَفَجَّرْنَا ٱلْأَرْضَ عُيُونًۭا فَٱلْتَقَى ٱلْمَآءُ عَلَىٰٓ أَمْرٍۢ قَدْ قُدِرَ ﴿١٢﴾\\
\textamh{13.\  } & وَحَمَلْنَـٰهُ عَلَىٰ ذَاتِ أَلْوَٟحٍۢ وَدُسُرٍۢ ﴿١٣﴾\\
\textamh{14.\  } & تَجْرِى بِأَعْيُنِنَا جَزَآءًۭ لِّمَن كَانَ كُفِرَ ﴿١٤﴾\\
\textamh{15.\  } & وَلَقَد تَّرَكْنَـٰهَآ ءَايَةًۭ فَهَلْ مِن مُّدَّكِرٍۢ ﴿١٥﴾\\
\textamh{16.\  } & فَكَيْفَ كَانَ عَذَابِى وَنُذُرِ ﴿١٦﴾\\
\textamh{17.\  } & وَلَقَدْ يَسَّرْنَا ٱلْقُرْءَانَ لِلذِّكْرِ فَهَلْ مِن مُّدَّكِرٍۢ ﴿١٧﴾\\
\textamh{18.\  } & كَذَّبَتْ عَادٌۭ فَكَيْفَ كَانَ عَذَابِى وَنُذُرِ ﴿١٨﴾\\
\textamh{19.\  } & إِنَّآ أَرْسَلْنَا عَلَيْهِمْ رِيحًۭا صَرْصَرًۭا فِى يَوْمِ نَحْسٍۢ مُّسْتَمِرٍّۢ ﴿١٩﴾\\
\textamh{20.\  } & تَنزِعُ ٱلنَّاسَ كَأَنَّهُمْ أَعْجَازُ نَخْلٍۢ مُّنقَعِرٍۢ ﴿٢٠﴾\\
\textamh{21.\  } & فَكَيْفَ كَانَ عَذَابِى وَنُذُرِ ﴿٢١﴾\\
\textamh{22.\  } & وَلَقَدْ يَسَّرْنَا ٱلْقُرْءَانَ لِلذِّكْرِ فَهَلْ مِن مُّدَّكِرٍۢ ﴿٢٢﴾\\
\textamh{23.\  } & كَذَّبَتْ ثَمُودُ بِٱلنُّذُرِ ﴿٢٣﴾\\
\textamh{24.\  } & فَقَالُوٓا۟ أَبَشَرًۭا مِّنَّا وَٟحِدًۭا نَّتَّبِعُهُۥٓ إِنَّآ إِذًۭا لَّفِى ضَلَـٰلٍۢ وَسُعُرٍ ﴿٢٤﴾\\
\textamh{25.\  } & أَءُلْقِىَ ٱلذِّكْرُ عَلَيْهِ مِنۢ بَيْنِنَا بَلْ هُوَ كَذَّابٌ أَشِرٌۭ ﴿٢٥﴾\\
\textamh{26.\  } & سَيَعْلَمُونَ غَدًۭا مَّنِ ٱلْكَذَّابُ ٱلْأَشِرُ ﴿٢٦﴾\\
\textamh{27.\  } & إِنَّا مُرْسِلُوا۟ ٱلنَّاقَةِ فِتْنَةًۭ لَّهُمْ فَٱرْتَقِبْهُمْ وَٱصْطَبِرْ ﴿٢٧﴾\\
\textamh{28.\  } & وَنَبِّئْهُمْ أَنَّ ٱلْمَآءَ قِسْمَةٌۢ بَيْنَهُمْ ۖ كُلُّ شِرْبٍۢ مُّحْتَضَرٌۭ ﴿٢٨﴾\\
\textamh{29.\  } & فَنَادَوْا۟ صَاحِبَهُمْ فَتَعَاطَىٰ فَعَقَرَ ﴿٢٩﴾\\
\textamh{30.\  } & فَكَيْفَ كَانَ عَذَابِى وَنُذُرِ ﴿٣٠﴾\\
\textamh{31.\  } & إِنَّآ أَرْسَلْنَا عَلَيْهِمْ صَيْحَةًۭ وَٟحِدَةًۭ فَكَانُوا۟ كَهَشِيمِ ٱلْمُحْتَظِرِ ﴿٣١﴾\\
\textamh{32.\  } & وَلَقَدْ يَسَّرْنَا ٱلْقُرْءَانَ لِلذِّكْرِ فَهَلْ مِن مُّدَّكِرٍۢ ﴿٣٢﴾\\
\textamh{33.\  } & كَذَّبَتْ قَوْمُ لُوطٍۭ بِٱلنُّذُرِ ﴿٣٣﴾\\
\textamh{34.\  } & إِنَّآ أَرْسَلْنَا عَلَيْهِمْ حَاصِبًا إِلَّآ ءَالَ لُوطٍۢ ۖ نَّجَّيْنَـٰهُم بِسَحَرٍۢ ﴿٣٤﴾\\
\textamh{35.\  } & نِّعْمَةًۭ مِّنْ عِندِنَا ۚ كَذَٟلِكَ نَجْزِى مَن شَكَرَ ﴿٣٥﴾\\
\textamh{36.\  } & وَلَقَدْ أَنذَرَهُم بَطْشَتَنَا فَتَمَارَوْا۟ بِٱلنُّذُرِ ﴿٣٦﴾\\
\textamh{37.\  } & وَلَقَدْ رَٰوَدُوهُ عَن ضَيْفِهِۦ فَطَمَسْنَآ أَعْيُنَهُمْ فَذُوقُوا۟ عَذَابِى وَنُذُرِ ﴿٣٧﴾\\
\textamh{38.\  } & وَلَقَدْ صَبَّحَهُم بُكْرَةً عَذَابٌۭ مُّسْتَقِرٌّۭ ﴿٣٨﴾\\
\textamh{39.\  } & فَذُوقُوا۟ عَذَابِى وَنُذُرِ ﴿٣٩﴾\\
\textamh{40.\  } & وَلَقَدْ يَسَّرْنَا ٱلْقُرْءَانَ لِلذِّكْرِ فَهَلْ مِن مُّدَّكِرٍۢ ﴿٤٠﴾\\
\textamh{41.\  } & وَلَقَدْ جَآءَ ءَالَ فِرْعَوْنَ ٱلنُّذُرُ ﴿٤١﴾\\
\textamh{42.\  } & كَذَّبُوا۟ بِـَٔايَـٰتِنَا كُلِّهَا فَأَخَذْنَـٰهُمْ أَخْذَ عَزِيزٍۢ مُّقْتَدِرٍ ﴿٤٢﴾\\
\textamh{43.\  } & أَكُفَّارُكُمْ خَيْرٌۭ مِّنْ أُو۟لَـٰٓئِكُمْ أَمْ لَكُم بَرَآءَةٌۭ فِى ٱلزُّبُرِ ﴿٤٣﴾\\
\textamh{44.\  } & أَمْ يَقُولُونَ نَحْنُ جَمِيعٌۭ مُّنتَصِرٌۭ ﴿٤٤﴾\\
\textamh{45.\  } & سَيُهْزَمُ ٱلْجَمْعُ وَيُوَلُّونَ ٱلدُّبُرَ ﴿٤٥﴾\\
\textamh{46.\  } & بَلِ ٱلسَّاعَةُ مَوْعِدُهُمْ وَٱلسَّاعَةُ أَدْهَىٰ وَأَمَرُّ ﴿٤٦﴾\\
\textamh{47.\  } & إِنَّ ٱلْمُجْرِمِينَ فِى ضَلَـٰلٍۢ وَسُعُرٍۢ ﴿٤٧﴾\\
\textamh{48.\  } & يَوْمَ يُسْحَبُونَ فِى ٱلنَّارِ عَلَىٰ وُجُوهِهِمْ ذُوقُوا۟ مَسَّ سَقَرَ ﴿٤٨﴾\\
\textamh{49.\  } & إِنَّا كُلَّ شَىْءٍ خَلَقْنَـٰهُ بِقَدَرٍۢ ﴿٤٩﴾\\
\textamh{50.\  } & وَمَآ أَمْرُنَآ إِلَّا وَٟحِدَةٌۭ كَلَمْحٍۭ بِٱلْبَصَرِ ﴿٥٠﴾\\
\textamh{51.\  } & وَلَقَدْ أَهْلَكْنَآ أَشْيَاعَكُمْ فَهَلْ مِن مُّدَّكِرٍۢ ﴿٥١﴾\\
\textamh{52.\  } & وَكُلُّ شَىْءٍۢ فَعَلُوهُ فِى ٱلزُّبُرِ ﴿٥٢﴾\\
\textamh{53.\  } & وَكُلُّ صَغِيرٍۢ وَكَبِيرٍۢ مُّسْتَطَرٌ ﴿٥٣﴾\\
\textamh{54.\  } & إِنَّ ٱلْمُتَّقِينَ فِى جَنَّـٰتٍۢ وَنَهَرٍۢ ﴿٥٤﴾\\
\textamh{55.\  } & فِى مَقْعَدِ صِدْقٍ عِندَ مَلِيكٍۢ مُّقْتَدِرٍۭ ﴿٥٥﴾\\
\end{longtable}
\clearpage
%% License: BSD style (Berkley) (i.e. Put the Copyright owner's name always)
%% Writer and Copyright (to): Bewketu(Bilal) Tadilo (2016-17)
\centering\section{\LR{\textamharic{ሱራቱ አርራህመን -}  \RL{سوره  الرحمن}}}
\begin{longtable}{%
  @{}
    p{.5\textwidth}
  @{~~~~~~~~~~~~~}
    p{.5\textwidth}
    @{}
}
\nopagebreak
\textamh{ቢስሚላሂ አራህመኒ ራሂይም } &  بِسْمِ ٱللَّهِ ٱلرَّحْمَـٰنِ ٱلرَّحِيمِ\\
\textamh{1.\  } &  ٱلرَّحْمَـٰنُ ﴿١﴾\\
\textamh{2.\  } & عَلَّمَ ٱلْقُرْءَانَ ﴿٢﴾\\
\textamh{3.\  } & خَلَقَ ٱلْإِنسَـٰنَ ﴿٣﴾\\
\textamh{4.\  } & عَلَّمَهُ ٱلْبَيَانَ ﴿٤﴾\\
\textamh{5.\  } & ٱلشَّمْسُ وَٱلْقَمَرُ بِحُسْبَانٍۢ ﴿٥﴾\\
\textamh{6.\  } & وَٱلنَّجْمُ وَٱلشَّجَرُ يَسْجُدَانِ ﴿٦﴾\\
\textamh{7.\  } & وَٱلسَّمَآءَ رَفَعَهَا وَوَضَعَ ٱلْمِيزَانَ ﴿٧﴾\\
\textamh{8.\  } & أَلَّا تَطْغَوْا۟ فِى ٱلْمِيزَانِ ﴿٨﴾\\
\textamh{9.\  } & وَأَقِيمُوا۟ ٱلْوَزْنَ بِٱلْقِسْطِ وَلَا تُخْسِرُوا۟ ٱلْمِيزَانَ ﴿٩﴾\\
\textamh{10.\  } & وَٱلْأَرْضَ وَضَعَهَا لِلْأَنَامِ ﴿١٠﴾\\
\textamh{11.\  } & فِيهَا فَـٰكِهَةٌۭ وَٱلنَّخْلُ ذَاتُ ٱلْأَكْمَامِ ﴿١١﴾\\
\textamh{12.\  } & وَٱلْحَبُّ ذُو ٱلْعَصْفِ وَٱلرَّيْحَانُ ﴿١٢﴾\\
\textamh{13.\  } & فَبِأَىِّ ءَالَآءِ رَبِّكُمَا تُكَذِّبَانِ ﴿١٣﴾\\
\textamh{14.\  } & خَلَقَ ٱلْإِنسَـٰنَ مِن صَلْصَـٰلٍۢ كَٱلْفَخَّارِ ﴿١٤﴾\\
\textamh{15.\  } & وَخَلَقَ ٱلْجَآنَّ مِن مَّارِجٍۢ مِّن نَّارٍۢ ﴿١٥﴾\\
\textamh{16.\  } & فَبِأَىِّ ءَالَآءِ رَبِّكُمَا تُكَذِّبَانِ ﴿١٦﴾\\
\textamh{17.\  } & رَبُّ ٱلْمَشْرِقَيْنِ وَرَبُّ ٱلْمَغْرِبَيْنِ ﴿١٧﴾\\
\textamh{18.\  } & فَبِأَىِّ ءَالَآءِ رَبِّكُمَا تُكَذِّبَانِ ﴿١٨﴾\\
\textamh{19.\  } & مَرَجَ ٱلْبَحْرَيْنِ يَلْتَقِيَانِ ﴿١٩﴾\\
\textamh{20.\  } & بَيْنَهُمَا بَرْزَخٌۭ لَّا يَبْغِيَانِ ﴿٢٠﴾\\
\textamh{21.\  } & فَبِأَىِّ ءَالَآءِ رَبِّكُمَا تُكَذِّبَانِ ﴿٢١﴾\\
\textamh{22.\  } & يَخْرُجُ مِنْهُمَا ٱللُّؤْلُؤُ وَٱلْمَرْجَانُ ﴿٢٢﴾\\
\textamh{23.\  } & فَبِأَىِّ ءَالَآءِ رَبِّكُمَا تُكَذِّبَانِ ﴿٢٣﴾\\
\textamh{24.\  } & وَلَهُ ٱلْجَوَارِ ٱلْمُنشَـَٔاتُ فِى ٱلْبَحْرِ كَٱلْأَعْلَـٰمِ ﴿٢٤﴾\\
\textamh{25.\  } & فَبِأَىِّ ءَالَآءِ رَبِّكُمَا تُكَذِّبَانِ ﴿٢٥﴾\\
\textamh{26.\  } & كُلُّ مَنْ عَلَيْهَا فَانٍۢ ﴿٢٦﴾\\
\textamh{27.\  } & وَيَبْقَىٰ وَجْهُ رَبِّكَ ذُو ٱلْجَلَـٰلِ وَٱلْإِكْرَامِ ﴿٢٧﴾\\
\textamh{28.\  } & فَبِأَىِّ ءَالَآءِ رَبِّكُمَا تُكَذِّبَانِ ﴿٢٨﴾\\
\textamh{29.\  } & يَسْـَٔلُهُۥ مَن فِى ٱلسَّمَـٰوَٟتِ وَٱلْأَرْضِ ۚ كُلَّ يَوْمٍ هُوَ فِى شَأْنٍۢ ﴿٢٩﴾\\
\textamh{30.\  } & فَبِأَىِّ ءَالَآءِ رَبِّكُمَا تُكَذِّبَانِ ﴿٣٠﴾\\
\textamh{31.\  } & سَنَفْرُغُ لَكُمْ أَيُّهَ ٱلثَّقَلَانِ ﴿٣١﴾\\
\textamh{32.\  } & فَبِأَىِّ ءَالَآءِ رَبِّكُمَا تُكَذِّبَانِ ﴿٣٢﴾\\
\textamh{33.\  } & يَـٰمَعْشَرَ ٱلْجِنِّ وَٱلْإِنسِ إِنِ ٱسْتَطَعْتُمْ أَن تَنفُذُوا۟ مِنْ أَقْطَارِ ٱلسَّمَـٰوَٟتِ وَٱلْأَرْضِ فَٱنفُذُوا۟ ۚ لَا تَنفُذُونَ إِلَّا بِسُلْطَٰنٍۢ ﴿٣٣﴾\\
\textamh{34.\  } & فَبِأَىِّ ءَالَآءِ رَبِّكُمَا تُكَذِّبَانِ ﴿٣٤﴾\\
\textamh{35.\  } & يُرْسَلُ عَلَيْكُمَا شُوَاظٌۭ مِّن نَّارٍۢ وَنُحَاسٌۭ فَلَا تَنتَصِرَانِ ﴿٣٥﴾\\
\textamh{36.\  } & فَبِأَىِّ ءَالَآءِ رَبِّكُمَا تُكَذِّبَانِ ﴿٣٦﴾\\
\textamh{37.\  } & فَإِذَا ٱنشَقَّتِ ٱلسَّمَآءُ فَكَانَتْ وَرْدَةًۭ كَٱلدِّهَانِ ﴿٣٧﴾\\
\textamh{38.\  } & فَبِأَىِّ ءَالَآءِ رَبِّكُمَا تُكَذِّبَانِ ﴿٣٨﴾\\
\textamh{39.\  } & فَيَوْمَئِذٍۢ لَّا يُسْـَٔلُ عَن ذَنۢبِهِۦٓ إِنسٌۭ وَلَا جَآنٌّۭ ﴿٣٩﴾\\
\textamh{40.\  } & فَبِأَىِّ ءَالَآءِ رَبِّكُمَا تُكَذِّبَانِ ﴿٤٠﴾\\
\textamh{41.\  } & يُعْرَفُ ٱلْمُجْرِمُونَ بِسِيمَـٰهُمْ فَيُؤْخَذُ بِٱلنَّوَٟصِى وَٱلْأَقْدَامِ ﴿٤١﴾\\
\textamh{42.\  } & فَبِأَىِّ ءَالَآءِ رَبِّكُمَا تُكَذِّبَانِ ﴿٤٢﴾\\
\textamh{43.\  } & هَـٰذِهِۦ جَهَنَّمُ ٱلَّتِى يُكَذِّبُ بِهَا ٱلْمُجْرِمُونَ ﴿٤٣﴾\\
\textamh{44.\  } & يَطُوفُونَ بَيْنَهَا وَبَيْنَ حَمِيمٍ ءَانٍۢ ﴿٤٤﴾\\
\textamh{45.\  } & فَبِأَىِّ ءَالَآءِ رَبِّكُمَا تُكَذِّبَانِ ﴿٤٥﴾\\
\textamh{46.\  } & وَلِمَنْ خَافَ مَقَامَ رَبِّهِۦ جَنَّتَانِ ﴿٤٦﴾\\
\textamh{47.\  } & فَبِأَىِّ ءَالَآءِ رَبِّكُمَا تُكَذِّبَانِ ﴿٤٧﴾\\
\textamh{48.\  } & ذَوَاتَآ أَفْنَانٍۢ ﴿٤٨﴾\\
\textamh{49.\  } & فَبِأَىِّ ءَالَآءِ رَبِّكُمَا تُكَذِّبَانِ ﴿٤٩﴾\\
\textamh{50.\  } & فِيهِمَا عَيْنَانِ تَجْرِيَانِ ﴿٥٠﴾\\
\textamh{51.\  } & فَبِأَىِّ ءَالَآءِ رَبِّكُمَا تُكَذِّبَانِ ﴿٥١﴾\\
\textamh{52.\  } & فِيهِمَا مِن كُلِّ فَـٰكِهَةٍۢ زَوْجَانِ ﴿٥٢﴾\\
\textamh{53.\  } & فَبِأَىِّ ءَالَآءِ رَبِّكُمَا تُكَذِّبَانِ ﴿٥٣﴾\\
\textamh{54.\  } & مُتَّكِـِٔينَ عَلَىٰ فُرُشٍۭ بَطَآئِنُهَا مِنْ إِسْتَبْرَقٍۢ ۚ وَجَنَى ٱلْجَنَّتَيْنِ دَانٍۢ ﴿٥٤﴾\\
\textamh{55.\  } & فَبِأَىِّ ءَالَآءِ رَبِّكُمَا تُكَذِّبَانِ ﴿٥٥﴾\\
\textamh{56.\  } & فِيهِنَّ قَـٰصِرَٰتُ ٱلطَّرْفِ لَمْ يَطْمِثْهُنَّ إِنسٌۭ قَبْلَهُمْ وَلَا جَآنٌّۭ ﴿٥٦﴾\\
\textamh{57.\  } & فَبِأَىِّ ءَالَآءِ رَبِّكُمَا تُكَذِّبَانِ ﴿٥٧﴾\\
\textamh{58.\  } & كَأَنَّهُنَّ ٱلْيَاقُوتُ وَٱلْمَرْجَانُ ﴿٥٨﴾\\
\textamh{59.\  } & فَبِأَىِّ ءَالَآءِ رَبِّكُمَا تُكَذِّبَانِ ﴿٥٩﴾\\
\textamh{60.\  } & هَلْ جَزَآءُ ٱلْإِحْسَـٰنِ إِلَّا ٱلْإِحْسَـٰنُ ﴿٦٠﴾\\
\textamh{61.\  } & فَبِأَىِّ ءَالَآءِ رَبِّكُمَا تُكَذِّبَانِ ﴿٦١﴾\\
\textamh{62.\  } & وَمِن دُونِهِمَا جَنَّتَانِ ﴿٦٢﴾\\
\textamh{63.\  } & فَبِأَىِّ ءَالَآءِ رَبِّكُمَا تُكَذِّبَانِ ﴿٦٣﴾\\
\textamh{64.\  } & مُدْهَآمَّتَانِ ﴿٦٤﴾\\
\textamh{65.\  } & فَبِأَىِّ ءَالَآءِ رَبِّكُمَا تُكَذِّبَانِ ﴿٦٥﴾\\
\textamh{66.\  } & فِيهِمَا عَيْنَانِ نَضَّاخَتَانِ ﴿٦٦﴾\\
\textamh{67.\  } & فَبِأَىِّ ءَالَآءِ رَبِّكُمَا تُكَذِّبَانِ ﴿٦٧﴾\\
\textamh{68.\  } & فِيهِمَا فَـٰكِهَةٌۭ وَنَخْلٌۭ وَرُمَّانٌۭ ﴿٦٨﴾\\
\textamh{69.\  } & فَبِأَىِّ ءَالَآءِ رَبِّكُمَا تُكَذِّبَانِ ﴿٦٩﴾\\
\textamh{70.\  } & فِيهِنَّ خَيْرَٰتٌ حِسَانٌۭ ﴿٧٠﴾\\
\textamh{71.\  } & فَبِأَىِّ ءَالَآءِ رَبِّكُمَا تُكَذِّبَانِ ﴿٧١﴾\\
\textamh{72.\  } & حُورٌۭ مَّقْصُورَٰتٌۭ فِى ٱلْخِيَامِ ﴿٧٢﴾\\
\textamh{73.\  } & فَبِأَىِّ ءَالَآءِ رَبِّكُمَا تُكَذِّبَانِ ﴿٧٣﴾\\
\textamh{74.\  } & لَمْ يَطْمِثْهُنَّ إِنسٌۭ قَبْلَهُمْ وَلَا جَآنٌّۭ ﴿٧٤﴾\\
\textamh{75.\  } & فَبِأَىِّ ءَالَآءِ رَبِّكُمَا تُكَذِّبَانِ ﴿٧٥﴾\\
\textamh{76.\  } & مُتَّكِـِٔينَ عَلَىٰ رَفْرَفٍ خُضْرٍۢ وَعَبْقَرِىٍّ حِسَانٍۢ ﴿٧٦﴾\\
\textamh{77.\  } & فَبِأَىِّ ءَالَآءِ رَبِّكُمَا تُكَذِّبَانِ ﴿٧٧﴾\\
\textamh{78.\  } & تَبَٰرَكَ ٱسْمُ رَبِّكَ ذِى ٱلْجَلَـٰلِ وَٱلْإِكْرَامِ ﴿٧٨﴾\\
\end{longtable}
\clearpage
%% License: BSD style (Berkley) (i.e. Put the Copyright owner's name always)
%% Writer and Copyright (to): Bewketu(Bilal) Tadilo (2016-17)
\centering\section{\LR{\textamharic{ሱራቱ አልዋቂያ -}  \RL{سوره  الواقعة}}}
\begin{longtable}{%
  @{}
    p{.5\textwidth}
  @{~~~~~~~~~~~~~}
    p{.5\textwidth}
    @{}
}
\nopagebreak
\textamh{\ \ \ \ \ \  ቢስሚላሂ አራህመኒ ራሂይም } &  بِسْمِ ٱللَّهِ ٱلرَّحْمَـٰنِ ٱلرَّحِيمِ\\
\textamh{1.\  } &  إِذَا وَقَعَتِ ٱلْوَاقِعَةُ ﴿١﴾\\
\textamh{2.\  } & لَيْسَ لِوَقْعَتِهَا كَاذِبَةٌ ﴿٢﴾\\
\textamh{3.\  } & خَافِضَةٌۭ رَّافِعَةٌ ﴿٣﴾\\
\textamh{4.\  } & إِذَا رُجَّتِ ٱلْأَرْضُ رَجًّۭا ﴿٤﴾\\
\textamh{5.\  } & وَبُسَّتِ ٱلْجِبَالُ بَسًّۭا ﴿٥﴾\\
\textamh{6.\  } & فَكَانَتْ هَبَآءًۭ مُّنۢبَثًّۭا ﴿٦﴾\\
\textamh{7.\  } & وَكُنتُمْ أَزْوَٟجًۭا ثَلَـٰثَةًۭ ﴿٧﴾\\
\textamh{8.\  } & فَأَصْحَـٰبُ ٱلْمَيْمَنَةِ مَآ أَصْحَـٰبُ ٱلْمَيْمَنَةِ ﴿٨﴾\\
\textamh{9.\  } & وَأَصْحَـٰبُ ٱلْمَشْـَٔمَةِ مَآ أَصْحَـٰبُ ٱلْمَشْـَٔمَةِ ﴿٩﴾\\
\textamh{10.\  } & وَٱلسَّٰبِقُونَ ٱلسَّٰبِقُونَ ﴿١٠﴾\\
\textamh{11.\  } & أُو۟لَـٰٓئِكَ ٱلْمُقَرَّبُونَ ﴿١١﴾\\
\textamh{12.\  } & فِى جَنَّـٰتِ ٱلنَّعِيمِ ﴿١٢﴾\\
\textamh{13.\  } & ثُلَّةٌۭ مِّنَ ٱلْأَوَّلِينَ ﴿١٣﴾\\
\textamh{14.\  } & وَقَلِيلٌۭ مِّنَ ٱلْءَاخِرِينَ ﴿١٤﴾\\
\textamh{15.\  } & عَلَىٰ سُرُرٍۢ مَّوْضُونَةٍۢ ﴿١٥﴾\\
\textamh{16.\  } & مُّتَّكِـِٔينَ عَلَيْهَا مُتَقَـٰبِلِينَ ﴿١٦﴾\\
\textamh{17.\  } & يَطُوفُ عَلَيْهِمْ وِلْدَٟنٌۭ مُّخَلَّدُونَ ﴿١٧﴾\\
\textamh{18.\  } & بِأَكْوَابٍۢ وَأَبَارِيقَ وَكَأْسٍۢ مِّن مَّعِينٍۢ ﴿١٨﴾\\
\textamh{19.\  } & لَّا يُصَدَّعُونَ عَنْهَا وَلَا يُنزِفُونَ ﴿١٩﴾\\
\textamh{20.\  } & وَفَـٰكِهَةٍۢ مِّمَّا يَتَخَيَّرُونَ ﴿٢٠﴾\\
\textamh{21.\  } & وَلَحْمِ طَيْرٍۢ مِّمَّا يَشْتَهُونَ ﴿٢١﴾\\
\textamh{22.\  } & وَحُورٌ عِينٌۭ ﴿٢٢﴾\\
\textamh{23.\  } & كَأَمْثَـٰلِ ٱللُّؤْلُؤِ ٱلْمَكْنُونِ ﴿٢٣﴾\\
\textamh{24.\  } & جَزَآءًۢ بِمَا كَانُوا۟ يَعْمَلُونَ ﴿٢٤﴾\\
\textamh{25.\  } & لَا يَسْمَعُونَ فِيهَا لَغْوًۭا وَلَا تَأْثِيمًا ﴿٢٥﴾\\
\textamh{26.\  } & إِلَّا قِيلًۭا سَلَـٰمًۭا سَلَـٰمًۭا ﴿٢٦﴾\\
\textamh{27.\  } & وَأَصْحَـٰبُ ٱلْيَمِينِ مَآ أَصْحَـٰبُ ٱلْيَمِينِ ﴿٢٧﴾\\
\textamh{28.\  } & فِى سِدْرٍۢ مَّخْضُودٍۢ ﴿٢٨﴾\\
\textamh{29.\  } & وَطَلْحٍۢ مَّنضُودٍۢ ﴿٢٩﴾\\
\textamh{30.\  } & وَظِلٍّۢ مَّمْدُودٍۢ ﴿٣٠﴾\\
\textamh{31.\  } & وَمَآءٍۢ مَّسْكُوبٍۢ ﴿٣١﴾\\
\textamh{32.\  } & وَفَـٰكِهَةٍۢ كَثِيرَةٍۢ ﴿٣٢﴾\\
\textamh{33.\  } & لَّا مَقْطُوعَةٍۢ وَلَا مَمْنُوعَةٍۢ ﴿٣٣﴾\\
\textamh{34.\  } & وَفُرُشٍۢ مَّرْفُوعَةٍ ﴿٣٤﴾\\
\textamh{35.\  } & إِنَّآ أَنشَأْنَـٰهُنَّ إِنشَآءًۭ ﴿٣٥﴾\\
\textamh{36.\  } & فَجَعَلْنَـٰهُنَّ أَبْكَارًا ﴿٣٦﴾\\
\textamh{37.\  } & عُرُبًا أَتْرَابًۭا ﴿٣٧﴾\\
\textamh{38.\  } & لِّأَصْحَـٰبِ ٱلْيَمِينِ ﴿٣٨﴾\\
\textamh{39.\  } & ثُلَّةٌۭ مِّنَ ٱلْأَوَّلِينَ ﴿٣٩﴾\\
\textamh{40.\  } & وَثُلَّةٌۭ مِّنَ ٱلْءَاخِرِينَ ﴿٤٠﴾\\
\textamh{41.\  } & وَأَصْحَـٰبُ ٱلشِّمَالِ مَآ أَصْحَـٰبُ ٱلشِّمَالِ ﴿٤١﴾\\
\textamh{42.\  } & فِى سَمُومٍۢ وَحَمِيمٍۢ ﴿٤٢﴾\\
\textamh{43.\  } & وَظِلٍّۢ مِّن يَحْمُومٍۢ ﴿٤٣﴾\\
\textamh{44.\  } & لَّا بَارِدٍۢ وَلَا كَرِيمٍ ﴿٤٤﴾\\
\textamh{45.\  } & إِنَّهُمْ كَانُوا۟ قَبْلَ ذَٟلِكَ مُتْرَفِينَ ﴿٤٥﴾\\
\textamh{46.\  } & وَكَانُوا۟ يُصِرُّونَ عَلَى ٱلْحِنثِ ٱلْعَظِيمِ ﴿٤٦﴾\\
\textamh{47.\  } & وَكَانُوا۟ يَقُولُونَ أَئِذَا مِتْنَا وَكُنَّا تُرَابًۭا وَعِظَـٰمًا أَءِنَّا لَمَبْعُوثُونَ ﴿٤٧﴾\\
\textamh{48.\  } & أَوَءَابَآؤُنَا ٱلْأَوَّلُونَ ﴿٤٨﴾\\
\textamh{49.\  } & قُلْ إِنَّ ٱلْأَوَّلِينَ وَٱلْءَاخِرِينَ ﴿٤٩﴾\\
\textamh{50.\  } & لَمَجْمُوعُونَ إِلَىٰ مِيقَـٰتِ يَوْمٍۢ مَّعْلُومٍۢ ﴿٥٠﴾\\
\textamh{51.\  } & ثُمَّ إِنَّكُمْ أَيُّهَا ٱلضَّآلُّونَ ٱلْمُكَذِّبُونَ ﴿٥١﴾\\
\textamh{52.\  } & لَءَاكِلُونَ مِن شَجَرٍۢ مِّن زَقُّومٍۢ ﴿٥٢﴾\\
\textamh{53.\  } & فَمَالِـُٔونَ مِنْهَا ٱلْبُطُونَ ﴿٥٣﴾\\
\textamh{54.\  } & فَشَـٰرِبُونَ عَلَيْهِ مِنَ ٱلْحَمِيمِ ﴿٥٤﴾\\
\textamh{55.\  } & فَشَـٰرِبُونَ شُرْبَ ٱلْهِيمِ ﴿٥٥﴾\\
\textamh{56.\  } & هَـٰذَا نُزُلُهُمْ يَوْمَ ٱلدِّينِ ﴿٥٦﴾\\
\textamh{57.\  } & نَحْنُ خَلَقْنَـٰكُمْ فَلَوْلَا تُصَدِّقُونَ ﴿٥٧﴾\\
\textamh{58.\  } & أَفَرَءَيْتُم مَّا تُمْنُونَ ﴿٥٨﴾\\
\textamh{59.\  } & ءَأَنتُمْ تَخْلُقُونَهُۥٓ أَمْ نَحْنُ ٱلْخَـٰلِقُونَ ﴿٥٩﴾\\
\textamh{60.\  } & نَحْنُ قَدَّرْنَا بَيْنَكُمُ ٱلْمَوْتَ وَمَا نَحْنُ بِمَسْبُوقِينَ ﴿٦٠﴾\\
\textamh{61.\  } & عَلَىٰٓ أَن نُّبَدِّلَ أَمْثَـٰلَكُمْ وَنُنشِئَكُمْ فِى مَا لَا تَعْلَمُونَ ﴿٦١﴾\\
\textamh{62.\  } & وَلَقَدْ عَلِمْتُمُ ٱلنَّشْأَةَ ٱلْأُولَىٰ فَلَوْلَا تَذَكَّرُونَ ﴿٦٢﴾\\
\textamh{63.\  } & أَفَرَءَيْتُم مَّا تَحْرُثُونَ ﴿٦٣﴾\\
\textamh{64.\  } & ءَأَنتُمْ تَزْرَعُونَهُۥٓ أَمْ نَحْنُ ٱلزَّٰرِعُونَ ﴿٦٤﴾\\
\textamh{65.\  } & لَوْ نَشَآءُ لَجَعَلْنَـٰهُ حُطَٰمًۭا فَظَلْتُمْ تَفَكَّهُونَ ﴿٦٥﴾\\
\textamh{66.\  } & إِنَّا لَمُغْرَمُونَ ﴿٦٦﴾\\
\textamh{67.\  } & بَلْ نَحْنُ مَحْرُومُونَ ﴿٦٧﴾\\
\textamh{68.\  } & أَفَرَءَيْتُمُ ٱلْمَآءَ ٱلَّذِى تَشْرَبُونَ ﴿٦٨﴾\\
\textamh{69.\  } & ءَأَنتُمْ أَنزَلْتُمُوهُ مِنَ ٱلْمُزْنِ أَمْ نَحْنُ ٱلْمُنزِلُونَ ﴿٦٩﴾\\
\textamh{70.\  } & لَوْ نَشَآءُ جَعَلْنَـٰهُ أُجَاجًۭا فَلَوْلَا تَشْكُرُونَ ﴿٧٠﴾\\
\textamh{71.\  } & أَفَرَءَيْتُمُ ٱلنَّارَ ٱلَّتِى تُورُونَ ﴿٧١﴾\\
\textamh{72.\  } & ءَأَنتُمْ أَنشَأْتُمْ شَجَرَتَهَآ أَمْ نَحْنُ ٱلْمُنشِـُٔونَ ﴿٧٢﴾\\
\textamh{73.\  } & نَحْنُ جَعَلْنَـٰهَا تَذْكِرَةًۭ وَمَتَـٰعًۭا لِّلْمُقْوِينَ ﴿٧٣﴾\\
\textamh{74.\  } & فَسَبِّحْ بِٱسْمِ رَبِّكَ ٱلْعَظِيمِ ﴿٧٤﴾\\
\textamh{75.\  } & ۞ فَلَآ أُقْسِمُ بِمَوَٟقِعِ ٱلنُّجُومِ ﴿٧٥﴾\\
\textamh{76.\  } & وَإِنَّهُۥ لَقَسَمٌۭ لَّوْ تَعْلَمُونَ عَظِيمٌ ﴿٧٦﴾\\
\textamh{77.\  } & إِنَّهُۥ لَقُرْءَانٌۭ كَرِيمٌۭ ﴿٧٧﴾\\
\textamh{78.\  } & فِى كِتَـٰبٍۢ مَّكْنُونٍۢ ﴿٧٨﴾\\
\textamh{79.\  } & لَّا يَمَسُّهُۥٓ إِلَّا ٱلْمُطَهَّرُونَ ﴿٧٩﴾\\
\textamh{80.\  } & تَنزِيلٌۭ مِّن رَّبِّ ٱلْعَـٰلَمِينَ ﴿٨٠﴾\\
\textamh{81.\  } & أَفَبِهَـٰذَا ٱلْحَدِيثِ أَنتُم مُّدْهِنُونَ ﴿٨١﴾\\
\textamh{82.\  } & وَتَجْعَلُونَ رِزْقَكُمْ أَنَّكُمْ تُكَذِّبُونَ ﴿٨٢﴾\\
\textamh{83.\  } & فَلَوْلَآ إِذَا بَلَغَتِ ٱلْحُلْقُومَ ﴿٨٣﴾\\
\textamh{84.\  } & وَأَنتُمْ حِينَئِذٍۢ تَنظُرُونَ ﴿٨٤﴾\\
\textamh{85.\  } & وَنَحْنُ أَقْرَبُ إِلَيْهِ مِنكُمْ وَلَـٰكِن لَّا تُبْصِرُونَ ﴿٨٥﴾\\
\textamh{86.\  } & فَلَوْلَآ إِن كُنتُمْ غَيْرَ مَدِينِينَ ﴿٨٦﴾\\
\textamh{87.\  } & تَرْجِعُونَهَآ إِن كُنتُمْ صَـٰدِقِينَ ﴿٨٧﴾\\
\textamh{88.\  } & فَأَمَّآ إِن كَانَ مِنَ ٱلْمُقَرَّبِينَ ﴿٨٨﴾\\
\textamh{89.\  } & فَرَوْحٌۭ وَرَيْحَانٌۭ وَجَنَّتُ نَعِيمٍۢ ﴿٨٩﴾\\
\textamh{90.\  } & وَأَمَّآ إِن كَانَ مِنْ أَصْحَـٰبِ ٱلْيَمِينِ ﴿٩٠﴾\\
\textamh{91.\  } & فَسَلَـٰمٌۭ لَّكَ مِنْ أَصْحَـٰبِ ٱلْيَمِينِ ﴿٩١﴾\\
\textamh{92.\  } & وَأَمَّآ إِن كَانَ مِنَ ٱلْمُكَذِّبِينَ ٱلضَّآلِّينَ ﴿٩٢﴾\\
\textamh{93.\  } & فَنُزُلٌۭ مِّنْ حَمِيمٍۢ ﴿٩٣﴾\\
\textamh{94.\  } & وَتَصْلِيَةُ جَحِيمٍ ﴿٩٤﴾\\
\textamh{95.\  } & إِنَّ هَـٰذَا لَهُوَ حَقُّ ٱلْيَقِينِ ﴿٩٥﴾\\
\textamh{96.\  } & فَسَبِّحْ بِٱسْمِ رَبِّكَ ٱلْعَظِيمِ ﴿٩٦﴾\\
\end{longtable} \newpage

%% License: BSD style (Berkley) (i.e. Put the Copyright owner's name always)
%% Writer and Copyright (to): Bewketu(Bilal) Tadilo (2016-17)
\begin{center}\section{\LR{\textamhsec{ሱራቱ አልሀዲይድ -}  \textarabic{سوره  الحديد}}}\end{center}
\begin{longtable}{%
  @{}
    p{.5\textwidth}
  @{~~~}
    p{.5\textwidth}
    @{}
}
\textamh{ቢስሚላሂ አራህመኒ ራሂይም } &  \mytextarabic{بِسْمِ ٱللَّهِ ٱلرَّحْمَـٰنِ ٱلرَّحِيمِ}\\
\textamh{1.\  } & \mytextarabic{ سَبَّحَ لِلَّهِ مَا فِى ٱلسَّمَـٰوَٟتِ وَٱلْأَرْضِ ۖ وَهُوَ ٱلْعَزِيزُ ٱلْحَكِيمُ ﴿١﴾}\\
\textamh{2.\  } & \mytextarabic{لَهُۥ مُلْكُ ٱلسَّمَـٰوَٟتِ وَٱلْأَرْضِ ۖ يُحْىِۦ وَيُمِيتُ ۖ وَهُوَ عَلَىٰ كُلِّ شَىْءٍۢ قَدِيرٌ ﴿٢﴾}\\
\textamh{3.\  } & \mytextarabic{هُوَ ٱلْأَوَّلُ وَٱلْءَاخِرُ وَٱلظَّـٰهِرُ وَٱلْبَاطِنُ ۖ وَهُوَ بِكُلِّ شَىْءٍ عَلِيمٌ ﴿٣﴾}\\
\textamh{4.\  } & \mytextarabic{هُوَ ٱلَّذِى خَلَقَ ٱلسَّمَـٰوَٟتِ وَٱلْأَرْضَ فِى سِتَّةِ أَيَّامٍۢ ثُمَّ ٱسْتَوَىٰ عَلَى ٱلْعَرْشِ ۚ يَعْلَمُ مَا يَلِجُ فِى ٱلْأَرْضِ وَمَا يَخْرُجُ مِنْهَا وَمَا يَنزِلُ مِنَ ٱلسَّمَآءِ وَمَا يَعْرُجُ فِيهَا ۖ وَهُوَ مَعَكُمْ أَيْنَ مَا كُنتُمْ ۚ وَٱللَّهُ بِمَا تَعْمَلُونَ بَصِيرٌۭ ﴿٤﴾}\\
\textamh{5.\  } & \mytextarabic{لَّهُۥ مُلْكُ ٱلسَّمَـٰوَٟتِ وَٱلْأَرْضِ ۚ وَإِلَى ٱللَّهِ تُرْجَعُ ٱلْأُمُورُ ﴿٥﴾}\\
\textamh{6.\  } & \mytextarabic{يُولِجُ ٱلَّيْلَ فِى ٱلنَّهَارِ وَيُولِجُ ٱلنَّهَارَ فِى ٱلَّيْلِ ۚ وَهُوَ عَلِيمٌۢ بِذَاتِ ٱلصُّدُورِ ﴿٦﴾}\\
\textamh{7.\  } & \mytextarabic{ءَامِنُوا۟ بِٱللَّهِ وَرَسُولِهِۦ وَأَنفِقُوا۟ مِمَّا جَعَلَكُم مُّسْتَخْلَفِينَ فِيهِ ۖ فَٱلَّذِينَ ءَامَنُوا۟ مِنكُمْ وَأَنفَقُوا۟ لَهُمْ أَجْرٌۭ كَبِيرٌۭ ﴿٧﴾}\\
\textamh{8.\  } & \mytextarabic{وَمَا لَكُمْ لَا تُؤْمِنُونَ بِٱللَّهِ ۙ وَٱلرَّسُولُ يَدْعُوكُمْ لِتُؤْمِنُوا۟ بِرَبِّكُمْ وَقَدْ أَخَذَ مِيثَـٰقَكُمْ إِن كُنتُم مُّؤْمِنِينَ ﴿٨﴾}\\
\textamh{9.\  } & \mytextarabic{هُوَ ٱلَّذِى يُنَزِّلُ عَلَىٰ عَبْدِهِۦٓ ءَايَـٰتٍۭ بَيِّنَـٰتٍۢ لِّيُخْرِجَكُم مِّنَ ٱلظُّلُمَـٰتِ إِلَى ٱلنُّورِ ۚ وَإِنَّ ٱللَّهَ بِكُمْ لَرَءُوفٌۭ رَّحِيمٌۭ ﴿٩﴾}\\
\textamh{10.\  } & \mytextarabic{وَمَا لَكُمْ أَلَّا تُنفِقُوا۟ فِى سَبِيلِ ٱللَّهِ وَلِلَّهِ مِيرَٰثُ ٱلسَّمَـٰوَٟتِ وَٱلْأَرْضِ ۚ لَا يَسْتَوِى مِنكُم مَّنْ أَنفَقَ مِن قَبْلِ ٱلْفَتْحِ وَقَـٰتَلَ ۚ أُو۟لَـٰٓئِكَ أَعْظَمُ دَرَجَةًۭ مِّنَ ٱلَّذِينَ أَنفَقُوا۟ مِنۢ بَعْدُ وَقَـٰتَلُوا۟ ۚ وَكُلًّۭا وَعَدَ ٱللَّهُ ٱلْحُسْنَىٰ ۚ وَٱللَّهُ بِمَا تَعْمَلُونَ خَبِيرٌۭ ﴿١٠﴾}\\
\textamh{11.\  } & \mytextarabic{مَّن ذَا ٱلَّذِى يُقْرِضُ ٱللَّهَ قَرْضًا حَسَنًۭا فَيُضَٰعِفَهُۥ لَهُۥ وَلَهُۥٓ أَجْرٌۭ كَرِيمٌۭ ﴿١١﴾}\\
\textamh{12.\  } & \mytextarabic{يَوْمَ تَرَى ٱلْمُؤْمِنِينَ وَٱلْمُؤْمِنَـٰتِ يَسْعَىٰ نُورُهُم بَيْنَ أَيْدِيهِمْ وَبِأَيْمَـٰنِهِم بُشْرَىٰكُمُ ٱلْيَوْمَ جَنَّـٰتٌۭ تَجْرِى مِن تَحْتِهَا ٱلْأَنْهَـٰرُ خَـٰلِدِينَ فِيهَا ۚ ذَٟلِكَ هُوَ ٱلْفَوْزُ ٱلْعَظِيمُ ﴿١٢﴾}\\
\textamh{13.\  } & \mytextarabic{يَوْمَ يَقُولُ ٱلْمُنَـٰفِقُونَ وَٱلْمُنَـٰفِقَـٰتُ لِلَّذِينَ ءَامَنُوا۟ ٱنظُرُونَا نَقْتَبِسْ مِن نُّورِكُمْ قِيلَ ٱرْجِعُوا۟ وَرَآءَكُمْ فَٱلْتَمِسُوا۟ نُورًۭا فَضُرِبَ بَيْنَهُم بِسُورٍۢ لَّهُۥ بَابٌۢ بَاطِنُهُۥ فِيهِ ٱلرَّحْمَةُ وَظَـٰهِرُهُۥ مِن قِبَلِهِ ٱلْعَذَابُ ﴿١٣﴾}\\
\textamh{14.\  } & \mytextarabic{يُنَادُونَهُمْ أَلَمْ نَكُن مَّعَكُمْ ۖ قَالُوا۟ بَلَىٰ وَلَـٰكِنَّكُمْ فَتَنتُمْ أَنفُسَكُمْ وَتَرَبَّصْتُمْ وَٱرْتَبْتُمْ وَغَرَّتْكُمُ ٱلْأَمَانِىُّ حَتَّىٰ جَآءَ أَمْرُ ٱللَّهِ وَغَرَّكُم بِٱللَّهِ ٱلْغَرُورُ ﴿١٤﴾}\\
\textamh{15.\  } & \mytextarabic{فَٱلْيَوْمَ لَا يُؤْخَذُ مِنكُمْ فِدْيَةٌۭ وَلَا مِنَ ٱلَّذِينَ كَفَرُوا۟ ۚ مَأْوَىٰكُمُ ٱلنَّارُ ۖ هِىَ مَوْلَىٰكُمْ ۖ وَبِئْسَ ٱلْمَصِيرُ ﴿١٥﴾}\\
\textamh{16.\  } & \mytextarabic{۞ أَلَمْ يَأْنِ لِلَّذِينَ ءَامَنُوٓا۟ أَن تَخْشَعَ قُلُوبُهُمْ لِذِكْرِ ٱللَّهِ وَمَا نَزَلَ مِنَ ٱلْحَقِّ وَلَا يَكُونُوا۟ كَٱلَّذِينَ أُوتُوا۟ ٱلْكِتَـٰبَ مِن قَبْلُ فَطَالَ عَلَيْهِمُ ٱلْأَمَدُ فَقَسَتْ قُلُوبُهُمْ ۖ وَكَثِيرٌۭ مِّنْهُمْ فَـٰسِقُونَ ﴿١٦﴾}\\
\textamh{17.\  } & \mytextarabic{ٱعْلَمُوٓا۟ أَنَّ ٱللَّهَ يُحْىِ ٱلْأَرْضَ بَعْدَ مَوْتِهَا ۚ قَدْ بَيَّنَّا لَكُمُ ٱلْءَايَـٰتِ لَعَلَّكُمْ تَعْقِلُونَ ﴿١٧﴾}\\
\textamh{18.\  } & \mytextarabic{إِنَّ ٱلْمُصَّدِّقِينَ وَٱلْمُصَّدِّقَـٰتِ وَأَقْرَضُوا۟ ٱللَّهَ قَرْضًا حَسَنًۭا يُضَٰعَفُ لَهُمْ وَلَهُمْ أَجْرٌۭ كَرِيمٌۭ ﴿١٨﴾}\\
\textamh{19.\  } & \mytextarabic{وَٱلَّذِينَ ءَامَنُوا۟ بِٱللَّهِ وَرُسُلِهِۦٓ أُو۟لَـٰٓئِكَ هُمُ ٱلصِّدِّيقُونَ ۖ وَٱلشُّهَدَآءُ عِندَ رَبِّهِمْ لَهُمْ أَجْرُهُمْ وَنُورُهُمْ ۖ وَٱلَّذِينَ كَفَرُوا۟ وَكَذَّبُوا۟ بِـَٔايَـٰتِنَآ أُو۟لَـٰٓئِكَ أَصْحَـٰبُ ٱلْجَحِيمِ ﴿١٩﴾}\\
\textamh{20.\  } & \mytextarabic{ٱعْلَمُوٓا۟ أَنَّمَا ٱلْحَيَوٰةُ ٱلدُّنْيَا لَعِبٌۭ وَلَهْوٌۭ وَزِينَةٌۭ وَتَفَاخُرٌۢ بَيْنَكُمْ وَتَكَاثُرٌۭ فِى ٱلْأَمْوَٟلِ وَٱلْأَوْلَـٰدِ ۖ كَمَثَلِ غَيْثٍ أَعْجَبَ ٱلْكُفَّارَ نَبَاتُهُۥ ثُمَّ يَهِيجُ فَتَرَىٰهُ مُصْفَرًّۭا ثُمَّ يَكُونُ حُطَٰمًۭا ۖ وَفِى ٱلْءَاخِرَةِ عَذَابٌۭ شَدِيدٌۭ وَمَغْفِرَةٌۭ مِّنَ ٱللَّهِ وَرِضْوَٟنٌۭ ۚ وَمَا ٱلْحَيَوٰةُ ٱلدُّنْيَآ إِلَّا مَتَـٰعُ ٱلْغُرُورِ ﴿٢٠﴾}\\
\textamh{21.\  } & \mytextarabic{سَابِقُوٓا۟ إِلَىٰ مَغْفِرَةٍۢ مِّن رَّبِّكُمْ وَجَنَّةٍ عَرْضُهَا كَعَرْضِ ٱلسَّمَآءِ وَٱلْأَرْضِ أُعِدَّتْ لِلَّذِينَ ءَامَنُوا۟ بِٱللَّهِ وَرُسُلِهِۦ ۚ ذَٟلِكَ فَضْلُ ٱللَّهِ يُؤْتِيهِ مَن يَشَآءُ ۚ وَٱللَّهُ ذُو ٱلْفَضْلِ ٱلْعَظِيمِ ﴿٢١﴾}\\
\textamh{22.\  } & \mytextarabic{مَآ أَصَابَ مِن مُّصِيبَةٍۢ فِى ٱلْأَرْضِ وَلَا فِىٓ أَنفُسِكُمْ إِلَّا فِى كِتَـٰبٍۢ مِّن قَبْلِ أَن نَّبْرَأَهَآ ۚ إِنَّ ذَٟلِكَ عَلَى ٱللَّهِ يَسِيرٌۭ ﴿٢٢﴾}\\
\textamh{23.\  } & \mytextarabic{لِّكَيْلَا تَأْسَوْا۟ عَلَىٰ مَا فَاتَكُمْ وَلَا تَفْرَحُوا۟ بِمَآ ءَاتَىٰكُمْ ۗ وَٱللَّهُ لَا يُحِبُّ كُلَّ مُخْتَالٍۢ فَخُورٍ ﴿٢٣﴾}\\
\textamh{24.\  } & \mytextarabic{ٱلَّذِينَ يَبْخَلُونَ وَيَأْمُرُونَ ٱلنَّاسَ بِٱلْبُخْلِ ۗ وَمَن يَتَوَلَّ فَإِنَّ ٱللَّهَ هُوَ ٱلْغَنِىُّ ٱلْحَمِيدُ ﴿٢٤﴾}\\
\textamh{25.\  } & \mytextarabic{لَقَدْ أَرْسَلْنَا رُسُلَنَا بِٱلْبَيِّنَـٰتِ وَأَنزَلْنَا مَعَهُمُ ٱلْكِتَـٰبَ وَٱلْمِيزَانَ لِيَقُومَ ٱلنَّاسُ بِٱلْقِسْطِ ۖ وَأَنزَلْنَا ٱلْحَدِيدَ فِيهِ بَأْسٌۭ شَدِيدٌۭ وَمَنَـٰفِعُ لِلنَّاسِ وَلِيَعْلَمَ ٱللَّهُ مَن يَنصُرُهُۥ وَرُسُلَهُۥ بِٱلْغَيْبِ ۚ إِنَّ ٱللَّهَ قَوِىٌّ عَزِيزٌۭ ﴿٢٥﴾}\\
\textamh{26.\  } & \mytextarabic{وَلَقَدْ أَرْسَلْنَا نُوحًۭا وَإِبْرَٰهِيمَ وَجَعَلْنَا فِى ذُرِّيَّتِهِمَا ٱلنُّبُوَّةَ وَٱلْكِتَـٰبَ ۖ فَمِنْهُم مُّهْتَدٍۢ ۖ وَكَثِيرٌۭ مِّنْهُمْ فَـٰسِقُونَ ﴿٢٦﴾}\\
\textamh{27.\  } & \mytextarabic{ثُمَّ قَفَّيْنَا عَلَىٰٓ ءَاثَـٰرِهِم بِرُسُلِنَا وَقَفَّيْنَا بِعِيسَى ٱبْنِ مَرْيَمَ وَءَاتَيْنَـٰهُ ٱلْإِنجِيلَ وَجَعَلْنَا فِى قُلُوبِ ٱلَّذِينَ ٱتَّبَعُوهُ رَأْفَةًۭ وَرَحْمَةًۭ وَرَهْبَانِيَّةً ٱبْتَدَعُوهَا مَا كَتَبْنَـٰهَا عَلَيْهِمْ إِلَّا ٱبْتِغَآءَ رِضْوَٟنِ ٱللَّهِ فَمَا رَعَوْهَا حَقَّ رِعَايَتِهَا ۖ فَـَٔاتَيْنَا ٱلَّذِينَ ءَامَنُوا۟ مِنْهُمْ أَجْرَهُمْ ۖ وَكَثِيرٌۭ مِّنْهُمْ فَـٰسِقُونَ ﴿٢٧﴾}\\
\textamh{28.\  } & \mytextarabic{يَـٰٓأَيُّهَا ٱلَّذِينَ ءَامَنُوا۟ ٱتَّقُوا۟ ٱللَّهَ وَءَامِنُوا۟ بِرَسُولِهِۦ يُؤْتِكُمْ كِفْلَيْنِ مِن رَّحْمَتِهِۦ وَيَجْعَل لَّكُمْ نُورًۭا تَمْشُونَ بِهِۦ وَيَغْفِرْ لَكُمْ ۚ وَٱللَّهُ غَفُورٌۭ رَّحِيمٌۭ ﴿٢٨﴾}\\
\textamh{29.\  } & \mytextarabic{لِّئَلَّا يَعْلَمَ أَهْلُ ٱلْكِتَـٰبِ أَلَّا يَقْدِرُونَ عَلَىٰ شَىْءٍۢ مِّن فَضْلِ ٱللَّهِ ۙ وَأَنَّ ٱلْفَضْلَ بِيَدِ ٱللَّهِ يُؤْتِيهِ مَن يَشَآءُ ۚ وَٱللَّهُ ذُو ٱلْفَضْلِ ٱلْعَظِيمِ ﴿٢٩﴾}\\
\end{longtable}
\clearpage
%% License: BSD style (Berkley) (i.e. Put the Copyright owner's name always)
%% Writer and Copyright (to): Bewketu(Bilal) Tadilo (2016-17)
\centering\section{\LR{\textamharic{ሱራቱ አልሙጀዲላ -}  \RL{سوره  المجادلة}}}
\begin{longtable}{%
  @{}
    p{.5\textwidth}
  @{~~~~~~~~~~~~~}
    p{.5\textwidth}
    @{}
}
\nopagebreak
\textamh{\ \ \ \ \ \  ቢስሚላሂ አራህመኒ ራሂይም } &  بِسْمِ ٱللَّهِ ٱلرَّحْمَـٰنِ ٱلرَّحِيمِ\\
\textamh{1.\  } &  قَدْ سَمِعَ ٱللَّهُ قَوْلَ ٱلَّتِى تُجَٰدِلُكَ فِى زَوْجِهَا وَتَشْتَكِىٓ إِلَى ٱللَّهِ وَٱللَّهُ يَسْمَعُ تَحَاوُرَكُمَآ ۚ إِنَّ ٱللَّهَ سَمِيعٌۢ بَصِيرٌ ﴿١﴾\\
\textamh{2.\  } & ٱلَّذِينَ يُظَـٰهِرُونَ مِنكُم مِّن نِّسَآئِهِم مَّا هُنَّ أُمَّهَـٰتِهِمْ ۖ إِنْ أُمَّهَـٰتُهُمْ إِلَّا ٱلَّٰٓـِٔى وَلَدْنَهُمْ ۚ وَإِنَّهُمْ لَيَقُولُونَ مُنكَرًۭا مِّنَ ٱلْقَوْلِ وَزُورًۭا ۚ وَإِنَّ ٱللَّهَ لَعَفُوٌّ غَفُورٌۭ ﴿٢﴾\\
\textamh{3.\  } & وَٱلَّذِينَ يُظَـٰهِرُونَ مِن نِّسَآئِهِمْ ثُمَّ يَعُودُونَ لِمَا قَالُوا۟ فَتَحْرِيرُ رَقَبَةٍۢ مِّن قَبْلِ أَن يَتَمَآسَّا ۚ ذَٟلِكُمْ تُوعَظُونَ بِهِۦ ۚ وَٱللَّهُ بِمَا تَعْمَلُونَ خَبِيرٌۭ ﴿٣﴾\\
\textamh{4.\  } & فَمَن لَّمْ يَجِدْ فَصِيَامُ شَهْرَيْنِ مُتَتَابِعَيْنِ مِن قَبْلِ أَن يَتَمَآسَّا ۖ فَمَن لَّمْ يَسْتَطِعْ فَإِطْعَامُ سِتِّينَ مِسْكِينًۭا ۚ ذَٟلِكَ لِتُؤْمِنُوا۟ بِٱللَّهِ وَرَسُولِهِۦ ۚ وَتِلْكَ حُدُودُ ٱللَّهِ ۗ وَلِلْكَـٰفِرِينَ عَذَابٌ أَلِيمٌ ﴿٤﴾\\
\textamh{5.\  } & إِنَّ ٱلَّذِينَ يُحَآدُّونَ ٱللَّهَ وَرَسُولَهُۥ كُبِتُوا۟ كَمَا كُبِتَ ٱلَّذِينَ مِن قَبْلِهِمْ ۚ وَقَدْ أَنزَلْنَآ ءَايَـٰتٍۭ بَيِّنَـٰتٍۢ ۚ وَلِلْكَـٰفِرِينَ عَذَابٌۭ مُّهِينٌۭ ﴿٥﴾\\
\textamh{6.\  } & يَوْمَ يَبْعَثُهُمُ ٱللَّهُ جَمِيعًۭا فَيُنَبِّئُهُم بِمَا عَمِلُوٓا۟ ۚ أَحْصَىٰهُ ٱللَّهُ وَنَسُوهُ ۚ وَٱللَّهُ عَلَىٰ كُلِّ شَىْءٍۢ شَهِيدٌ ﴿٦﴾\\
\textamh{7.\  } & أَلَمْ تَرَ أَنَّ ٱللَّهَ يَعْلَمُ مَا فِى ٱلسَّمَـٰوَٟتِ وَمَا فِى ٱلْأَرْضِ ۖ مَا يَكُونُ مِن نَّجْوَىٰ ثَلَـٰثَةٍ إِلَّا هُوَ رَابِعُهُمْ وَلَا خَمْسَةٍ إِلَّا هُوَ سَادِسُهُمْ وَلَآ أَدْنَىٰ مِن ذَٟلِكَ وَلَآ أَكْثَرَ إِلَّا هُوَ مَعَهُمْ أَيْنَ مَا كَانُوا۟ ۖ ثُمَّ يُنَبِّئُهُم بِمَا عَمِلُوا۟ يَوْمَ ٱلْقِيَـٰمَةِ ۚ إِنَّ ٱللَّهَ بِكُلِّ شَىْءٍ عَلِيمٌ ﴿٧﴾\\
\textamh{8.\  } & أَلَمْ تَرَ إِلَى ٱلَّذِينَ نُهُوا۟ عَنِ ٱلنَّجْوَىٰ ثُمَّ يَعُودُونَ لِمَا نُهُوا۟ عَنْهُ وَيَتَنَـٰجَوْنَ بِٱلْإِثْمِ وَٱلْعُدْوَٟنِ وَمَعْصِيَتِ ٱلرَّسُولِ وَإِذَا جَآءُوكَ حَيَّوْكَ بِمَا لَمْ يُحَيِّكَ بِهِ ٱللَّهُ وَيَقُولُونَ فِىٓ أَنفُسِهِمْ لَوْلَا يُعَذِّبُنَا ٱللَّهُ بِمَا نَقُولُ ۚ حَسْبُهُمْ جَهَنَّمُ يَصْلَوْنَهَا ۖ فَبِئْسَ ٱلْمَصِيرُ ﴿٨﴾\\
\textamh{9.\  } & يَـٰٓأَيُّهَا ٱلَّذِينَ ءَامَنُوٓا۟ إِذَا تَنَـٰجَيْتُمْ فَلَا تَتَنَـٰجَوْا۟ بِٱلْإِثْمِ وَٱلْعُدْوَٟنِ وَمَعْصِيَتِ ٱلرَّسُولِ وَتَنَـٰجَوْا۟ بِٱلْبِرِّ وَٱلتَّقْوَىٰ ۖ وَٱتَّقُوا۟ ٱللَّهَ ٱلَّذِىٓ إِلَيْهِ تُحْشَرُونَ ﴿٩﴾\\
\textamh{10.\  } & إِنَّمَا ٱلنَّجْوَىٰ مِنَ ٱلشَّيْطَٰنِ لِيَحْزُنَ ٱلَّذِينَ ءَامَنُوا۟ وَلَيْسَ بِضَآرِّهِمْ شَيْـًٔا إِلَّا بِإِذْنِ ٱللَّهِ ۚ وَعَلَى ٱللَّهِ فَلْيَتَوَكَّلِ ٱلْمُؤْمِنُونَ ﴿١٠﴾\\
\textamh{11.\  } & يَـٰٓأَيُّهَا ٱلَّذِينَ ءَامَنُوٓا۟ إِذَا قِيلَ لَكُمْ تَفَسَّحُوا۟ فِى ٱلْمَجَٰلِسِ فَٱفْسَحُوا۟ يَفْسَحِ ٱللَّهُ لَكُمْ ۖ وَإِذَا قِيلَ ٱنشُزُوا۟ فَٱنشُزُوا۟ يَرْفَعِ ٱللَّهُ ٱلَّذِينَ ءَامَنُوا۟ مِنكُمْ وَٱلَّذِينَ أُوتُوا۟ ٱلْعِلْمَ دَرَجَٰتٍۢ ۚ وَٱللَّهُ بِمَا تَعْمَلُونَ خَبِيرٌۭ ﴿١١﴾\\
\textamh{12.\  } & يَـٰٓأَيُّهَا ٱلَّذِينَ ءَامَنُوٓا۟ إِذَا نَـٰجَيْتُمُ ٱلرَّسُولَ فَقَدِّمُوا۟ بَيْنَ يَدَىْ نَجْوَىٰكُمْ صَدَقَةًۭ ۚ ذَٟلِكَ خَيْرٌۭ لَّكُمْ وَأَطْهَرُ ۚ فَإِن لَّمْ تَجِدُوا۟ فَإِنَّ ٱللَّهَ غَفُورٌۭ رَّحِيمٌ ﴿١٢﴾\\
\textamh{13.\  } & ءَأَشْفَقْتُمْ أَن تُقَدِّمُوا۟ بَيْنَ يَدَىْ نَجْوَىٰكُمْ صَدَقَـٰتٍۢ ۚ فَإِذْ لَمْ تَفْعَلُوا۟ وَتَابَ ٱللَّهُ عَلَيْكُمْ فَأَقِيمُوا۟ ٱلصَّلَوٰةَ وَءَاتُوا۟ ٱلزَّكَوٰةَ وَأَطِيعُوا۟ ٱللَّهَ وَرَسُولَهُۥ ۚ وَٱللَّهُ خَبِيرٌۢ بِمَا تَعْمَلُونَ ﴿١٣﴾\\
\textamh{14.\  } & ۞ أَلَمْ تَرَ إِلَى ٱلَّذِينَ تَوَلَّوْا۟ قَوْمًا غَضِبَ ٱللَّهُ عَلَيْهِم مَّا هُم مِّنكُمْ وَلَا مِنْهُمْ وَيَحْلِفُونَ عَلَى ٱلْكَذِبِ وَهُمْ يَعْلَمُونَ ﴿١٤﴾\\
\textamh{15.\  } & أَعَدَّ ٱللَّهُ لَهُمْ عَذَابًۭا شَدِيدًا ۖ إِنَّهُمْ سَآءَ مَا كَانُوا۟ يَعْمَلُونَ ﴿١٥﴾\\
\textamh{16.\  } & ٱتَّخَذُوٓا۟ أَيْمَـٰنَهُمْ جُنَّةًۭ فَصَدُّوا۟ عَن سَبِيلِ ٱللَّهِ فَلَهُمْ عَذَابٌۭ مُّهِينٌۭ ﴿١٦﴾\\
\textamh{17.\  } & لَّن تُغْنِىَ عَنْهُمْ أَمْوَٟلُهُمْ وَلَآ أَوْلَـٰدُهُم مِّنَ ٱللَّهِ شَيْـًٔا ۚ أُو۟لَـٰٓئِكَ أَصْحَـٰبُ ٱلنَّارِ ۖ هُمْ فِيهَا خَـٰلِدُونَ ﴿١٧﴾\\
\textamh{18.\  } & يَوْمَ يَبْعَثُهُمُ ٱللَّهُ جَمِيعًۭا فَيَحْلِفُونَ لَهُۥ كَمَا يَحْلِفُونَ لَكُمْ ۖ وَيَحْسَبُونَ أَنَّهُمْ عَلَىٰ شَىْءٍ ۚ أَلَآ إِنَّهُمْ هُمُ ٱلْكَـٰذِبُونَ ﴿١٨﴾\\
\textamh{19.\  } & ٱسْتَحْوَذَ عَلَيْهِمُ ٱلشَّيْطَٰنُ فَأَنسَىٰهُمْ ذِكْرَ ٱللَّهِ ۚ أُو۟لَـٰٓئِكَ حِزْبُ ٱلشَّيْطَٰنِ ۚ أَلَآ إِنَّ حِزْبَ ٱلشَّيْطَٰنِ هُمُ ٱلْخَـٰسِرُونَ ﴿١٩﴾\\
\textamh{20.\  } & إِنَّ ٱلَّذِينَ يُحَآدُّونَ ٱللَّهَ وَرَسُولَهُۥٓ أُو۟لَـٰٓئِكَ فِى ٱلْأَذَلِّينَ ﴿٢٠﴾\\
\textamh{21.\  } & كَتَبَ ٱللَّهُ لَأَغْلِبَنَّ أَنَا۠ وَرُسُلِىٓ ۚ إِنَّ ٱللَّهَ قَوِىٌّ عَزِيزٌۭ ﴿٢١﴾\\
\textamh{22.\  } & لَّا تَجِدُ قَوْمًۭا يُؤْمِنُونَ بِٱللَّهِ وَٱلْيَوْمِ ٱلْءَاخِرِ يُوَآدُّونَ مَنْ حَآدَّ ٱللَّهَ وَرَسُولَهُۥ وَلَوْ كَانُوٓا۟ ءَابَآءَهُمْ أَوْ أَبْنَآءَهُمْ أَوْ إِخْوَٟنَهُمْ أَوْ عَشِيرَتَهُمْ ۚ أُو۟لَـٰٓئِكَ كَتَبَ فِى قُلُوبِهِمُ ٱلْإِيمَـٰنَ وَأَيَّدَهُم بِرُوحٍۢ مِّنْهُ ۖ وَيُدْخِلُهُمْ جَنَّـٰتٍۢ تَجْرِى مِن تَحْتِهَا ٱلْأَنْهَـٰرُ خَـٰلِدِينَ فِيهَا ۚ رَضِىَ ٱللَّهُ عَنْهُمْ وَرَضُوا۟ عَنْهُ ۚ أُو۟لَـٰٓئِكَ حِزْبُ ٱللَّهِ ۚ أَلَآ إِنَّ حِزْبَ ٱللَّهِ هُمُ ٱلْمُفْلِحُونَ ﴿٢٢﴾\\
\end{longtable} \newpage

%% License: BSD style (Berkley) (i.e. Put the Copyright owner's name always)
%% Writer and Copyright (to): Bewketu(Bilal) Tadilo (2016-17)
\centering\section{\LR{\textamharic{ሱራቱ አልሀሽር -}  \RL{سوره  الحشر}}}
\begin{longtable}{%
  @{}
    p{.5\textwidth}
  @{~~~~~~~~~~~~}
    p{.5\textwidth}
    @{}
}
\nopagebreak
\textamh{ቢስሚላሂ አራህመኒ ራሂይም } &  بِسْمِ ٱللَّهِ ٱلرَّحْمَـٰنِ ٱلرَّحِيمِ\\
\textamh{1.\  } &  سَبَّحَ لِلَّهِ مَا فِى ٱلسَّمَـٰوَٟتِ وَمَا فِى ٱلْأَرْضِ ۖ وَهُوَ ٱلْعَزِيزُ ٱلْحَكِيمُ ﴿١﴾\\
\textamh{2.\  } & هُوَ ٱلَّذِىٓ أَخْرَجَ ٱلَّذِينَ كَفَرُوا۟ مِنْ أَهْلِ ٱلْكِتَـٰبِ مِن دِيَـٰرِهِمْ لِأَوَّلِ ٱلْحَشْرِ ۚ مَا ظَنَنتُمْ أَن يَخْرُجُوا۟ ۖ وَظَنُّوٓا۟ أَنَّهُم مَّانِعَتُهُمْ حُصُونُهُم مِّنَ ٱللَّهِ فَأَتَىٰهُمُ ٱللَّهُ مِنْ حَيْثُ لَمْ يَحْتَسِبُوا۟ ۖ وَقَذَفَ فِى قُلُوبِهِمُ ٱلرُّعْبَ ۚ يُخْرِبُونَ بُيُوتَهُم بِأَيْدِيهِمْ وَأَيْدِى ٱلْمُؤْمِنِينَ فَٱعْتَبِرُوا۟ يَـٰٓأُو۟لِى ٱلْأَبْصَـٰرِ ﴿٢﴾\\
\textamh{3.\  } & وَلَوْلَآ أَن كَتَبَ ٱللَّهُ عَلَيْهِمُ ٱلْجَلَآءَ لَعَذَّبَهُمْ فِى ٱلدُّنْيَا ۖ وَلَهُمْ فِى ٱلْءَاخِرَةِ عَذَابُ ٱلنَّارِ ﴿٣﴾\\
\textamh{4.\  } & ذَٟلِكَ بِأَنَّهُمْ شَآقُّوا۟ ٱللَّهَ وَرَسُولَهُۥ ۖ وَمَن يُشَآقِّ ٱللَّهَ فَإِنَّ ٱللَّهَ شَدِيدُ ٱلْعِقَابِ ﴿٤﴾\\
\textamh{5.\  } & مَا قَطَعْتُم مِّن لِّينَةٍ أَوْ تَرَكْتُمُوهَا قَآئِمَةً عَلَىٰٓ أُصُولِهَا فَبِإِذْنِ ٱللَّهِ وَلِيُخْزِىَ ٱلْفَـٰسِقِينَ ﴿٥﴾\\
\textamh{6.\  } & وَمَآ أَفَآءَ ٱللَّهُ عَلَىٰ رَسُولِهِۦ مِنْهُمْ فَمَآ أَوْجَفْتُمْ عَلَيْهِ مِنْ خَيْلٍۢ وَلَا رِكَابٍۢ وَلَـٰكِنَّ ٱللَّهَ يُسَلِّطُ رُسُلَهُۥ عَلَىٰ مَن يَشَآءُ ۚ وَٱللَّهُ عَلَىٰ كُلِّ شَىْءٍۢ قَدِيرٌۭ ﴿٦﴾\\
\textamh{7.\  } & مَّآ أَفَآءَ ٱللَّهُ عَلَىٰ رَسُولِهِۦ مِنْ أَهْلِ ٱلْقُرَىٰ فَلِلَّهِ وَلِلرَّسُولِ وَلِذِى ٱلْقُرْبَىٰ وَٱلْيَتَـٰمَىٰ وَٱلْمَسَـٰكِينِ وَٱبْنِ ٱلسَّبِيلِ كَىْ لَا يَكُونَ دُولَةًۢ بَيْنَ ٱلْأَغْنِيَآءِ مِنكُمْ ۚ وَمَآ ءَاتَىٰكُمُ ٱلرَّسُولُ فَخُذُوهُ وَمَا نَهَىٰكُمْ عَنْهُ فَٱنتَهُوا۟ ۚ وَٱتَّقُوا۟ ٱللَّهَ ۖ إِنَّ ٱللَّهَ شَدِيدُ ٱلْعِقَابِ ﴿٧﴾\\
\textamh{8.\  } & لِلْفُقَرَآءِ ٱلْمُهَـٰجِرِينَ ٱلَّذِينَ أُخْرِجُوا۟ مِن دِيَـٰرِهِمْ وَأَمْوَٟلِهِمْ يَبْتَغُونَ فَضْلًۭا مِّنَ ٱللَّهِ وَرِضْوَٟنًۭا وَيَنصُرُونَ ٱللَّهَ وَرَسُولَهُۥٓ ۚ أُو۟لَـٰٓئِكَ هُمُ ٱلصَّـٰدِقُونَ ﴿٨﴾\\
\textamh{9.\  } & وَٱلَّذِينَ تَبَوَّءُو ٱلدَّارَ وَٱلْإِيمَـٰنَ مِن قَبْلِهِمْ يُحِبُّونَ مَنْ هَاجَرَ إِلَيْهِمْ وَلَا يَجِدُونَ فِى صُدُورِهِمْ حَاجَةًۭ مِّمَّآ أُوتُوا۟ وَيُؤْثِرُونَ عَلَىٰٓ أَنفُسِهِمْ وَلَوْ كَانَ بِهِمْ خَصَاصَةٌۭ ۚ وَمَن يُوقَ شُحَّ نَفْسِهِۦ فَأُو۟لَـٰٓئِكَ هُمُ ٱلْمُفْلِحُونَ ﴿٩﴾\\
\textamh{10.\  } & وَٱلَّذِينَ جَآءُو مِنۢ بَعْدِهِمْ يَقُولُونَ رَبَّنَا ٱغْفِرْ لَنَا وَلِإِخْوَٟنِنَا ٱلَّذِينَ سَبَقُونَا بِٱلْإِيمَـٰنِ وَلَا تَجْعَلْ فِى قُلُوبِنَا غِلًّۭا لِّلَّذِينَ ءَامَنُوا۟ رَبَّنَآ إِنَّكَ رَءُوفٌۭ رَّحِيمٌ ﴿١٠﴾\\
\textamh{11.\  } & ۞ أَلَمْ تَرَ إِلَى ٱلَّذِينَ نَافَقُوا۟ يَقُولُونَ لِإِخْوَٟنِهِمُ ٱلَّذِينَ كَفَرُوا۟ مِنْ أَهْلِ ٱلْكِتَـٰبِ لَئِنْ أُخْرِجْتُمْ لَنَخْرُجَنَّ مَعَكُمْ وَلَا نُطِيعُ فِيكُمْ أَحَدًا أَبَدًۭا وَإِن قُوتِلْتُمْ لَنَنصُرَنَّكُمْ وَٱللَّهُ يَشْهَدُ إِنَّهُمْ لَكَـٰذِبُونَ ﴿١١﴾\\
\textamh{12.\  } & لَئِنْ أُخْرِجُوا۟ لَا يَخْرُجُونَ مَعَهُمْ وَلَئِن قُوتِلُوا۟ لَا يَنصُرُونَهُمْ وَلَئِن نَّصَرُوهُمْ لَيُوَلُّنَّ ٱلْأَدْبَٰرَ ثُمَّ لَا يُنصَرُونَ ﴿١٢﴾\\
\textamh{13.\  } & لَأَنتُمْ أَشَدُّ رَهْبَةًۭ فِى صُدُورِهِم مِّنَ ٱللَّهِ ۚ ذَٟلِكَ بِأَنَّهُمْ قَوْمٌۭ لَّا يَفْقَهُونَ ﴿١٣﴾\\
\textamh{14.\  } & لَا يُقَـٰتِلُونَكُمْ جَمِيعًا إِلَّا فِى قُرًۭى مُّحَصَّنَةٍ أَوْ مِن وَرَآءِ جُدُرٍۭ ۚ بَأْسُهُم بَيْنَهُمْ شَدِيدٌۭ ۚ تَحْسَبُهُمْ جَمِيعًۭا وَقُلُوبُهُمْ شَتَّىٰ ۚ ذَٟلِكَ بِأَنَّهُمْ قَوْمٌۭ لَّا يَعْقِلُونَ ﴿١٤﴾\\
\textamh{15.\  } & كَمَثَلِ ٱلَّذِينَ مِن قَبْلِهِمْ قَرِيبًۭا ۖ ذَاقُوا۟ وَبَالَ أَمْرِهِمْ وَلَهُمْ عَذَابٌ أَلِيمٌۭ ﴿١٥﴾\\
\textamh{16.\  } & كَمَثَلِ ٱلشَّيْطَٰنِ إِذْ قَالَ لِلْإِنسَـٰنِ ٱكْفُرْ فَلَمَّا كَفَرَ قَالَ إِنِّى بَرِىٓءٌۭ مِّنكَ إِنِّىٓ أَخَافُ ٱللَّهَ رَبَّ ٱلْعَـٰلَمِينَ ﴿١٦﴾\\
\textamh{17.\  } & فَكَانَ عَـٰقِبَتَهُمَآ أَنَّهُمَا فِى ٱلنَّارِ خَـٰلِدَيْنِ فِيهَا ۚ وَذَٟلِكَ جَزَٰٓؤُا۟ ٱلظَّـٰلِمِينَ ﴿١٧﴾\\
\textamh{18.\  } & يَـٰٓأَيُّهَا ٱلَّذِينَ ءَامَنُوا۟ ٱتَّقُوا۟ ٱللَّهَ وَلْتَنظُرْ نَفْسٌۭ مَّا قَدَّمَتْ لِغَدٍۢ ۖ وَٱتَّقُوا۟ ٱللَّهَ ۚ إِنَّ ٱللَّهَ خَبِيرٌۢ بِمَا تَعْمَلُونَ ﴿١٨﴾\\
\textamh{19.\  } & وَلَا تَكُونُوا۟ كَٱلَّذِينَ نَسُوا۟ ٱللَّهَ فَأَنسَىٰهُمْ أَنفُسَهُمْ ۚ أُو۟لَـٰٓئِكَ هُمُ ٱلْفَـٰسِقُونَ ﴿١٩﴾\\
\textamh{20.\  } & لَا يَسْتَوِىٓ أَصْحَـٰبُ ٱلنَّارِ وَأَصْحَـٰبُ ٱلْجَنَّةِ ۚ أَصْحَـٰبُ ٱلْجَنَّةِ هُمُ ٱلْفَآئِزُونَ ﴿٢٠﴾\\
\textamh{21.\  } & لَوْ أَنزَلْنَا هَـٰذَا ٱلْقُرْءَانَ عَلَىٰ جَبَلٍۢ لَّرَأَيْتَهُۥ خَـٰشِعًۭا مُّتَصَدِّعًۭا مِّنْ خَشْيَةِ ٱللَّهِ ۚ وَتِلْكَ ٱلْأَمْثَـٰلُ نَضْرِبُهَا لِلنَّاسِ لَعَلَّهُمْ يَتَفَكَّرُونَ ﴿٢١﴾\\
\textamh{22.\  } & هُوَ ٱللَّهُ ٱلَّذِى لَآ إِلَـٰهَ إِلَّا هُوَ ۖ عَـٰلِمُ ٱلْغَيْبِ وَٱلشَّهَـٰدَةِ ۖ هُوَ ٱلرَّحْمَـٰنُ ٱلرَّحِيمُ ﴿٢٢﴾\\
\textamh{23.\  } & هُوَ ٱللَّهُ ٱلَّذِى لَآ إِلَـٰهَ إِلَّا هُوَ ٱلْمَلِكُ ٱلْقُدُّوسُ ٱلسَّلَـٰمُ ٱلْمُؤْمِنُ ٱلْمُهَيْمِنُ ٱلْعَزِيزُ ٱلْجَبَّارُ ٱلْمُتَكَبِّرُ ۚ سُبْحَـٰنَ ٱللَّهِ عَمَّا يُشْرِكُونَ ﴿٢٣﴾\\
\textamh{24.\  } & هُوَ ٱللَّهُ ٱلْخَـٰلِقُ ٱلْبَارِئُ ٱلْمُصَوِّرُ ۖ لَهُ ٱلْأَسْمَآءُ ٱلْحُسْنَىٰ ۚ يُسَبِّحُ لَهُۥ مَا فِى ٱلسَّمَـٰوَٟتِ وَٱلْأَرْضِ ۖ وَهُوَ ٱلْعَزِيزُ ٱلْحَكِيمُ ﴿٢٤﴾\\
\end{longtable}
\clearpage
%% License: BSD style (Berkley) (i.e. Put the Copyright owner's name always)
%% Writer and Copyright (to): Bewketu(Bilal) Tadilo (2016-17)
\centering\section{\LR{\textamharic{ሱራቱ አልሙምታሂና -}  \RL{سوره  الممتحنة}}}
\begin{longtable}{%
  @{}
    p{.5\textwidth}
  @{~~~~~~~~~~~~~}
    p{.5\textwidth}
    @{}
}
\nopagebreak
\textamh{ቢስሚላሂ አራህመኒ ራሂይም } &  بِسْمِ ٱللَّهِ ٱلرَّحْمَـٰنِ ٱلرَّحِيمِ\\
\textamh{1.\  } &  يَـٰٓأَيُّهَا ٱلَّذِينَ ءَامَنُوا۟ لَا تَتَّخِذُوا۟ عَدُوِّى وَعَدُوَّكُمْ أَوْلِيَآءَ تُلْقُونَ إِلَيْهِم بِٱلْمَوَدَّةِ وَقَدْ كَفَرُوا۟ بِمَا جَآءَكُم مِّنَ ٱلْحَقِّ يُخْرِجُونَ ٱلرَّسُولَ وَإِيَّاكُمْ ۙ أَن تُؤْمِنُوا۟ بِٱللَّهِ رَبِّكُمْ إِن كُنتُمْ خَرَجْتُمْ جِهَـٰدًۭا فِى سَبِيلِى وَٱبْتِغَآءَ مَرْضَاتِى ۚ تُسِرُّونَ إِلَيْهِم بِٱلْمَوَدَّةِ وَأَنَا۠ أَعْلَمُ بِمَآ أَخْفَيْتُمْ وَمَآ أَعْلَنتُمْ ۚ وَمَن يَفْعَلْهُ مِنكُمْ فَقَدْ ضَلَّ سَوَآءَ ٱلسَّبِيلِ ﴿١﴾\\
\textamh{2.\  } & إِن يَثْقَفُوكُمْ يَكُونُوا۟ لَكُمْ أَعْدَآءًۭ وَيَبْسُطُوٓا۟ إِلَيْكُمْ أَيْدِيَهُمْ وَأَلْسِنَتَهُم بِٱلسُّوٓءِ وَوَدُّوا۟ لَوْ تَكْفُرُونَ ﴿٢﴾\\
\textamh{3.\  } & لَن تَنفَعَكُمْ أَرْحَامُكُمْ وَلَآ أَوْلَـٰدُكُمْ ۚ يَوْمَ ٱلْقِيَـٰمَةِ يَفْصِلُ بَيْنَكُمْ ۚ وَٱللَّهُ بِمَا تَعْمَلُونَ بَصِيرٌۭ ﴿٣﴾\\
\textamh{4.\  } & قَدْ كَانَتْ لَكُمْ أُسْوَةٌ حَسَنَةٌۭ فِىٓ إِبْرَٰهِيمَ وَٱلَّذِينَ مَعَهُۥٓ إِذْ قَالُوا۟ لِقَوْمِهِمْ إِنَّا بُرَءَٰٓؤُا۟ مِنكُمْ وَمِمَّا تَعْبُدُونَ مِن دُونِ ٱللَّهِ كَفَرْنَا بِكُمْ وَبَدَا بَيْنَنَا وَبَيْنَكُمُ ٱلْعَدَٟوَةُ وَٱلْبَغْضَآءُ أَبَدًا حَتَّىٰ تُؤْمِنُوا۟ بِٱللَّهِ وَحْدَهُۥٓ إِلَّا قَوْلَ إِبْرَٰهِيمَ لِأَبِيهِ لَأَسْتَغْفِرَنَّ لَكَ وَمَآ أَمْلِكُ لَكَ مِنَ ٱللَّهِ مِن شَىْءٍۢ ۖ رَّبَّنَا عَلَيْكَ تَوَكَّلْنَا وَإِلَيْكَ أَنَبْنَا وَإِلَيْكَ ٱلْمَصِيرُ ﴿٤﴾\\
\textamh{5.\  } & رَبَّنَا لَا تَجْعَلْنَا فِتْنَةًۭ لِّلَّذِينَ كَفَرُوا۟ وَٱغْفِرْ لَنَا رَبَّنَآ ۖ إِنَّكَ أَنتَ ٱلْعَزِيزُ ٱلْحَكِيمُ ﴿٥﴾\\
\textamh{6.\  } & لَقَدْ كَانَ لَكُمْ فِيهِمْ أُسْوَةٌ حَسَنَةٌۭ لِّمَن كَانَ يَرْجُوا۟ ٱللَّهَ وَٱلْيَوْمَ ٱلْءَاخِرَ ۚ وَمَن يَتَوَلَّ فَإِنَّ ٱللَّهَ هُوَ ٱلْغَنِىُّ ٱلْحَمِيدُ ﴿٦﴾\\
\textamh{7.\  } & ۞ عَسَى ٱللَّهُ أَن يَجْعَلَ بَيْنَكُمْ وَبَيْنَ ٱلَّذِينَ عَادَيْتُم مِّنْهُم مَّوَدَّةًۭ ۚ وَٱللَّهُ قَدِيرٌۭ ۚ وَٱللَّهُ غَفُورٌۭ رَّحِيمٌۭ ﴿٧﴾\\
\textamh{8.\  } & لَّا يَنْهَىٰكُمُ ٱللَّهُ عَنِ ٱلَّذِينَ لَمْ يُقَـٰتِلُوكُمْ فِى ٱلدِّينِ وَلَمْ يُخْرِجُوكُم مِّن دِيَـٰرِكُمْ أَن تَبَرُّوهُمْ وَتُقْسِطُوٓا۟ إِلَيْهِمْ ۚ إِنَّ ٱللَّهَ يُحِبُّ ٱلْمُقْسِطِينَ ﴿٨﴾\\
\textamh{9.\  } & إِنَّمَا يَنْهَىٰكُمُ ٱللَّهُ عَنِ ٱلَّذِينَ قَـٰتَلُوكُمْ فِى ٱلدِّينِ وَأَخْرَجُوكُم مِّن دِيَـٰرِكُمْ وَظَـٰهَرُوا۟ عَلَىٰٓ إِخْرَاجِكُمْ أَن تَوَلَّوْهُمْ ۚ وَمَن يَتَوَلَّهُمْ فَأُو۟لَـٰٓئِكَ هُمُ ٱلظَّـٰلِمُونَ ﴿٩﴾\\
\textamh{10.\  } & يَـٰٓأَيُّهَا ٱلَّذِينَ ءَامَنُوٓا۟ إِذَا جَآءَكُمُ ٱلْمُؤْمِنَـٰتُ مُهَـٰجِرَٰتٍۢ فَٱمْتَحِنُوهُنَّ ۖ ٱللَّهُ أَعْلَمُ بِإِيمَـٰنِهِنَّ ۖ فَإِنْ عَلِمْتُمُوهُنَّ مُؤْمِنَـٰتٍۢ فَلَا تَرْجِعُوهُنَّ إِلَى ٱلْكُفَّارِ ۖ لَا هُنَّ حِلٌّۭ لَّهُمْ وَلَا هُمْ يَحِلُّونَ لَهُنَّ ۖ وَءَاتُوهُم مَّآ أَنفَقُوا۟ ۚ وَلَا جُنَاحَ عَلَيْكُمْ أَن تَنكِحُوهُنَّ إِذَآ ءَاتَيْتُمُوهُنَّ أُجُورَهُنَّ ۚ وَلَا تُمْسِكُوا۟ بِعِصَمِ ٱلْكَوَافِرِ وَسْـَٔلُوا۟ مَآ أَنفَقْتُمْ وَلْيَسْـَٔلُوا۟ مَآ أَنفَقُوا۟ ۚ ذَٟلِكُمْ حُكْمُ ٱللَّهِ ۖ يَحْكُمُ بَيْنَكُمْ ۚ وَٱللَّهُ عَلِيمٌ حَكِيمٌۭ ﴿١٠﴾\\
\textamh{11.\  } & وَإِن فَاتَكُمْ شَىْءٌۭ مِّنْ أَزْوَٟجِكُمْ إِلَى ٱلْكُفَّارِ فَعَاقَبْتُمْ فَـَٔاتُوا۟ ٱلَّذِينَ ذَهَبَتْ أَزْوَٟجُهُم مِّثْلَ مَآ أَنفَقُوا۟ ۚ وَٱتَّقُوا۟ ٱللَّهَ ٱلَّذِىٓ أَنتُم بِهِۦ مُؤْمِنُونَ ﴿١١﴾\\
\textamh{12.\  } & يَـٰٓأَيُّهَا ٱلنَّبِىُّ إِذَا جَآءَكَ ٱلْمُؤْمِنَـٰتُ يُبَايِعْنَكَ عَلَىٰٓ أَن لَّا يُشْرِكْنَ بِٱللَّهِ شَيْـًۭٔا وَلَا يَسْرِقْنَ وَلَا يَزْنِينَ وَلَا يَقْتُلْنَ أَوْلَـٰدَهُنَّ وَلَا يَأْتِينَ بِبُهْتَـٰنٍۢ يَفْتَرِينَهُۥ بَيْنَ أَيْدِيهِنَّ وَأَرْجُلِهِنَّ وَلَا يَعْصِينَكَ فِى مَعْرُوفٍۢ ۙ فَبَايِعْهُنَّ وَٱسْتَغْفِرْ لَهُنَّ ٱللَّهَ ۖ إِنَّ ٱللَّهَ غَفُورٌۭ رَّحِيمٌۭ ﴿١٢﴾\\
\textamh{13.\  } & يَـٰٓأَيُّهَا ٱلَّذِينَ ءَامَنُوا۟ لَا تَتَوَلَّوْا۟ قَوْمًا غَضِبَ ٱللَّهُ عَلَيْهِمْ قَدْ يَئِسُوا۟ مِنَ ٱلْءَاخِرَةِ كَمَا يَئِسَ ٱلْكُفَّارُ مِنْ أَصْحَـٰبِ ٱلْقُبُورِ ﴿١٣﴾\\
\end{longtable}
\clearpage
%% License: BSD style (Berkley) (i.e. Put the Copyright owner's name always)
%% Writer and Copyright (to): Bewketu(Bilal) Tadilo (2016-17)
\centering\section{\LR{\textamharic{ሱራቱ አስሳፍ -}  \RL{سوره  الصف}}}
\begin{longtable}{%
  @{}
    p{.5\textwidth}
  @{~~~~~~~~~~~~~}
    p{.5\textwidth}
    @{}
}
\nopagebreak
\textamh{ቢስሚላሂ አራህመኒ ራሂይም } &  بِسْمِ ٱللَّهِ ٱلرَّحْمَـٰنِ ٱلرَّحِيمِ\\
\textamh{1.\  } &  سَبَّحَ لِلَّهِ مَا فِى ٱلسَّمَـٰوَٟتِ وَمَا فِى ٱلْأَرْضِ ۖ وَهُوَ ٱلْعَزِيزُ ٱلْحَكِيمُ ﴿١﴾\\
\textamh{2.\  } & يَـٰٓأَيُّهَا ٱلَّذِينَ ءَامَنُوا۟ لِمَ تَقُولُونَ مَا لَا تَفْعَلُونَ ﴿٢﴾\\
\textamh{3.\  } & كَبُرَ مَقْتًا عِندَ ٱللَّهِ أَن تَقُولُوا۟ مَا لَا تَفْعَلُونَ ﴿٣﴾\\
\textamh{4.\  } & إِنَّ ٱللَّهَ يُحِبُّ ٱلَّذِينَ يُقَـٰتِلُونَ فِى سَبِيلِهِۦ صَفًّۭا كَأَنَّهُم بُنْيَـٰنٌۭ مَّرْصُوصٌۭ ﴿٤﴾\\
\textamh{5.\  } & وَإِذْ قَالَ مُوسَىٰ لِقَوْمِهِۦ يَـٰقَوْمِ لِمَ تُؤْذُونَنِى وَقَد تَّعْلَمُونَ أَنِّى رَسُولُ ٱللَّهِ إِلَيْكُمْ ۖ فَلَمَّا زَاغُوٓا۟ أَزَاغَ ٱللَّهُ قُلُوبَهُمْ ۚ وَٱللَّهُ لَا يَهْدِى ٱلْقَوْمَ ٱلْفَـٰسِقِينَ ﴿٥﴾\\
\textamh{6.\  } & وَإِذْ قَالَ عِيسَى ٱبْنُ مَرْيَمَ يَـٰبَنِىٓ إِسْرَٰٓءِيلَ إِنِّى رَسُولُ ٱللَّهِ إِلَيْكُم مُّصَدِّقًۭا لِّمَا بَيْنَ يَدَىَّ مِنَ ٱلتَّوْرَىٰةِ وَمُبَشِّرًۢا بِرَسُولٍۢ يَأْتِى مِنۢ بَعْدِى ٱسْمُهُۥٓ أَحْمَدُ ۖ فَلَمَّا جَآءَهُم بِٱلْبَيِّنَـٰتِ قَالُوا۟ هَـٰذَا سِحْرٌۭ مُّبِينٌۭ ﴿٦﴾\\
\textamh{7.\  } & وَمَنْ أَظْلَمُ مِمَّنِ ٱفْتَرَىٰ عَلَى ٱللَّهِ ٱلْكَذِبَ وَهُوَ يُدْعَىٰٓ إِلَى ٱلْإِسْلَـٰمِ ۚ وَٱللَّهُ لَا يَهْدِى ٱلْقَوْمَ ٱلظَّـٰلِمِينَ ﴿٧﴾\\
\textamh{8.\  } & يُرِيدُونَ لِيُطْفِـُٔوا۟ نُورَ ٱللَّهِ بِأَفْوَٟهِهِمْ وَٱللَّهُ مُتِمُّ نُورِهِۦ وَلَوْ كَرِهَ ٱلْكَـٰفِرُونَ ﴿٨﴾\\
\textamh{9.\  } & هُوَ ٱلَّذِىٓ أَرْسَلَ رَسُولَهُۥ بِٱلْهُدَىٰ وَدِينِ ٱلْحَقِّ لِيُظْهِرَهُۥ عَلَى ٱلدِّينِ كُلِّهِۦ وَلَوْ كَرِهَ ٱلْمُشْرِكُونَ ﴿٩﴾\\
\textamh{10.\  } & يَـٰٓأَيُّهَا ٱلَّذِينَ ءَامَنُوا۟ هَلْ أَدُلُّكُمْ عَلَىٰ تِجَٰرَةٍۢ تُنجِيكُم مِّنْ عَذَابٍ أَلِيمٍۢ ﴿١٠﴾\\
\textamh{11.\  } & تُؤْمِنُونَ بِٱللَّهِ وَرَسُولِهِۦ وَتُجَٰهِدُونَ فِى سَبِيلِ ٱللَّهِ بِأَمْوَٟلِكُمْ وَأَنفُسِكُمْ ۚ ذَٟلِكُمْ خَيْرٌۭ لَّكُمْ إِن كُنتُمْ تَعْلَمُونَ ﴿١١﴾\\
\textamh{12.\  } & يَغْفِرْ لَكُمْ ذُنُوبَكُمْ وَيُدْخِلْكُمْ جَنَّـٰتٍۢ تَجْرِى مِن تَحْتِهَا ٱلْأَنْهَـٰرُ وَمَسَـٰكِنَ طَيِّبَةًۭ فِى جَنَّـٰتِ عَدْنٍۢ ۚ ذَٟلِكَ ٱلْفَوْزُ ٱلْعَظِيمُ ﴿١٢﴾\\
\textamh{13.\  } & وَأُخْرَىٰ تُحِبُّونَهَا ۖ نَصْرٌۭ مِّنَ ٱللَّهِ وَفَتْحٌۭ قَرِيبٌۭ ۗ وَبَشِّرِ ٱلْمُؤْمِنِينَ ﴿١٣﴾\\
\textamh{14.\  } & يَـٰٓأَيُّهَا ٱلَّذِينَ ءَامَنُوا۟ كُونُوٓا۟ أَنصَارَ ٱللَّهِ كَمَا قَالَ عِيسَى ٱبْنُ مَرْيَمَ لِلْحَوَارِيِّۦنَ مَنْ أَنصَارِىٓ إِلَى ٱللَّهِ ۖ قَالَ ٱلْحَوَارِيُّونَ نَحْنُ أَنصَارُ ٱللَّهِ ۖ فَـَٔامَنَت طَّآئِفَةٌۭ مِّنۢ بَنِىٓ إِسْرَٰٓءِيلَ وَكَفَرَت طَّآئِفَةٌۭ ۖ فَأَيَّدْنَا ٱلَّذِينَ ءَامَنُوا۟ عَلَىٰ عَدُوِّهِمْ فَأَصْبَحُوا۟ ظَـٰهِرِينَ ﴿١٤﴾\\
\end{longtable}
\clearpage
%% License: BSD style (Berkley) (i.e. Put the Copyright owner's name always)
%% Writer and Copyright (to): Bewketu(Bilal) Tadilo (2016-17)
\centering\section{\LR{\textamharic{ሱራቱ አልጁሙኣት -}  \RL{سوره  الجمعة}}}
\begin{longtable}{%
  @{}
    p{.5\textwidth}
  @{~~~~~~~~~~~~}
    p{.5\textwidth}
    @{}
}
\nopagebreak
\textamh{ቢስሚላሂ አራህመኒ ራሂይም } &  بِسْمِ ٱللَّهِ ٱلرَّحْمَـٰنِ ٱلرَّحِيمِ\\
\textamh{1.\  } &  يُسَبِّحُ لِلَّهِ مَا فِى ٱلسَّمَـٰوَٟتِ وَمَا فِى ٱلْأَرْضِ ٱلْمَلِكِ ٱلْقُدُّوسِ ٱلْعَزِيزِ ٱلْحَكِيمِ ﴿١﴾\\
\textamh{2.\  } & هُوَ ٱلَّذِى بَعَثَ فِى ٱلْأُمِّيِّۦنَ رَسُولًۭا مِّنْهُمْ يَتْلُوا۟ عَلَيْهِمْ ءَايَـٰتِهِۦ وَيُزَكِّيهِمْ وَيُعَلِّمُهُمُ ٱلْكِتَـٰبَ وَٱلْحِكْمَةَ وَإِن كَانُوا۟ مِن قَبْلُ لَفِى ضَلَـٰلٍۢ مُّبِينٍۢ ﴿٢﴾\\
\textamh{3.\  } & وَءَاخَرِينَ مِنْهُمْ لَمَّا يَلْحَقُوا۟ بِهِمْ ۚ وَهُوَ ٱلْعَزِيزُ ٱلْحَكِيمُ ﴿٣﴾\\
\textamh{4.\  } & ذَٟلِكَ فَضْلُ ٱللَّهِ يُؤْتِيهِ مَن يَشَآءُ ۚ وَٱللَّهُ ذُو ٱلْفَضْلِ ٱلْعَظِيمِ ﴿٤﴾\\
\textamh{5.\  } & مَثَلُ ٱلَّذِينَ حُمِّلُوا۟ ٱلتَّوْرَىٰةَ ثُمَّ لَمْ يَحْمِلُوهَا كَمَثَلِ ٱلْحِمَارِ يَحْمِلُ أَسْفَارًۢا ۚ بِئْسَ مَثَلُ ٱلْقَوْمِ ٱلَّذِينَ كَذَّبُوا۟ بِـَٔايَـٰتِ ٱللَّهِ ۚ وَٱللَّهُ لَا يَهْدِى ٱلْقَوْمَ ٱلظَّـٰلِمِينَ ﴿٥﴾\\
\textamh{6.\  } & قُلْ يَـٰٓأَيُّهَا ٱلَّذِينَ هَادُوٓا۟ إِن زَعَمْتُمْ أَنَّكُمْ أَوْلِيَآءُ لِلَّهِ مِن دُونِ ٱلنَّاسِ فَتَمَنَّوُا۟ ٱلْمَوْتَ إِن كُنتُمْ صَـٰدِقِينَ ﴿٦﴾\\
\textamh{7.\  } & وَلَا يَتَمَنَّوْنَهُۥٓ أَبَدًۢا بِمَا قَدَّمَتْ أَيْدِيهِمْ ۚ وَٱللَّهُ عَلِيمٌۢ بِٱلظَّـٰلِمِينَ ﴿٧﴾\\
\textamh{8.\  } & قُلْ إِنَّ ٱلْمَوْتَ ٱلَّذِى تَفِرُّونَ مِنْهُ فَإِنَّهُۥ مُلَـٰقِيكُمْ ۖ ثُمَّ تُرَدُّونَ إِلَىٰ عَـٰلِمِ ٱلْغَيْبِ وَٱلشَّهَـٰدَةِ فَيُنَبِّئُكُم بِمَا كُنتُمْ تَعْمَلُونَ ﴿٨﴾\\
\textamh{9.\  } & يَـٰٓأَيُّهَا ٱلَّذِينَ ءَامَنُوٓا۟ إِذَا نُودِىَ لِلصَّلَوٰةِ مِن يَوْمِ ٱلْجُمُعَةِ فَٱسْعَوْا۟ إِلَىٰ ذِكْرِ ٱللَّهِ وَذَرُوا۟ ٱلْبَيْعَ ۚ ذَٟلِكُمْ خَيْرٌۭ لَّكُمْ إِن كُنتُمْ تَعْلَمُونَ ﴿٩﴾\\
\textamh{10.\  } & فَإِذَا قُضِيَتِ ٱلصَّلَوٰةُ فَٱنتَشِرُوا۟ فِى ٱلْأَرْضِ وَٱبْتَغُوا۟ مِن فَضْلِ ٱللَّهِ وَٱذْكُرُوا۟ ٱللَّهَ كَثِيرًۭا لَّعَلَّكُمْ تُفْلِحُونَ ﴿١٠﴾\\
\textamh{11.\  } & وَإِذَا رَأَوْا۟ تِجَٰرَةً أَوْ لَهْوًا ٱنفَضُّوٓا۟ إِلَيْهَا وَتَرَكُوكَ قَآئِمًۭا ۚ قُلْ مَا عِندَ ٱللَّهِ خَيْرٌۭ مِّنَ ٱللَّهْوِ وَمِنَ ٱلتِّجَٰرَةِ ۚ وَٱللَّهُ خَيْرُ ٱلرَّٟزِقِينَ ﴿١١﴾\\
\end{longtable}
\clearpage
%% License: BSD style (Berkley) (i.e. Put the Copyright owner's name always)
%% Writer and Copyright (to): Bewketu(Bilal) Tadilo (2016-17)
\begin{center}\section{\LR{\textamhsec{ሱራቱ አልሙናፊቁን -}  \textarabic{سوره  المنافقون}}}\end{center}
\begin{longtable}{%
  @{}
    p{.5\textwidth}
  @{~~~}
    p{.5\textwidth}
    @{}
}
\textamh{ቢስሚላሂ አራህመኒ ራሂይም } &  \mytextarabic{بِسْمِ ٱللَّهِ ٱلرَّحْمَـٰنِ ٱلرَّحِيمِ}\\
\textamh{1.\  } & \mytextarabic{ إِذَا جَآءَكَ ٱلْمُنَـٰفِقُونَ قَالُوا۟ نَشْهَدُ إِنَّكَ لَرَسُولُ ٱللَّهِ ۗ وَٱللَّهُ يَعْلَمُ إِنَّكَ لَرَسُولُهُۥ وَٱللَّهُ يَشْهَدُ إِنَّ ٱلْمُنَـٰفِقِينَ لَكَـٰذِبُونَ ﴿١﴾}\\
\textamh{2.\  } & \mytextarabic{ٱتَّخَذُوٓا۟ أَيْمَـٰنَهُمْ جُنَّةًۭ فَصَدُّوا۟ عَن سَبِيلِ ٱللَّهِ ۚ إِنَّهُمْ سَآءَ مَا كَانُوا۟ يَعْمَلُونَ ﴿٢﴾}\\
\textamh{3.\  } & \mytextarabic{ذَٟلِكَ بِأَنَّهُمْ ءَامَنُوا۟ ثُمَّ كَفَرُوا۟ فَطُبِعَ عَلَىٰ قُلُوبِهِمْ فَهُمْ لَا يَفْقَهُونَ ﴿٣﴾}\\
\textamh{4.\  } & \mytextarabic{۞ وَإِذَا رَأَيْتَهُمْ تُعْجِبُكَ أَجْسَامُهُمْ ۖ وَإِن يَقُولُوا۟ تَسْمَعْ لِقَوْلِهِمْ ۖ كَأَنَّهُمْ خُشُبٌۭ مُّسَنَّدَةٌۭ ۖ يَحْسَبُونَ كُلَّ صَيْحَةٍ عَلَيْهِمْ ۚ هُمُ ٱلْعَدُوُّ فَٱحْذَرْهُمْ ۚ قَـٰتَلَهُمُ ٱللَّهُ ۖ أَنَّىٰ يُؤْفَكُونَ ﴿٤﴾}\\
\textamh{5.\  } & \mytextarabic{وَإِذَا قِيلَ لَهُمْ تَعَالَوْا۟ يَسْتَغْفِرْ لَكُمْ رَسُولُ ٱللَّهِ لَوَّوْا۟ رُءُوسَهُمْ وَرَأَيْتَهُمْ يَصُدُّونَ وَهُم مُّسْتَكْبِرُونَ ﴿٥﴾}\\
\textamh{6.\  } & \mytextarabic{سَوَآءٌ عَلَيْهِمْ أَسْتَغْفَرْتَ لَهُمْ أَمْ لَمْ تَسْتَغْفِرْ لَهُمْ لَن يَغْفِرَ ٱللَّهُ لَهُمْ ۚ إِنَّ ٱللَّهَ لَا يَهْدِى ٱلْقَوْمَ ٱلْفَـٰسِقِينَ ﴿٦﴾}\\
\textamh{7.\  } & \mytextarabic{هُمُ ٱلَّذِينَ يَقُولُونَ لَا تُنفِقُوا۟ عَلَىٰ مَنْ عِندَ رَسُولِ ٱللَّهِ حَتَّىٰ يَنفَضُّوا۟ ۗ وَلِلَّهِ خَزَآئِنُ ٱلسَّمَـٰوَٟتِ وَٱلْأَرْضِ وَلَـٰكِنَّ ٱلْمُنَـٰفِقِينَ لَا يَفْقَهُونَ ﴿٧﴾}\\
\textamh{8.\  } & \mytextarabic{يَقُولُونَ لَئِن رَّجَعْنَآ إِلَى ٱلْمَدِينَةِ لَيُخْرِجَنَّ ٱلْأَعَزُّ مِنْهَا ٱلْأَذَلَّ ۚ وَلِلَّهِ ٱلْعِزَّةُ وَلِرَسُولِهِۦ وَلِلْمُؤْمِنِينَ وَلَـٰكِنَّ ٱلْمُنَـٰفِقِينَ لَا يَعْلَمُونَ ﴿٨﴾}\\
\textamh{9.\  } & \mytextarabic{يَـٰٓأَيُّهَا ٱلَّذِينَ ءَامَنُوا۟ لَا تُلْهِكُمْ أَمْوَٟلُكُمْ وَلَآ أَوْلَـٰدُكُمْ عَن ذِكْرِ ٱللَّهِ ۚ وَمَن يَفْعَلْ ذَٟلِكَ فَأُو۟لَـٰٓئِكَ هُمُ ٱلْخَـٰسِرُونَ ﴿٩﴾}\\
\textamh{10.\  } & \mytextarabic{وَأَنفِقُوا۟ مِن مَّا رَزَقْنَـٰكُم مِّن قَبْلِ أَن يَأْتِىَ أَحَدَكُمُ ٱلْمَوْتُ فَيَقُولَ رَبِّ لَوْلَآ أَخَّرْتَنِىٓ إِلَىٰٓ أَجَلٍۢ قَرِيبٍۢ فَأَصَّدَّقَ وَأَكُن مِّنَ ٱلصَّـٰلِحِينَ ﴿١٠﴾}\\
\textamh{11.\  } & \mytextarabic{وَلَن يُؤَخِّرَ ٱللَّهُ نَفْسًا إِذَا جَآءَ أَجَلُهَا ۚ وَٱللَّهُ خَبِيرٌۢ بِمَا تَعْمَلُونَ ﴿١١﴾}\\
\end{longtable}
\clearpage
%% License: BSD style (Berkley) (i.e. Put the Copyright owner's name always)
%% Writer and Copyright (to): Bewketu(Bilal) Tadilo (2016-17)
\centering\section{\LR{\textamharic{ሱራቱ አልተጋቡን -}  \RL{سوره  التغابن}}}
\begin{longtable}{%
  @{}
    p{.5\textwidth}
  @{~~~~~~~~~~~~~}
    p{.5\textwidth}
    @{}
}
\nopagebreak
\textamh{\ \ \ \ \ \  ቢስሚላሂ አራህመኒ ራሂይም } &  بِسْمِ ٱللَّهِ ٱلرَّحْمَـٰنِ ٱلرَّحِيمِ\\
\textamh{1.\  } &  يُسَبِّحُ لِلَّهِ مَا فِى ٱلسَّمَـٰوَٟتِ وَمَا فِى ٱلْأَرْضِ ۖ لَهُ ٱلْمُلْكُ وَلَهُ ٱلْحَمْدُ ۖ وَهُوَ عَلَىٰ كُلِّ شَىْءٍۢ قَدِيرٌ ﴿١﴾\\
\textamh{2.\  } & هُوَ ٱلَّذِى خَلَقَكُمْ فَمِنكُمْ كَافِرٌۭ وَمِنكُم مُّؤْمِنٌۭ ۚ وَٱللَّهُ بِمَا تَعْمَلُونَ بَصِيرٌ ﴿٢﴾\\
\textamh{3.\  } & خَلَقَ ٱلسَّمَـٰوَٟتِ وَٱلْأَرْضَ بِٱلْحَقِّ وَصَوَّرَكُمْ فَأَحْسَنَ صُوَرَكُمْ ۖ وَإِلَيْهِ ٱلْمَصِيرُ ﴿٣﴾\\
\textamh{4.\  } & يَعْلَمُ مَا فِى ٱلسَّمَـٰوَٟتِ وَٱلْأَرْضِ وَيَعْلَمُ مَا تُسِرُّونَ وَمَا تُعْلِنُونَ ۚ وَٱللَّهُ عَلِيمٌۢ بِذَاتِ ٱلصُّدُورِ ﴿٤﴾\\
\textamh{5.\  } & أَلَمْ يَأْتِكُمْ نَبَؤُا۟ ٱلَّذِينَ كَفَرُوا۟ مِن قَبْلُ فَذَاقُوا۟ وَبَالَ أَمْرِهِمْ وَلَهُمْ عَذَابٌ أَلِيمٌۭ ﴿٥﴾\\
\textamh{6.\  } & ذَٟلِكَ بِأَنَّهُۥ كَانَت تَّأْتِيهِمْ رُسُلُهُم بِٱلْبَيِّنَـٰتِ فَقَالُوٓا۟ أَبَشَرٌۭ يَهْدُونَنَا فَكَفَرُوا۟ وَتَوَلَّوا۟ ۚ وَّٱسْتَغْنَى ٱللَّهُ ۚ وَٱللَّهُ غَنِىٌّ حَمِيدٌۭ ﴿٦﴾\\
\textamh{7.\  } & زَعَمَ ٱلَّذِينَ كَفَرُوٓا۟ أَن لَّن يُبْعَثُوا۟ ۚ قُلْ بَلَىٰ وَرَبِّى لَتُبْعَثُنَّ ثُمَّ لَتُنَبَّؤُنَّ بِمَا عَمِلْتُمْ ۚ وَذَٟلِكَ عَلَى ٱللَّهِ يَسِيرٌۭ ﴿٧﴾\\
\textamh{8.\  } & فَـَٔامِنُوا۟ بِٱللَّهِ وَرَسُولِهِۦ وَٱلنُّورِ ٱلَّذِىٓ أَنزَلْنَا ۚ وَٱللَّهُ بِمَا تَعْمَلُونَ خَبِيرٌۭ ﴿٨﴾\\
\textamh{9.\  } & يَوْمَ يَجْمَعُكُمْ لِيَوْمِ ٱلْجَمْعِ ۖ ذَٟلِكَ يَوْمُ ٱلتَّغَابُنِ ۗ وَمَن يُؤْمِنۢ بِٱللَّهِ وَيَعْمَلْ صَـٰلِحًۭا يُكَفِّرْ عَنْهُ سَيِّـَٔاتِهِۦ وَيُدْخِلْهُ جَنَّـٰتٍۢ تَجْرِى مِن تَحْتِهَا ٱلْأَنْهَـٰرُ خَـٰلِدِينَ فِيهَآ أَبَدًۭا ۚ ذَٟلِكَ ٱلْفَوْزُ ٱلْعَظِيمُ ﴿٩﴾\\
\textamh{10.\  } & وَٱلَّذِينَ كَفَرُوا۟ وَكَذَّبُوا۟ بِـَٔايَـٰتِنَآ أُو۟لَـٰٓئِكَ أَصْحَـٰبُ ٱلنَّارِ خَـٰلِدِينَ فِيهَا ۖ وَبِئْسَ ٱلْمَصِيرُ ﴿١٠﴾\\
\textamh{11.\  } & مَآ أَصَابَ مِن مُّصِيبَةٍ إِلَّا بِإِذْنِ ٱللَّهِ ۗ وَمَن يُؤْمِنۢ بِٱللَّهِ يَهْدِ قَلْبَهُۥ ۚ وَٱللَّهُ بِكُلِّ شَىْءٍ عَلِيمٌۭ ﴿١١﴾\\
\textamh{12.\  } & وَأَطِيعُوا۟ ٱللَّهَ وَأَطِيعُوا۟ ٱلرَّسُولَ ۚ فَإِن تَوَلَّيْتُمْ فَإِنَّمَا عَلَىٰ رَسُولِنَا ٱلْبَلَـٰغُ ٱلْمُبِينُ ﴿١٢﴾\\
\textamh{13.\  } & ٱللَّهُ لَآ إِلَـٰهَ إِلَّا هُوَ ۚ وَعَلَى ٱللَّهِ فَلْيَتَوَكَّلِ ٱلْمُؤْمِنُونَ ﴿١٣﴾\\
\textamh{14.\  } & يَـٰٓأَيُّهَا ٱلَّذِينَ ءَامَنُوٓا۟ إِنَّ مِنْ أَزْوَٟجِكُمْ وَأَوْلَـٰدِكُمْ عَدُوًّۭا لَّكُمْ فَٱحْذَرُوهُمْ ۚ وَإِن تَعْفُوا۟ وَتَصْفَحُوا۟ وَتَغْفِرُوا۟ فَإِنَّ ٱللَّهَ غَفُورٌۭ رَّحِيمٌ ﴿١٤﴾\\
\textamh{15.\  } & إِنَّمَآ أَمْوَٟلُكُمْ وَأَوْلَـٰدُكُمْ فِتْنَةٌۭ ۚ وَٱللَّهُ عِندَهُۥٓ أَجْرٌ عَظِيمٌۭ ﴿١٥﴾\\
\textamh{16.\  } & فَٱتَّقُوا۟ ٱللَّهَ مَا ٱسْتَطَعْتُمْ وَٱسْمَعُوا۟ وَأَطِيعُوا۟ وَأَنفِقُوا۟ خَيْرًۭا لِّأَنفُسِكُمْ ۗ وَمَن يُوقَ شُحَّ نَفْسِهِۦ فَأُو۟لَـٰٓئِكَ هُمُ ٱلْمُفْلِحُونَ ﴿١٦﴾\\
\textamh{17.\  } & إِن تُقْرِضُوا۟ ٱللَّهَ قَرْضًا حَسَنًۭا يُضَٰعِفْهُ لَكُمْ وَيَغْفِرْ لَكُمْ ۚ وَٱللَّهُ شَكُورٌ حَلِيمٌ ﴿١٧﴾\\
\textamh{18.\  } & عَـٰلِمُ ٱلْغَيْبِ وَٱلشَّهَـٰدَةِ ٱلْعَزِيزُ ٱلْحَكِيمُ ﴿١٨﴾\\
\end{longtable} \newpage

%% License: BSD style (Berkley) (i.e. Put the Copyright owner's name always)
%% Writer and Copyright (to): Bewketu(Bilal) Tadilo (2016-17)
\centering\section{\LR{\textamharic{ሱራቱ አጥጠለቅ -}  \RL{سوره  الطلاق}}}
\begin{longtable}{%
  @{}
    p{.5\textwidth}
  @{~~~~~~~~~~~~}
    p{.5\textwidth}
    @{}
}
\nopagebreak
\textamh{ቢስሚላሂ አራህመኒ ራሂይም } &  بِسْمِ ٱللَّهِ ٱلرَّحْمَـٰنِ ٱلرَّحِيمِ\\
\textamh{1.\  } &  يَـٰٓأَيُّهَا ٱلنَّبِىُّ إِذَا طَلَّقْتُمُ ٱلنِّسَآءَ فَطَلِّقُوهُنَّ لِعِدَّتِهِنَّ وَأَحْصُوا۟ ٱلْعِدَّةَ ۖ وَٱتَّقُوا۟ ٱللَّهَ رَبَّكُمْ ۖ لَا تُخْرِجُوهُنَّ مِنۢ بُيُوتِهِنَّ وَلَا يَخْرُجْنَ إِلَّآ أَن يَأْتِينَ بِفَـٰحِشَةٍۢ مُّبَيِّنَةٍۢ ۚ وَتِلْكَ حُدُودُ ٱللَّهِ ۚ وَمَن يَتَعَدَّ حُدُودَ ٱللَّهِ فَقَدْ ظَلَمَ نَفْسَهُۥ ۚ لَا تَدْرِى لَعَلَّ ٱللَّهَ يُحْدِثُ بَعْدَ ذَٟلِكَ أَمْرًۭا ﴿١﴾\\
\textamh{2.\  } & فَإِذَا بَلَغْنَ أَجَلَهُنَّ فَأَمْسِكُوهُنَّ بِمَعْرُوفٍ أَوْ فَارِقُوهُنَّ بِمَعْرُوفٍۢ وَأَشْهِدُوا۟ ذَوَىْ عَدْلٍۢ مِّنكُمْ وَأَقِيمُوا۟ ٱلشَّهَـٰدَةَ لِلَّهِ ۚ ذَٟلِكُمْ يُوعَظُ بِهِۦ مَن كَانَ يُؤْمِنُ بِٱللَّهِ وَٱلْيَوْمِ ٱلْءَاخِرِ ۚ وَمَن يَتَّقِ ٱللَّهَ يَجْعَل لَّهُۥ مَخْرَجًۭا ﴿٢﴾\\
\textamh{3.\  } & وَيَرْزُقْهُ مِنْ حَيْثُ لَا يَحْتَسِبُ ۚ وَمَن يَتَوَكَّلْ عَلَى ٱللَّهِ فَهُوَ حَسْبُهُۥٓ ۚ إِنَّ ٱللَّهَ بَٰلِغُ أَمْرِهِۦ ۚ قَدْ جَعَلَ ٱللَّهُ لِكُلِّ شَىْءٍۢ قَدْرًۭا ﴿٣﴾\\
\textamh{4.\  } & وَٱلَّٰٓـِٔى يَئِسْنَ مِنَ ٱلْمَحِيضِ مِن نِّسَآئِكُمْ إِنِ ٱرْتَبْتُمْ فَعِدَّتُهُنَّ ثَلَـٰثَةُ أَشْهُرٍۢ وَٱلَّٰٓـِٔى لَمْ يَحِضْنَ ۚ وَأُو۟لَـٰتُ ٱلْأَحْمَالِ أَجَلُهُنَّ أَن يَضَعْنَ حَمْلَهُنَّ ۚ وَمَن يَتَّقِ ٱللَّهَ يَجْعَل لَّهُۥ مِنْ أَمْرِهِۦ يُسْرًۭا ﴿٤﴾\\
\textamh{5.\  } & ذَٟلِكَ أَمْرُ ٱللَّهِ أَنزَلَهُۥٓ إِلَيْكُمْ ۚ وَمَن يَتَّقِ ٱللَّهَ يُكَفِّرْ عَنْهُ سَيِّـَٔاتِهِۦ وَيُعْظِمْ لَهُۥٓ أَجْرًا ﴿٥﴾\\
\textamh{6.\  } & أَسْكِنُوهُنَّ مِنْ حَيْثُ سَكَنتُم مِّن وُجْدِكُمْ وَلَا تُضَآرُّوهُنَّ لِتُضَيِّقُوا۟ عَلَيْهِنَّ ۚ وَإِن كُنَّ أُو۟لَـٰتِ حَمْلٍۢ فَأَنفِقُوا۟ عَلَيْهِنَّ حَتَّىٰ يَضَعْنَ حَمْلَهُنَّ ۚ فَإِنْ أَرْضَعْنَ لَكُمْ فَـَٔاتُوهُنَّ أُجُورَهُنَّ ۖ وَأْتَمِرُوا۟ بَيْنَكُم بِمَعْرُوفٍۢ ۖ وَإِن تَعَاسَرْتُمْ فَسَتُرْضِعُ لَهُۥٓ أُخْرَىٰ ﴿٦﴾\\
\textamh{7.\  } & لِيُنفِقْ ذُو سَعَةٍۢ مِّن سَعَتِهِۦ ۖ وَمَن قُدِرَ عَلَيْهِ رِزْقُهُۥ فَلْيُنفِقْ مِمَّآ ءَاتَىٰهُ ٱللَّهُ ۚ لَا يُكَلِّفُ ٱللَّهُ نَفْسًا إِلَّا مَآ ءَاتَىٰهَا ۚ سَيَجْعَلُ ٱللَّهُ بَعْدَ عُسْرٍۢ يُسْرًۭا ﴿٧﴾\\
\textamh{8.\  } & وَكَأَيِّن مِّن قَرْيَةٍ عَتَتْ عَنْ أَمْرِ رَبِّهَا وَرُسُلِهِۦ فَحَاسَبْنَـٰهَا حِسَابًۭا شَدِيدًۭا وَعَذَّبْنَـٰهَا عَذَابًۭا نُّكْرًۭا ﴿٨﴾\\
\textamh{9.\  } & فَذَاقَتْ وَبَالَ أَمْرِهَا وَكَانَ عَـٰقِبَةُ أَمْرِهَا خُسْرًا ﴿٩﴾\\
\textamh{10.\  } & أَعَدَّ ٱللَّهُ لَهُمْ عَذَابًۭا شَدِيدًۭا ۖ فَٱتَّقُوا۟ ٱللَّهَ يَـٰٓأُو۟لِى ٱلْأَلْبَٰبِ ٱلَّذِينَ ءَامَنُوا۟ ۚ قَدْ أَنزَلَ ٱللَّهُ إِلَيْكُمْ ذِكْرًۭا ﴿١٠﴾\\
\textamh{11.\  } & رَّسُولًۭا يَتْلُوا۟ عَلَيْكُمْ ءَايَـٰتِ ٱللَّهِ مُبَيِّنَـٰتٍۢ لِّيُخْرِجَ ٱلَّذِينَ ءَامَنُوا۟ وَعَمِلُوا۟ ٱلصَّـٰلِحَـٰتِ مِنَ ٱلظُّلُمَـٰتِ إِلَى ٱلنُّورِ ۚ وَمَن يُؤْمِنۢ بِٱللَّهِ وَيَعْمَلْ صَـٰلِحًۭا يُدْخِلْهُ جَنَّـٰتٍۢ تَجْرِى مِن تَحْتِهَا ٱلْأَنْهَـٰرُ خَـٰلِدِينَ فِيهَآ أَبَدًۭا ۖ قَدْ أَحْسَنَ ٱللَّهُ لَهُۥ رِزْقًا ﴿١١﴾\\
\textamh{12.\  } & ٱللَّهُ ٱلَّذِى خَلَقَ سَبْعَ سَمَـٰوَٟتٍۢ وَمِنَ ٱلْأَرْضِ مِثْلَهُنَّ يَتَنَزَّلُ ٱلْأَمْرُ بَيْنَهُنَّ لِتَعْلَمُوٓا۟ أَنَّ ٱللَّهَ عَلَىٰ كُلِّ شَىْءٍۢ قَدِيرٌۭ وَأَنَّ ٱللَّهَ قَدْ أَحَاطَ بِكُلِّ شَىْءٍ عِلْمًۢا ﴿١٢﴾\\
\end{longtable}
\clearpage
%% License: BSD style (Berkley) (i.e. Put the Copyright owner's name always)
%% Writer and Copyright (to): Bewketu(Bilal) Tadilo (2016-17)
\centering\section{\LR{\textamharic{ሱራቱ አትታህሪይም -}  \RL{سوره  التحريم}}}
\begin{longtable}{%
  @{}
    p{.5\textwidth}
  @{~~~~~~~~~~~~~}
    p{.5\textwidth}
    @{}
}
\nopagebreak
\textamh{\ \ \ \ \ \  ቢስሚላሂ አራህመኒ ራሂይም } &  بِسْمِ ٱللَّهِ ٱلرَّحْمَـٰنِ ٱلرَّحِيمِ\\
\textamh{1.\  } &  يَـٰٓأَيُّهَا ٱلنَّبِىُّ لِمَ تُحَرِّمُ مَآ أَحَلَّ ٱللَّهُ لَكَ ۖ تَبْتَغِى مَرْضَاتَ أَزْوَٟجِكَ ۚ وَٱللَّهُ غَفُورٌۭ رَّحِيمٌۭ ﴿١﴾\\
\textamh{2.\  } & قَدْ فَرَضَ ٱللَّهُ لَكُمْ تَحِلَّةَ أَيْمَـٰنِكُمْ ۚ وَٱللَّهُ مَوْلَىٰكُمْ ۖ وَهُوَ ٱلْعَلِيمُ ٱلْحَكِيمُ ﴿٢﴾\\
\textamh{3.\  } & وَإِذْ أَسَرَّ ٱلنَّبِىُّ إِلَىٰ بَعْضِ أَزْوَٟجِهِۦ حَدِيثًۭا فَلَمَّا نَبَّأَتْ بِهِۦ وَأَظْهَرَهُ ٱللَّهُ عَلَيْهِ عَرَّفَ بَعْضَهُۥ وَأَعْرَضَ عَنۢ بَعْضٍۢ ۖ فَلَمَّا نَبَّأَهَا بِهِۦ قَالَتْ مَنْ أَنۢبَأَكَ هَـٰذَا ۖ قَالَ نَبَّأَنِىَ ٱلْعَلِيمُ ٱلْخَبِيرُ ﴿٣﴾\\
\textamh{4.\  } & إِن تَتُوبَآ إِلَى ٱللَّهِ فَقَدْ صَغَتْ قُلُوبُكُمَا ۖ وَإِن تَظَـٰهَرَا عَلَيْهِ فَإِنَّ ٱللَّهَ هُوَ مَوْلَىٰهُ وَجِبْرِيلُ وَصَـٰلِحُ ٱلْمُؤْمِنِينَ ۖ وَٱلْمَلَـٰٓئِكَةُ بَعْدَ ذَٟلِكَ ظَهِيرٌ ﴿٤﴾\\
\textamh{5.\  } & عَسَىٰ رَبُّهُۥٓ إِن طَلَّقَكُنَّ أَن يُبْدِلَهُۥٓ أَزْوَٟجًا خَيْرًۭا مِّنكُنَّ مُسْلِمَـٰتٍۢ مُّؤْمِنَـٰتٍۢ قَـٰنِتَـٰتٍۢ تَـٰٓئِبَٰتٍ عَـٰبِدَٟتٍۢ سَـٰٓئِحَـٰتٍۢ ثَيِّبَٰتٍۢ وَأَبْكَارًۭا ﴿٥﴾\\
\textamh{6.\  } & يَـٰٓأَيُّهَا ٱلَّذِينَ ءَامَنُوا۟ قُوٓا۟ أَنفُسَكُمْ وَأَهْلِيكُمْ نَارًۭا وَقُودُهَا ٱلنَّاسُ وَٱلْحِجَارَةُ عَلَيْهَا مَلَـٰٓئِكَةٌ غِلَاظٌۭ شِدَادٌۭ لَّا يَعْصُونَ ٱللَّهَ مَآ أَمَرَهُمْ وَيَفْعَلُونَ مَا يُؤْمَرُونَ ﴿٦﴾\\
\textamh{7.\  } & يَـٰٓأَيُّهَا ٱلَّذِينَ كَفَرُوا۟ لَا تَعْتَذِرُوا۟ ٱلْيَوْمَ ۖ إِنَّمَا تُجْزَوْنَ مَا كُنتُمْ تَعْمَلُونَ ﴿٧﴾\\
\textamh{8.\  } & يَـٰٓأَيُّهَا ٱلَّذِينَ ءَامَنُوا۟ تُوبُوٓا۟ إِلَى ٱللَّهِ تَوْبَةًۭ نَّصُوحًا عَسَىٰ رَبُّكُمْ أَن يُكَفِّرَ عَنكُمْ سَيِّـَٔاتِكُمْ وَيُدْخِلَكُمْ جَنَّـٰتٍۢ تَجْرِى مِن تَحْتِهَا ٱلْأَنْهَـٰرُ يَوْمَ لَا يُخْزِى ٱللَّهُ ٱلنَّبِىَّ وَٱلَّذِينَ ءَامَنُوا۟ مَعَهُۥ ۖ نُورُهُمْ يَسْعَىٰ بَيْنَ أَيْدِيهِمْ وَبِأَيْمَـٰنِهِمْ يَقُولُونَ رَبَّنَآ أَتْمِمْ لَنَا نُورَنَا وَٱغْفِرْ لَنَآ ۖ إِنَّكَ عَلَىٰ كُلِّ شَىْءٍۢ قَدِيرٌۭ ﴿٨﴾\\
\textamh{9.\  } & يَـٰٓأَيُّهَا ٱلنَّبِىُّ جَٰهِدِ ٱلْكُفَّارَ وَٱلْمُنَـٰفِقِينَ وَٱغْلُظْ عَلَيْهِمْ ۚ وَمَأْوَىٰهُمْ جَهَنَّمُ ۖ وَبِئْسَ ٱلْمَصِيرُ ﴿٩﴾\\
\textamh{10.\  } & ضَرَبَ ٱللَّهُ مَثَلًۭا لِّلَّذِينَ كَفَرُوا۟ ٱمْرَأَتَ نُوحٍۢ وَٱمْرَأَتَ لُوطٍۢ ۖ كَانَتَا تَحْتَ عَبْدَيْنِ مِنْ عِبَادِنَا صَـٰلِحَيْنِ فَخَانَتَاهُمَا فَلَمْ يُغْنِيَا عَنْهُمَا مِنَ ٱللَّهِ شَيْـًۭٔا وَقِيلَ ٱدْخُلَا ٱلنَّارَ مَعَ ٱلدَّٰخِلِينَ ﴿١٠﴾\\
\textamh{11.\  } & وَضَرَبَ ٱللَّهُ مَثَلًۭا لِّلَّذِينَ ءَامَنُوا۟ ٱمْرَأَتَ فِرْعَوْنَ إِذْ قَالَتْ رَبِّ ٱبْنِ لِى عِندَكَ بَيْتًۭا فِى ٱلْجَنَّةِ وَنَجِّنِى مِن فِرْعَوْنَ وَعَمَلِهِۦ وَنَجِّنِى مِنَ ٱلْقَوْمِ ٱلظَّـٰلِمِينَ ﴿١١﴾\\
\textamh{12.\  } & وَمَرْيَمَ ٱبْنَتَ عِمْرَٰنَ ٱلَّتِىٓ أَحْصَنَتْ فَرْجَهَا فَنَفَخْنَا فِيهِ مِن رُّوحِنَا وَصَدَّقَتْ بِكَلِمَـٰتِ رَبِّهَا وَكُتُبِهِۦ وَكَانَتْ مِنَ ٱلْقَـٰنِتِينَ ﴿١٢﴾\\
\end{longtable} \newpage

%% License: BSD style (Berkley) (i.e. Put the Copyright owner's name always)
%% Writer and Copyright (to): Bewketu(Bilal) Tadilo (2016-17)
\centering\section{\LR{\textamharic{ሱራቱ አልሙልክ -}  \RL{سوره  الملك}}}
\begin{longtable}{%
  @{}
    p{.5\textwidth}
  @{~~~~~~~~~~~~~}
    p{.5\textwidth}
    @{}
}
\nopagebreak
\textamh{ቢስሚላሂ አራህመኒ ራሂይም } &  بِسْمِ ٱللَّهِ ٱلرَّحْمَـٰنِ ٱلرَّحِيمِ\\
\textamh{1.\  } &  تَبَٰرَكَ ٱلَّذِى بِيَدِهِ ٱلْمُلْكُ وَهُوَ عَلَىٰ كُلِّ شَىْءٍۢ قَدِيرٌ ﴿١﴾\\
\textamh{2.\  } & ٱلَّذِى خَلَقَ ٱلْمَوْتَ وَٱلْحَيَوٰةَ لِيَبْلُوَكُمْ أَيُّكُمْ أَحْسَنُ عَمَلًۭا ۚ وَهُوَ ٱلْعَزِيزُ ٱلْغَفُورُ ﴿٢﴾\\
\textamh{3.\  } & ٱلَّذِى خَلَقَ سَبْعَ سَمَـٰوَٟتٍۢ طِبَاقًۭا ۖ مَّا تَرَىٰ فِى خَلْقِ ٱلرَّحْمَـٰنِ مِن تَفَـٰوُتٍۢ ۖ فَٱرْجِعِ ٱلْبَصَرَ هَلْ تَرَىٰ مِن فُطُورٍۢ ﴿٣﴾\\
\textamh{4.\  } & ثُمَّ ٱرْجِعِ ٱلْبَصَرَ كَرَّتَيْنِ يَنقَلِبْ إِلَيْكَ ٱلْبَصَرُ خَاسِئًۭا وَهُوَ حَسِيرٌۭ ﴿٤﴾\\
\textamh{5.\  } & وَلَقَدْ زَيَّنَّا ٱلسَّمَآءَ ٱلدُّنْيَا بِمَصَـٰبِيحَ وَجَعَلْنَـٰهَا رُجُومًۭا لِّلشَّيَـٰطِينِ ۖ وَأَعْتَدْنَا لَهُمْ عَذَابَ ٱلسَّعِيرِ ﴿٥﴾\\
\textamh{6.\  } & وَلِلَّذِينَ كَفَرُوا۟ بِرَبِّهِمْ عَذَابُ جَهَنَّمَ ۖ وَبِئْسَ ٱلْمَصِيرُ ﴿٦﴾\\
\textamh{7.\  } & إِذَآ أُلْقُوا۟ فِيهَا سَمِعُوا۟ لَهَا شَهِيقًۭا وَهِىَ تَفُورُ ﴿٧﴾\\
\textamh{8.\  } & تَكَادُ تَمَيَّزُ مِنَ ٱلْغَيْظِ ۖ كُلَّمَآ أُلْقِىَ فِيهَا فَوْجٌۭ سَأَلَهُمْ خَزَنَتُهَآ أَلَمْ يَأْتِكُمْ نَذِيرٌۭ ﴿٨﴾\\
\textamh{9.\  } & قَالُوا۟ بَلَىٰ قَدْ جَآءَنَا نَذِيرٌۭ فَكَذَّبْنَا وَقُلْنَا مَا نَزَّلَ ٱللَّهُ مِن شَىْءٍ إِنْ أَنتُمْ إِلَّا فِى ضَلَـٰلٍۢ كَبِيرٍۢ ﴿٩﴾\\
\textamh{10.\  } & وَقَالُوا۟ لَوْ كُنَّا نَسْمَعُ أَوْ نَعْقِلُ مَا كُنَّا فِىٓ أَصْحَـٰبِ ٱلسَّعِيرِ ﴿١٠﴾\\
\textamh{11.\  } & فَٱعْتَرَفُوا۟ بِذَنۢبِهِمْ فَسُحْقًۭا لِّأَصْحَـٰبِ ٱلسَّعِيرِ ﴿١١﴾\\
\textamh{12.\  } & إِنَّ ٱلَّذِينَ يَخْشَوْنَ رَبَّهُم بِٱلْغَيْبِ لَهُم مَّغْفِرَةٌۭ وَأَجْرٌۭ كَبِيرٌۭ ﴿١٢﴾\\
\textamh{13.\  } & وَأَسِرُّوا۟ قَوْلَكُمْ أَوِ ٱجْهَرُوا۟ بِهِۦٓ ۖ إِنَّهُۥ عَلِيمٌۢ بِذَاتِ ٱلصُّدُورِ ﴿١٣﴾\\
\textamh{14.\  } & أَلَا يَعْلَمُ مَنْ خَلَقَ وَهُوَ ٱللَّطِيفُ ٱلْخَبِيرُ ﴿١٤﴾\\
\textamh{15.\  } & هُوَ ٱلَّذِى جَعَلَ لَكُمُ ٱلْأَرْضَ ذَلُولًۭا فَٱمْشُوا۟ فِى مَنَاكِبِهَا وَكُلُوا۟ مِن رِّزْقِهِۦ ۖ وَإِلَيْهِ ٱلنُّشُورُ ﴿١٥﴾\\
\textamh{16.\  } & ءَأَمِنتُم مَّن فِى ٱلسَّمَآءِ أَن يَخْسِفَ بِكُمُ ٱلْأَرْضَ فَإِذَا هِىَ تَمُورُ ﴿١٦﴾\\
\textamh{17.\  } & أَمْ أَمِنتُم مَّن فِى ٱلسَّمَآءِ أَن يُرْسِلَ عَلَيْكُمْ حَاصِبًۭا ۖ فَسَتَعْلَمُونَ كَيْفَ نَذِيرِ ﴿١٧﴾\\
\textamh{18.\  } & وَلَقَدْ كَذَّبَ ٱلَّذِينَ مِن قَبْلِهِمْ فَكَيْفَ كَانَ نَكِيرِ ﴿١٨﴾\\
\textamh{19.\  } & أَوَلَمْ يَرَوْا۟ إِلَى ٱلطَّيْرِ فَوْقَهُمْ صَـٰٓفَّٰتٍۢ وَيَقْبِضْنَ ۚ مَا يُمْسِكُهُنَّ إِلَّا ٱلرَّحْمَـٰنُ ۚ إِنَّهُۥ بِكُلِّ شَىْءٍۭ بَصِيرٌ ﴿١٩﴾\\
\textamh{20.\  } & أَمَّنْ هَـٰذَا ٱلَّذِى هُوَ جُندٌۭ لَّكُمْ يَنصُرُكُم مِّن دُونِ ٱلرَّحْمَـٰنِ ۚ إِنِ ٱلْكَـٰفِرُونَ إِلَّا فِى غُرُورٍ ﴿٢٠﴾\\
\textamh{21.\  } & أَمَّنْ هَـٰذَا ٱلَّذِى يَرْزُقُكُمْ إِنْ أَمْسَكَ رِزْقَهُۥ ۚ بَل لَّجُّوا۟ فِى عُتُوٍّۢ وَنُفُورٍ ﴿٢١﴾\\
\textamh{22.\  } & أَفَمَن يَمْشِى مُكِبًّا عَلَىٰ وَجْهِهِۦٓ أَهْدَىٰٓ أَمَّن يَمْشِى سَوِيًّا عَلَىٰ صِرَٰطٍۢ مُّسْتَقِيمٍۢ ﴿٢٢﴾\\
\textamh{23.\  } & قُلْ هُوَ ٱلَّذِىٓ أَنشَأَكُمْ وَجَعَلَ لَكُمُ ٱلسَّمْعَ وَٱلْأَبْصَـٰرَ وَٱلْأَفْـِٔدَةَ ۖ قَلِيلًۭا مَّا تَشْكُرُونَ ﴿٢٣﴾\\
\textamh{24.\  } & قُلْ هُوَ ٱلَّذِى ذَرَأَكُمْ فِى ٱلْأَرْضِ وَإِلَيْهِ تُحْشَرُونَ ﴿٢٤﴾\\
\textamh{25.\  } & وَيَقُولُونَ مَتَىٰ هَـٰذَا ٱلْوَعْدُ إِن كُنتُمْ صَـٰدِقِينَ ﴿٢٥﴾\\
\textamh{26.\  } & قُلْ إِنَّمَا ٱلْعِلْمُ عِندَ ٱللَّهِ وَإِنَّمَآ أَنَا۠ نَذِيرٌۭ مُّبِينٌۭ ﴿٢٦﴾\\
\textamh{27.\  } & فَلَمَّا رَأَوْهُ زُلْفَةًۭ سِيٓـَٔتْ وُجُوهُ ٱلَّذِينَ كَفَرُوا۟ وَقِيلَ هَـٰذَا ٱلَّذِى كُنتُم بِهِۦ تَدَّعُونَ ﴿٢٧﴾\\
\textamh{28.\  } & قُلْ أَرَءَيْتُمْ إِنْ أَهْلَكَنِىَ ٱللَّهُ وَمَن مَّعِىَ أَوْ رَحِمَنَا فَمَن يُجِيرُ ٱلْكَـٰفِرِينَ مِنْ عَذَابٍ أَلِيمٍۢ ﴿٢٨﴾\\
\textamh{29.\  } & قُلْ هُوَ ٱلرَّحْمَـٰنُ ءَامَنَّا بِهِۦ وَعَلَيْهِ تَوَكَّلْنَا ۖ فَسَتَعْلَمُونَ مَنْ هُوَ فِى ضَلَـٰلٍۢ مُّبِينٍۢ ﴿٢٩﴾\\
\textamh{30.\  } & قُلْ أَرَءَيْتُمْ إِنْ أَصْبَحَ مَآؤُكُمْ غَوْرًۭا فَمَن يَأْتِيكُم بِمَآءٍۢ مَّعِينٍۭ ﴿٣٠﴾\\
\end{longtable}
\clearpage
%% License: BSD style (Berkley) (i.e. Put the Copyright owner's name always)
%% Writer and Copyright (to): Bewketu(Bilal) Tadilo (2016-17)
\centering\section{\LR{\textamharic{ሱራቱ አልቀለም -}  \RL{سوره  القلم}}}
\begin{longtable}{%
  @{}
    p{.5\textwidth}
  @{~~~~~~~~~~~~}
    p{.5\textwidth}
    @{}
}
\nopagebreak
\textamh{ቢስሚላሂ አራህመኒ ራሂይም } &  بِسْمِ ٱللَّهِ ٱلرَّحْمَـٰنِ ٱلرَّحِيمِ\\
\textamh{1.\  } &  نٓ ۚ وَٱلْقَلَمِ وَمَا يَسْطُرُونَ ﴿١﴾\\
\textamh{2.\  } & مَآ أَنتَ بِنِعْمَةِ رَبِّكَ بِمَجْنُونٍۢ ﴿٢﴾\\
\textamh{3.\  } & وَإِنَّ لَكَ لَأَجْرًا غَيْرَ مَمْنُونٍۢ ﴿٣﴾\\
\textamh{4.\  } & وَإِنَّكَ لَعَلَىٰ خُلُقٍ عَظِيمٍۢ ﴿٤﴾\\
\textamh{5.\  } & فَسَتُبْصِرُ وَيُبْصِرُونَ ﴿٥﴾\\
\textamh{6.\  } & بِأَييِّكُمُ ٱلْمَفْتُونُ ﴿٦﴾\\
\textamh{7.\  } & إِنَّ رَبَّكَ هُوَ أَعْلَمُ بِمَن ضَلَّ عَن سَبِيلِهِۦ وَهُوَ أَعْلَمُ بِٱلْمُهْتَدِينَ ﴿٧﴾\\
\textamh{8.\  } & فَلَا تُطِعِ ٱلْمُكَذِّبِينَ ﴿٨﴾\\
\textamh{9.\  } & وَدُّوا۟ لَوْ تُدْهِنُ فَيُدْهِنُونَ ﴿٩﴾\\
\textamh{10.\  } & وَلَا تُطِعْ كُلَّ حَلَّافٍۢ مَّهِينٍ ﴿١٠﴾\\
\textamh{11.\  } & هَمَّازٍۢ مَّشَّآءٍۭ بِنَمِيمٍۢ ﴿١١﴾\\
\textamh{12.\  } & مَّنَّاعٍۢ لِّلْخَيْرِ مُعْتَدٍ أَثِيمٍ ﴿١٢﴾\\
\textamh{13.\  } & عُتُلٍّۭ بَعْدَ ذَٟلِكَ زَنِيمٍ ﴿١٣﴾\\
\textamh{14.\  } & أَن كَانَ ذَا مَالٍۢ وَبَنِينَ ﴿١٤﴾\\
\textamh{15.\  } & إِذَا تُتْلَىٰ عَلَيْهِ ءَايَـٰتُنَا قَالَ أَسَـٰطِيرُ ٱلْأَوَّلِينَ ﴿١٥﴾\\
\textamh{16.\  } & سَنَسِمُهُۥ عَلَى ٱلْخُرْطُومِ ﴿١٦﴾\\
\textamh{17.\  } & إِنَّا بَلَوْنَـٰهُمْ كَمَا بَلَوْنَآ أَصْحَـٰبَ ٱلْجَنَّةِ إِذْ أَقْسَمُوا۟ لَيَصْرِمُنَّهَا مُصْبِحِينَ ﴿١٧﴾\\
\textamh{18.\  } & وَلَا يَسْتَثْنُونَ ﴿١٨﴾\\
\textamh{19.\  } & فَطَافَ عَلَيْهَا طَآئِفٌۭ مِّن رَّبِّكَ وَهُمْ نَآئِمُونَ ﴿١٩﴾\\
\textamh{20.\  } & فَأَصْبَحَتْ كَٱلصَّرِيمِ ﴿٢٠﴾\\
\textamh{21.\  } & فَتَنَادَوْا۟ مُصْبِحِينَ ﴿٢١﴾\\
\textamh{22.\  } & أَنِ ٱغْدُوا۟ عَلَىٰ حَرْثِكُمْ إِن كُنتُمْ صَـٰرِمِينَ ﴿٢٢﴾\\
\textamh{23.\  } & فَٱنطَلَقُوا۟ وَهُمْ يَتَخَـٰفَتُونَ ﴿٢٣﴾\\
\textamh{24.\  } & أَن لَّا يَدْخُلَنَّهَا ٱلْيَوْمَ عَلَيْكُم مِّسْكِينٌۭ ﴿٢٤﴾\\
\textamh{25.\  } & وَغَدَوْا۟ عَلَىٰ حَرْدٍۢ قَـٰدِرِينَ ﴿٢٥﴾\\
\textamh{26.\  } & فَلَمَّا رَأَوْهَا قَالُوٓا۟ إِنَّا لَضَآلُّونَ ﴿٢٦﴾\\
\textamh{27.\  } & بَلْ نَحْنُ مَحْرُومُونَ ﴿٢٧﴾\\
\textamh{28.\  } & قَالَ أَوْسَطُهُمْ أَلَمْ أَقُل لَّكُمْ لَوْلَا تُسَبِّحُونَ ﴿٢٨﴾\\
\textamh{29.\  } & قَالُوا۟ سُبْحَـٰنَ رَبِّنَآ إِنَّا كُنَّا ظَـٰلِمِينَ ﴿٢٩﴾\\
\textamh{30.\  } & فَأَقْبَلَ بَعْضُهُمْ عَلَىٰ بَعْضٍۢ يَتَلَـٰوَمُونَ ﴿٣٠﴾\\
\textamh{31.\  } & قَالُوا۟ يَـٰوَيْلَنَآ إِنَّا كُنَّا طَٰغِينَ ﴿٣١﴾\\
\textamh{32.\  } & عَسَىٰ رَبُّنَآ أَن يُبْدِلَنَا خَيْرًۭا مِّنْهَآ إِنَّآ إِلَىٰ رَبِّنَا رَٰغِبُونَ ﴿٣٢﴾\\
\textamh{33.\  } & كَذَٟلِكَ ٱلْعَذَابُ ۖ وَلَعَذَابُ ٱلْءَاخِرَةِ أَكْبَرُ ۚ لَوْ كَانُوا۟ يَعْلَمُونَ ﴿٣٣﴾\\
\textamh{34.\  } & إِنَّ لِلْمُتَّقِينَ عِندَ رَبِّهِمْ جَنَّـٰتِ ٱلنَّعِيمِ ﴿٣٤﴾\\
\textamh{35.\  } & أَفَنَجْعَلُ ٱلْمُسْلِمِينَ كَٱلْمُجْرِمِينَ ﴿٣٥﴾\\
\textamh{36.\  } & مَا لَكُمْ كَيْفَ تَحْكُمُونَ ﴿٣٦﴾\\
\textamh{37.\  } & أَمْ لَكُمْ كِتَـٰبٌۭ فِيهِ تَدْرُسُونَ ﴿٣٧﴾\\
\textamh{38.\  } & إِنَّ لَكُمْ فِيهِ لَمَا تَخَيَّرُونَ ﴿٣٨﴾\\
\textamh{39.\  } & أَمْ لَكُمْ أَيْمَـٰنٌ عَلَيْنَا بَٰلِغَةٌ إِلَىٰ يَوْمِ ٱلْقِيَـٰمَةِ ۙ إِنَّ لَكُمْ لَمَا تَحْكُمُونَ ﴿٣٩﴾\\
\textamh{40.\  } & سَلْهُمْ أَيُّهُم بِذَٟلِكَ زَعِيمٌ ﴿٤٠﴾\\
\textamh{41.\  } & أَمْ لَهُمْ شُرَكَآءُ فَلْيَأْتُوا۟ بِشُرَكَآئِهِمْ إِن كَانُوا۟ صَـٰدِقِينَ ﴿٤١﴾\\
\textamh{42.\  } & يَوْمَ يُكْشَفُ عَن سَاقٍۢ وَيُدْعَوْنَ إِلَى ٱلسُّجُودِ فَلَا يَسْتَطِيعُونَ ﴿٤٢﴾\\
\textamh{43.\  } & خَـٰشِعَةً أَبْصَـٰرُهُمْ تَرْهَقُهُمْ ذِلَّةٌۭ ۖ وَقَدْ كَانُوا۟ يُدْعَوْنَ إِلَى ٱلسُّجُودِ وَهُمْ سَـٰلِمُونَ ﴿٤٣﴾\\
\textamh{44.\  } & فَذَرْنِى وَمَن يُكَذِّبُ بِهَـٰذَا ٱلْحَدِيثِ ۖ سَنَسْتَدْرِجُهُم مِّنْ حَيْثُ لَا يَعْلَمُونَ ﴿٤٤﴾\\
\textamh{45.\  } & وَأُمْلِى لَهُمْ ۚ إِنَّ كَيْدِى مَتِينٌ ﴿٤٥﴾\\
\textamh{46.\  } & أَمْ تَسْـَٔلُهُمْ أَجْرًۭا فَهُم مِّن مَّغْرَمٍۢ مُّثْقَلُونَ ﴿٤٦﴾\\
\textamh{47.\  } & أَمْ عِندَهُمُ ٱلْغَيْبُ فَهُمْ يَكْتُبُونَ ﴿٤٧﴾\\
\textamh{48.\  } & فَٱصْبِرْ لِحُكْمِ رَبِّكَ وَلَا تَكُن كَصَاحِبِ ٱلْحُوتِ إِذْ نَادَىٰ وَهُوَ مَكْظُومٌۭ ﴿٤٨﴾\\
\textamh{49.\  } & لَّوْلَآ أَن تَدَٟرَكَهُۥ نِعْمَةٌۭ مِّن رَّبِّهِۦ لَنُبِذَ بِٱلْعَرَآءِ وَهُوَ مَذْمُومٌۭ ﴿٤٩﴾\\
\textamh{50.\  } & فَٱجْتَبَٰهُ رَبُّهُۥ فَجَعَلَهُۥ مِنَ ٱلصَّـٰلِحِينَ ﴿٥٠﴾\\
\textamh{51.\  } & وَإِن يَكَادُ ٱلَّذِينَ كَفَرُوا۟ لَيُزْلِقُونَكَ بِأَبْصَـٰرِهِمْ لَمَّا سَمِعُوا۟ ٱلذِّكْرَ وَيَقُولُونَ إِنَّهُۥ لَمَجْنُونٌۭ ﴿٥١﴾\\
\textamh{52.\  } & وَمَا هُوَ إِلَّا ذِكْرٌۭ لِّلْعَـٰلَمِينَ ﴿٥٢﴾\\
\end{longtable}
\clearpage
%% License: BSD style (Berkley) (i.e. Put the Copyright owner's name always)
%% Writer and Copyright (to): Bewketu(Bilal) Tadilo (2016-17)
\begin{center}\section{\LR{\textamhsec{ሱራቱ አልሀቃ -}  \textarabic{سوره  الحاقة}}}\end{center}
\begin{longtable}{%
  @{}
    p{.5\textwidth}
  @{~~~}
    p{.5\textwidth}
    @{}
}
\textamh{ቢስሚላሂ አራህመኒ ራሂይም } &  \mytextarabic{بِسْمِ ٱللَّهِ ٱلرَّحْمَـٰنِ ٱلرَّحِيمِ}\\
\textamh{1.\  } & \mytextarabic{ ٱلْحَآقَّةُ ﴿١﴾}\\
\textamh{2.\  } & \mytextarabic{مَا ٱلْحَآقَّةُ ﴿٢﴾}\\
\textamh{3.\  } & \mytextarabic{وَمَآ أَدْرَىٰكَ مَا ٱلْحَآقَّةُ ﴿٣﴾}\\
\textamh{4.\  } & \mytextarabic{كَذَّبَتْ ثَمُودُ وَعَادٌۢ بِٱلْقَارِعَةِ ﴿٤﴾}\\
\textamh{5.\  } & \mytextarabic{فَأَمَّا ثَمُودُ فَأُهْلِكُوا۟ بِٱلطَّاغِيَةِ ﴿٥﴾}\\
\textamh{6.\  } & \mytextarabic{وَأَمَّا عَادٌۭ فَأُهْلِكُوا۟ بِرِيحٍۢ صَرْصَرٍ عَاتِيَةٍۢ ﴿٦﴾}\\
\textamh{7.\  } & \mytextarabic{سَخَّرَهَا عَلَيْهِمْ سَبْعَ لَيَالٍۢ وَثَمَـٰنِيَةَ أَيَّامٍ حُسُومًۭا فَتَرَى ٱلْقَوْمَ فِيهَا صَرْعَىٰ كَأَنَّهُمْ أَعْجَازُ نَخْلٍ خَاوِيَةٍۢ ﴿٧﴾}\\
\textamh{8.\  } & \mytextarabic{فَهَلْ تَرَىٰ لَهُم مِّنۢ بَاقِيَةٍۢ ﴿٨﴾}\\
\textamh{9.\  } & \mytextarabic{وَجَآءَ فِرْعَوْنُ وَمَن قَبْلَهُۥ وَٱلْمُؤْتَفِكَـٰتُ بِٱلْخَاطِئَةِ ﴿٩﴾}\\
\textamh{10.\  } & \mytextarabic{فَعَصَوْا۟ رَسُولَ رَبِّهِمْ فَأَخَذَهُمْ أَخْذَةًۭ رَّابِيَةً ﴿١٠﴾}\\
\textamh{11.\  } & \mytextarabic{إِنَّا لَمَّا طَغَا ٱلْمَآءُ حَمَلْنَـٰكُمْ فِى ٱلْجَارِيَةِ ﴿١١﴾}\\
\textamh{12.\  } & \mytextarabic{لِنَجْعَلَهَا لَكُمْ تَذْكِرَةًۭ وَتَعِيَهَآ أُذُنٌۭ وَٟعِيَةٌۭ ﴿١٢﴾}\\
\textamh{13.\  } & \mytextarabic{فَإِذَا نُفِخَ فِى ٱلصُّورِ نَفْخَةٌۭ وَٟحِدَةٌۭ ﴿١٣﴾}\\
\textamh{14.\  } & \mytextarabic{وَحُمِلَتِ ٱلْأَرْضُ وَٱلْجِبَالُ فَدُكَّتَا دَكَّةًۭ وَٟحِدَةًۭ ﴿١٤﴾}\\
\textamh{15.\  } & \mytextarabic{فَيَوْمَئِذٍۢ وَقَعَتِ ٱلْوَاقِعَةُ ﴿١٥﴾}\\
\textamh{16.\  } & \mytextarabic{وَٱنشَقَّتِ ٱلسَّمَآءُ فَهِىَ يَوْمَئِذٍۢ وَاهِيَةٌۭ ﴿١٦﴾}\\
\textamh{17.\  } & \mytextarabic{وَٱلْمَلَكُ عَلَىٰٓ أَرْجَآئِهَا ۚ وَيَحْمِلُ عَرْشَ رَبِّكَ فَوْقَهُمْ يَوْمَئِذٍۢ ثَمَـٰنِيَةٌۭ ﴿١٧﴾}\\
\textamh{18.\  } & \mytextarabic{يَوْمَئِذٍۢ تُعْرَضُونَ لَا تَخْفَىٰ مِنكُمْ خَافِيَةٌۭ ﴿١٨﴾}\\
\textamh{19.\  } & \mytextarabic{فَأَمَّا مَنْ أُوتِىَ كِتَـٰبَهُۥ بِيَمِينِهِۦ فَيَقُولُ هَآؤُمُ ٱقْرَءُوا۟ كِتَـٰبِيَهْ ﴿١٩﴾}\\
\textamh{20.\  } & \mytextarabic{إِنِّى ظَنَنتُ أَنِّى مُلَـٰقٍ حِسَابِيَهْ ﴿٢٠﴾}\\
\textamh{21.\  } & \mytextarabic{فَهُوَ فِى عِيشَةٍۢ رَّاضِيَةٍۢ ﴿٢١﴾}\\
\textamh{22.\  } & \mytextarabic{فِى جَنَّةٍ عَالِيَةٍۢ ﴿٢٢﴾}\\
\textamh{23.\  } & \mytextarabic{قُطُوفُهَا دَانِيَةٌۭ ﴿٢٣﴾}\\
\textamh{24.\  } & \mytextarabic{كُلُوا۟ وَٱشْرَبُوا۟ هَنِيٓـًٔۢا بِمَآ أَسْلَفْتُمْ فِى ٱلْأَيَّامِ ٱلْخَالِيَةِ ﴿٢٤﴾}\\
\textamh{25.\  } & \mytextarabic{وَأَمَّا مَنْ أُوتِىَ كِتَـٰبَهُۥ بِشِمَالِهِۦ فَيَقُولُ يَـٰلَيْتَنِى لَمْ أُوتَ كِتَـٰبِيَهْ ﴿٢٥﴾}\\
\textamh{26.\  } & \mytextarabic{وَلَمْ أَدْرِ مَا حِسَابِيَهْ ﴿٢٦﴾}\\
\textamh{27.\  } & \mytextarabic{يَـٰلَيْتَهَا كَانَتِ ٱلْقَاضِيَةَ ﴿٢٧﴾}\\
\textamh{28.\  } & \mytextarabic{مَآ أَغْنَىٰ عَنِّى مَالِيَهْ ۜ ﴿٢٨﴾}\\
\textamh{29.\  } & \mytextarabic{هَلَكَ عَنِّى سُلْطَٰنِيَهْ ﴿٢٩﴾}\\
\textamh{30.\  } & \mytextarabic{خُذُوهُ فَغُلُّوهُ ﴿٣٠﴾}\\
\textamh{31.\  } & \mytextarabic{ثُمَّ ٱلْجَحِيمَ صَلُّوهُ ﴿٣١﴾}\\
\textamh{32.\  } & \mytextarabic{ثُمَّ فِى سِلْسِلَةٍۢ ذَرْعُهَا سَبْعُونَ ذِرَاعًۭا فَٱسْلُكُوهُ ﴿٣٢﴾}\\
\textamh{33.\  } & \mytextarabic{إِنَّهُۥ كَانَ لَا يُؤْمِنُ بِٱللَّهِ ٱلْعَظِيمِ ﴿٣٣﴾}\\
\textamh{34.\  } & \mytextarabic{وَلَا يَحُضُّ عَلَىٰ طَعَامِ ٱلْمِسْكِينِ ﴿٣٤﴾}\\
\textamh{35.\  } & \mytextarabic{فَلَيْسَ لَهُ ٱلْيَوْمَ هَـٰهُنَا حَمِيمٌۭ ﴿٣٥﴾}\\
\textamh{36.\  } & \mytextarabic{وَلَا طَعَامٌ إِلَّا مِنْ غِسْلِينٍۢ ﴿٣٦﴾}\\
\textamh{37.\  } & \mytextarabic{لَّا يَأْكُلُهُۥٓ إِلَّا ٱلْخَـٰطِـُٔونَ ﴿٣٧﴾}\\
\textamh{38.\  } & \mytextarabic{فَلَآ أُقْسِمُ بِمَا تُبْصِرُونَ ﴿٣٨﴾}\\
\textamh{39.\  } & \mytextarabic{وَمَا لَا تُبْصِرُونَ ﴿٣٩﴾}\\
\textamh{40.\  } & \mytextarabic{إِنَّهُۥ لَقَوْلُ رَسُولٍۢ كَرِيمٍۢ ﴿٤٠﴾}\\
\textamh{41.\  } & \mytextarabic{وَمَا هُوَ بِقَوْلِ شَاعِرٍۢ ۚ قَلِيلًۭا مَّا تُؤْمِنُونَ ﴿٤١﴾}\\
\textamh{42.\  } & \mytextarabic{وَلَا بِقَوْلِ كَاهِنٍۢ ۚ قَلِيلًۭا مَّا تَذَكَّرُونَ ﴿٤٢﴾}\\
\textamh{43.\  } & \mytextarabic{تَنزِيلٌۭ مِّن رَّبِّ ٱلْعَـٰلَمِينَ ﴿٤٣﴾}\\
\textamh{44.\  } & \mytextarabic{وَلَوْ تَقَوَّلَ عَلَيْنَا بَعْضَ ٱلْأَقَاوِيلِ ﴿٤٤﴾}\\
\textamh{45.\  } & \mytextarabic{لَأَخَذْنَا مِنْهُ بِٱلْيَمِينِ ﴿٤٥﴾}\\
\textamh{46.\  } & \mytextarabic{ثُمَّ لَقَطَعْنَا مِنْهُ ٱلْوَتِينَ ﴿٤٦﴾}\\
\textamh{47.\  } & \mytextarabic{فَمَا مِنكُم مِّنْ أَحَدٍ عَنْهُ حَـٰجِزِينَ ﴿٤٧﴾}\\
\textamh{48.\  } & \mytextarabic{وَإِنَّهُۥ لَتَذْكِرَةٌۭ لِّلْمُتَّقِينَ ﴿٤٨﴾}\\
\textamh{49.\  } & \mytextarabic{وَإِنَّا لَنَعْلَمُ أَنَّ مِنكُم مُّكَذِّبِينَ ﴿٤٩﴾}\\
\textamh{50.\  } & \mytextarabic{وَإِنَّهُۥ لَحَسْرَةٌ عَلَى ٱلْكَـٰفِرِينَ ﴿٥٠﴾}\\
\textamh{51.\  } & \mytextarabic{وَإِنَّهُۥ لَحَقُّ ٱلْيَقِينِ ﴿٥١﴾}\\
\textamh{52.\  } & \mytextarabic{فَسَبِّحْ بِٱسْمِ رَبِّكَ ٱلْعَظِيمِ ﴿٥٢﴾}\\
\end{longtable}
\clearpage
%% License: BSD style (Berkley) (i.e. Put the Copyright owner's name always)
%% Writer and Copyright (to): Bewketu(Bilal) Tadilo (2016-17)
\centering\section{\LR{\textamharic{ሱራቱ አልመኣሪጅ -}  \RL{سوره  المعارج}}}
\begin{longtable}{%
  @{}
    p{.5\textwidth}
  @{~~~~~~~~~~~~}
    p{.5\textwidth}
    @{}
}
\nopagebreak
\textamh{ቢስሚላሂ አራህመኒ ራሂይም } &  بِسْمِ ٱللَّهِ ٱلرَّحْمَـٰنِ ٱلرَّحِيمِ\\
\textamh{1.\  } &  سَأَلَ سَآئِلٌۢ بِعَذَابٍۢ وَاقِعٍۢ ﴿١﴾\\
\textamh{2.\  } & لِّلْكَـٰفِرِينَ لَيْسَ لَهُۥ دَافِعٌۭ ﴿٢﴾\\
\textamh{3.\  } & مِّنَ ٱللَّهِ ذِى ٱلْمَعَارِجِ ﴿٣﴾\\
\textamh{4.\  } & تَعْرُجُ ٱلْمَلَـٰٓئِكَةُ وَٱلرُّوحُ إِلَيْهِ فِى يَوْمٍۢ كَانَ مِقْدَارُهُۥ خَمْسِينَ أَلْفَ سَنَةٍۢ ﴿٤﴾\\
\textamh{5.\  } & فَٱصْبِرْ صَبْرًۭا جَمِيلًا ﴿٥﴾\\
\textamh{6.\  } & إِنَّهُمْ يَرَوْنَهُۥ بَعِيدًۭا ﴿٦﴾\\
\textamh{7.\  } & وَنَرَىٰهُ قَرِيبًۭا ﴿٧﴾\\
\textamh{8.\  } & يَوْمَ تَكُونُ ٱلسَّمَآءُ كَٱلْمُهْلِ ﴿٨﴾\\
\textamh{9.\  } & وَتَكُونُ ٱلْجِبَالُ كَٱلْعِهْنِ ﴿٩﴾\\
\textamh{10.\  } & وَلَا يَسْـَٔلُ حَمِيمٌ حَمِيمًۭا ﴿١٠﴾\\
\textamh{11.\  } & يُبَصَّرُونَهُمْ ۚ يَوَدُّ ٱلْمُجْرِمُ لَوْ يَفْتَدِى مِنْ عَذَابِ يَوْمِئِذٍۭ بِبَنِيهِ ﴿١١﴾\\
\textamh{12.\  } & وَصَـٰحِبَتِهِۦ وَأَخِيهِ ﴿١٢﴾\\
\textamh{13.\  } & وَفَصِيلَتِهِ ٱلَّتِى تُـْٔوِيهِ ﴿١٣﴾\\
\textamh{14.\  } & وَمَن فِى ٱلْأَرْضِ جَمِيعًۭا ثُمَّ يُنجِيهِ ﴿١٤﴾\\
\textamh{15.\  } & كَلَّآ ۖ إِنَّهَا لَظَىٰ ﴿١٥﴾\\
\textamh{16.\  } & نَزَّاعَةًۭ لِّلشَّوَىٰ ﴿١٦﴾\\
\textamh{17.\  } & تَدْعُوا۟ مَنْ أَدْبَرَ وَتَوَلَّىٰ ﴿١٧﴾\\
\textamh{18.\  } & وَجَمَعَ فَأَوْعَىٰٓ ﴿١٨﴾\\
\textamh{19.\  } & ۞ إِنَّ ٱلْإِنسَـٰنَ خُلِقَ هَلُوعًا ﴿١٩﴾\\
\textamh{20.\  } & إِذَا مَسَّهُ ٱلشَّرُّ جَزُوعًۭا ﴿٢٠﴾\\
\textamh{21.\  } & وَإِذَا مَسَّهُ ٱلْخَيْرُ مَنُوعًا ﴿٢١﴾\\
\textamh{22.\  } & إِلَّا ٱلْمُصَلِّينَ ﴿٢٢﴾\\
\textamh{23.\  } & ٱلَّذِينَ هُمْ عَلَىٰ صَلَاتِهِمْ دَآئِمُونَ ﴿٢٣﴾\\
\textamh{24.\  } & وَٱلَّذِينَ فِىٓ أَمْوَٟلِهِمْ حَقٌّۭ مَّعْلُومٌۭ ﴿٢٤﴾\\
\textamh{25.\  } & لِّلسَّآئِلِ وَٱلْمَحْرُومِ ﴿٢٥﴾\\
\textamh{26.\  } & وَٱلَّذِينَ يُصَدِّقُونَ بِيَوْمِ ٱلدِّينِ ﴿٢٦﴾\\
\textamh{27.\  } & وَٱلَّذِينَ هُم مِّنْ عَذَابِ رَبِّهِم مُّشْفِقُونَ ﴿٢٧﴾\\
\textamh{28.\  } & إِنَّ عَذَابَ رَبِّهِمْ غَيْرُ مَأْمُونٍۢ ﴿٢٨﴾\\
\textamh{29.\  } & وَٱلَّذِينَ هُمْ لِفُرُوجِهِمْ حَـٰفِظُونَ ﴿٢٩﴾\\
\textamh{30.\  } & إِلَّا عَلَىٰٓ أَزْوَٟجِهِمْ أَوْ مَا مَلَكَتْ أَيْمَـٰنُهُمْ فَإِنَّهُمْ غَيْرُ مَلُومِينَ ﴿٣٠﴾\\
\textamh{31.\  } & فَمَنِ ٱبْتَغَىٰ وَرَآءَ ذَٟلِكَ فَأُو۟لَـٰٓئِكَ هُمُ ٱلْعَادُونَ ﴿٣١﴾\\
\textamh{32.\  } & وَٱلَّذِينَ هُمْ لِأَمَـٰنَـٰتِهِمْ وَعَهْدِهِمْ رَٰعُونَ ﴿٣٢﴾\\
\textamh{33.\  } & وَٱلَّذِينَ هُم بِشَهَـٰدَٟتِهِمْ قَآئِمُونَ ﴿٣٣﴾\\
\textamh{34.\  } & وَٱلَّذِينَ هُمْ عَلَىٰ صَلَاتِهِمْ يُحَافِظُونَ ﴿٣٤﴾\\
\textamh{35.\  } & أُو۟لَـٰٓئِكَ فِى جَنَّـٰتٍۢ مُّكْرَمُونَ ﴿٣٥﴾\\
\textamh{36.\  } & فَمَالِ ٱلَّذِينَ كَفَرُوا۟ قِبَلَكَ مُهْطِعِينَ ﴿٣٦﴾\\
\textamh{37.\  } & عَنِ ٱلْيَمِينِ وَعَنِ ٱلشِّمَالِ عِزِينَ ﴿٣٧﴾\\
\textamh{38.\  } & أَيَطْمَعُ كُلُّ ٱمْرِئٍۢ مِّنْهُمْ أَن يُدْخَلَ جَنَّةَ نَعِيمٍۢ ﴿٣٨﴾\\
\textamh{39.\  } & كَلَّآ ۖ إِنَّا خَلَقْنَـٰهُم مِّمَّا يَعْلَمُونَ ﴿٣٩﴾\\
\textamh{40.\  } & فَلَآ أُقْسِمُ بِرَبِّ ٱلْمَشَـٰرِقِ وَٱلْمَغَٰرِبِ إِنَّا لَقَـٰدِرُونَ ﴿٤٠﴾\\
\textamh{41.\  } & عَلَىٰٓ أَن نُّبَدِّلَ خَيْرًۭا مِّنْهُمْ وَمَا نَحْنُ بِمَسْبُوقِينَ ﴿٤١﴾\\
\textamh{42.\  } & فَذَرْهُمْ يَخُوضُوا۟ وَيَلْعَبُوا۟ حَتَّىٰ يُلَـٰقُوا۟ يَوْمَهُمُ ٱلَّذِى يُوعَدُونَ ﴿٤٢﴾\\
\textamh{43.\  } & يَوْمَ يَخْرُجُونَ مِنَ ٱلْأَجْدَاثِ سِرَاعًۭا كَأَنَّهُمْ إِلَىٰ نُصُبٍۢ يُوفِضُونَ ﴿٤٣﴾\\
\textamh{44.\  } & خَـٰشِعَةً أَبْصَـٰرُهُمْ تَرْهَقُهُمْ ذِلَّةٌۭ ۚ ذَٟلِكَ ٱلْيَوْمُ ٱلَّذِى كَانُوا۟ يُوعَدُونَ ﴿٤٤﴾\\
\end{longtable}
\clearpage
%% License: BSD style (Berkley) (i.e. Put the Copyright owner's name always)
%% Writer and Copyright (to): Bewketu(Bilal) Tadilo (2016-17)
\centering\section{\LR{\textamharic{ሱራቱ ኑህ -}  \RL{سوره  نوح}}}
\begin{longtable}{%
  @{}
    p{.5\textwidth}
  @{~~~~~~~~~~~~~}
    p{.5\textwidth}
    @{}
}
\nopagebreak
\textamh{ቢስሚላሂ አራህመኒ ራሂይም } &  بِسْمِ ٱللَّهِ ٱلرَّحْمَـٰنِ ٱلرَّحِيمِ\\
\textamh{1.\  } &  إِنَّآ أَرْسَلْنَا نُوحًا إِلَىٰ قَوْمِهِۦٓ أَنْ أَنذِرْ قَوْمَكَ مِن قَبْلِ أَن يَأْتِيَهُمْ عَذَابٌ أَلِيمٌۭ ﴿١﴾\\
\textamh{2.\  } & قَالَ يَـٰقَوْمِ إِنِّى لَكُمْ نَذِيرٌۭ مُّبِينٌ ﴿٢﴾\\
\textamh{3.\  } & أَنِ ٱعْبُدُوا۟ ٱللَّهَ وَٱتَّقُوهُ وَأَطِيعُونِ ﴿٣﴾\\
\textamh{4.\  } & يَغْفِرْ لَكُم مِّن ذُنُوبِكُمْ وَيُؤَخِّرْكُمْ إِلَىٰٓ أَجَلٍۢ مُّسَمًّى ۚ إِنَّ أَجَلَ ٱللَّهِ إِذَا جَآءَ لَا يُؤَخَّرُ ۖ لَوْ كُنتُمْ تَعْلَمُونَ ﴿٤﴾\\
\textamh{5.\  } & قَالَ رَبِّ إِنِّى دَعَوْتُ قَوْمِى لَيْلًۭا وَنَهَارًۭا ﴿٥﴾\\
\textamh{6.\  } & فَلَمْ يَزِدْهُمْ دُعَآءِىٓ إِلَّا فِرَارًۭا ﴿٦﴾\\
\textamh{7.\  } & وَإِنِّى كُلَّمَا دَعَوْتُهُمْ لِتَغْفِرَ لَهُمْ جَعَلُوٓا۟ أَصَـٰبِعَهُمْ فِىٓ ءَاذَانِهِمْ وَٱسْتَغْشَوْا۟ ثِيَابَهُمْ وَأَصَرُّوا۟ وَٱسْتَكْبَرُوا۟ ٱسْتِكْبَارًۭا ﴿٧﴾\\
\textamh{8.\  } & ثُمَّ إِنِّى دَعَوْتُهُمْ جِهَارًۭا ﴿٨﴾\\
\textamh{9.\  } & ثُمَّ إِنِّىٓ أَعْلَنتُ لَهُمْ وَأَسْرَرْتُ لَهُمْ إِسْرَارًۭا ﴿٩﴾\\
\textamh{10.\  } & فَقُلْتُ ٱسْتَغْفِرُوا۟ رَبَّكُمْ إِنَّهُۥ كَانَ غَفَّارًۭا ﴿١٠﴾\\
\textamh{11.\  } & يُرْسِلِ ٱلسَّمَآءَ عَلَيْكُم مِّدْرَارًۭا ﴿١١﴾\\
\textamh{12.\  } & وَيُمْدِدْكُم بِأَمْوَٟلٍۢ وَبَنِينَ وَيَجْعَل لَّكُمْ جَنَّـٰتٍۢ وَيَجْعَل لَّكُمْ أَنْهَـٰرًۭا ﴿١٢﴾\\
\textamh{13.\  } & مَّا لَكُمْ لَا تَرْجُونَ لِلَّهِ وَقَارًۭا ﴿١٣﴾\\
\textamh{14.\  } & وَقَدْ خَلَقَكُمْ أَطْوَارًا ﴿١٤﴾\\
\textamh{15.\  } & أَلَمْ تَرَوْا۟ كَيْفَ خَلَقَ ٱللَّهُ سَبْعَ سَمَـٰوَٟتٍۢ طِبَاقًۭا ﴿١٥﴾\\
\textamh{16.\  } & وَجَعَلَ ٱلْقَمَرَ فِيهِنَّ نُورًۭا وَجَعَلَ ٱلشَّمْسَ سِرَاجًۭا ﴿١٦﴾\\
\textamh{17.\  } & وَٱللَّهُ أَنۢبَتَكُم مِّنَ ٱلْأَرْضِ نَبَاتًۭا ﴿١٧﴾\\
\textamh{18.\  } & ثُمَّ يُعِيدُكُمْ فِيهَا وَيُخْرِجُكُمْ إِخْرَاجًۭا ﴿١٨﴾\\
\textamh{19.\  } & وَٱللَّهُ جَعَلَ لَكُمُ ٱلْأَرْضَ بِسَاطًۭا ﴿١٩﴾\\
\textamh{20.\  } & لِّتَسْلُكُوا۟ مِنْهَا سُبُلًۭا فِجَاجًۭا ﴿٢٠﴾\\
\textamh{21.\  } & قَالَ نُوحٌۭ رَّبِّ إِنَّهُمْ عَصَوْنِى وَٱتَّبَعُوا۟ مَن لَّمْ يَزِدْهُ مَالُهُۥ وَوَلَدُهُۥٓ إِلَّا خَسَارًۭا ﴿٢١﴾\\
\textamh{22.\  } & وَمَكَرُوا۟ مَكْرًۭا كُبَّارًۭا ﴿٢٢﴾\\
\textamh{23.\  } & وَقَالُوا۟ لَا تَذَرُنَّ ءَالِهَتَكُمْ وَلَا تَذَرُنَّ وَدًّۭا وَلَا سُوَاعًۭا وَلَا يَغُوثَ وَيَعُوقَ وَنَسْرًۭا ﴿٢٣﴾\\
\textamh{24.\  } & وَقَدْ أَضَلُّوا۟ كَثِيرًۭا ۖ وَلَا تَزِدِ ٱلظَّـٰلِمِينَ إِلَّا ضَلَـٰلًۭا ﴿٢٤﴾\\
\textamh{25.\  } & مِّمَّا خَطِيٓـَٰٔتِهِمْ أُغْرِقُوا۟ فَأُدْخِلُوا۟ نَارًۭا فَلَمْ يَجِدُوا۟ لَهُم مِّن دُونِ ٱللَّهِ أَنصَارًۭا ﴿٢٥﴾\\
\textamh{26.\  } & وَقَالَ نُوحٌۭ رَّبِّ لَا تَذَرْ عَلَى ٱلْأَرْضِ مِنَ ٱلْكَـٰفِرِينَ دَيَّارًا ﴿٢٦﴾\\
\textamh{27.\  } & إِنَّكَ إِن تَذَرْهُمْ يُضِلُّوا۟ عِبَادَكَ وَلَا يَلِدُوٓا۟ إِلَّا فَاجِرًۭا كَفَّارًۭا ﴿٢٧﴾\\
\textamh{28.\  } & رَّبِّ ٱغْفِرْ لِى وَلِوَٟلِدَىَّ وَلِمَن دَخَلَ بَيْتِىَ مُؤْمِنًۭا وَلِلْمُؤْمِنِينَ وَٱلْمُؤْمِنَـٰتِ وَلَا تَزِدِ ٱلظَّـٰلِمِينَ إِلَّا تَبَارًۢا ﴿٢٨﴾\\
\end{longtable}
\clearpage
%% License: BSD style (Berkley) (i.e. Put the Copyright owner's name always)
%% Writer and Copyright (to): Bewketu(Bilal) Tadilo (2016-17)
\centering\section{\LR{\textamharic{ሱራቱ አልጂን -}  \RL{سوره  الجن}}}
\begin{longtable}{%
  @{}
    p{.5\textwidth}
  @{~~~~~~~~~~~~~}
    p{.5\textwidth}
    @{}
}
\nopagebreak
\textamh{ቢስሚላሂ አራህመኒ ራሂይም } &  بِسْمِ ٱللَّهِ ٱلرَّحْمَـٰنِ ٱلرَّحِيمِ\\
\textamh{1.\  } &  قُلْ أُوحِىَ إِلَىَّ أَنَّهُ ٱسْتَمَعَ نَفَرٌۭ مِّنَ ٱلْجِنِّ فَقَالُوٓا۟ إِنَّا سَمِعْنَا قُرْءَانًا عَجَبًۭا ﴿١﴾\\
\textamh{2.\  } & يَهْدِىٓ إِلَى ٱلرُّشْدِ فَـَٔامَنَّا بِهِۦ ۖ وَلَن نُّشْرِكَ بِرَبِّنَآ أَحَدًۭا ﴿٢﴾\\
\textamh{3.\  } & وَأَنَّهُۥ تَعَـٰلَىٰ جَدُّ رَبِّنَا مَا ٱتَّخَذَ صَـٰحِبَةًۭ وَلَا وَلَدًۭا ﴿٣﴾\\
\textamh{4.\  } & وَأَنَّهُۥ كَانَ يَقُولُ سَفِيهُنَا عَلَى ٱللَّهِ شَطَطًۭا ﴿٤﴾\\
\textamh{5.\  } & وَأَنَّا ظَنَنَّآ أَن لَّن تَقُولَ ٱلْإِنسُ وَٱلْجِنُّ عَلَى ٱللَّهِ كَذِبًۭا ﴿٥﴾\\
\textamh{6.\  } & وَأَنَّهُۥ كَانَ رِجَالٌۭ مِّنَ ٱلْإِنسِ يَعُوذُونَ بِرِجَالٍۢ مِّنَ ٱلْجِنِّ فَزَادُوهُمْ رَهَقًۭا ﴿٦﴾\\
\textamh{7.\  } & وَأَنَّهُمْ ظَنُّوا۟ كَمَا ظَنَنتُمْ أَن لَّن يَبْعَثَ ٱللَّهُ أَحَدًۭا ﴿٧﴾\\
\textamh{8.\  } & وَأَنَّا لَمَسْنَا ٱلسَّمَآءَ فَوَجَدْنَـٰهَا مُلِئَتْ حَرَسًۭا شَدِيدًۭا وَشُهُبًۭا ﴿٨﴾\\
\textamh{9.\  } & وَأَنَّا كُنَّا نَقْعُدُ مِنْهَا مَقَـٰعِدَ لِلسَّمْعِ ۖ فَمَن يَسْتَمِعِ ٱلْءَانَ يَجِدْ لَهُۥ شِهَابًۭا رَّصَدًۭا ﴿٩﴾\\
\textamh{10.\  } & وَأَنَّا لَا نَدْرِىٓ أَشَرٌّ أُرِيدَ بِمَن فِى ٱلْأَرْضِ أَمْ أَرَادَ بِهِمْ رَبُّهُمْ رَشَدًۭا ﴿١٠﴾\\
\textamh{11.\  } & وَأَنَّا مِنَّا ٱلصَّـٰلِحُونَ وَمِنَّا دُونَ ذَٟلِكَ ۖ كُنَّا طَرَآئِقَ قِدَدًۭا ﴿١١﴾\\
\textamh{12.\  } & وَأَنَّا ظَنَنَّآ أَن لَّن نُّعْجِزَ ٱللَّهَ فِى ٱلْأَرْضِ وَلَن نُّعْجِزَهُۥ هَرَبًۭا ﴿١٢﴾\\
\textamh{13.\  } & وَأَنَّا لَمَّا سَمِعْنَا ٱلْهُدَىٰٓ ءَامَنَّا بِهِۦ ۖ فَمَن يُؤْمِنۢ بِرَبِّهِۦ فَلَا يَخَافُ بَخْسًۭا وَلَا رَهَقًۭا ﴿١٣﴾\\
\textamh{14.\  } & وَأَنَّا مِنَّا ٱلْمُسْلِمُونَ وَمِنَّا ٱلْقَـٰسِطُونَ ۖ فَمَنْ أَسْلَمَ فَأُو۟لَـٰٓئِكَ تَحَرَّوْا۟ رَشَدًۭا ﴿١٤﴾\\
\textamh{15.\  } & وَأَمَّا ٱلْقَـٰسِطُونَ فَكَانُوا۟ لِجَهَنَّمَ حَطَبًۭا ﴿١٥﴾\\
\textamh{16.\  } & وَأَلَّوِ ٱسْتَقَـٰمُوا۟ عَلَى ٱلطَّرِيقَةِ لَأَسْقَيْنَـٰهُم مَّآءً غَدَقًۭا ﴿١٦﴾\\
\textamh{17.\  } & لِّنَفْتِنَهُمْ فِيهِ ۚ وَمَن يُعْرِضْ عَن ذِكْرِ رَبِّهِۦ يَسْلُكْهُ عَذَابًۭا صَعَدًۭا ﴿١٧﴾\\
\textamh{18.\  } & وَأَنَّ ٱلْمَسَـٰجِدَ لِلَّهِ فَلَا تَدْعُوا۟ مَعَ ٱللَّهِ أَحَدًۭا ﴿١٨﴾\\
\textamh{19.\  } & وَأَنَّهُۥ لَمَّا قَامَ عَبْدُ ٱللَّهِ يَدْعُوهُ كَادُوا۟ يَكُونُونَ عَلَيْهِ لِبَدًۭا ﴿١٩﴾\\
\textamh{20.\  } & قُلْ إِنَّمَآ أَدْعُوا۟ رَبِّى وَلَآ أُشْرِكُ بِهِۦٓ أَحَدًۭا ﴿٢٠﴾\\
\textamh{21.\  } & قُلْ إِنِّى لَآ أَمْلِكُ لَكُمْ ضَرًّۭا وَلَا رَشَدًۭا ﴿٢١﴾\\
\textamh{22.\  } & قُلْ إِنِّى لَن يُجِيرَنِى مِنَ ٱللَّهِ أَحَدٌۭ وَلَنْ أَجِدَ مِن دُونِهِۦ مُلْتَحَدًا ﴿٢٢﴾\\
\textamh{23.\  } & إِلَّا بَلَـٰغًۭا مِّنَ ٱللَّهِ وَرِسَـٰلَـٰتِهِۦ ۚ وَمَن يَعْصِ ٱللَّهَ وَرَسُولَهُۥ فَإِنَّ لَهُۥ نَارَ جَهَنَّمَ خَـٰلِدِينَ فِيهَآ أَبَدًا ﴿٢٣﴾\\
\textamh{24.\  } & حَتَّىٰٓ إِذَا رَأَوْا۟ مَا يُوعَدُونَ فَسَيَعْلَمُونَ مَنْ أَضْعَفُ نَاصِرًۭا وَأَقَلُّ عَدَدًۭا ﴿٢٤﴾\\
\textamh{25.\  } & قُلْ إِنْ أَدْرِىٓ أَقَرِيبٌۭ مَّا تُوعَدُونَ أَمْ يَجْعَلُ لَهُۥ رَبِّىٓ أَمَدًا ﴿٢٥﴾\\
\textamh{26.\  } & عَـٰلِمُ ٱلْغَيْبِ فَلَا يُظْهِرُ عَلَىٰ غَيْبِهِۦٓ أَحَدًا ﴿٢٦﴾\\
\textamh{27.\  } & إِلَّا مَنِ ٱرْتَضَىٰ مِن رَّسُولٍۢ فَإِنَّهُۥ يَسْلُكُ مِنۢ بَيْنِ يَدَيْهِ وَمِنْ خَلْفِهِۦ رَصَدًۭا ﴿٢٧﴾\\
\textamh{28.\  } & لِّيَعْلَمَ أَن قَدْ أَبْلَغُوا۟ رِسَـٰلَـٰتِ رَبِّهِمْ وَأَحَاطَ بِمَا لَدَيْهِمْ وَأَحْصَىٰ كُلَّ شَىْءٍ عَدَدًۢا ﴿٢٨﴾\\
\end{longtable}
\clearpage
%% License: BSD style (Berkley) (i.e. Put the Copyright owner's name always)
%% Writer and Copyright (to): Bewketu(Bilal) Tadilo (2016-17)
\centering\section{\LR{\textamharic{ሱራቱ አልሙዘሚል -}  \RL{سوره  المزمل}}}
\begin{longtable}{%
  @{}
    p{.5\textwidth}
  @{~~~~~~~~~~~~~}
    p{.5\textwidth}
    @{}
}
\nopagebreak
\textamh{ቢስሚላሂ አራህመኒ ራሂይም } &  بِسْمِ ٱللَّهِ ٱلرَّحْمَـٰنِ ٱلرَّحِيمِ\\
\textamh{1.\  } &  يَـٰٓأَيُّهَا ٱلْمُزَّمِّلُ ﴿١﴾\\
\textamh{2.\  } & قُمِ ٱلَّيْلَ إِلَّا قَلِيلًۭا ﴿٢﴾\\
\textamh{3.\  } & نِّصْفَهُۥٓ أَوِ ٱنقُصْ مِنْهُ قَلِيلًا ﴿٣﴾\\
\textamh{4.\  } & أَوْ زِدْ عَلَيْهِ وَرَتِّلِ ٱلْقُرْءَانَ تَرْتِيلًا ﴿٤﴾\\
\textamh{5.\  } & إِنَّا سَنُلْقِى عَلَيْكَ قَوْلًۭا ثَقِيلًا ﴿٥﴾\\
\textamh{6.\  } & إِنَّ نَاشِئَةَ ٱلَّيْلِ هِىَ أَشَدُّ وَطْـًۭٔا وَأَقْوَمُ قِيلًا ﴿٦﴾\\
\textamh{7.\  } & إِنَّ لَكَ فِى ٱلنَّهَارِ سَبْحًۭا طَوِيلًۭا ﴿٧﴾\\
\textamh{8.\  } & وَٱذْكُرِ ٱسْمَ رَبِّكَ وَتَبَتَّلْ إِلَيْهِ تَبْتِيلًۭا ﴿٨﴾\\
\textamh{9.\  } & رَّبُّ ٱلْمَشْرِقِ وَٱلْمَغْرِبِ لَآ إِلَـٰهَ إِلَّا هُوَ فَٱتَّخِذْهُ وَكِيلًۭا ﴿٩﴾\\
\textamh{10.\  } & وَٱصْبِرْ عَلَىٰ مَا يَقُولُونَ وَٱهْجُرْهُمْ هَجْرًۭا جَمِيلًۭا ﴿١٠﴾\\
\textamh{11.\  } & وَذَرْنِى وَٱلْمُكَذِّبِينَ أُو۟لِى ٱلنَّعْمَةِ وَمَهِّلْهُمْ قَلِيلًا ﴿١١﴾\\
\textamh{12.\  } & إِنَّ لَدَيْنَآ أَنكَالًۭا وَجَحِيمًۭا ﴿١٢﴾\\
\textamh{13.\  } & وَطَعَامًۭا ذَا غُصَّةٍۢ وَعَذَابًا أَلِيمًۭا ﴿١٣﴾\\
\textamh{14.\  } & يَوْمَ تَرْجُفُ ٱلْأَرْضُ وَٱلْجِبَالُ وَكَانَتِ ٱلْجِبَالُ كَثِيبًۭا مَّهِيلًا ﴿١٤﴾\\
\textamh{15.\  } & إِنَّآ أَرْسَلْنَآ إِلَيْكُمْ رَسُولًۭا شَـٰهِدًا عَلَيْكُمْ كَمَآ أَرْسَلْنَآ إِلَىٰ فِرْعَوْنَ رَسُولًۭا ﴿١٥﴾\\
\textamh{16.\  } & فَعَصَىٰ فِرْعَوْنُ ٱلرَّسُولَ فَأَخَذْنَـٰهُ أَخْذًۭا وَبِيلًۭا ﴿١٦﴾\\
\textamh{17.\  } & فَكَيْفَ تَتَّقُونَ إِن كَفَرْتُمْ يَوْمًۭا يَجْعَلُ ٱلْوِلْدَٟنَ شِيبًا ﴿١٧﴾\\
\textamh{18.\  } & ٱلسَّمَآءُ مُنفَطِرٌۢ بِهِۦ ۚ كَانَ وَعْدُهُۥ مَفْعُولًا ﴿١٨﴾\\
\textamh{19.\  } & إِنَّ هَـٰذِهِۦ تَذْكِرَةٌۭ ۖ فَمَن شَآءَ ٱتَّخَذَ إِلَىٰ رَبِّهِۦ سَبِيلًا ﴿١٩﴾\\
\textamh{20.\  } & ۞ إِنَّ رَبَّكَ يَعْلَمُ أَنَّكَ تَقُومُ أَدْنَىٰ مِن ثُلُثَىِ ٱلَّيْلِ وَنِصْفَهُۥ وَثُلُثَهُۥ وَطَآئِفَةٌۭ مِّنَ ٱلَّذِينَ مَعَكَ ۚ وَٱللَّهُ يُقَدِّرُ ٱلَّيْلَ وَٱلنَّهَارَ ۚ عَلِمَ أَن لَّن تُحْصُوهُ فَتَابَ عَلَيْكُمْ ۖ فَٱقْرَءُوا۟ مَا تَيَسَّرَ مِنَ ٱلْقُرْءَانِ ۚ عَلِمَ أَن سَيَكُونُ مِنكُم مَّرْضَىٰ ۙ وَءَاخَرُونَ يَضْرِبُونَ فِى ٱلْأَرْضِ يَبْتَغُونَ مِن فَضْلِ ٱللَّهِ ۙ وَءَاخَرُونَ يُقَـٰتِلُونَ فِى سَبِيلِ ٱللَّهِ ۖ فَٱقْرَءُوا۟ مَا تَيَسَّرَ مِنْهُ ۚ وَأَقِيمُوا۟ ٱلصَّلَوٰةَ وَءَاتُوا۟ ٱلزَّكَوٰةَ وَأَقْرِضُوا۟ ٱللَّهَ قَرْضًا حَسَنًۭا ۚ وَمَا تُقَدِّمُوا۟ لِأَنفُسِكُم مِّنْ خَيْرٍۢ تَجِدُوهُ عِندَ ٱللَّهِ هُوَ خَيْرًۭا وَأَعْظَمَ أَجْرًۭا ۚ وَٱسْتَغْفِرُوا۟ ٱللَّهَ ۖ إِنَّ ٱللَّهَ غَفُورٌۭ رَّحِيمٌۢ ﴿٢٠﴾\\
\end{longtable}
\clearpage
%% License: BSD style (Berkley) (i.e. Put the Copyright owner's name always)
%% Writer and Copyright (to): Bewketu(Bilal) Tadilo (2016-17)
\centering\section{\LR{\textamharic{ሱራቱ አልሙደቲር -}  \RL{سوره  المدثر}}}
\begin{longtable}{%
  @{}
    p{.5\textwidth}
  @{~~~~~~~~~~~~}
    p{.5\textwidth}
    @{}
}
\nopagebreak
\textamh{ቢስሚላሂ አራህመኒ ራሂይም } &  بِسْمِ ٱللَّهِ ٱلرَّحْمَـٰنِ ٱلرَّحِيمِ\\
\textamh{1.\  } &  يَـٰٓأَيُّهَا ٱلْمُدَّثِّرُ ﴿١﴾\\
\textamh{2.\  } & قُمْ فَأَنذِرْ ﴿٢﴾\\
\textamh{3.\  } & وَرَبَّكَ فَكَبِّرْ ﴿٣﴾\\
\textamh{4.\  } & وَثِيَابَكَ فَطَهِّرْ ﴿٤﴾\\
\textamh{5.\  } & وَٱلرُّجْزَ فَٱهْجُرْ ﴿٥﴾\\
\textamh{6.\  } & وَلَا تَمْنُن تَسْتَكْثِرُ ﴿٦﴾\\
\textamh{7.\  } & وَلِرَبِّكَ فَٱصْبِرْ ﴿٧﴾\\
\textamh{8.\  } & فَإِذَا نُقِرَ فِى ٱلنَّاقُورِ ﴿٨﴾\\
\textamh{9.\  } & فَذَٟلِكَ يَوْمَئِذٍۢ يَوْمٌ عَسِيرٌ ﴿٩﴾\\
\textamh{10.\  } & عَلَى ٱلْكَـٰفِرِينَ غَيْرُ يَسِيرٍۢ ﴿١٠﴾\\
\textamh{11.\  } & ذَرْنِى وَمَنْ خَلَقْتُ وَحِيدًۭا ﴿١١﴾\\
\textamh{12.\  } & وَجَعَلْتُ لَهُۥ مَالًۭا مَّمْدُودًۭا ﴿١٢﴾\\
\textamh{13.\  } & وَبَنِينَ شُهُودًۭا ﴿١٣﴾\\
\textamh{14.\  } & وَمَهَّدتُّ لَهُۥ تَمْهِيدًۭا ﴿١٤﴾\\
\textamh{15.\  } & ثُمَّ يَطْمَعُ أَنْ أَزِيدَ ﴿١٥﴾\\
\textamh{16.\  } & كَلَّآ ۖ إِنَّهُۥ كَانَ لِءَايَـٰتِنَا عَنِيدًۭا ﴿١٦﴾\\
\textamh{17.\  } & سَأُرْهِقُهُۥ صَعُودًا ﴿١٧﴾\\
\textamh{18.\  } & إِنَّهُۥ فَكَّرَ وَقَدَّرَ ﴿١٨﴾\\
\textamh{19.\  } & فَقُتِلَ كَيْفَ قَدَّرَ ﴿١٩﴾\\
\textamh{20.\  } & ثُمَّ قُتِلَ كَيْفَ قَدَّرَ ﴿٢٠﴾\\
\textamh{21.\  } & ثُمَّ نَظَرَ ﴿٢١﴾\\
\textamh{22.\  } & ثُمَّ عَبَسَ وَبَسَرَ ﴿٢٢﴾\\
\textamh{23.\  } & ثُمَّ أَدْبَرَ وَٱسْتَكْبَرَ ﴿٢٣﴾\\
\textamh{24.\  } & فَقَالَ إِنْ هَـٰذَآ إِلَّا سِحْرٌۭ يُؤْثَرُ ﴿٢٤﴾\\
\textamh{25.\  } & إِنْ هَـٰذَآ إِلَّا قَوْلُ ٱلْبَشَرِ ﴿٢٥﴾\\
\textamh{26.\  } & سَأُصْلِيهِ سَقَرَ ﴿٢٦﴾\\
\textamh{27.\  } & وَمَآ أَدْرَىٰكَ مَا سَقَرُ ﴿٢٧﴾\\
\textamh{28.\  } & لَا تُبْقِى وَلَا تَذَرُ ﴿٢٨﴾\\
\textamh{29.\  } & لَوَّاحَةٌۭ لِّلْبَشَرِ ﴿٢٩﴾\\
\textamh{30.\  } & عَلَيْهَا تِسْعَةَ عَشَرَ ﴿٣٠﴾\\
\textamh{31.\  } & وَمَا جَعَلْنَآ أَصْحَـٰبَ ٱلنَّارِ إِلَّا مَلَـٰٓئِكَةًۭ ۙ وَمَا جَعَلْنَا عِدَّتَهُمْ إِلَّا فِتْنَةًۭ لِّلَّذِينَ كَفَرُوا۟ لِيَسْتَيْقِنَ ٱلَّذِينَ أُوتُوا۟ ٱلْكِتَـٰبَ وَيَزْدَادَ ٱلَّذِينَ ءَامَنُوٓا۟ إِيمَـٰنًۭا ۙ وَلَا يَرْتَابَ ٱلَّذِينَ أُوتُوا۟ ٱلْكِتَـٰبَ وَٱلْمُؤْمِنُونَ ۙ وَلِيَقُولَ ٱلَّذِينَ فِى قُلُوبِهِم مَّرَضٌۭ وَٱلْكَـٰفِرُونَ مَاذَآ أَرَادَ ٱللَّهُ بِهَـٰذَا مَثَلًۭا ۚ كَذَٟلِكَ يُضِلُّ ٱللَّهُ مَن يَشَآءُ وَيَهْدِى مَن يَشَآءُ ۚ وَمَا يَعْلَمُ جُنُودَ رَبِّكَ إِلَّا هُوَ ۚ وَمَا هِىَ إِلَّا ذِكْرَىٰ لِلْبَشَرِ ﴿٣١﴾\\
\textamh{32.\  } & كَلَّا وَٱلْقَمَرِ ﴿٣٢﴾\\
\textamh{33.\  } & وَٱلَّيْلِ إِذْ أَدْبَرَ ﴿٣٣﴾\\
\textamh{34.\  } & وَٱلصُّبْحِ إِذَآ أَسْفَرَ ﴿٣٤﴾\\
\textamh{35.\  } & إِنَّهَا لَإِحْدَى ٱلْكُبَرِ ﴿٣٥﴾\\
\textamh{36.\  } & نَذِيرًۭا لِّلْبَشَرِ ﴿٣٦﴾\\
\textamh{37.\  } & لِمَن شَآءَ مِنكُمْ أَن يَتَقَدَّمَ أَوْ يَتَأَخَّرَ ﴿٣٧﴾\\
\textamh{38.\  } & كُلُّ نَفْسٍۭ بِمَا كَسَبَتْ رَهِينَةٌ ﴿٣٨﴾\\
\textamh{39.\  } & إِلَّآ أَصْحَـٰبَ ٱلْيَمِينِ ﴿٣٩﴾\\
\textamh{40.\  } & فِى جَنَّـٰتٍۢ يَتَسَآءَلُونَ ﴿٤٠﴾\\
\textamh{41.\  } & عَنِ ٱلْمُجْرِمِينَ ﴿٤١﴾\\
\textamh{42.\  } & مَا سَلَكَكُمْ فِى سَقَرَ ﴿٤٢﴾\\
\textamh{43.\  } & قَالُوا۟ لَمْ نَكُ مِنَ ٱلْمُصَلِّينَ ﴿٤٣﴾\\
\textamh{44.\  } & وَلَمْ نَكُ نُطْعِمُ ٱلْمِسْكِينَ ﴿٤٤﴾\\
\textamh{45.\  } & وَكُنَّا نَخُوضُ مَعَ ٱلْخَآئِضِينَ ﴿٤٥﴾\\
\textamh{46.\  } & وَكُنَّا نُكَذِّبُ بِيَوْمِ ٱلدِّينِ ﴿٤٦﴾\\
\textamh{47.\  } & حَتَّىٰٓ أَتَىٰنَا ٱلْيَقِينُ ﴿٤٧﴾\\
\textamh{48.\  } & فَمَا تَنفَعُهُمْ شَفَـٰعَةُ ٱلشَّـٰفِعِينَ ﴿٤٨﴾\\
\textamh{49.\  } & فَمَا لَهُمْ عَنِ ٱلتَّذْكِرَةِ مُعْرِضِينَ ﴿٤٩﴾\\
\textamh{50.\  } & كَأَنَّهُمْ حُمُرٌۭ مُّسْتَنفِرَةٌۭ ﴿٥٠﴾\\
\textamh{51.\  } & فَرَّتْ مِن قَسْوَرَةٍۭ ﴿٥١﴾\\
\textamh{52.\  } & بَلْ يُرِيدُ كُلُّ ٱمْرِئٍۢ مِّنْهُمْ أَن يُؤْتَىٰ صُحُفًۭا مُّنَشَّرَةًۭ ﴿٥٢﴾\\
\textamh{53.\  } & كَلَّا ۖ بَل لَّا يَخَافُونَ ٱلْءَاخِرَةَ ﴿٥٣﴾\\
\textamh{54.\  } & كَلَّآ إِنَّهُۥ تَذْكِرَةٌۭ ﴿٥٤﴾\\
\textamh{55.\  } & فَمَن شَآءَ ذَكَرَهُۥ ﴿٥٥﴾\\
\textamh{56.\  } & وَمَا يَذْكُرُونَ إِلَّآ أَن يَشَآءَ ٱللَّهُ ۚ هُوَ أَهْلُ ٱلتَّقْوَىٰ وَأَهْلُ ٱلْمَغْفِرَةِ ﴿٥٦﴾\\
\end{longtable}
\clearpage
%% License: BSD style (Berkley) (i.e. Put the Copyright owner's name always)
%% Writer and Copyright (to): Bewketu(Bilal) Tadilo (2016-17)
\centering\section{\LR{\textamharic{ሱራቱ አልቂያማ -}  \RL{سوره  القيامة}}}
\begin{longtable}{%
  @{}
    p{.5\textwidth}
  @{~~~~~~~~~~~~~}
    p{.5\textwidth}
    @{}
}
\nopagebreak
\textamh{ቢስሚላሂ አራህመኒ ራሂይም } &  بِسْمِ ٱللَّهِ ٱلرَّحْمَـٰنِ ٱلرَّحِيمِ\\
\textamh{1.\  } &  لَآ أُقْسِمُ بِيَوْمِ ٱلْقِيَـٰمَةِ ﴿١﴾\\
\textamh{2.\  } & وَلَآ أُقْسِمُ بِٱلنَّفْسِ ٱللَّوَّامَةِ ﴿٢﴾\\
\textamh{3.\  } & أَيَحْسَبُ ٱلْإِنسَـٰنُ أَلَّن نَّجْمَعَ عِظَامَهُۥ ﴿٣﴾\\
\textamh{4.\  } & بَلَىٰ قَـٰدِرِينَ عَلَىٰٓ أَن نُّسَوِّىَ بَنَانَهُۥ ﴿٤﴾\\
\textamh{5.\  } & بَلْ يُرِيدُ ٱلْإِنسَـٰنُ لِيَفْجُرَ أَمَامَهُۥ ﴿٥﴾\\
\textamh{6.\  } & يَسْـَٔلُ أَيَّانَ يَوْمُ ٱلْقِيَـٰمَةِ ﴿٦﴾\\
\textamh{7.\  } & فَإِذَا بَرِقَ ٱلْبَصَرُ ﴿٧﴾\\
\textamh{8.\  } & وَخَسَفَ ٱلْقَمَرُ ﴿٨﴾\\
\textamh{9.\  } & وَجُمِعَ ٱلشَّمْسُ وَٱلْقَمَرُ ﴿٩﴾\\
\textamh{10.\  } & يَقُولُ ٱلْإِنسَـٰنُ يَوْمَئِذٍ أَيْنَ ٱلْمَفَرُّ ﴿١٠﴾\\
\textamh{11.\  } & كَلَّا لَا وَزَرَ ﴿١١﴾\\
\textamh{12.\  } & إِلَىٰ رَبِّكَ يَوْمَئِذٍ ٱلْمُسْتَقَرُّ ﴿١٢﴾\\
\textamh{13.\  } & يُنَبَّؤُا۟ ٱلْإِنسَـٰنُ يَوْمَئِذٍۭ بِمَا قَدَّمَ وَأَخَّرَ ﴿١٣﴾\\
\textamh{14.\  } & بَلِ ٱلْإِنسَـٰنُ عَلَىٰ نَفْسِهِۦ بَصِيرَةٌۭ ﴿١٤﴾\\
\textamh{15.\  } & وَلَوْ أَلْقَىٰ مَعَاذِيرَهُۥ ﴿١٥﴾\\
\textamh{16.\  } & لَا تُحَرِّكْ بِهِۦ لِسَانَكَ لِتَعْجَلَ بِهِۦٓ ﴿١٦﴾\\
\textamh{17.\  } & إِنَّ عَلَيْنَا جَمْعَهُۥ وَقُرْءَانَهُۥ ﴿١٧﴾\\
\textamh{18.\  } & فَإِذَا قَرَأْنَـٰهُ فَٱتَّبِعْ قُرْءَانَهُۥ ﴿١٨﴾\\
\textamh{19.\  } & ثُمَّ إِنَّ عَلَيْنَا بَيَانَهُۥ ﴿١٩﴾\\
\textamh{20.\  } & كَلَّا بَلْ تُحِبُّونَ ٱلْعَاجِلَةَ ﴿٢٠﴾\\
\textamh{21.\  } & وَتَذَرُونَ ٱلْءَاخِرَةَ ﴿٢١﴾\\
\textamh{22.\  } & وُجُوهٌۭ يَوْمَئِذٍۢ نَّاضِرَةٌ ﴿٢٢﴾\\
\textamh{23.\  } & إِلَىٰ رَبِّهَا نَاظِرَةٌۭ ﴿٢٣﴾\\
\textamh{24.\  } & وَوُجُوهٌۭ يَوْمَئِذٍۭ بَاسِرَةٌۭ ﴿٢٤﴾\\
\textamh{25.\  } & تَظُنُّ أَن يُفْعَلَ بِهَا فَاقِرَةٌۭ ﴿٢٥﴾\\
\textamh{26.\  } & كَلَّآ إِذَا بَلَغَتِ ٱلتَّرَاقِىَ ﴿٢٦﴾\\
\textamh{27.\  } & وَقِيلَ مَنْ ۜ رَاقٍۢ ﴿٢٧﴾\\
\textamh{28.\  } & وَظَنَّ أَنَّهُ ٱلْفِرَاقُ ﴿٢٨﴾\\
\textamh{29.\  } & وَٱلْتَفَّتِ ٱلسَّاقُ بِٱلسَّاقِ ﴿٢٩﴾\\
\textamh{30.\  } & إِلَىٰ رَبِّكَ يَوْمَئِذٍ ٱلْمَسَاقُ ﴿٣٠﴾\\
\textamh{31.\  } & فَلَا صَدَّقَ وَلَا صَلَّىٰ ﴿٣١﴾\\
\textamh{32.\  } & وَلَـٰكِن كَذَّبَ وَتَوَلَّىٰ ﴿٣٢﴾\\
\textamh{33.\  } & ثُمَّ ذَهَبَ إِلَىٰٓ أَهْلِهِۦ يَتَمَطَّىٰٓ ﴿٣٣﴾\\
\textamh{34.\  } & أَوْلَىٰ لَكَ فَأَوْلَىٰ ﴿٣٤﴾\\
\textamh{35.\  } & ثُمَّ أَوْلَىٰ لَكَ فَأَوْلَىٰٓ ﴿٣٥﴾\\
\textamh{36.\  } & أَيَحْسَبُ ٱلْإِنسَـٰنُ أَن يُتْرَكَ سُدًى ﴿٣٦﴾\\
\textamh{37.\  } & أَلَمْ يَكُ نُطْفَةًۭ مِّن مَّنِىٍّۢ يُمْنَىٰ ﴿٣٧﴾\\
\textamh{38.\  } & ثُمَّ كَانَ عَلَقَةًۭ فَخَلَقَ فَسَوَّىٰ ﴿٣٨﴾\\
\textamh{39.\  } & فَجَعَلَ مِنْهُ ٱلزَّوْجَيْنِ ٱلذَّكَرَ وَٱلْأُنثَىٰٓ ﴿٣٩﴾\\
\textamh{40.\  } & أَلَيْسَ ذَٟلِكَ بِقَـٰدِرٍ عَلَىٰٓ أَن يُحْۦِىَ ٱلْمَوْتَىٰ ﴿٤٠﴾\\
\end{longtable}
\clearpage
%% License: BSD style (Berkley) (i.e. Put the Copyright owner's name always)
%% Writer and Copyright (to): Bewketu(Bilal) Tadilo (2016-17)
\centering\section{\LR{\textamharic{ሱራቱ አልኢንሳን -}  \RL{سوره  الانسان}}}
\begin{longtable}{%
  @{}
    p{.5\textwidth}
  @{~~~~~~~~~~~~}
    p{.5\textwidth}
    @{}
}
\nopagebreak
\textamh{ቢስሚላሂ አራህመኒ ራሂይም } &  بِسْمِ ٱللَّهِ ٱلرَّحْمَـٰنِ ٱلرَّحِيمِ\\
\textamh{1.\  } &  هَلْ أَتَىٰ عَلَى ٱلْإِنسَـٰنِ حِينٌۭ مِّنَ ٱلدَّهْرِ لَمْ يَكُن شَيْـًۭٔا مَّذْكُورًا ﴿١﴾\\
\textamh{2.\  } & إِنَّا خَلَقْنَا ٱلْإِنسَـٰنَ مِن نُّطْفَةٍ أَمْشَاجٍۢ نَّبْتَلِيهِ فَجَعَلْنَـٰهُ سَمِيعًۢا بَصِيرًا ﴿٢﴾\\
\textamh{3.\  } & إِنَّا هَدَيْنَـٰهُ ٱلسَّبِيلَ إِمَّا شَاكِرًۭا وَإِمَّا كَفُورًا ﴿٣﴾\\
\textamh{4.\  } & إِنَّآ أَعْتَدْنَا لِلْكَـٰفِرِينَ سَلَـٰسِلَا۟ وَأَغْلَـٰلًۭا وَسَعِيرًا ﴿٤﴾\\
\textamh{5.\  } & إِنَّ ٱلْأَبْرَارَ يَشْرَبُونَ مِن كَأْسٍۢ كَانَ مِزَاجُهَا كَافُورًا ﴿٥﴾\\
\textamh{6.\  } & عَيْنًۭا يَشْرَبُ بِهَا عِبَادُ ٱللَّهِ يُفَجِّرُونَهَا تَفْجِيرًۭا ﴿٦﴾\\
\textamh{7.\  } & يُوفُونَ بِٱلنَّذْرِ وَيَخَافُونَ يَوْمًۭا كَانَ شَرُّهُۥ مُسْتَطِيرًۭا ﴿٧﴾\\
\textamh{8.\  } & وَيُطْعِمُونَ ٱلطَّعَامَ عَلَىٰ حُبِّهِۦ مِسْكِينًۭا وَيَتِيمًۭا وَأَسِيرًا ﴿٨﴾\\
\textamh{9.\  } & إِنَّمَا نُطْعِمُكُمْ لِوَجْهِ ٱللَّهِ لَا نُرِيدُ مِنكُمْ جَزَآءًۭ وَلَا شُكُورًا ﴿٩﴾\\
\textamh{10.\  } & إِنَّا نَخَافُ مِن رَّبِّنَا يَوْمًا عَبُوسًۭا قَمْطَرِيرًۭا ﴿١٠﴾\\
\textamh{11.\  } & فَوَقَىٰهُمُ ٱللَّهُ شَرَّ ذَٟلِكَ ٱلْيَوْمِ وَلَقَّىٰهُمْ نَضْرَةًۭ وَسُرُورًۭا ﴿١١﴾\\
\textamh{12.\  } & وَجَزَىٰهُم بِمَا صَبَرُوا۟ جَنَّةًۭ وَحَرِيرًۭا ﴿١٢﴾\\
\textamh{13.\  } & مُّتَّكِـِٔينَ فِيهَا عَلَى ٱلْأَرَآئِكِ ۖ لَا يَرَوْنَ فِيهَا شَمْسًۭا وَلَا زَمْهَرِيرًۭا ﴿١٣﴾\\
\textamh{14.\  } & وَدَانِيَةً عَلَيْهِمْ ظِلَـٰلُهَا وَذُلِّلَتْ قُطُوفُهَا تَذْلِيلًۭا ﴿١٤﴾\\
\textamh{15.\  } & وَيُطَافُ عَلَيْهِم بِـَٔانِيَةٍۢ مِّن فِضَّةٍۢ وَأَكْوَابٍۢ كَانَتْ قَوَارِيرَا۠ ﴿١٥﴾\\
\textamh{16.\  } & قَوَارِيرَا۟ مِن فِضَّةٍۢ قَدَّرُوهَا تَقْدِيرًۭا ﴿١٦﴾\\
\textamh{17.\  } & وَيُسْقَوْنَ فِيهَا كَأْسًۭا كَانَ مِزَاجُهَا زَنجَبِيلًا ﴿١٧﴾\\
\textamh{18.\  } & عَيْنًۭا فِيهَا تُسَمَّىٰ سَلْسَبِيلًۭا ﴿١٨﴾\\
\textamh{19.\  } & ۞ وَيَطُوفُ عَلَيْهِمْ وِلْدَٟنٌۭ مُّخَلَّدُونَ إِذَا رَأَيْتَهُمْ حَسِبْتَهُمْ لُؤْلُؤًۭا مَّنثُورًۭا ﴿١٩﴾\\
\textamh{20.\  } & وَإِذَا رَأَيْتَ ثَمَّ رَأَيْتَ نَعِيمًۭا وَمُلْكًۭا كَبِيرًا ﴿٢٠﴾\\
\textamh{21.\  } & عَـٰلِيَهُمْ ثِيَابُ سُندُسٍ خُضْرٌۭ وَإِسْتَبْرَقٌۭ ۖ وَحُلُّوٓا۟ أَسَاوِرَ مِن فِضَّةٍۢ وَسَقَىٰهُمْ رَبُّهُمْ شَرَابًۭا طَهُورًا ﴿٢١﴾\\
\textamh{22.\  } & إِنَّ هَـٰذَا كَانَ لَكُمْ جَزَآءًۭ وَكَانَ سَعْيُكُم مَّشْكُورًا ﴿٢٢﴾\\
\textamh{23.\  } & إِنَّا نَحْنُ نَزَّلْنَا عَلَيْكَ ٱلْقُرْءَانَ تَنزِيلًۭا ﴿٢٣﴾\\
\textamh{24.\  } & فَٱصْبِرْ لِحُكْمِ رَبِّكَ وَلَا تُطِعْ مِنْهُمْ ءَاثِمًا أَوْ كَفُورًۭا ﴿٢٤﴾\\
\textamh{25.\  } & وَٱذْكُرِ ٱسْمَ رَبِّكَ بُكْرَةًۭ وَأَصِيلًۭا ﴿٢٥﴾\\
\textamh{26.\  } & وَمِنَ ٱلَّيْلِ فَٱسْجُدْ لَهُۥ وَسَبِّحْهُ لَيْلًۭا طَوِيلًا ﴿٢٦﴾\\
\textamh{27.\  } & إِنَّ هَـٰٓؤُلَآءِ يُحِبُّونَ ٱلْعَاجِلَةَ وَيَذَرُونَ وَرَآءَهُمْ يَوْمًۭا ثَقِيلًۭا ﴿٢٧﴾\\
\textamh{28.\  } & نَّحْنُ خَلَقْنَـٰهُمْ وَشَدَدْنَآ أَسْرَهُمْ ۖ وَإِذَا شِئْنَا بَدَّلْنَآ أَمْثَـٰلَهُمْ تَبْدِيلًا ﴿٢٨﴾\\
\textamh{29.\  } & إِنَّ هَـٰذِهِۦ تَذْكِرَةٌۭ ۖ فَمَن شَآءَ ٱتَّخَذَ إِلَىٰ رَبِّهِۦ سَبِيلًۭا ﴿٢٩﴾\\
\textamh{30.\  } & وَمَا تَشَآءُونَ إِلَّآ أَن يَشَآءَ ٱللَّهُ ۚ إِنَّ ٱللَّهَ كَانَ عَلِيمًا حَكِيمًۭا ﴿٣٠﴾\\
\textamh{31.\  } & يُدْخِلُ مَن يَشَآءُ فِى رَحْمَتِهِۦ ۚ وَٱلظَّـٰلِمِينَ أَعَدَّ لَهُمْ عَذَابًا أَلِيمًۢا ﴿٣١﴾\\
\end{longtable}
\clearpage
%% License: BSD style (Berkley) (i.e. Put the Copyright owner's name always)
%% Writer and Copyright (to): Bewketu(Bilal) Tadilo (2016-17)
\centering\section{\LR{\textamharic{ሱራቱ አልሙርሰላት -}  \RL{سوره  المرسلات}}}
\begin{longtable}{%
  @{}
    p{.5\textwidth}
  @{~~~~~~~~~~~~~}
    p{.5\textwidth}
    @{}
}
\nopagebreak
\textamh{ቢስሚላሂ አራህመኒ ራሂይም } &  بِسْمِ ٱللَّهِ ٱلرَّحْمَـٰنِ ٱلرَّحِيمِ\\
\textamh{1.\  } &  وَٱلْمُرْسَلَـٰتِ عُرْفًۭا ﴿١﴾\\
\textamh{2.\  } & فَٱلْعَـٰصِفَـٰتِ عَصْفًۭا ﴿٢﴾\\
\textamh{3.\  } & وَٱلنَّـٰشِرَٰتِ نَشْرًۭا ﴿٣﴾\\
\textamh{4.\  } & فَٱلْفَـٰرِقَـٰتِ فَرْقًۭا ﴿٤﴾\\
\textamh{5.\  } & فَٱلْمُلْقِيَـٰتِ ذِكْرًا ﴿٥﴾\\
\textamh{6.\  } & عُذْرًا أَوْ نُذْرًا ﴿٦﴾\\
\textamh{7.\  } & إِنَّمَا تُوعَدُونَ لَوَٟقِعٌۭ ﴿٧﴾\\
\textamh{8.\  } & فَإِذَا ٱلنُّجُومُ طُمِسَتْ ﴿٨﴾\\
\textamh{9.\  } & وَإِذَا ٱلسَّمَآءُ فُرِجَتْ ﴿٩﴾\\
\textamh{10.\  } & وَإِذَا ٱلْجِبَالُ نُسِفَتْ ﴿١٠﴾\\
\textamh{11.\  } & وَإِذَا ٱلرُّسُلُ أُقِّتَتْ ﴿١١﴾\\
\textamh{12.\  } & لِأَىِّ يَوْمٍ أُجِّلَتْ ﴿١٢﴾\\
\textamh{13.\  } & لِيَوْمِ ٱلْفَصْلِ ﴿١٣﴾\\
\textamh{14.\  } & وَمَآ أَدْرَىٰكَ مَا يَوْمُ ٱلْفَصْلِ ﴿١٤﴾\\
\textamh{15.\  } & وَيْلٌۭ يَوْمَئِذٍۢ لِّلْمُكَذِّبِينَ ﴿١٥﴾\\
\textamh{16.\  } & أَلَمْ نُهْلِكِ ٱلْأَوَّلِينَ ﴿١٦﴾\\
\textamh{17.\  } & ثُمَّ نُتْبِعُهُمُ ٱلْءَاخِرِينَ ﴿١٧﴾\\
\textamh{18.\  } & كَذَٟلِكَ نَفْعَلُ بِٱلْمُجْرِمِينَ ﴿١٨﴾\\
\textamh{19.\  } & وَيْلٌۭ يَوْمَئِذٍۢ لِّلْمُكَذِّبِينَ ﴿١٩﴾\\
\textamh{20.\  } & أَلَمْ نَخْلُقكُّم مِّن مَّآءٍۢ مَّهِينٍۢ ﴿٢٠﴾\\
\textamh{21.\  } & فَجَعَلْنَـٰهُ فِى قَرَارٍۢ مَّكِينٍ ﴿٢١﴾\\
\textamh{22.\  } & إِلَىٰ قَدَرٍۢ مَّعْلُومٍۢ ﴿٢٢﴾\\
\textamh{23.\  } & فَقَدَرْنَا فَنِعْمَ ٱلْقَـٰدِرُونَ ﴿٢٣﴾\\
\textamh{24.\  } & وَيْلٌۭ يَوْمَئِذٍۢ لِّلْمُكَذِّبِينَ ﴿٢٤﴾\\
\textamh{25.\  } & أَلَمْ نَجْعَلِ ٱلْأَرْضَ كِفَاتًا ﴿٢٥﴾\\
\textamh{26.\  } & أَحْيَآءًۭ وَأَمْوَٟتًۭا ﴿٢٦﴾\\
\textamh{27.\  } & وَجَعَلْنَا فِيهَا رَوَٟسِىَ شَـٰمِخَـٰتٍۢ وَأَسْقَيْنَـٰكُم مَّآءًۭ فُرَاتًۭا ﴿٢٧﴾\\
\textamh{28.\  } & وَيْلٌۭ يَوْمَئِذٍۢ لِّلْمُكَذِّبِينَ ﴿٢٨﴾\\
\textamh{29.\  } & ٱنطَلِقُوٓا۟ إِلَىٰ مَا كُنتُم بِهِۦ تُكَذِّبُونَ ﴿٢٩﴾\\
\textamh{30.\  } & ٱنطَلِقُوٓا۟ إِلَىٰ ظِلٍّۢ ذِى ثَلَـٰثِ شُعَبٍۢ ﴿٣٠﴾\\
\textamh{31.\  } & لَّا ظَلِيلٍۢ وَلَا يُغْنِى مِنَ ٱللَّهَبِ ﴿٣١﴾\\
\textamh{32.\  } & إِنَّهَا تَرْمِى بِشَرَرٍۢ كَٱلْقَصْرِ ﴿٣٢﴾\\
\textamh{33.\  } & كَأَنَّهُۥ جِمَـٰلَتٌۭ صُفْرٌۭ ﴿٣٣﴾\\
\textamh{34.\  } & وَيْلٌۭ يَوْمَئِذٍۢ لِّلْمُكَذِّبِينَ ﴿٣٤﴾\\
\textamh{35.\  } & هَـٰذَا يَوْمُ لَا يَنطِقُونَ ﴿٣٥﴾\\
\textamh{36.\  } & وَلَا يُؤْذَنُ لَهُمْ فَيَعْتَذِرُونَ ﴿٣٦﴾\\
\textamh{37.\  } & وَيْلٌۭ يَوْمَئِذٍۢ لِّلْمُكَذِّبِينَ ﴿٣٧﴾\\
\textamh{38.\  } & هَـٰذَا يَوْمُ ٱلْفَصْلِ ۖ جَمَعْنَـٰكُمْ وَٱلْأَوَّلِينَ ﴿٣٨﴾\\
\textamh{39.\  } & فَإِن كَانَ لَكُمْ كَيْدٌۭ فَكِيدُونِ ﴿٣٩﴾\\
\textamh{40.\  } & وَيْلٌۭ يَوْمَئِذٍۢ لِّلْمُكَذِّبِينَ ﴿٤٠﴾\\
\textamh{41.\  } & إِنَّ ٱلْمُتَّقِينَ فِى ظِلَـٰلٍۢ وَعُيُونٍۢ ﴿٤١﴾\\
\textamh{42.\  } & وَفَوَٟكِهَ مِمَّا يَشْتَهُونَ ﴿٤٢﴾\\
\textamh{43.\  } & كُلُوا۟ وَٱشْرَبُوا۟ هَنِيٓـًٔۢا بِمَا كُنتُمْ تَعْمَلُونَ ﴿٤٣﴾\\
\textamh{44.\  } & إِنَّا كَذَٟلِكَ نَجْزِى ٱلْمُحْسِنِينَ ﴿٤٤﴾\\
\textamh{45.\  } & وَيْلٌۭ يَوْمَئِذٍۢ لِّلْمُكَذِّبِينَ ﴿٤٥﴾\\
\textamh{46.\  } & كُلُوا۟ وَتَمَتَّعُوا۟ قَلِيلًا إِنَّكُم مُّجْرِمُونَ ﴿٤٦﴾\\
\textamh{47.\  } & وَيْلٌۭ يَوْمَئِذٍۢ لِّلْمُكَذِّبِينَ ﴿٤٧﴾\\
\textamh{48.\  } & وَإِذَا قِيلَ لَهُمُ ٱرْكَعُوا۟ لَا يَرْكَعُونَ ﴿٤٨﴾\\
\textamh{49.\  } & وَيْلٌۭ يَوْمَئِذٍۢ لِّلْمُكَذِّبِينَ ﴿٤٩﴾\\
\textamh{50.\  } & فَبِأَىِّ حَدِيثٍۭ بَعْدَهُۥ يُؤْمِنُونَ ﴿٥٠﴾\\
\end{longtable}
\clearpage
%% License: BSD style (Berkley) (i.e. Put the Copyright owner's name always)
%% Writer and Copyright (to): Bewketu(Bilal) Tadilo (2016-17)
\centering\section{\LR{\textamharic{ሱራቱ አንነባኢ -}  \RL{سوره  النبإ}}}
\begin{longtable}{%
  @{}
    p{.5\textwidth}
  @{~~~~~~~~~~~~~}
    p{.5\textwidth}
    @{}
}
\nopagebreak
\textamh{ቢስሚላሂ አራህመኒ ራሂይም } &  بِسْمِ ٱللَّهِ ٱلرَّحْمَـٰنِ ٱلرَّحِيمِ\\
\textamh{1.\  } &  عَمَّ يَتَسَآءَلُونَ ﴿١﴾\\
\textamh{2.\  } & عَنِ ٱلنَّبَإِ ٱلْعَظِيمِ ﴿٢﴾\\
\textamh{3.\  } & ٱلَّذِى هُمْ فِيهِ مُخْتَلِفُونَ ﴿٣﴾\\
\textamh{4.\  } & كَلَّا سَيَعْلَمُونَ ﴿٤﴾\\
\textamh{5.\  } & ثُمَّ كَلَّا سَيَعْلَمُونَ ﴿٥﴾\\
\textamh{6.\  } & أَلَمْ نَجْعَلِ ٱلْأَرْضَ مِهَـٰدًۭا ﴿٦﴾\\
\textamh{7.\  } & وَٱلْجِبَالَ أَوْتَادًۭا ﴿٧﴾\\
\textamh{8.\  } & وَخَلَقْنَـٰكُمْ أَزْوَٟجًۭا ﴿٨﴾\\
\textamh{9.\  } & وَجَعَلْنَا نَوْمَكُمْ سُبَاتًۭا ﴿٩﴾\\
\textamh{10.\  } & وَجَعَلْنَا ٱلَّيْلَ لِبَاسًۭا ﴿١٠﴾\\
\textamh{11.\  } & وَجَعَلْنَا ٱلنَّهَارَ مَعَاشًۭا ﴿١١﴾\\
\textamh{12.\  } & وَبَنَيْنَا فَوْقَكُمْ سَبْعًۭا شِدَادًۭا ﴿١٢﴾\\
\textamh{13.\  } & وَجَعَلْنَا سِرَاجًۭا وَهَّاجًۭا ﴿١٣﴾\\
\textamh{14.\  } & وَأَنزَلْنَا مِنَ ٱلْمُعْصِرَٰتِ مَآءًۭ ثَجَّاجًۭا ﴿١٤﴾\\
\textamh{15.\  } & لِّنُخْرِجَ بِهِۦ حَبًّۭا وَنَبَاتًۭا ﴿١٥﴾\\
\textamh{16.\  } & وَجَنَّـٰتٍ أَلْفَافًا ﴿١٦﴾\\
\textamh{17.\  } & إِنَّ يَوْمَ ٱلْفَصْلِ كَانَ مِيقَـٰتًۭا ﴿١٧﴾\\
\textamh{18.\  } & يَوْمَ يُنفَخُ فِى ٱلصُّورِ فَتَأْتُونَ أَفْوَاجًۭا ﴿١٨﴾\\
\textamh{19.\  } & وَفُتِحَتِ ٱلسَّمَآءُ فَكَانَتْ أَبْوَٟبًۭا ﴿١٩﴾\\
\textamh{20.\  } & وَسُيِّرَتِ ٱلْجِبَالُ فَكَانَتْ سَرَابًا ﴿٢٠﴾\\
\textamh{21.\  } & إِنَّ جَهَنَّمَ كَانَتْ مِرْصَادًۭا ﴿٢١﴾\\
\textamh{22.\  } & لِّلطَّٰغِينَ مَـَٔابًۭا ﴿٢٢﴾\\
\textamh{23.\  } & لَّٰبِثِينَ فِيهَآ أَحْقَابًۭا ﴿٢٣﴾\\
\textamh{24.\  } & لَّا يَذُوقُونَ فِيهَا بَرْدًۭا وَلَا شَرَابًا ﴿٢٤﴾\\
\textamh{25.\  } & إِلَّا حَمِيمًۭا وَغَسَّاقًۭا ﴿٢٥﴾\\
\textamh{26.\  } & جَزَآءًۭ وِفَاقًا ﴿٢٦﴾\\
\textamh{27.\  } & إِنَّهُمْ كَانُوا۟ لَا يَرْجُونَ حِسَابًۭا ﴿٢٧﴾\\
\textamh{28.\  } & وَكَذَّبُوا۟ بِـَٔايَـٰتِنَا كِذَّابًۭا ﴿٢٨﴾\\
\textamh{29.\  } & وَكُلَّ شَىْءٍ أَحْصَيْنَـٰهُ كِتَـٰبًۭا ﴿٢٩﴾\\
\textamh{30.\  } & فَذُوقُوا۟ فَلَن نَّزِيدَكُمْ إِلَّا عَذَابًا ﴿٣٠﴾\\
\textamh{31.\  } & إِنَّ لِلْمُتَّقِينَ مَفَازًا ﴿٣١﴾\\
\textamh{32.\  } & حَدَآئِقَ وَأَعْنَـٰبًۭا ﴿٣٢﴾\\
\textamh{33.\  } & وَكَوَاعِبَ أَتْرَابًۭا ﴿٣٣﴾\\
\textamh{34.\  } & وَكَأْسًۭا دِهَاقًۭا ﴿٣٤﴾\\
\textamh{35.\  } & لَّا يَسْمَعُونَ فِيهَا لَغْوًۭا وَلَا كِذَّٰبًۭا ﴿٣٥﴾\\
\textamh{36.\  } & جَزَآءًۭ مِّن رَّبِّكَ عَطَآءً حِسَابًۭا ﴿٣٦﴾\\
\textamh{37.\  } & رَّبِّ ٱلسَّمَـٰوَٟتِ وَٱلْأَرْضِ وَمَا بَيْنَهُمَا ٱلرَّحْمَـٰنِ ۖ لَا يَمْلِكُونَ مِنْهُ خِطَابًۭا ﴿٣٧﴾\\
\textamh{38.\  } & يَوْمَ يَقُومُ ٱلرُّوحُ وَٱلْمَلَـٰٓئِكَةُ صَفًّۭا ۖ لَّا يَتَكَلَّمُونَ إِلَّا مَنْ أَذِنَ لَهُ ٱلرَّحْمَـٰنُ وَقَالَ صَوَابًۭا ﴿٣٨﴾\\
\textamh{39.\  } & ذَٟلِكَ ٱلْيَوْمُ ٱلْحَقُّ ۖ فَمَن شَآءَ ٱتَّخَذَ إِلَىٰ رَبِّهِۦ مَـَٔابًا ﴿٣٩﴾\\
\textamh{40.\  } & إِنَّآ أَنذَرْنَـٰكُمْ عَذَابًۭا قَرِيبًۭا يَوْمَ يَنظُرُ ٱلْمَرْءُ مَا قَدَّمَتْ يَدَاهُ وَيَقُولُ ٱلْكَافِرُ يَـٰلَيْتَنِى كُنتُ تُرَٰبًۢا ﴿٤٠﴾\\
\end{longtable}
\clearpage
%% License: BSD style (Berkley) (i.e. Put the Copyright owner's name always)
%% Writer and Copyright (to): Bewketu(Bilal) Tadilo (2016-17)
\centering\section{\LR{\textamharic{ሱራቱ አንነዚኣት -}  \RL{سوره  النازعات}}}
\begin{longtable}{%
  @{}
    p{.5\textwidth}
  @{~~~~~~~~~~~~}
    p{.5\textwidth}
    @{}
}
\nopagebreak
\textamh{ቢስሚላሂ አራህመኒ ራሂይም } &  بِسْمِ ٱللَّهِ ٱلرَّحْمَـٰنِ ٱلرَّحِيمِ\\
\textamh{1.\  } &  وَٱلنَّـٰزِعَـٰتِ غَرْقًۭا ﴿١﴾\\
\textamh{2.\  } & وَٱلنَّـٰشِطَٰتِ نَشْطًۭا ﴿٢﴾\\
\textamh{3.\  } & وَٱلسَّٰبِحَـٰتِ سَبْحًۭا ﴿٣﴾\\
\textamh{4.\  } & فَٱلسَّٰبِقَـٰتِ سَبْقًۭا ﴿٤﴾\\
\textamh{5.\  } & فَٱلْمُدَبِّرَٰتِ أَمْرًۭا ﴿٥﴾\\
\textamh{6.\  } & يَوْمَ تَرْجُفُ ٱلرَّاجِفَةُ ﴿٦﴾\\
\textamh{7.\  } & تَتْبَعُهَا ٱلرَّادِفَةُ ﴿٧﴾\\
\textamh{8.\  } & قُلُوبٌۭ يَوْمَئِذٍۢ وَاجِفَةٌ ﴿٨﴾\\
\textamh{9.\  } & أَبْصَـٰرُهَا خَـٰشِعَةٌۭ ﴿٩﴾\\
\textamh{10.\  } & يَقُولُونَ أَءِنَّا لَمَرْدُودُونَ فِى ٱلْحَافِرَةِ ﴿١٠﴾\\
\textamh{11.\  } & أَءِذَا كُنَّا عِظَـٰمًۭا نَّخِرَةًۭ ﴿١١﴾\\
\textamh{12.\  } & قَالُوا۟ تِلْكَ إِذًۭا كَرَّةٌ خَاسِرَةٌۭ ﴿١٢﴾\\
\textamh{13.\  } & فَإِنَّمَا هِىَ زَجْرَةٌۭ وَٟحِدَةٌۭ ﴿١٣﴾\\
\textamh{14.\  } & فَإِذَا هُم بِٱلسَّاهِرَةِ ﴿١٤﴾\\
\textamh{15.\  } & هَلْ أَتَىٰكَ حَدِيثُ مُوسَىٰٓ ﴿١٥﴾\\
\textamh{16.\  } & إِذْ نَادَىٰهُ رَبُّهُۥ بِٱلْوَادِ ٱلْمُقَدَّسِ طُوًى ﴿١٦﴾\\
\textamh{17.\  } & ٱذْهَبْ إِلَىٰ فِرْعَوْنَ إِنَّهُۥ طَغَىٰ ﴿١٧﴾\\
\textamh{18.\  } & فَقُلْ هَل لَّكَ إِلَىٰٓ أَن تَزَكَّىٰ ﴿١٨﴾\\
\textamh{19.\  } & وَأَهْدِيَكَ إِلَىٰ رَبِّكَ فَتَخْشَىٰ ﴿١٩﴾\\
\textamh{20.\  } & فَأَرَىٰهُ ٱلْءَايَةَ ٱلْكُبْرَىٰ ﴿٢٠﴾\\
\textamh{21.\  } & فَكَذَّبَ وَعَصَىٰ ﴿٢١﴾\\
\textamh{22.\  } & ثُمَّ أَدْبَرَ يَسْعَىٰ ﴿٢٢﴾\\
\textamh{23.\  } & فَحَشَرَ فَنَادَىٰ ﴿٢٣﴾\\
\textamh{24.\  } & فَقَالَ أَنَا۠ رَبُّكُمُ ٱلْأَعْلَىٰ ﴿٢٤﴾\\
\textamh{25.\  } & فَأَخَذَهُ ٱللَّهُ نَكَالَ ٱلْءَاخِرَةِ وَٱلْأُولَىٰٓ ﴿٢٥﴾\\
\textamh{26.\  } & إِنَّ فِى ذَٟلِكَ لَعِبْرَةًۭ لِّمَن يَخْشَىٰٓ ﴿٢٦﴾\\
\textamh{27.\  } & ءَأَنتُمْ أَشَدُّ خَلْقًا أَمِ ٱلسَّمَآءُ ۚ بَنَىٰهَا ﴿٢٧﴾\\
\textamh{28.\  } & رَفَعَ سَمْكَهَا فَسَوَّىٰهَا ﴿٢٨﴾\\
\textamh{29.\  } & وَأَغْطَشَ لَيْلَهَا وَأَخْرَجَ ضُحَىٰهَا ﴿٢٩﴾\\
\textamh{30.\  } & وَٱلْأَرْضَ بَعْدَ ذَٟلِكَ دَحَىٰهَآ ﴿٣٠﴾\\
\textamh{31.\  } & أَخْرَجَ مِنْهَا مَآءَهَا وَمَرْعَىٰهَا ﴿٣١﴾\\
\textamh{32.\  } & وَٱلْجِبَالَ أَرْسَىٰهَا ﴿٣٢﴾\\
\textamh{33.\  } & مَتَـٰعًۭا لَّكُمْ وَلِأَنْعَـٰمِكُمْ ﴿٣٣﴾\\
\textamh{34.\  } & فَإِذَا جَآءَتِ ٱلطَّآمَّةُ ٱلْكُبْرَىٰ ﴿٣٤﴾\\
\textamh{35.\  } & يَوْمَ يَتَذَكَّرُ ٱلْإِنسَـٰنُ مَا سَعَىٰ ﴿٣٥﴾\\
\textamh{36.\  } & وَبُرِّزَتِ ٱلْجَحِيمُ لِمَن يَرَىٰ ﴿٣٦﴾\\
\textamh{37.\  } & فَأَمَّا مَن طَغَىٰ ﴿٣٧﴾\\
\textamh{38.\  } & وَءَاثَرَ ٱلْحَيَوٰةَ ٱلدُّنْيَا ﴿٣٨﴾\\
\textamh{39.\  } & فَإِنَّ ٱلْجَحِيمَ هِىَ ٱلْمَأْوَىٰ ﴿٣٩﴾\\
\textamh{40.\  } & وَأَمَّا مَنْ خَافَ مَقَامَ رَبِّهِۦ وَنَهَى ٱلنَّفْسَ عَنِ ٱلْهَوَىٰ ﴿٤٠﴾\\
\textamh{41.\  } & فَإِنَّ ٱلْجَنَّةَ هِىَ ٱلْمَأْوَىٰ ﴿٤١﴾\\
\textamh{42.\  } & يَسْـَٔلُونَكَ عَنِ ٱلسَّاعَةِ أَيَّانَ مُرْسَىٰهَا ﴿٤٢﴾\\
\textamh{43.\  } & فِيمَ أَنتَ مِن ذِكْرَىٰهَآ ﴿٤٣﴾\\
\textamh{44.\  } & إِلَىٰ رَبِّكَ مُنتَهَىٰهَآ ﴿٤٤﴾\\
\textamh{45.\  } & إِنَّمَآ أَنتَ مُنذِرُ مَن يَخْشَىٰهَا ﴿٤٥﴾\\
\textamh{46.\  } & كَأَنَّهُمْ يَوْمَ يَرَوْنَهَا لَمْ يَلْبَثُوٓا۟ إِلَّا عَشِيَّةً أَوْ ضُحَىٰهَا ﴿٤٦﴾\\
\end{longtable}
\clearpage
%% License: BSD style (Berkley) (i.e. Put the Copyright owner's name always)
%% Writer and Copyright (to): Bewketu(Bilal) Tadilo (2016-17)
\centering\section{\LR{\textamharic{ሱራቱ አበሳ -}  \RL{سوره  عبس}}}
\begin{longtable}{%
  @{}
    p{.5\textwidth}
  @{~~~~~~~~~~~~~}
    p{.5\textwidth}
    @{}
}
\nopagebreak
\textamh{\ \ \ \ \ \  ቢስሚላሂ አራህመኒ ራሂይም } &  بِسْمِ ٱللَّهِ ٱلرَّحْمَـٰنِ ٱلرَّحِيمِ\\
\textamh{1.\  } &  عَبَسَ وَتَوَلَّىٰٓ ﴿١﴾\\
\textamh{2.\  } & أَن جَآءَهُ ٱلْأَعْمَىٰ ﴿٢﴾\\
\textamh{3.\  } & وَمَا يُدْرِيكَ لَعَلَّهُۥ يَزَّكَّىٰٓ ﴿٣﴾\\
\textamh{4.\  } & أَوْ يَذَّكَّرُ فَتَنفَعَهُ ٱلذِّكْرَىٰٓ ﴿٤﴾\\
\textamh{5.\  } & أَمَّا مَنِ ٱسْتَغْنَىٰ ﴿٥﴾\\
\textamh{6.\  } & فَأَنتَ لَهُۥ تَصَدَّىٰ ﴿٦﴾\\
\textamh{7.\  } & وَمَا عَلَيْكَ أَلَّا يَزَّكَّىٰ ﴿٧﴾\\
\textamh{8.\  } & وَأَمَّا مَن جَآءَكَ يَسْعَىٰ ﴿٨﴾\\
\textamh{9.\  } & وَهُوَ يَخْشَىٰ ﴿٩﴾\\
\textamh{10.\  } & فَأَنتَ عَنْهُ تَلَهَّىٰ ﴿١٠﴾\\
\textamh{11.\  } & كَلَّآ إِنَّهَا تَذْكِرَةٌۭ ﴿١١﴾\\
\textamh{12.\  } & فَمَن شَآءَ ذَكَرَهُۥ ﴿١٢﴾\\
\textamh{13.\  } & فِى صُحُفٍۢ مُّكَرَّمَةٍۢ ﴿١٣﴾\\
\textamh{14.\  } & مَّرْفُوعَةٍۢ مُّطَهَّرَةٍۭ ﴿١٤﴾\\
\textamh{15.\  } & بِأَيْدِى سَفَرَةٍۢ ﴿١٥﴾\\
\textamh{16.\  } & كِرَامٍۭ بَرَرَةٍۢ ﴿١٦﴾\\
\textamh{17.\  } & قُتِلَ ٱلْإِنسَـٰنُ مَآ أَكْفَرَهُۥ ﴿١٧﴾\\
\textamh{18.\  } & مِنْ أَىِّ شَىْءٍ خَلَقَهُۥ ﴿١٨﴾\\
\textamh{19.\  } & مِن نُّطْفَةٍ خَلَقَهُۥ فَقَدَّرَهُۥ ﴿١٩﴾\\
\textamh{20.\  } & ثُمَّ ٱلسَّبِيلَ يَسَّرَهُۥ ﴿٢٠﴾\\
\textamh{21.\  } & ثُمَّ أَمَاتَهُۥ فَأَقْبَرَهُۥ ﴿٢١﴾\\
\textamh{22.\  } & ثُمَّ إِذَا شَآءَ أَنشَرَهُۥ ﴿٢٢﴾\\
\textamh{23.\  } & كَلَّا لَمَّا يَقْضِ مَآ أَمَرَهُۥ ﴿٢٣﴾\\
\textamh{24.\  } & فَلْيَنظُرِ ٱلْإِنسَـٰنُ إِلَىٰ طَعَامِهِۦٓ ﴿٢٤﴾\\
\textamh{25.\  } & أَنَّا صَبَبْنَا ٱلْمَآءَ صَبًّۭا ﴿٢٥﴾\\
\textamh{26.\  } & ثُمَّ شَقَقْنَا ٱلْأَرْضَ شَقًّۭا ﴿٢٦﴾\\
\textamh{27.\  } & فَأَنۢبَتْنَا فِيهَا حَبًّۭا ﴿٢٧﴾\\
\textamh{28.\  } & وَعِنَبًۭا وَقَضْبًۭا ﴿٢٨﴾\\
\textamh{29.\  } & وَزَيْتُونًۭا وَنَخْلًۭا ﴿٢٩﴾\\
\textamh{30.\  } & وَحَدَآئِقَ غُلْبًۭا ﴿٣٠﴾\\
\textamh{31.\  } & وَفَـٰكِهَةًۭ وَأَبًّۭا ﴿٣١﴾\\
\textamh{32.\  } & مَّتَـٰعًۭا لَّكُمْ وَلِأَنْعَـٰمِكُمْ ﴿٣٢﴾\\
\textamh{33.\  } & فَإِذَا جَآءَتِ ٱلصَّآخَّةُ ﴿٣٣﴾\\
\textamh{34.\  } & يَوْمَ يَفِرُّ ٱلْمَرْءُ مِنْ أَخِيهِ ﴿٣٤﴾\\
\textamh{35.\  } & وَأُمِّهِۦ وَأَبِيهِ ﴿٣٥﴾\\
\textamh{36.\  } & وَصَـٰحِبَتِهِۦ وَبَنِيهِ ﴿٣٦﴾\\
\textamh{37.\  } & لِكُلِّ ٱمْرِئٍۢ مِّنْهُمْ يَوْمَئِذٍۢ شَأْنٌۭ يُغْنِيهِ ﴿٣٧﴾\\
\textamh{38.\  } & وُجُوهٌۭ يَوْمَئِذٍۢ مُّسْفِرَةٌۭ ﴿٣٨﴾\\
\textamh{39.\  } & ضَاحِكَةٌۭ مُّسْتَبْشِرَةٌۭ ﴿٣٩﴾\\
\textamh{40.\  } & وَوُجُوهٌۭ يَوْمَئِذٍ عَلَيْهَا غَبَرَةٌۭ ﴿٤٠﴾\\
\textamh{41.\  } & تَرْهَقُهَا قَتَرَةٌ ﴿٤١﴾\\
\textamh{42.\  } & أُو۟لَـٰٓئِكَ هُمُ ٱلْكَفَرَةُ ٱلْفَجَرَةُ ﴿٤٢﴾\\
\end{longtable} \newpage

%% License: BSD style (Berkley) (i.e. Put the Copyright owner's name always)
%% Writer and Copyright (to): Bewketu(Bilal) Tadilo (2016-17)
\centering\section{\LR{\textamharic{ሱራቱ አትተካዊያር -}  \RL{سوره  التكوير}}}
\begin{longtable}{%
  @{}
    p{.5\textwidth}
  @{~~~~~~~~~~~~~}
    p{.5\textwidth}
    @{}
}
\nopagebreak
\textamh{\ \ \ \ \ \  ቢስሚላሂ አራህመኒ ራሂይም } &  بِسْمِ ٱللَّهِ ٱلرَّحْمَـٰنِ ٱلرَّحِيمِ\\
\textamh{1.\  } &  إِذَا ٱلشَّمْسُ كُوِّرَتْ ﴿١﴾\\
\textamh{2.\  } & وَإِذَا ٱلنُّجُومُ ٱنكَدَرَتْ ﴿٢﴾\\
\textamh{3.\  } & وَإِذَا ٱلْجِبَالُ سُيِّرَتْ ﴿٣﴾\\
\textamh{4.\  } & وَإِذَا ٱلْعِشَارُ عُطِّلَتْ ﴿٤﴾\\
\textamh{5.\  } & وَإِذَا ٱلْوُحُوشُ حُشِرَتْ ﴿٥﴾\\
\textamh{6.\  } & وَإِذَا ٱلْبِحَارُ سُجِّرَتْ ﴿٦﴾\\
\textamh{7.\  } & وَإِذَا ٱلنُّفُوسُ زُوِّجَتْ ﴿٧﴾\\
\textamh{8.\  } & وَإِذَا ٱلْمَوْءُۥدَةُ سُئِلَتْ ﴿٨﴾\\
\textamh{9.\  } & بِأَىِّ ذَنۢبٍۢ قُتِلَتْ ﴿٩﴾\\
\textamh{10.\  } & وَإِذَا ٱلصُّحُفُ نُشِرَتْ ﴿١٠﴾\\
\textamh{11.\  } & وَإِذَا ٱلسَّمَآءُ كُشِطَتْ ﴿١١﴾\\
\textamh{12.\  } & وَإِذَا ٱلْجَحِيمُ سُعِّرَتْ ﴿١٢﴾\\
\textamh{13.\  } & وَإِذَا ٱلْجَنَّةُ أُزْلِفَتْ ﴿١٣﴾\\
\textamh{14.\  } & عَلِمَتْ نَفْسٌۭ مَّآ أَحْضَرَتْ ﴿١٤﴾\\
\textamh{15.\  } & فَلَآ أُقْسِمُ بِٱلْخُنَّسِ ﴿١٥﴾\\
\textamh{16.\  } & ٱلْجَوَارِ ٱلْكُنَّسِ ﴿١٦﴾\\
\textamh{17.\  } & وَٱلَّيْلِ إِذَا عَسْعَسَ ﴿١٧﴾\\
\textamh{18.\  } & وَٱلصُّبْحِ إِذَا تَنَفَّسَ ﴿١٨﴾\\
\textamh{19.\  } & إِنَّهُۥ لَقَوْلُ رَسُولٍۢ كَرِيمٍۢ ﴿١٩﴾\\
\textamh{20.\  } & ذِى قُوَّةٍ عِندَ ذِى ٱلْعَرْشِ مَكِينٍۢ ﴿٢٠﴾\\
\textamh{21.\  } & مُّطَاعٍۢ ثَمَّ أَمِينٍۢ ﴿٢١﴾\\
\textamh{22.\  } & وَمَا صَاحِبُكُم بِمَجْنُونٍۢ ﴿٢٢﴾\\
\textamh{23.\  } & وَلَقَدْ رَءَاهُ بِٱلْأُفُقِ ٱلْمُبِينِ ﴿٢٣﴾\\
\textamh{24.\  } & وَمَا هُوَ عَلَى ٱلْغَيْبِ بِضَنِينٍۢ ﴿٢٤﴾\\
\textamh{25.\  } & وَمَا هُوَ بِقَوْلِ شَيْطَٰنٍۢ رَّجِيمٍۢ ﴿٢٥﴾\\
\textamh{26.\  } & فَأَيْنَ تَذْهَبُونَ ﴿٢٦﴾\\
\textamh{27.\  } & إِنْ هُوَ إِلَّا ذِكْرٌۭ لِّلْعَـٰلَمِينَ ﴿٢٧﴾\\
\textamh{28.\  } & لِمَن شَآءَ مِنكُمْ أَن يَسْتَقِيمَ ﴿٢٨﴾\\
\textamh{29.\  } & وَمَا تَشَآءُونَ إِلَّآ أَن يَشَآءَ ٱللَّهُ رَبُّ ٱلْعَـٰلَمِينَ ﴿٢٩﴾\\
\end{longtable} \newpage

%% License: BSD style (Berkley) (i.e. Put the Copyright owner's name always)
%% Writer and Copyright (to): Bewketu(Bilal) Tadilo (2016-17)
\begin{center}\section{\LR{\textamhsec{ሱራቱ አልአንፊጣሪ -}  \textarabic{سوره  الإنفطار}}}\end{center}
\begin{longtable}{%
  @{}
    p{.5\textwidth}
  @{~~~}
    p{.5\textwidth}
    @{}
}
\textamh{ቢስሚላሂ አራህመኒ ራሂይም } &  \mytextarabic{بِسْمِ ٱللَّهِ ٱلرَّحْمَـٰنِ ٱلرَّحِيمِ}\\
\textamh{1.\  } & \mytextarabic{ إِذَا ٱلسَّمَآءُ ٱنفَطَرَتْ ﴿١﴾}\\
\textamh{2.\  } & \mytextarabic{وَإِذَا ٱلْكَوَاكِبُ ٱنتَثَرَتْ ﴿٢﴾}\\
\textamh{3.\  } & \mytextarabic{وَإِذَا ٱلْبِحَارُ فُجِّرَتْ ﴿٣﴾}\\
\textamh{4.\  } & \mytextarabic{وَإِذَا ٱلْقُبُورُ بُعْثِرَتْ ﴿٤﴾}\\
\textamh{5.\  } & \mytextarabic{عَلِمَتْ نَفْسٌۭ مَّا قَدَّمَتْ وَأَخَّرَتْ ﴿٥﴾}\\
\textamh{6.\  } & \mytextarabic{يَـٰٓأَيُّهَا ٱلْإِنسَـٰنُ مَا غَرَّكَ بِرَبِّكَ ٱلْكَرِيمِ ﴿٦﴾}\\
\textamh{7.\  } & \mytextarabic{ٱلَّذِى خَلَقَكَ فَسَوَّىٰكَ فَعَدَلَكَ ﴿٧﴾}\\
\textamh{8.\  } & \mytextarabic{فِىٓ أَىِّ صُورَةٍۢ مَّا شَآءَ رَكَّبَكَ ﴿٨﴾}\\
\textamh{9.\  } & \mytextarabic{كَلَّا بَلْ تُكَذِّبُونَ بِٱلدِّينِ ﴿٩﴾}\\
\textamh{10.\  } & \mytextarabic{وَإِنَّ عَلَيْكُمْ لَحَـٰفِظِينَ ﴿١٠﴾}\\
\textamh{11.\  } & \mytextarabic{كِرَامًۭا كَـٰتِبِينَ ﴿١١﴾}\\
\textamh{12.\  } & \mytextarabic{يَعْلَمُونَ مَا تَفْعَلُونَ ﴿١٢﴾}\\
\textamh{13.\  } & \mytextarabic{إِنَّ ٱلْأَبْرَارَ لَفِى نَعِيمٍۢ ﴿١٣﴾}\\
\textamh{14.\  } & \mytextarabic{وَإِنَّ ٱلْفُجَّارَ لَفِى جَحِيمٍۢ ﴿١٤﴾}\\
\textamh{15.\  } & \mytextarabic{يَصْلَوْنَهَا يَوْمَ ٱلدِّينِ ﴿١٥﴾}\\
\textamh{16.\  } & \mytextarabic{وَمَا هُمْ عَنْهَا بِغَآئِبِينَ ﴿١٦﴾}\\
\textamh{17.\  } & \mytextarabic{وَمَآ أَدْرَىٰكَ مَا يَوْمُ ٱلدِّينِ ﴿١٧﴾}\\
\textamh{18.\  } & \mytextarabic{ثُمَّ مَآ أَدْرَىٰكَ مَا يَوْمُ ٱلدِّينِ ﴿١٨﴾}\\
\textamh{19.\  } & \mytextarabic{يَوْمَ لَا تَمْلِكُ نَفْسٌۭ لِّنَفْسٍۢ شَيْـًۭٔا ۖ وَٱلْأَمْرُ يَوْمَئِذٍۢ لِّلَّهِ ﴿١٩﴾}\\
\end{longtable}
\clearpage
%% License: BSD style (Berkley) (i.e. Put the Copyright owner's name always)
%% Writer and Copyright (to): Bewketu(Bilal) Tadilo (2016-17)
\centering\section{\LR{\textamharic{ሱራቱ አልሙጠፊፊይን -}  \RL{سوره  المطففين}}}
\begin{longtable}{%
  @{}
    p{.5\textwidth}
  @{~~~~~~~~~~~~}
    p{.5\textwidth}
    @{}
}
\nopagebreak
\textamh{ቢስሚላሂ አራህመኒ ራሂይም } &  بِسْمِ ٱللَّهِ ٱلرَّحْمَـٰنِ ٱلرَّحِيمِ\\
\textamh{1.\  } &  وَيْلٌۭ لِّلْمُطَفِّفِينَ ﴿١﴾\\
\textamh{2.\  } & ٱلَّذِينَ إِذَا ٱكْتَالُوا۟ عَلَى ٱلنَّاسِ يَسْتَوْفُونَ ﴿٢﴾\\
\textamh{3.\  } & وَإِذَا كَالُوهُمْ أَو وَّزَنُوهُمْ يُخْسِرُونَ ﴿٣﴾\\
\textamh{4.\  } & أَلَا يَظُنُّ أُو۟لَـٰٓئِكَ أَنَّهُم مَّبْعُوثُونَ ﴿٤﴾\\
\textamh{5.\  } & لِيَوْمٍ عَظِيمٍۢ ﴿٥﴾\\
\textamh{6.\  } & يَوْمَ يَقُومُ ٱلنَّاسُ لِرَبِّ ٱلْعَـٰلَمِينَ ﴿٦﴾\\
\textamh{7.\  } & كَلَّآ إِنَّ كِتَـٰبَ ٱلْفُجَّارِ لَفِى سِجِّينٍۢ ﴿٧﴾\\
\textamh{8.\  } & وَمَآ أَدْرَىٰكَ مَا سِجِّينٌۭ ﴿٨﴾\\
\textamh{9.\  } & كِتَـٰبٌۭ مَّرْقُومٌۭ ﴿٩﴾\\
\textamh{10.\  } & وَيْلٌۭ يَوْمَئِذٍۢ لِّلْمُكَذِّبِينَ ﴿١٠﴾\\
\textamh{11.\  } & ٱلَّذِينَ يُكَذِّبُونَ بِيَوْمِ ٱلدِّينِ ﴿١١﴾\\
\textamh{12.\  } & وَمَا يُكَذِّبُ بِهِۦٓ إِلَّا كُلُّ مُعْتَدٍ أَثِيمٍ ﴿١٢﴾\\
\textamh{13.\  } & إِذَا تُتْلَىٰ عَلَيْهِ ءَايَـٰتُنَا قَالَ أَسَـٰطِيرُ ٱلْأَوَّلِينَ ﴿١٣﴾\\
\textamh{14.\  } & كَلَّا ۖ بَلْ ۜ رَانَ عَلَىٰ قُلُوبِهِم مَّا كَانُوا۟ يَكْسِبُونَ ﴿١٤﴾\\
\textamh{15.\  } & كَلَّآ إِنَّهُمْ عَن رَّبِّهِمْ يَوْمَئِذٍۢ لَّمَحْجُوبُونَ ﴿١٥﴾\\
\textamh{16.\  } & ثُمَّ إِنَّهُمْ لَصَالُوا۟ ٱلْجَحِيمِ ﴿١٦﴾\\
\textamh{17.\  } & ثُمَّ يُقَالُ هَـٰذَا ٱلَّذِى كُنتُم بِهِۦ تُكَذِّبُونَ ﴿١٧﴾\\
\textamh{18.\  } & كَلَّآ إِنَّ كِتَـٰبَ ٱلْأَبْرَارِ لَفِى عِلِّيِّينَ ﴿١٨﴾\\
\textamh{19.\  } & وَمَآ أَدْرَىٰكَ مَا عِلِّيُّونَ ﴿١٩﴾\\
\textamh{20.\  } & كِتَـٰبٌۭ مَّرْقُومٌۭ ﴿٢٠﴾\\
\textamh{21.\  } & يَشْهَدُهُ ٱلْمُقَرَّبُونَ ﴿٢١﴾\\
\textamh{22.\  } & إِنَّ ٱلْأَبْرَارَ لَفِى نَعِيمٍ ﴿٢٢﴾\\
\textamh{23.\  } & عَلَى ٱلْأَرَآئِكِ يَنظُرُونَ ﴿٢٣﴾\\
\textamh{24.\  } & تَعْرِفُ فِى وُجُوهِهِمْ نَضْرَةَ ٱلنَّعِيمِ ﴿٢٤﴾\\
\textamh{25.\  } & يُسْقَوْنَ مِن رَّحِيقٍۢ مَّخْتُومٍ ﴿٢٥﴾\\
\textamh{26.\  } & خِتَـٰمُهُۥ مِسْكٌۭ ۚ وَفِى ذَٟلِكَ فَلْيَتَنَافَسِ ٱلْمُتَنَـٰفِسُونَ ﴿٢٦﴾\\
\textamh{27.\  } & وَمِزَاجُهُۥ مِن تَسْنِيمٍ ﴿٢٧﴾\\
\textamh{28.\  } & عَيْنًۭا يَشْرَبُ بِهَا ٱلْمُقَرَّبُونَ ﴿٢٨﴾\\
\textamh{29.\  } & إِنَّ ٱلَّذِينَ أَجْرَمُوا۟ كَانُوا۟ مِنَ ٱلَّذِينَ ءَامَنُوا۟ يَضْحَكُونَ ﴿٢٩﴾\\
\textamh{30.\  } & وَإِذَا مَرُّوا۟ بِهِمْ يَتَغَامَزُونَ ﴿٣٠﴾\\
\textamh{31.\  } & وَإِذَا ٱنقَلَبُوٓا۟ إِلَىٰٓ أَهْلِهِمُ ٱنقَلَبُوا۟ فَكِهِينَ ﴿٣١﴾\\
\textamh{32.\  } & وَإِذَا رَأَوْهُمْ قَالُوٓا۟ إِنَّ هَـٰٓؤُلَآءِ لَضَآلُّونَ ﴿٣٢﴾\\
\textamh{33.\  } & وَمَآ أُرْسِلُوا۟ عَلَيْهِمْ حَـٰفِظِينَ ﴿٣٣﴾\\
\textamh{34.\  } & فَٱلْيَوْمَ ٱلَّذِينَ ءَامَنُوا۟ مِنَ ٱلْكُفَّارِ يَضْحَكُونَ ﴿٣٤﴾\\
\textamh{35.\  } & عَلَى ٱلْأَرَآئِكِ يَنظُرُونَ ﴿٣٥﴾\\
\textamh{36.\  } & هَلْ ثُوِّبَ ٱلْكُفَّارُ مَا كَانُوا۟ يَفْعَلُونَ ﴿٣٦﴾\\
\end{longtable}
\clearpage
%% License: BSD style (Berkley) (i.e. Put the Copyright owner's name always)
%% Writer and Copyright (to): Bewketu(Bilal) Tadilo (2016-17)
\begin{center}\section{\LR{\textamhsec{ሱራቱ አልአነሺቃቅ -}  \textarabic{سوره  الإنشقاق}}}\end{center}
\begin{longtable}{%
  @{}
    p{.5\textwidth}
  @{~~~}
    p{.5\textwidth}
    @{}
}
\textamh{ቢስሚላሂ አራህመኒ ራሂይም } &  \mytextarabic{بِسْمِ ٱللَّهِ ٱلرَّحْمَـٰنِ ٱلرَّحِيمِ}\\
\textamh{1.\  } & \mytextarabic{ إِذَا ٱلسَّمَآءُ ٱنشَقَّتْ ﴿١﴾}\\
\textamh{2.\  } & \mytextarabic{وَأَذِنَتْ لِرَبِّهَا وَحُقَّتْ ﴿٢﴾}\\
\textamh{3.\  } & \mytextarabic{وَإِذَا ٱلْأَرْضُ مُدَّتْ ﴿٣﴾}\\
\textamh{4.\  } & \mytextarabic{وَأَلْقَتْ مَا فِيهَا وَتَخَلَّتْ ﴿٤﴾}\\
\textamh{5.\  } & \mytextarabic{وَأَذِنَتْ لِرَبِّهَا وَحُقَّتْ ﴿٥﴾}\\
\textamh{6.\  } & \mytextarabic{يَـٰٓأَيُّهَا ٱلْإِنسَـٰنُ إِنَّكَ كَادِحٌ إِلَىٰ رَبِّكَ كَدْحًۭا فَمُلَـٰقِيهِ ﴿٦﴾}\\
\textamh{7.\  } & \mytextarabic{فَأَمَّا مَنْ أُوتِىَ كِتَـٰبَهُۥ بِيَمِينِهِۦ ﴿٧﴾}\\
\textamh{8.\  } & \mytextarabic{فَسَوْفَ يُحَاسَبُ حِسَابًۭا يَسِيرًۭا ﴿٨﴾}\\
\textamh{9.\  } & \mytextarabic{وَيَنقَلِبُ إِلَىٰٓ أَهْلِهِۦ مَسْرُورًۭا ﴿٩﴾}\\
\textamh{10.\  } & \mytextarabic{وَأَمَّا مَنْ أُوتِىَ كِتَـٰبَهُۥ وَرَآءَ ظَهْرِهِۦ ﴿١٠﴾}\\
\textamh{11.\  } & \mytextarabic{فَسَوْفَ يَدْعُوا۟ ثُبُورًۭا ﴿١١﴾}\\
\textamh{12.\  } & \mytextarabic{وَيَصْلَىٰ سَعِيرًا ﴿١٢﴾}\\
\textamh{13.\  } & \mytextarabic{إِنَّهُۥ كَانَ فِىٓ أَهْلِهِۦ مَسْرُورًا ﴿١٣﴾}\\
\textamh{14.\  } & \mytextarabic{إِنَّهُۥ ظَنَّ أَن لَّن يَحُورَ ﴿١٤﴾}\\
\textamh{15.\  } & \mytextarabic{بَلَىٰٓ إِنَّ رَبَّهُۥ كَانَ بِهِۦ بَصِيرًۭا ﴿١٥﴾}\\
\textamh{16.\  } & \mytextarabic{فَلَآ أُقْسِمُ بِٱلشَّفَقِ ﴿١٦﴾}\\
\textamh{17.\  } & \mytextarabic{وَٱلَّيْلِ وَمَا وَسَقَ ﴿١٧﴾}\\
\textamh{18.\  } & \mytextarabic{وَٱلْقَمَرِ إِذَا ٱتَّسَقَ ﴿١٨﴾}\\
\textamh{19.\  } & \mytextarabic{لَتَرْكَبُنَّ طَبَقًا عَن طَبَقٍۢ ﴿١٩﴾}\\
\textamh{20.\  } & \mytextarabic{فَمَا لَهُمْ لَا يُؤْمِنُونَ ﴿٢٠﴾}\\
\textamh{21.\  } & \mytextarabic{وَإِذَا قُرِئَ عَلَيْهِمُ ٱلْقُرْءَانُ لَا يَسْجُدُونَ ۩ ﴿٢١﴾}\\
\textamh{22.\  } & \mytextarabic{بَلِ ٱلَّذِينَ كَفَرُوا۟ يُكَذِّبُونَ ﴿٢٢﴾}\\
\textamh{23.\  } & \mytextarabic{وَٱللَّهُ أَعْلَمُ بِمَا يُوعُونَ ﴿٢٣﴾}\\
\textamh{24.\  } & \mytextarabic{فَبَشِّرْهُم بِعَذَابٍ أَلِيمٍ ﴿٢٤﴾}\\
\textamh{25.\  } & \mytextarabic{إِلَّا ٱلَّذِينَ ءَامَنُوا۟ وَعَمِلُوا۟ ٱلصَّـٰلِحَـٰتِ لَهُمْ أَجْرٌ غَيْرُ مَمْنُونٍۭ ﴿٢٥﴾}\\
\end{longtable}
\clearpage
%% License: BSD style (Berkley) (i.e. Put the Copyright owner's name always)
%% Writer and Copyright (to): Bewketu(Bilal) Tadilo (2016-17)
\centering\section{\LR{\textamharic{ሱራቱ አልቡሩዉጅ -}  \RL{سوره  البروج}}}
\begin{longtable}{%
  @{}
    p{.5\textwidth}
  @{~~~~~~~~~~~~~}
    p{.5\textwidth}
    @{}
}
\nopagebreak
\textamh{\ \ \ \ \ \  ቢስሚላሂ አራህመኒ ራሂይም } &  بِسْمِ ٱللَّهِ ٱلرَّحْمَـٰنِ ٱلرَّحِيمِ\\
\textamh{1.\  } &  وَٱلسَّمَآءِ ذَاتِ ٱلْبُرُوجِ ﴿١﴾\\
\textamh{2.\  } & وَٱلْيَوْمِ ٱلْمَوْعُودِ ﴿٢﴾\\
\textamh{3.\  } & وَشَاهِدٍۢ وَمَشْهُودٍۢ ﴿٣﴾\\
\textamh{4.\  } & قُتِلَ أَصْحَـٰبُ ٱلْأُخْدُودِ ﴿٤﴾\\
\textamh{5.\  } & ٱلنَّارِ ذَاتِ ٱلْوَقُودِ ﴿٥﴾\\
\textamh{6.\  } & إِذْ هُمْ عَلَيْهَا قُعُودٌۭ ﴿٦﴾\\
\textamh{7.\  } & وَهُمْ عَلَىٰ مَا يَفْعَلُونَ بِٱلْمُؤْمِنِينَ شُهُودٌۭ ﴿٧﴾\\
\textamh{8.\  } & وَمَا نَقَمُوا۟ مِنْهُمْ إِلَّآ أَن يُؤْمِنُوا۟ بِٱللَّهِ ٱلْعَزِيزِ ٱلْحَمِيدِ ﴿٨﴾\\
\textamh{9.\  } & ٱلَّذِى لَهُۥ مُلْكُ ٱلسَّمَـٰوَٟتِ وَٱلْأَرْضِ ۚ وَٱللَّهُ عَلَىٰ كُلِّ شَىْءٍۢ شَهِيدٌ ﴿٩﴾\\
\textamh{10.\  } & إِنَّ ٱلَّذِينَ فَتَنُوا۟ ٱلْمُؤْمِنِينَ وَٱلْمُؤْمِنَـٰتِ ثُمَّ لَمْ يَتُوبُوا۟ فَلَهُمْ عَذَابُ جَهَنَّمَ وَلَهُمْ عَذَابُ ٱلْحَرِيقِ ﴿١٠﴾\\
\textamh{11.\  } & إِنَّ ٱلَّذِينَ ءَامَنُوا۟ وَعَمِلُوا۟ ٱلصَّـٰلِحَـٰتِ لَهُمْ جَنَّـٰتٌۭ تَجْرِى مِن تَحْتِهَا ٱلْأَنْهَـٰرُ ۚ ذَٟلِكَ ٱلْفَوْزُ ٱلْكَبِيرُ ﴿١١﴾\\
\textamh{12.\  } & إِنَّ بَطْشَ رَبِّكَ لَشَدِيدٌ ﴿١٢﴾\\
\textamh{13.\  } & إِنَّهُۥ هُوَ يُبْدِئُ وَيُعِيدُ ﴿١٣﴾\\
\textamh{14.\  } & وَهُوَ ٱلْغَفُورُ ٱلْوَدُودُ ﴿١٤﴾\\
\textamh{15.\  } & ذُو ٱلْعَرْشِ ٱلْمَجِيدُ ﴿١٥﴾\\
\textamh{16.\  } & فَعَّالٌۭ لِّمَا يُرِيدُ ﴿١٦﴾\\
\textamh{17.\  } & هَلْ أَتَىٰكَ حَدِيثُ ٱلْجُنُودِ ﴿١٧﴾\\
\textamh{18.\  } & فِرْعَوْنَ وَثَمُودَ ﴿١٨﴾\\
\textamh{19.\  } & بَلِ ٱلَّذِينَ كَفَرُوا۟ فِى تَكْذِيبٍۢ ﴿١٩﴾\\
\textamh{20.\  } & وَٱللَّهُ مِن وَرَآئِهِم مُّحِيطٌۢ ﴿٢٠﴾\\
\textamh{21.\  } & بَلْ هُوَ قُرْءَانٌۭ مَّجِيدٌۭ ﴿٢١﴾\\
\textamh{22.\  } & فِى لَوْحٍۢ مَّحْفُوظٍۭ ﴿٢٢﴾\\
\end{longtable} \newpage

%% License: BSD style (Berkley) (i.e. Put the Copyright owner's name always)
%% Writer and Copyright (to): Bewketu(Bilal) Tadilo (2016-17)
\centering\section{\LR{\textamharic{ሱራቱ አጥጣሪቂ -}  \RL{سوره  الطارق}}}
\begin{longtable}{%
  @{}
    p{.5\textwidth}
  @{~~~~~~~~~~~~~}
    p{.5\textwidth}
    @{}
}
\nopagebreak
\textamh{ቢስሚላሂ አራህመኒ ራሂይም } &  بِسْمِ ٱللَّهِ ٱلرَّحْمَـٰنِ ٱلرَّحِيمِ\\
\textamh{1.\  } &  وَٱلسَّمَآءِ وَٱلطَّارِقِ ﴿١﴾\\
\textamh{2.\  } & وَمَآ أَدْرَىٰكَ مَا ٱلطَّارِقُ ﴿٢﴾\\
\textamh{3.\  } & ٱلنَّجْمُ ٱلثَّاقِبُ ﴿٣﴾\\
\textamh{4.\  } & إِن كُلُّ نَفْسٍۢ لَّمَّا عَلَيْهَا حَافِظٌۭ ﴿٤﴾\\
\textamh{5.\  } & فَلْيَنظُرِ ٱلْإِنسَـٰنُ مِمَّ خُلِقَ ﴿٥﴾\\
\textamh{6.\  } & خُلِقَ مِن مَّآءٍۢ دَافِقٍۢ ﴿٦﴾\\
\textamh{7.\  } & يَخْرُجُ مِنۢ بَيْنِ ٱلصُّلْبِ وَٱلتَّرَآئِبِ ﴿٧﴾\\
\textamh{8.\  } & إِنَّهُۥ عَلَىٰ رَجْعِهِۦ لَقَادِرٌۭ ﴿٨﴾\\
\textamh{9.\  } & يَوْمَ تُبْلَى ٱلسَّرَآئِرُ ﴿٩﴾\\
\textamh{10.\  } & فَمَا لَهُۥ مِن قُوَّةٍۢ وَلَا نَاصِرٍۢ ﴿١٠﴾\\
\textamh{11.\  } & وَٱلسَّمَآءِ ذَاتِ ٱلرَّجْعِ ﴿١١﴾\\
\textamh{12.\  } & وَٱلْأَرْضِ ذَاتِ ٱلصَّدْعِ ﴿١٢﴾\\
\textamh{13.\  } & إِنَّهُۥ لَقَوْلٌۭ فَصْلٌۭ ﴿١٣﴾\\
\textamh{14.\  } & وَمَا هُوَ بِٱلْهَزْلِ ﴿١٤﴾\\
\textamh{15.\  } & إِنَّهُمْ يَكِيدُونَ كَيْدًۭا ﴿١٥﴾\\
\textamh{16.\  } & وَأَكِيدُ كَيْدًۭا ﴿١٦﴾\\
\textamh{17.\  } & فَمَهِّلِ ٱلْكَـٰفِرِينَ أَمْهِلْهُمْ رُوَيْدًۢا ﴿١٧﴾\\
\end{longtable}
\clearpage
%% License: BSD style (Berkley) (i.e. Put the Copyright owner's name always)
%% Writer and Copyright (to): Bewketu(Bilal) Tadilo (2016-17)
\centering\section{\LR{\textamharic{ሱራቱ አልኣእላ -}  \RL{سوره  الأعلى}}}
\begin{longtable}{%
  @{}
    p{.5\textwidth}
  @{~~~~~~~~~~~~}
    p{.5\textwidth}
    @{}
}
\nopagebreak
\textamh{ቢስሚላሂ አራህመኒ ራሂይም } &  بِسْمِ ٱللَّهِ ٱلرَّحْمَـٰنِ ٱلرَّحِيمِ\\
\textamh{1.\  } &  سَبِّحِ ٱسْمَ رَبِّكَ ٱلْأَعْلَى ﴿١﴾\\
\textamh{2.\  } & ٱلَّذِى خَلَقَ فَسَوَّىٰ ﴿٢﴾\\
\textamh{3.\  } & وَٱلَّذِى قَدَّرَ فَهَدَىٰ ﴿٣﴾\\
\textamh{4.\  } & وَٱلَّذِىٓ أَخْرَجَ ٱلْمَرْعَىٰ ﴿٤﴾\\
\textamh{5.\  } & فَجَعَلَهُۥ غُثَآءً أَحْوَىٰ ﴿٥﴾\\
\textamh{6.\  } & سَنُقْرِئُكَ فَلَا تَنسَىٰٓ ﴿٦﴾\\
\textamh{7.\  } & إِلَّا مَا شَآءَ ٱللَّهُ ۚ إِنَّهُۥ يَعْلَمُ ٱلْجَهْرَ وَمَا يَخْفَىٰ ﴿٧﴾\\
\textamh{8.\  } & وَنُيَسِّرُكَ لِلْيُسْرَىٰ ﴿٨﴾\\
\textamh{9.\  } & فَذَكِّرْ إِن نَّفَعَتِ ٱلذِّكْرَىٰ ﴿٩﴾\\
\textamh{10.\  } & سَيَذَّكَّرُ مَن يَخْشَىٰ ﴿١٠﴾\\
\textamh{11.\  } & وَيَتَجَنَّبُهَا ٱلْأَشْقَى ﴿١١﴾\\
\textamh{12.\  } & ٱلَّذِى يَصْلَى ٱلنَّارَ ٱلْكُبْرَىٰ ﴿١٢﴾\\
\textamh{13.\  } & ثُمَّ لَا يَمُوتُ فِيهَا وَلَا يَحْيَىٰ ﴿١٣﴾\\
\textamh{14.\  } & قَدْ أَفْلَحَ مَن تَزَكَّىٰ ﴿١٤﴾\\
\textamh{15.\  } & وَذَكَرَ ٱسْمَ رَبِّهِۦ فَصَلَّىٰ ﴿١٥﴾\\
\textamh{16.\  } & بَلْ تُؤْثِرُونَ ٱلْحَيَوٰةَ ٱلدُّنْيَا ﴿١٦﴾\\
\textamh{17.\  } & وَٱلْءَاخِرَةُ خَيْرٌۭ وَأَبْقَىٰٓ ﴿١٧﴾\\
\textamh{18.\  } & إِنَّ هَـٰذَا لَفِى ٱلصُّحُفِ ٱلْأُولَىٰ ﴿١٨﴾\\
\textamh{19.\  } & صُحُفِ إِبْرَٰهِيمَ وَمُوسَىٰ ﴿١٩﴾\\
\end{longtable}
\clearpage
%% License: BSD style (Berkley) (i.e. Put the Copyright owner's name always)
%% Writer and Copyright (to): Bewketu(Bilal) Tadilo (2016-17)
\centering\section{\LR{\textamharic{ሱራቱ አልጋሺያ -}  \RL{سوره  الغاشية}}}
\begin{longtable}{%
  @{}
    p{.5\textwidth}
  @{~~~~~~~~~~~~~}
    p{.5\textwidth}
    @{}
}
\nopagebreak
\textamh{\ \ \ \ \ \  ቢስሚላሂ አራህመኒ ራሂይም } &  بِسْمِ ٱللَّهِ ٱلرَّحْمَـٰنِ ٱلرَّحِيمِ\\
\textamh{1.\  } &  هَلْ أَتَىٰكَ حَدِيثُ ٱلْغَٰشِيَةِ ﴿١﴾\\
\textamh{2.\  } & وُجُوهٌۭ يَوْمَئِذٍ خَـٰشِعَةٌ ﴿٢﴾\\
\textamh{3.\  } & عَامِلَةٌۭ نَّاصِبَةٌۭ ﴿٣﴾\\
\textamh{4.\  } & تَصْلَىٰ نَارًا حَامِيَةًۭ ﴿٤﴾\\
\textamh{5.\  } & تُسْقَىٰ مِنْ عَيْنٍ ءَانِيَةٍۢ ﴿٥﴾\\
\textamh{6.\  } & لَّيْسَ لَهُمْ طَعَامٌ إِلَّا مِن ضَرِيعٍۢ ﴿٦﴾\\
\textamh{7.\  } & لَّا يُسْمِنُ وَلَا يُغْنِى مِن جُوعٍۢ ﴿٧﴾\\
\textamh{8.\  } & وُجُوهٌۭ يَوْمَئِذٍۢ نَّاعِمَةٌۭ ﴿٨﴾\\
\textamh{9.\  } & لِّسَعْيِهَا رَاضِيَةٌۭ ﴿٩﴾\\
\textamh{10.\  } & فِى جَنَّةٍ عَالِيَةٍۢ ﴿١٠﴾\\
\textamh{11.\  } & لَّا تَسْمَعُ فِيهَا لَـٰغِيَةًۭ ﴿١١﴾\\
\textamh{12.\  } & فِيهَا عَيْنٌۭ جَارِيَةٌۭ ﴿١٢﴾\\
\textamh{13.\  } & فِيهَا سُرُرٌۭ مَّرْفُوعَةٌۭ ﴿١٣﴾\\
\textamh{14.\  } & وَأَكْوَابٌۭ مَّوْضُوعَةٌۭ ﴿١٤﴾\\
\textamh{15.\  } & وَنَمَارِقُ مَصْفُوفَةٌۭ ﴿١٥﴾\\
\textamh{16.\  } & وَزَرَابِىُّ مَبْثُوثَةٌ ﴿١٦﴾\\
\textamh{17.\  } & أَفَلَا يَنظُرُونَ إِلَى ٱلْإِبِلِ كَيْفَ خُلِقَتْ ﴿١٧﴾\\
\textamh{18.\  } & وَإِلَى ٱلسَّمَآءِ كَيْفَ رُفِعَتْ ﴿١٨﴾\\
\textamh{19.\  } & وَإِلَى ٱلْجِبَالِ كَيْفَ نُصِبَتْ ﴿١٩﴾\\
\textamh{20.\  } & وَإِلَى ٱلْأَرْضِ كَيْفَ سُطِحَتْ ﴿٢٠﴾\\
\textamh{21.\  } & فَذَكِّرْ إِنَّمَآ أَنتَ مُذَكِّرٌۭ ﴿٢١﴾\\
\textamh{22.\  } & لَّسْتَ عَلَيْهِم بِمُصَيْطِرٍ ﴿٢٢﴾\\
\textamh{23.\  } & إِلَّا مَن تَوَلَّىٰ وَكَفَرَ ﴿٢٣﴾\\
\textamh{24.\  } & فَيُعَذِّبُهُ ٱللَّهُ ٱلْعَذَابَ ٱلْأَكْبَرَ ﴿٢٤﴾\\
\textamh{25.\  } & إِنَّ إِلَيْنَآ إِيَابَهُمْ ﴿٢٥﴾\\
\textamh{26.\  } & ثُمَّ إِنَّ عَلَيْنَا حِسَابَهُم ﴿٢٦﴾\\
\end{longtable} \newpage

%% License: BSD style (Berkley) (i.e. Put the Copyright owner's name always)
%% Writer and Copyright (to): Bewketu(Bilal) Tadilo (2016-17)
\centering\section{\LR{\textamharic{ሱራቱ አልፈጅር -}  \RL{سوره  الفجر}}}
\begin{longtable}{%
  @{}
    p{.5\textwidth}
  @{~~~~~~~~~~~~~}
    p{.5\textwidth}
    @{}
}
\nopagebreak
\textamh{ቢስሚላሂ አራህመኒ ራሂይም } &  بِسْمِ ٱللَّهِ ٱلرَّحْمَـٰنِ ٱلرَّحِيمِ\\
\textamh{1.\  } &  وَٱلْفَجْرِ ﴿١﴾\\
\textamh{2.\  } & وَلَيَالٍ عَشْرٍۢ ﴿٢﴾\\
\textamh{3.\  } & وَٱلشَّفْعِ وَٱلْوَتْرِ ﴿٣﴾\\
\textamh{4.\  } & وَٱلَّيْلِ إِذَا يَسْرِ ﴿٤﴾\\
\textamh{5.\  } & هَلْ فِى ذَٟلِكَ قَسَمٌۭ لِّذِى حِجْرٍ ﴿٥﴾\\
\textamh{6.\  } & أَلَمْ تَرَ كَيْفَ فَعَلَ رَبُّكَ بِعَادٍ ﴿٦﴾\\
\textamh{7.\  } & إِرَمَ ذَاتِ ٱلْعِمَادِ ﴿٧﴾\\
\textamh{8.\  } & ٱلَّتِى لَمْ يُخْلَقْ مِثْلُهَا فِى ٱلْبِلَـٰدِ ﴿٨﴾\\
\textamh{9.\  } & وَثَمُودَ ٱلَّذِينَ جَابُوا۟ ٱلصَّخْرَ بِٱلْوَادِ ﴿٩﴾\\
\textamh{10.\  } & وَفِرْعَوْنَ ذِى ٱلْأَوْتَادِ ﴿١٠﴾\\
\textamh{11.\  } & ٱلَّذِينَ طَغَوْا۟ فِى ٱلْبِلَـٰدِ ﴿١١﴾\\
\textamh{12.\  } & فَأَكْثَرُوا۟ فِيهَا ٱلْفَسَادَ ﴿١٢﴾\\
\textamh{13.\  } & فَصَبَّ عَلَيْهِمْ رَبُّكَ سَوْطَ عَذَابٍ ﴿١٣﴾\\
\textamh{14.\  } & إِنَّ رَبَّكَ لَبِٱلْمِرْصَادِ ﴿١٤﴾\\
\textamh{15.\  } & فَأَمَّا ٱلْإِنسَـٰنُ إِذَا مَا ٱبْتَلَىٰهُ رَبُّهُۥ فَأَكْرَمَهُۥ وَنَعَّمَهُۥ فَيَقُولُ رَبِّىٓ أَكْرَمَنِ ﴿١٥﴾\\
\textamh{16.\  } & وَأَمَّآ إِذَا مَا ٱبْتَلَىٰهُ فَقَدَرَ عَلَيْهِ رِزْقَهُۥ فَيَقُولُ رَبِّىٓ أَهَـٰنَنِ ﴿١٦﴾\\
\textamh{17.\  } & كَلَّا ۖ بَل لَّا تُكْرِمُونَ ٱلْيَتِيمَ ﴿١٧﴾\\
\textamh{18.\  } & وَلَا تَحَـٰٓضُّونَ عَلَىٰ طَعَامِ ٱلْمِسْكِينِ ﴿١٨﴾\\
\textamh{19.\  } & وَتَأْكُلُونَ ٱلتُّرَاثَ أَكْلًۭا لَّمًّۭا ﴿١٩﴾\\
\textamh{20.\  } & وَتُحِبُّونَ ٱلْمَالَ حُبًّۭا جَمًّۭا ﴿٢٠﴾\\
\textamh{21.\  } & كَلَّآ إِذَا دُكَّتِ ٱلْأَرْضُ دَكًّۭا دَكًّۭا ﴿٢١﴾\\
\textamh{22.\  } & وَجَآءَ رَبُّكَ وَٱلْمَلَكُ صَفًّۭا صَفًّۭا ﴿٢٢﴾\\
\textamh{23.\  } & وَجِا۟ىٓءَ يَوْمَئِذٍۭ بِجَهَنَّمَ ۚ يَوْمَئِذٍۢ يَتَذَكَّرُ ٱلْإِنسَـٰنُ وَأَنَّىٰ لَهُ ٱلذِّكْرَىٰ ﴿٢٣﴾\\
\textamh{24.\  } & يَقُولُ يَـٰلَيْتَنِى قَدَّمْتُ لِحَيَاتِى ﴿٢٤﴾\\
\textamh{25.\  } & فَيَوْمَئِذٍۢ لَّا يُعَذِّبُ عَذَابَهُۥٓ أَحَدٌۭ ﴿٢٥﴾\\
\textamh{26.\  } & وَلَا يُوثِقُ وَثَاقَهُۥٓ أَحَدٌۭ ﴿٢٦﴾\\
\textamh{27.\  } & يَـٰٓأَيَّتُهَا ٱلنَّفْسُ ٱلْمُطْمَئِنَّةُ ﴿٢٧﴾\\
\textamh{28.\  } & ٱرْجِعِىٓ إِلَىٰ رَبِّكِ رَاضِيَةًۭ مَّرْضِيَّةًۭ ﴿٢٨﴾\\
\textamh{29.\  } & فَٱدْخُلِى فِى عِبَٰدِى ﴿٢٩﴾\\
\textamh{30.\  } & وَٱدْخُلِى جَنَّتِى ﴿٣٠﴾\\
\end{longtable}
\clearpage
%% License: BSD style (Berkley) (i.e. Put the Copyright owner's name always)
%% Writer and Copyright (to): Bewketu(Bilal) Tadilo (2016-17)
\centering\section{\LR{\textamharic{ሱራቱ አልበለድ -}  \RL{سوره  البلد}}}
\begin{longtable}{%
  @{}
    p{.5\textwidth}
  @{~~~~~~~~~~~~~}
    p{.5\textwidth}
    @{}
}
\nopagebreak
\textamh{ቢስሚላሂ አራህመኒ ራሂይም } &  بِسْمِ ٱللَّهِ ٱلرَّحْمَـٰنِ ٱلرَّحِيمِ\\
\textamh{1.\  } &  لَآ أُقْسِمُ بِهَـٰذَا ٱلْبَلَدِ ﴿١﴾\\
\textamh{2.\  } & وَأَنتَ حِلٌّۢ بِهَـٰذَا ٱلْبَلَدِ ﴿٢﴾\\
\textamh{3.\  } & وَوَالِدٍۢ وَمَا وَلَدَ ﴿٣﴾\\
\textamh{4.\  } & لَقَدْ خَلَقْنَا ٱلْإِنسَـٰنَ فِى كَبَدٍ ﴿٤﴾\\
\textamh{5.\  } & أَيَحْسَبُ أَن لَّن يَقْدِرَ عَلَيْهِ أَحَدٌۭ ﴿٥﴾\\
\textamh{6.\  } & يَقُولُ أَهْلَكْتُ مَالًۭا لُّبَدًا ﴿٦﴾\\
\textamh{7.\  } & أَيَحْسَبُ أَن لَّمْ يَرَهُۥٓ أَحَدٌ ﴿٧﴾\\
\textamh{8.\  } & أَلَمْ نَجْعَل لَّهُۥ عَيْنَيْنِ ﴿٨﴾\\
\textamh{9.\  } & وَلِسَانًۭا وَشَفَتَيْنِ ﴿٩﴾\\
\textamh{10.\  } & وَهَدَيْنَـٰهُ ٱلنَّجْدَيْنِ ﴿١٠﴾\\
\textamh{11.\  } & فَلَا ٱقْتَحَمَ ٱلْعَقَبَةَ ﴿١١﴾\\
\textamh{12.\  } & وَمَآ أَدْرَىٰكَ مَا ٱلْعَقَبَةُ ﴿١٢﴾\\
\textamh{13.\  } & فَكُّ رَقَبَةٍ ﴿١٣﴾\\
\textamh{14.\  } & أَوْ إِطْعَـٰمٌۭ فِى يَوْمٍۢ ذِى مَسْغَبَةٍۢ ﴿١٤﴾\\
\textamh{15.\  } & يَتِيمًۭا ذَا مَقْرَبَةٍ ﴿١٥﴾\\
\textamh{16.\  } & أَوْ مِسْكِينًۭا ذَا مَتْرَبَةٍۢ ﴿١٦﴾\\
\textamh{17.\  } & ثُمَّ كَانَ مِنَ ٱلَّذِينَ ءَامَنُوا۟ وَتَوَاصَوْا۟ بِٱلصَّبْرِ وَتَوَاصَوْا۟ بِٱلْمَرْحَمَةِ ﴿١٧﴾\\
\textamh{18.\  } & أُو۟لَـٰٓئِكَ أَصْحَـٰبُ ٱلْمَيْمَنَةِ ﴿١٨﴾\\
\textamh{19.\  } & وَٱلَّذِينَ كَفَرُوا۟ بِـَٔايَـٰتِنَا هُمْ أَصْحَـٰبُ ٱلْمَشْـَٔمَةِ ﴿١٩﴾\\
\textamh{20.\  } & عَلَيْهِمْ نَارٌۭ مُّؤْصَدَةٌۢ ﴿٢٠﴾\\
\end{longtable}
\clearpage
%% License: BSD style (Berkley) (i.e. Put the Copyright owner's name always)
%% Writer and Copyright (to): Bewketu(Bilal) Tadilo (2016-17)
\centering\section{\LR{\textamharic{ሱራቱ አልሸምስ -}  \RL{سوره  الشمس}}}
\begin{longtable}{%
  @{}
    p{.5\textwidth}
  @{~~~~~~~~~~~~}
    p{.5\textwidth}
    @{}
}
\nopagebreak
\textamh{ቢስሚላሂ አራህመኒ ራሂይም } &  بِسْمِ ٱللَّهِ ٱلرَّحْمَـٰنِ ٱلرَّحِيمِ\\
\textamh{1.\  } &  وَٱلشَّمْسِ وَضُحَىٰهَا ﴿١﴾\\
\textamh{2.\  } & وَٱلْقَمَرِ إِذَا تَلَىٰهَا ﴿٢﴾\\
\textamh{3.\  } & وَٱلنَّهَارِ إِذَا جَلَّىٰهَا ﴿٣﴾\\
\textamh{4.\  } & وَٱلَّيْلِ إِذَا يَغْشَىٰهَا ﴿٤﴾\\
\textamh{5.\  } & وَٱلسَّمَآءِ وَمَا بَنَىٰهَا ﴿٥﴾\\
\textamh{6.\  } & وَٱلْأَرْضِ وَمَا طَحَىٰهَا ﴿٦﴾\\
\textamh{7.\  } & وَنَفْسٍۢ وَمَا سَوَّىٰهَا ﴿٧﴾\\
\textamh{8.\  } & فَأَلْهَمَهَا فُجُورَهَا وَتَقْوَىٰهَا ﴿٨﴾\\
\textamh{9.\  } & قَدْ أَفْلَحَ مَن زَكَّىٰهَا ﴿٩﴾\\
\textamh{10.\  } & وَقَدْ خَابَ مَن دَسَّىٰهَا ﴿١٠﴾\\
\textamh{11.\  } & كَذَّبَتْ ثَمُودُ بِطَغْوَىٰهَآ ﴿١١﴾\\
\textamh{12.\  } & إِذِ ٱنۢبَعَثَ أَشْقَىٰهَا ﴿١٢﴾\\
\textamh{13.\  } & فَقَالَ لَهُمْ رَسُولُ ٱللَّهِ نَاقَةَ ٱللَّهِ وَسُقْيَـٰهَا ﴿١٣﴾\\
\textamh{14.\  } & فَكَذَّبُوهُ فَعَقَرُوهَا فَدَمْدَمَ عَلَيْهِمْ رَبُّهُم بِذَنۢبِهِمْ فَسَوَّىٰهَا ﴿١٤﴾\\
\textamh{15.\  } & وَلَا يَخَافُ عُقْبَٰهَا ﴿١٥﴾\\
\end{longtable}
\clearpage
%% License: BSD style (Berkley) (i.e. Put the Copyright owner's name always)
%% Writer and Copyright (to): Bewketu(Bilal) Tadilo (2016-17)
\centering\section{\LR{\textamharic{ሱራቱ አልለይል -}  \RL{سوره  الليل}}}
\begin{longtable}{%
  @{}
    p{.5\textwidth}
  @{~~~~~~~~~~~~~}
    p{.5\textwidth}
    @{}
}
\nopagebreak
\textamh{ቢስሚላሂ አራህመኒ ራሂይም } &  بِسْمِ ٱللَّهِ ٱلرَّحْمَـٰنِ ٱلرَّحِيمِ\\
\textamh{1.\  } &  وَٱلَّيْلِ إِذَا يَغْشَىٰ ﴿١﴾\\
\textamh{2.\  } & وَٱلنَّهَارِ إِذَا تَجَلَّىٰ ﴿٢﴾\\
\textamh{3.\  } & وَمَا خَلَقَ ٱلذَّكَرَ وَٱلْأُنثَىٰٓ ﴿٣﴾\\
\textamh{4.\  } & إِنَّ سَعْيَكُمْ لَشَتَّىٰ ﴿٤﴾\\
\textamh{5.\  } & فَأَمَّا مَنْ أَعْطَىٰ وَٱتَّقَىٰ ﴿٥﴾\\
\textamh{6.\  } & وَصَدَّقَ بِٱلْحُسْنَىٰ ﴿٦﴾\\
\textamh{7.\  } & فَسَنُيَسِّرُهُۥ لِلْيُسْرَىٰ ﴿٧﴾\\
\textamh{8.\  } & وَأَمَّا مَنۢ بَخِلَ وَٱسْتَغْنَىٰ ﴿٨﴾\\
\textamh{9.\  } & وَكَذَّبَ بِٱلْحُسْنَىٰ ﴿٩﴾\\
\textamh{10.\  } & فَسَنُيَسِّرُهُۥ لِلْعُسْرَىٰ ﴿١٠﴾\\
\textamh{11.\  } & وَمَا يُغْنِى عَنْهُ مَالُهُۥٓ إِذَا تَرَدَّىٰٓ ﴿١١﴾\\
\textamh{12.\  } & إِنَّ عَلَيْنَا لَلْهُدَىٰ ﴿١٢﴾\\
\textamh{13.\  } & وَإِنَّ لَنَا لَلْءَاخِرَةَ وَٱلْأُولَىٰ ﴿١٣﴾\\
\textamh{14.\  } & فَأَنذَرْتُكُمْ نَارًۭا تَلَظَّىٰ ﴿١٤﴾\\
\textamh{15.\  } & لَا يَصْلَىٰهَآ إِلَّا ٱلْأَشْقَى ﴿١٥﴾\\
\textamh{16.\  } & ٱلَّذِى كَذَّبَ وَتَوَلَّىٰ ﴿١٦﴾\\
\textamh{17.\  } & وَسَيُجَنَّبُهَا ٱلْأَتْقَى ﴿١٧﴾\\
\textamh{18.\  } & ٱلَّذِى يُؤْتِى مَالَهُۥ يَتَزَكَّىٰ ﴿١٨﴾\\
\textamh{19.\  } & وَمَا لِأَحَدٍ عِندَهُۥ مِن نِّعْمَةٍۢ تُجْزَىٰٓ ﴿١٩﴾\\
\textamh{20.\  } & إِلَّا ٱبْتِغَآءَ وَجْهِ رَبِّهِ ٱلْأَعْلَىٰ ﴿٢٠﴾\\
\textamh{21.\  } & وَلَسَوْفَ يَرْضَىٰ ﴿٢١﴾\\
\end{longtable}
\clearpage
%% License: BSD style (Berkley) (i.e. Put the Copyright owner's name always)
%% Writer and Copyright (to): Bewketu(Bilal) Tadilo (2016-17)
\centering\section{\LR{\textamharic{ሱራቱ አድዱሀ -}  \RL{سوره  الضحى}}}
\begin{longtable}{%
  @{}
    p{.5\textwidth}
  @{~~~~~~~~~~~~~}
    p{.5\textwidth}
    @{}
}
\nopagebreak
\textamh{ቢስሚላሂ አራህመኒ ራሂይም } &  بِسْمِ ٱللَّهِ ٱلرَّحْمَـٰنِ ٱلرَّحِيمِ\\
\textamh{1.\  } &  وَٱلضُّحَىٰ ﴿١﴾\\
\textamh{2.\  } & وَٱلَّيْلِ إِذَا سَجَىٰ ﴿٢﴾\\
\textamh{3.\  } & مَا وَدَّعَكَ رَبُّكَ وَمَا قَلَىٰ ﴿٣﴾\\
\textamh{4.\  } & وَلَلْءَاخِرَةُ خَيْرٌۭ لَّكَ مِنَ ٱلْأُولَىٰ ﴿٤﴾\\
\textamh{5.\  } & وَلَسَوْفَ يُعْطِيكَ رَبُّكَ فَتَرْضَىٰٓ ﴿٥﴾\\
\textamh{6.\  } & أَلَمْ يَجِدْكَ يَتِيمًۭا فَـَٔاوَىٰ ﴿٦﴾\\
\textamh{7.\  } & وَوَجَدَكَ ضَآلًّۭا فَهَدَىٰ ﴿٧﴾\\
\textamh{8.\  } & وَوَجَدَكَ عَآئِلًۭا فَأَغْنَىٰ ﴿٨﴾\\
\textamh{9.\  } & فَأَمَّا ٱلْيَتِيمَ فَلَا تَقْهَرْ ﴿٩﴾\\
\textamh{10.\  } & وَأَمَّا ٱلسَّآئِلَ فَلَا تَنْهَرْ ﴿١٠﴾\\
\textamh{11.\  } & وَأَمَّا بِنِعْمَةِ رَبِّكَ فَحَدِّثْ ﴿١١﴾\\
\end{longtable}
\clearpage
%% License: BSD style (Berkley) (i.e. Put the Copyright owner's name always)
%% Writer and Copyright (to): Bewketu(Bilal) Tadilo (2016-17)
\begin{center}\section{\LR{\textamhsec{ሱራቱ አሽሸርህ -}  \textarabic{سوره  الشرح}}}\end{center}
\begin{longtable}{%
  @{}
    p{.5\textwidth}
  @{~~~}
    p{.5\textwidth}
    @{}
}
\textamh{ቢስሚላሂ አራህመኒ ራሂይም } &  \mytextarabic{بِسْمِ ٱللَّهِ ٱلرَّحْمَـٰنِ ٱلرَّحِيمِ}\\
\textamh{1.\  } & \mytextarabic{ أَلَمْ نَشْرَحْ لَكَ صَدْرَكَ ﴿١﴾}\\
\textamh{2.\  } & \mytextarabic{وَوَضَعْنَا عَنكَ وِزْرَكَ ﴿٢﴾}\\
\textamh{3.\  } & \mytextarabic{ٱلَّذِىٓ أَنقَضَ ظَهْرَكَ ﴿٣﴾}\\
\textamh{4.\  } & \mytextarabic{وَرَفَعْنَا لَكَ ذِكْرَكَ ﴿٤﴾}\\
\textamh{5.\  } & \mytextarabic{فَإِنَّ مَعَ ٱلْعُسْرِ يُسْرًا ﴿٥﴾}\\
\textamh{6.\  } & \mytextarabic{إِنَّ مَعَ ٱلْعُسْرِ يُسْرًۭا ﴿٦﴾}\\
\textamh{7.\  } & \mytextarabic{فَإِذَا فَرَغْتَ فَٱنصَبْ ﴿٧﴾}\\
\textamh{8.\  } & \mytextarabic{وَإِلَىٰ رَبِّكَ فَٱرْغَب ﴿٨﴾}\\
\end{longtable}
\clearpage
%% License: BSD style (Berkley) (i.e. Put the Copyright owner's name always)
%% Writer and Copyright (to): Bewketu(Bilal) Tadilo (2016-17)
\begin{center}\section{\LR{\textamhsec{ሱራቱ አትቲይን -}  \textarabic{سوره  التين}}}\end{center}
\begin{longtable}{%
  @{}
    p{.5\textwidth}
  @{~~~}
    p{.5\textwidth}
    @{}
}
\textamh{ቢስሚላሂ አራህመኒ ራሂይም } &  \mytextarabic{بِسْمِ ٱللَّهِ ٱلرَّحْمَـٰنِ ٱلرَّحِيمِ}\\
\textamh{1.\  } & \mytextarabic{بِّسْمِ ٱللَّهِ ٱلرَّحْمَـٰنِ ٱلرَّحِيمِ وَٱلتِّينِ وَٱلزَّيْتُونِ ﴿١﴾}\\
\textamh{2.\  } & \mytextarabic{وَطُورِ سِينِينَ ﴿٢﴾}\\
\textamh{3.\  } & \mytextarabic{وَهَـٰذَا ٱلْبَلَدِ ٱلْأَمِينِ ﴿٣﴾}\\
\textamh{4.\  } & \mytextarabic{لَقَدْ خَلَقْنَا ٱلْإِنسَـٰنَ فِىٓ أَحْسَنِ تَقْوِيمٍۢ ﴿٤﴾}\\
\textamh{5.\  } & \mytextarabic{ثُمَّ رَدَدْنَـٰهُ أَسْفَلَ سَـٰفِلِينَ ﴿٥﴾}\\
\textamh{6.\  } & \mytextarabic{إِلَّا ٱلَّذِينَ ءَامَنُوا۟ وَعَمِلُوا۟ ٱلصَّـٰلِحَـٰتِ فَلَهُمْ أَجْرٌ غَيْرُ مَمْنُونٍۢ ﴿٦﴾}\\
\textamh{7.\  } & \mytextarabic{فَمَا يُكَذِّبُكَ بَعْدُ بِٱلدِّينِ ﴿٧﴾}\\
\textamh{8.\  } & \mytextarabic{أَلَيْسَ ٱللَّهُ بِأَحْكَمِ ٱلْحَـٰكِمِينَ ﴿٨﴾}\\
\end{longtable}
\clearpage
%% License: BSD style (Berkley) (i.e. Put the Copyright owner's name always)
%% Writer and Copyright (to): Bewketu(Bilal) Tadilo (2016-17)
\centering\section{\LR{\textamharic{ሱራቱ አልአለቅ -}  \RL{سوره  العلق}}}
\begin{longtable}{%
  @{}
    p{.5\textwidth}
  @{~~~~~~~~~~~~~}
    p{.5\textwidth}
    @{}
}
\nopagebreak
\textamh{\ \ \ \ \ \  ቢስሚላሂ አራህመኒ ራሂይም } &  بِسْمِ ٱللَّهِ ٱلرَّحْمَـٰنِ ٱلرَّحِيمِ\\
\textamh{1.\  } &  ٱقْرَأْ بِٱسْمِ رَبِّكَ ٱلَّذِى خَلَقَ ﴿١﴾\\
\textamh{2.\  } & خَلَقَ ٱلْإِنسَـٰنَ مِنْ عَلَقٍ ﴿٢﴾\\
\textamh{3.\  } & ٱقْرَأْ وَرَبُّكَ ٱلْأَكْرَمُ ﴿٣﴾\\
\textamh{4.\  } & ٱلَّذِى عَلَّمَ بِٱلْقَلَمِ ﴿٤﴾\\
\textamh{5.\  } & عَلَّمَ ٱلْإِنسَـٰنَ مَا لَمْ يَعْلَمْ ﴿٥﴾\\
\textamh{6.\  } & كَلَّآ إِنَّ ٱلْإِنسَـٰنَ لَيَطْغَىٰٓ ﴿٦﴾\\
\textamh{7.\  } & أَن رَّءَاهُ ٱسْتَغْنَىٰٓ ﴿٧﴾\\
\textamh{8.\  } & إِنَّ إِلَىٰ رَبِّكَ ٱلرُّجْعَىٰٓ ﴿٨﴾\\
\textamh{9.\  } & أَرَءَيْتَ ٱلَّذِى يَنْهَىٰ ﴿٩﴾\\
\textamh{10.\  } & عَبْدًا إِذَا صَلَّىٰٓ ﴿١٠﴾\\
\textamh{11.\  } & أَرَءَيْتَ إِن كَانَ عَلَى ٱلْهُدَىٰٓ ﴿١١﴾\\
\textamh{12.\  } & أَوْ أَمَرَ بِٱلتَّقْوَىٰٓ ﴿١٢﴾\\
\textamh{13.\  } & أَرَءَيْتَ إِن كَذَّبَ وَتَوَلَّىٰٓ ﴿١٣﴾\\
\textamh{14.\  } & أَلَمْ يَعْلَم بِأَنَّ ٱللَّهَ يَرَىٰ ﴿١٤﴾\\
\textamh{15.\  } & كَلَّا لَئِن لَّمْ يَنتَهِ لَنَسْفَعًۢا بِٱلنَّاصِيَةِ ﴿١٥﴾\\
\textamh{16.\  } & نَاصِيَةٍۢ كَـٰذِبَةٍ خَاطِئَةٍۢ ﴿١٦﴾\\
\textamh{17.\  } & فَلْيَدْعُ نَادِيَهُۥ ﴿١٧﴾\\
\textamh{18.\  } & سَنَدْعُ ٱلزَّبَانِيَةَ ﴿١٨﴾\\
\textamh{19.\  } & كَلَّا لَا تُطِعْهُ وَٱسْجُدْ وَٱقْتَرِب ۩ ﴿١٩﴾\\
\end{longtable} \newpage

%% License: BSD style (Berkley) (i.e. Put the Copyright owner's name always)
%% Writer and Copyright (to): Bewketu(Bilal) Tadilo (2016-17)
\centering\section{\LR{\textamharic{ሱራቱ አልቀድር -}  \RL{سوره  القدر}}}
\begin{longtable}{%
  @{}
    p{.5\textwidth}
  @{~~~~~~~~~~~~~}
    p{.5\textwidth}
    @{}
}
\nopagebreak
\textamh{ቢስሚላሂ አራህመኒ ራሂይም } &  بِسْمِ ٱللَّهِ ٱلرَّحْمَـٰنِ ٱلرَّحِيمِ\\
\textamh{1.\  } & بِّسْمِ ٱللَّهِ ٱلرَّحْمَـٰنِ ٱلرَّحِيمِ إِنَّآ أَنزَلْنَـٰهُ فِى لَيْلَةِ ٱلْقَدْرِ ﴿١﴾\\
\textamh{2.\  } & وَمَآ أَدْرَىٰكَ مَا لَيْلَةُ ٱلْقَدْرِ ﴿٢﴾\\
\textamh{3.\  } & لَيْلَةُ ٱلْقَدْرِ خَيْرٌۭ مِّنْ أَلْفِ شَهْرٍۢ ﴿٣﴾\\
\textamh{4.\  } & تَنَزَّلُ ٱلْمَلَـٰٓئِكَةُ وَٱلرُّوحُ فِيهَا بِإِذْنِ رَبِّهِم مِّن كُلِّ أَمْرٍۢ ﴿٤﴾\\
\textamh{5.\  } & سَلَـٰمٌ هِىَ حَتَّىٰ مَطْلَعِ ٱلْفَجْرِ ﴿٥﴾\\
\end{longtable}
\clearpage
%% License: BSD style (Berkley) (i.e. Put the Copyright owner's name always)
%% Writer and Copyright (to): Bewketu(Bilal) Tadilo (2016-17)
\begin{center}\section{\LR{\textamhsec{ሱራቱ አልበይና -}  \textarabic{سوره  البينة}}}\end{center}
\begin{longtable}{%
  @{}
    p{.5\textwidth}
  @{~~~}
    p{.5\textwidth}
    @{}
}
\textamh{ቢስሚላሂ አራህመኒ ራሂይም } &  \mytextarabic{بِسْمِ ٱللَّهِ ٱلرَّحْمَـٰنِ ٱلرَّحِيمِ}\\
\textamh{1.\  } & \mytextarabic{ لَمْ يَكُنِ ٱلَّذِينَ كَفَرُوا۟ مِنْ أَهْلِ ٱلْكِتَـٰبِ وَٱلْمُشْرِكِينَ مُنفَكِّينَ حَتَّىٰ تَأْتِيَهُمُ ٱلْبَيِّنَةُ ﴿١﴾}\\
\textamh{2.\  } & \mytextarabic{رَسُولٌۭ مِّنَ ٱللَّهِ يَتْلُوا۟ صُحُفًۭا مُّطَهَّرَةًۭ ﴿٢﴾}\\
\textamh{3.\  } & \mytextarabic{فِيهَا كُتُبٌۭ قَيِّمَةٌۭ ﴿٣﴾}\\
\textamh{4.\  } & \mytextarabic{وَمَا تَفَرَّقَ ٱلَّذِينَ أُوتُوا۟ ٱلْكِتَـٰبَ إِلَّا مِنۢ بَعْدِ مَا جَآءَتْهُمُ ٱلْبَيِّنَةُ ﴿٤﴾}\\
\textamh{5.\  } & \mytextarabic{وَمَآ أُمِرُوٓا۟ إِلَّا لِيَعْبُدُوا۟ ٱللَّهَ مُخْلِصِينَ لَهُ ٱلدِّينَ حُنَفَآءَ وَيُقِيمُوا۟ ٱلصَّلَوٰةَ وَيُؤْتُوا۟ ٱلزَّكَوٰةَ ۚ وَذَٟلِكَ دِينُ ٱلْقَيِّمَةِ ﴿٥﴾}\\
\textamh{6.\  } & \mytextarabic{إِنَّ ٱلَّذِينَ كَفَرُوا۟ مِنْ أَهْلِ ٱلْكِتَـٰبِ وَٱلْمُشْرِكِينَ فِى نَارِ جَهَنَّمَ خَـٰلِدِينَ فِيهَآ ۚ أُو۟لَـٰٓئِكَ هُمْ شَرُّ ٱلْبَرِيَّةِ ﴿٦﴾}\\
\textamh{7.\  } & \mytextarabic{إِنَّ ٱلَّذِينَ ءَامَنُوا۟ وَعَمِلُوا۟ ٱلصَّـٰلِحَـٰتِ أُو۟لَـٰٓئِكَ هُمْ خَيْرُ ٱلْبَرِيَّةِ ﴿٧﴾}\\
\textamh{8.\  } & \mytextarabic{جَزَآؤُهُمْ عِندَ رَبِّهِمْ جَنَّـٰتُ عَدْنٍۢ تَجْرِى مِن تَحْتِهَا ٱلْأَنْهَـٰرُ خَـٰلِدِينَ فِيهَآ أَبَدًۭا ۖ رَّضِىَ ٱللَّهُ عَنْهُمْ وَرَضُوا۟ عَنْهُ ۚ ذَٟلِكَ لِمَنْ خَشِىَ رَبَّهُۥ ﴿٨﴾}\\
\end{longtable}
\clearpage
%% License: BSD style (Berkley) (i.e. Put the Copyright owner's name always)
%% Writer and Copyright (to): Bewketu(Bilal) Tadilo (2016-17)
\begin{center}\section{\LR{\textamhsec{ሱራቱ አልዘልዘላ -}  \textarabic{سوره  الزلزلة}}}\end{center}
\begin{longtable}{%
  @{}
    p{.5\textwidth}
  @{~~~}
    p{.5\textwidth}
    @{}
}
\textamh{ቢስሚላሂ አራህመኒ ራሂይም } &  \mytextarabic{بِسْمِ ٱللَّهِ ٱلرَّحْمَـٰنِ ٱلرَّحِيمِ}\\
\textamh{1.\  } & \mytextarabic{ إِذَا زُلْزِلَتِ ٱلْأَرْضُ زِلْزَالَهَا ﴿١﴾}\\
\textamh{2.\  } & \mytextarabic{وَأَخْرَجَتِ ٱلْأَرْضُ أَثْقَالَهَا ﴿٢﴾}\\
\textamh{3.\  } & \mytextarabic{وَقَالَ ٱلْإِنسَـٰنُ مَا لَهَا ﴿٣﴾}\\
\textamh{4.\  } & \mytextarabic{يَوْمَئِذٍۢ تُحَدِّثُ أَخْبَارَهَا ﴿٤﴾}\\
\textamh{5.\  } & \mytextarabic{بِأَنَّ رَبَّكَ أَوْحَىٰ لَهَا ﴿٥﴾}\\
\textamh{6.\  } & \mytextarabic{يَوْمَئِذٍۢ يَصْدُرُ ٱلنَّاسُ أَشْتَاتًۭا لِّيُرَوْا۟ أَعْمَـٰلَهُمْ ﴿٦﴾}\\
\textamh{7.\  } & \mytextarabic{فَمَن يَعْمَلْ مِثْقَالَ ذَرَّةٍ خَيْرًۭا يَرَهُۥ ﴿٧﴾}\\
\textamh{8.\  } & \mytextarabic{وَمَن يَعْمَلْ مِثْقَالَ ذَرَّةٍۢ شَرًّۭا يَرَهُۥ ﴿٨﴾}\\
\end{longtable}
\clearpage
%% License: BSD style (Berkley) (i.e. Put the Copyright owner's name always)
%% Writer and Copyright (to): Bewketu(Bilal) Tadilo (2016-17)
\centering\section{\LR{\textamharic{ሱራቱ አልአዲያ -}  \RL{سوره  العاديات}}}
\begin{longtable}{%
  @{}
    p{.5\textwidth}
  @{~~~~~~~~~~~~}
    p{.5\textwidth}
    @{}
}
\nopagebreak
\textamh{ቢስሚላሂ አራህመኒ ራሂይም } &  بِسْمِ ٱللَّهِ ٱلرَّحْمَـٰنِ ٱلرَّحِيمِ\\
\textamh{1.\  } &  وَٱلْعَـٰدِيَـٰتِ ضَبْحًۭا ﴿١﴾\\
\textamh{2.\  } & فَٱلْمُورِيَـٰتِ قَدْحًۭا ﴿٢﴾\\
\textamh{3.\  } & فَٱلْمُغِيرَٰتِ صُبْحًۭا ﴿٣﴾\\
\textamh{4.\  } & فَأَثَرْنَ بِهِۦ نَقْعًۭا ﴿٤﴾\\
\textamh{5.\  } & فَوَسَطْنَ بِهِۦ جَمْعًا ﴿٥﴾\\
\textamh{6.\  } & إِنَّ ٱلْإِنسَـٰنَ لِرَبِّهِۦ لَكَنُودٌۭ ﴿٦﴾\\
\textamh{7.\  } & وَإِنَّهُۥ عَلَىٰ ذَٟلِكَ لَشَهِيدٌۭ ﴿٧﴾\\
\textamh{8.\  } & وَإِنَّهُۥ لِحُبِّ ٱلْخَيْرِ لَشَدِيدٌ ﴿٨﴾\\
\textamh{9.\  } & ۞ أَفَلَا يَعْلَمُ إِذَا بُعْثِرَ مَا فِى ٱلْقُبُورِ ﴿٩﴾\\
\textamh{10.\  } & وَحُصِّلَ مَا فِى ٱلصُّدُورِ ﴿١٠﴾\\
\textamh{11.\  } & إِنَّ رَبَّهُم بِهِمْ يَوْمَئِذٍۢ لَّخَبِيرٌۢ ﴿١١﴾\\
\end{longtable}
\clearpage
%% License: BSD style (Berkley) (i.e. Put the Copyright owner's name always)
%% Writer and Copyright (to): Bewketu(Bilal) Tadilo (2016-17)
\centering\section{\LR{\textamharic{ሱራቱ አልቃሪያ -}  \RL{سوره  القارعة}}}
\begin{longtable}{%
  @{}
    p{.5\textwidth}
  @{~~~~~~~~~~~~~}
    p{.5\textwidth}
    @{}
}
\nopagebreak
\textamh{ቢስሚላሂ አራህመኒ ራሂይም } &  بِسْمِ ٱللَّهِ ٱلرَّحْمَـٰنِ ٱلرَّحِيمِ\\
\textamh{1.\  } &  ٱلْقَارِعَةُ ﴿١﴾\\
\textamh{2.\  } & مَا ٱلْقَارِعَةُ ﴿٢﴾\\
\textamh{3.\  } & وَمَآ أَدْرَىٰكَ مَا ٱلْقَارِعَةُ ﴿٣﴾\\
\textamh{4.\  } & يَوْمَ يَكُونُ ٱلنَّاسُ كَٱلْفَرَاشِ ٱلْمَبْثُوثِ ﴿٤﴾\\
\textamh{5.\  } & وَتَكُونُ ٱلْجِبَالُ كَٱلْعِهْنِ ٱلْمَنفُوشِ ﴿٥﴾\\
\textamh{6.\  } & فَأَمَّا مَن ثَقُلَتْ مَوَٟزِينُهُۥ ﴿٦﴾\\
\textamh{7.\  } & فَهُوَ فِى عِيشَةٍۢ رَّاضِيَةٍۢ ﴿٧﴾\\
\textamh{8.\  } & وَأَمَّا مَنْ خَفَّتْ مَوَٟزِينُهُۥ ﴿٨﴾\\
\textamh{9.\  } & فَأُمُّهُۥ هَاوِيَةٌۭ ﴿٩﴾\\
\textamh{10.\  } & وَمَآ أَدْرَىٰكَ مَا هِيَهْ ﴿١٠﴾\\
\textamh{11.\  } & نَارٌ حَامِيَةٌۢ ﴿١١﴾\\
\end{longtable}
\clearpage
%% License: BSD style (Berkley) (i.e. Put the Copyright owner's name always)
%% Writer and Copyright (to): Bewketu(Bilal) Tadilo (2016-17)
\begin{center}\section{\LR{\textamhsec{ሱራቱ አትተካቱር -}  \textarabic{سوره  التكاثر}}}\end{center}
\begin{longtable}{%
  @{}
    p{.5\textwidth}
  @{~~~}
    p{.5\textwidth}
    @{}
}
\textamh{ቢስሚላሂ አራህመኒ ራሂይም } &  \mytextarabic{بِسْمِ ٱللَّهِ ٱلرَّحْمَـٰنِ ٱلرَّحِيمِ}\\
\textamh{1.\  } & \mytextarabic{ أَلْهَىٰكُمُ ٱلتَّكَاثُرُ ﴿١﴾}\\
\textamh{2.\  } & \mytextarabic{حَتَّىٰ زُرْتُمُ ٱلْمَقَابِرَ ﴿٢﴾}\\
\textamh{3.\  } & \mytextarabic{كَلَّا سَوْفَ تَعْلَمُونَ ﴿٣﴾}\\
\textamh{4.\  } & \mytextarabic{ثُمَّ كَلَّا سَوْفَ تَعْلَمُونَ ﴿٤﴾}\\
\textamh{5.\  } & \mytextarabic{كَلَّا لَوْ تَعْلَمُونَ عِلْمَ ٱلْيَقِينِ ﴿٥﴾}\\
\textamh{6.\  } & \mytextarabic{لَتَرَوُنَّ ٱلْجَحِيمَ ﴿٦﴾}\\
\textamh{7.\  } & \mytextarabic{ثُمَّ لَتَرَوُنَّهَا عَيْنَ ٱلْيَقِينِ ﴿٧﴾}\\
\textamh{8.\  } & \mytextarabic{ثُمَّ لَتُسْـَٔلُنَّ يَوْمَئِذٍ عَنِ ٱلنَّعِيمِ ﴿٨﴾}\\
\end{longtable}
\clearpage
%% License: BSD style (Berkley) (i.e. Put the Copyright owner's name always)
%% Writer and Copyright (to): Bewketu(Bilal) Tadilo (2016-17)
\begin{center}\section{\LR{\textamhsec{ሱራቱ አልአስር -}  \textarabic{سوره  العصر}}}\end{center}
\begin{longtable}{%
  @{}
    p{.5\textwidth}
  @{~~~}
    p{.5\textwidth}
    @{}
}
\textamh{ቢስሚላሂ አራህመኒ ራሂይም } &  \mytextarabic{بِسْمِ ٱللَّهِ ٱلرَّحْمَـٰنِ ٱلرَّحِيمِ}\\
\textamh{1.\  } & \mytextarabic{ وَٱلْعَصْرِ ﴿١﴾}\\
\textamh{2.\  } & \mytextarabic{إِنَّ ٱلْإِنسَـٰنَ لَفِى خُسْرٍ ﴿٢﴾}\\
\textamh{3.\  } & \mytextarabic{إِلَّا ٱلَّذِينَ ءَامَنُوا۟ وَعَمِلُوا۟ ٱلصَّـٰلِحَـٰتِ وَتَوَاصَوْا۟ بِٱلْحَقِّ وَتَوَاصَوْا۟ بِٱلصَّبْرِ ﴿٣﴾}\\
\end{longtable}
\clearpage
%% License: BSD style (Berkley) (i.e. Put the Copyright owner's name always)
%% Writer and Copyright (to): Bewketu(Bilal) Tadilo (2016-17)
\centering\section{\LR{\textamharic{ሱራቱ አልሁመዛ -}  \RL{سوره  الهمزة}}}
\begin{longtable}{%
  @{}
    p{.5\textwidth}
  @{~~~~~~~~~~~~}
    p{.5\textwidth}
    @{}
}
\nopagebreak
\textamh{ቢስሚላሂ አራህመኒ ራሂይም } &  بِسْمِ ٱللَّهِ ٱلرَّحْمَـٰنِ ٱلرَّحِيمِ\\
\textamh{1.\  } &  وَيْلٌۭ لِّكُلِّ هُمَزَةٍۢ لُّمَزَةٍ ﴿١﴾\\
\textamh{2.\  } & ٱلَّذِى جَمَعَ مَالًۭا وَعَدَّدَهُۥ ﴿٢﴾\\
\textamh{3.\  } & يَحْسَبُ أَنَّ مَالَهُۥٓ أَخْلَدَهُۥ ﴿٣﴾\\
\textamh{4.\  } & كَلَّا ۖ لَيُنۢبَذَنَّ فِى ٱلْحُطَمَةِ ﴿٤﴾\\
\textamh{5.\  } & وَمَآ أَدْرَىٰكَ مَا ٱلْحُطَمَةُ ﴿٥﴾\\
\textamh{6.\  } & نَارُ ٱللَّهِ ٱلْمُوقَدَةُ ﴿٦﴾\\
\textamh{7.\  } & ٱلَّتِى تَطَّلِعُ عَلَى ٱلْأَفْـِٔدَةِ ﴿٧﴾\\
\textamh{8.\  } & إِنَّهَا عَلَيْهِم مُّؤْصَدَةٌۭ ﴿٨﴾\\
\textamh{9.\  } & فِى عَمَدٍۢ مُّمَدَّدَةٍۭ ﴿٩﴾\\
\end{longtable}
\clearpage
%% License: BSD style (Berkley) (i.e. Put the Copyright owner's name always)
%% Writer and Copyright (to): Bewketu(Bilal) Tadilo (2016-17)
\centering\section{\LR{\textamharic{ሱራቱ አልፊይል -}  \RL{سوره  الفيل}}}
\begin{longtable}{%
  @{}
    p{.5\textwidth}
  @{~~~~~~~~~~~~}
    p{.5\textwidth}
    @{}
}
\nopagebreak
\textamh{ቢስሚላሂ አራህመኒ ራሂይም } &  بِسْمِ ٱللَّهِ ٱلرَّحْمَـٰنِ ٱلرَّحِيمِ\\
\textamh{1.\  } &  أَلَمْ تَرَ كَيْفَ فَعَلَ رَبُّكَ بِأَصْحَـٰبِ ٱلْفِيلِ ﴿١﴾\\
\textamh{2.\  } & أَلَمْ يَجْعَلْ كَيْدَهُمْ فِى تَضْلِيلٍۢ ﴿٢﴾\\
\textamh{3.\  } & وَأَرْسَلَ عَلَيْهِمْ طَيْرًا أَبَابِيلَ ﴿٣﴾\\
\textamh{4.\  } & تَرْمِيهِم بِحِجَارَةٍۢ مِّن سِجِّيلٍۢ ﴿٤﴾\\
\textamh{5.\  } & فَجَعَلَهُمْ كَعَصْفٍۢ مَّأْكُولٍۭ ﴿٥﴾\\
\end{longtable}
\clearpage
%% License: BSD style (Berkley) (i.e. Put the Copyright owner's name always)
%% Writer and Copyright (to): Bewketu(Bilal) Tadilo (2016-17)
\centering\section{\LR{\textamharic{ሱራቱ ቁሬይሽ -}  \RL{سوره  قريش}}}
\begin{longtable}{%
  @{}
    p{.5\textwidth}
  @{~~~~~~~~~~~~~}
    p{.5\textwidth}
    @{}
}
\nopagebreak
\textamh{ቢስሚላሂ አራህመኒ ራሂይም } &  بِسْمِ ٱللَّهِ ٱلرَّحْمَـٰنِ ٱلرَّحِيمِ\\
\textamh{1.\  } &  لِإِيلَـٰفِ قُرَيْشٍ ﴿١﴾\\
\textamh{2.\  } & إِۦلَـٰفِهِمْ رِحْلَةَ ٱلشِّتَآءِ وَٱلصَّيْفِ ﴿٢﴾\\
\textamh{3.\  } & فَلْيَعْبُدُوا۟ رَبَّ هَـٰذَا ٱلْبَيْتِ ﴿٣﴾\\
\textamh{4.\  } & ٱلَّذِىٓ أَطْعَمَهُم مِّن جُوعٍۢ وَءَامَنَهُم مِّنْ خَوْفٍۭ ﴿٤﴾\\
\end{longtable}
\clearpage
%% License: BSD style (Berkley) (i.e. Put the Copyright owner's name always)
%% Writer and Copyright (to): Bewketu(Bilal) Tadilo (2016-17)
\centering\section{\LR{\textamharic{ሱራቱ አልማኣዉን -}  \RL{سوره  الماعون}}}
\begin{longtable}{%
  @{}
    p{.5\textwidth}
  @{~~~~~~~~~~~~~}
    p{.5\textwidth}
    @{}
}
\nopagebreak
\textamh{\ \ \ \ \ \  ቢስሚላሂ አራህመኒ ራሂይም } &  بِسْمِ ٱللَّهِ ٱلرَّحْمَـٰنِ ٱلرَّحِيمِ\\
\textamh{1.\  } &  أَرَءَيْتَ ٱلَّذِى يُكَذِّبُ بِٱلدِّينِ ﴿١﴾\\
\textamh{2.\  } & فَذَٟلِكَ ٱلَّذِى يَدُعُّ ٱلْيَتِيمَ ﴿٢﴾\\
\textamh{3.\  } & وَلَا يَحُضُّ عَلَىٰ طَعَامِ ٱلْمِسْكِينِ ﴿٣﴾\\
\textamh{4.\  } & فَوَيْلٌۭ لِّلْمُصَلِّينَ ﴿٤﴾\\
\textamh{5.\  } & ٱلَّذِينَ هُمْ عَن صَلَاتِهِمْ سَاهُونَ ﴿٥﴾\\
\textamh{6.\  } & ٱلَّذِينَ هُمْ يُرَآءُونَ ﴿٦﴾\\
\textamh{7.\  } & وَيَمْنَعُونَ ٱلْمَاعُونَ ﴿٧﴾\\
\end{longtable} \newpage

%% License: BSD style (Berkley) (i.e. Put the Copyright owner's name always)
%% Writer and Copyright (to): Bewketu(Bilal) Tadilo (2016-17)
\centering\section{\LR{\textamharic{ሱራቱ አልከውታር -}  \RL{سوره  الكوثر}}}
\begin{longtable}{%
  @{}
    p{.5\textwidth}
  @{~~~~~~~~~~~~}
    p{.5\textwidth}
    @{}
}
\nopagebreak
\textamh{ቢስሚላሂ አራህመኒ ራሂይም } &  بِسْمِ ٱللَّهِ ٱلرَّحْمَـٰنِ ٱلرَّحِيمِ\\
\textamh{1.\  } &  إِنَّآ أَعْطَيْنَـٰكَ ٱلْكَوْثَرَ ﴿١﴾\\
\textamh{2.\  } & فَصَلِّ لِرَبِّكَ وَٱنْحَرْ ﴿٢﴾\\
\textamh{3.\  } & إِنَّ شَانِئَكَ هُوَ ٱلْأَبْتَرُ ﴿٣﴾\\
\end{longtable}
\clearpage
%% License: BSD style (Berkley) (i.e. Put the Copyright owner's name always)
%% Writer and Copyright (to): Bewketu(Bilal) Tadilo (2016-17)
\centering\section{\LR{\textamharic{ሱራቱ አልካፊሩውን -}  \RL{سوره  الكافرون}}}
\begin{longtable}{%
  @{}
    p{.5\textwidth}
  @{~~~~~~~~~~~~~}
    p{.5\textwidth}
    @{}
}
\nopagebreak
\textamh{\ \ \ \ \ \  ቢስሚላሂ አራህመኒ ራሂይም } &  بِسْمِ ٱللَّهِ ٱلرَّحْمَـٰنِ ٱلرَّحِيمِ\\
\textamh{1.\  } &  قُلْ يَـٰٓأَيُّهَا ٱلْكَـٰفِرُونَ ﴿١﴾\\
\textamh{2.\  } & لَآ أَعْبُدُ مَا تَعْبُدُونَ ﴿٢﴾\\
\textamh{3.\  } & وَلَآ أَنتُمْ عَـٰبِدُونَ مَآ أَعْبُدُ ﴿٣﴾\\
\textamh{4.\  } & وَلَآ أَنَا۠ عَابِدٌۭ مَّا عَبَدتُّمْ ﴿٤﴾\\
\textamh{5.\  } & وَلَآ أَنتُمْ عَـٰبِدُونَ مَآ أَعْبُدُ ﴿٥﴾\\
\textamh{6.\  } & لَكُمْ دِينُكُمْ وَلِىَ دِينِ ﴿٦﴾\\
\end{longtable} \newpage

%% License: BSD style (Berkley) (i.e. Put the Copyright owner's name always)
%% Writer and Copyright (to): Bewketu(Bilal) Tadilo (2016-17)
\centering\section{\LR{\textamharic{ሱራቱ አንነስር -}  \RL{سوره  النصر}}}
\begin{longtable}{%
  @{}
    p{.5\textwidth}
  @{~~~~~~~~~~~~~}
    p{.5\textwidth}
    @{}
}
\nopagebreak
\textamh{\ \ \ \ \ \  ቢስሚላሂ አራህመኒ ራሂይም } &  بِسْمِ ٱللَّهِ ٱلرَّحْمَـٰنِ ٱلرَّحِيمِ\\
\textamh{1.\  } &  إِذَا جَآءَ نَصْرُ ٱللَّهِ وَٱلْفَتْحُ ﴿١﴾\\
\textamh{2.\  } & وَرَأَيْتَ ٱلنَّاسَ يَدْخُلُونَ فِى دِينِ ٱللَّهِ أَفْوَاجًۭا ﴿٢﴾\\
\textamh{3.\  } & فَسَبِّحْ بِحَمْدِ رَبِّكَ وَٱسْتَغْفِرْهُ ۚ إِنَّهُۥ كَانَ تَوَّابًۢا ﴿٣﴾\\
\end{longtable} \newpage

%% License: BSD style (Berkley) (i.e. Put the Copyright owner's name always)
%% Writer and Copyright (to): Bewketu(Bilal) Tadilo (2016-17)
\centering\section{\LR{\textamharic{ሱራቱ አልመሰድ -}  \RL{سوره  المسد}}}
\begin{longtable}{%
  @{}
    p{.5\textwidth}
  @{~~~~~~~~~~~~~}
    p{.5\textwidth}
    @{}
}
\nopagebreak
\textamh{\ \ \ \ \ \  ቢስሚላሂ አራህመኒ ራሂይም } &  بِسْمِ ٱللَّهِ ٱلرَّحْمَـٰنِ ٱلرَّحِيمِ\\
\textamh{1.\  } &  تَبَّتْ يَدَآ أَبِى لَهَبٍۢ وَتَبَّ ﴿١﴾\\
\textamh{2.\  } & مَآ أَغْنَىٰ عَنْهُ مَالُهُۥ وَمَا كَسَبَ ﴿٢﴾\\
\textamh{3.\  } & سَيَصْلَىٰ نَارًۭا ذَاتَ لَهَبٍۢ ﴿٣﴾\\
\textamh{4.\  } & وَٱمْرَأَتُهُۥ حَمَّالَةَ ٱلْحَطَبِ ﴿٤﴾\\
\textamh{5.\  } & فِى جِيدِهَا حَبْلٌۭ مِّن مَّسَدٍۭ ﴿٥﴾\\
\end{longtable} \newpage

%% License: BSD style (Berkley) (i.e. Put the Copyright owner's name always)
%% Writer and Copyright (to): Bewketu(Bilal) Tadilo (2016-17)
\centering\section{\LR{\textamharic{ሱራቱ አልኢኽላስ -}  \RL{سوره  الإخلاص}}}
\begin{longtable}{%
  @{}
    p{.5\textwidth}
  @{~~~~~~~~~~~~~}
    p{.5\textwidth}
    @{}
}
\nopagebreak
\textamh{\ \ \ \ \ \  ቢስሚላሂ አራህመኒ ራሂይም } &  بِسْمِ ٱللَّهِ ٱلرَّحْمَـٰنِ ٱلرَّحِيمِ\\
\textamh{1.\  } &  قُلْ هُوَ ٱللَّهُ أَحَدٌ ﴿١﴾\\
\textamh{2.\  } & ٱللَّهُ ٱلصَّمَدُ ﴿٢﴾\\
\textamh{3.\  } & لَمْ يَلِدْ وَلَمْ يُولَدْ ﴿٣﴾\\
\textamh{4.\  } & وَلَمْ يَكُن لَّهُۥ كُفُوًا أَحَدٌۢ ﴿٤﴾\\
\end{longtable} \newpage

%% License: BSD style (Berkley) (i.e. Put the Copyright owner's name always)
%% Writer and Copyright (to): Bewketu(Bilal) Tadilo (2016-17)
\centering\section{\LR{\textamharic{ሱራቱ አልፈለቅ -}  \RL{سوره  الفلق}}}
\begin{longtable}{%
  @{}
    p{.5\textwidth}
  @{~~~~~~~~~~~~}
    p{.5\textwidth}
    @{}
}
\nopagebreak
\textamh{ቢስሚላሂ አራህመኒ ራሂይም } &  بِسْمِ ٱللَّهِ ٱلرَّحْمَـٰنِ ٱلرَّحِيمِ\\
\textamh{1.\  } &  قُلْ أَعُوذُ بِرَبِّ ٱلْفَلَقِ ﴿١﴾\\
\textamh{2.\  } & مِن شَرِّ مَا خَلَقَ ﴿٢﴾\\
\textamh{3.\  } & وَمِن شَرِّ غَاسِقٍ إِذَا وَقَبَ ﴿٣﴾\\
\textamh{4.\  } & وَمِن شَرِّ ٱلنَّفَّٰثَـٰتِ فِى ٱلْعُقَدِ ﴿٤﴾\\
\textamh{5.\  } & وَمِن شَرِّ حَاسِدٍ إِذَا حَسَدَ ﴿٥﴾\\
\end{longtable}
\clearpage
%% License: BSD style (Berkley) (i.e. Put the Copyright owner's name always)
%% Writer and Copyright (to): Bewketu(Bilal) Tadilo (2016-17)
\centering\section{\LR{\textamharic{ሱራቱ አንናስ -}  \RL{سوره  الناس}}}
\begin{longtable}{%
  @{}
    p{.5\textwidth}
  @{~~~~~~~~~~~~}
    p{.5\textwidth}
    @{}
}
\nopagebreak
\textamh{ቢስሚላሂ አራህመኒ ራሂይም } &  بِسْمِ ٱللَّهِ ٱلرَّحْمَـٰنِ ٱلرَّحِيمِ\\
\textamh{1.\  } &  قُلْ أَعُوذُ بِرَبِّ ٱلنَّاسِ ﴿١﴾\\
\textamh{2.\  } & مَلِكِ ٱلنَّاسِ ﴿٢﴾\\
\textamh{3.\  } & إِلَـٰهِ ٱلنَّاسِ ﴿٣﴾\\
\textamh{4.\  } & مِن شَرِّ ٱلْوَسْوَاسِ ٱلْخَنَّاسِ ﴿٤﴾\\
\textamh{5.\  } & ٱلَّذِى يُوَسْوِسُ فِى صُدُورِ ٱلنَّاسِ ﴿٥﴾\\
\textamh{6.\  } & مِنَ ٱلْجِنَّةِ وَٱلنَّاس ﴿٦﴾\\
\end{longtable}
\end{document}
