%% License: BSD style (Berkley) (i.e. Put the Copyright owner's name always)
%% Writer and Copyright (to): Bewketu(Bilal) Tadilo (2016-17)
\documentclass[11pt,a4paper,oneside,openleft]{article}%, ,fleqn oneside]
\usepackage[top=3cm,bottom=3cm,left=3.2cm,right=3.2cm,headsep=10pt,a4paper]{geometry}
\usepackage{fontspec}
\usepackage{graphicx}
%usepackage{arabxetex}
%usepackage{titletoc}
\usepackage{booktabs}
\usepackage{longtable}
\usepackage{polyglossia}
\usepackage{ifxetex}
\usepackage{xcolor} % Required for specifying colors by name

\usepackage{hyperref}
%usepackage{fancyhdr}
%citecolor={dark-blue},urlcolor={dark-blue}
\hypersetup{colorlinks,linkcolor={black}, pdfauthor= {Bilal Al Gonder/x-bewketu}, pdftitle={ቁርኣን ከሪም-ኢሥላም}, pdfsubject={ቁርኣን,ኢሥላም}}

%\\*titlecontents{section}[1 25cm] % Indentation
%{\\*addvspace{5pt}\bfseries} % Spacing and font options for sections
%{\\*contentslabel[\thecontentslabel]{1
%25cm}} % Section number
%{}
%{\\*hfill\color{black}\thecontentspage} % Page number
%[]
%usepackage{underscore}
\newfontfamily\arabicfont[Script=Arabic, Scale = 1.5]{mequran}%KFGQPC Uthmanic Script HAFS}% Taha Naskh}
%\\*newfontfamily\arabicfonttt[Script = Arabic, Scale = 1.5]{Amiri}%KFGQPC Uthmanic Script HAFS}% Taha Naskh}
\newfontfamily\amharicfont[Script=Ethiopic, Scale = 1.1]{Abyssinica SIL}
\setmainlanguage[locale=mashriq]{arabic}
\setotherlanguage{amharic}
\newcommand{\mytextarabic}[1]{\RL{ #1 \flushright}}
%usepackage{array}[numerals=mashriq]

%newcommand{\textLANGUAGENAME}[1]{\begin{langname}#1\end{langname}}
 %put here your language name
\newcommand{\textamh}[1]{\LR{\begin{amharic}#1\flushleft\end{amharic}}} 
\newcommand{\textamhsec}[1]{\begin{amharic} #1 \end{amharic}}
\let\sbkslg\longtable
\def\longtable{\noindent\noindent\noindent\noindent\sbkslg}
%\selectbackgroundlanguage{arabic}
%title{}
%\\*author{}
\begin{document}
%\\*maketitle
%\\*Hijritoday[0] 
%\\*textamharic{\today-} 
\tableofcontents{}
\%%%%%%%%%%%%%%%%%%%%%%%%%
%TODO: please substitute alif which is broken by      ـٰ
%%%%%%%%%%%%%%%%%%%%%%
\cleardoublepage
\pagenumbering{arabic}
%pagestyle{fancy}
\centering\section{\LR{\textamharic{ሱራቱ አልፈቲሃ - } \RL{سوره  الفاتحة}}}
\begin{longtable}{%
  @{}
    p{.4\textwidth}
  @{~~~~~~~~~~~~}
    p{.4\textwidth}
    @{}
}
\nopagebreak
\textamh{1.\ ቢስሚላሂ አራህመኒ ራሂይም } &  بِسْمِ ٱللَّهِ الرَّحْمَـٰنِ الرَّحِيمِ﴿١﴾}     \\*
\textamh{2.\ (ኣልሃምዱሊላሂ) ምስጋና ሁሉ ለኣላህ የአለሚን (የሰዎች፥ ጅኖች፥ ያለ ነገር ሁሉ) ጌታ } & ٱلْحَمْدُ لِلَّهِ رَبِّ ٱلْعَـٰلَمِينَ﴿٢﴾} \\*
\textamh{3.\ ከሁሉም በላይ ሰጪ፥ ከሁሉም በላይ ምህረተኛው } & ٱلرَّحْمَـٰنِ ٱلرَّحِيمِ﴿٣﴾}   \\*
\textamh{4.\ የዛች ቀን (የፍርድ ቀን) ብቸኛ ባለቤት } &   مَـٰلِكِ يَوْمِ ٱلدِّينِ ﴿٤﴾}   \\*
\textamh{5.\ አንተን ብቻ እናመልካለን፤ አንተን ብቻ እርዳታ እንጠይቃለን } &  إِيَّاكَ نَعْبُدُ وَإِيَّاكَ نَسْتَعِينُ ﴿٥﴾}   \\*
\textamh{6.\ ምራነ በቀጥተኛው (በትክክለኛው)  መንገድ } &  ٱهْدِنَا ٱلصِّرَٟطَ ٱلْمُسْتَقِيمَ ﴿٦﴾}  \\*
\textamh{7.\ የአንተን ፀጋ ያደረግክላቸውን (ሰዎች)  መንገድ፥ የአንተን ቁጣ እንዳተርፉት (እንደይሁዶች) ሳይሆን ፥እንደሳቱትም (እንደክርስቲያኖች) ሳይሆን } &   صِرَٟطَ ٱلَّذِينَ أَنْعَمْتَ عَلَيْهِمْ غَيْرِ ٱلْمَغْضُوبِ عَلَيْهِمْ وَلَا ٱلضَّآلِّينَ ﴿٧﴾} 
\end{longtable}
\clearpage
\end{document}
