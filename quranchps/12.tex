%% License: BSD style (Berkley) (i.e. Put the Copyright owner's name always)
%% Writer and Copyright (to): Bewketu(Bilal) Tadilo (2016-17)
\begin{center}\section{ሱራቱ ዩሱፍ -  \textarabic{سوره  يوسف}}\end{center}
\begin{longtable}{%
  @{}
    p{.5\textwidth}
  @{~~~}
    p{.5\textwidth}
    @{}
}
ቢስሚላሂ አራህመኒ ራሂይም &  \mytextarabic{بِسْمِ ٱللَّهِ ٱلرَّحْمَـٰنِ ٱلرَّحِيمِ}\\
1.\  & \mytextarabic{ الٓر ۚ تِلْكَ ءَايَـٰتُ ٱلْكِتَـٰبِ ٱلْمُبِينِ ﴿١﴾}\\
2.\  & \mytextarabic{إِنَّآ أَنزَلْنَـٰهُ قُرْءَٰنًا عَرَبِيًّۭا لَّعَلَّكُمْ تَعْقِلُونَ ﴿٢﴾}\\
3.\  & \mytextarabic{نَحْنُ نَقُصُّ عَلَيْكَ أَحْسَنَ ٱلْقَصَصِ بِمَآ أَوْحَيْنَآ إِلَيْكَ هَـٰذَا ٱلْقُرْءَانَ وَإِن كُنتَ مِن قَبْلِهِۦ لَمِنَ ٱلْغَٰفِلِينَ ﴿٣﴾}\\
4.\  & \mytextarabic{إِذْ قَالَ يُوسُفُ لِأَبِيهِ يَـٰٓأَبَتِ إِنِّى رَأَيْتُ أَحَدَ عَشَرَ كَوْكَبًۭا وَٱلشَّمْسَ وَٱلْقَمَرَ رَأَيْتُهُمْ لِى سَـٰجِدِينَ ﴿٤﴾}\\
5.\  & \mytextarabic{قَالَ يَـٰبُنَىَّ لَا تَقْصُصْ رُءْيَاكَ عَلَىٰٓ إِخْوَتِكَ فَيَكِيدُوا۟ لَكَ كَيْدًا ۖ إِنَّ ٱلشَّيْطَٰنَ لِلْإِنسَـٰنِ عَدُوٌّۭ مُّبِينٌۭ ﴿٥﴾}\\
6.\  & \mytextarabic{وَكَذَٟلِكَ يَجْتَبِيكَ رَبُّكَ وَيُعَلِّمُكَ مِن تَأْوِيلِ ٱلْأَحَادِيثِ وَيُتِمُّ نِعْمَتَهُۥ عَلَيْكَ وَعَلَىٰٓ ءَالِ يَعْقُوبَ كَمَآ أَتَمَّهَا عَلَىٰٓ أَبَوَيْكَ مِن قَبْلُ إِبْرَٰهِيمَ وَإِسْحَـٰقَ ۚ إِنَّ رَبَّكَ عَلِيمٌ حَكِيمٌۭ ﴿٦﴾}\\
7.\  & \mytextarabic{۞ لَّقَدْ كَانَ فِى يُوسُفَ وَإِخْوَتِهِۦٓ ءَايَـٰتٌۭ لِّلسَّآئِلِينَ ﴿٧﴾}\\
8.\  & \mytextarabic{إِذْ قَالُوا۟ لَيُوسُفُ وَأَخُوهُ أَحَبُّ إِلَىٰٓ أَبِينَا مِنَّا وَنَحْنُ عُصْبَةٌ إِنَّ أَبَانَا لَفِى ضَلَـٰلٍۢ مُّبِينٍ ﴿٨﴾}\\
9.\  & \mytextarabic{ٱقْتُلُوا۟ يُوسُفَ أَوِ ٱطْرَحُوهُ أَرْضًۭا يَخْلُ لَكُمْ وَجْهُ أَبِيكُمْ وَتَكُونُوا۟ مِنۢ بَعْدِهِۦ قَوْمًۭا صَـٰلِحِينَ ﴿٩﴾}\\
10.\  & \mytextarabic{قَالَ قَآئِلٌۭ مِّنْهُمْ لَا تَقْتُلُوا۟ يُوسُفَ وَأَلْقُوهُ فِى غَيَـٰبَتِ ٱلْجُبِّ يَلْتَقِطْهُ بَعْضُ ٱلسَّيَّارَةِ إِن كُنتُمْ فَـٰعِلِينَ ﴿١٠﴾}\\
11.\  & \mytextarabic{قَالُوا۟ يَـٰٓأَبَانَا مَا لَكَ لَا تَأْمَ۫نَّا عَلَىٰ يُوسُفَ وَإِنَّا لَهُۥ لَنَـٰصِحُونَ ﴿١١﴾}\\
12.\  & \mytextarabic{أَرْسِلْهُ مَعَنَا غَدًۭا يَرْتَعْ وَيَلْعَبْ وَإِنَّا لَهُۥ لَحَـٰفِظُونَ ﴿١٢﴾}\\
13.\  & \mytextarabic{قَالَ إِنِّى لَيَحْزُنُنِىٓ أَن تَذْهَبُوا۟ بِهِۦ وَأَخَافُ أَن يَأْكُلَهُ ٱلذِّئْبُ وَأَنتُمْ عَنْهُ غَٰفِلُونَ ﴿١٣﴾}\\
14.\  & \mytextarabic{قَالُوا۟ لَئِنْ أَكَلَهُ ٱلذِّئْبُ وَنَحْنُ عُصْبَةٌ إِنَّآ إِذًۭا لَّخَـٰسِرُونَ ﴿١٤﴾}\\
15.\  & \mytextarabic{فَلَمَّا ذَهَبُوا۟ بِهِۦ وَأَجْمَعُوٓا۟ أَن يَجْعَلُوهُ فِى غَيَـٰبَتِ ٱلْجُبِّ ۚ وَأَوْحَيْنَآ إِلَيْهِ لَتُنَبِّئَنَّهُم بِأَمْرِهِمْ هَـٰذَا وَهُمْ لَا يَشْعُرُونَ ﴿١٥﴾}\\
16.\  & \mytextarabic{وَجَآءُوٓ أَبَاهُمْ عِشَآءًۭ يَبْكُونَ ﴿١٦﴾}\\
17.\  & \mytextarabic{قَالُوا۟ يَـٰٓأَبَانَآ إِنَّا ذَهَبْنَا نَسْتَبِقُ وَتَرَكْنَا يُوسُفَ عِندَ مَتَـٰعِنَا فَأَكَلَهُ ٱلذِّئْبُ ۖ وَمَآ أَنتَ بِمُؤْمِنٍۢ لَّنَا وَلَوْ كُنَّا صَـٰدِقِينَ ﴿١٧﴾}\\
18.\  & \mytextarabic{وَجَآءُو عَلَىٰ قَمِيصِهِۦ بِدَمٍۢ كَذِبٍۢ ۚ قَالَ بَلْ سَوَّلَتْ لَكُمْ أَنفُسُكُمْ أَمْرًۭا ۖ فَصَبْرٌۭ جَمِيلٌۭ ۖ وَٱللَّهُ ٱلْمُسْتَعَانُ عَلَىٰ مَا تَصِفُونَ ﴿١٨﴾}\\
19.\  & \mytextarabic{وَجَآءَتْ سَيَّارَةٌۭ فَأَرْسَلُوا۟ وَارِدَهُمْ فَأَدْلَىٰ دَلْوَهُۥ ۖ قَالَ يَـٰبُشْرَىٰ هَـٰذَا غُلَـٰمٌۭ ۚ وَأَسَرُّوهُ بِضَٰعَةًۭ ۚ وَٱللَّهُ عَلِيمٌۢ بِمَا يَعْمَلُونَ ﴿١٩﴾}\\
20.\  & \mytextarabic{وَشَرَوْهُ بِثَمَنٍۭ بَخْسٍۢ دَرَٰهِمَ مَعْدُودَةٍۢ وَكَانُوا۟ فِيهِ مِنَ ٱلزَّٰهِدِينَ ﴿٢٠﴾}\\
21.\  & \mytextarabic{وَقَالَ ٱلَّذِى ٱشْتَرَىٰهُ مِن مِّصْرَ لِٱمْرَأَتِهِۦٓ أَكْرِمِى مَثْوَىٰهُ عَسَىٰٓ أَن يَنفَعَنَآ أَوْ نَتَّخِذَهُۥ وَلَدًۭا ۚ وَكَذَٟلِكَ مَكَّنَّا لِيُوسُفَ فِى ٱلْأَرْضِ وَلِنُعَلِّمَهُۥ مِن تَأْوِيلِ ٱلْأَحَادِيثِ ۚ وَٱللَّهُ غَالِبٌ عَلَىٰٓ أَمْرِهِۦ وَلَـٰكِنَّ أَكْثَرَ ٱلنَّاسِ لَا يَعْلَمُونَ ﴿٢١﴾}\\
22.\  & \mytextarabic{وَلَمَّا بَلَغَ أَشُدَّهُۥٓ ءَاتَيْنَـٰهُ حُكْمًۭا وَعِلْمًۭا ۚ وَكَذَٟلِكَ نَجْزِى ٱلْمُحْسِنِينَ ﴿٢٢﴾}\\
23.\  & \mytextarabic{وَرَٰوَدَتْهُ ٱلَّتِى هُوَ فِى بَيْتِهَا عَن نَّفْسِهِۦ وَغَلَّقَتِ ٱلْأَبْوَٟبَ وَقَالَتْ هَيْتَ لَكَ ۚ قَالَ مَعَاذَ ٱللَّهِ ۖ إِنَّهُۥ رَبِّىٓ أَحْسَنَ مَثْوَاىَ ۖ إِنَّهُۥ لَا يُفْلِحُ ٱلظَّـٰلِمُونَ ﴿٢٣﴾}\\
24.\  & \mytextarabic{وَلَقَدْ هَمَّتْ بِهِۦ ۖ وَهَمَّ بِهَا لَوْلَآ أَن رَّءَا بُرْهَـٰنَ رَبِّهِۦ ۚ كَذَٟلِكَ لِنَصْرِفَ عَنْهُ ٱلسُّوٓءَ وَٱلْفَحْشَآءَ ۚ إِنَّهُۥ مِنْ عِبَادِنَا ٱلْمُخْلَصِينَ ﴿٢٤﴾}\\
25.\  & \mytextarabic{وَٱسْتَبَقَا ٱلْبَابَ وَقَدَّتْ قَمِيصَهُۥ مِن دُبُرٍۢ وَأَلْفَيَا سَيِّدَهَا لَدَا ٱلْبَابِ ۚ قَالَتْ مَا جَزَآءُ مَنْ أَرَادَ بِأَهْلِكَ سُوٓءًا إِلَّآ أَن يُسْجَنَ أَوْ عَذَابٌ أَلِيمٌۭ ﴿٢٥﴾}\\
26.\  & \mytextarabic{قَالَ هِىَ رَٰوَدَتْنِى عَن نَّفْسِى ۚ وَشَهِدَ شَاهِدٌۭ مِّنْ أَهْلِهَآ إِن كَانَ قَمِيصُهُۥ قُدَّ مِن قُبُلٍۢ فَصَدَقَتْ وَهُوَ مِنَ ٱلْكَـٰذِبِينَ ﴿٢٦﴾}\\
27.\  & \mytextarabic{وَإِن كَانَ قَمِيصُهُۥ قُدَّ مِن دُبُرٍۢ فَكَذَبَتْ وَهُوَ مِنَ ٱلصَّـٰدِقِينَ ﴿٢٧﴾}\\
28.\  & \mytextarabic{فَلَمَّا رَءَا قَمِيصَهُۥ قُدَّ مِن دُبُرٍۢ قَالَ إِنَّهُۥ مِن كَيْدِكُنَّ ۖ إِنَّ كَيْدَكُنَّ عَظِيمٌۭ ﴿٢٨﴾}\\
29.\  & \mytextarabic{يُوسُفُ أَعْرِضْ عَنْ هَـٰذَا ۚ وَٱسْتَغْفِرِى لِذَنۢبِكِ ۖ إِنَّكِ كُنتِ مِنَ ٱلْخَاطِـِٔينَ ﴿٢٩﴾}\\
30.\  & \mytextarabic{۞ وَقَالَ نِسْوَةٌۭ فِى ٱلْمَدِينَةِ ٱمْرَأَتُ ٱلْعَزِيزِ تُرَٰوِدُ فَتَىٰهَا عَن نَّفْسِهِۦ ۖ قَدْ شَغَفَهَا حُبًّا ۖ إِنَّا لَنَرَىٰهَا فِى ضَلَـٰلٍۢ مُّبِينٍۢ ﴿٣٠﴾}\\
31.\  & \mytextarabic{فَلَمَّا سَمِعَتْ بِمَكْرِهِنَّ أَرْسَلَتْ إِلَيْهِنَّ وَأَعْتَدَتْ لَهُنَّ مُتَّكَـًۭٔا وَءَاتَتْ كُلَّ وَٟحِدَةٍۢ مِّنْهُنَّ سِكِّينًۭا وَقَالَتِ ٱخْرُجْ عَلَيْهِنَّ ۖ فَلَمَّا رَأَيْنَهُۥٓ أَكْبَرْنَهُۥ وَقَطَّعْنَ أَيْدِيَهُنَّ وَقُلْنَ حَـٰشَ لِلَّهِ مَا هَـٰذَا بَشَرًا إِنْ هَـٰذَآ إِلَّا مَلَكٌۭ كَرِيمٌۭ ﴿٣١﴾}\\
32.\  & \mytextarabic{قَالَتْ فَذَٟلِكُنَّ ٱلَّذِى لُمْتُنَّنِى فِيهِ ۖ وَلَقَدْ رَٰوَدتُّهُۥ عَن نَّفْسِهِۦ فَٱسْتَعْصَمَ ۖ وَلَئِن لَّمْ يَفْعَلْ مَآ ءَامُرُهُۥ لَيُسْجَنَنَّ وَلَيَكُونًۭا مِّنَ ٱلصَّـٰغِرِينَ ﴿٣٢﴾}\\
33.\  & \mytextarabic{قَالَ رَبِّ ٱلسِّجْنُ أَحَبُّ إِلَىَّ مِمَّا يَدْعُونَنِىٓ إِلَيْهِ ۖ وَإِلَّا تَصْرِفْ عَنِّى كَيْدَهُنَّ أَصْبُ إِلَيْهِنَّ وَأَكُن مِّنَ ٱلْجَٰهِلِينَ ﴿٣٣﴾}\\
34.\  & \mytextarabic{فَٱسْتَجَابَ لَهُۥ رَبُّهُۥ فَصَرَفَ عَنْهُ كَيْدَهُنَّ ۚ إِنَّهُۥ هُوَ ٱلسَّمِيعُ ٱلْعَلِيمُ ﴿٣٤﴾}\\
35.\  & \mytextarabic{ثُمَّ بَدَا لَهُم مِّنۢ بَعْدِ مَا رَأَوُا۟ ٱلْءَايَـٰتِ لَيَسْجُنُنَّهُۥ حَتَّىٰ حِينٍۢ ﴿٣٥﴾}\\
36.\  & \mytextarabic{وَدَخَلَ مَعَهُ ٱلسِّجْنَ فَتَيَانِ ۖ قَالَ أَحَدُهُمَآ إِنِّىٓ أَرَىٰنِىٓ أَعْصِرُ خَمْرًۭا ۖ وَقَالَ ٱلْءَاخَرُ إِنِّىٓ أَرَىٰنِىٓ أَحْمِلُ فَوْقَ رَأْسِى خُبْزًۭا تَأْكُلُ ٱلطَّيْرُ مِنْهُ ۖ نَبِّئْنَا بِتَأْوِيلِهِۦٓ ۖ إِنَّا نَرَىٰكَ مِنَ ٱلْمُحْسِنِينَ ﴿٣٦﴾}\\
37.\  & \mytextarabic{قَالَ لَا يَأْتِيكُمَا طَعَامٌۭ تُرْزَقَانِهِۦٓ إِلَّا نَبَّأْتُكُمَا بِتَأْوِيلِهِۦ قَبْلَ أَن يَأْتِيَكُمَا ۚ ذَٟلِكُمَا مِمَّا عَلَّمَنِى رَبِّىٓ ۚ إِنِّى تَرَكْتُ مِلَّةَ قَوْمٍۢ لَّا يُؤْمِنُونَ بِٱللَّهِ وَهُم بِٱلْءَاخِرَةِ هُمْ كَـٰفِرُونَ ﴿٣٧﴾}\\
38.\  & \mytextarabic{وَٱتَّبَعْتُ مِلَّةَ ءَابَآءِىٓ إِبْرَٰهِيمَ وَإِسْحَـٰقَ وَيَعْقُوبَ ۚ مَا كَانَ لَنَآ أَن نُّشْرِكَ بِٱللَّهِ مِن شَىْءٍۢ ۚ ذَٟلِكَ مِن فَضْلِ ٱللَّهِ عَلَيْنَا وَعَلَى ٱلنَّاسِ وَلَـٰكِنَّ أَكْثَرَ ٱلنَّاسِ لَا يَشْكُرُونَ ﴿٣٨﴾}\\
39.\  & \mytextarabic{يَـٰصَىٰحِبَىِ ٱلسِّجْنِ ءَأَرْبَابٌۭ مُّتَفَرِّقُونَ خَيْرٌ أَمِ ٱللَّهُ ٱلْوَٟحِدُ ٱلْقَهَّارُ ﴿٣٩﴾}\\
40.\  & \mytextarabic{مَا تَعْبُدُونَ مِن دُونِهِۦٓ إِلَّآ أَسْمَآءًۭ سَمَّيْتُمُوهَآ أَنتُمْ وَءَابَآؤُكُم مَّآ أَنزَلَ ٱللَّهُ بِهَا مِن سُلْطَٰنٍ ۚ إِنِ ٱلْحُكْمُ إِلَّا لِلَّهِ ۚ أَمَرَ أَلَّا تَعْبُدُوٓا۟ إِلَّآ إِيَّاهُ ۚ ذَٟلِكَ ٱلدِّينُ ٱلْقَيِّمُ وَلَـٰكِنَّ أَكْثَرَ ٱلنَّاسِ لَا يَعْلَمُونَ ﴿٤٠﴾}\\
41.\  & \mytextarabic{يَـٰصَىٰحِبَىِ ٱلسِّجْنِ أَمَّآ أَحَدُكُمَا فَيَسْقِى رَبَّهُۥ خَمْرًۭا ۖ وَأَمَّا ٱلْءَاخَرُ فَيُصْلَبُ فَتَأْكُلُ ٱلطَّيْرُ مِن رَّأْسِهِۦ ۚ قُضِىَ ٱلْأَمْرُ ٱلَّذِى فِيهِ تَسْتَفْتِيَانِ ﴿٤١﴾}\\
42.\  & \mytextarabic{وَقَالَ لِلَّذِى ظَنَّ أَنَّهُۥ نَاجٍۢ مِّنْهُمَا ٱذْكُرْنِى عِندَ رَبِّكَ فَأَنسَىٰهُ ٱلشَّيْطَٰنُ ذِكْرَ رَبِّهِۦ فَلَبِثَ فِى ٱلسِّجْنِ بِضْعَ سِنِينَ ﴿٤٢﴾}\\
43.\  & \mytextarabic{وَقَالَ ٱلْمَلِكُ إِنِّىٓ أَرَىٰ سَبْعَ بَقَرَٰتٍۢ سِمَانٍۢ يَأْكُلُهُنَّ سَبْعٌ عِجَافٌۭ وَسَبْعَ سُنۢبُلَـٰتٍ خُضْرٍۢ وَأُخَرَ يَابِسَـٰتٍۢ ۖ يَـٰٓأَيُّهَا ٱلْمَلَأُ أَفْتُونِى فِى رُءْيَـٰىَ إِن كُنتُمْ لِلرُّءْيَا تَعْبُرُونَ ﴿٤٣﴾}\\
44.\  & \mytextarabic{قَالُوٓا۟ أَضْغَٰثُ أَحْلَـٰمٍۢ ۖ وَمَا نَحْنُ بِتَأْوِيلِ ٱلْأَحْلَـٰمِ بِعَـٰلِمِينَ ﴿٤٤﴾}\\
45.\  & \mytextarabic{وَقَالَ ٱلَّذِى نَجَا مِنْهُمَا وَٱدَّكَرَ بَعْدَ أُمَّةٍ أَنَا۠ أُنَبِّئُكُم بِتَأْوِيلِهِۦ فَأَرْسِلُونِ ﴿٤٥﴾}\\
46.\  & \mytextarabic{يُوسُفُ أَيُّهَا ٱلصِّدِّيقُ أَفْتِنَا فِى سَبْعِ بَقَرَٰتٍۢ سِمَانٍۢ يَأْكُلُهُنَّ سَبْعٌ عِجَافٌۭ وَسَبْعِ سُنۢبُلَـٰتٍ خُضْرٍۢ وَأُخَرَ يَابِسَـٰتٍۢ لَّعَلِّىٓ أَرْجِعُ إِلَى ٱلنَّاسِ لَعَلَّهُمْ يَعْلَمُونَ ﴿٤٦﴾}\\
47.\  & \mytextarabic{قَالَ تَزْرَعُونَ سَبْعَ سِنِينَ دَأَبًۭا فَمَا حَصَدتُّمْ فَذَرُوهُ فِى سُنۢبُلِهِۦٓ إِلَّا قَلِيلًۭا مِّمَّا تَأْكُلُونَ ﴿٤٧﴾}\\
48.\  & \mytextarabic{ثُمَّ يَأْتِى مِنۢ بَعْدِ ذَٟلِكَ سَبْعٌۭ شِدَادٌۭ يَأْكُلْنَ مَا قَدَّمْتُمْ لَهُنَّ إِلَّا قَلِيلًۭا مِّمَّا تُحْصِنُونَ ﴿٤٨﴾}\\
49.\  & \mytextarabic{ثُمَّ يَأْتِى مِنۢ بَعْدِ ذَٟلِكَ عَامٌۭ فِيهِ يُغَاثُ ٱلنَّاسُ وَفِيهِ يَعْصِرُونَ ﴿٤٩﴾}\\
50.\  & \mytextarabic{وَقَالَ ٱلْمَلِكُ ٱئْتُونِى بِهِۦ ۖ فَلَمَّا جَآءَهُ ٱلرَّسُولُ قَالَ ٱرْجِعْ إِلَىٰ رَبِّكَ فَسْـَٔلْهُ مَا بَالُ ٱلنِّسْوَةِ ٱلَّٰتِى قَطَّعْنَ أَيْدِيَهُنَّ ۚ إِنَّ رَبِّى بِكَيْدِهِنَّ عَلِيمٌۭ ﴿٥٠﴾}\\
51.\  & \mytextarabic{قَالَ مَا خَطْبُكُنَّ إِذْ رَٰوَدتُّنَّ يُوسُفَ عَن نَّفْسِهِۦ ۚ قُلْنَ حَـٰشَ لِلَّهِ مَا عَلِمْنَا عَلَيْهِ مِن سُوٓءٍۢ ۚ قَالَتِ ٱمْرَأَتُ ٱلْعَزِيزِ ٱلْـَٰٔنَ حَصْحَصَ ٱلْحَقُّ أَنَا۠ رَٰوَدتُّهُۥ عَن نَّفْسِهِۦ وَإِنَّهُۥ لَمِنَ ٱلصَّـٰدِقِينَ ﴿٥١﴾}\\
52.\  & \mytextarabic{ذَٟلِكَ لِيَعْلَمَ أَنِّى لَمْ أَخُنْهُ بِٱلْغَيْبِ وَأَنَّ ٱللَّهَ لَا يَهْدِى كَيْدَ ٱلْخَآئِنِينَ ﴿٥٢﴾}\\
53.\  & \mytextarabic{۞ وَمَآ أُبَرِّئُ نَفْسِىٓ ۚ إِنَّ ٱلنَّفْسَ لَأَمَّارَةٌۢ بِٱلسُّوٓءِ إِلَّا مَا رَحِمَ رَبِّىٓ ۚ إِنَّ رَبِّى غَفُورٌۭ رَّحِيمٌۭ ﴿٥٣﴾}\\
54.\  & \mytextarabic{وَقَالَ ٱلْمَلِكُ ٱئْتُونِى بِهِۦٓ أَسْتَخْلِصْهُ لِنَفْسِى ۖ فَلَمَّا كَلَّمَهُۥ قَالَ إِنَّكَ ٱلْيَوْمَ لَدَيْنَا مَكِينٌ أَمِينٌۭ ﴿٥٤﴾}\\
55.\  & \mytextarabic{قَالَ ٱجْعَلْنِى عَلَىٰ خَزَآئِنِ ٱلْأَرْضِ ۖ إِنِّى حَفِيظٌ عَلِيمٌۭ ﴿٥٥﴾}\\
56.\  & \mytextarabic{وَكَذَٟلِكَ مَكَّنَّا لِيُوسُفَ فِى ٱلْأَرْضِ يَتَبَوَّأُ مِنْهَا حَيْثُ يَشَآءُ ۚ نُصِيبُ بِرَحْمَتِنَا مَن نَّشَآءُ ۖ وَلَا نُضِيعُ أَجْرَ ٱلْمُحْسِنِينَ ﴿٥٦﴾}\\
57.\  & \mytextarabic{وَلَأَجْرُ ٱلْءَاخِرَةِ خَيْرٌۭ لِّلَّذِينَ ءَامَنُوا۟ وَكَانُوا۟ يَتَّقُونَ ﴿٥٧﴾}\\
58.\  & \mytextarabic{وَجَآءَ إِخْوَةُ يُوسُفَ فَدَخَلُوا۟ عَلَيْهِ فَعَرَفَهُمْ وَهُمْ لَهُۥ مُنكِرُونَ ﴿٥٨﴾}\\
59.\  & \mytextarabic{وَلَمَّا جَهَّزَهُم بِجَهَازِهِمْ قَالَ ٱئْتُونِى بِأَخٍۢ لَّكُم مِّنْ أَبِيكُمْ ۚ أَلَا تَرَوْنَ أَنِّىٓ أُوفِى ٱلْكَيْلَ وَأَنَا۠ خَيْرُ ٱلْمُنزِلِينَ ﴿٥٩﴾}\\
60.\  & \mytextarabic{فَإِن لَّمْ تَأْتُونِى بِهِۦ فَلَا كَيْلَ لَكُمْ عِندِى وَلَا تَقْرَبُونِ ﴿٦٠﴾}\\
61.\  & \mytextarabic{قَالُوا۟ سَنُرَٰوِدُ عَنْهُ أَبَاهُ وَإِنَّا لَفَـٰعِلُونَ ﴿٦١﴾}\\
62.\  & \mytextarabic{وَقَالَ لِفِتْيَـٰنِهِ ٱجْعَلُوا۟ بِضَٰعَتَهُمْ فِى رِحَالِهِمْ لَعَلَّهُمْ يَعْرِفُونَهَآ إِذَا ٱنقَلَبُوٓا۟ إِلَىٰٓ أَهْلِهِمْ لَعَلَّهُمْ يَرْجِعُونَ ﴿٦٢﴾}\\
63.\  & \mytextarabic{فَلَمَّا رَجَعُوٓا۟ إِلَىٰٓ أَبِيهِمْ قَالُوا۟ يَـٰٓأَبَانَا مُنِعَ مِنَّا ٱلْكَيْلُ فَأَرْسِلْ مَعَنَآ أَخَانَا نَكْتَلْ وَإِنَّا لَهُۥ لَحَـٰفِظُونَ ﴿٦٣﴾}\\
64.\  & \mytextarabic{قَالَ هَلْ ءَامَنُكُمْ عَلَيْهِ إِلَّا كَمَآ أَمِنتُكُمْ عَلَىٰٓ أَخِيهِ مِن قَبْلُ ۖ فَٱللَّهُ خَيْرٌ حَـٰفِظًۭا ۖ وَهُوَ أَرْحَمُ ٱلرَّٟحِمِينَ ﴿٦٤﴾}\\
65.\  & \mytextarabic{وَلَمَّا فَتَحُوا۟ مَتَـٰعَهُمْ وَجَدُوا۟ بِضَٰعَتَهُمْ رُدَّتْ إِلَيْهِمْ ۖ قَالُوا۟ يَـٰٓأَبَانَا مَا نَبْغِى ۖ هَـٰذِهِۦ بِضَٰعَتُنَا رُدَّتْ إِلَيْنَا ۖ وَنَمِيرُ أَهْلَنَا وَنَحْفَظُ أَخَانَا وَنَزْدَادُ كَيْلَ بَعِيرٍۢ ۖ ذَٟلِكَ كَيْلٌۭ يَسِيرٌۭ ﴿٦٥﴾}\\
66.\  & \mytextarabic{قَالَ لَنْ أُرْسِلَهُۥ مَعَكُمْ حَتَّىٰ تُؤْتُونِ مَوْثِقًۭا مِّنَ ٱللَّهِ لَتَأْتُنَّنِى بِهِۦٓ إِلَّآ أَن يُحَاطَ بِكُمْ ۖ فَلَمَّآ ءَاتَوْهُ مَوْثِقَهُمْ قَالَ ٱللَّهُ عَلَىٰ مَا نَقُولُ وَكِيلٌۭ ﴿٦٦﴾}\\
67.\  & \mytextarabic{وَقَالَ يَـٰبَنِىَّ لَا تَدْخُلُوا۟ مِنۢ بَابٍۢ وَٟحِدٍۢ وَٱدْخُلُوا۟ مِنْ أَبْوَٟبٍۢ مُّتَفَرِّقَةٍۢ ۖ وَمَآ أُغْنِى عَنكُم مِّنَ ٱللَّهِ مِن شَىْءٍ ۖ إِنِ ٱلْحُكْمُ إِلَّا لِلَّهِ ۖ عَلَيْهِ تَوَكَّلْتُ ۖ وَعَلَيْهِ فَلْيَتَوَكَّلِ ٱلْمُتَوَكِّلُونَ ﴿٦٧﴾}\\
68.\  & \mytextarabic{وَلَمَّا دَخَلُوا۟ مِنْ حَيْثُ أَمَرَهُمْ أَبُوهُم مَّا كَانَ يُغْنِى عَنْهُم مِّنَ ٱللَّهِ مِن شَىْءٍ إِلَّا حَاجَةًۭ فِى نَفْسِ يَعْقُوبَ قَضَىٰهَا ۚ وَإِنَّهُۥ لَذُو عِلْمٍۢ لِّمَا عَلَّمْنَـٰهُ وَلَـٰكِنَّ أَكْثَرَ ٱلنَّاسِ لَا يَعْلَمُونَ ﴿٦٨﴾}\\
69.\  & \mytextarabic{وَلَمَّا دَخَلُوا۟ عَلَىٰ يُوسُفَ ءَاوَىٰٓ إِلَيْهِ أَخَاهُ ۖ قَالَ إِنِّىٓ أَنَا۠ أَخُوكَ فَلَا تَبْتَئِسْ بِمَا كَانُوا۟ يَعْمَلُونَ ﴿٦٩﴾}\\
70.\  & \mytextarabic{فَلَمَّا جَهَّزَهُم بِجَهَازِهِمْ جَعَلَ ٱلسِّقَايَةَ فِى رَحْلِ أَخِيهِ ثُمَّ أَذَّنَ مُؤَذِّنٌ أَيَّتُهَا ٱلْعِيرُ إِنَّكُمْ لَسَـٰرِقُونَ ﴿٧٠﴾}\\
71.\  & \mytextarabic{قَالُوا۟ وَأَقْبَلُوا۟ عَلَيْهِم مَّاذَا تَفْقِدُونَ ﴿٧١﴾}\\
72.\  & \mytextarabic{قَالُوا۟ نَفْقِدُ صُوَاعَ ٱلْمَلِكِ وَلِمَن جَآءَ بِهِۦ حِمْلُ بَعِيرٍۢ وَأَنَا۠ بِهِۦ زَعِيمٌۭ ﴿٧٢﴾}\\
73.\  & \mytextarabic{قَالُوا۟ تَٱللَّهِ لَقَدْ عَلِمْتُم مَّا جِئْنَا لِنُفْسِدَ فِى ٱلْأَرْضِ وَمَا كُنَّا سَـٰرِقِينَ ﴿٧٣﴾}\\
74.\  & \mytextarabic{قَالُوا۟ فَمَا جَزَٰٓؤُهُۥٓ إِن كُنتُمْ كَـٰذِبِينَ ﴿٧٤﴾}\\
75.\  & \mytextarabic{قَالُوا۟ جَزَٰٓؤُهُۥ مَن وُجِدَ فِى رَحْلِهِۦ فَهُوَ جَزَٰٓؤُهُۥ ۚ كَذَٟلِكَ نَجْزِى ٱلظَّـٰلِمِينَ ﴿٧٥﴾}\\
76.\  & \mytextarabic{فَبَدَأَ بِأَوْعِيَتِهِمْ قَبْلَ وِعَآءِ أَخِيهِ ثُمَّ ٱسْتَخْرَجَهَا مِن وِعَآءِ أَخِيهِ ۚ كَذَٟلِكَ كِدْنَا لِيُوسُفَ ۖ مَا كَانَ لِيَأْخُذَ أَخَاهُ فِى دِينِ ٱلْمَلِكِ إِلَّآ أَن يَشَآءَ ٱللَّهُ ۚ نَرْفَعُ دَرَجَٰتٍۢ مَّن نَّشَآءُ ۗ وَفَوْقَ كُلِّ ذِى عِلْمٍ عَلِيمٌۭ ﴿٧٦﴾}\\
77.\  & \mytextarabic{۞ قَالُوٓا۟ إِن يَسْرِقْ فَقَدْ سَرَقَ أَخٌۭ لَّهُۥ مِن قَبْلُ ۚ فَأَسَرَّهَا يُوسُفُ فِى نَفْسِهِۦ وَلَمْ يُبْدِهَا لَهُمْ ۚ قَالَ أَنتُمْ شَرٌّۭ مَّكَانًۭا ۖ وَٱللَّهُ أَعْلَمُ بِمَا تَصِفُونَ ﴿٧٧﴾}\\
78.\  & \mytextarabic{قَالُوا۟ يَـٰٓأَيُّهَا ٱلْعَزِيزُ إِنَّ لَهُۥٓ أَبًۭا شَيْخًۭا كَبِيرًۭا فَخُذْ أَحَدَنَا مَكَانَهُۥٓ ۖ إِنَّا نَرَىٰكَ مِنَ ٱلْمُحْسِنِينَ ﴿٧٨﴾}\\
79.\  & \mytextarabic{قَالَ مَعَاذَ ٱللَّهِ أَن نَّأْخُذَ إِلَّا مَن وَجَدْنَا مَتَـٰعَنَا عِندَهُۥٓ إِنَّآ إِذًۭا لَّظَـٰلِمُونَ ﴿٧٩﴾}\\
80.\  & \mytextarabic{فَلَمَّا ٱسْتَيْـَٔسُوا۟ مِنْهُ خَلَصُوا۟ نَجِيًّۭا ۖ قَالَ كَبِيرُهُمْ أَلَمْ تَعْلَمُوٓا۟ أَنَّ أَبَاكُمْ قَدْ أَخَذَ عَلَيْكُم مَّوْثِقًۭا مِّنَ ٱللَّهِ وَمِن قَبْلُ مَا فَرَّطتُمْ فِى يُوسُفَ ۖ فَلَنْ أَبْرَحَ ٱلْأَرْضَ حَتَّىٰ يَأْذَنَ لِىٓ أَبِىٓ أَوْ يَحْكُمَ ٱللَّهُ لِى ۖ وَهُوَ خَيْرُ ٱلْحَـٰكِمِينَ ﴿٨٠﴾}\\
81.\  & \mytextarabic{ٱرْجِعُوٓا۟ إِلَىٰٓ أَبِيكُمْ فَقُولُوا۟ يَـٰٓأَبَانَآ إِنَّ ٱبْنَكَ سَرَقَ وَمَا شَهِدْنَآ إِلَّا بِمَا عَلِمْنَا وَمَا كُنَّا لِلْغَيْبِ حَـٰفِظِينَ ﴿٨١﴾}\\
82.\  & \mytextarabic{وَسْـَٔلِ ٱلْقَرْيَةَ ٱلَّتِى كُنَّا فِيهَا وَٱلْعِيرَ ٱلَّتِىٓ أَقْبَلْنَا فِيهَا ۖ وَإِنَّا لَصَـٰدِقُونَ ﴿٨٢﴾}\\
83.\  & \mytextarabic{قَالَ بَلْ سَوَّلَتْ لَكُمْ أَنفُسُكُمْ أَمْرًۭا ۖ فَصَبْرٌۭ جَمِيلٌ ۖ عَسَى ٱللَّهُ أَن يَأْتِيَنِى بِهِمْ جَمِيعًا ۚ إِنَّهُۥ هُوَ ٱلْعَلِيمُ ٱلْحَكِيمُ ﴿٨٣﴾}\\
84.\  & \mytextarabic{وَتَوَلَّىٰ عَنْهُمْ وَقَالَ يَـٰٓأَسَفَىٰ عَلَىٰ يُوسُفَ وَٱبْيَضَّتْ عَيْنَاهُ مِنَ ٱلْحُزْنِ فَهُوَ كَظِيمٌۭ ﴿٨٤﴾}\\
85.\  & \mytextarabic{قَالُوا۟ تَٱللَّهِ تَفْتَؤُا۟ تَذْكُرُ يُوسُفَ حَتَّىٰ تَكُونَ حَرَضًا أَوْ تَكُونَ مِنَ ٱلْهَـٰلِكِينَ ﴿٨٥﴾}\\
86.\  & \mytextarabic{قَالَ إِنَّمَآ أَشْكُوا۟ بَثِّى وَحُزْنِىٓ إِلَى ٱللَّهِ وَأَعْلَمُ مِنَ ٱللَّهِ مَا لَا تَعْلَمُونَ ﴿٨٦﴾}\\
87.\  & \mytextarabic{يَـٰبَنِىَّ ٱذْهَبُوا۟ فَتَحَسَّسُوا۟ مِن يُوسُفَ وَأَخِيهِ وَلَا تَا۟يْـَٔسُوا۟ مِن رَّوْحِ ٱللَّهِ ۖ إِنَّهُۥ لَا يَا۟يْـَٔسُ مِن رَّوْحِ ٱللَّهِ إِلَّا ٱلْقَوْمُ ٱلْكَـٰفِرُونَ ﴿٨٧﴾}\\
88.\  & \mytextarabic{فَلَمَّا دَخَلُوا۟ عَلَيْهِ قَالُوا۟ يَـٰٓأَيُّهَا ٱلْعَزِيزُ مَسَّنَا وَأَهْلَنَا ٱلضُّرُّ وَجِئْنَا بِبِضَٰعَةٍۢ مُّزْجَىٰةٍۢ فَأَوْفِ لَنَا ٱلْكَيْلَ وَتَصَدَّقْ عَلَيْنَآ ۖ إِنَّ ٱللَّهَ يَجْزِى ٱلْمُتَصَدِّقِينَ ﴿٨٨﴾}\\
89.\  & \mytextarabic{قَالَ هَلْ عَلِمْتُم مَّا فَعَلْتُم بِيُوسُفَ وَأَخِيهِ إِذْ أَنتُمْ جَٰهِلُونَ ﴿٨٩﴾}\\
90.\  & \mytextarabic{قَالُوٓا۟ أَءِنَّكَ لَأَنتَ يُوسُفُ ۖ قَالَ أَنَا۠ يُوسُفُ وَهَـٰذَآ أَخِى ۖ قَدْ مَنَّ ٱللَّهُ عَلَيْنَآ ۖ إِنَّهُۥ مَن يَتَّقِ وَيَصْبِرْ فَإِنَّ ٱللَّهَ لَا يُضِيعُ أَجْرَ ٱلْمُحْسِنِينَ ﴿٩٠﴾}\\
91.\  & \mytextarabic{قَالُوا۟ تَٱللَّهِ لَقَدْ ءَاثَرَكَ ٱللَّهُ عَلَيْنَا وَإِن كُنَّا لَخَـٰطِـِٔينَ ﴿٩١﴾}\\
92.\  & \mytextarabic{قَالَ لَا تَثْرِيبَ عَلَيْكُمُ ٱلْيَوْمَ ۖ يَغْفِرُ ٱللَّهُ لَكُمْ ۖ وَهُوَ أَرْحَمُ ٱلرَّٟحِمِينَ ﴿٩٢﴾}\\
93.\  & \mytextarabic{ٱذْهَبُوا۟ بِقَمِيصِى هَـٰذَا فَأَلْقُوهُ عَلَىٰ وَجْهِ أَبِى يَأْتِ بَصِيرًۭا وَأْتُونِى بِأَهْلِكُمْ أَجْمَعِينَ ﴿٩٣﴾}\\
94.\  & \mytextarabic{وَلَمَّا فَصَلَتِ ٱلْعِيرُ قَالَ أَبُوهُمْ إِنِّى لَأَجِدُ رِيحَ يُوسُفَ ۖ لَوْلَآ أَن تُفَنِّدُونِ ﴿٩٤﴾}\\
95.\  & \mytextarabic{قَالُوا۟ تَٱللَّهِ إِنَّكَ لَفِى ضَلَـٰلِكَ ٱلْقَدِيمِ ﴿٩٥﴾}\\
96.\  & \mytextarabic{فَلَمَّآ أَن جَآءَ ٱلْبَشِيرُ أَلْقَىٰهُ عَلَىٰ وَجْهِهِۦ فَٱرْتَدَّ بَصِيرًۭا ۖ قَالَ أَلَمْ أَقُل لَّكُمْ إِنِّىٓ أَعْلَمُ مِنَ ٱللَّهِ مَا لَا تَعْلَمُونَ ﴿٩٦﴾}\\
97.\  & \mytextarabic{قَالُوا۟ يَـٰٓأَبَانَا ٱسْتَغْفِرْ لَنَا ذُنُوبَنَآ إِنَّا كُنَّا خَـٰطِـِٔينَ ﴿٩٧﴾}\\
98.\  & \mytextarabic{قَالَ سَوْفَ أَسْتَغْفِرُ لَكُمْ رَبِّىٓ ۖ إِنَّهُۥ هُوَ ٱلْغَفُورُ ٱلرَّحِيمُ ﴿٩٨﴾}\\
99.\  & \mytextarabic{فَلَمَّا دَخَلُوا۟ عَلَىٰ يُوسُفَ ءَاوَىٰٓ إِلَيْهِ أَبَوَيْهِ وَقَالَ ٱدْخُلُوا۟ مِصْرَ إِن شَآءَ ٱللَّهُ ءَامِنِينَ ﴿٩٩﴾}\\
100.\  & \mytextarabic{وَرَفَعَ أَبَوَيْهِ عَلَى ٱلْعَرْشِ وَخَرُّوا۟ لَهُۥ سُجَّدًۭا ۖ وَقَالَ يَـٰٓأَبَتِ هَـٰذَا تَأْوِيلُ رُءْيَـٰىَ مِن قَبْلُ قَدْ جَعَلَهَا رَبِّى حَقًّۭا ۖ وَقَدْ أَحْسَنَ بِىٓ إِذْ أَخْرَجَنِى مِنَ ٱلسِّجْنِ وَجَآءَ بِكُم مِّنَ ٱلْبَدْوِ مِنۢ بَعْدِ أَن نَّزَغَ ٱلشَّيْطَٰنُ بَيْنِى وَبَيْنَ إِخْوَتِىٓ ۚ إِنَّ رَبِّى لَطِيفٌۭ لِّمَا يَشَآءُ ۚ إِنَّهُۥ هُوَ ٱلْعَلِيمُ ٱلْحَكِيمُ ﴿١٠٠﴾}\\
101.\  & \mytextarabic{۞ رَبِّ قَدْ ءَاتَيْتَنِى مِنَ ٱلْمُلْكِ وَعَلَّمْتَنِى مِن تَأْوِيلِ ٱلْأَحَادِيثِ ۚ فَاطِرَ ٱلسَّمَـٰوَٟتِ وَٱلْأَرْضِ أَنتَ وَلِىِّۦ فِى ٱلدُّنْيَا وَٱلْءَاخِرَةِ ۖ تَوَفَّنِى مُسْلِمًۭا وَأَلْحِقْنِى بِٱلصَّـٰلِحِينَ ﴿١٠١﴾}\\
102.\  & \mytextarabic{ذَٟلِكَ مِنْ أَنۢبَآءِ ٱلْغَيْبِ نُوحِيهِ إِلَيْكَ ۖ وَمَا كُنتَ لَدَيْهِمْ إِذْ أَجْمَعُوٓا۟ أَمْرَهُمْ وَهُمْ يَمْكُرُونَ ﴿١٠٢﴾}\\
103.\  & \mytextarabic{وَمَآ أَكْثَرُ ٱلنَّاسِ وَلَوْ حَرَصْتَ بِمُؤْمِنِينَ ﴿١٠٣﴾}\\
104.\  & \mytextarabic{وَمَا تَسْـَٔلُهُمْ عَلَيْهِ مِنْ أَجْرٍ ۚ إِنْ هُوَ إِلَّا ذِكْرٌۭ لِّلْعَـٰلَمِينَ ﴿١٠٤﴾}\\
105.\  & \mytextarabic{وَكَأَيِّن مِّنْ ءَايَةٍۢ فِى ٱلسَّمَـٰوَٟتِ وَٱلْأَرْضِ يَمُرُّونَ عَلَيْهَا وَهُمْ عَنْهَا مُعْرِضُونَ ﴿١٠٥﴾}\\
106.\  & \mytextarabic{وَمَا يُؤْمِنُ أَكْثَرُهُم بِٱللَّهِ إِلَّا وَهُم مُّشْرِكُونَ ﴿١٠٦﴾}\\
107.\  & \mytextarabic{أَفَأَمِنُوٓا۟ أَن تَأْتِيَهُمْ غَٰشِيَةٌۭ مِّنْ عَذَابِ ٱللَّهِ أَوْ تَأْتِيَهُمُ ٱلسَّاعَةُ بَغْتَةًۭ وَهُمْ لَا يَشْعُرُونَ ﴿١٠٧﴾}\\
108.\  & \mytextarabic{قُلْ هَـٰذِهِۦ سَبِيلِىٓ أَدْعُوٓا۟ إِلَى ٱللَّهِ ۚ عَلَىٰ بَصِيرَةٍ أَنَا۠ وَمَنِ ٱتَّبَعَنِى ۖ وَسُبْحَـٰنَ ٱللَّهِ وَمَآ أَنَا۠ مِنَ ٱلْمُشْرِكِينَ ﴿١٠٨﴾}\\
109.\  & \mytextarabic{وَمَآ أَرْسَلْنَا مِن قَبْلِكَ إِلَّا رِجَالًۭا نُّوحِىٓ إِلَيْهِم مِّنْ أَهْلِ ٱلْقُرَىٰٓ ۗ أَفَلَمْ يَسِيرُوا۟ فِى ٱلْأَرْضِ فَيَنظُرُوا۟ كَيْفَ كَانَ عَـٰقِبَةُ ٱلَّذِينَ مِن قَبْلِهِمْ ۗ وَلَدَارُ ٱلْءَاخِرَةِ خَيْرٌۭ لِّلَّذِينَ ٱتَّقَوْا۟ ۗ أَفَلَا تَعْقِلُونَ ﴿١٠٩﴾}\\
110.\  & \mytextarabic{حَتَّىٰٓ إِذَا ٱسْتَيْـَٔسَ ٱلرُّسُلُ وَظَنُّوٓا۟ أَنَّهُمْ قَدْ كُذِبُوا۟ جَآءَهُمْ نَصْرُنَا فَنُجِّىَ مَن نَّشَآءُ ۖ وَلَا يُرَدُّ بَأْسُنَا عَنِ ٱلْقَوْمِ ٱلْمُجْرِمِينَ ﴿١١٠﴾}\\
111.\  & \mytextarabic{لَقَدْ كَانَ فِى قَصَصِهِمْ عِبْرَةٌۭ لِّأُو۟لِى ٱلْأَلْبَٰبِ ۗ مَا كَانَ حَدِيثًۭا يُفْتَرَىٰ وَلَـٰكِن تَصْدِيقَ ٱلَّذِى بَيْنَ يَدَيْهِ وَتَفْصِيلَ كُلِّ شَىْءٍۢ وَهُدًۭى وَرَحْمَةًۭ لِّقَوْمٍۢ يُؤْمِنُونَ ﴿١١١﴾}\\
\end{longtable}
\clearpage