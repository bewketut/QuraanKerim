\begin{center}\section{ሱራቱ አልኢስራኣ -  \textarabic{سوره  الإسراء}}\end{center}
\begin{longtable}{%
  @{}
    p{.5\textwidth}
  @{~~~}
    p{.5\textwidth}
    @{}
}
ቢስሚላሂ አራህመኒ ራሂይም &  \mytextarabic{بِسْمِ ٱللَّهِ ٱلرَّحْمَـٰنِ ٱلرَّحِيمِ}\\
1.\  & \mytextarabic{ سُبْحَـٰنَ ٱلَّذِىٓ أَسْرَىٰ بِعَبْدِهِۦ لَيْلًۭا مِّنَ ٱلْمَسْجِدِ ٱلْحَرَامِ إِلَى ٱلْمَسْجِدِ ٱلْأَقْصَا ٱلَّذِى بَٰرَكْنَا حَوْلَهُۥ لِنُرِيَهُۥ مِنْ ءَايَـٰتِنَآ ۚ إِنَّهُۥ هُوَ ٱلسَّمِيعُ ٱلْبَصِيرُ ﴿١﴾}\\
2.\  & \mytextarabic{وَءَاتَيْنَا مُوسَى ٱلْكِتَـٰبَ وَجَعَلْنَـٰهُ هُدًۭى لِّبَنِىٓ إِسْرَٰٓءِيلَ أَلَّا تَتَّخِذُوا۟ مِن دُونِى وَكِيلًۭا ﴿٢﴾}\\
3.\  & \mytextarabic{ذُرِّيَّةَ مَنْ حَمَلْنَا مَعَ نُوحٍ ۚ إِنَّهُۥ كَانَ عَبْدًۭا شَكُورًۭا ﴿٣﴾}\\
4.\  & \mytextarabic{وَقَضَيْنَآ إِلَىٰ بَنِىٓ إِسْرَٰٓءِيلَ فِى ٱلْكِتَـٰبِ لَتُفْسِدُنَّ فِى ٱلْأَرْضِ مَرَّتَيْنِ وَلَتَعْلُنَّ عُلُوًّۭا كَبِيرًۭا ﴿٤﴾}\\
5.\  & \mytextarabic{فَإِذَا جَآءَ وَعْدُ أُولَىٰهُمَا بَعَثْنَا عَلَيْكُمْ عِبَادًۭا لَّنَآ أُو۟لِى بَأْسٍۢ شَدِيدٍۢ فَجَاسُوا۟ خِلَـٰلَ ٱلدِّيَارِ ۚ وَكَانَ وَعْدًۭا مَّفْعُولًۭا ﴿٥﴾}\\
6.\  & \mytextarabic{ثُمَّ رَدَدْنَا لَكُمُ ٱلْكَرَّةَ عَلَيْهِمْ وَأَمْدَدْنَـٰكُم بِأَمْوَٟلٍۢ وَبَنِينَ وَجَعَلْنَـٰكُمْ أَكْثَرَ نَفِيرًا ﴿٦﴾}\\
7.\  & \mytextarabic{إِنْ أَحْسَنتُمْ أَحْسَنتُمْ لِأَنفُسِكُمْ ۖ وَإِنْ أَسَأْتُمْ فَلَهَا ۚ فَإِذَا جَآءَ وَعْدُ ٱلْءَاخِرَةِ لِيَسُۥٓـُٔوا۟ وُجُوهَكُمْ وَلِيَدْخُلُوا۟ ٱلْمَسْجِدَ كَمَا دَخَلُوهُ أَوَّلَ مَرَّةٍۢ وَلِيُتَبِّرُوا۟ مَا عَلَوْا۟ تَتْبِيرًا ﴿٧﴾}\\
8.\  & \mytextarabic{عَسَىٰ رَبُّكُمْ أَن يَرْحَمَكُمْ ۚ وَإِنْ عُدتُّمْ عُدْنَا ۘ وَجَعَلْنَا جَهَنَّمَ لِلْكَـٰفِرِينَ حَصِيرًا ﴿٨﴾}\\
9.\  & \mytextarabic{إِنَّ هَـٰذَا ٱلْقُرْءَانَ يَهْدِى لِلَّتِى هِىَ أَقْوَمُ وَيُبَشِّرُ ٱلْمُؤْمِنِينَ ٱلَّذِينَ يَعْمَلُونَ ٱلصَّـٰلِحَـٰتِ أَنَّ لَهُمْ أَجْرًۭا كَبِيرًۭا ﴿٩﴾}\\
10.\  & \mytextarabic{وَأَنَّ ٱلَّذِينَ لَا يُؤْمِنُونَ بِٱلْءَاخِرَةِ أَعْتَدْنَا لَهُمْ عَذَابًا أَلِيمًۭا ﴿١٠﴾}\\
11.\  & \mytextarabic{وَيَدْعُ ٱلْإِنسَـٰنُ بِٱلشَّرِّ دُعَآءَهُۥ بِٱلْخَيْرِ ۖ وَكَانَ ٱلْإِنسَـٰنُ عَجُولًۭا ﴿١١﴾}\\
12.\  & \mytextarabic{وَجَعَلْنَا ٱلَّيْلَ وَٱلنَّهَارَ ءَايَتَيْنِ ۖ فَمَحَوْنَآ ءَايَةَ ٱلَّيْلِ وَجَعَلْنَآ ءَايَةَ ٱلنَّهَارِ مُبْصِرَةًۭ لِّتَبْتَغُوا۟ فَضْلًۭا مِّن رَّبِّكُمْ وَلِتَعْلَمُوا۟ عَدَدَ ٱلسِّنِينَ وَٱلْحِسَابَ ۚ وَكُلَّ شَىْءٍۢ فَصَّلْنَـٰهُ تَفْصِيلًۭا ﴿١٢﴾}\\
13.\  & \mytextarabic{وَكُلَّ إِنسَـٰنٍ أَلْزَمْنَـٰهُ طَٰٓئِرَهُۥ فِى عُنُقِهِۦ ۖ وَنُخْرِجُ لَهُۥ يَوْمَ ٱلْقِيَـٰمَةِ كِتَـٰبًۭا يَلْقَىٰهُ مَنشُورًا ﴿١٣﴾}\\
14.\  & \mytextarabic{ٱقْرَأْ كِتَـٰبَكَ كَفَىٰ بِنَفْسِكَ ٱلْيَوْمَ عَلَيْكَ حَسِيبًۭا ﴿١٤﴾}\\
15.\  & \mytextarabic{مَّنِ ٱهْتَدَىٰ فَإِنَّمَا يَهْتَدِى لِنَفْسِهِۦ ۖ وَمَن ضَلَّ فَإِنَّمَا يَضِلُّ عَلَيْهَا ۚ وَلَا تَزِرُ وَازِرَةٌۭ وِزْرَ أُخْرَىٰ ۗ وَمَا كُنَّا مُعَذِّبِينَ حَتَّىٰ نَبْعَثَ رَسُولًۭا ﴿١٥﴾}\\
16.\  & \mytextarabic{وَإِذَآ أَرَدْنَآ أَن نُّهْلِكَ قَرْيَةً أَمَرْنَا مُتْرَفِيهَا فَفَسَقُوا۟ فِيهَا فَحَقَّ عَلَيْهَا ٱلْقَوْلُ فَدَمَّرْنَـٰهَا تَدْمِيرًۭا ﴿١٦﴾}\\
17.\  & \mytextarabic{وَكَمْ أَهْلَكْنَا مِنَ ٱلْقُرُونِ مِنۢ بَعْدِ نُوحٍۢ ۗ وَكَفَىٰ بِرَبِّكَ بِذُنُوبِ عِبَادِهِۦ خَبِيرًۢا بَصِيرًۭا ﴿١٧﴾}\\
18.\  & \mytextarabic{مَّن كَانَ يُرِيدُ ٱلْعَاجِلَةَ عَجَّلْنَا لَهُۥ فِيهَا مَا نَشَآءُ لِمَن نُّرِيدُ ثُمَّ جَعَلْنَا لَهُۥ جَهَنَّمَ يَصْلَىٰهَا مَذْمُومًۭا مَّدْحُورًۭا ﴿١٨﴾}\\
19.\  & \mytextarabic{وَمَنْ أَرَادَ ٱلْءَاخِرَةَ وَسَعَىٰ لَهَا سَعْيَهَا وَهُوَ مُؤْمِنٌۭ فَأُو۟لَـٰٓئِكَ كَانَ سَعْيُهُم مَّشْكُورًۭا ﴿١٩﴾}\\
20.\  & \mytextarabic{كُلًّۭا نُّمِدُّ هَـٰٓؤُلَآءِ وَهَـٰٓؤُلَآءِ مِنْ عَطَآءِ رَبِّكَ ۚ وَمَا كَانَ عَطَآءُ رَبِّكَ مَحْظُورًا ﴿٢٠﴾}\\
21.\  & \mytextarabic{ٱنظُرْ كَيْفَ فَضَّلْنَا بَعْضَهُمْ عَلَىٰ بَعْضٍۢ ۚ وَلَلْءَاخِرَةُ أَكْبَرُ دَرَجَٰتٍۢ وَأَكْبَرُ تَفْضِيلًۭا ﴿٢١﴾}\\
22.\  & \mytextarabic{لَّا تَجْعَلْ مَعَ ٱللَّهِ إِلَـٰهًا ءَاخَرَ فَتَقْعُدَ مَذْمُومًۭا مَّخْذُولًۭا ﴿٢٢﴾}\\
23.\  & \mytextarabic{۞ وَقَضَىٰ رَبُّكَ أَلَّا تَعْبُدُوٓا۟ إِلَّآ إِيَّاهُ وَبِٱلْوَٟلِدَيْنِ إِحْسَـٰنًا ۚ إِمَّا يَبْلُغَنَّ عِندَكَ ٱلْكِبَرَ أَحَدُهُمَآ أَوْ كِلَاهُمَا فَلَا تَقُل لَّهُمَآ أُفٍّۢ وَلَا تَنْهَرْهُمَا وَقُل لَّهُمَا قَوْلًۭا كَرِيمًۭا ﴿٢٣﴾}\\
24.\  & \mytextarabic{وَٱخْفِضْ لَهُمَا جَنَاحَ ٱلذُّلِّ مِنَ ٱلرَّحْمَةِ وَقُل رَّبِّ ٱرْحَمْهُمَا كَمَا رَبَّيَانِى صَغِيرًۭا ﴿٢٤﴾}\\
25.\  & \mytextarabic{رَّبُّكُمْ أَعْلَمُ بِمَا فِى نُفُوسِكُمْ ۚ إِن تَكُونُوا۟ صَـٰلِحِينَ فَإِنَّهُۥ كَانَ لِلْأَوَّٰبِينَ غَفُورًۭا ﴿٢٥﴾}\\
26.\  & \mytextarabic{وَءَاتِ ذَا ٱلْقُرْبَىٰ حَقَّهُۥ وَٱلْمِسْكِينَ وَٱبْنَ ٱلسَّبِيلِ وَلَا تُبَذِّرْ تَبْذِيرًا ﴿٢٦﴾}\\
27.\  & \mytextarabic{إِنَّ ٱلْمُبَذِّرِينَ كَانُوٓا۟ إِخْوَٟنَ ٱلشَّيَـٰطِينِ ۖ وَكَانَ ٱلشَّيْطَٰنُ لِرَبِّهِۦ كَفُورًۭا ﴿٢٧﴾}\\
28.\  & \mytextarabic{وَإِمَّا تُعْرِضَنَّ عَنْهُمُ ٱبْتِغَآءَ رَحْمَةٍۢ مِّن رَّبِّكَ تَرْجُوهَا فَقُل لَّهُمْ قَوْلًۭا مَّيْسُورًۭا ﴿٢٨﴾}\\
29.\  & \mytextarabic{وَلَا تَجْعَلْ يَدَكَ مَغْلُولَةً إِلَىٰ عُنُقِكَ وَلَا تَبْسُطْهَا كُلَّ ٱلْبَسْطِ فَتَقْعُدَ مَلُومًۭا مَّحْسُورًا ﴿٢٩﴾}\\
30.\  & \mytextarabic{إِنَّ رَبَّكَ يَبْسُطُ ٱلرِّزْقَ لِمَن يَشَآءُ وَيَقْدِرُ ۚ إِنَّهُۥ كَانَ بِعِبَادِهِۦ خَبِيرًۢا بَصِيرًۭا ﴿٣٠﴾}\\
31.\  & \mytextarabic{وَلَا تَقْتُلُوٓا۟ أَوْلَـٰدَكُمْ خَشْيَةَ إِمْلَـٰقٍۢ ۖ نَّحْنُ نَرْزُقُهُمْ وَإِيَّاكُمْ ۚ إِنَّ قَتْلَهُمْ كَانَ خِطْـًۭٔا كَبِيرًۭا ﴿٣١﴾}\\
32.\  & \mytextarabic{وَلَا تَقْرَبُوا۟ ٱلزِّنَىٰٓ ۖ إِنَّهُۥ كَانَ فَـٰحِشَةًۭ وَسَآءَ سَبِيلًۭا ﴿٣٢﴾}\\
33.\  & \mytextarabic{وَلَا تَقْتُلُوا۟ ٱلنَّفْسَ ٱلَّتِى حَرَّمَ ٱللَّهُ إِلَّا بِٱلْحَقِّ ۗ وَمَن قُتِلَ مَظْلُومًۭا فَقَدْ جَعَلْنَا لِوَلِيِّهِۦ سُلْطَٰنًۭا فَلَا يُسْرِف فِّى ٱلْقَتْلِ ۖ إِنَّهُۥ كَانَ مَنصُورًۭا ﴿٣٣﴾}\\
34.\  & \mytextarabic{وَلَا تَقْرَبُوا۟ مَالَ ٱلْيَتِيمِ إِلَّا بِٱلَّتِى هِىَ أَحْسَنُ حَتَّىٰ يَبْلُغَ أَشُدَّهُۥ ۚ وَأَوْفُوا۟ بِٱلْعَهْدِ ۖ إِنَّ ٱلْعَهْدَ كَانَ مَسْـُٔولًۭا ﴿٣٤﴾}\\
35.\  & \mytextarabic{وَأَوْفُوا۟ ٱلْكَيْلَ إِذَا كِلْتُمْ وَزِنُوا۟ بِٱلْقِسْطَاسِ ٱلْمُسْتَقِيمِ ۚ ذَٟلِكَ خَيْرٌۭ وَأَحْسَنُ تَأْوِيلًۭا ﴿٣٥﴾}\\
36.\  & \mytextarabic{وَلَا تَقْفُ مَا لَيْسَ لَكَ بِهِۦ عِلْمٌ ۚ إِنَّ ٱلسَّمْعَ وَٱلْبَصَرَ وَٱلْفُؤَادَ كُلُّ أُو۟لَـٰٓئِكَ كَانَ عَنْهُ مَسْـُٔولًۭا ﴿٣٦﴾}\\
37.\  & \mytextarabic{وَلَا تَمْشِ فِى ٱلْأَرْضِ مَرَحًا ۖ إِنَّكَ لَن تَخْرِقَ ٱلْأَرْضَ وَلَن تَبْلُغَ ٱلْجِبَالَ طُولًۭا ﴿٣٧﴾}\\
38.\  & \mytextarabic{كُلُّ ذَٟلِكَ كَانَ سَيِّئُهُۥ عِندَ رَبِّكَ مَكْرُوهًۭا ﴿٣٨﴾}\\
39.\  & \mytextarabic{ذَٟلِكَ مِمَّآ أَوْحَىٰٓ إِلَيْكَ رَبُّكَ مِنَ ٱلْحِكْمَةِ ۗ وَلَا تَجْعَلْ مَعَ ٱللَّهِ إِلَـٰهًا ءَاخَرَ فَتُلْقَىٰ فِى جَهَنَّمَ مَلُومًۭا مَّدْحُورًا ﴿٣٩﴾}\\
40.\  & \mytextarabic{أَفَأَصْفَىٰكُمْ رَبُّكُم بِٱلْبَنِينَ وَٱتَّخَذَ مِنَ ٱلْمَلَـٰٓئِكَةِ إِنَـٰثًا ۚ إِنَّكُمْ لَتَقُولُونَ قَوْلًا عَظِيمًۭا ﴿٤٠﴾}\\
41.\  & \mytextarabic{وَلَقَدْ صَرَّفْنَا فِى هَـٰذَا ٱلْقُرْءَانِ لِيَذَّكَّرُوا۟ وَمَا يَزِيدُهُمْ إِلَّا نُفُورًۭا ﴿٤١﴾}\\
42.\  & \mytextarabic{قُل لَّوْ كَانَ مَعَهُۥٓ ءَالِهَةٌۭ كَمَا يَقُولُونَ إِذًۭا لَّٱبْتَغَوْا۟ إِلَىٰ ذِى ٱلْعَرْشِ سَبِيلًۭا ﴿٤٢﴾}\\
43.\  & \mytextarabic{سُبْحَـٰنَهُۥ وَتَعَـٰلَىٰ عَمَّا يَقُولُونَ عُلُوًّۭا كَبِيرًۭا ﴿٤٣﴾}\\
44.\  & \mytextarabic{تُسَبِّحُ لَهُ ٱلسَّمَـٰوَٟتُ ٱلسَّبْعُ وَٱلْأَرْضُ وَمَن فِيهِنَّ ۚ وَإِن مِّن شَىْءٍ إِلَّا يُسَبِّحُ بِحَمْدِهِۦ وَلَـٰكِن لَّا تَفْقَهُونَ تَسْبِيحَهُمْ ۗ إِنَّهُۥ كَانَ حَلِيمًا غَفُورًۭا ﴿٤٤﴾}\\
45.\  & \mytextarabic{وَإِذَا قَرَأْتَ ٱلْقُرْءَانَ جَعَلْنَا بَيْنَكَ وَبَيْنَ ٱلَّذِينَ لَا يُؤْمِنُونَ بِٱلْءَاخِرَةِ حِجَابًۭا مَّسْتُورًۭا ﴿٤٥﴾}\\
46.\  & \mytextarabic{وَجَعَلْنَا عَلَىٰ قُلُوبِهِمْ أَكِنَّةً أَن يَفْقَهُوهُ وَفِىٓ ءَاذَانِهِمْ وَقْرًۭا ۚ وَإِذَا ذَكَرْتَ رَبَّكَ فِى ٱلْقُرْءَانِ وَحْدَهُۥ وَلَّوْا۟ عَلَىٰٓ أَدْبَٰرِهِمْ نُفُورًۭا ﴿٤٦﴾}\\
47.\  & \mytextarabic{نَّحْنُ أَعْلَمُ بِمَا يَسْتَمِعُونَ بِهِۦٓ إِذْ يَسْتَمِعُونَ إِلَيْكَ وَإِذْ هُمْ نَجْوَىٰٓ إِذْ يَقُولُ ٱلظَّـٰلِمُونَ إِن تَتَّبِعُونَ إِلَّا رَجُلًۭا مَّسْحُورًا ﴿٤٧﴾}\\
48.\  & \mytextarabic{ٱنظُرْ كَيْفَ ضَرَبُوا۟ لَكَ ٱلْأَمْثَالَ فَضَلُّوا۟ فَلَا يَسْتَطِيعُونَ سَبِيلًۭا ﴿٤٨﴾}\\
49.\  & \mytextarabic{وَقَالُوٓا۟ أَءِذَا كُنَّا عِظَـٰمًۭا وَرُفَـٰتًا أَءِنَّا لَمَبْعُوثُونَ خَلْقًۭا جَدِيدًۭا ﴿٤٩﴾}\\
50.\  & \mytextarabic{۞ قُلْ كُونُوا۟ حِجَارَةً أَوْ حَدِيدًا ﴿٥٠﴾}\\
51.\  & \mytextarabic{أَوْ خَلْقًۭا مِّمَّا يَكْبُرُ فِى صُدُورِكُمْ ۚ فَسَيَقُولُونَ مَن يُعِيدُنَا ۖ قُلِ ٱلَّذِى فَطَرَكُمْ أَوَّلَ مَرَّةٍۢ ۚ فَسَيُنْغِضُونَ إِلَيْكَ رُءُوسَهُمْ وَيَقُولُونَ مَتَىٰ هُوَ ۖ قُلْ عَسَىٰٓ أَن يَكُونَ قَرِيبًۭا ﴿٥١﴾}\\
52.\  & \mytextarabic{يَوْمَ يَدْعُوكُمْ فَتَسْتَجِيبُونَ بِحَمْدِهِۦ وَتَظُنُّونَ إِن لَّبِثْتُمْ إِلَّا قَلِيلًۭا ﴿٥٢﴾}\\
53.\  & \mytextarabic{وَقُل لِّعِبَادِى يَقُولُوا۟ ٱلَّتِى هِىَ أَحْسَنُ ۚ إِنَّ ٱلشَّيْطَٰنَ يَنزَغُ بَيْنَهُمْ ۚ إِنَّ ٱلشَّيْطَٰنَ كَانَ لِلْإِنسَـٰنِ عَدُوًّۭا مُّبِينًۭا ﴿٥٣﴾}\\
54.\  & \mytextarabic{رَّبُّكُمْ أَعْلَمُ بِكُمْ ۖ إِن يَشَأْ يَرْحَمْكُمْ أَوْ إِن يَشَأْ يُعَذِّبْكُمْ ۚ وَمَآ أَرْسَلْنَـٰكَ عَلَيْهِمْ وَكِيلًۭا ﴿٥٤﴾}\\
55.\  & \mytextarabic{وَرَبُّكَ أَعْلَمُ بِمَن فِى ٱلسَّمَـٰوَٟتِ وَٱلْأَرْضِ ۗ وَلَقَدْ فَضَّلْنَا بَعْضَ ٱلنَّبِيِّۦنَ عَلَىٰ بَعْضٍۢ ۖ وَءَاتَيْنَا دَاوُۥدَ زَبُورًۭا ﴿٥٥﴾}\\
56.\  & \mytextarabic{قُلِ ٱدْعُوا۟ ٱلَّذِينَ زَعَمْتُم مِّن دُونِهِۦ فَلَا يَمْلِكُونَ كَشْفَ ٱلضُّرِّ عَنكُمْ وَلَا تَحْوِيلًا ﴿٥٦﴾}\\
57.\  & \mytextarabic{أُو۟لَـٰٓئِكَ ٱلَّذِينَ يَدْعُونَ يَبْتَغُونَ إِلَىٰ رَبِّهِمُ ٱلْوَسِيلَةَ أَيُّهُمْ أَقْرَبُ وَيَرْجُونَ رَحْمَتَهُۥ وَيَخَافُونَ عَذَابَهُۥٓ ۚ إِنَّ عَذَابَ رَبِّكَ كَانَ مَحْذُورًۭا ﴿٥٧﴾}\\
58.\  & \mytextarabic{وَإِن مِّن قَرْيَةٍ إِلَّا نَحْنُ مُهْلِكُوهَا قَبْلَ يَوْمِ ٱلْقِيَـٰمَةِ أَوْ مُعَذِّبُوهَا عَذَابًۭا شَدِيدًۭا ۚ كَانَ ذَٟلِكَ فِى ٱلْكِتَـٰبِ مَسْطُورًۭا ﴿٥٨﴾}\\
59.\  & \mytextarabic{وَمَا مَنَعَنَآ أَن نُّرْسِلَ بِٱلْءَايَـٰتِ إِلَّآ أَن كَذَّبَ بِهَا ٱلْأَوَّلُونَ ۚ وَءَاتَيْنَا ثَمُودَ ٱلنَّاقَةَ مُبْصِرَةًۭ فَظَلَمُوا۟ بِهَا ۚ وَمَا نُرْسِلُ بِٱلْءَايَـٰتِ إِلَّا تَخْوِيفًۭا ﴿٥٩﴾}\\
60.\  & \mytextarabic{وَإِذْ قُلْنَا لَكَ إِنَّ رَبَّكَ أَحَاطَ بِٱلنَّاسِ ۚ وَمَا جَعَلْنَا ٱلرُّءْيَا ٱلَّتِىٓ أَرَيْنَـٰكَ إِلَّا فِتْنَةًۭ لِّلنَّاسِ وَٱلشَّجَرَةَ ٱلْمَلْعُونَةَ فِى ٱلْقُرْءَانِ ۚ وَنُخَوِّفُهُمْ فَمَا يَزِيدُهُمْ إِلَّا طُغْيَـٰنًۭا كَبِيرًۭا ﴿٦٠﴾}\\
61.\  & \mytextarabic{وَإِذْ قُلْنَا لِلْمَلَـٰٓئِكَةِ ٱسْجُدُوا۟ لِءَادَمَ فَسَجَدُوٓا۟ إِلَّآ إِبْلِيسَ قَالَ ءَأَسْجُدُ لِمَنْ خَلَقْتَ طِينًۭا ﴿٦١﴾}\\
62.\  & \mytextarabic{قَالَ أَرَءَيْتَكَ هَـٰذَا ٱلَّذِى كَرَّمْتَ عَلَىَّ لَئِنْ أَخَّرْتَنِ إِلَىٰ يَوْمِ ٱلْقِيَـٰمَةِ لَأَحْتَنِكَنَّ ذُرِّيَّتَهُۥٓ إِلَّا قَلِيلًۭا ﴿٦٢﴾}\\
63.\  & \mytextarabic{قَالَ ٱذْهَبْ فَمَن تَبِعَكَ مِنْهُمْ فَإِنَّ جَهَنَّمَ جَزَآؤُكُمْ جَزَآءًۭ مَّوْفُورًۭا ﴿٦٣﴾}\\
64.\  & \mytextarabic{وَٱسْتَفْزِزْ مَنِ ٱسْتَطَعْتَ مِنْهُم بِصَوْتِكَ وَأَجْلِبْ عَلَيْهِم بِخَيْلِكَ وَرَجِلِكَ وَشَارِكْهُمْ فِى ٱلْأَمْوَٟلِ وَٱلْأَوْلَـٰدِ وَعِدْهُمْ ۚ وَمَا يَعِدُهُمُ ٱلشَّيْطَٰنُ إِلَّا غُرُورًا ﴿٦٤﴾}\\
65.\  & \mytextarabic{إِنَّ عِبَادِى لَيْسَ لَكَ عَلَيْهِمْ سُلْطَٰنٌۭ ۚ وَكَفَىٰ بِرَبِّكَ وَكِيلًۭا ﴿٦٥﴾}\\
66.\  & \mytextarabic{رَّبُّكُمُ ٱلَّذِى يُزْجِى لَكُمُ ٱلْفُلْكَ فِى ٱلْبَحْرِ لِتَبْتَغُوا۟ مِن فَضْلِهِۦٓ ۚ إِنَّهُۥ كَانَ بِكُمْ رَحِيمًۭا ﴿٦٦﴾}\\
67.\  & \mytextarabic{وَإِذَا مَسَّكُمُ ٱلضُّرُّ فِى ٱلْبَحْرِ ضَلَّ مَن تَدْعُونَ إِلَّآ إِيَّاهُ ۖ فَلَمَّا نَجَّىٰكُمْ إِلَى ٱلْبَرِّ أَعْرَضْتُمْ ۚ وَكَانَ ٱلْإِنسَـٰنُ كَفُورًا ﴿٦٧﴾}\\
68.\  & \mytextarabic{أَفَأَمِنتُمْ أَن يَخْسِفَ بِكُمْ جَانِبَ ٱلْبَرِّ أَوْ يُرْسِلَ عَلَيْكُمْ حَاصِبًۭا ثُمَّ لَا تَجِدُوا۟ لَكُمْ وَكِيلًا ﴿٦٨﴾}\\
69.\  & \mytextarabic{أَمْ أَمِنتُمْ أَن يُعِيدَكُمْ فِيهِ تَارَةً أُخْرَىٰ فَيُرْسِلَ عَلَيْكُمْ قَاصِفًۭا مِّنَ ٱلرِّيحِ فَيُغْرِقَكُم بِمَا كَفَرْتُمْ ۙ ثُمَّ لَا تَجِدُوا۟ لَكُمْ عَلَيْنَا بِهِۦ تَبِيعًۭا ﴿٦٩﴾}\\
70.\  & \mytextarabic{۞ وَلَقَدْ كَرَّمْنَا بَنِىٓ ءَادَمَ وَحَمَلْنَـٰهُمْ فِى ٱلْبَرِّ وَٱلْبَحْرِ وَرَزَقْنَـٰهُم مِّنَ ٱلطَّيِّبَٰتِ وَفَضَّلْنَـٰهُمْ عَلَىٰ كَثِيرٍۢ مِّمَّنْ خَلَقْنَا تَفْضِيلًۭا ﴿٧٠﴾}\\
71.\  & \mytextarabic{يَوْمَ نَدْعُوا۟ كُلَّ أُنَاسٍۭ بِإِمَـٰمِهِمْ ۖ فَمَنْ أُوتِىَ كِتَـٰبَهُۥ بِيَمِينِهِۦ فَأُو۟لَـٰٓئِكَ يَقْرَءُونَ كِتَـٰبَهُمْ وَلَا يُظْلَمُونَ فَتِيلًۭا ﴿٧١﴾}\\
72.\  & \mytextarabic{وَمَن كَانَ فِى هَـٰذِهِۦٓ أَعْمَىٰ فَهُوَ فِى ٱلْءَاخِرَةِ أَعْمَىٰ وَأَضَلُّ سَبِيلًۭا ﴿٧٢﴾}\\
73.\  & \mytextarabic{وَإِن كَادُوا۟ لَيَفْتِنُونَكَ عَنِ ٱلَّذِىٓ أَوْحَيْنَآ إِلَيْكَ لِتَفْتَرِىَ عَلَيْنَا غَيْرَهُۥ ۖ وَإِذًۭا لَّٱتَّخَذُوكَ خَلِيلًۭا ﴿٧٣﴾}\\
74.\  & \mytextarabic{وَلَوْلَآ أَن ثَبَّتْنَـٰكَ لَقَدْ كِدتَّ تَرْكَنُ إِلَيْهِمْ شَيْـًۭٔا قَلِيلًا ﴿٧٤﴾}\\
75.\  & \mytextarabic{إِذًۭا لَّأَذَقْنَـٰكَ ضِعْفَ ٱلْحَيَوٰةِ وَضِعْفَ ٱلْمَمَاتِ ثُمَّ لَا تَجِدُ لَكَ عَلَيْنَا نَصِيرًۭا ﴿٧٥﴾}\\
76.\  & \mytextarabic{وَإِن كَادُوا۟ لَيَسْتَفِزُّونَكَ مِنَ ٱلْأَرْضِ لِيُخْرِجُوكَ مِنْهَا ۖ وَإِذًۭا لَّا يَلْبَثُونَ خِلَـٰفَكَ إِلَّا قَلِيلًۭا ﴿٧٦﴾}\\
77.\  & \mytextarabic{سُنَّةَ مَن قَدْ أَرْسَلْنَا قَبْلَكَ مِن رُّسُلِنَا ۖ وَلَا تَجِدُ لِسُنَّتِنَا تَحْوِيلًا ﴿٧٧﴾}\\
78.\  & \mytextarabic{أَقِمِ ٱلصَّلَوٰةَ لِدُلُوكِ ٱلشَّمْسِ إِلَىٰ غَسَقِ ٱلَّيْلِ وَقُرْءَانَ ٱلْفَجْرِ ۖ إِنَّ قُرْءَانَ ٱلْفَجْرِ كَانَ مَشْهُودًۭا ﴿٧٨﴾}\\
79.\  & \mytextarabic{وَمِنَ ٱلَّيْلِ فَتَهَجَّدْ بِهِۦ نَافِلَةًۭ لَّكَ عَسَىٰٓ أَن يَبْعَثَكَ رَبُّكَ مَقَامًۭا مَّحْمُودًۭا ﴿٧٩﴾}\\
80.\  & \mytextarabic{وَقُل رَّبِّ أَدْخِلْنِى مُدْخَلَ صِدْقٍۢ وَأَخْرِجْنِى مُخْرَجَ صِدْقٍۢ وَٱجْعَل لِّى مِن لَّدُنكَ سُلْطَٰنًۭا نَّصِيرًۭا ﴿٨٠﴾}\\
81.\  & \mytextarabic{وَقُلْ جَآءَ ٱلْحَقُّ وَزَهَقَ ٱلْبَٰطِلُ ۚ إِنَّ ٱلْبَٰطِلَ كَانَ زَهُوقًۭا ﴿٨١﴾}\\
82.\  & \mytextarabic{وَنُنَزِّلُ مِنَ ٱلْقُرْءَانِ مَا هُوَ شِفَآءٌۭ وَرَحْمَةٌۭ لِّلْمُؤْمِنِينَ ۙ وَلَا يَزِيدُ ٱلظَّـٰلِمِينَ إِلَّا خَسَارًۭا ﴿٨٢﴾}\\
83.\  & \mytextarabic{وَإِذَآ أَنْعَمْنَا عَلَى ٱلْإِنسَـٰنِ أَعْرَضَ وَنَـَٔا بِجَانِبِهِۦ ۖ وَإِذَا مَسَّهُ ٱلشَّرُّ كَانَ يَـُٔوسًۭا ﴿٨٣﴾}\\
84.\  & \mytextarabic{قُلْ كُلٌّۭ يَعْمَلُ عَلَىٰ شَاكِلَتِهِۦ فَرَبُّكُمْ أَعْلَمُ بِمَنْ هُوَ أَهْدَىٰ سَبِيلًۭا ﴿٨٤﴾}\\
85.\  & \mytextarabic{وَيَسْـَٔلُونَكَ عَنِ ٱلرُّوحِ ۖ قُلِ ٱلرُّوحُ مِنْ أَمْرِ رَبِّى وَمَآ أُوتِيتُم مِّنَ ٱلْعِلْمِ إِلَّا قَلِيلًۭا ﴿٨٥﴾}\\
86.\  & \mytextarabic{وَلَئِن شِئْنَا لَنَذْهَبَنَّ بِٱلَّذِىٓ أَوْحَيْنَآ إِلَيْكَ ثُمَّ لَا تَجِدُ لَكَ بِهِۦ عَلَيْنَا وَكِيلًا ﴿٨٦﴾}\\
87.\  & \mytextarabic{إِلَّا رَحْمَةًۭ مِّن رَّبِّكَ ۚ إِنَّ فَضْلَهُۥ كَانَ عَلَيْكَ كَبِيرًۭا ﴿٨٧﴾}\\
88.\  & \mytextarabic{قُل لَّئِنِ ٱجْتَمَعَتِ ٱلْإِنسُ وَٱلْجِنُّ عَلَىٰٓ أَن يَأْتُوا۟ بِمِثْلِ هَـٰذَا ٱلْقُرْءَانِ لَا يَأْتُونَ بِمِثْلِهِۦ وَلَوْ كَانَ بَعْضُهُمْ لِبَعْضٍۢ ظَهِيرًۭا ﴿٨٨﴾}\\
89.\  & \mytextarabic{وَلَقَدْ صَرَّفْنَا لِلنَّاسِ فِى هَـٰذَا ٱلْقُرْءَانِ مِن كُلِّ مَثَلٍۢ فَأَبَىٰٓ أَكْثَرُ ٱلنَّاسِ إِلَّا كُفُورًۭا ﴿٨٩﴾}\\
90.\  & \mytextarabic{وَقَالُوا۟ لَن نُّؤْمِنَ لَكَ حَتَّىٰ تَفْجُرَ لَنَا مِنَ ٱلْأَرْضِ يَنۢبُوعًا ﴿٩٠﴾}\\
91.\  & \mytextarabic{أَوْ تَكُونَ لَكَ جَنَّةٌۭ مِّن نَّخِيلٍۢ وَعِنَبٍۢ فَتُفَجِّرَ ٱلْأَنْهَـٰرَ خِلَـٰلَهَا تَفْجِيرًا ﴿٩١﴾}\\
92.\  & \mytextarabic{أَوْ تُسْقِطَ ٱلسَّمَآءَ كَمَا زَعَمْتَ عَلَيْنَا كِسَفًا أَوْ تَأْتِىَ بِٱللَّهِ وَٱلْمَلَـٰٓئِكَةِ قَبِيلًا ﴿٩٢﴾}\\
93.\  & \mytextarabic{أَوْ يَكُونَ لَكَ بَيْتٌۭ مِّن زُخْرُفٍ أَوْ تَرْقَىٰ فِى ٱلسَّمَآءِ وَلَن نُّؤْمِنَ لِرُقِيِّكَ حَتَّىٰ تُنَزِّلَ عَلَيْنَا كِتَـٰبًۭا نَّقْرَؤُهُۥ ۗ قُلْ سُبْحَانَ رَبِّى هَلْ كُنتُ إِلَّا بَشَرًۭا رَّسُولًۭا ﴿٩٣﴾}\\
94.\  & \mytextarabic{وَمَا مَنَعَ ٱلنَّاسَ أَن يُؤْمِنُوٓا۟ إِذْ جَآءَهُمُ ٱلْهُدَىٰٓ إِلَّآ أَن قَالُوٓا۟ أَبَعَثَ ٱللَّهُ بَشَرًۭا رَّسُولًۭا ﴿٩٤﴾}\\
95.\  & \mytextarabic{قُل لَّوْ كَانَ فِى ٱلْأَرْضِ مَلَـٰٓئِكَةٌۭ يَمْشُونَ مُطْمَئِنِّينَ لَنَزَّلْنَا عَلَيْهِم مِّنَ ٱلسَّمَآءِ مَلَكًۭا رَّسُولًۭا ﴿٩٥﴾}\\
96.\  & \mytextarabic{قُلْ كَفَىٰ بِٱللَّهِ شَهِيدًۢا بَيْنِى وَبَيْنَكُمْ ۚ إِنَّهُۥ كَانَ بِعِبَادِهِۦ خَبِيرًۢا بَصِيرًۭا ﴿٩٦﴾}\\
97.\  & \mytextarabic{وَمَن يَهْدِ ٱللَّهُ فَهُوَ ٱلْمُهْتَدِ ۖ وَمَن يُضْلِلْ فَلَن تَجِدَ لَهُمْ أَوْلِيَآءَ مِن دُونِهِۦ ۖ وَنَحْشُرُهُمْ يَوْمَ ٱلْقِيَـٰمَةِ عَلَىٰ وُجُوهِهِمْ عُمْيًۭا وَبُكْمًۭا وَصُمًّۭا ۖ مَّأْوَىٰهُمْ جَهَنَّمُ ۖ كُلَّمَا خَبَتْ زِدْنَـٰهُمْ سَعِيرًۭا ﴿٩٧﴾}\\
98.\  & \mytextarabic{ذَٟلِكَ جَزَآؤُهُم بِأَنَّهُمْ كَفَرُوا۟ بِـَٔايَـٰتِنَا وَقَالُوٓا۟ أَءِذَا كُنَّا عِظَـٰمًۭا وَرُفَـٰتًا أَءِنَّا لَمَبْعُوثُونَ خَلْقًۭا جَدِيدًا ﴿٩٨﴾}\\
99.\  & \mytextarabic{۞ أَوَلَمْ يَرَوْا۟ أَنَّ ٱللَّهَ ٱلَّذِى خَلَقَ ٱلسَّمَـٰوَٟتِ وَٱلْأَرْضَ قَادِرٌ عَلَىٰٓ أَن يَخْلُقَ مِثْلَهُمْ وَجَعَلَ لَهُمْ أَجَلًۭا لَّا رَيْبَ فِيهِ فَأَبَى ٱلظَّـٰلِمُونَ إِلَّا كُفُورًۭا ﴿٩٩﴾}\\
100.\  & \mytextarabic{قُل لَّوْ أَنتُمْ تَمْلِكُونَ خَزَآئِنَ رَحْمَةِ رَبِّىٓ إِذًۭا لَّأَمْسَكْتُمْ خَشْيَةَ ٱلْإِنفَاقِ ۚ وَكَانَ ٱلْإِنسَـٰنُ قَتُورًۭا ﴿١٠٠﴾}\\
101.\  & \mytextarabic{وَلَقَدْ ءَاتَيْنَا مُوسَىٰ تِسْعَ ءَايَـٰتٍۭ بَيِّنَـٰتٍۢ ۖ فَسْـَٔلْ بَنِىٓ إِسْرَٰٓءِيلَ إِذْ جَآءَهُمْ فَقَالَ لَهُۥ فِرْعَوْنُ إِنِّى لَأَظُنُّكَ يَـٰمُوسَىٰ مَسْحُورًۭا ﴿١٠١﴾}\\
102.\  & \mytextarabic{قَالَ لَقَدْ عَلِمْتَ مَآ أَنزَلَ هَـٰٓؤُلَآءِ إِلَّا رَبُّ ٱلسَّمَـٰوَٟتِ وَٱلْأَرْضِ بَصَآئِرَ وَإِنِّى لَأَظُنُّكَ يَـٰفِرْعَوْنُ مَثْبُورًۭا ﴿١٠٢﴾}\\
103.\  & \mytextarabic{فَأَرَادَ أَن يَسْتَفِزَّهُم مِّنَ ٱلْأَرْضِ فَأَغْرَقْنَـٰهُ وَمَن مَّعَهُۥ جَمِيعًۭا ﴿١٠٣﴾}\\
104.\  & \mytextarabic{وَقُلْنَا مِنۢ بَعْدِهِۦ لِبَنِىٓ إِسْرَٰٓءِيلَ ٱسْكُنُوا۟ ٱلْأَرْضَ فَإِذَا جَآءَ وَعْدُ ٱلْءَاخِرَةِ جِئْنَا بِكُمْ لَفِيفًۭا ﴿١٠٤﴾}\\
105.\  & \mytextarabic{وَبِٱلْحَقِّ أَنزَلْنَـٰهُ وَبِٱلْحَقِّ نَزَلَ ۗ وَمَآ أَرْسَلْنَـٰكَ إِلَّا مُبَشِّرًۭا وَنَذِيرًۭا ﴿١٠٥﴾}\\
106.\  & \mytextarabic{وَقُرْءَانًۭا فَرَقْنَـٰهُ لِتَقْرَأَهُۥ عَلَى ٱلنَّاسِ عَلَىٰ مُكْثٍۢ وَنَزَّلْنَـٰهُ تَنزِيلًۭا ﴿١٠٦﴾}\\
107.\  & \mytextarabic{قُلْ ءَامِنُوا۟ بِهِۦٓ أَوْ لَا تُؤْمِنُوٓا۟ ۚ إِنَّ ٱلَّذِينَ أُوتُوا۟ ٱلْعِلْمَ مِن قَبْلِهِۦٓ إِذَا يُتْلَىٰ عَلَيْهِمْ يَخِرُّونَ لِلْأَذْقَانِ سُجَّدًۭا ﴿١٠٧﴾}\\
108.\  & \mytextarabic{وَيَقُولُونَ سُبْحَـٰنَ رَبِّنَآ إِن كَانَ وَعْدُ رَبِّنَا لَمَفْعُولًۭا ﴿١٠٨﴾}\\
109.\  & \mytextarabic{وَيَخِرُّونَ لِلْأَذْقَانِ يَبْكُونَ وَيَزِيدُهُمْ خُشُوعًۭا ۩ ﴿١٠٩﴾}\\
110.\  & \mytextarabic{قُلِ ٱدْعُوا۟ ٱللَّهَ أَوِ ٱدْعُوا۟ ٱلرَّحْمَـٰنَ ۖ أَيًّۭا مَّا تَدْعُوا۟ فَلَهُ ٱلْأَسْمَآءُ ٱلْحُسْنَىٰ ۚ وَلَا تَجْهَرْ بِصَلَاتِكَ وَلَا تُخَافِتْ بِهَا وَٱبْتَغِ بَيْنَ ذَٟلِكَ سَبِيلًۭا ﴿١١٠﴾}\\
111.\  & \mytextarabic{وَقُلِ ٱلْحَمْدُ لِلَّهِ ٱلَّذِى لَمْ يَتَّخِذْ وَلَدًۭا وَلَمْ يَكُن لَّهُۥ شَرِيكٌۭ فِى ٱلْمُلْكِ وَلَمْ يَكُن لَّهُۥ وَلِىٌّۭ مِّنَ ٱلذُّلِّ ۖ وَكَبِّرْهُ تَكْبِيرًۢا ﴿١١١﴾}\\
\end{longtable}
\clearpage