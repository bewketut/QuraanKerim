\begin{center}\section{ሱራቱ ማሪያም -  \textarabic{سوره  مريم}}\end{center}
\begin{longtable}{%
  @{}
    p{.5\textwidth}
  @{~~~}
    p{.5\textwidth}
    @{}
}
ቢስሚላሂ አራህመኒ ራሂይም &  \mytextarabic{بِسْمِ ٱللَّهِ ٱلرَّحْمَـٰنِ ٱلرَّحِيمِ}\\
1.\  & \mytextarabic{ كٓهيعٓصٓ ﴿١﴾}\\
2.\  & \mytextarabic{ذِكْرُ رَحْمَتِ رَبِّكَ عَبْدَهُۥ زَكَرِيَّآ ﴿٢﴾}\\
3.\  & \mytextarabic{إِذْ نَادَىٰ رَبَّهُۥ نِدَآءً خَفِيًّۭا ﴿٣﴾}\\
4.\  & \mytextarabic{قَالَ رَبِّ إِنِّى وَهَنَ ٱلْعَظْمُ مِنِّى وَٱشْتَعَلَ ٱلرَّأْسُ شَيْبًۭا وَلَمْ أَكُنۢ بِدُعَآئِكَ رَبِّ شَقِيًّۭا ﴿٤﴾}\\
5.\  & \mytextarabic{وَإِنِّى خِفْتُ ٱلْمَوَٟلِىَ مِن وَرَآءِى وَكَانَتِ ٱمْرَأَتِى عَاقِرًۭا فَهَبْ لِى مِن لَّدُنكَ وَلِيًّۭا ﴿٥﴾}\\
6.\  & \mytextarabic{يَرِثُنِى وَيَرِثُ مِنْ ءَالِ يَعْقُوبَ ۖ وَٱجْعَلْهُ رَبِّ رَضِيًّۭا ﴿٦﴾}\\
7.\  & \mytextarabic{يَـٰزَكَرِيَّآ إِنَّا نُبَشِّرُكَ بِغُلَـٰمٍ ٱسْمُهُۥ يَحْيَىٰ لَمْ نَجْعَل لَّهُۥ مِن قَبْلُ سَمِيًّۭا ﴿٧﴾}\\
8.\  & \mytextarabic{قَالَ رَبِّ أَنَّىٰ يَكُونُ لِى غُلَـٰمٌۭ وَكَانَتِ ٱمْرَأَتِى عَاقِرًۭا وَقَدْ بَلَغْتُ مِنَ ٱلْكِبَرِ عِتِيًّۭا ﴿٨﴾}\\
9.\  & \mytextarabic{قَالَ كَذَٟلِكَ قَالَ رَبُّكَ هُوَ عَلَىَّ هَيِّنٌۭ وَقَدْ خَلَقْتُكَ مِن قَبْلُ وَلَمْ تَكُ شَيْـًۭٔا ﴿٩﴾}\\
10.\  & \mytextarabic{قَالَ رَبِّ ٱجْعَل لِّىٓ ءَايَةًۭ ۚ قَالَ ءَايَتُكَ أَلَّا تُكَلِّمَ ٱلنَّاسَ ثَلَـٰثَ لَيَالٍۢ سَوِيًّۭا ﴿١٠﴾}\\
11.\  & \mytextarabic{فَخَرَجَ عَلَىٰ قَوْمِهِۦ مِنَ ٱلْمِحْرَابِ فَأَوْحَىٰٓ إِلَيْهِمْ أَن سَبِّحُوا۟ بُكْرَةًۭ وَعَشِيًّۭا ﴿١١﴾}\\
12.\  & \mytextarabic{يَـٰيَحْيَىٰ خُذِ ٱلْكِتَـٰبَ بِقُوَّةٍۢ ۖ وَءَاتَيْنَـٰهُ ٱلْحُكْمَ صَبِيًّۭا ﴿١٢﴾}\\
13.\  & \mytextarabic{وَحَنَانًۭا مِّن لَّدُنَّا وَزَكَوٰةًۭ ۖ وَكَانَ تَقِيًّۭا ﴿١٣﴾}\\
14.\  & \mytextarabic{وَبَرًّۢا بِوَٟلِدَيْهِ وَلَمْ يَكُن جَبَّارًا عَصِيًّۭا ﴿١٤﴾}\\
15.\  & \mytextarabic{وَسَلَـٰمٌ عَلَيْهِ يَوْمَ وُلِدَ وَيَوْمَ يَمُوتُ وَيَوْمَ يُبْعَثُ حَيًّۭا ﴿١٥﴾}\\
16.\  & \mytextarabic{وَٱذْكُرْ فِى ٱلْكِتَـٰبِ مَرْيَمَ إِذِ ٱنتَبَذَتْ مِنْ أَهْلِهَا مَكَانًۭا شَرْقِيًّۭا ﴿١٦﴾}\\
17.\  & \mytextarabic{فَٱتَّخَذَتْ مِن دُونِهِمْ حِجَابًۭا فَأَرْسَلْنَآ إِلَيْهَا رُوحَنَا فَتَمَثَّلَ لَهَا بَشَرًۭا سَوِيًّۭا ﴿١٧﴾}\\
18.\  & \mytextarabic{قَالَتْ إِنِّىٓ أَعُوذُ بِٱلرَّحْمَـٰنِ مِنكَ إِن كُنتَ تَقِيًّۭا ﴿١٨﴾}\\
19.\  & \mytextarabic{قَالَ إِنَّمَآ أَنَا۠ رَسُولُ رَبِّكِ لِأَهَبَ لَكِ غُلَـٰمًۭا زَكِيًّۭا ﴿١٩﴾}\\
20.\  & \mytextarabic{قَالَتْ أَنَّىٰ يَكُونُ لِى غُلَـٰمٌۭ وَلَمْ يَمْسَسْنِى بَشَرٌۭ وَلَمْ أَكُ بَغِيًّۭا ﴿٢٠﴾}\\
21.\  & \mytextarabic{قَالَ كَذَٟلِكِ قَالَ رَبُّكِ هُوَ عَلَىَّ هَيِّنٌۭ ۖ وَلِنَجْعَلَهُۥٓ ءَايَةًۭ لِّلنَّاسِ وَرَحْمَةًۭ مِّنَّا ۚ وَكَانَ أَمْرًۭا مَّقْضِيًّۭا ﴿٢١﴾}\\
22.\  & \mytextarabic{۞ فَحَمَلَتْهُ فَٱنتَبَذَتْ بِهِۦ مَكَانًۭا قَصِيًّۭا ﴿٢٢﴾}\\
23.\  & \mytextarabic{فَأَجَآءَهَا ٱلْمَخَاضُ إِلَىٰ جِذْعِ ٱلنَّخْلَةِ قَالَتْ يَـٰلَيْتَنِى مِتُّ قَبْلَ هَـٰذَا وَكُنتُ نَسْيًۭا مَّنسِيًّۭا ﴿٢٣﴾}\\
24.\  & \mytextarabic{فَنَادَىٰهَا مِن تَحْتِهَآ أَلَّا تَحْزَنِى قَدْ جَعَلَ رَبُّكِ تَحْتَكِ سَرِيًّۭا ﴿٢٤﴾}\\
25.\  & \mytextarabic{وَهُزِّىٓ إِلَيْكِ بِجِذْعِ ٱلنَّخْلَةِ تُسَـٰقِطْ عَلَيْكِ رُطَبًۭا جَنِيًّۭا ﴿٢٥﴾}\\
26.\  & \mytextarabic{فَكُلِى وَٱشْرَبِى وَقَرِّى عَيْنًۭا ۖ فَإِمَّا تَرَيِنَّ مِنَ ٱلْبَشَرِ أَحَدًۭا فَقُولِىٓ إِنِّى نَذَرْتُ لِلرَّحْمَـٰنِ صَوْمًۭا فَلَنْ أُكَلِّمَ ٱلْيَوْمَ إِنسِيًّۭا ﴿٢٦﴾}\\
27.\  & \mytextarabic{فَأَتَتْ بِهِۦ قَوْمَهَا تَحْمِلُهُۥ ۖ قَالُوا۟ يَـٰمَرْيَمُ لَقَدْ جِئْتِ شَيْـًۭٔا فَرِيًّۭا ﴿٢٧﴾}\\
28.\  & \mytextarabic{يَـٰٓأُخْتَ هَـٰرُونَ مَا كَانَ أَبُوكِ ٱمْرَأَ سَوْءٍۢ وَمَا كَانَتْ أُمُّكِ بَغِيًّۭا ﴿٢٨﴾}\\
29.\  & \mytextarabic{فَأَشَارَتْ إِلَيْهِ ۖ قَالُوا۟ كَيْفَ نُكَلِّمُ مَن كَانَ فِى ٱلْمَهْدِ صَبِيًّۭا ﴿٢٩﴾}\\
30.\  & \mytextarabic{قَالَ إِنِّى عَبْدُ ٱللَّهِ ءَاتَىٰنِىَ ٱلْكِتَـٰبَ وَجَعَلَنِى نَبِيًّۭا ﴿٣٠﴾}\\
31.\  & \mytextarabic{وَجَعَلَنِى مُبَارَكًا أَيْنَ مَا كُنتُ وَأَوْصَـٰنِى بِٱلصَّلَوٰةِ وَٱلزَّكَوٰةِ مَا دُمْتُ حَيًّۭا ﴿٣١﴾}\\
32.\  & \mytextarabic{وَبَرًّۢا بِوَٟلِدَتِى وَلَمْ يَجْعَلْنِى جَبَّارًۭا شَقِيًّۭا ﴿٣٢﴾}\\
33.\  & \mytextarabic{وَٱلسَّلَـٰمُ عَلَىَّ يَوْمَ وُلِدتُّ وَيَوْمَ أَمُوتُ وَيَوْمَ أُبْعَثُ حَيًّۭا ﴿٣٣﴾}\\
34.\  & \mytextarabic{ذَٟلِكَ عِيسَى ٱبْنُ مَرْيَمَ ۚ قَوْلَ ٱلْحَقِّ ٱلَّذِى فِيهِ يَمْتَرُونَ ﴿٣٤﴾}\\
35.\  & \mytextarabic{مَا كَانَ لِلَّهِ أَن يَتَّخِذَ مِن وَلَدٍۢ ۖ سُبْحَـٰنَهُۥٓ ۚ إِذَا قَضَىٰٓ أَمْرًۭا فَإِنَّمَا يَقُولُ لَهُۥ كُن فَيَكُونُ ﴿٣٥﴾}\\
36.\  & \mytextarabic{وَإِنَّ ٱللَّهَ رَبِّى وَرَبُّكُمْ فَٱعْبُدُوهُ ۚ هَـٰذَا صِرَٰطٌۭ مُّسْتَقِيمٌۭ ﴿٣٦﴾}\\
37.\  & \mytextarabic{فَٱخْتَلَفَ ٱلْأَحْزَابُ مِنۢ بَيْنِهِمْ ۖ فَوَيْلٌۭ لِّلَّذِينَ كَفَرُوا۟ مِن مَّشْهَدِ يَوْمٍ عَظِيمٍ ﴿٣٧﴾}\\
38.\  & \mytextarabic{أَسْمِعْ بِهِمْ وَأَبْصِرْ يَوْمَ يَأْتُونَنَا ۖ لَـٰكِنِ ٱلظَّـٰلِمُونَ ٱلْيَوْمَ فِى ضَلَـٰلٍۢ مُّبِينٍۢ ﴿٣٨﴾}\\
39.\  & \mytextarabic{وَأَنذِرْهُمْ يَوْمَ ٱلْحَسْرَةِ إِذْ قُضِىَ ٱلْأَمْرُ وَهُمْ فِى غَفْلَةٍۢ وَهُمْ لَا يُؤْمِنُونَ ﴿٣٩﴾}\\
40.\  & \mytextarabic{إِنَّا نَحْنُ نَرِثُ ٱلْأَرْضَ وَمَنْ عَلَيْهَا وَإِلَيْنَا يُرْجَعُونَ ﴿٤٠﴾}\\
41.\  & \mytextarabic{وَٱذْكُرْ فِى ٱلْكِتَـٰبِ إِبْرَٰهِيمَ ۚ إِنَّهُۥ كَانَ صِدِّيقًۭا نَّبِيًّا ﴿٤١﴾}\\
42.\  & \mytextarabic{إِذْ قَالَ لِأَبِيهِ يَـٰٓأَبَتِ لِمَ تَعْبُدُ مَا لَا يَسْمَعُ وَلَا يُبْصِرُ وَلَا يُغْنِى عَنكَ شَيْـًۭٔا ﴿٤٢﴾}\\
43.\  & \mytextarabic{يَـٰٓأَبَتِ إِنِّى قَدْ جَآءَنِى مِنَ ٱلْعِلْمِ مَا لَمْ يَأْتِكَ فَٱتَّبِعْنِىٓ أَهْدِكَ صِرَٰطًۭا سَوِيًّۭا ﴿٤٣﴾}\\
44.\  & \mytextarabic{يَـٰٓأَبَتِ لَا تَعْبُدِ ٱلشَّيْطَٰنَ ۖ إِنَّ ٱلشَّيْطَٰنَ كَانَ لِلرَّحْمَـٰنِ عَصِيًّۭا ﴿٤٤﴾}\\
45.\  & \mytextarabic{يَـٰٓأَبَتِ إِنِّىٓ أَخَافُ أَن يَمَسَّكَ عَذَابٌۭ مِّنَ ٱلرَّحْمَـٰنِ فَتَكُونَ لِلشَّيْطَٰنِ وَلِيًّۭا ﴿٤٥﴾}\\
46.\  & \mytextarabic{قَالَ أَرَاغِبٌ أَنتَ عَنْ ءَالِهَتِى يَـٰٓإِبْرَٰهِيمُ ۖ لَئِن لَّمْ تَنتَهِ لَأَرْجُمَنَّكَ ۖ وَٱهْجُرْنِى مَلِيًّۭا ﴿٤٦﴾}\\
47.\  & \mytextarabic{قَالَ سَلَـٰمٌ عَلَيْكَ ۖ سَأَسْتَغْفِرُ لَكَ رَبِّىٓ ۖ إِنَّهُۥ كَانَ بِى حَفِيًّۭا ﴿٤٧﴾}\\
48.\  & \mytextarabic{وَأَعْتَزِلُكُمْ وَمَا تَدْعُونَ مِن دُونِ ٱللَّهِ وَأَدْعُوا۟ رَبِّى عَسَىٰٓ أَلَّآ أَكُونَ بِدُعَآءِ رَبِّى شَقِيًّۭا ﴿٤٨﴾}\\
49.\  & \mytextarabic{فَلَمَّا ٱعْتَزَلَهُمْ وَمَا يَعْبُدُونَ مِن دُونِ ٱللَّهِ وَهَبْنَا لَهُۥٓ إِسْحَـٰقَ وَيَعْقُوبَ ۖ وَكُلًّۭا جَعَلْنَا نَبِيًّۭا ﴿٤٩﴾}\\
50.\  & \mytextarabic{وَوَهَبْنَا لَهُم مِّن رَّحْمَتِنَا وَجَعَلْنَا لَهُمْ لِسَانَ صِدْقٍ عَلِيًّۭا ﴿٥٠﴾}\\
51.\  & \mytextarabic{وَٱذْكُرْ فِى ٱلْكِتَـٰبِ مُوسَىٰٓ ۚ إِنَّهُۥ كَانَ مُخْلَصًۭا وَكَانَ رَسُولًۭا نَّبِيًّۭا ﴿٥١﴾}\\
52.\  & \mytextarabic{وَنَـٰدَيْنَـٰهُ مِن جَانِبِ ٱلطُّورِ ٱلْأَيْمَنِ وَقَرَّبْنَـٰهُ نَجِيًّۭا ﴿٥٢﴾}\\
53.\  & \mytextarabic{وَوَهَبْنَا لَهُۥ مِن رَّحْمَتِنَآ أَخَاهُ هَـٰرُونَ نَبِيًّۭا ﴿٥٣﴾}\\
54.\  & \mytextarabic{وَٱذْكُرْ فِى ٱلْكِتَـٰبِ إِسْمَـٰعِيلَ ۚ إِنَّهُۥ كَانَ صَادِقَ ٱلْوَعْدِ وَكَانَ رَسُولًۭا نَّبِيًّۭا ﴿٥٤﴾}\\
55.\  & \mytextarabic{وَكَانَ يَأْمُرُ أَهْلَهُۥ بِٱلصَّلَوٰةِ وَٱلزَّكَوٰةِ وَكَانَ عِندَ رَبِّهِۦ مَرْضِيًّۭا ﴿٥٥﴾}\\
56.\  & \mytextarabic{وَٱذْكُرْ فِى ٱلْكِتَـٰبِ إِدْرِيسَ ۚ إِنَّهُۥ كَانَ صِدِّيقًۭا نَّبِيًّۭا ﴿٥٦﴾}\\
57.\  & \mytextarabic{وَرَفَعْنَـٰهُ مَكَانًا عَلِيًّا ﴿٥٧﴾}\\
58.\  & \mytextarabic{أُو۟لَـٰٓئِكَ ٱلَّذِينَ أَنْعَمَ ٱللَّهُ عَلَيْهِم مِّنَ ٱلنَّبِيِّۦنَ مِن ذُرِّيَّةِ ءَادَمَ وَمِمَّنْ حَمَلْنَا مَعَ نُوحٍۢ وَمِن ذُرِّيَّةِ إِبْرَٰهِيمَ وَإِسْرَٰٓءِيلَ وَمِمَّنْ هَدَيْنَا وَٱجْتَبَيْنَآ ۚ إِذَا تُتْلَىٰ عَلَيْهِمْ ءَايَـٰتُ ٱلرَّحْمَـٰنِ خَرُّوا۟ سُجَّدًۭا وَبُكِيًّۭا ۩ ﴿٥٨﴾}\\
59.\  & \mytextarabic{۞ فَخَلَفَ مِنۢ بَعْدِهِمْ خَلْفٌ أَضَاعُوا۟ ٱلصَّلَوٰةَ وَٱتَّبَعُوا۟ ٱلشَّهَوَٟتِ ۖ فَسَوْفَ يَلْقَوْنَ غَيًّا ﴿٥٩﴾}\\
60.\  & \mytextarabic{إِلَّا مَن تَابَ وَءَامَنَ وَعَمِلَ صَـٰلِحًۭا فَأُو۟لَـٰٓئِكَ يَدْخُلُونَ ٱلْجَنَّةَ وَلَا يُظْلَمُونَ شَيْـًۭٔا ﴿٦٠﴾}\\
61.\  & \mytextarabic{جَنَّـٰتِ عَدْنٍ ٱلَّتِى وَعَدَ ٱلرَّحْمَـٰنُ عِبَادَهُۥ بِٱلْغَيْبِ ۚ إِنَّهُۥ كَانَ وَعْدُهُۥ مَأْتِيًّۭا ﴿٦١﴾}\\
62.\  & \mytextarabic{لَّا يَسْمَعُونَ فِيهَا لَغْوًا إِلَّا سَلَـٰمًۭا ۖ وَلَهُمْ رِزْقُهُمْ فِيهَا بُكْرَةًۭ وَعَشِيًّۭا ﴿٦٢﴾}\\
63.\  & \mytextarabic{تِلْكَ ٱلْجَنَّةُ ٱلَّتِى نُورِثُ مِنْ عِبَادِنَا مَن كَانَ تَقِيًّۭا ﴿٦٣﴾}\\
64.\  & \mytextarabic{وَمَا نَتَنَزَّلُ إِلَّا بِأَمْرِ رَبِّكَ ۖ لَهُۥ مَا بَيْنَ أَيْدِينَا وَمَا خَلْفَنَا وَمَا بَيْنَ ذَٟلِكَ ۚ وَمَا كَانَ رَبُّكَ نَسِيًّۭا ﴿٦٤﴾}\\
65.\  & \mytextarabic{رَّبُّ ٱلسَّمَـٰوَٟتِ وَٱلْأَرْضِ وَمَا بَيْنَهُمَا فَٱعْبُدْهُ وَٱصْطَبِرْ لِعِبَٰدَتِهِۦ ۚ هَلْ تَعْلَمُ لَهُۥ سَمِيًّۭا ﴿٦٥﴾}\\
66.\  & \mytextarabic{وَيَقُولُ ٱلْإِنسَـٰنُ أَءِذَا مَا مِتُّ لَسَوْفَ أُخْرَجُ حَيًّا ﴿٦٦﴾}\\
67.\  & \mytextarabic{أَوَلَا يَذْكُرُ ٱلْإِنسَـٰنُ أَنَّا خَلَقْنَـٰهُ مِن قَبْلُ وَلَمْ يَكُ شَيْـًۭٔا ﴿٦٧﴾}\\
68.\  & \mytextarabic{فَوَرَبِّكَ لَنَحْشُرَنَّهُمْ وَٱلشَّيَـٰطِينَ ثُمَّ لَنُحْضِرَنَّهُمْ حَوْلَ جَهَنَّمَ جِثِيًّۭا ﴿٦٨﴾}\\
69.\  & \mytextarabic{ثُمَّ لَنَنزِعَنَّ مِن كُلِّ شِيعَةٍ أَيُّهُمْ أَشَدُّ عَلَى ٱلرَّحْمَـٰنِ عِتِيًّۭا ﴿٦٩﴾}\\
70.\  & \mytextarabic{ثُمَّ لَنَحْنُ أَعْلَمُ بِٱلَّذِينَ هُمْ أَوْلَىٰ بِهَا صِلِيًّۭا ﴿٧٠﴾}\\
71.\  & \mytextarabic{وَإِن مِّنكُمْ إِلَّا وَارِدُهَا ۚ كَانَ عَلَىٰ رَبِّكَ حَتْمًۭا مَّقْضِيًّۭا ﴿٧١﴾}\\
72.\  & \mytextarabic{ثُمَّ نُنَجِّى ٱلَّذِينَ ٱتَّقَوا۟ وَّنَذَرُ ٱلظَّـٰلِمِينَ فِيهَا جِثِيًّۭا ﴿٧٢﴾}\\
73.\  & \mytextarabic{وَإِذَا تُتْلَىٰ عَلَيْهِمْ ءَايَـٰتُنَا بَيِّنَـٰتٍۢ قَالَ ٱلَّذِينَ كَفَرُوا۟ لِلَّذِينَ ءَامَنُوٓا۟ أَىُّ ٱلْفَرِيقَيْنِ خَيْرٌۭ مَّقَامًۭا وَأَحْسَنُ نَدِيًّۭا ﴿٧٣﴾}\\
74.\  & \mytextarabic{وَكَمْ أَهْلَكْنَا قَبْلَهُم مِّن قَرْنٍ هُمْ أَحْسَنُ أَثَـٰثًۭا وَرِءْيًۭا ﴿٧٤﴾}\\
75.\  & \mytextarabic{قُلْ مَن كَانَ فِى ٱلضَّلَـٰلَةِ فَلْيَمْدُدْ لَهُ ٱلرَّحْمَـٰنُ مَدًّا ۚ حَتَّىٰٓ إِذَا رَأَوْا۟ مَا يُوعَدُونَ إِمَّا ٱلْعَذَابَ وَإِمَّا ٱلسَّاعَةَ فَسَيَعْلَمُونَ مَنْ هُوَ شَرٌّۭ مَّكَانًۭا وَأَضْعَفُ جُندًۭا ﴿٧٥﴾}\\
76.\  & \mytextarabic{وَيَزِيدُ ٱللَّهُ ٱلَّذِينَ ٱهْتَدَوْا۟ هُدًۭى ۗ وَٱلْبَٰقِيَـٰتُ ٱلصَّـٰلِحَـٰتُ خَيْرٌ عِندَ رَبِّكَ ثَوَابًۭا وَخَيْرٌۭ مَّرَدًّا ﴿٧٦﴾}\\
77.\  & \mytextarabic{أَفَرَءَيْتَ ٱلَّذِى كَفَرَ بِـَٔايَـٰتِنَا وَقَالَ لَأُوتَيَنَّ مَالًۭا وَوَلَدًا ﴿٧٧﴾}\\
78.\  & \mytextarabic{أَطَّلَعَ ٱلْغَيْبَ أَمِ ٱتَّخَذَ عِندَ ٱلرَّحْمَـٰنِ عَهْدًۭا ﴿٧٨﴾}\\
79.\  & \mytextarabic{كَلَّا ۚ سَنَكْتُبُ مَا يَقُولُ وَنَمُدُّ لَهُۥ مِنَ ٱلْعَذَابِ مَدًّۭا ﴿٧٩﴾}\\
80.\  & \mytextarabic{وَنَرِثُهُۥ مَا يَقُولُ وَيَأْتِينَا فَرْدًۭا ﴿٨٠﴾}\\
81.\  & \mytextarabic{وَٱتَّخَذُوا۟ مِن دُونِ ٱللَّهِ ءَالِهَةًۭ لِّيَكُونُوا۟ لَهُمْ عِزًّۭا ﴿٨١﴾}\\
82.\  & \mytextarabic{كَلَّا ۚ سَيَكْفُرُونَ بِعِبَادَتِهِمْ وَيَكُونُونَ عَلَيْهِمْ ضِدًّا ﴿٨٢﴾}\\
83.\  & \mytextarabic{أَلَمْ تَرَ أَنَّآ أَرْسَلْنَا ٱلشَّيَـٰطِينَ عَلَى ٱلْكَـٰفِرِينَ تَؤُزُّهُمْ أَزًّۭا ﴿٨٣﴾}\\
84.\  & \mytextarabic{فَلَا تَعْجَلْ عَلَيْهِمْ ۖ إِنَّمَا نَعُدُّ لَهُمْ عَدًّۭا ﴿٨٤﴾}\\
85.\  & \mytextarabic{يَوْمَ نَحْشُرُ ٱلْمُتَّقِينَ إِلَى ٱلرَّحْمَـٰنِ وَفْدًۭا ﴿٨٥﴾}\\
86.\  & \mytextarabic{وَنَسُوقُ ٱلْمُجْرِمِينَ إِلَىٰ جَهَنَّمَ وِرْدًۭا ﴿٨٦﴾}\\
87.\  & \mytextarabic{لَّا يَمْلِكُونَ ٱلشَّفَـٰعَةَ إِلَّا مَنِ ٱتَّخَذَ عِندَ ٱلرَّحْمَـٰنِ عَهْدًۭا ﴿٨٧﴾}\\
88.\  & \mytextarabic{وَقَالُوا۟ ٱتَّخَذَ ٱلرَّحْمَـٰنُ وَلَدًۭا ﴿٨٨﴾}\\
89.\  & \mytextarabic{لَّقَدْ جِئْتُمْ شَيْـًٔا إِدًّۭا ﴿٨٩﴾}\\
90.\  & \mytextarabic{تَكَادُ ٱلسَّمَـٰوَٟتُ يَتَفَطَّرْنَ مِنْهُ وَتَنشَقُّ ٱلْأَرْضُ وَتَخِرُّ ٱلْجِبَالُ هَدًّا ﴿٩٠﴾}\\
91.\  & \mytextarabic{أَن دَعَوْا۟ لِلرَّحْمَـٰنِ وَلَدًۭا ﴿٩١﴾}\\
92.\  & \mytextarabic{وَمَا يَنۢبَغِى لِلرَّحْمَـٰنِ أَن يَتَّخِذَ وَلَدًا ﴿٩٢﴾}\\
93.\  & \mytextarabic{إِن كُلُّ مَن فِى ٱلسَّمَـٰوَٟتِ وَٱلْأَرْضِ إِلَّآ ءَاتِى ٱلرَّحْمَـٰنِ عَبْدًۭا ﴿٩٣﴾}\\
94.\  & \mytextarabic{لَّقَدْ أَحْصَىٰهُمْ وَعَدَّهُمْ عَدًّۭا ﴿٩٤﴾}\\
95.\  & \mytextarabic{وَكُلُّهُمْ ءَاتِيهِ يَوْمَ ٱلْقِيَـٰمَةِ فَرْدًا ﴿٩٥﴾}\\
96.\  & \mytextarabic{إِنَّ ٱلَّذِينَ ءَامَنُوا۟ وَعَمِلُوا۟ ٱلصَّـٰلِحَـٰتِ سَيَجْعَلُ لَهُمُ ٱلرَّحْمَـٰنُ وُدًّۭا ﴿٩٦﴾}\\
97.\  & \mytextarabic{فَإِنَّمَا يَسَّرْنَـٰهُ بِلِسَانِكَ لِتُبَشِّرَ بِهِ ٱلْمُتَّقِينَ وَتُنذِرَ بِهِۦ قَوْمًۭا لُّدًّۭا ﴿٩٧﴾}\\
98.\  & \mytextarabic{وَكَمْ أَهْلَكْنَا قَبْلَهُم مِّن قَرْنٍ هَلْ تُحِسُّ مِنْهُم مِّنْ أَحَدٍ أَوْ تَسْمَعُ لَهُمْ رِكْزًۢا ﴿٩٨﴾}\\
\end{longtable}
\clearpage