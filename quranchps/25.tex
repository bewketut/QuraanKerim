%% License: BSD style (Berkley) (i.e. Put the Copyright owner's name always)
%% Writer and Copyright (to): Bewketu(Bilal) Tadilo (2016-17)
\begin{center}\section{ሱራቱ አልፉርቃን -  \textarabic{سوره  الفرقان}}\end{center}
\begin{longtable}{%
  @{}
    p{.5\textwidth}
  @{~~~}
    p{.5\textwidth}
    @{}
}
ቢስሚላሂ አራህመኒ ራሂይም &  \mytextarabic{بِسْمِ ٱللَّهِ ٱلرَّحْمَـٰنِ ٱلرَّحِيمِ}\\
1.\  & \mytextarabic{ تَبَارَكَ ٱلَّذِى نَزَّلَ ٱلْفُرْقَانَ عَلَىٰ عَبْدِهِۦ لِيَكُونَ لِلْعَـٰلَمِينَ نَذِيرًا ﴿١﴾}\\
2.\  & \mytextarabic{ٱلَّذِى لَهُۥ مُلْكُ ٱلسَّمَـٰوَٟتِ وَٱلْأَرْضِ وَلَمْ يَتَّخِذْ وَلَدًۭا وَلَمْ يَكُن لَّهُۥ شَرِيكٌۭ فِى ٱلْمُلْكِ وَخَلَقَ كُلَّ شَىْءٍۢ فَقَدَّرَهُۥ تَقْدِيرًۭا ﴿٢﴾}\\
3.\  & \mytextarabic{وَٱتَّخَذُوا۟ مِن دُونِهِۦٓ ءَالِهَةًۭ لَّا يَخْلُقُونَ شَيْـًۭٔا وَهُمْ يُخْلَقُونَ وَلَا يَمْلِكُونَ لِأَنفُسِهِمْ ضَرًّۭا وَلَا نَفْعًۭا وَلَا يَمْلِكُونَ مَوْتًۭا وَلَا حَيَوٰةًۭ وَلَا نُشُورًۭا ﴿٣﴾}\\
4.\  & \mytextarabic{وَقَالَ ٱلَّذِينَ كَفَرُوٓا۟ إِنْ هَـٰذَآ إِلَّآ إِفْكٌ ٱفْتَرَىٰهُ وَأَعَانَهُۥ عَلَيْهِ قَوْمٌ ءَاخَرُونَ ۖ فَقَدْ جَآءُو ظُلْمًۭا وَزُورًۭا ﴿٤﴾}\\
5.\  & \mytextarabic{وَقَالُوٓا۟ أَسَـٰطِيرُ ٱلْأَوَّلِينَ ٱكْتَتَبَهَا فَهِىَ تُمْلَىٰ عَلَيْهِ بُكْرَةًۭ وَأَصِيلًۭا ﴿٥﴾}\\
6.\  & \mytextarabic{قُلْ أَنزَلَهُ ٱلَّذِى يَعْلَمُ ٱلسِّرَّ فِى ٱلسَّمَـٰوَٟتِ وَٱلْأَرْضِ ۚ إِنَّهُۥ كَانَ غَفُورًۭا رَّحِيمًۭا ﴿٦﴾}\\
7.\  & \mytextarabic{وَقَالُوا۟ مَالِ هَـٰذَا ٱلرَّسُولِ يَأْكُلُ ٱلطَّعَامَ وَيَمْشِى فِى ٱلْأَسْوَاقِ ۙ لَوْلَآ أُنزِلَ إِلَيْهِ مَلَكٌۭ فَيَكُونَ مَعَهُۥ نَذِيرًا ﴿٧﴾}\\
8.\  & \mytextarabic{أَوْ يُلْقَىٰٓ إِلَيْهِ كَنزٌ أَوْ تَكُونُ لَهُۥ جَنَّةٌۭ يَأْكُلُ مِنْهَا ۚ وَقَالَ ٱلظَّـٰلِمُونَ إِن تَتَّبِعُونَ إِلَّا رَجُلًۭا مَّسْحُورًا ﴿٨﴾}\\
9.\  & \mytextarabic{ٱنظُرْ كَيْفَ ضَرَبُوا۟ لَكَ ٱلْأَمْثَـٰلَ فَضَلُّوا۟ فَلَا يَسْتَطِيعُونَ سَبِيلًۭا ﴿٩﴾}\\
10.\  & \mytextarabic{تَبَارَكَ ٱلَّذِىٓ إِن شَآءَ جَعَلَ لَكَ خَيْرًۭا مِّن ذَٟلِكَ جَنَّـٰتٍۢ تَجْرِى مِن تَحْتِهَا ٱلْأَنْهَـٰرُ وَيَجْعَل لَّكَ قُصُورًۢا ﴿١٠﴾}\\
11.\  & \mytextarabic{بَلْ كَذَّبُوا۟ بِٱلسَّاعَةِ ۖ وَأَعْتَدْنَا لِمَن كَذَّبَ بِٱلسَّاعَةِ سَعِيرًا ﴿١١﴾}\\
12.\  & \mytextarabic{إِذَا رَأَتْهُم مِّن مَّكَانٍۭ بَعِيدٍۢ سَمِعُوا۟ لَهَا تَغَيُّظًۭا وَزَفِيرًۭا ﴿١٢﴾}\\
13.\  & \mytextarabic{وَإِذَآ أُلْقُوا۟ مِنْهَا مَكَانًۭا ضَيِّقًۭا مُّقَرَّنِينَ دَعَوْا۟ هُنَالِكَ ثُبُورًۭا ﴿١٣﴾}\\
14.\  & \mytextarabic{لَّا تَدْعُوا۟ ٱلْيَوْمَ ثُبُورًۭا وَٟحِدًۭا وَٱدْعُوا۟ ثُبُورًۭا كَثِيرًۭا ﴿١٤﴾}\\
15.\  & \mytextarabic{قُلْ أَذَٟلِكَ خَيْرٌ أَمْ جَنَّةُ ٱلْخُلْدِ ٱلَّتِى وُعِدَ ٱلْمُتَّقُونَ ۚ كَانَتْ لَهُمْ جَزَآءًۭ وَمَصِيرًۭا ﴿١٥﴾}\\
16.\  & \mytextarabic{لَّهُمْ فِيهَا مَا يَشَآءُونَ خَـٰلِدِينَ ۚ كَانَ عَلَىٰ رَبِّكَ وَعْدًۭا مَّسْـُٔولًۭا ﴿١٦﴾}\\
17.\  & \mytextarabic{وَيَوْمَ يَحْشُرُهُمْ وَمَا يَعْبُدُونَ مِن دُونِ ٱللَّهِ فَيَقُولُ ءَأَنتُمْ أَضْلَلْتُمْ عِبَادِى هَـٰٓؤُلَآءِ أَمْ هُمْ ضَلُّوا۟ ٱلسَّبِيلَ ﴿١٧﴾}\\
18.\  & \mytextarabic{قَالُوا۟ سُبْحَـٰنَكَ مَا كَانَ يَنۢبَغِى لَنَآ أَن نَّتَّخِذَ مِن دُونِكَ مِنْ أَوْلِيَآءَ وَلَـٰكِن مَّتَّعْتَهُمْ وَءَابَآءَهُمْ حَتَّىٰ نَسُوا۟ ٱلذِّكْرَ وَكَانُوا۟ قَوْمًۢا بُورًۭا ﴿١٨﴾}\\
19.\  & \mytextarabic{فَقَدْ كَذَّبُوكُم بِمَا تَقُولُونَ فَمَا تَسْتَطِيعُونَ صَرْفًۭا وَلَا نَصْرًۭا ۚ وَمَن يَظْلِم مِّنكُمْ نُذِقْهُ عَذَابًۭا كَبِيرًۭا ﴿١٩﴾}\\
20.\  & \mytextarabic{وَمَآ أَرْسَلْنَا قَبْلَكَ مِنَ ٱلْمُرْسَلِينَ إِلَّآ إِنَّهُمْ لَيَأْكُلُونَ ٱلطَّعَامَ وَيَمْشُونَ فِى ٱلْأَسْوَاقِ ۗ وَجَعَلْنَا بَعْضَكُمْ لِبَعْضٍۢ فِتْنَةً أَتَصْبِرُونَ ۗ وَكَانَ رَبُّكَ بَصِيرًۭا ﴿٢٠﴾}\\
21.\  & \mytextarabic{۞ وَقَالَ ٱلَّذِينَ لَا يَرْجُونَ لِقَآءَنَا لَوْلَآ أُنزِلَ عَلَيْنَا ٱلْمَلَـٰٓئِكَةُ أَوْ نَرَىٰ رَبَّنَا ۗ لَقَدِ ٱسْتَكْبَرُوا۟ فِىٓ أَنفُسِهِمْ وَعَتَوْ عُتُوًّۭا كَبِيرًۭا ﴿٢١﴾}\\
22.\  & \mytextarabic{يَوْمَ يَرَوْنَ ٱلْمَلَـٰٓئِكَةَ لَا بُشْرَىٰ يَوْمَئِذٍۢ لِّلْمُجْرِمِينَ وَيَقُولُونَ حِجْرًۭا مَّحْجُورًۭا ﴿٢٢﴾}\\
23.\  & \mytextarabic{وَقَدِمْنَآ إِلَىٰ مَا عَمِلُوا۟ مِنْ عَمَلٍۢ فَجَعَلْنَـٰهُ هَبَآءًۭ مَّنثُورًا ﴿٢٣﴾}\\
24.\  & \mytextarabic{أَصْحَـٰبُ ٱلْجَنَّةِ يَوْمَئِذٍ خَيْرٌۭ مُّسْتَقَرًّۭا وَأَحْسَنُ مَقِيلًۭا ﴿٢٤﴾}\\
25.\  & \mytextarabic{وَيَوْمَ تَشَقَّقُ ٱلسَّمَآءُ بِٱلْغَمَـٰمِ وَنُزِّلَ ٱلْمَلَـٰٓئِكَةُ تَنزِيلًا ﴿٢٥﴾}\\
26.\  & \mytextarabic{ٱلْمُلْكُ يَوْمَئِذٍ ٱلْحَقُّ لِلرَّحْمَـٰنِ ۚ وَكَانَ يَوْمًا عَلَى ٱلْكَـٰفِرِينَ عَسِيرًۭا ﴿٢٦﴾}\\
27.\  & \mytextarabic{وَيَوْمَ يَعَضُّ ٱلظَّالِمُ عَلَىٰ يَدَيْهِ يَقُولُ يَـٰلَيْتَنِى ٱتَّخَذْتُ مَعَ ٱلرَّسُولِ سَبِيلًۭا ﴿٢٧﴾}\\
28.\  & \mytextarabic{يَـٰوَيْلَتَىٰ لَيْتَنِى لَمْ أَتَّخِذْ فُلَانًا خَلِيلًۭا ﴿٢٨﴾}\\
29.\  & \mytextarabic{لَّقَدْ أَضَلَّنِى عَنِ ٱلذِّكْرِ بَعْدَ إِذْ جَآءَنِى ۗ وَكَانَ ٱلشَّيْطَٰنُ لِلْإِنسَـٰنِ خَذُولًۭا ﴿٢٩﴾}\\
30.\  & \mytextarabic{وَقَالَ ٱلرَّسُولُ يَـٰرَبِّ إِنَّ قَوْمِى ٱتَّخَذُوا۟ هَـٰذَا ٱلْقُرْءَانَ مَهْجُورًۭا ﴿٣٠﴾}\\
31.\  & \mytextarabic{وَكَذَٟلِكَ جَعَلْنَا لِكُلِّ نَبِىٍّ عَدُوًّۭا مِّنَ ٱلْمُجْرِمِينَ ۗ وَكَفَىٰ بِرَبِّكَ هَادِيًۭا وَنَصِيرًۭا ﴿٣١﴾}\\
32.\  & \mytextarabic{وَقَالَ ٱلَّذِينَ كَفَرُوا۟ لَوْلَا نُزِّلَ عَلَيْهِ ٱلْقُرْءَانُ جُمْلَةًۭ وَٟحِدَةًۭ ۚ كَذَٟلِكَ لِنُثَبِّتَ بِهِۦ فُؤَادَكَ ۖ وَرَتَّلْنَـٰهُ تَرْتِيلًۭا ﴿٣٢﴾}\\
33.\  & \mytextarabic{وَلَا يَأْتُونَكَ بِمَثَلٍ إِلَّا جِئْنَـٰكَ بِٱلْحَقِّ وَأَحْسَنَ تَفْسِيرًا ﴿٣٣﴾}\\
34.\  & \mytextarabic{ٱلَّذِينَ يُحْشَرُونَ عَلَىٰ وُجُوهِهِمْ إِلَىٰ جَهَنَّمَ أُو۟لَـٰٓئِكَ شَرٌّۭ مَّكَانًۭا وَأَضَلُّ سَبِيلًۭا ﴿٣٤﴾}\\
35.\  & \mytextarabic{وَلَقَدْ ءَاتَيْنَا مُوسَى ٱلْكِتَـٰبَ وَجَعَلْنَا مَعَهُۥٓ أَخَاهُ هَـٰرُونَ وَزِيرًۭا ﴿٣٥﴾}\\
36.\  & \mytextarabic{فَقُلْنَا ٱذْهَبَآ إِلَى ٱلْقَوْمِ ٱلَّذِينَ كَذَّبُوا۟ بِـَٔايَـٰتِنَا فَدَمَّرْنَـٰهُمْ تَدْمِيرًۭا ﴿٣٦﴾}\\
37.\  & \mytextarabic{وَقَوْمَ نُوحٍۢ لَّمَّا كَذَّبُوا۟ ٱلرُّسُلَ أَغْرَقْنَـٰهُمْ وَجَعَلْنَـٰهُمْ لِلنَّاسِ ءَايَةًۭ ۖ وَأَعْتَدْنَا لِلظَّـٰلِمِينَ عَذَابًا أَلِيمًۭا ﴿٣٧﴾}\\
38.\  & \mytextarabic{وَعَادًۭا وَثَمُودَا۟ وَأَصْحَـٰبَ ٱلرَّسِّ وَقُرُونًۢا بَيْنَ ذَٟلِكَ كَثِيرًۭا ﴿٣٨﴾}\\
39.\  & \mytextarabic{وَكُلًّۭا ضَرَبْنَا لَهُ ٱلْأَمْثَـٰلَ ۖ وَكُلًّۭا تَبَّرْنَا تَتْبِيرًۭا ﴿٣٩﴾}\\
40.\  & \mytextarabic{وَلَقَدْ أَتَوْا۟ عَلَى ٱلْقَرْيَةِ ٱلَّتِىٓ أُمْطِرَتْ مَطَرَ ٱلسَّوْءِ ۚ أَفَلَمْ يَكُونُوا۟ يَرَوْنَهَا ۚ بَلْ كَانُوا۟ لَا يَرْجُونَ نُشُورًۭا ﴿٤٠﴾}\\
41.\  & \mytextarabic{وَإِذَا رَأَوْكَ إِن يَتَّخِذُونَكَ إِلَّا هُزُوًا أَهَـٰذَا ٱلَّذِى بَعَثَ ٱللَّهُ رَسُولًا ﴿٤١﴾}\\
42.\  & \mytextarabic{إِن كَادَ لَيُضِلُّنَا عَنْ ءَالِهَتِنَا لَوْلَآ أَن صَبَرْنَا عَلَيْهَا ۚ وَسَوْفَ يَعْلَمُونَ حِينَ يَرَوْنَ ٱلْعَذَابَ مَنْ أَضَلُّ سَبِيلًا ﴿٤٢﴾}\\
43.\  & \mytextarabic{أَرَءَيْتَ مَنِ ٱتَّخَذَ إِلَـٰهَهُۥ هَوَىٰهُ أَفَأَنتَ تَكُونُ عَلَيْهِ وَكِيلًا ﴿٤٣﴾}\\
44.\  & \mytextarabic{أَمْ تَحْسَبُ أَنَّ أَكْثَرَهُمْ يَسْمَعُونَ أَوْ يَعْقِلُونَ ۚ إِنْ هُمْ إِلَّا كَٱلْأَنْعَـٰمِ ۖ بَلْ هُمْ أَضَلُّ سَبِيلًا ﴿٤٤﴾}\\
45.\  & \mytextarabic{أَلَمْ تَرَ إِلَىٰ رَبِّكَ كَيْفَ مَدَّ ٱلظِّلَّ وَلَوْ شَآءَ لَجَعَلَهُۥ سَاكِنًۭا ثُمَّ جَعَلْنَا ٱلشَّمْسَ عَلَيْهِ دَلِيلًۭا ﴿٤٥﴾}\\
46.\  & \mytextarabic{ثُمَّ قَبَضْنَـٰهُ إِلَيْنَا قَبْضًۭا يَسِيرًۭا ﴿٤٦﴾}\\
47.\  & \mytextarabic{وَهُوَ ٱلَّذِى جَعَلَ لَكُمُ ٱلَّيْلَ لِبَاسًۭا وَٱلنَّوْمَ سُبَاتًۭا وَجَعَلَ ٱلنَّهَارَ نُشُورًۭا ﴿٤٧﴾}\\
48.\  & \mytextarabic{وَهُوَ ٱلَّذِىٓ أَرْسَلَ ٱلرِّيَـٰحَ بُشْرًۢا بَيْنَ يَدَىْ رَحْمَتِهِۦ ۚ وَأَنزَلْنَا مِنَ ٱلسَّمَآءِ مَآءًۭ طَهُورًۭا ﴿٤٨﴾}\\
49.\  & \mytextarabic{لِّنُحْۦِىَ بِهِۦ بَلْدَةًۭ مَّيْتًۭا وَنُسْقِيَهُۥ مِمَّا خَلَقْنَآ أَنْعَـٰمًۭا وَأَنَاسِىَّ كَثِيرًۭا ﴿٤٩﴾}\\
50.\  & \mytextarabic{وَلَقَدْ صَرَّفْنَـٰهُ بَيْنَهُمْ لِيَذَّكَّرُوا۟ فَأَبَىٰٓ أَكْثَرُ ٱلنَّاسِ إِلَّا كُفُورًۭا ﴿٥٠﴾}\\
51.\  & \mytextarabic{وَلَوْ شِئْنَا لَبَعَثْنَا فِى كُلِّ قَرْيَةٍۢ نَّذِيرًۭا ﴿٥١﴾}\\
52.\  & \mytextarabic{فَلَا تُطِعِ ٱلْكَـٰفِرِينَ وَجَٰهِدْهُم بِهِۦ جِهَادًۭا كَبِيرًۭا ﴿٥٢﴾}\\
53.\  & \mytextarabic{۞ وَهُوَ ٱلَّذِى مَرَجَ ٱلْبَحْرَيْنِ هَـٰذَا عَذْبٌۭ فُرَاتٌۭ وَهَـٰذَا مِلْحٌ أُجَاجٌۭ وَجَعَلَ بَيْنَهُمَا بَرْزَخًۭا وَحِجْرًۭا مَّحْجُورًۭا ﴿٥٣﴾}\\
54.\  & \mytextarabic{وَهُوَ ٱلَّذِى خَلَقَ مِنَ ٱلْمَآءِ بَشَرًۭا فَجَعَلَهُۥ نَسَبًۭا وَصِهْرًۭا ۗ وَكَانَ رَبُّكَ قَدِيرًۭا ﴿٥٤﴾}\\
55.\  & \mytextarabic{وَيَعْبُدُونَ مِن دُونِ ٱللَّهِ مَا لَا يَنفَعُهُمْ وَلَا يَضُرُّهُمْ ۗ وَكَانَ ٱلْكَافِرُ عَلَىٰ رَبِّهِۦ ظَهِيرًۭا ﴿٥٥﴾}\\
56.\  & \mytextarabic{وَمَآ أَرْسَلْنَـٰكَ إِلَّا مُبَشِّرًۭا وَنَذِيرًۭا ﴿٥٦﴾}\\
57.\  & \mytextarabic{قُلْ مَآ أَسْـَٔلُكُمْ عَلَيْهِ مِنْ أَجْرٍ إِلَّا مَن شَآءَ أَن يَتَّخِذَ إِلَىٰ رَبِّهِۦ سَبِيلًۭا ﴿٥٧﴾}\\
58.\  & \mytextarabic{وَتَوَكَّلْ عَلَى ٱلْحَىِّ ٱلَّذِى لَا يَمُوتُ وَسَبِّحْ بِحَمْدِهِۦ ۚ وَكَفَىٰ بِهِۦ بِذُنُوبِ عِبَادِهِۦ خَبِيرًا ﴿٥٨﴾}\\
59.\  & \mytextarabic{ٱلَّذِى خَلَقَ ٱلسَّمَـٰوَٟتِ وَٱلْأَرْضَ وَمَا بَيْنَهُمَا فِى سِتَّةِ أَيَّامٍۢ ثُمَّ ٱسْتَوَىٰ عَلَى ٱلْعَرْشِ ۚ ٱلرَّحْمَـٰنُ فَسْـَٔلْ بِهِۦ خَبِيرًۭا ﴿٥٩﴾}\\
60.\  & \mytextarabic{وَإِذَا قِيلَ لَهُمُ ٱسْجُدُوا۟ لِلرَّحْمَـٰنِ قَالُوا۟ وَمَا ٱلرَّحْمَـٰنُ أَنَسْجُدُ لِمَا تَأْمُرُنَا وَزَادَهُمْ نُفُورًۭا ۩ ﴿٦٠﴾}\\
61.\  & \mytextarabic{تَبَارَكَ ٱلَّذِى جَعَلَ فِى ٱلسَّمَآءِ بُرُوجًۭا وَجَعَلَ فِيهَا سِرَٰجًۭا وَقَمَرًۭا مُّنِيرًۭا ﴿٦١﴾}\\
62.\  & \mytextarabic{وَهُوَ ٱلَّذِى جَعَلَ ٱلَّيْلَ وَٱلنَّهَارَ خِلْفَةًۭ لِّمَنْ أَرَادَ أَن يَذَّكَّرَ أَوْ أَرَادَ شُكُورًۭا ﴿٦٢﴾}\\
63.\  & \mytextarabic{وَعِبَادُ ٱلرَّحْمَـٰنِ ٱلَّذِينَ يَمْشُونَ عَلَى ٱلْأَرْضِ هَوْنًۭا وَإِذَا خَاطَبَهُمُ ٱلْجَٰهِلُونَ قَالُوا۟ سَلَـٰمًۭا ﴿٦٣﴾}\\
64.\  & \mytextarabic{وَٱلَّذِينَ يَبِيتُونَ لِرَبِّهِمْ سُجَّدًۭا وَقِيَـٰمًۭا ﴿٦٤﴾}\\
65.\  & \mytextarabic{وَٱلَّذِينَ يَقُولُونَ رَبَّنَا ٱصْرِفْ عَنَّا عَذَابَ جَهَنَّمَ ۖ إِنَّ عَذَابَهَا كَانَ غَرَامًا ﴿٦٥﴾}\\
66.\  & \mytextarabic{إِنَّهَا سَآءَتْ مُسْتَقَرًّۭا وَمُقَامًۭا ﴿٦٦﴾}\\
67.\  & \mytextarabic{وَٱلَّذِينَ إِذَآ أَنفَقُوا۟ لَمْ يُسْرِفُوا۟ وَلَمْ يَقْتُرُوا۟ وَكَانَ بَيْنَ ذَٟلِكَ قَوَامًۭا ﴿٦٧﴾}\\
68.\  & \mytextarabic{وَٱلَّذِينَ لَا يَدْعُونَ مَعَ ٱللَّهِ إِلَـٰهًا ءَاخَرَ وَلَا يَقْتُلُونَ ٱلنَّفْسَ ٱلَّتِى حَرَّمَ ٱللَّهُ إِلَّا بِٱلْحَقِّ وَلَا يَزْنُونَ ۚ وَمَن يَفْعَلْ ذَٟلِكَ يَلْقَ أَثَامًۭا ﴿٦٨﴾}\\
69.\  & \mytextarabic{يُضَٰعَفْ لَهُ ٱلْعَذَابُ يَوْمَ ٱلْقِيَـٰمَةِ وَيَخْلُدْ فِيهِۦ مُهَانًا ﴿٦٩﴾}\\
70.\  & \mytextarabic{إِلَّا مَن تَابَ وَءَامَنَ وَعَمِلَ عَمَلًۭا صَـٰلِحًۭا فَأُو۟لَـٰٓئِكَ يُبَدِّلُ ٱللَّهُ سَيِّـَٔاتِهِمْ حَسَنَـٰتٍۢ ۗ وَكَانَ ٱللَّهُ غَفُورًۭا رَّحِيمًۭا ﴿٧٠﴾}\\
71.\  & \mytextarabic{وَمَن تَابَ وَعَمِلَ صَـٰلِحًۭا فَإِنَّهُۥ يَتُوبُ إِلَى ٱللَّهِ مَتَابًۭا ﴿٧١﴾}\\
72.\  & \mytextarabic{وَٱلَّذِينَ لَا يَشْهَدُونَ ٱلزُّورَ وَإِذَا مَرُّوا۟ بِٱللَّغْوِ مَرُّوا۟ كِرَامًۭا ﴿٧٢﴾}\\
73.\  & \mytextarabic{وَٱلَّذِينَ إِذَا ذُكِّرُوا۟ بِـَٔايَـٰتِ رَبِّهِمْ لَمْ يَخِرُّوا۟ عَلَيْهَا صُمًّۭا وَعُمْيَانًۭا ﴿٧٣﴾}\\
74.\  & \mytextarabic{وَٱلَّذِينَ يَقُولُونَ رَبَّنَا هَبْ لَنَا مِنْ أَزْوَٟجِنَا وَذُرِّيَّٰتِنَا قُرَّةَ أَعْيُنٍۢ وَٱجْعَلْنَا لِلْمُتَّقِينَ إِمَامًا ﴿٧٤﴾}\\
75.\  & \mytextarabic{أُو۟لَـٰٓئِكَ يُجْزَوْنَ ٱلْغُرْفَةَ بِمَا صَبَرُوا۟ وَيُلَقَّوْنَ فِيهَا تَحِيَّةًۭ وَسَلَـٰمًا ﴿٧٥﴾}\\
76.\  & \mytextarabic{خَـٰلِدِينَ فِيهَا ۚ حَسُنَتْ مُسْتَقَرًّۭا وَمُقَامًۭا ﴿٧٦﴾}\\
77.\  & \mytextarabic{قُلْ مَا يَعْبَؤُا۟ بِكُمْ رَبِّى لَوْلَا دُعَآؤُكُمْ ۖ فَقَدْ كَذَّبْتُمْ فَسَوْفَ يَكُونُ لِزَامًۢا ﴿٧٧﴾}\\
\end{longtable}
\clearpage