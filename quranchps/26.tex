%% License: BSD style (Berkley) (i.e. Put the Copyright owner's name always)
%% Writer and Copyright (to): Bewketu(Bilal) Tadilo (2016-17)
\begin{center}\section{ሱራቱ አሹኣራኣ -  \textarabic{سوره  الشعراء}}\end{center}
\begin{longtable}{%
  @{}
    p{.5\textwidth}
  @{~~~}
    p{.5\textwidth}
    @{}
}
ቢስሚላሂ አራህመኒ ራሂይም &  \mytextarabic{بِسْمِ ٱللَّهِ ٱلرَّحْمَـٰنِ ٱلرَّحِيمِ}\\
1.\  & \mytextarabic{ طسٓمٓ ﴿١﴾}\\
2.\  & \mytextarabic{تِلْكَ ءَايَـٰتُ ٱلْكِتَـٰبِ ٱلْمُبِينِ ﴿٢﴾}\\
3.\  & \mytextarabic{لَعَلَّكَ بَٰخِعٌۭ نَّفْسَكَ أَلَّا يَكُونُوا۟ مُؤْمِنِينَ ﴿٣﴾}\\
4.\  & \mytextarabic{إِن نَّشَأْ نُنَزِّلْ عَلَيْهِم مِّنَ ٱلسَّمَآءِ ءَايَةًۭ فَظَلَّتْ أَعْنَـٰقُهُمْ لَهَا خَـٰضِعِينَ ﴿٤﴾}\\
5.\  & \mytextarabic{وَمَا يَأْتِيهِم مِّن ذِكْرٍۢ مِّنَ ٱلرَّحْمَـٰنِ مُحْدَثٍ إِلَّا كَانُوا۟ عَنْهُ مُعْرِضِينَ ﴿٥﴾}\\
6.\  & \mytextarabic{فَقَدْ كَذَّبُوا۟ فَسَيَأْتِيهِمْ أَنۢبَٰٓؤُا۟ مَا كَانُوا۟ بِهِۦ يَسْتَهْزِءُونَ ﴿٦﴾}\\
7.\  & \mytextarabic{أَوَلَمْ يَرَوْا۟ إِلَى ٱلْأَرْضِ كَمْ أَنۢبَتْنَا فِيهَا مِن كُلِّ زَوْجٍۢ كَرِيمٍ ﴿٧﴾}\\
8.\  & \mytextarabic{إِنَّ فِى ذَٟلِكَ لَءَايَةًۭ ۖ وَمَا كَانَ أَكْثَرُهُم مُّؤْمِنِينَ ﴿٨﴾}\\
9.\  & \mytextarabic{وَإِنَّ رَبَّكَ لَهُوَ ٱلْعَزِيزُ ٱلرَّحِيمُ ﴿٩﴾}\\
10.\  & \mytextarabic{وَإِذْ نَادَىٰ رَبُّكَ مُوسَىٰٓ أَنِ ٱئْتِ ٱلْقَوْمَ ٱلظَّـٰلِمِينَ ﴿١٠﴾}\\
11.\  & \mytextarabic{قَوْمَ فِرْعَوْنَ ۚ أَلَا يَتَّقُونَ ﴿١١﴾}\\
12.\  & \mytextarabic{قَالَ رَبِّ إِنِّىٓ أَخَافُ أَن يُكَذِّبُونِ ﴿١٢﴾}\\
13.\  & \mytextarabic{وَيَضِيقُ صَدْرِى وَلَا يَنطَلِقُ لِسَانِى فَأَرْسِلْ إِلَىٰ هَـٰرُونَ ﴿١٣﴾}\\
14.\  & \mytextarabic{وَلَهُمْ عَلَىَّ ذَنۢبٌۭ فَأَخَافُ أَن يَقْتُلُونِ ﴿١٤﴾}\\
15.\  & \mytextarabic{قَالَ كَلَّا ۖ فَٱذْهَبَا بِـَٔايَـٰتِنَآ ۖ إِنَّا مَعَكُم مُّسْتَمِعُونَ ﴿١٥﴾}\\
16.\  & \mytextarabic{فَأْتِيَا فِرْعَوْنَ فَقُولَآ إِنَّا رَسُولُ رَبِّ ٱلْعَـٰلَمِينَ ﴿١٦﴾}\\
17.\  & \mytextarabic{أَنْ أَرْسِلْ مَعَنَا بَنِىٓ إِسْرَٰٓءِيلَ ﴿١٧﴾}\\
18.\  & \mytextarabic{قَالَ أَلَمْ نُرَبِّكَ فِينَا وَلِيدًۭا وَلَبِثْتَ فِينَا مِنْ عُمُرِكَ سِنِينَ ﴿١٨﴾}\\
19.\  & \mytextarabic{وَفَعَلْتَ فَعْلَتَكَ ٱلَّتِى فَعَلْتَ وَأَنتَ مِنَ ٱلْكَـٰفِرِينَ ﴿١٩﴾}\\
20.\  & \mytextarabic{قَالَ فَعَلْتُهَآ إِذًۭا وَأَنَا۠ مِنَ ٱلضَّآلِّينَ ﴿٢٠﴾}\\
21.\  & \mytextarabic{فَفَرَرْتُ مِنكُمْ لَمَّا خِفْتُكُمْ فَوَهَبَ لِى رَبِّى حُكْمًۭا وَجَعَلَنِى مِنَ ٱلْمُرْسَلِينَ ﴿٢١﴾}\\
22.\  & \mytextarabic{وَتِلْكَ نِعْمَةٌۭ تَمُنُّهَا عَلَىَّ أَنْ عَبَّدتَّ بَنِىٓ إِسْرَٰٓءِيلَ ﴿٢٢﴾}\\
23.\  & \mytextarabic{قَالَ فِرْعَوْنُ وَمَا رَبُّ ٱلْعَـٰلَمِينَ ﴿٢٣﴾}\\
24.\  & \mytextarabic{قَالَ رَبُّ ٱلسَّمَـٰوَٟتِ وَٱلْأَرْضِ وَمَا بَيْنَهُمَآ ۖ إِن كُنتُم مُّوقِنِينَ ﴿٢٤﴾}\\
25.\  & \mytextarabic{قَالَ لِمَنْ حَوْلَهُۥٓ أَلَا تَسْتَمِعُونَ ﴿٢٥﴾}\\
26.\  & \mytextarabic{قَالَ رَبُّكُمْ وَرَبُّ ءَابَآئِكُمُ ٱلْأَوَّلِينَ ﴿٢٦﴾}\\
27.\  & \mytextarabic{قَالَ إِنَّ رَسُولَكُمُ ٱلَّذِىٓ أُرْسِلَ إِلَيْكُمْ لَمَجْنُونٌۭ ﴿٢٧﴾}\\
28.\  & \mytextarabic{قَالَ رَبُّ ٱلْمَشْرِقِ وَٱلْمَغْرِبِ وَمَا بَيْنَهُمَآ ۖ إِن كُنتُمْ تَعْقِلُونَ ﴿٢٨﴾}\\
29.\  & \mytextarabic{قَالَ لَئِنِ ٱتَّخَذْتَ إِلَـٰهًا غَيْرِى لَأَجْعَلَنَّكَ مِنَ ٱلْمَسْجُونِينَ ﴿٢٩﴾}\\
30.\  & \mytextarabic{قَالَ أَوَلَوْ جِئْتُكَ بِشَىْءٍۢ مُّبِينٍۢ ﴿٣٠﴾}\\
31.\  & \mytextarabic{قَالَ فَأْتِ بِهِۦٓ إِن كُنتَ مِنَ ٱلصَّـٰدِقِينَ ﴿٣١﴾}\\
32.\  & \mytextarabic{فَأَلْقَىٰ عَصَاهُ فَإِذَا هِىَ ثُعْبَانٌۭ مُّبِينٌۭ ﴿٣٢﴾}\\
33.\  & \mytextarabic{وَنَزَعَ يَدَهُۥ فَإِذَا هِىَ بَيْضَآءُ لِلنَّـٰظِرِينَ ﴿٣٣﴾}\\
34.\  & \mytextarabic{قَالَ لِلْمَلَإِ حَوْلَهُۥٓ إِنَّ هَـٰذَا لَسَـٰحِرٌ عَلِيمٌۭ ﴿٣٤﴾}\\
35.\  & \mytextarabic{يُرِيدُ أَن يُخْرِجَكُم مِّنْ أَرْضِكُم بِسِحْرِهِۦ فَمَاذَا تَأْمُرُونَ ﴿٣٥﴾}\\
36.\  & \mytextarabic{قَالُوٓا۟ أَرْجِهْ وَأَخَاهُ وَٱبْعَثْ فِى ٱلْمَدَآئِنِ حَـٰشِرِينَ ﴿٣٦﴾}\\
37.\  & \mytextarabic{يَأْتُوكَ بِكُلِّ سَحَّارٍ عَلِيمٍۢ ﴿٣٧﴾}\\
38.\  & \mytextarabic{فَجُمِعَ ٱلسَّحَرَةُ لِمِيقَـٰتِ يَوْمٍۢ مَّعْلُومٍۢ ﴿٣٨﴾}\\
39.\  & \mytextarabic{وَقِيلَ لِلنَّاسِ هَلْ أَنتُم مُّجْتَمِعُونَ ﴿٣٩﴾}\\
40.\  & \mytextarabic{لَعَلَّنَا نَتَّبِعُ ٱلسَّحَرَةَ إِن كَانُوا۟ هُمُ ٱلْغَٰلِبِينَ ﴿٤٠﴾}\\
41.\  & \mytextarabic{فَلَمَّا جَآءَ ٱلسَّحَرَةُ قَالُوا۟ لِفِرْعَوْنَ أَئِنَّ لَنَا لَأَجْرًا إِن كُنَّا نَحْنُ ٱلْغَٰلِبِينَ ﴿٤١﴾}\\
42.\  & \mytextarabic{قَالَ نَعَمْ وَإِنَّكُمْ إِذًۭا لَّمِنَ ٱلْمُقَرَّبِينَ ﴿٤٢﴾}\\
43.\  & \mytextarabic{قَالَ لَهُم مُّوسَىٰٓ أَلْقُوا۟ مَآ أَنتُم مُّلْقُونَ ﴿٤٣﴾}\\
44.\  & \mytextarabic{فَأَلْقَوْا۟ حِبَالَهُمْ وَعِصِيَّهُمْ وَقَالُوا۟ بِعِزَّةِ فِرْعَوْنَ إِنَّا لَنَحْنُ ٱلْغَٰلِبُونَ ﴿٤٤﴾}\\
45.\  & \mytextarabic{فَأَلْقَىٰ مُوسَىٰ عَصَاهُ فَإِذَا هِىَ تَلْقَفُ مَا يَأْفِكُونَ ﴿٤٥﴾}\\
46.\  & \mytextarabic{فَأُلْقِىَ ٱلسَّحَرَةُ سَـٰجِدِينَ ﴿٤٦﴾}\\
47.\  & \mytextarabic{قَالُوٓا۟ ءَامَنَّا بِرَبِّ ٱلْعَـٰلَمِينَ ﴿٤٧﴾}\\
48.\  & \mytextarabic{رَبِّ مُوسَىٰ وَهَـٰرُونَ ﴿٤٨﴾}\\
49.\  & \mytextarabic{قَالَ ءَامَنتُمْ لَهُۥ قَبْلَ أَنْ ءَاذَنَ لَكُمْ ۖ إِنَّهُۥ لَكَبِيرُكُمُ ٱلَّذِى عَلَّمَكُمُ ٱلسِّحْرَ فَلَسَوْفَ تَعْلَمُونَ ۚ لَأُقَطِّعَنَّ أَيْدِيَكُمْ وَأَرْجُلَكُم مِّنْ خِلَـٰفٍۢ وَلَأُصَلِّبَنَّكُمْ أَجْمَعِينَ ﴿٤٩﴾}\\
50.\  & \mytextarabic{قَالُوا۟ لَا ضَيْرَ ۖ إِنَّآ إِلَىٰ رَبِّنَا مُنقَلِبُونَ ﴿٥٠﴾}\\
51.\  & \mytextarabic{إِنَّا نَطْمَعُ أَن يَغْفِرَ لَنَا رَبُّنَا خَطَٰيَـٰنَآ أَن كُنَّآ أَوَّلَ ٱلْمُؤْمِنِينَ ﴿٥١﴾}\\
52.\  & \mytextarabic{۞ وَأَوْحَيْنَآ إِلَىٰ مُوسَىٰٓ أَنْ أَسْرِ بِعِبَادِىٓ إِنَّكُم مُّتَّبَعُونَ ﴿٥٢﴾}\\
53.\  & \mytextarabic{فَأَرْسَلَ فِرْعَوْنُ فِى ٱلْمَدَآئِنِ حَـٰشِرِينَ ﴿٥٣﴾}\\
54.\  & \mytextarabic{إِنَّ هَـٰٓؤُلَآءِ لَشِرْذِمَةٌۭ قَلِيلُونَ ﴿٥٤﴾}\\
55.\  & \mytextarabic{وَإِنَّهُمْ لَنَا لَغَآئِظُونَ ﴿٥٥﴾}\\
56.\  & \mytextarabic{وَإِنَّا لَجَمِيعٌ حَـٰذِرُونَ ﴿٥٦﴾}\\
57.\  & \mytextarabic{فَأَخْرَجْنَـٰهُم مِّن جَنَّـٰتٍۢ وَعُيُونٍۢ ﴿٥٧﴾}\\
58.\  & \mytextarabic{وَكُنُوزٍۢ وَمَقَامٍۢ كَرِيمٍۢ ﴿٥٨﴾}\\
59.\  & \mytextarabic{كَذَٟلِكَ وَأَوْرَثْنَـٰهَا بَنِىٓ إِسْرَٰٓءِيلَ ﴿٥٩﴾}\\
60.\  & \mytextarabic{فَأَتْبَعُوهُم مُّشْرِقِينَ ﴿٦٠﴾}\\
61.\  & \mytextarabic{فَلَمَّا تَرَٰٓءَا ٱلْجَمْعَانِ قَالَ أَصْحَـٰبُ مُوسَىٰٓ إِنَّا لَمُدْرَكُونَ ﴿٦١﴾}\\
62.\  & \mytextarabic{قَالَ كَلَّآ ۖ إِنَّ مَعِىَ رَبِّى سَيَهْدِينِ ﴿٦٢﴾}\\
63.\  & \mytextarabic{فَأَوْحَيْنَآ إِلَىٰ مُوسَىٰٓ أَنِ ٱضْرِب بِّعَصَاكَ ٱلْبَحْرَ ۖ فَٱنفَلَقَ فَكَانَ كُلُّ فِرْقٍۢ كَٱلطَّوْدِ ٱلْعَظِيمِ ﴿٦٣﴾}\\
64.\  & \mytextarabic{وَأَزْلَفْنَا ثَمَّ ٱلْءَاخَرِينَ ﴿٦٤﴾}\\
65.\  & \mytextarabic{وَأَنجَيْنَا مُوسَىٰ وَمَن مَّعَهُۥٓ أَجْمَعِينَ ﴿٦٥﴾}\\
66.\  & \mytextarabic{ثُمَّ أَغْرَقْنَا ٱلْءَاخَرِينَ ﴿٦٦﴾}\\
67.\  & \mytextarabic{إِنَّ فِى ذَٟلِكَ لَءَايَةًۭ ۖ وَمَا كَانَ أَكْثَرُهُم مُّؤْمِنِينَ ﴿٦٧﴾}\\
68.\  & \mytextarabic{وَإِنَّ رَبَّكَ لَهُوَ ٱلْعَزِيزُ ٱلرَّحِيمُ ﴿٦٨﴾}\\
69.\  & \mytextarabic{وَٱتْلُ عَلَيْهِمْ نَبَأَ إِبْرَٰهِيمَ ﴿٦٩﴾}\\
70.\  & \mytextarabic{إِذْ قَالَ لِأَبِيهِ وَقَوْمِهِۦ مَا تَعْبُدُونَ ﴿٧٠﴾}\\
71.\  & \mytextarabic{قَالُوا۟ نَعْبُدُ أَصْنَامًۭا فَنَظَلُّ لَهَا عَـٰكِفِينَ ﴿٧١﴾}\\
72.\  & \mytextarabic{قَالَ هَلْ يَسْمَعُونَكُمْ إِذْ تَدْعُونَ ﴿٧٢﴾}\\
73.\  & \mytextarabic{أَوْ يَنفَعُونَكُمْ أَوْ يَضُرُّونَ ﴿٧٣﴾}\\
74.\  & \mytextarabic{قَالُوا۟ بَلْ وَجَدْنَآ ءَابَآءَنَا كَذَٟلِكَ يَفْعَلُونَ ﴿٧٤﴾}\\
75.\  & \mytextarabic{قَالَ أَفَرَءَيْتُم مَّا كُنتُمْ تَعْبُدُونَ ﴿٧٥﴾}\\
76.\  & \mytextarabic{أَنتُمْ وَءَابَآؤُكُمُ ٱلْأَقْدَمُونَ ﴿٧٦﴾}\\
77.\  & \mytextarabic{فَإِنَّهُمْ عَدُوٌّۭ لِّىٓ إِلَّا رَبَّ ٱلْعَـٰلَمِينَ ﴿٧٧﴾}\\
78.\  & \mytextarabic{ٱلَّذِى خَلَقَنِى فَهُوَ يَهْدِينِ ﴿٧٨﴾}\\
79.\  & \mytextarabic{وَٱلَّذِى هُوَ يُطْعِمُنِى وَيَسْقِينِ ﴿٧٩﴾}\\
80.\  & \mytextarabic{وَإِذَا مَرِضْتُ فَهُوَ يَشْفِينِ ﴿٨٠﴾}\\
81.\  & \mytextarabic{وَٱلَّذِى يُمِيتُنِى ثُمَّ يُحْيِينِ ﴿٨١﴾}\\
82.\  & \mytextarabic{وَٱلَّذِىٓ أَطْمَعُ أَن يَغْفِرَ لِى خَطِيٓـَٔتِى يَوْمَ ٱلدِّينِ ﴿٨٢﴾}\\
83.\  & \mytextarabic{رَبِّ هَبْ لِى حُكْمًۭا وَأَلْحِقْنِى بِٱلصَّـٰلِحِينَ ﴿٨٣﴾}\\
84.\  & \mytextarabic{وَٱجْعَل لِّى لِسَانَ صِدْقٍۢ فِى ٱلْءَاخِرِينَ ﴿٨٤﴾}\\
85.\  & \mytextarabic{وَٱجْعَلْنِى مِن وَرَثَةِ جَنَّةِ ٱلنَّعِيمِ ﴿٨٥﴾}\\
86.\  & \mytextarabic{وَٱغْفِرْ لِأَبِىٓ إِنَّهُۥ كَانَ مِنَ ٱلضَّآلِّينَ ﴿٨٦﴾}\\
87.\  & \mytextarabic{وَلَا تُخْزِنِى يَوْمَ يُبْعَثُونَ ﴿٨٧﴾}\\
88.\  & \mytextarabic{يَوْمَ لَا يَنفَعُ مَالٌۭ وَلَا بَنُونَ ﴿٨٨﴾}\\
89.\  & \mytextarabic{إِلَّا مَنْ أَتَى ٱللَّهَ بِقَلْبٍۢ سَلِيمٍۢ ﴿٨٩﴾}\\
90.\  & \mytextarabic{وَأُزْلِفَتِ ٱلْجَنَّةُ لِلْمُتَّقِينَ ﴿٩٠﴾}\\
91.\  & \mytextarabic{وَبُرِّزَتِ ٱلْجَحِيمُ لِلْغَاوِينَ ﴿٩١﴾}\\
92.\  & \mytextarabic{وَقِيلَ لَهُمْ أَيْنَ مَا كُنتُمْ تَعْبُدُونَ ﴿٩٢﴾}\\
93.\  & \mytextarabic{مِن دُونِ ٱللَّهِ هَلْ يَنصُرُونَكُمْ أَوْ يَنتَصِرُونَ ﴿٩٣﴾}\\
94.\  & \mytextarabic{فَكُبْكِبُوا۟ فِيهَا هُمْ وَٱلْغَاوُۥنَ ﴿٩٤﴾}\\
95.\  & \mytextarabic{وَجُنُودُ إِبْلِيسَ أَجْمَعُونَ ﴿٩٥﴾}\\
96.\  & \mytextarabic{قَالُوا۟ وَهُمْ فِيهَا يَخْتَصِمُونَ ﴿٩٦﴾}\\
97.\  & \mytextarabic{تَٱللَّهِ إِن كُنَّا لَفِى ضَلَـٰلٍۢ مُّبِينٍ ﴿٩٧﴾}\\
98.\  & \mytextarabic{إِذْ نُسَوِّيكُم بِرَبِّ ٱلْعَـٰلَمِينَ ﴿٩٨﴾}\\
99.\  & \mytextarabic{وَمَآ أَضَلَّنَآ إِلَّا ٱلْمُجْرِمُونَ ﴿٩٩﴾}\\
100.\  & \mytextarabic{فَمَا لَنَا مِن شَـٰفِعِينَ ﴿١٠٠﴾}\\
101.\  & \mytextarabic{وَلَا صَدِيقٍ حَمِيمٍۢ ﴿١٠١﴾}\\
102.\  & \mytextarabic{فَلَوْ أَنَّ لَنَا كَرَّةًۭ فَنَكُونَ مِنَ ٱلْمُؤْمِنِينَ ﴿١٠٢﴾}\\
103.\  & \mytextarabic{إِنَّ فِى ذَٟلِكَ لَءَايَةًۭ ۖ وَمَا كَانَ أَكْثَرُهُم مُّؤْمِنِينَ ﴿١٠٣﴾}\\
104.\  & \mytextarabic{وَإِنَّ رَبَّكَ لَهُوَ ٱلْعَزِيزُ ٱلرَّحِيمُ ﴿١٠٤﴾}\\
105.\  & \mytextarabic{كَذَّبَتْ قَوْمُ نُوحٍ ٱلْمُرْسَلِينَ ﴿١٠٥﴾}\\
106.\  & \mytextarabic{إِذْ قَالَ لَهُمْ أَخُوهُمْ نُوحٌ أَلَا تَتَّقُونَ ﴿١٠٦﴾}\\
107.\  & \mytextarabic{إِنِّى لَكُمْ رَسُولٌ أَمِينٌۭ ﴿١٠٧﴾}\\
108.\  & \mytextarabic{فَٱتَّقُوا۟ ٱللَّهَ وَأَطِيعُونِ ﴿١٠٨﴾}\\
109.\  & \mytextarabic{وَمَآ أَسْـَٔلُكُمْ عَلَيْهِ مِنْ أَجْرٍ ۖ إِنْ أَجْرِىَ إِلَّا عَلَىٰ رَبِّ ٱلْعَـٰلَمِينَ ﴿١٠٩﴾}\\
110.\  & \mytextarabic{فَٱتَّقُوا۟ ٱللَّهَ وَأَطِيعُونِ ﴿١١٠﴾}\\
111.\  & \mytextarabic{۞ قَالُوٓا۟ أَنُؤْمِنُ لَكَ وَٱتَّبَعَكَ ٱلْأَرْذَلُونَ ﴿١١١﴾}\\
112.\  & \mytextarabic{قَالَ وَمَا عِلْمِى بِمَا كَانُوا۟ يَعْمَلُونَ ﴿١١٢﴾}\\
113.\  & \mytextarabic{إِنْ حِسَابُهُمْ إِلَّا عَلَىٰ رَبِّى ۖ لَوْ تَشْعُرُونَ ﴿١١٣﴾}\\
114.\  & \mytextarabic{وَمَآ أَنَا۠ بِطَارِدِ ٱلْمُؤْمِنِينَ ﴿١١٤﴾}\\
115.\  & \mytextarabic{إِنْ أَنَا۠ إِلَّا نَذِيرٌۭ مُّبِينٌۭ ﴿١١٥﴾}\\
116.\  & \mytextarabic{قَالُوا۟ لَئِن لَّمْ تَنتَهِ يَـٰنُوحُ لَتَكُونَنَّ مِنَ ٱلْمَرْجُومِينَ ﴿١١٦﴾}\\
117.\  & \mytextarabic{قَالَ رَبِّ إِنَّ قَوْمِى كَذَّبُونِ ﴿١١٧﴾}\\
118.\  & \mytextarabic{فَٱفْتَحْ بَيْنِى وَبَيْنَهُمْ فَتْحًۭا وَنَجِّنِى وَمَن مَّعِىَ مِنَ ٱلْمُؤْمِنِينَ ﴿١١٨﴾}\\
119.\  & \mytextarabic{فَأَنجَيْنَـٰهُ وَمَن مَّعَهُۥ فِى ٱلْفُلْكِ ٱلْمَشْحُونِ ﴿١١٩﴾}\\
120.\  & \mytextarabic{ثُمَّ أَغْرَقْنَا بَعْدُ ٱلْبَاقِينَ ﴿١٢٠﴾}\\
121.\  & \mytextarabic{إِنَّ فِى ذَٟلِكَ لَءَايَةًۭ ۖ وَمَا كَانَ أَكْثَرُهُم مُّؤْمِنِينَ ﴿١٢١﴾}\\
122.\  & \mytextarabic{وَإِنَّ رَبَّكَ لَهُوَ ٱلْعَزِيزُ ٱلرَّحِيمُ ﴿١٢٢﴾}\\
123.\  & \mytextarabic{كَذَّبَتْ عَادٌ ٱلْمُرْسَلِينَ ﴿١٢٣﴾}\\
124.\  & \mytextarabic{إِذْ قَالَ لَهُمْ أَخُوهُمْ هُودٌ أَلَا تَتَّقُونَ ﴿١٢٤﴾}\\
125.\  & \mytextarabic{إِنِّى لَكُمْ رَسُولٌ أَمِينٌۭ ﴿١٢٥﴾}\\
126.\  & \mytextarabic{فَٱتَّقُوا۟ ٱللَّهَ وَأَطِيعُونِ ﴿١٢٦﴾}\\
127.\  & \mytextarabic{وَمَآ أَسْـَٔلُكُمْ عَلَيْهِ مِنْ أَجْرٍ ۖ إِنْ أَجْرِىَ إِلَّا عَلَىٰ رَبِّ ٱلْعَـٰلَمِينَ ﴿١٢٧﴾}\\
128.\  & \mytextarabic{أَتَبْنُونَ بِكُلِّ رِيعٍ ءَايَةًۭ تَعْبَثُونَ ﴿١٢٨﴾}\\
129.\  & \mytextarabic{وَتَتَّخِذُونَ مَصَانِعَ لَعَلَّكُمْ تَخْلُدُونَ ﴿١٢٩﴾}\\
130.\  & \mytextarabic{وَإِذَا بَطَشْتُم بَطَشْتُمْ جَبَّارِينَ ﴿١٣٠﴾}\\
131.\  & \mytextarabic{فَٱتَّقُوا۟ ٱللَّهَ وَأَطِيعُونِ ﴿١٣١﴾}\\
132.\  & \mytextarabic{وَٱتَّقُوا۟ ٱلَّذِىٓ أَمَدَّكُم بِمَا تَعْلَمُونَ ﴿١٣٢﴾}\\
133.\  & \mytextarabic{أَمَدَّكُم بِأَنْعَـٰمٍۢ وَبَنِينَ ﴿١٣٣﴾}\\
134.\  & \mytextarabic{وَجَنَّـٰتٍۢ وَعُيُونٍ ﴿١٣٤﴾}\\
135.\  & \mytextarabic{إِنِّىٓ أَخَافُ عَلَيْكُمْ عَذَابَ يَوْمٍ عَظِيمٍۢ ﴿١٣٥﴾}\\
136.\  & \mytextarabic{قَالُوا۟ سَوَآءٌ عَلَيْنَآ أَوَعَظْتَ أَمْ لَمْ تَكُن مِّنَ ٱلْوَٟعِظِينَ ﴿١٣٦﴾}\\
137.\  & \mytextarabic{إِنْ هَـٰذَآ إِلَّا خُلُقُ ٱلْأَوَّلِينَ ﴿١٣٧﴾}\\
138.\  & \mytextarabic{وَمَا نَحْنُ بِمُعَذَّبِينَ ﴿١٣٨﴾}\\
139.\  & \mytextarabic{فَكَذَّبُوهُ فَأَهْلَكْنَـٰهُمْ ۗ إِنَّ فِى ذَٟلِكَ لَءَايَةًۭ ۖ وَمَا كَانَ أَكْثَرُهُم مُّؤْمِنِينَ ﴿١٣٩﴾}\\
140.\  & \mytextarabic{وَإِنَّ رَبَّكَ لَهُوَ ٱلْعَزِيزُ ٱلرَّحِيمُ ﴿١٤٠﴾}\\
141.\  & \mytextarabic{كَذَّبَتْ ثَمُودُ ٱلْمُرْسَلِينَ ﴿١٤١﴾}\\
142.\  & \mytextarabic{إِذْ قَالَ لَهُمْ أَخُوهُمْ صَـٰلِحٌ أَلَا تَتَّقُونَ ﴿١٤٢﴾}\\
143.\  & \mytextarabic{إِنِّى لَكُمْ رَسُولٌ أَمِينٌۭ ﴿١٤٣﴾}\\
144.\  & \mytextarabic{فَٱتَّقُوا۟ ٱللَّهَ وَأَطِيعُونِ ﴿١٤٤﴾}\\
145.\  & \mytextarabic{وَمَآ أَسْـَٔلُكُمْ عَلَيْهِ مِنْ أَجْرٍ ۖ إِنْ أَجْرِىَ إِلَّا عَلَىٰ رَبِّ ٱلْعَـٰلَمِينَ ﴿١٤٥﴾}\\
146.\  & \mytextarabic{أَتُتْرَكُونَ فِى مَا هَـٰهُنَآ ءَامِنِينَ ﴿١٤٦﴾}\\
147.\  & \mytextarabic{فِى جَنَّـٰتٍۢ وَعُيُونٍۢ ﴿١٤٧﴾}\\
148.\  & \mytextarabic{وَزُرُوعٍۢ وَنَخْلٍۢ طَلْعُهَا هَضِيمٌۭ ﴿١٤٨﴾}\\
149.\  & \mytextarabic{وَتَنْحِتُونَ مِنَ ٱلْجِبَالِ بُيُوتًۭا فَـٰرِهِينَ ﴿١٤٩﴾}\\
150.\  & \mytextarabic{فَٱتَّقُوا۟ ٱللَّهَ وَأَطِيعُونِ ﴿١٥٠﴾}\\
151.\  & \mytextarabic{وَلَا تُطِيعُوٓا۟ أَمْرَ ٱلْمُسْرِفِينَ ﴿١٥١﴾}\\
152.\  & \mytextarabic{ٱلَّذِينَ يُفْسِدُونَ فِى ٱلْأَرْضِ وَلَا يُصْلِحُونَ ﴿١٥٢﴾}\\
153.\  & \mytextarabic{قَالُوٓا۟ إِنَّمَآ أَنتَ مِنَ ٱلْمُسَحَّرِينَ ﴿١٥٣﴾}\\
154.\  & \mytextarabic{مَآ أَنتَ إِلَّا بَشَرٌۭ مِّثْلُنَا فَأْتِ بِـَٔايَةٍ إِن كُنتَ مِنَ ٱلصَّـٰدِقِينَ ﴿١٥٤﴾}\\
155.\  & \mytextarabic{قَالَ هَـٰذِهِۦ نَاقَةٌۭ لَّهَا شِرْبٌۭ وَلَكُمْ شِرْبُ يَوْمٍۢ مَّعْلُومٍۢ ﴿١٥٥﴾}\\
156.\  & \mytextarabic{وَلَا تَمَسُّوهَا بِسُوٓءٍۢ فَيَأْخُذَكُمْ عَذَابُ يَوْمٍ عَظِيمٍۢ ﴿١٥٦﴾}\\
157.\  & \mytextarabic{فَعَقَرُوهَا فَأَصْبَحُوا۟ نَـٰدِمِينَ ﴿١٥٧﴾}\\
158.\  & \mytextarabic{فَأَخَذَهُمُ ٱلْعَذَابُ ۗ إِنَّ فِى ذَٟلِكَ لَءَايَةًۭ ۖ وَمَا كَانَ أَكْثَرُهُم مُّؤْمِنِينَ ﴿١٥٨﴾}\\
159.\  & \mytextarabic{وَإِنَّ رَبَّكَ لَهُوَ ٱلْعَزِيزُ ٱلرَّحِيمُ ﴿١٥٩﴾}\\
160.\  & \mytextarabic{كَذَّبَتْ قَوْمُ لُوطٍ ٱلْمُرْسَلِينَ ﴿١٦٠﴾}\\
161.\  & \mytextarabic{إِذْ قَالَ لَهُمْ أَخُوهُمْ لُوطٌ أَلَا تَتَّقُونَ ﴿١٦١﴾}\\
162.\  & \mytextarabic{إِنِّى لَكُمْ رَسُولٌ أَمِينٌۭ ﴿١٦٢﴾}\\
163.\  & \mytextarabic{فَٱتَّقُوا۟ ٱللَّهَ وَأَطِيعُونِ ﴿١٦٣﴾}\\
164.\  & \mytextarabic{وَمَآ أَسْـَٔلُكُمْ عَلَيْهِ مِنْ أَجْرٍ ۖ إِنْ أَجْرِىَ إِلَّا عَلَىٰ رَبِّ ٱلْعَـٰلَمِينَ ﴿١٦٤﴾}\\
165.\  & \mytextarabic{أَتَأْتُونَ ٱلذُّكْرَانَ مِنَ ٱلْعَـٰلَمِينَ ﴿١٦٥﴾}\\
166.\  & \mytextarabic{وَتَذَرُونَ مَا خَلَقَ لَكُمْ رَبُّكُم مِّنْ أَزْوَٟجِكُم ۚ بَلْ أَنتُمْ قَوْمٌ عَادُونَ ﴿١٦٦﴾}\\
167.\  & \mytextarabic{قَالُوا۟ لَئِن لَّمْ تَنتَهِ يَـٰلُوطُ لَتَكُونَنَّ مِنَ ٱلْمُخْرَجِينَ ﴿١٦٧﴾}\\
168.\  & \mytextarabic{قَالَ إِنِّى لِعَمَلِكُم مِّنَ ٱلْقَالِينَ ﴿١٦٨﴾}\\
169.\  & \mytextarabic{رَبِّ نَجِّنِى وَأَهْلِى مِمَّا يَعْمَلُونَ ﴿١٦٩﴾}\\
170.\  & \mytextarabic{فَنَجَّيْنَـٰهُ وَأَهْلَهُۥٓ أَجْمَعِينَ ﴿١٧٠﴾}\\
171.\  & \mytextarabic{إِلَّا عَجُوزًۭا فِى ٱلْغَٰبِرِينَ ﴿١٧١﴾}\\
172.\  & \mytextarabic{ثُمَّ دَمَّرْنَا ٱلْءَاخَرِينَ ﴿١٧٢﴾}\\
173.\  & \mytextarabic{وَأَمْطَرْنَا عَلَيْهِم مَّطَرًۭا ۖ فَسَآءَ مَطَرُ ٱلْمُنذَرِينَ ﴿١٧٣﴾}\\
174.\  & \mytextarabic{إِنَّ فِى ذَٟلِكَ لَءَايَةًۭ ۖ وَمَا كَانَ أَكْثَرُهُم مُّؤْمِنِينَ ﴿١٧٤﴾}\\
175.\  & \mytextarabic{وَإِنَّ رَبَّكَ لَهُوَ ٱلْعَزِيزُ ٱلرَّحِيمُ ﴿١٧٥﴾}\\
176.\  & \mytextarabic{كَذَّبَ أَصْحَـٰبُ لْـَٔيْكَةِ ٱلْمُرْسَلِينَ ﴿١٧٦﴾}\\
177.\  & \mytextarabic{إِذْ قَالَ لَهُمْ شُعَيْبٌ أَلَا تَتَّقُونَ ﴿١٧٧﴾}\\
178.\  & \mytextarabic{إِنِّى لَكُمْ رَسُولٌ أَمِينٌۭ ﴿١٧٨﴾}\\
179.\  & \mytextarabic{فَٱتَّقُوا۟ ٱللَّهَ وَأَطِيعُونِ ﴿١٧٩﴾}\\
180.\  & \mytextarabic{وَمَآ أَسْـَٔلُكُمْ عَلَيْهِ مِنْ أَجْرٍ ۖ إِنْ أَجْرِىَ إِلَّا عَلَىٰ رَبِّ ٱلْعَـٰلَمِينَ ﴿١٨٠﴾}\\
181.\  & \mytextarabic{۞ أَوْفُوا۟ ٱلْكَيْلَ وَلَا تَكُونُوا۟ مِنَ ٱلْمُخْسِرِينَ ﴿١٨١﴾}\\
182.\  & \mytextarabic{وَزِنُوا۟ بِٱلْقِسْطَاسِ ٱلْمُسْتَقِيمِ ﴿١٨٢﴾}\\
183.\  & \mytextarabic{وَلَا تَبْخَسُوا۟ ٱلنَّاسَ أَشْيَآءَهُمْ وَلَا تَعْثَوْا۟ فِى ٱلْأَرْضِ مُفْسِدِينَ ﴿١٨٣﴾}\\
184.\  & \mytextarabic{وَٱتَّقُوا۟ ٱلَّذِى خَلَقَكُمْ وَٱلْجِبِلَّةَ ٱلْأَوَّلِينَ ﴿١٨٤﴾}\\
185.\  & \mytextarabic{قَالُوٓا۟ إِنَّمَآ أَنتَ مِنَ ٱلْمُسَحَّرِينَ ﴿١٨٥﴾}\\
186.\  & \mytextarabic{وَمَآ أَنتَ إِلَّا بَشَرٌۭ مِّثْلُنَا وَإِن نَّظُنُّكَ لَمِنَ ٱلْكَـٰذِبِينَ ﴿١٨٦﴾}\\
187.\  & \mytextarabic{فَأَسْقِطْ عَلَيْنَا كِسَفًۭا مِّنَ ٱلسَّمَآءِ إِن كُنتَ مِنَ ٱلصَّـٰدِقِينَ ﴿١٨٧﴾}\\
188.\  & \mytextarabic{قَالَ رَبِّىٓ أَعْلَمُ بِمَا تَعْمَلُونَ ﴿١٨٨﴾}\\
189.\  & \mytextarabic{فَكَذَّبُوهُ فَأَخَذَهُمْ عَذَابُ يَوْمِ ٱلظُّلَّةِ ۚ إِنَّهُۥ كَانَ عَذَابَ يَوْمٍ عَظِيمٍ ﴿١٨٩﴾}\\
190.\  & \mytextarabic{إِنَّ فِى ذَٟلِكَ لَءَايَةًۭ ۖ وَمَا كَانَ أَكْثَرُهُم مُّؤْمِنِينَ ﴿١٩٠﴾}\\
191.\  & \mytextarabic{وَإِنَّ رَبَّكَ لَهُوَ ٱلْعَزِيزُ ٱلرَّحِيمُ ﴿١٩١﴾}\\
192.\  & \mytextarabic{وَإِنَّهُۥ لَتَنزِيلُ رَبِّ ٱلْعَـٰلَمِينَ ﴿١٩٢﴾}\\
193.\  & \mytextarabic{نَزَلَ بِهِ ٱلرُّوحُ ٱلْأَمِينُ ﴿١٩٣﴾}\\
194.\  & \mytextarabic{عَلَىٰ قَلْبِكَ لِتَكُونَ مِنَ ٱلْمُنذِرِينَ ﴿١٩٤﴾}\\
195.\  & \mytextarabic{بِلِسَانٍ عَرَبِىٍّۢ مُّبِينٍۢ ﴿١٩٥﴾}\\
196.\  & \mytextarabic{وَإِنَّهُۥ لَفِى زُبُرِ ٱلْأَوَّلِينَ ﴿١٩٦﴾}\\
197.\  & \mytextarabic{أَوَلَمْ يَكُن لَّهُمْ ءَايَةً أَن يَعْلَمَهُۥ عُلَمَـٰٓؤُا۟ بَنِىٓ إِسْرَٰٓءِيلَ ﴿١٩٧﴾}\\
198.\  & \mytextarabic{وَلَوْ نَزَّلْنَـٰهُ عَلَىٰ بَعْضِ ٱلْأَعْجَمِينَ ﴿١٩٨﴾}\\
199.\  & \mytextarabic{فَقَرَأَهُۥ عَلَيْهِم مَّا كَانُوا۟ بِهِۦ مُؤْمِنِينَ ﴿١٩٩﴾}\\
200.\  & \mytextarabic{كَذَٟلِكَ سَلَكْنَـٰهُ فِى قُلُوبِ ٱلْمُجْرِمِينَ ﴿٢٠٠﴾}\\
201.\  & \mytextarabic{لَا يُؤْمِنُونَ بِهِۦ حَتَّىٰ يَرَوُا۟ ٱلْعَذَابَ ٱلْأَلِيمَ ﴿٢٠١﴾}\\
202.\  & \mytextarabic{فَيَأْتِيَهُم بَغْتَةًۭ وَهُمْ لَا يَشْعُرُونَ ﴿٢٠٢﴾}\\
203.\  & \mytextarabic{فَيَقُولُوا۟ هَلْ نَحْنُ مُنظَرُونَ ﴿٢٠٣﴾}\\
204.\  & \mytextarabic{أَفَبِعَذَابِنَا يَسْتَعْجِلُونَ ﴿٢٠٤﴾}\\
205.\  & \mytextarabic{أَفَرَءَيْتَ إِن مَّتَّعْنَـٰهُمْ سِنِينَ ﴿٢٠٥﴾}\\
206.\  & \mytextarabic{ثُمَّ جَآءَهُم مَّا كَانُوا۟ يُوعَدُونَ ﴿٢٠٦﴾}\\
207.\  & \mytextarabic{مَآ أَغْنَىٰ عَنْهُم مَّا كَانُوا۟ يُمَتَّعُونَ ﴿٢٠٧﴾}\\
208.\  & \mytextarabic{وَمَآ أَهْلَكْنَا مِن قَرْيَةٍ إِلَّا لَهَا مُنذِرُونَ ﴿٢٠٨﴾}\\
209.\  & \mytextarabic{ذِكْرَىٰ وَمَا كُنَّا ظَـٰلِمِينَ ﴿٢٠٩﴾}\\
210.\  & \mytextarabic{وَمَا تَنَزَّلَتْ بِهِ ٱلشَّيَـٰطِينُ ﴿٢١٠﴾}\\
211.\  & \mytextarabic{وَمَا يَنۢبَغِى لَهُمْ وَمَا يَسْتَطِيعُونَ ﴿٢١١﴾}\\
212.\  & \mytextarabic{إِنَّهُمْ عَنِ ٱلسَّمْعِ لَمَعْزُولُونَ ﴿٢١٢﴾}\\
213.\  & \mytextarabic{فَلَا تَدْعُ مَعَ ٱللَّهِ إِلَـٰهًا ءَاخَرَ فَتَكُونَ مِنَ ٱلْمُعَذَّبِينَ ﴿٢١٣﴾}\\
214.\  & \mytextarabic{وَأَنذِرْ عَشِيرَتَكَ ٱلْأَقْرَبِينَ ﴿٢١٤﴾}\\
215.\  & \mytextarabic{وَٱخْفِضْ جَنَاحَكَ لِمَنِ ٱتَّبَعَكَ مِنَ ٱلْمُؤْمِنِينَ ﴿٢١٥﴾}\\
216.\  & \mytextarabic{فَإِنْ عَصَوْكَ فَقُلْ إِنِّى بَرِىٓءٌۭ مِّمَّا تَعْمَلُونَ ﴿٢١٦﴾}\\
217.\  & \mytextarabic{وَتَوَكَّلْ عَلَى ٱلْعَزِيزِ ٱلرَّحِيمِ ﴿٢١٧﴾}\\
218.\  & \mytextarabic{ٱلَّذِى يَرَىٰكَ حِينَ تَقُومُ ﴿٢١٨﴾}\\
219.\  & \mytextarabic{وَتَقَلُّبَكَ فِى ٱلسَّٰجِدِينَ ﴿٢١٩﴾}\\
220.\  & \mytextarabic{إِنَّهُۥ هُوَ ٱلسَّمِيعُ ٱلْعَلِيمُ ﴿٢٢٠﴾}\\
221.\  & \mytextarabic{هَلْ أُنَبِّئُكُمْ عَلَىٰ مَن تَنَزَّلُ ٱلشَّيَـٰطِينُ ﴿٢٢١﴾}\\
222.\  & \mytextarabic{تَنَزَّلُ عَلَىٰ كُلِّ أَفَّاكٍ أَثِيمٍۢ ﴿٢٢٢﴾}\\
223.\  & \mytextarabic{يُلْقُونَ ٱلسَّمْعَ وَأَكْثَرُهُمْ كَـٰذِبُونَ ﴿٢٢٣﴾}\\
224.\  & \mytextarabic{وَٱلشُّعَرَآءُ يَتَّبِعُهُمُ ٱلْغَاوُۥنَ ﴿٢٢٤﴾}\\
225.\  & \mytextarabic{أَلَمْ تَرَ أَنَّهُمْ فِى كُلِّ وَادٍۢ يَهِيمُونَ ﴿٢٢٥﴾}\\
226.\  & \mytextarabic{وَأَنَّهُمْ يَقُولُونَ مَا لَا يَفْعَلُونَ ﴿٢٢٦﴾}\\
227.\  & \mytextarabic{إِلَّا ٱلَّذِينَ ءَامَنُوا۟ وَعَمِلُوا۟ ٱلصَّـٰلِحَـٰتِ وَذَكَرُوا۟ ٱللَّهَ كَثِيرًۭا وَٱنتَصَرُوا۟ مِنۢ بَعْدِ مَا ظُلِمُوا۟ ۗ وَسَيَعْلَمُ ٱلَّذِينَ ظَلَمُوٓا۟ أَىَّ مُنقَلَبٍۢ يَنقَلِبُونَ ﴿٢٢٧﴾}\\
\end{longtable}
\clearpage