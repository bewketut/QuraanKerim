\begin{center}\section{ሱራቱ ጋፊር -  \textarabic{سوره  غافر}}\end{center}
\begin{longtable}{%
  @{}
    p{.5\textwidth}
  @{~~~}
    p{.5\textwidth}
    @{}
}
ቢስሚላሂ አራህመኒ ራሂይም &  \mytextarabic{بِسْمِ ٱللَّهِ ٱلرَّحْمَـٰنِ ٱلرَّحِيمِ}\\
1.\  & \mytextarabic{ حمٓ ﴿١﴾}\\
2.\  & \mytextarabic{تَنزِيلُ ٱلْكِتَـٰبِ مِنَ ٱللَّهِ ٱلْعَزِيزِ ٱلْعَلِيمِ ﴿٢﴾}\\
3.\  & \mytextarabic{غَافِرِ ٱلذَّنۢبِ وَقَابِلِ ٱلتَّوْبِ شَدِيدِ ٱلْعِقَابِ ذِى ٱلطَّوْلِ ۖ لَآ إِلَـٰهَ إِلَّا هُوَ ۖ إِلَيْهِ ٱلْمَصِيرُ ﴿٣﴾}\\
4.\  & \mytextarabic{مَا يُجَٰدِلُ فِىٓ ءَايَـٰتِ ٱللَّهِ إِلَّا ٱلَّذِينَ كَفَرُوا۟ فَلَا يَغْرُرْكَ تَقَلُّبُهُمْ فِى ٱلْبِلَـٰدِ ﴿٤﴾}\\
5.\  & \mytextarabic{كَذَّبَتْ قَبْلَهُمْ قَوْمُ نُوحٍۢ وَٱلْأَحْزَابُ مِنۢ بَعْدِهِمْ ۖ وَهَمَّتْ كُلُّ أُمَّةٍۭ بِرَسُولِهِمْ لِيَأْخُذُوهُ ۖ وَجَٰدَلُوا۟ بِٱلْبَٰطِلِ لِيُدْحِضُوا۟ بِهِ ٱلْحَقَّ فَأَخَذْتُهُمْ ۖ فَكَيْفَ كَانَ عِقَابِ ﴿٥﴾}\\
6.\  & \mytextarabic{وَكَذَٟلِكَ حَقَّتْ كَلِمَتُ رَبِّكَ عَلَى ٱلَّذِينَ كَفَرُوٓا۟ أَنَّهُمْ أَصْحَـٰبُ ٱلنَّارِ ﴿٦﴾}\\
7.\  & \mytextarabic{ٱلَّذِينَ يَحْمِلُونَ ٱلْعَرْشَ وَمَنْ حَوْلَهُۥ يُسَبِّحُونَ بِحَمْدِ رَبِّهِمْ وَيُؤْمِنُونَ بِهِۦ وَيَسْتَغْفِرُونَ لِلَّذِينَ ءَامَنُوا۟ رَبَّنَا وَسِعْتَ كُلَّ شَىْءٍۢ رَّحْمَةًۭ وَعِلْمًۭا فَٱغْفِرْ لِلَّذِينَ تَابُوا۟ وَٱتَّبَعُوا۟ سَبِيلَكَ وَقِهِمْ عَذَابَ ٱلْجَحِيمِ ﴿٧﴾}\\
8.\  & \mytextarabic{رَبَّنَا وَأَدْخِلْهُمْ جَنَّـٰتِ عَدْنٍ ٱلَّتِى وَعَدتَّهُمْ وَمَن صَلَحَ مِنْ ءَابَآئِهِمْ وَأَزْوَٟجِهِمْ وَذُرِّيَّٰتِهِمْ ۚ إِنَّكَ أَنتَ ٱلْعَزِيزُ ٱلْحَكِيمُ ﴿٨﴾}\\
9.\  & \mytextarabic{وَقِهِمُ ٱلسَّيِّـَٔاتِ ۚ وَمَن تَقِ ٱلسَّيِّـَٔاتِ يَوْمَئِذٍۢ فَقَدْ رَحِمْتَهُۥ ۚ وَذَٟلِكَ هُوَ ٱلْفَوْزُ ٱلْعَظِيمُ ﴿٩﴾}\\
10.\  & \mytextarabic{إِنَّ ٱلَّذِينَ كَفَرُوا۟ يُنَادَوْنَ لَمَقْتُ ٱللَّهِ أَكْبَرُ مِن مَّقْتِكُمْ أَنفُسَكُمْ إِذْ تُدْعَوْنَ إِلَى ٱلْإِيمَـٰنِ فَتَكْفُرُونَ ﴿١٠﴾}\\
11.\  & \mytextarabic{قَالُوا۟ رَبَّنَآ أَمَتَّنَا ٱثْنَتَيْنِ وَأَحْيَيْتَنَا ٱثْنَتَيْنِ فَٱعْتَرَفْنَا بِذُنُوبِنَا فَهَلْ إِلَىٰ خُرُوجٍۢ مِّن سَبِيلٍۢ ﴿١١﴾}\\
12.\  & \mytextarabic{ذَٟلِكُم بِأَنَّهُۥٓ إِذَا دُعِىَ ٱللَّهُ وَحْدَهُۥ كَفَرْتُمْ ۖ وَإِن يُشْرَكْ بِهِۦ تُؤْمِنُوا۟ ۚ فَٱلْحُكْمُ لِلَّهِ ٱلْعَلِىِّ ٱلْكَبِيرِ ﴿١٢﴾}\\
13.\  & \mytextarabic{هُوَ ٱلَّذِى يُرِيكُمْ ءَايَـٰتِهِۦ وَيُنَزِّلُ لَكُم مِّنَ ٱلسَّمَآءِ رِزْقًۭا ۚ وَمَا يَتَذَكَّرُ إِلَّا مَن يُنِيبُ ﴿١٣﴾}\\
14.\  & \mytextarabic{فَٱدْعُوا۟ ٱللَّهَ مُخْلِصِينَ لَهُ ٱلدِّينَ وَلَوْ كَرِهَ ٱلْكَـٰفِرُونَ ﴿١٤﴾}\\
15.\  & \mytextarabic{رَفِيعُ ٱلدَّرَجَٰتِ ذُو ٱلْعَرْشِ يُلْقِى ٱلرُّوحَ مِنْ أَمْرِهِۦ عَلَىٰ مَن يَشَآءُ مِنْ عِبَادِهِۦ لِيُنذِرَ يَوْمَ ٱلتَّلَاقِ ﴿١٥﴾}\\
16.\  & \mytextarabic{يَوْمَ هُم بَٰرِزُونَ ۖ لَا يَخْفَىٰ عَلَى ٱللَّهِ مِنْهُمْ شَىْءٌۭ ۚ لِّمَنِ ٱلْمُلْكُ ٱلْيَوْمَ ۖ لِلَّهِ ٱلْوَٟحِدِ ٱلْقَهَّارِ ﴿١٦﴾}\\
17.\  & \mytextarabic{ٱلْيَوْمَ تُجْزَىٰ كُلُّ نَفْسٍۭ بِمَا كَسَبَتْ ۚ لَا ظُلْمَ ٱلْيَوْمَ ۚ إِنَّ ٱللَّهَ سَرِيعُ ٱلْحِسَابِ ﴿١٧﴾}\\
18.\  & \mytextarabic{وَأَنذِرْهُمْ يَوْمَ ٱلْءَازِفَةِ إِذِ ٱلْقُلُوبُ لَدَى ٱلْحَنَاجِرِ كَـٰظِمِينَ ۚ مَا لِلظَّـٰلِمِينَ مِنْ حَمِيمٍۢ وَلَا شَفِيعٍۢ يُطَاعُ ﴿١٨﴾}\\
19.\  & \mytextarabic{يَعْلَمُ خَآئِنَةَ ٱلْأَعْيُنِ وَمَا تُخْفِى ٱلصُّدُورُ ﴿١٩﴾}\\
20.\  & \mytextarabic{وَٱللَّهُ يَقْضِى بِٱلْحَقِّ ۖ وَٱلَّذِينَ يَدْعُونَ مِن دُونِهِۦ لَا يَقْضُونَ بِشَىْءٍ ۗ إِنَّ ٱللَّهَ هُوَ ٱلسَّمِيعُ ٱلْبَصِيرُ ﴿٢٠﴾}\\
21.\  & \mytextarabic{۞ أَوَلَمْ يَسِيرُوا۟ فِى ٱلْأَرْضِ فَيَنظُرُوا۟ كَيْفَ كَانَ عَـٰقِبَةُ ٱلَّذِينَ كَانُوا۟ مِن قَبْلِهِمْ ۚ كَانُوا۟ هُمْ أَشَدَّ مِنْهُمْ قُوَّةًۭ وَءَاثَارًۭا فِى ٱلْأَرْضِ فَأَخَذَهُمُ ٱللَّهُ بِذُنُوبِهِمْ وَمَا كَانَ لَهُم مِّنَ ٱللَّهِ مِن وَاقٍۢ ﴿٢١﴾}\\
22.\  & \mytextarabic{ذَٟلِكَ بِأَنَّهُمْ كَانَت تَّأْتِيهِمْ رُسُلُهُم بِٱلْبَيِّنَـٰتِ فَكَفَرُوا۟ فَأَخَذَهُمُ ٱللَّهُ ۚ إِنَّهُۥ قَوِىٌّۭ شَدِيدُ ٱلْعِقَابِ ﴿٢٢﴾}\\
23.\  & \mytextarabic{وَلَقَدْ أَرْسَلْنَا مُوسَىٰ بِـَٔايَـٰتِنَا وَسُلْطَٰنٍۢ مُّبِينٍ ﴿٢٣﴾}\\
24.\  & \mytextarabic{إِلَىٰ فِرْعَوْنَ وَهَـٰمَـٰنَ وَقَـٰرُونَ فَقَالُوا۟ سَـٰحِرٌۭ كَذَّابٌۭ ﴿٢٤﴾}\\
25.\  & \mytextarabic{فَلَمَّا جَآءَهُم بِٱلْحَقِّ مِنْ عِندِنَا قَالُوا۟ ٱقْتُلُوٓا۟ أَبْنَآءَ ٱلَّذِينَ ءَامَنُوا۟ مَعَهُۥ وَٱسْتَحْيُوا۟ نِسَآءَهُمْ ۚ وَمَا كَيْدُ ٱلْكَـٰفِرِينَ إِلَّا فِى ضَلَـٰلٍۢ ﴿٢٥﴾}\\
26.\  & \mytextarabic{وَقَالَ فِرْعَوْنُ ذَرُونِىٓ أَقْتُلْ مُوسَىٰ وَلْيَدْعُ رَبَّهُۥٓ ۖ إِنِّىٓ أَخَافُ أَن يُبَدِّلَ دِينَكُمْ أَوْ أَن يُظْهِرَ فِى ٱلْأَرْضِ ٱلْفَسَادَ ﴿٢٦﴾}\\
27.\  & \mytextarabic{وَقَالَ مُوسَىٰٓ إِنِّى عُذْتُ بِرَبِّى وَرَبِّكُم مِّن كُلِّ مُتَكَبِّرٍۢ لَّا يُؤْمِنُ بِيَوْمِ ٱلْحِسَابِ ﴿٢٧﴾}\\
28.\  & \mytextarabic{وَقَالَ رَجُلٌۭ مُّؤْمِنٌۭ مِّنْ ءَالِ فِرْعَوْنَ يَكْتُمُ إِيمَـٰنَهُۥٓ أَتَقْتُلُونَ رَجُلًا أَن يَقُولَ رَبِّىَ ٱللَّهُ وَقَدْ جَآءَكُم بِٱلْبَيِّنَـٰتِ مِن رَّبِّكُمْ ۖ وَإِن يَكُ كَـٰذِبًۭا فَعَلَيْهِ كَذِبُهُۥ ۖ وَإِن يَكُ صَادِقًۭا يُصِبْكُم بَعْضُ ٱلَّذِى يَعِدُكُمْ ۖ إِنَّ ٱللَّهَ لَا يَهْدِى مَنْ هُوَ مُسْرِفٌۭ كَذَّابٌۭ ﴿٢٨﴾}\\
29.\  & \mytextarabic{يَـٰقَوْمِ لَكُمُ ٱلْمُلْكُ ٱلْيَوْمَ ظَـٰهِرِينَ فِى ٱلْأَرْضِ فَمَن يَنصُرُنَا مِنۢ بَأْسِ ٱللَّهِ إِن جَآءَنَا ۚ قَالَ فِرْعَوْنُ مَآ أُرِيكُمْ إِلَّا مَآ أَرَىٰ وَمَآ أَهْدِيكُمْ إِلَّا سَبِيلَ ٱلرَّشَادِ ﴿٢٩﴾}\\
30.\  & \mytextarabic{وَقَالَ ٱلَّذِىٓ ءَامَنَ يَـٰقَوْمِ إِنِّىٓ أَخَافُ عَلَيْكُم مِّثْلَ يَوْمِ ٱلْأَحْزَابِ ﴿٣٠﴾}\\
31.\  & \mytextarabic{مِثْلَ دَأْبِ قَوْمِ نُوحٍۢ وَعَادٍۢ وَثَمُودَ وَٱلَّذِينَ مِنۢ بَعْدِهِمْ ۚ وَمَا ٱللَّهُ يُرِيدُ ظُلْمًۭا لِّلْعِبَادِ ﴿٣١﴾}\\
32.\  & \mytextarabic{وَيَـٰقَوْمِ إِنِّىٓ أَخَافُ عَلَيْكُمْ يَوْمَ ٱلتَّنَادِ ﴿٣٢﴾}\\
33.\  & \mytextarabic{يَوْمَ تُوَلُّونَ مُدْبِرِينَ مَا لَكُم مِّنَ ٱللَّهِ مِنْ عَاصِمٍۢ ۗ وَمَن يُضْلِلِ ٱللَّهُ فَمَا لَهُۥ مِنْ هَادٍۢ ﴿٣٣﴾}\\
34.\  & \mytextarabic{وَلَقَدْ جَآءَكُمْ يُوسُفُ مِن قَبْلُ بِٱلْبَيِّنَـٰتِ فَمَا زِلْتُمْ فِى شَكٍّۢ مِّمَّا جَآءَكُم بِهِۦ ۖ حَتَّىٰٓ إِذَا هَلَكَ قُلْتُمْ لَن يَبْعَثَ ٱللَّهُ مِنۢ بَعْدِهِۦ رَسُولًۭا ۚ كَذَٟلِكَ يُضِلُّ ٱللَّهُ مَنْ هُوَ مُسْرِفٌۭ مُّرْتَابٌ ﴿٣٤﴾}\\
35.\  & \mytextarabic{ٱلَّذِينَ يُجَٰدِلُونَ فِىٓ ءَايَـٰتِ ٱللَّهِ بِغَيْرِ سُلْطَٰنٍ أَتَىٰهُمْ ۖ كَبُرَ مَقْتًا عِندَ ٱللَّهِ وَعِندَ ٱلَّذِينَ ءَامَنُوا۟ ۚ كَذَٟلِكَ يَطْبَعُ ٱللَّهُ عَلَىٰ كُلِّ قَلْبِ مُتَكَبِّرٍۢ جَبَّارٍۢ ﴿٣٥﴾}\\
36.\  & \mytextarabic{وَقَالَ فِرْعَوْنُ يَـٰهَـٰمَـٰنُ ٱبْنِ لِى صَرْحًۭا لَّعَلِّىٓ أَبْلُغُ ٱلْأَسْبَٰبَ ﴿٣٦﴾}\\
37.\  & \mytextarabic{أَسْبَٰبَ ٱلسَّمَـٰوَٟتِ فَأَطَّلِعَ إِلَىٰٓ إِلَـٰهِ مُوسَىٰ وَإِنِّى لَأَظُنُّهُۥ كَـٰذِبًۭا ۚ وَكَذَٟلِكَ زُيِّنَ لِفِرْعَوْنَ سُوٓءُ عَمَلِهِۦ وَصُدَّ عَنِ ٱلسَّبِيلِ ۚ وَمَا كَيْدُ فِرْعَوْنَ إِلَّا فِى تَبَابٍۢ ﴿٣٧﴾}\\
38.\  & \mytextarabic{وَقَالَ ٱلَّذِىٓ ءَامَنَ يَـٰقَوْمِ ٱتَّبِعُونِ أَهْدِكُمْ سَبِيلَ ٱلرَّشَادِ ﴿٣٨﴾}\\
39.\  & \mytextarabic{يَـٰقَوْمِ إِنَّمَا هَـٰذِهِ ٱلْحَيَوٰةُ ٱلدُّنْيَا مَتَـٰعٌۭ وَإِنَّ ٱلْءَاخِرَةَ هِىَ دَارُ ٱلْقَرَارِ ﴿٣٩﴾}\\
40.\  & \mytextarabic{مَنْ عَمِلَ سَيِّئَةًۭ فَلَا يُجْزَىٰٓ إِلَّا مِثْلَهَا ۖ وَمَنْ عَمِلَ صَـٰلِحًۭا مِّن ذَكَرٍ أَوْ أُنثَىٰ وَهُوَ مُؤْمِنٌۭ فَأُو۟لَـٰٓئِكَ يَدْخُلُونَ ٱلْجَنَّةَ يُرْزَقُونَ فِيهَا بِغَيْرِ حِسَابٍۢ ﴿٤٠﴾}\\
41.\  & \mytextarabic{۞ وَيَـٰقَوْمِ مَا لِىٓ أَدْعُوكُمْ إِلَى ٱلنَّجَوٰةِ وَتَدْعُونَنِىٓ إِلَى ٱلنَّارِ ﴿٤١﴾}\\
42.\  & \mytextarabic{تَدْعُونَنِى لِأَكْفُرَ بِٱللَّهِ وَأُشْرِكَ بِهِۦ مَا لَيْسَ لِى بِهِۦ عِلْمٌۭ وَأَنَا۠ أَدْعُوكُمْ إِلَى ٱلْعَزِيزِ ٱلْغَفَّٰرِ ﴿٤٢﴾}\\
43.\  & \mytextarabic{لَا جَرَمَ أَنَّمَا تَدْعُونَنِىٓ إِلَيْهِ لَيْسَ لَهُۥ دَعْوَةٌۭ فِى ٱلدُّنْيَا وَلَا فِى ٱلْءَاخِرَةِ وَأَنَّ مَرَدَّنَآ إِلَى ٱللَّهِ وَأَنَّ ٱلْمُسْرِفِينَ هُمْ أَصْحَـٰبُ ٱلنَّارِ ﴿٤٣﴾}\\
44.\  & \mytextarabic{فَسَتَذْكُرُونَ مَآ أَقُولُ لَكُمْ ۚ وَأُفَوِّضُ أَمْرِىٓ إِلَى ٱللَّهِ ۚ إِنَّ ٱللَّهَ بَصِيرٌۢ بِٱلْعِبَادِ ﴿٤٤﴾}\\
45.\  & \mytextarabic{فَوَقَىٰهُ ٱللَّهُ سَيِّـَٔاتِ مَا مَكَرُوا۟ ۖ وَحَاقَ بِـَٔالِ فِرْعَوْنَ سُوٓءُ ٱلْعَذَابِ ﴿٤٥﴾}\\
46.\  & \mytextarabic{ٱلنَّارُ يُعْرَضُونَ عَلَيْهَا غُدُوًّۭا وَعَشِيًّۭا ۖ وَيَوْمَ تَقُومُ ٱلسَّاعَةُ أَدْخِلُوٓا۟ ءَالَ فِرْعَوْنَ أَشَدَّ ٱلْعَذَابِ ﴿٤٦﴾}\\
47.\  & \mytextarabic{وَإِذْ يَتَحَآجُّونَ فِى ٱلنَّارِ فَيَقُولُ ٱلضُّعَفَـٰٓؤُا۟ لِلَّذِينَ ٱسْتَكْبَرُوٓا۟ إِنَّا كُنَّا لَكُمْ تَبَعًۭا فَهَلْ أَنتُم مُّغْنُونَ عَنَّا نَصِيبًۭا مِّنَ ٱلنَّارِ ﴿٤٧﴾}\\
48.\  & \mytextarabic{قَالَ ٱلَّذِينَ ٱسْتَكْبَرُوٓا۟ إِنَّا كُلٌّۭ فِيهَآ إِنَّ ٱللَّهَ قَدْ حَكَمَ بَيْنَ ٱلْعِبَادِ ﴿٤٨﴾}\\
49.\  & \mytextarabic{وَقَالَ ٱلَّذِينَ فِى ٱلنَّارِ لِخَزَنَةِ جَهَنَّمَ ٱدْعُوا۟ رَبَّكُمْ يُخَفِّفْ عَنَّا يَوْمًۭا مِّنَ ٱلْعَذَابِ ﴿٤٩﴾}\\
50.\  & \mytextarabic{قَالُوٓا۟ أَوَلَمْ تَكُ تَأْتِيكُمْ رُسُلُكُم بِٱلْبَيِّنَـٰتِ ۖ قَالُوا۟ بَلَىٰ ۚ قَالُوا۟ فَٱدْعُوا۟ ۗ وَمَا دُعَـٰٓؤُا۟ ٱلْكَـٰفِرِينَ إِلَّا فِى ضَلَـٰلٍ ﴿٥٠﴾}\\
51.\  & \mytextarabic{إِنَّا لَنَنصُرُ رُسُلَنَا وَٱلَّذِينَ ءَامَنُوا۟ فِى ٱلْحَيَوٰةِ ٱلدُّنْيَا وَيَوْمَ يَقُومُ ٱلْأَشْهَـٰدُ ﴿٥١﴾}\\
52.\  & \mytextarabic{يَوْمَ لَا يَنفَعُ ٱلظَّـٰلِمِينَ مَعْذِرَتُهُمْ ۖ وَلَهُمُ ٱللَّعْنَةُ وَلَهُمْ سُوٓءُ ٱلدَّارِ ﴿٥٢﴾}\\
53.\  & \mytextarabic{وَلَقَدْ ءَاتَيْنَا مُوسَى ٱلْهُدَىٰ وَأَوْرَثْنَا بَنِىٓ إِسْرَٰٓءِيلَ ٱلْكِتَـٰبَ ﴿٥٣﴾}\\
54.\  & \mytextarabic{هُدًۭى وَذِكْرَىٰ لِأُو۟لِى ٱلْأَلْبَٰبِ ﴿٥٤﴾}\\
55.\  & \mytextarabic{فَٱصْبِرْ إِنَّ وَعْدَ ٱللَّهِ حَقٌّۭ وَٱسْتَغْفِرْ لِذَنۢبِكَ وَسَبِّحْ بِحَمْدِ رَبِّكَ بِٱلْعَشِىِّ وَٱلْإِبْكَـٰرِ ﴿٥٥﴾}\\
56.\  & \mytextarabic{إِنَّ ٱلَّذِينَ يُجَٰدِلُونَ فِىٓ ءَايَـٰتِ ٱللَّهِ بِغَيْرِ سُلْطَٰنٍ أَتَىٰهُمْ ۙ إِن فِى صُدُورِهِمْ إِلَّا كِبْرٌۭ مَّا هُم بِبَٰلِغِيهِ ۚ فَٱسْتَعِذْ بِٱللَّهِ ۖ إِنَّهُۥ هُوَ ٱلسَّمِيعُ ٱلْبَصِيرُ ﴿٥٦﴾}\\
57.\  & \mytextarabic{لَخَلْقُ ٱلسَّمَـٰوَٟتِ وَٱلْأَرْضِ أَكْبَرُ مِنْ خَلْقِ ٱلنَّاسِ وَلَـٰكِنَّ أَكْثَرَ ٱلنَّاسِ لَا يَعْلَمُونَ ﴿٥٧﴾}\\
58.\  & \mytextarabic{وَمَا يَسْتَوِى ٱلْأَعْمَىٰ وَٱلْبَصِيرُ وَٱلَّذِينَ ءَامَنُوا۟ وَعَمِلُوا۟ ٱلصَّـٰلِحَـٰتِ وَلَا ٱلْمُسِىٓءُ ۚ قَلِيلًۭا مَّا تَتَذَكَّرُونَ ﴿٥٨﴾}\\
59.\  & \mytextarabic{إِنَّ ٱلسَّاعَةَ لَءَاتِيَةٌۭ لَّا رَيْبَ فِيهَا وَلَـٰكِنَّ أَكْثَرَ ٱلنَّاسِ لَا يُؤْمِنُونَ ﴿٥٩﴾}\\
60.\  & \mytextarabic{وَقَالَ رَبُّكُمُ ٱدْعُونِىٓ أَسْتَجِبْ لَكُمْ ۚ إِنَّ ٱلَّذِينَ يَسْتَكْبِرُونَ عَنْ عِبَادَتِى سَيَدْخُلُونَ جَهَنَّمَ دَاخِرِينَ ﴿٦٠﴾}\\
61.\  & \mytextarabic{ٱللَّهُ ٱلَّذِى جَعَلَ لَكُمُ ٱلَّيْلَ لِتَسْكُنُوا۟ فِيهِ وَٱلنَّهَارَ مُبْصِرًا ۚ إِنَّ ٱللَّهَ لَذُو فَضْلٍ عَلَى ٱلنَّاسِ وَلَـٰكِنَّ أَكْثَرَ ٱلنَّاسِ لَا يَشْكُرُونَ ﴿٦١﴾}\\
62.\  & \mytextarabic{ذَٟلِكُمُ ٱللَّهُ رَبُّكُمْ خَـٰلِقُ كُلِّ شَىْءٍۢ لَّآ إِلَـٰهَ إِلَّا هُوَ ۖ فَأَنَّىٰ تُؤْفَكُونَ ﴿٦٢﴾}\\
63.\  & \mytextarabic{كَذَٟلِكَ يُؤْفَكُ ٱلَّذِينَ كَانُوا۟ بِـَٔايَـٰتِ ٱللَّهِ يَجْحَدُونَ ﴿٦٣﴾}\\
64.\  & \mytextarabic{ٱللَّهُ ٱلَّذِى جَعَلَ لَكُمُ ٱلْأَرْضَ قَرَارًۭا وَٱلسَّمَآءَ بِنَآءًۭ وَصَوَّرَكُمْ فَأَحْسَنَ صُوَرَكُمْ وَرَزَقَكُم مِّنَ ٱلطَّيِّبَٰتِ ۚ ذَٟلِكُمُ ٱللَّهُ رَبُّكُمْ ۖ فَتَبَارَكَ ٱللَّهُ رَبُّ ٱلْعَـٰلَمِينَ ﴿٦٤﴾}\\
65.\  & \mytextarabic{هُوَ ٱلْحَىُّ لَآ إِلَـٰهَ إِلَّا هُوَ فَٱدْعُوهُ مُخْلِصِينَ لَهُ ٱلدِّينَ ۗ ٱلْحَمْدُ لِلَّهِ رَبِّ ٱلْعَـٰلَمِينَ ﴿٦٥﴾}\\
66.\  & \mytextarabic{۞ قُلْ إِنِّى نُهِيتُ أَنْ أَعْبُدَ ٱلَّذِينَ تَدْعُونَ مِن دُونِ ٱللَّهِ لَمَّا جَآءَنِىَ ٱلْبَيِّنَـٰتُ مِن رَّبِّى وَأُمِرْتُ أَنْ أُسْلِمَ لِرَبِّ ٱلْعَـٰلَمِينَ ﴿٦٦﴾}\\
67.\  & \mytextarabic{هُوَ ٱلَّذِى خَلَقَكُم مِّن تُرَابٍۢ ثُمَّ مِن نُّطْفَةٍۢ ثُمَّ مِنْ عَلَقَةٍۢ ثُمَّ يُخْرِجُكُمْ طِفْلًۭا ثُمَّ لِتَبْلُغُوٓا۟ أَشُدَّكُمْ ثُمَّ لِتَكُونُوا۟ شُيُوخًۭا ۚ وَمِنكُم مَّن يُتَوَفَّىٰ مِن قَبْلُ ۖ وَلِتَبْلُغُوٓا۟ أَجَلًۭا مُّسَمًّۭى وَلَعَلَّكُمْ تَعْقِلُونَ ﴿٦٧﴾}\\
68.\  & \mytextarabic{هُوَ ٱلَّذِى يُحْىِۦ وَيُمِيتُ ۖ فَإِذَا قَضَىٰٓ أَمْرًۭا فَإِنَّمَا يَقُولُ لَهُۥ كُن فَيَكُونُ ﴿٦٨﴾}\\
69.\  & \mytextarabic{أَلَمْ تَرَ إِلَى ٱلَّذِينَ يُجَٰدِلُونَ فِىٓ ءَايَـٰتِ ٱللَّهِ أَنَّىٰ يُصْرَفُونَ ﴿٦٩﴾}\\
70.\  & \mytextarabic{ٱلَّذِينَ كَذَّبُوا۟ بِٱلْكِتَـٰبِ وَبِمَآ أَرْسَلْنَا بِهِۦ رُسُلَنَا ۖ فَسَوْفَ يَعْلَمُونَ ﴿٧٠﴾}\\
71.\  & \mytextarabic{إِذِ ٱلْأَغْلَـٰلُ فِىٓ أَعْنَـٰقِهِمْ وَٱلسَّلَـٰسِلُ يُسْحَبُونَ ﴿٧١﴾}\\
72.\  & \mytextarabic{فِى ٱلْحَمِيمِ ثُمَّ فِى ٱلنَّارِ يُسْجَرُونَ ﴿٧٢﴾}\\
73.\  & \mytextarabic{ثُمَّ قِيلَ لَهُمْ أَيْنَ مَا كُنتُمْ تُشْرِكُونَ ﴿٧٣﴾}\\
74.\  & \mytextarabic{مِن دُونِ ٱللَّهِ ۖ قَالُوا۟ ضَلُّوا۟ عَنَّا بَل لَّمْ نَكُن نَّدْعُوا۟ مِن قَبْلُ شَيْـًۭٔا ۚ كَذَٟلِكَ يُضِلُّ ٱللَّهُ ٱلْكَـٰفِرِينَ ﴿٧٤﴾}\\
75.\  & \mytextarabic{ذَٟلِكُم بِمَا كُنتُمْ تَفْرَحُونَ فِى ٱلْأَرْضِ بِغَيْرِ ٱلْحَقِّ وَبِمَا كُنتُمْ تَمْرَحُونَ ﴿٧٥﴾}\\
76.\  & \mytextarabic{ٱدْخُلُوٓا۟ أَبْوَٟبَ جَهَنَّمَ خَـٰلِدِينَ فِيهَا ۖ فَبِئْسَ مَثْوَى ٱلْمُتَكَبِّرِينَ ﴿٧٦﴾}\\
77.\  & \mytextarabic{فَٱصْبِرْ إِنَّ وَعْدَ ٱللَّهِ حَقٌّۭ ۚ فَإِمَّا نُرِيَنَّكَ بَعْضَ ٱلَّذِى نَعِدُهُمْ أَوْ نَتَوَفَّيَنَّكَ فَإِلَيْنَا يُرْجَعُونَ ﴿٧٧﴾}\\
78.\  & \mytextarabic{وَلَقَدْ أَرْسَلْنَا رُسُلًۭا مِّن قَبْلِكَ مِنْهُم مَّن قَصَصْنَا عَلَيْكَ وَمِنْهُم مَّن لَّمْ نَقْصُصْ عَلَيْكَ ۗ وَمَا كَانَ لِرَسُولٍ أَن يَأْتِىَ بِـَٔايَةٍ إِلَّا بِإِذْنِ ٱللَّهِ ۚ فَإِذَا جَآءَ أَمْرُ ٱللَّهِ قُضِىَ بِٱلْحَقِّ وَخَسِرَ هُنَالِكَ ٱلْمُبْطِلُونَ ﴿٧٨﴾}\\
79.\  & \mytextarabic{ٱللَّهُ ٱلَّذِى جَعَلَ لَكُمُ ٱلْأَنْعَـٰمَ لِتَرْكَبُوا۟ مِنْهَا وَمِنْهَا تَأْكُلُونَ ﴿٧٩﴾}\\
80.\  & \mytextarabic{وَلَكُمْ فِيهَا مَنَـٰفِعُ وَلِتَبْلُغُوا۟ عَلَيْهَا حَاجَةًۭ فِى صُدُورِكُمْ وَعَلَيْهَا وَعَلَى ٱلْفُلْكِ تُحْمَلُونَ ﴿٨٠﴾}\\
81.\  & \mytextarabic{وَيُرِيكُمْ ءَايَـٰتِهِۦ فَأَىَّ ءَايَـٰتِ ٱللَّهِ تُنكِرُونَ ﴿٨١﴾}\\
82.\  & \mytextarabic{أَفَلَمْ يَسِيرُوا۟ فِى ٱلْأَرْضِ فَيَنظُرُوا۟ كَيْفَ كَانَ عَـٰقِبَةُ ٱلَّذِينَ مِن قَبْلِهِمْ ۚ كَانُوٓا۟ أَكْثَرَ مِنْهُمْ وَأَشَدَّ قُوَّةًۭ وَءَاثَارًۭا فِى ٱلْأَرْضِ فَمَآ أَغْنَىٰ عَنْهُم مَّا كَانُوا۟ يَكْسِبُونَ ﴿٨٢﴾}\\
83.\  & \mytextarabic{فَلَمَّا جَآءَتْهُمْ رُسُلُهُم بِٱلْبَيِّنَـٰتِ فَرِحُوا۟ بِمَا عِندَهُم مِّنَ ٱلْعِلْمِ وَحَاقَ بِهِم مَّا كَانُوا۟ بِهِۦ يَسْتَهْزِءُونَ ﴿٨٣﴾}\\
84.\  & \mytextarabic{فَلَمَّا رَأَوْا۟ بَأْسَنَا قَالُوٓا۟ ءَامَنَّا بِٱللَّهِ وَحْدَهُۥ وَكَفَرْنَا بِمَا كُنَّا بِهِۦ مُشْرِكِينَ ﴿٨٤﴾}\\
85.\  & \mytextarabic{فَلَمْ يَكُ يَنفَعُهُمْ إِيمَـٰنُهُمْ لَمَّا رَأَوْا۟ بَأْسَنَا ۖ سُنَّتَ ٱللَّهِ ٱلَّتِى قَدْ خَلَتْ فِى عِبَادِهِۦ ۖ وَخَسِرَ هُنَالِكَ ٱلْكَـٰفِرُونَ ﴿٨٥﴾}\\
\end{longtable}
\clearpage