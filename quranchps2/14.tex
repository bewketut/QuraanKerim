%% License: BSD style (Berkley) (i.e. Put the Copyright owner's name always)
%% Writer and Copyright (to): Bewketu(Bilal) Tadilo (2016-17)
\begin{center}\section{\LR{\textamhsec{ሱራቱ ኢብራሂም -}  \textarabic{سوره  ابراهيم}}}\end{center}
\begin{longtable}{%
  @{}
    p{.5\textwidth}
  @{~~~}
    p{.5\textwidth}
    @{}
}
\textamh{ቢስሚላሂ አራህመኒ ራሂይም } &  \mytextarabic{بِسْمِ ٱللَّهِ ٱلرَّحْمَـٰنِ ٱلرَّحِيمِ}\\
\textamh{1.\  } & \mytextarabic{ الٓر ۚ كِتَـٰبٌ أَنزَلْنَـٰهُ إِلَيْكَ لِتُخْرِجَ ٱلنَّاسَ مِنَ ٱلظُّلُمَـٰتِ إِلَى ٱلنُّورِ بِإِذْنِ رَبِّهِمْ إِلَىٰ صِرَٰطِ ٱلْعَزِيزِ ٱلْحَمِيدِ ﴿١﴾}\\
\textamh{2.\  } & \mytextarabic{ٱللَّهِ ٱلَّذِى لَهُۥ مَا فِى ٱلسَّمَـٰوَٟتِ وَمَا فِى ٱلْأَرْضِ ۗ وَوَيْلٌۭ لِّلْكَـٰفِرِينَ مِنْ عَذَابٍۢ شَدِيدٍ ﴿٢﴾}\\
\textamh{3.\  } & \mytextarabic{ٱلَّذِينَ يَسْتَحِبُّونَ ٱلْحَيَوٰةَ ٱلدُّنْيَا عَلَى ٱلْءَاخِرَةِ وَيَصُدُّونَ عَن سَبِيلِ ٱللَّهِ وَيَبْغُونَهَا عِوَجًا ۚ أُو۟لَـٰٓئِكَ فِى ضَلَـٰلٍۭ بَعِيدٍۢ ﴿٣﴾}\\
\textamh{4.\  } & \mytextarabic{وَمَآ أَرْسَلْنَا مِن رَّسُولٍ إِلَّا بِلِسَانِ قَوْمِهِۦ لِيُبَيِّنَ لَهُمْ ۖ فَيُضِلُّ ٱللَّهُ مَن يَشَآءُ وَيَهْدِى مَن يَشَآءُ ۚ وَهُوَ ٱلْعَزِيزُ ٱلْحَكِيمُ ﴿٤﴾}\\
\textamh{5.\  } & \mytextarabic{وَلَقَدْ أَرْسَلْنَا مُوسَىٰ بِـَٔايَـٰتِنَآ أَنْ أَخْرِجْ قَوْمَكَ مِنَ ٱلظُّلُمَـٰتِ إِلَى ٱلنُّورِ وَذَكِّرْهُم بِأَيَّىٰمِ ٱللَّهِ ۚ إِنَّ فِى ذَٟلِكَ لَءَايَـٰتٍۢ لِّكُلِّ صَبَّارٍۢ شَكُورٍۢ ﴿٥﴾}\\
\textamh{6.\  } & \mytextarabic{وَإِذْ قَالَ مُوسَىٰ لِقَوْمِهِ ٱذْكُرُوا۟ نِعْمَةَ ٱللَّهِ عَلَيْكُمْ إِذْ أَنجَىٰكُم مِّنْ ءَالِ فِرْعَوْنَ يَسُومُونَكُمْ سُوٓءَ ٱلْعَذَابِ وَيُذَبِّحُونَ أَبْنَآءَكُمْ وَيَسْتَحْيُونَ نِسَآءَكُمْ ۚ وَفِى ذَٟلِكُم بَلَآءٌۭ مِّن رَّبِّكُمْ عَظِيمٌۭ ﴿٦﴾}\\
\textamh{7.\  } & \mytextarabic{وَإِذْ تَأَذَّنَ رَبُّكُمْ لَئِن شَكَرْتُمْ لَأَزِيدَنَّكُمْ ۖ وَلَئِن كَفَرْتُمْ إِنَّ عَذَابِى لَشَدِيدٌۭ ﴿٧﴾}\\
\textamh{8.\  } & \mytextarabic{وَقَالَ مُوسَىٰٓ إِن تَكْفُرُوٓا۟ أَنتُمْ وَمَن فِى ٱلْأَرْضِ جَمِيعًۭا فَإِنَّ ٱللَّهَ لَغَنِىٌّ حَمِيدٌ ﴿٨﴾}\\
\textamh{9.\  } & \mytextarabic{أَلَمْ يَأْتِكُمْ نَبَؤُا۟ ٱلَّذِينَ مِن قَبْلِكُمْ قَوْمِ نُوحٍۢ وَعَادٍۢ وَثَمُودَ ۛ وَٱلَّذِينَ مِنۢ بَعْدِهِمْ ۛ لَا يَعْلَمُهُمْ إِلَّا ٱللَّهُ ۚ جَآءَتْهُمْ رُسُلُهُم بِٱلْبَيِّنَـٰتِ فَرَدُّوٓا۟ أَيْدِيَهُمْ فِىٓ أَفْوَٟهِهِمْ وَقَالُوٓا۟ إِنَّا كَفَرْنَا بِمَآ أُرْسِلْتُم بِهِۦ وَإِنَّا لَفِى شَكٍّۢ مِّمَّا تَدْعُونَنَآ إِلَيْهِ مُرِيبٍۢ ﴿٩﴾}\\
\textamh{10.\  } & \mytextarabic{۞ قَالَتْ رُسُلُهُمْ أَفِى ٱللَّهِ شَكٌّۭ فَاطِرِ ٱلسَّمَـٰوَٟتِ وَٱلْأَرْضِ ۖ يَدْعُوكُمْ لِيَغْفِرَ لَكُم مِّن ذُنُوبِكُمْ وَيُؤَخِّرَكُمْ إِلَىٰٓ أَجَلٍۢ مُّسَمًّۭى ۚ قَالُوٓا۟ إِنْ أَنتُمْ إِلَّا بَشَرٌۭ مِّثْلُنَا تُرِيدُونَ أَن تَصُدُّونَا عَمَّا كَانَ يَعْبُدُ ءَابَآؤُنَا فَأْتُونَا بِسُلْطَٰنٍۢ مُّبِينٍۢ ﴿١٠﴾}\\
\textamh{11.\  } & \mytextarabic{قَالَتْ لَهُمْ رُسُلُهُمْ إِن نَّحْنُ إِلَّا بَشَرٌۭ مِّثْلُكُمْ وَلَـٰكِنَّ ٱللَّهَ يَمُنُّ عَلَىٰ مَن يَشَآءُ مِنْ عِبَادِهِۦ ۖ وَمَا كَانَ لَنَآ أَن نَّأْتِيَكُم بِسُلْطَٰنٍ إِلَّا بِإِذْنِ ٱللَّهِ ۚ وَعَلَى ٱللَّهِ فَلْيَتَوَكَّلِ ٱلْمُؤْمِنُونَ ﴿١١﴾}\\
\textamh{12.\  } & \mytextarabic{وَمَا لَنَآ أَلَّا نَتَوَكَّلَ عَلَى ٱللَّهِ وَقَدْ هَدَىٰنَا سُبُلَنَا ۚ وَلَنَصْبِرَنَّ عَلَىٰ مَآ ءَاذَيْتُمُونَا ۚ وَعَلَى ٱللَّهِ فَلْيَتَوَكَّلِ ٱلْمُتَوَكِّلُونَ ﴿١٢﴾}\\
\textamh{13.\  } & \mytextarabic{وَقَالَ ٱلَّذِينَ كَفَرُوا۟ لِرُسُلِهِمْ لَنُخْرِجَنَّكُم مِّنْ أَرْضِنَآ أَوْ لَتَعُودُنَّ فِى مِلَّتِنَا ۖ فَأَوْحَىٰٓ إِلَيْهِمْ رَبُّهُمْ لَنُهْلِكَنَّ ٱلظَّـٰلِمِينَ ﴿١٣﴾}\\
\textamh{14.\  } & \mytextarabic{وَلَنُسْكِنَنَّكُمُ ٱلْأَرْضَ مِنۢ بَعْدِهِمْ ۚ ذَٟلِكَ لِمَنْ خَافَ مَقَامِى وَخَافَ وَعِيدِ ﴿١٤﴾}\\
\textamh{15.\  } & \mytextarabic{وَٱسْتَفْتَحُوا۟ وَخَابَ كُلُّ جَبَّارٍ عَنِيدٍۢ ﴿١٥﴾}\\
\textamh{16.\  } & \mytextarabic{مِّن وَرَآئِهِۦ جَهَنَّمُ وَيُسْقَىٰ مِن مَّآءٍۢ صَدِيدٍۢ ﴿١٦﴾}\\
\textamh{17.\  } & \mytextarabic{يَتَجَرَّعُهُۥ وَلَا يَكَادُ يُسِيغُهُۥ وَيَأْتِيهِ ٱلْمَوْتُ مِن كُلِّ مَكَانٍۢ وَمَا هُوَ بِمَيِّتٍۢ ۖ وَمِن وَرَآئِهِۦ عَذَابٌ غَلِيظٌۭ ﴿١٧﴾}\\
\textamh{18.\  } & \mytextarabic{مَّثَلُ ٱلَّذِينَ كَفَرُوا۟ بِرَبِّهِمْ ۖ أَعْمَـٰلُهُمْ كَرَمَادٍ ٱشْتَدَّتْ بِهِ ٱلرِّيحُ فِى يَوْمٍ عَاصِفٍۢ ۖ لَّا يَقْدِرُونَ مِمَّا كَسَبُوا۟ عَلَىٰ شَىْءٍۢ ۚ ذَٟلِكَ هُوَ ٱلضَّلَـٰلُ ٱلْبَعِيدُ ﴿١٨﴾}\\
\textamh{19.\  } & \mytextarabic{أَلَمْ تَرَ أَنَّ ٱللَّهَ خَلَقَ ٱلسَّمَـٰوَٟتِ وَٱلْأَرْضَ بِٱلْحَقِّ ۚ إِن يَشَأْ يُذْهِبْكُمْ وَيَأْتِ بِخَلْقٍۢ جَدِيدٍۢ ﴿١٩﴾}\\
\textamh{20.\  } & \mytextarabic{وَمَا ذَٟلِكَ عَلَى ٱللَّهِ بِعَزِيزٍۢ ﴿٢٠﴾}\\
\textamh{21.\  } & \mytextarabic{وَبَرَزُوا۟ لِلَّهِ جَمِيعًۭا فَقَالَ ٱلضُّعَفَـٰٓؤُا۟ لِلَّذِينَ ٱسْتَكْبَرُوٓا۟ إِنَّا كُنَّا لَكُمْ تَبَعًۭا فَهَلْ أَنتُم مُّغْنُونَ عَنَّا مِنْ عَذَابِ ٱللَّهِ مِن شَىْءٍۢ ۚ قَالُوا۟ لَوْ هَدَىٰنَا ٱللَّهُ لَهَدَيْنَـٰكُمْ ۖ سَوَآءٌ عَلَيْنَآ أَجَزِعْنَآ أَمْ صَبَرْنَا مَا لَنَا مِن مَّحِيصٍۢ ﴿٢١﴾}\\
\textamh{22.\  } & \mytextarabic{وَقَالَ ٱلشَّيْطَٰنُ لَمَّا قُضِىَ ٱلْأَمْرُ إِنَّ ٱللَّهَ وَعَدَكُمْ وَعْدَ ٱلْحَقِّ وَوَعَدتُّكُمْ فَأَخْلَفْتُكُمْ ۖ وَمَا كَانَ لِىَ عَلَيْكُم مِّن سُلْطَٰنٍ إِلَّآ أَن دَعَوْتُكُمْ فَٱسْتَجَبْتُمْ لِى ۖ فَلَا تَلُومُونِى وَلُومُوٓا۟ أَنفُسَكُم ۖ مَّآ أَنَا۠ بِمُصْرِخِكُمْ وَمَآ أَنتُم بِمُصْرِخِىَّ ۖ إِنِّى كَفَرْتُ بِمَآ أَشْرَكْتُمُونِ مِن قَبْلُ ۗ إِنَّ ٱلظَّـٰلِمِينَ لَهُمْ عَذَابٌ أَلِيمٌۭ ﴿٢٢﴾}\\
\textamh{23.\  } & \mytextarabic{وَأُدْخِلَ ٱلَّذِينَ ءَامَنُوا۟ وَعَمِلُوا۟ ٱلصَّـٰلِحَـٰتِ جَنَّـٰتٍۢ تَجْرِى مِن تَحْتِهَا ٱلْأَنْهَـٰرُ خَـٰلِدِينَ فِيهَا بِإِذْنِ رَبِّهِمْ ۖ تَحِيَّتُهُمْ فِيهَا سَلَـٰمٌ ﴿٢٣﴾}\\
\textamh{24.\  } & \mytextarabic{أَلَمْ تَرَ كَيْفَ ضَرَبَ ٱللَّهُ مَثَلًۭا كَلِمَةًۭ طَيِّبَةًۭ كَشَجَرَةٍۢ طَيِّبَةٍ أَصْلُهَا ثَابِتٌۭ وَفَرْعُهَا فِى ٱلسَّمَآءِ ﴿٢٤﴾}\\
\textamh{25.\  } & \mytextarabic{تُؤْتِىٓ أُكُلَهَا كُلَّ حِينٍۭ بِإِذْنِ رَبِّهَا ۗ وَيَضْرِبُ ٱللَّهُ ٱلْأَمْثَالَ لِلنَّاسِ لَعَلَّهُمْ يَتَذَكَّرُونَ ﴿٢٥﴾}\\
\textamh{26.\  } & \mytextarabic{وَمَثَلُ كَلِمَةٍ خَبِيثَةٍۢ كَشَجَرَةٍ خَبِيثَةٍ ٱجْتُثَّتْ مِن فَوْقِ ٱلْأَرْضِ مَا لَهَا مِن قَرَارٍۢ ﴿٢٦﴾}\\
\textamh{27.\  } & \mytextarabic{يُثَبِّتُ ٱللَّهُ ٱلَّذِينَ ءَامَنُوا۟ بِٱلْقَوْلِ ٱلثَّابِتِ فِى ٱلْحَيَوٰةِ ٱلدُّنْيَا وَفِى ٱلْءَاخِرَةِ ۖ وَيُضِلُّ ٱللَّهُ ٱلظَّـٰلِمِينَ ۚ وَيَفْعَلُ ٱللَّهُ مَا يَشَآءُ ﴿٢٧﴾}\\
\textamh{28.\  } & \mytextarabic{۞ أَلَمْ تَرَ إِلَى ٱلَّذِينَ بَدَّلُوا۟ نِعْمَتَ ٱللَّهِ كُفْرًۭا وَأَحَلُّوا۟ قَوْمَهُمْ دَارَ ٱلْبَوَارِ ﴿٢٨﴾}\\
\textamh{29.\  } & \mytextarabic{جَهَنَّمَ يَصْلَوْنَهَا ۖ وَبِئْسَ ٱلْقَرَارُ ﴿٢٩﴾}\\
\textamh{30.\  } & \mytextarabic{وَجَعَلُوا۟ لِلَّهِ أَندَادًۭا لِّيُضِلُّوا۟ عَن سَبِيلِهِۦ ۗ قُلْ تَمَتَّعُوا۟ فَإِنَّ مَصِيرَكُمْ إِلَى ٱلنَّارِ ﴿٣٠﴾}\\
\textamh{31.\  } & \mytextarabic{قُل لِّعِبَادِىَ ٱلَّذِينَ ءَامَنُوا۟ يُقِيمُوا۟ ٱلصَّلَوٰةَ وَيُنفِقُوا۟ مِمَّا رَزَقْنَـٰهُمْ سِرًّۭا وَعَلَانِيَةًۭ مِّن قَبْلِ أَن يَأْتِىَ يَوْمٌۭ لَّا بَيْعٌۭ فِيهِ وَلَا خِلَـٰلٌ ﴿٣١﴾}\\
\textamh{32.\  } & \mytextarabic{ٱللَّهُ ٱلَّذِى خَلَقَ ٱلسَّمَـٰوَٟتِ وَٱلْأَرْضَ وَأَنزَلَ مِنَ ٱلسَّمَآءِ مَآءًۭ فَأَخْرَجَ بِهِۦ مِنَ ٱلثَّمَرَٰتِ رِزْقًۭا لَّكُمْ ۖ وَسَخَّرَ لَكُمُ ٱلْفُلْكَ لِتَجْرِىَ فِى ٱلْبَحْرِ بِأَمْرِهِۦ ۖ وَسَخَّرَ لَكُمُ ٱلْأَنْهَـٰرَ ﴿٣٢﴾}\\
\textamh{33.\  } & \mytextarabic{وَسَخَّرَ لَكُمُ ٱلشَّمْسَ وَٱلْقَمَرَ دَآئِبَيْنِ ۖ وَسَخَّرَ لَكُمُ ٱلَّيْلَ وَٱلنَّهَارَ ﴿٣٣﴾}\\
\textamh{34.\  } & \mytextarabic{وَءَاتَىٰكُم مِّن كُلِّ مَا سَأَلْتُمُوهُ ۚ وَإِن تَعُدُّوا۟ نِعْمَتَ ٱللَّهِ لَا تُحْصُوهَآ ۗ إِنَّ ٱلْإِنسَـٰنَ لَظَلُومٌۭ كَفَّارٌۭ ﴿٣٤﴾}\\
\textamh{35.\  } & \mytextarabic{وَإِذْ قَالَ إِبْرَٰهِيمُ رَبِّ ٱجْعَلْ هَـٰذَا ٱلْبَلَدَ ءَامِنًۭا وَٱجْنُبْنِى وَبَنِىَّ أَن نَّعْبُدَ ٱلْأَصْنَامَ ﴿٣٥﴾}\\
\textamh{36.\  } & \mytextarabic{رَبِّ إِنَّهُنَّ أَضْلَلْنَ كَثِيرًۭا مِّنَ ٱلنَّاسِ ۖ فَمَن تَبِعَنِى فَإِنَّهُۥ مِنِّى ۖ وَمَنْ عَصَانِى فَإِنَّكَ غَفُورٌۭ رَّحِيمٌۭ ﴿٣٦﴾}\\
\textamh{37.\  } & \mytextarabic{رَّبَّنَآ إِنِّىٓ أَسْكَنتُ مِن ذُرِّيَّتِى بِوَادٍ غَيْرِ ذِى زَرْعٍ عِندَ بَيْتِكَ ٱلْمُحَرَّمِ رَبَّنَا لِيُقِيمُوا۟ ٱلصَّلَوٰةَ فَٱجْعَلْ أَفْـِٔدَةًۭ مِّنَ ٱلنَّاسِ تَهْوِىٓ إِلَيْهِمْ وَٱرْزُقْهُم مِّنَ ٱلثَّمَرَٰتِ لَعَلَّهُمْ يَشْكُرُونَ ﴿٣٧﴾}\\
\textamh{38.\  } & \mytextarabic{رَبَّنَآ إِنَّكَ تَعْلَمُ مَا نُخْفِى وَمَا نُعْلِنُ ۗ وَمَا يَخْفَىٰ عَلَى ٱللَّهِ مِن شَىْءٍۢ فِى ٱلْأَرْضِ وَلَا فِى ٱلسَّمَآءِ ﴿٣٨﴾}\\
\textamh{39.\  } & \mytextarabic{ٱلْحَمْدُ لِلَّهِ ٱلَّذِى وَهَبَ لِى عَلَى ٱلْكِبَرِ إِسْمَـٰعِيلَ وَإِسْحَـٰقَ ۚ إِنَّ رَبِّى لَسَمِيعُ ٱلدُّعَآءِ ﴿٣٩﴾}\\
\textamh{40.\  } & \mytextarabic{رَبِّ ٱجْعَلْنِى مُقِيمَ ٱلصَّلَوٰةِ وَمِن ذُرِّيَّتِى ۚ رَبَّنَا وَتَقَبَّلْ دُعَآءِ ﴿٤٠﴾}\\
\textamh{41.\  } & \mytextarabic{رَبَّنَا ٱغْفِرْ لِى وَلِوَٟلِدَىَّ وَلِلْمُؤْمِنِينَ يَوْمَ يَقُومُ ٱلْحِسَابُ ﴿٤١﴾}\\
\textamh{42.\  } & \mytextarabic{وَلَا تَحْسَبَنَّ ٱللَّهَ غَٰفِلًا عَمَّا يَعْمَلُ ٱلظَّـٰلِمُونَ ۚ إِنَّمَا يُؤَخِّرُهُمْ لِيَوْمٍۢ تَشْخَصُ فِيهِ ٱلْأَبْصَـٰرُ ﴿٤٢﴾}\\
\textamh{43.\  } & \mytextarabic{مُهْطِعِينَ مُقْنِعِى رُءُوسِهِمْ لَا يَرْتَدُّ إِلَيْهِمْ طَرْفُهُمْ ۖ وَأَفْـِٔدَتُهُمْ هَوَآءٌۭ ﴿٤٣﴾}\\
\textamh{44.\  } & \mytextarabic{وَأَنذِرِ ٱلنَّاسَ يَوْمَ يَأْتِيهِمُ ٱلْعَذَابُ فَيَقُولُ ٱلَّذِينَ ظَلَمُوا۟ رَبَّنَآ أَخِّرْنَآ إِلَىٰٓ أَجَلٍۢ قَرِيبٍۢ نُّجِبْ دَعْوَتَكَ وَنَتَّبِعِ ٱلرُّسُلَ ۗ أَوَلَمْ تَكُونُوٓا۟ أَقْسَمْتُم مِّن قَبْلُ مَا لَكُم مِّن زَوَالٍۢ ﴿٤٤﴾}\\
\textamh{45.\  } & \mytextarabic{وَسَكَنتُمْ فِى مَسَـٰكِنِ ٱلَّذِينَ ظَلَمُوٓا۟ أَنفُسَهُمْ وَتَبَيَّنَ لَكُمْ كَيْفَ فَعَلْنَا بِهِمْ وَضَرَبْنَا لَكُمُ ٱلْأَمْثَالَ ﴿٤٥﴾}\\
\textamh{46.\  } & \mytextarabic{وَقَدْ مَكَرُوا۟ مَكْرَهُمْ وَعِندَ ٱللَّهِ مَكْرُهُمْ وَإِن كَانَ مَكْرُهُمْ لِتَزُولَ مِنْهُ ٱلْجِبَالُ ﴿٤٦﴾}\\
\textamh{47.\  } & \mytextarabic{فَلَا تَحْسَبَنَّ ٱللَّهَ مُخْلِفَ وَعْدِهِۦ رُسُلَهُۥٓ ۗ إِنَّ ٱللَّهَ عَزِيزٌۭ ذُو ٱنتِقَامٍۢ ﴿٤٧﴾}\\
\textamh{48.\  } & \mytextarabic{يَوْمَ تُبَدَّلُ ٱلْأَرْضُ غَيْرَ ٱلْأَرْضِ وَٱلسَّمَـٰوَٟتُ ۖ وَبَرَزُوا۟ لِلَّهِ ٱلْوَٟحِدِ ٱلْقَهَّارِ ﴿٤٨﴾}\\
\textamh{49.\  } & \mytextarabic{وَتَرَى ٱلْمُجْرِمِينَ يَوْمَئِذٍۢ مُّقَرَّنِينَ فِى ٱلْأَصْفَادِ ﴿٤٩﴾}\\
\textamh{50.\  } & \mytextarabic{سَرَابِيلُهُم مِّن قَطِرَانٍۢ وَتَغْشَىٰ وُجُوهَهُمُ ٱلنَّارُ ﴿٥٠﴾}\\
\textamh{51.\  } & \mytextarabic{لِيَجْزِىَ ٱللَّهُ كُلَّ نَفْسٍۢ مَّا كَسَبَتْ ۚ إِنَّ ٱللَّهَ سَرِيعُ ٱلْحِسَابِ ﴿٥١﴾}\\
\textamh{52.\  } & \mytextarabic{هَـٰذَا بَلَـٰغٌۭ لِّلنَّاسِ وَلِيُنذَرُوا۟ بِهِۦ وَلِيَعْلَمُوٓا۟ أَنَّمَا هُوَ إِلَـٰهٌۭ وَٟحِدٌۭ وَلِيَذَّكَّرَ أُو۟لُوا۟ ٱلْأَلْبَٰبِ ﴿٥٢﴾}\\
\end{longtable}
\clearpage