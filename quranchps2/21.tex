%% License: BSD style (Berkley) (i.e. Put the Copyright owner's name always)
%% Writer and Copyright (to): Bewketu(Bilal) Tadilo (2016-17)
\begin{center}\section{\LR{\textamhsec{ሱራቱ አልአንቢያ -}  \textarabic{سوره  الأنبياء}}}\end{center}
\begin{longtable}{%
  @{}
    p{.5\textwidth}
  @{~~~}
    p{.5\textwidth}
    @{}
}
\textamh{ቢስሚላሂ አራህመኒ ራሂይም } &  \mytextarabic{بِسْمِ ٱللَّهِ ٱلرَّحْمَـٰنِ ٱلرَّحِيمِ}\\
\textamh{1.\  } & \mytextarabic{ ٱقْتَرَبَ لِلنَّاسِ حِسَابُهُمْ وَهُمْ فِى غَفْلَةٍۢ مُّعْرِضُونَ ﴿١﴾}\\
\textamh{2.\  } & \mytextarabic{مَا يَأْتِيهِم مِّن ذِكْرٍۢ مِّن رَّبِّهِم مُّحْدَثٍ إِلَّا ٱسْتَمَعُوهُ وَهُمْ يَلْعَبُونَ ﴿٢﴾}\\
\textamh{3.\  } & \mytextarabic{لَاهِيَةًۭ قُلُوبُهُمْ ۗ وَأَسَرُّوا۟ ٱلنَّجْوَى ٱلَّذِينَ ظَلَمُوا۟ هَلْ هَـٰذَآ إِلَّا بَشَرٌۭ مِّثْلُكُمْ ۖ أَفَتَأْتُونَ ٱلسِّحْرَ وَأَنتُمْ تُبْصِرُونَ ﴿٣﴾}\\
\textamh{4.\  } & \mytextarabic{قَالَ رَبِّى يَعْلَمُ ٱلْقَوْلَ فِى ٱلسَّمَآءِ وَٱلْأَرْضِ ۖ وَهُوَ ٱلسَّمِيعُ ٱلْعَلِيمُ ﴿٤﴾}\\
\textamh{5.\  } & \mytextarabic{بَلْ قَالُوٓا۟ أَضْغَٰثُ أَحْلَـٰمٍۭ بَلِ ٱفْتَرَىٰهُ بَلْ هُوَ شَاعِرٌۭ فَلْيَأْتِنَا بِـَٔايَةٍۢ كَمَآ أُرْسِلَ ٱلْأَوَّلُونَ ﴿٥﴾}\\
\textamh{6.\  } & \mytextarabic{مَآ ءَامَنَتْ قَبْلَهُم مِّن قَرْيَةٍ أَهْلَكْنَـٰهَآ ۖ أَفَهُمْ يُؤْمِنُونَ ﴿٦﴾}\\
\textamh{7.\  } & \mytextarabic{وَمَآ أَرْسَلْنَا قَبْلَكَ إِلَّا رِجَالًۭا نُّوحِىٓ إِلَيْهِمْ ۖ فَسْـَٔلُوٓا۟ أَهْلَ ٱلذِّكْرِ إِن كُنتُمْ لَا تَعْلَمُونَ ﴿٧﴾}\\
\textamh{8.\  } & \mytextarabic{وَمَا جَعَلْنَـٰهُمْ جَسَدًۭا لَّا يَأْكُلُونَ ٱلطَّعَامَ وَمَا كَانُوا۟ خَـٰلِدِينَ ﴿٨﴾}\\
\textamh{9.\  } & \mytextarabic{ثُمَّ صَدَقْنَـٰهُمُ ٱلْوَعْدَ فَأَنجَيْنَـٰهُمْ وَمَن نَّشَآءُ وَأَهْلَكْنَا ٱلْمُسْرِفِينَ ﴿٩﴾}\\
\textamh{10.\  } & \mytextarabic{لَقَدْ أَنزَلْنَآ إِلَيْكُمْ كِتَـٰبًۭا فِيهِ ذِكْرُكُمْ ۖ أَفَلَا تَعْقِلُونَ ﴿١٠﴾}\\
\textamh{11.\  } & \mytextarabic{وَكَمْ قَصَمْنَا مِن قَرْيَةٍۢ كَانَتْ ظَالِمَةًۭ وَأَنشَأْنَا بَعْدَهَا قَوْمًا ءَاخَرِينَ ﴿١١﴾}\\
\textamh{12.\  } & \mytextarabic{فَلَمَّآ أَحَسُّوا۟ بَأْسَنَآ إِذَا هُم مِّنْهَا يَرْكُضُونَ ﴿١٢﴾}\\
\textamh{13.\  } & \mytextarabic{لَا تَرْكُضُوا۟ وَٱرْجِعُوٓا۟ إِلَىٰ مَآ أُتْرِفْتُمْ فِيهِ وَمَسَـٰكِنِكُمْ لَعَلَّكُمْ تُسْـَٔلُونَ ﴿١٣﴾}\\
\textamh{14.\  } & \mytextarabic{قَالُوا۟ يَـٰوَيْلَنَآ إِنَّا كُنَّا ظَـٰلِمِينَ ﴿١٤﴾}\\
\textamh{15.\  } & \mytextarabic{فَمَا زَالَت تِّلْكَ دَعْوَىٰهُمْ حَتَّىٰ جَعَلْنَـٰهُمْ حَصِيدًا خَـٰمِدِينَ ﴿١٥﴾}\\
\textamh{16.\  } & \mytextarabic{وَمَا خَلَقْنَا ٱلسَّمَآءَ وَٱلْأَرْضَ وَمَا بَيْنَهُمَا لَـٰعِبِينَ ﴿١٦﴾}\\
\textamh{17.\  } & \mytextarabic{لَوْ أَرَدْنَآ أَن نَّتَّخِذَ لَهْوًۭا لَّٱتَّخَذْنَـٰهُ مِن لَّدُنَّآ إِن كُنَّا فَـٰعِلِينَ ﴿١٧﴾}\\
\textamh{18.\  } & \mytextarabic{بَلْ نَقْذِفُ بِٱلْحَقِّ عَلَى ٱلْبَٰطِلِ فَيَدْمَغُهُۥ فَإِذَا هُوَ زَاهِقٌۭ ۚ وَلَكُمُ ٱلْوَيْلُ مِمَّا تَصِفُونَ ﴿١٨﴾}\\
\textamh{19.\  } & \mytextarabic{وَلَهُۥ مَن فِى ٱلسَّمَـٰوَٟتِ وَٱلْأَرْضِ ۚ وَمَنْ عِندَهُۥ لَا يَسْتَكْبِرُونَ عَنْ عِبَادَتِهِۦ وَلَا يَسْتَحْسِرُونَ ﴿١٩﴾}\\
\textamh{20.\  } & \mytextarabic{يُسَبِّحُونَ ٱلَّيْلَ وَٱلنَّهَارَ لَا يَفْتُرُونَ ﴿٢٠﴾}\\
\textamh{21.\  } & \mytextarabic{أَمِ ٱتَّخَذُوٓا۟ ءَالِهَةًۭ مِّنَ ٱلْأَرْضِ هُمْ يُنشِرُونَ ﴿٢١﴾}\\
\textamh{22.\  } & \mytextarabic{لَوْ كَانَ فِيهِمَآ ءَالِهَةٌ إِلَّا ٱللَّهُ لَفَسَدَتَا ۚ فَسُبْحَـٰنَ ٱللَّهِ رَبِّ ٱلْعَرْشِ عَمَّا يَصِفُونَ ﴿٢٢﴾}\\
\textamh{23.\  } & \mytextarabic{لَا يُسْـَٔلُ عَمَّا يَفْعَلُ وَهُمْ يُسْـَٔلُونَ ﴿٢٣﴾}\\
\textamh{24.\  } & \mytextarabic{أَمِ ٱتَّخَذُوا۟ مِن دُونِهِۦٓ ءَالِهَةًۭ ۖ قُلْ هَاتُوا۟ بُرْهَـٰنَكُمْ ۖ هَـٰذَا ذِكْرُ مَن مَّعِىَ وَذِكْرُ مَن قَبْلِى ۗ بَلْ أَكْثَرُهُمْ لَا يَعْلَمُونَ ٱلْحَقَّ ۖ فَهُم مُّعْرِضُونَ ﴿٢٤﴾}\\
\textamh{25.\  } & \mytextarabic{وَمَآ أَرْسَلْنَا مِن قَبْلِكَ مِن رَّسُولٍ إِلَّا نُوحِىٓ إِلَيْهِ أَنَّهُۥ لَآ إِلَـٰهَ إِلَّآ أَنَا۠ فَٱعْبُدُونِ ﴿٢٥﴾}\\
\textamh{26.\  } & \mytextarabic{وَقَالُوا۟ ٱتَّخَذَ ٱلرَّحْمَـٰنُ وَلَدًۭا ۗ سُبْحَـٰنَهُۥ ۚ بَلْ عِبَادٌۭ مُّكْرَمُونَ ﴿٢٦﴾}\\
\textamh{27.\  } & \mytextarabic{لَا يَسْبِقُونَهُۥ بِٱلْقَوْلِ وَهُم بِأَمْرِهِۦ يَعْمَلُونَ ﴿٢٧﴾}\\
\textamh{28.\  } & \mytextarabic{يَعْلَمُ مَا بَيْنَ أَيْدِيهِمْ وَمَا خَلْفَهُمْ وَلَا يَشْفَعُونَ إِلَّا لِمَنِ ٱرْتَضَىٰ وَهُم مِّنْ خَشْيَتِهِۦ مُشْفِقُونَ ﴿٢٨﴾}\\
\textamh{29.\  } & \mytextarabic{۞ وَمَن يَقُلْ مِنْهُمْ إِنِّىٓ إِلَـٰهٌۭ مِّن دُونِهِۦ فَذَٟلِكَ نَجْزِيهِ جَهَنَّمَ ۚ كَذَٟلِكَ نَجْزِى ٱلظَّـٰلِمِينَ ﴿٢٩﴾}\\
\textamh{30.\  } & \mytextarabic{أَوَلَمْ يَرَ ٱلَّذِينَ كَفَرُوٓا۟ أَنَّ ٱلسَّمَـٰوَٟتِ وَٱلْأَرْضَ كَانَتَا رَتْقًۭا فَفَتَقْنَـٰهُمَا ۖ وَجَعَلْنَا مِنَ ٱلْمَآءِ كُلَّ شَىْءٍ حَىٍّ ۖ أَفَلَا يُؤْمِنُونَ ﴿٣٠﴾}\\
\textamh{31.\  } & \mytextarabic{وَجَعَلْنَا فِى ٱلْأَرْضِ رَوَٟسِىَ أَن تَمِيدَ بِهِمْ وَجَعَلْنَا فِيهَا فِجَاجًۭا سُبُلًۭا لَّعَلَّهُمْ يَهْتَدُونَ ﴿٣١﴾}\\
\textamh{32.\  } & \mytextarabic{وَجَعَلْنَا ٱلسَّمَآءَ سَقْفًۭا مَّحْفُوظًۭا ۖ وَهُمْ عَنْ ءَايَـٰتِهَا مُعْرِضُونَ ﴿٣٢﴾}\\
\textamh{33.\  } & \mytextarabic{وَهُوَ ٱلَّذِى خَلَقَ ٱلَّيْلَ وَٱلنَّهَارَ وَٱلشَّمْسَ وَٱلْقَمَرَ ۖ كُلٌّۭ فِى فَلَكٍۢ يَسْبَحُونَ ﴿٣٣﴾}\\
\textamh{34.\  } & \mytextarabic{وَمَا جَعَلْنَا لِبَشَرٍۢ مِّن قَبْلِكَ ٱلْخُلْدَ ۖ أَفَإِي۟ن مِّتَّ فَهُمُ ٱلْخَـٰلِدُونَ ﴿٣٤﴾}\\
\textamh{35.\  } & \mytextarabic{كُلُّ نَفْسٍۢ ذَآئِقَةُ ٱلْمَوْتِ ۗ وَنَبْلُوكُم بِٱلشَّرِّ وَٱلْخَيْرِ فِتْنَةًۭ ۖ وَإِلَيْنَا تُرْجَعُونَ ﴿٣٥﴾}\\
\textamh{36.\  } & \mytextarabic{وَإِذَا رَءَاكَ ٱلَّذِينَ كَفَرُوٓا۟ إِن يَتَّخِذُونَكَ إِلَّا هُزُوًا أَهَـٰذَا ٱلَّذِى يَذْكُرُ ءَالِهَتَكُمْ وَهُم بِذِكْرِ ٱلرَّحْمَـٰنِ هُمْ كَـٰفِرُونَ ﴿٣٦﴾}\\
\textamh{37.\  } & \mytextarabic{خُلِقَ ٱلْإِنسَـٰنُ مِنْ عَجَلٍۢ ۚ سَأُو۟رِيكُمْ ءَايَـٰتِى فَلَا تَسْتَعْجِلُونِ ﴿٣٧﴾}\\
\textamh{38.\  } & \mytextarabic{وَيَقُولُونَ مَتَىٰ هَـٰذَا ٱلْوَعْدُ إِن كُنتُمْ صَـٰدِقِينَ ﴿٣٨﴾}\\
\textamh{39.\  } & \mytextarabic{لَوْ يَعْلَمُ ٱلَّذِينَ كَفَرُوا۟ حِينَ لَا يَكُفُّونَ عَن وُجُوهِهِمُ ٱلنَّارَ وَلَا عَن ظُهُورِهِمْ وَلَا هُمْ يُنصَرُونَ ﴿٣٩﴾}\\
\textamh{40.\  } & \mytextarabic{بَلْ تَأْتِيهِم بَغْتَةًۭ فَتَبْهَتُهُمْ فَلَا يَسْتَطِيعُونَ رَدَّهَا وَلَا هُمْ يُنظَرُونَ ﴿٤٠﴾}\\
\textamh{41.\  } & \mytextarabic{وَلَقَدِ ٱسْتُهْزِئَ بِرُسُلٍۢ مِّن قَبْلِكَ فَحَاقَ بِٱلَّذِينَ سَخِرُوا۟ مِنْهُم مَّا كَانُوا۟ بِهِۦ يَسْتَهْزِءُونَ ﴿٤١﴾}\\
\textamh{42.\  } & \mytextarabic{قُلْ مَن يَكْلَؤُكُم بِٱلَّيْلِ وَٱلنَّهَارِ مِنَ ٱلرَّحْمَـٰنِ ۗ بَلْ هُمْ عَن ذِكْرِ رَبِّهِم مُّعْرِضُونَ ﴿٤٢﴾}\\
\textamh{43.\  } & \mytextarabic{أَمْ لَهُمْ ءَالِهَةٌۭ تَمْنَعُهُم مِّن دُونِنَا ۚ لَا يَسْتَطِيعُونَ نَصْرَ أَنفُسِهِمْ وَلَا هُم مِّنَّا يُصْحَبُونَ ﴿٤٣﴾}\\
\textamh{44.\  } & \mytextarabic{بَلْ مَتَّعْنَا هَـٰٓؤُلَآءِ وَءَابَآءَهُمْ حَتَّىٰ طَالَ عَلَيْهِمُ ٱلْعُمُرُ ۗ أَفَلَا يَرَوْنَ أَنَّا نَأْتِى ٱلْأَرْضَ نَنقُصُهَا مِنْ أَطْرَافِهَآ ۚ أَفَهُمُ ٱلْغَٰلِبُونَ ﴿٤٤﴾}\\
\textamh{45.\  } & \mytextarabic{قُلْ إِنَّمَآ أُنذِرُكُم بِٱلْوَحْىِ ۚ وَلَا يَسْمَعُ ٱلصُّمُّ ٱلدُّعَآءَ إِذَا مَا يُنذَرُونَ ﴿٤٥﴾}\\
\textamh{46.\  } & \mytextarabic{وَلَئِن مَّسَّتْهُمْ نَفْحَةٌۭ مِّنْ عَذَابِ رَبِّكَ لَيَقُولُنَّ يَـٰوَيْلَنَآ إِنَّا كُنَّا ظَـٰلِمِينَ ﴿٤٦﴾}\\
\textamh{47.\  } & \mytextarabic{وَنَضَعُ ٱلْمَوَٟزِينَ ٱلْقِسْطَ لِيَوْمِ ٱلْقِيَـٰمَةِ فَلَا تُظْلَمُ نَفْسٌۭ شَيْـًۭٔا ۖ وَإِن كَانَ مِثْقَالَ حَبَّةٍۢ مِّنْ خَرْدَلٍ أَتَيْنَا بِهَا ۗ وَكَفَىٰ بِنَا حَـٰسِبِينَ ﴿٤٧﴾}\\
\textamh{48.\  } & \mytextarabic{وَلَقَدْ ءَاتَيْنَا مُوسَىٰ وَهَـٰرُونَ ٱلْفُرْقَانَ وَضِيَآءًۭ وَذِكْرًۭا لِّلْمُتَّقِينَ ﴿٤٨﴾}\\
\textamh{49.\  } & \mytextarabic{ٱلَّذِينَ يَخْشَوْنَ رَبَّهُم بِٱلْغَيْبِ وَهُم مِّنَ ٱلسَّاعَةِ مُشْفِقُونَ ﴿٤٩﴾}\\
\textamh{50.\  } & \mytextarabic{وَهَـٰذَا ذِكْرٌۭ مُّبَارَكٌ أَنزَلْنَـٰهُ ۚ أَفَأَنتُمْ لَهُۥ مُنكِرُونَ ﴿٥٠﴾}\\
\textamh{51.\  } & \mytextarabic{۞ وَلَقَدْ ءَاتَيْنَآ إِبْرَٰهِيمَ رُشْدَهُۥ مِن قَبْلُ وَكُنَّا بِهِۦ عَـٰلِمِينَ ﴿٥١﴾}\\
\textamh{52.\  } & \mytextarabic{إِذْ قَالَ لِأَبِيهِ وَقَوْمِهِۦ مَا هَـٰذِهِ ٱلتَّمَاثِيلُ ٱلَّتِىٓ أَنتُمْ لَهَا عَـٰكِفُونَ ﴿٥٢﴾}\\
\textamh{53.\  } & \mytextarabic{قَالُوا۟ وَجَدْنَآ ءَابَآءَنَا لَهَا عَـٰبِدِينَ ﴿٥٣﴾}\\
\textamh{54.\  } & \mytextarabic{قَالَ لَقَدْ كُنتُمْ أَنتُمْ وَءَابَآؤُكُمْ فِى ضَلَـٰلٍۢ مُّبِينٍۢ ﴿٥٤﴾}\\
\textamh{55.\  } & \mytextarabic{قَالُوٓا۟ أَجِئْتَنَا بِٱلْحَقِّ أَمْ أَنتَ مِنَ ٱللَّٰعِبِينَ ﴿٥٥﴾}\\
\textamh{56.\  } & \mytextarabic{قَالَ بَل رَّبُّكُمْ رَبُّ ٱلسَّمَـٰوَٟتِ وَٱلْأَرْضِ ٱلَّذِى فَطَرَهُنَّ وَأَنَا۠ عَلَىٰ ذَٟلِكُم مِّنَ ٱلشَّـٰهِدِينَ ﴿٥٦﴾}\\
\textamh{57.\  } & \mytextarabic{وَتَٱللَّهِ لَأَكِيدَنَّ أَصْنَـٰمَكُم بَعْدَ أَن تُوَلُّوا۟ مُدْبِرِينَ ﴿٥٧﴾}\\
\textamh{58.\  } & \mytextarabic{فَجَعَلَهُمْ جُذَٟذًا إِلَّا كَبِيرًۭا لَّهُمْ لَعَلَّهُمْ إِلَيْهِ يَرْجِعُونَ ﴿٥٨﴾}\\
\textamh{59.\  } & \mytextarabic{قَالُوا۟ مَن فَعَلَ هَـٰذَا بِـَٔالِهَتِنَآ إِنَّهُۥ لَمِنَ ٱلظَّـٰلِمِينَ ﴿٥٩﴾}\\
\textamh{60.\  } & \mytextarabic{قَالُوا۟ سَمِعْنَا فَتًۭى يَذْكُرُهُمْ يُقَالُ لَهُۥٓ إِبْرَٰهِيمُ ﴿٦٠﴾}\\
\textamh{61.\  } & \mytextarabic{قَالُوا۟ فَأْتُوا۟ بِهِۦ عَلَىٰٓ أَعْيُنِ ٱلنَّاسِ لَعَلَّهُمْ يَشْهَدُونَ ﴿٦١﴾}\\
\textamh{62.\  } & \mytextarabic{قَالُوٓا۟ ءَأَنتَ فَعَلْتَ هَـٰذَا بِـَٔالِهَتِنَا يَـٰٓإِبْرَٰهِيمُ ﴿٦٢﴾}\\
\textamh{63.\  } & \mytextarabic{قَالَ بَلْ فَعَلَهُۥ كَبِيرُهُمْ هَـٰذَا فَسْـَٔلُوهُمْ إِن كَانُوا۟ يَنطِقُونَ ﴿٦٣﴾}\\
\textamh{64.\  } & \mytextarabic{فَرَجَعُوٓا۟ إِلَىٰٓ أَنفُسِهِمْ فَقَالُوٓا۟ إِنَّكُمْ أَنتُمُ ٱلظَّـٰلِمُونَ ﴿٦٤﴾}\\
\textamh{65.\  } & \mytextarabic{ثُمَّ نُكِسُوا۟ عَلَىٰ رُءُوسِهِمْ لَقَدْ عَلِمْتَ مَا هَـٰٓؤُلَآءِ يَنطِقُونَ ﴿٦٥﴾}\\
\textamh{66.\  } & \mytextarabic{قَالَ أَفَتَعْبُدُونَ مِن دُونِ ٱللَّهِ مَا لَا يَنفَعُكُمْ شَيْـًۭٔا وَلَا يَضُرُّكُمْ ﴿٦٦﴾}\\
\textamh{67.\  } & \mytextarabic{أُفٍّۢ لَّكُمْ وَلِمَا تَعْبُدُونَ مِن دُونِ ٱللَّهِ ۖ أَفَلَا تَعْقِلُونَ ﴿٦٧﴾}\\
\textamh{68.\  } & \mytextarabic{قَالُوا۟ حَرِّقُوهُ وَٱنصُرُوٓا۟ ءَالِهَتَكُمْ إِن كُنتُمْ فَـٰعِلِينَ ﴿٦٨﴾}\\
\textamh{69.\  } & \mytextarabic{قُلْنَا يَـٰنَارُ كُونِى بَرْدًۭا وَسَلَـٰمًا عَلَىٰٓ إِبْرَٰهِيمَ ﴿٦٩﴾}\\
\textamh{70.\  } & \mytextarabic{وَأَرَادُوا۟ بِهِۦ كَيْدًۭا فَجَعَلْنَـٰهُمُ ٱلْأَخْسَرِينَ ﴿٧٠﴾}\\
\textamh{71.\  } & \mytextarabic{وَنَجَّيْنَـٰهُ وَلُوطًا إِلَى ٱلْأَرْضِ ٱلَّتِى بَٰرَكْنَا فِيهَا لِلْعَـٰلَمِينَ ﴿٧١﴾}\\
\textamh{72.\  } & \mytextarabic{وَوَهَبْنَا لَهُۥٓ إِسْحَـٰقَ وَيَعْقُوبَ نَافِلَةًۭ ۖ وَكُلًّۭا جَعَلْنَا صَـٰلِحِينَ ﴿٧٢﴾}\\
\textamh{73.\  } & \mytextarabic{وَجَعَلْنَـٰهُمْ أَئِمَّةًۭ يَهْدُونَ بِأَمْرِنَا وَأَوْحَيْنَآ إِلَيْهِمْ فِعْلَ ٱلْخَيْرَٰتِ وَإِقَامَ ٱلصَّلَوٰةِ وَإِيتَآءَ ٱلزَّكَوٰةِ ۖ وَكَانُوا۟ لَنَا عَـٰبِدِينَ ﴿٧٣﴾}\\
\textamh{74.\  } & \mytextarabic{وَلُوطًا ءَاتَيْنَـٰهُ حُكْمًۭا وَعِلْمًۭا وَنَجَّيْنَـٰهُ مِنَ ٱلْقَرْيَةِ ٱلَّتِى كَانَت تَّعْمَلُ ٱلْخَبَٰٓئِثَ ۗ إِنَّهُمْ كَانُوا۟ قَوْمَ سَوْءٍۢ فَـٰسِقِينَ ﴿٧٤﴾}\\
\textamh{75.\  } & \mytextarabic{وَأَدْخَلْنَـٰهُ فِى رَحْمَتِنَآ ۖ إِنَّهُۥ مِنَ ٱلصَّـٰلِحِينَ ﴿٧٥﴾}\\
\textamh{76.\  } & \mytextarabic{وَنُوحًا إِذْ نَادَىٰ مِن قَبْلُ فَٱسْتَجَبْنَا لَهُۥ فَنَجَّيْنَـٰهُ وَأَهْلَهُۥ مِنَ ٱلْكَرْبِ ٱلْعَظِيمِ ﴿٧٦﴾}\\
\textamh{77.\  } & \mytextarabic{وَنَصَرْنَـٰهُ مِنَ ٱلْقَوْمِ ٱلَّذِينَ كَذَّبُوا۟ بِـَٔايَـٰتِنَآ ۚ إِنَّهُمْ كَانُوا۟ قَوْمَ سَوْءٍۢ فَأَغْرَقْنَـٰهُمْ أَجْمَعِينَ ﴿٧٧﴾}\\
\textamh{78.\  } & \mytextarabic{وَدَاوُۥدَ وَسُلَيْمَـٰنَ إِذْ يَحْكُمَانِ فِى ٱلْحَرْثِ إِذْ نَفَشَتْ فِيهِ غَنَمُ ٱلْقَوْمِ وَكُنَّا لِحُكْمِهِمْ شَـٰهِدِينَ ﴿٧٨﴾}\\
\textamh{79.\  } & \mytextarabic{فَفَهَّمْنَـٰهَا سُلَيْمَـٰنَ ۚ وَكُلًّا ءَاتَيْنَا حُكْمًۭا وَعِلْمًۭا ۚ وَسَخَّرْنَا مَعَ دَاوُۥدَ ٱلْجِبَالَ يُسَبِّحْنَ وَٱلطَّيْرَ ۚ وَكُنَّا فَـٰعِلِينَ ﴿٧٩﴾}\\
\textamh{80.\  } & \mytextarabic{وَعَلَّمْنَـٰهُ صَنْعَةَ لَبُوسٍۢ لَّكُمْ لِتُحْصِنَكُم مِّنۢ بَأْسِكُمْ ۖ فَهَلْ أَنتُمْ شَـٰكِرُونَ ﴿٨٠﴾}\\
\textamh{81.\  } & \mytextarabic{وَلِسُلَيْمَـٰنَ ٱلرِّيحَ عَاصِفَةًۭ تَجْرِى بِأَمْرِهِۦٓ إِلَى ٱلْأَرْضِ ٱلَّتِى بَٰرَكْنَا فِيهَا ۚ وَكُنَّا بِكُلِّ شَىْءٍ عَـٰلِمِينَ ﴿٨١﴾}\\
\textamh{82.\  } & \mytextarabic{وَمِنَ ٱلشَّيَـٰطِينِ مَن يَغُوصُونَ لَهُۥ وَيَعْمَلُونَ عَمَلًۭا دُونَ ذَٟلِكَ ۖ وَكُنَّا لَهُمْ حَـٰفِظِينَ ﴿٨٢﴾}\\
\textamh{83.\  } & \mytextarabic{۞ وَأَيُّوبَ إِذْ نَادَىٰ رَبَّهُۥٓ أَنِّى مَسَّنِىَ ٱلضُّرُّ وَأَنتَ أَرْحَمُ ٱلرَّٟحِمِينَ ﴿٨٣﴾}\\
\textamh{84.\  } & \mytextarabic{فَٱسْتَجَبْنَا لَهُۥ فَكَشَفْنَا مَا بِهِۦ مِن ضُرٍّۢ ۖ وَءَاتَيْنَـٰهُ أَهْلَهُۥ وَمِثْلَهُم مَّعَهُمْ رَحْمَةًۭ مِّنْ عِندِنَا وَذِكْرَىٰ لِلْعَـٰبِدِينَ ﴿٨٤﴾}\\
\textamh{85.\  } & \mytextarabic{وَإِسْمَـٰعِيلَ وَإِدْرِيسَ وَذَا ٱلْكِفْلِ ۖ كُلٌّۭ مِّنَ ٱلصَّـٰبِرِينَ ﴿٨٥﴾}\\
\textamh{86.\  } & \mytextarabic{وَأَدْخَلْنَـٰهُمْ فِى رَحْمَتِنَآ ۖ إِنَّهُم مِّنَ ٱلصَّـٰلِحِينَ ﴿٨٦﴾}\\
\textamh{87.\  } & \mytextarabic{وَذَا ٱلنُّونِ إِذ ذَّهَبَ مُغَٰضِبًۭا فَظَنَّ أَن لَّن نَّقْدِرَ عَلَيْهِ فَنَادَىٰ فِى ٱلظُّلُمَـٰتِ أَن لَّآ إِلَـٰهَ إِلَّآ أَنتَ سُبْحَـٰنَكَ إِنِّى كُنتُ مِنَ ٱلظَّـٰلِمِينَ ﴿٨٧﴾}\\
\textamh{88.\  } & \mytextarabic{فَٱسْتَجَبْنَا لَهُۥ وَنَجَّيْنَـٰهُ مِنَ ٱلْغَمِّ ۚ وَكَذَٟلِكَ نُۨجِى ٱلْمُؤْمِنِينَ ﴿٨٨﴾}\\
\textamh{89.\  } & \mytextarabic{وَزَكَرِيَّآ إِذْ نَادَىٰ رَبَّهُۥ رَبِّ لَا تَذَرْنِى فَرْدًۭا وَأَنتَ خَيْرُ ٱلْوَٟرِثِينَ ﴿٨٩﴾}\\
\textamh{90.\  } & \mytextarabic{فَٱسْتَجَبْنَا لَهُۥ وَوَهَبْنَا لَهُۥ يَحْيَىٰ وَأَصْلَحْنَا لَهُۥ زَوْجَهُۥٓ ۚ إِنَّهُمْ كَانُوا۟ يُسَـٰرِعُونَ فِى ٱلْخَيْرَٰتِ وَيَدْعُونَنَا رَغَبًۭا وَرَهَبًۭا ۖ وَكَانُوا۟ لَنَا خَـٰشِعِينَ ﴿٩٠﴾}\\
\textamh{91.\  } & \mytextarabic{وَٱلَّتِىٓ أَحْصَنَتْ فَرْجَهَا فَنَفَخْنَا فِيهَا مِن رُّوحِنَا وَجَعَلْنَـٰهَا وَٱبْنَهَآ ءَايَةًۭ لِّلْعَـٰلَمِينَ ﴿٩١﴾}\\
\textamh{92.\  } & \mytextarabic{إِنَّ هَـٰذِهِۦٓ أُمَّتُكُمْ أُمَّةًۭ وَٟحِدَةًۭ وَأَنَا۠ رَبُّكُمْ فَٱعْبُدُونِ ﴿٩٢﴾}\\
\textamh{93.\  } & \mytextarabic{وَتَقَطَّعُوٓا۟ أَمْرَهُم بَيْنَهُمْ ۖ كُلٌّ إِلَيْنَا رَٰجِعُونَ ﴿٩٣﴾}\\
\textamh{94.\  } & \mytextarabic{فَمَن يَعْمَلْ مِنَ ٱلصَّـٰلِحَـٰتِ وَهُوَ مُؤْمِنٌۭ فَلَا كُفْرَانَ لِسَعْيِهِۦ وَإِنَّا لَهُۥ كَـٰتِبُونَ ﴿٩٤﴾}\\
\textamh{95.\  } & \mytextarabic{وَحَرَٰمٌ عَلَىٰ قَرْيَةٍ أَهْلَكْنَـٰهَآ أَنَّهُمْ لَا يَرْجِعُونَ ﴿٩٥﴾}\\
\textamh{96.\  } & \mytextarabic{حَتَّىٰٓ إِذَا فُتِحَتْ يَأْجُوجُ وَمَأْجُوجُ وَهُم مِّن كُلِّ حَدَبٍۢ يَنسِلُونَ ﴿٩٦﴾}\\
\textamh{97.\  } & \mytextarabic{وَٱقْتَرَبَ ٱلْوَعْدُ ٱلْحَقُّ فَإِذَا هِىَ شَـٰخِصَةٌ أَبْصَـٰرُ ٱلَّذِينَ كَفَرُوا۟ يَـٰوَيْلَنَا قَدْ كُنَّا فِى غَفْلَةٍۢ مِّنْ هَـٰذَا بَلْ كُنَّا ظَـٰلِمِينَ ﴿٩٧﴾}\\
\textamh{98.\  } & \mytextarabic{إِنَّكُمْ وَمَا تَعْبُدُونَ مِن دُونِ ٱللَّهِ حَصَبُ جَهَنَّمَ أَنتُمْ لَهَا وَٟرِدُونَ ﴿٩٨﴾}\\
\textamh{99.\  } & \mytextarabic{لَوْ كَانَ هَـٰٓؤُلَآءِ ءَالِهَةًۭ مَّا وَرَدُوهَا ۖ وَكُلٌّۭ فِيهَا خَـٰلِدُونَ ﴿٩٩﴾}\\
\textamh{100.\  } & \mytextarabic{لَهُمْ فِيهَا زَفِيرٌۭ وَهُمْ فِيهَا لَا يَسْمَعُونَ ﴿١٠٠﴾}\\
\textamh{101.\  } & \mytextarabic{إِنَّ ٱلَّذِينَ سَبَقَتْ لَهُم مِّنَّا ٱلْحُسْنَىٰٓ أُو۟لَـٰٓئِكَ عَنْهَا مُبْعَدُونَ ﴿١٠١﴾}\\
\textamh{102.\  } & \mytextarabic{لَا يَسْمَعُونَ حَسِيسَهَا ۖ وَهُمْ فِى مَا ٱشْتَهَتْ أَنفُسُهُمْ خَـٰلِدُونَ ﴿١٠٢﴾}\\
\textamh{103.\  } & \mytextarabic{لَا يَحْزُنُهُمُ ٱلْفَزَعُ ٱلْأَكْبَرُ وَتَتَلَقَّىٰهُمُ ٱلْمَلَـٰٓئِكَةُ هَـٰذَا يَوْمُكُمُ ٱلَّذِى كُنتُمْ تُوعَدُونَ ﴿١٠٣﴾}\\
\textamh{104.\  } & \mytextarabic{يَوْمَ نَطْوِى ٱلسَّمَآءَ كَطَىِّ ٱلسِّجِلِّ لِلْكُتُبِ ۚ كَمَا بَدَأْنَآ أَوَّلَ خَلْقٍۢ نُّعِيدُهُۥ ۚ وَعْدًا عَلَيْنَآ ۚ إِنَّا كُنَّا فَـٰعِلِينَ ﴿١٠٤﴾}\\
\textamh{105.\  } & \mytextarabic{وَلَقَدْ كَتَبْنَا فِى ٱلزَّبُورِ مِنۢ بَعْدِ ٱلذِّكْرِ أَنَّ ٱلْأَرْضَ يَرِثُهَا عِبَادِىَ ٱلصَّـٰلِحُونَ ﴿١٠٥﴾}\\
\textamh{106.\  } & \mytextarabic{إِنَّ فِى هَـٰذَا لَبَلَـٰغًۭا لِّقَوْمٍ عَـٰبِدِينَ ﴿١٠٦﴾}\\
\textamh{107.\  } & \mytextarabic{وَمَآ أَرْسَلْنَـٰكَ إِلَّا رَحْمَةًۭ لِّلْعَـٰلَمِينَ ﴿١٠٧﴾}\\
\textamh{108.\  } & \mytextarabic{قُلْ إِنَّمَا يُوحَىٰٓ إِلَىَّ أَنَّمَآ إِلَـٰهُكُمْ إِلَـٰهٌۭ وَٟحِدٌۭ ۖ فَهَلْ أَنتُم مُّسْلِمُونَ ﴿١٠٨﴾}\\
\textamh{109.\  } & \mytextarabic{فَإِن تَوَلَّوْا۟ فَقُلْ ءَاذَنتُكُمْ عَلَىٰ سَوَآءٍۢ ۖ وَإِنْ أَدْرِىٓ أَقَرِيبٌ أَم بَعِيدٌۭ مَّا تُوعَدُونَ ﴿١٠٩﴾}\\
\textamh{110.\  } & \mytextarabic{إِنَّهُۥ يَعْلَمُ ٱلْجَهْرَ مِنَ ٱلْقَوْلِ وَيَعْلَمُ مَا تَكْتُمُونَ ﴿١١٠﴾}\\
\textamh{111.\  } & \mytextarabic{وَإِنْ أَدْرِى لَعَلَّهُۥ فِتْنَةٌۭ لَّكُمْ وَمَتَـٰعٌ إِلَىٰ حِينٍۢ ﴿١١١﴾}\\
\textamh{112.\  } & \mytextarabic{قَـٰلَ رَبِّ ٱحْكُم بِٱلْحَقِّ ۗ وَرَبُّنَا ٱلرَّحْمَـٰنُ ٱلْمُسْتَعَانُ عَلَىٰ مَا تَصِفُونَ ﴿١١٢﴾}\\
\end{longtable}
\clearpage