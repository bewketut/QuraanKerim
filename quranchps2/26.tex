%% License: BSD style (Berkley) (i.e. Put the Copyright owner's name always)
%% Writer and Copyright (to): Bewketu(Bilal) Tadilo (2016-17)
\centering\section{\LR{\textamharic{ሱራቱ አሹኣራኣ -}  \RL{سوره  الشعراء}}}
\begin{longtable}{%
  @{}
    p{.5\textwidth}
  @{~~~~~~~~~~~~~}
    p{.5\textwidth}
    @{}
}
\nopagebreak
\textamh{\ \ \ \ \ \  ቢስሚላሂ አራህመኒ ራሂይም } &  بِسْمِ ٱللَّهِ ٱلرَّحْمَـٰنِ ٱلرَّحِيمِ\\
\textamh{1.\  } &  طسٓمٓ ﴿١﴾\\
\textamh{2.\  } & تِلْكَ ءَايَـٰتُ ٱلْكِتَـٰبِ ٱلْمُبِينِ ﴿٢﴾\\
\textamh{3.\  } & لَعَلَّكَ بَٰخِعٌۭ نَّفْسَكَ أَلَّا يَكُونُوا۟ مُؤْمِنِينَ ﴿٣﴾\\
\textamh{4.\  } & إِن نَّشَأْ نُنَزِّلْ عَلَيْهِم مِّنَ ٱلسَّمَآءِ ءَايَةًۭ فَظَلَّتْ أَعْنَـٰقُهُمْ لَهَا خَـٰضِعِينَ ﴿٤﴾\\
\textamh{5.\  } & وَمَا يَأْتِيهِم مِّن ذِكْرٍۢ مِّنَ ٱلرَّحْمَـٰنِ مُحْدَثٍ إِلَّا كَانُوا۟ عَنْهُ مُعْرِضِينَ ﴿٥﴾\\
\textamh{6.\  } & فَقَدْ كَذَّبُوا۟ فَسَيَأْتِيهِمْ أَنۢبَٰٓؤُا۟ مَا كَانُوا۟ بِهِۦ يَسْتَهْزِءُونَ ﴿٦﴾\\
\textamh{7.\  } & أَوَلَمْ يَرَوْا۟ إِلَى ٱلْأَرْضِ كَمْ أَنۢبَتْنَا فِيهَا مِن كُلِّ زَوْجٍۢ كَرِيمٍ ﴿٧﴾\\
\textamh{8.\  } & إِنَّ فِى ذَٟلِكَ لَءَايَةًۭ ۖ وَمَا كَانَ أَكْثَرُهُم مُّؤْمِنِينَ ﴿٨﴾\\
\textamh{9.\  } & وَإِنَّ رَبَّكَ لَهُوَ ٱلْعَزِيزُ ٱلرَّحِيمُ ﴿٩﴾\\
\textamh{10.\  } & وَإِذْ نَادَىٰ رَبُّكَ مُوسَىٰٓ أَنِ ٱئْتِ ٱلْقَوْمَ ٱلظَّـٰلِمِينَ ﴿١٠﴾\\
\textamh{11.\  } & قَوْمَ فِرْعَوْنَ ۚ أَلَا يَتَّقُونَ ﴿١١﴾\\
\textamh{12.\  } & قَالَ رَبِّ إِنِّىٓ أَخَافُ أَن يُكَذِّبُونِ ﴿١٢﴾\\
\textamh{13.\  } & وَيَضِيقُ صَدْرِى وَلَا يَنطَلِقُ لِسَانِى فَأَرْسِلْ إِلَىٰ هَـٰرُونَ ﴿١٣﴾\\
\textamh{14.\  } & وَلَهُمْ عَلَىَّ ذَنۢبٌۭ فَأَخَافُ أَن يَقْتُلُونِ ﴿١٤﴾\\
\textamh{15.\  } & قَالَ كَلَّا ۖ فَٱذْهَبَا بِـَٔايَـٰتِنَآ ۖ إِنَّا مَعَكُم مُّسْتَمِعُونَ ﴿١٥﴾\\
\textamh{16.\  } & فَأْتِيَا فِرْعَوْنَ فَقُولَآ إِنَّا رَسُولُ رَبِّ ٱلْعَـٰلَمِينَ ﴿١٦﴾\\
\textamh{17.\  } & أَنْ أَرْسِلْ مَعَنَا بَنِىٓ إِسْرَٰٓءِيلَ ﴿١٧﴾\\
\textamh{18.\  } & قَالَ أَلَمْ نُرَبِّكَ فِينَا وَلِيدًۭا وَلَبِثْتَ فِينَا مِنْ عُمُرِكَ سِنِينَ ﴿١٨﴾\\
\textamh{19.\  } & وَفَعَلْتَ فَعْلَتَكَ ٱلَّتِى فَعَلْتَ وَأَنتَ مِنَ ٱلْكَـٰفِرِينَ ﴿١٩﴾\\
\textamh{20.\  } & قَالَ فَعَلْتُهَآ إِذًۭا وَأَنَا۠ مِنَ ٱلضَّآلِّينَ ﴿٢٠﴾\\
\textamh{21.\  } & فَفَرَرْتُ مِنكُمْ لَمَّا خِفْتُكُمْ فَوَهَبَ لِى رَبِّى حُكْمًۭا وَجَعَلَنِى مِنَ ٱلْمُرْسَلِينَ ﴿٢١﴾\\
\textamh{22.\  } & وَتِلْكَ نِعْمَةٌۭ تَمُنُّهَا عَلَىَّ أَنْ عَبَّدتَّ بَنِىٓ إِسْرَٰٓءِيلَ ﴿٢٢﴾\\
\textamh{23.\  } & قَالَ فِرْعَوْنُ وَمَا رَبُّ ٱلْعَـٰلَمِينَ ﴿٢٣﴾\\
\textamh{24.\  } & قَالَ رَبُّ ٱلسَّمَـٰوَٟتِ وَٱلْأَرْضِ وَمَا بَيْنَهُمَآ ۖ إِن كُنتُم مُّوقِنِينَ ﴿٢٤﴾\\
\textamh{25.\  } & قَالَ لِمَنْ حَوْلَهُۥٓ أَلَا تَسْتَمِعُونَ ﴿٢٥﴾\\
\textamh{26.\  } & قَالَ رَبُّكُمْ وَرَبُّ ءَابَآئِكُمُ ٱلْأَوَّلِينَ ﴿٢٦﴾\\
\textamh{27.\  } & قَالَ إِنَّ رَسُولَكُمُ ٱلَّذِىٓ أُرْسِلَ إِلَيْكُمْ لَمَجْنُونٌۭ ﴿٢٧﴾\\
\textamh{28.\  } & قَالَ رَبُّ ٱلْمَشْرِقِ وَٱلْمَغْرِبِ وَمَا بَيْنَهُمَآ ۖ إِن كُنتُمْ تَعْقِلُونَ ﴿٢٨﴾\\
\textamh{29.\  } & قَالَ لَئِنِ ٱتَّخَذْتَ إِلَـٰهًا غَيْرِى لَأَجْعَلَنَّكَ مِنَ ٱلْمَسْجُونِينَ ﴿٢٩﴾\\
\textamh{30.\  } & قَالَ أَوَلَوْ جِئْتُكَ بِشَىْءٍۢ مُّبِينٍۢ ﴿٣٠﴾\\
\textamh{31.\  } & قَالَ فَأْتِ بِهِۦٓ إِن كُنتَ مِنَ ٱلصَّـٰدِقِينَ ﴿٣١﴾\\
\textamh{32.\  } & فَأَلْقَىٰ عَصَاهُ فَإِذَا هِىَ ثُعْبَانٌۭ مُّبِينٌۭ ﴿٣٢﴾\\
\textamh{33.\  } & وَنَزَعَ يَدَهُۥ فَإِذَا هِىَ بَيْضَآءُ لِلنَّـٰظِرِينَ ﴿٣٣﴾\\
\textamh{34.\  } & قَالَ لِلْمَلَإِ حَوْلَهُۥٓ إِنَّ هَـٰذَا لَسَـٰحِرٌ عَلِيمٌۭ ﴿٣٤﴾\\
\textamh{35.\  } & يُرِيدُ أَن يُخْرِجَكُم مِّنْ أَرْضِكُم بِسِحْرِهِۦ فَمَاذَا تَأْمُرُونَ ﴿٣٥﴾\\
\textamh{36.\  } & قَالُوٓا۟ أَرْجِهْ وَأَخَاهُ وَٱبْعَثْ فِى ٱلْمَدَآئِنِ حَـٰشِرِينَ ﴿٣٦﴾\\
\textamh{37.\  } & يَأْتُوكَ بِكُلِّ سَحَّارٍ عَلِيمٍۢ ﴿٣٧﴾\\
\textamh{38.\  } & فَجُمِعَ ٱلسَّحَرَةُ لِمِيقَـٰتِ يَوْمٍۢ مَّعْلُومٍۢ ﴿٣٨﴾\\
\textamh{39.\  } & وَقِيلَ لِلنَّاسِ هَلْ أَنتُم مُّجْتَمِعُونَ ﴿٣٩﴾\\
\textamh{40.\  } & لَعَلَّنَا نَتَّبِعُ ٱلسَّحَرَةَ إِن كَانُوا۟ هُمُ ٱلْغَٰلِبِينَ ﴿٤٠﴾\\
\textamh{41.\  } & فَلَمَّا جَآءَ ٱلسَّحَرَةُ قَالُوا۟ لِفِرْعَوْنَ أَئِنَّ لَنَا لَأَجْرًا إِن كُنَّا نَحْنُ ٱلْغَٰلِبِينَ ﴿٤١﴾\\
\textamh{42.\  } & قَالَ نَعَمْ وَإِنَّكُمْ إِذًۭا لَّمِنَ ٱلْمُقَرَّبِينَ ﴿٤٢﴾\\
\textamh{43.\  } & قَالَ لَهُم مُّوسَىٰٓ أَلْقُوا۟ مَآ أَنتُم مُّلْقُونَ ﴿٤٣﴾\\
\textamh{44.\  } & فَأَلْقَوْا۟ حِبَالَهُمْ وَعِصِيَّهُمْ وَقَالُوا۟ بِعِزَّةِ فِرْعَوْنَ إِنَّا لَنَحْنُ ٱلْغَٰلِبُونَ ﴿٤٤﴾\\
\textamh{45.\  } & فَأَلْقَىٰ مُوسَىٰ عَصَاهُ فَإِذَا هِىَ تَلْقَفُ مَا يَأْفِكُونَ ﴿٤٥﴾\\
\textamh{46.\  } & فَأُلْقِىَ ٱلسَّحَرَةُ سَـٰجِدِينَ ﴿٤٦﴾\\
\textamh{47.\  } & قَالُوٓا۟ ءَامَنَّا بِرَبِّ ٱلْعَـٰلَمِينَ ﴿٤٧﴾\\
\textamh{48.\  } & رَبِّ مُوسَىٰ وَهَـٰرُونَ ﴿٤٨﴾\\
\textamh{49.\  } & قَالَ ءَامَنتُمْ لَهُۥ قَبْلَ أَنْ ءَاذَنَ لَكُمْ ۖ إِنَّهُۥ لَكَبِيرُكُمُ ٱلَّذِى عَلَّمَكُمُ ٱلسِّحْرَ فَلَسَوْفَ تَعْلَمُونَ ۚ لَأُقَطِّعَنَّ أَيْدِيَكُمْ وَأَرْجُلَكُم مِّنْ خِلَـٰفٍۢ وَلَأُصَلِّبَنَّكُمْ أَجْمَعِينَ ﴿٤٩﴾\\
\textamh{50.\  } & قَالُوا۟ لَا ضَيْرَ ۖ إِنَّآ إِلَىٰ رَبِّنَا مُنقَلِبُونَ ﴿٥٠﴾\\
\textamh{51.\  } & إِنَّا نَطْمَعُ أَن يَغْفِرَ لَنَا رَبُّنَا خَطَٰيَـٰنَآ أَن كُنَّآ أَوَّلَ ٱلْمُؤْمِنِينَ ﴿٥١﴾\\
\textamh{52.\  } & ۞ وَأَوْحَيْنَآ إِلَىٰ مُوسَىٰٓ أَنْ أَسْرِ بِعِبَادِىٓ إِنَّكُم مُّتَّبَعُونَ ﴿٥٢﴾\\
\textamh{53.\  } & فَأَرْسَلَ فِرْعَوْنُ فِى ٱلْمَدَآئِنِ حَـٰشِرِينَ ﴿٥٣﴾\\
\textamh{54.\  } & إِنَّ هَـٰٓؤُلَآءِ لَشِرْذِمَةٌۭ قَلِيلُونَ ﴿٥٤﴾\\
\textamh{55.\  } & وَإِنَّهُمْ لَنَا لَغَآئِظُونَ ﴿٥٥﴾\\
\textamh{56.\  } & وَإِنَّا لَجَمِيعٌ حَـٰذِرُونَ ﴿٥٦﴾\\
\textamh{57.\  } & فَأَخْرَجْنَـٰهُم مِّن جَنَّـٰتٍۢ وَعُيُونٍۢ ﴿٥٧﴾\\
\textamh{58.\  } & وَكُنُوزٍۢ وَمَقَامٍۢ كَرِيمٍۢ ﴿٥٨﴾\\
\textamh{59.\  } & كَذَٟلِكَ وَأَوْرَثْنَـٰهَا بَنِىٓ إِسْرَٰٓءِيلَ ﴿٥٩﴾\\
\textamh{60.\  } & فَأَتْبَعُوهُم مُّشْرِقِينَ ﴿٦٠﴾\\
\textamh{61.\  } & فَلَمَّا تَرَٰٓءَا ٱلْجَمْعَانِ قَالَ أَصْحَـٰبُ مُوسَىٰٓ إِنَّا لَمُدْرَكُونَ ﴿٦١﴾\\
\textamh{62.\  } & قَالَ كَلَّآ ۖ إِنَّ مَعِىَ رَبِّى سَيَهْدِينِ ﴿٦٢﴾\\
\textamh{63.\  } & فَأَوْحَيْنَآ إِلَىٰ مُوسَىٰٓ أَنِ ٱضْرِب بِّعَصَاكَ ٱلْبَحْرَ ۖ فَٱنفَلَقَ فَكَانَ كُلُّ فِرْقٍۢ كَٱلطَّوْدِ ٱلْعَظِيمِ ﴿٦٣﴾\\
\textamh{64.\  } & وَأَزْلَفْنَا ثَمَّ ٱلْءَاخَرِينَ ﴿٦٤﴾\\
\textamh{65.\  } & وَأَنجَيْنَا مُوسَىٰ وَمَن مَّعَهُۥٓ أَجْمَعِينَ ﴿٦٥﴾\\
\textamh{66.\  } & ثُمَّ أَغْرَقْنَا ٱلْءَاخَرِينَ ﴿٦٦﴾\\
\textamh{67.\  } & إِنَّ فِى ذَٟلِكَ لَءَايَةًۭ ۖ وَمَا كَانَ أَكْثَرُهُم مُّؤْمِنِينَ ﴿٦٧﴾\\
\textamh{68.\  } & وَإِنَّ رَبَّكَ لَهُوَ ٱلْعَزِيزُ ٱلرَّحِيمُ ﴿٦٨﴾\\
\textamh{69.\  } & وَٱتْلُ عَلَيْهِمْ نَبَأَ إِبْرَٰهِيمَ ﴿٦٩﴾\\
\textamh{70.\  } & إِذْ قَالَ لِأَبِيهِ وَقَوْمِهِۦ مَا تَعْبُدُونَ ﴿٧٠﴾\\
\textamh{71.\  } & قَالُوا۟ نَعْبُدُ أَصْنَامًۭا فَنَظَلُّ لَهَا عَـٰكِفِينَ ﴿٧١﴾\\
\textamh{72.\  } & قَالَ هَلْ يَسْمَعُونَكُمْ إِذْ تَدْعُونَ ﴿٧٢﴾\\
\textamh{73.\  } & أَوْ يَنفَعُونَكُمْ أَوْ يَضُرُّونَ ﴿٧٣﴾\\
\textamh{74.\  } & قَالُوا۟ بَلْ وَجَدْنَآ ءَابَآءَنَا كَذَٟلِكَ يَفْعَلُونَ ﴿٧٤﴾\\
\textamh{75.\  } & قَالَ أَفَرَءَيْتُم مَّا كُنتُمْ تَعْبُدُونَ ﴿٧٥﴾\\
\textamh{76.\  } & أَنتُمْ وَءَابَآؤُكُمُ ٱلْأَقْدَمُونَ ﴿٧٦﴾\\
\textamh{77.\  } & فَإِنَّهُمْ عَدُوٌّۭ لِّىٓ إِلَّا رَبَّ ٱلْعَـٰلَمِينَ ﴿٧٧﴾\\
\textamh{78.\  } & ٱلَّذِى خَلَقَنِى فَهُوَ يَهْدِينِ ﴿٧٨﴾\\
\textamh{79.\  } & وَٱلَّذِى هُوَ يُطْعِمُنِى وَيَسْقِينِ ﴿٧٩﴾\\
\textamh{80.\  } & وَإِذَا مَرِضْتُ فَهُوَ يَشْفِينِ ﴿٨٠﴾\\
\textamh{81.\  } & وَٱلَّذِى يُمِيتُنِى ثُمَّ يُحْيِينِ ﴿٨١﴾\\
\textamh{82.\  } & وَٱلَّذِىٓ أَطْمَعُ أَن يَغْفِرَ لِى خَطِيٓـَٔتِى يَوْمَ ٱلدِّينِ ﴿٨٢﴾\\
\textamh{83.\  } & رَبِّ هَبْ لِى حُكْمًۭا وَأَلْحِقْنِى بِٱلصَّـٰلِحِينَ ﴿٨٣﴾\\
\textamh{84.\  } & وَٱجْعَل لِّى لِسَانَ صِدْقٍۢ فِى ٱلْءَاخِرِينَ ﴿٨٤﴾\\
\textamh{85.\  } & وَٱجْعَلْنِى مِن وَرَثَةِ جَنَّةِ ٱلنَّعِيمِ ﴿٨٥﴾\\
\textamh{86.\  } & وَٱغْفِرْ لِأَبِىٓ إِنَّهُۥ كَانَ مِنَ ٱلضَّآلِّينَ ﴿٨٦﴾\\
\textamh{87.\  } & وَلَا تُخْزِنِى يَوْمَ يُبْعَثُونَ ﴿٨٧﴾\\
\textamh{88.\  } & يَوْمَ لَا يَنفَعُ مَالٌۭ وَلَا بَنُونَ ﴿٨٨﴾\\
\textamh{89.\  } & إِلَّا مَنْ أَتَى ٱللَّهَ بِقَلْبٍۢ سَلِيمٍۢ ﴿٨٩﴾\\
\textamh{90.\  } & وَأُزْلِفَتِ ٱلْجَنَّةُ لِلْمُتَّقِينَ ﴿٩٠﴾\\
\textamh{91.\  } & وَبُرِّزَتِ ٱلْجَحِيمُ لِلْغَاوِينَ ﴿٩١﴾\\
\textamh{92.\  } & وَقِيلَ لَهُمْ أَيْنَ مَا كُنتُمْ تَعْبُدُونَ ﴿٩٢﴾\\
\textamh{93.\  } & مِن دُونِ ٱللَّهِ هَلْ يَنصُرُونَكُمْ أَوْ يَنتَصِرُونَ ﴿٩٣﴾\\
\textamh{94.\  } & فَكُبْكِبُوا۟ فِيهَا هُمْ وَٱلْغَاوُۥنَ ﴿٩٤﴾\\
\textamh{95.\  } & وَجُنُودُ إِبْلِيسَ أَجْمَعُونَ ﴿٩٥﴾\\
\textamh{96.\  } & قَالُوا۟ وَهُمْ فِيهَا يَخْتَصِمُونَ ﴿٩٦﴾\\
\textamh{97.\  } & تَٱللَّهِ إِن كُنَّا لَفِى ضَلَـٰلٍۢ مُّبِينٍ ﴿٩٧﴾\\
\textamh{98.\  } & إِذْ نُسَوِّيكُم بِرَبِّ ٱلْعَـٰلَمِينَ ﴿٩٨﴾\\
\textamh{99.\  } & وَمَآ أَضَلَّنَآ إِلَّا ٱلْمُجْرِمُونَ ﴿٩٩﴾\\
\textamh{100.\  } & فَمَا لَنَا مِن شَـٰفِعِينَ ﴿١٠٠﴾\\
\textamh{101.\  } & وَلَا صَدِيقٍ حَمِيمٍۢ ﴿١٠١﴾\\
\textamh{102.\  } & فَلَوْ أَنَّ لَنَا كَرَّةًۭ فَنَكُونَ مِنَ ٱلْمُؤْمِنِينَ ﴿١٠٢﴾\\
\textamh{103.\  } & إِنَّ فِى ذَٟلِكَ لَءَايَةًۭ ۖ وَمَا كَانَ أَكْثَرُهُم مُّؤْمِنِينَ ﴿١٠٣﴾\\
\textamh{104.\  } & وَإِنَّ رَبَّكَ لَهُوَ ٱلْعَزِيزُ ٱلرَّحِيمُ ﴿١٠٤﴾\\
\textamh{105.\  } & كَذَّبَتْ قَوْمُ نُوحٍ ٱلْمُرْسَلِينَ ﴿١٠٥﴾\\
\textamh{106.\  } & إِذْ قَالَ لَهُمْ أَخُوهُمْ نُوحٌ أَلَا تَتَّقُونَ ﴿١٠٦﴾\\
\textamh{107.\  } & إِنِّى لَكُمْ رَسُولٌ أَمِينٌۭ ﴿١٠٧﴾\\
\textamh{108.\  } & فَٱتَّقُوا۟ ٱللَّهَ وَأَطِيعُونِ ﴿١٠٨﴾\\
\textamh{109.\  } & وَمَآ أَسْـَٔلُكُمْ عَلَيْهِ مِنْ أَجْرٍ ۖ إِنْ أَجْرِىَ إِلَّا عَلَىٰ رَبِّ ٱلْعَـٰلَمِينَ ﴿١٠٩﴾\\
\textamh{110.\  } & فَٱتَّقُوا۟ ٱللَّهَ وَأَطِيعُونِ ﴿١١٠﴾\\
\textamh{111.\  } & ۞ قَالُوٓا۟ أَنُؤْمِنُ لَكَ وَٱتَّبَعَكَ ٱلْأَرْذَلُونَ ﴿١١١﴾\\
\textamh{112.\  } & قَالَ وَمَا عِلْمِى بِمَا كَانُوا۟ يَعْمَلُونَ ﴿١١٢﴾\\
\textamh{113.\  } & إِنْ حِسَابُهُمْ إِلَّا عَلَىٰ رَبِّى ۖ لَوْ تَشْعُرُونَ ﴿١١٣﴾\\
\textamh{114.\  } & وَمَآ أَنَا۠ بِطَارِدِ ٱلْمُؤْمِنِينَ ﴿١١٤﴾\\
\textamh{115.\  } & إِنْ أَنَا۠ إِلَّا نَذِيرٌۭ مُّبِينٌۭ ﴿١١٥﴾\\
\textamh{116.\  } & قَالُوا۟ لَئِن لَّمْ تَنتَهِ يَـٰنُوحُ لَتَكُونَنَّ مِنَ ٱلْمَرْجُومِينَ ﴿١١٦﴾\\
\textamh{117.\  } & قَالَ رَبِّ إِنَّ قَوْمِى كَذَّبُونِ ﴿١١٧﴾\\
\textamh{118.\  } & فَٱفْتَحْ بَيْنِى وَبَيْنَهُمْ فَتْحًۭا وَنَجِّنِى وَمَن مَّعِىَ مِنَ ٱلْمُؤْمِنِينَ ﴿١١٨﴾\\
\textamh{119.\  } & فَأَنجَيْنَـٰهُ وَمَن مَّعَهُۥ فِى ٱلْفُلْكِ ٱلْمَشْحُونِ ﴿١١٩﴾\\
\textamh{120.\  } & ثُمَّ أَغْرَقْنَا بَعْدُ ٱلْبَاقِينَ ﴿١٢٠﴾\\
\textamh{121.\  } & إِنَّ فِى ذَٟلِكَ لَءَايَةًۭ ۖ وَمَا كَانَ أَكْثَرُهُم مُّؤْمِنِينَ ﴿١٢١﴾\\
\textamh{122.\  } & وَإِنَّ رَبَّكَ لَهُوَ ٱلْعَزِيزُ ٱلرَّحِيمُ ﴿١٢٢﴾\\
\textamh{123.\  } & كَذَّبَتْ عَادٌ ٱلْمُرْسَلِينَ ﴿١٢٣﴾\\
\textamh{124.\  } & إِذْ قَالَ لَهُمْ أَخُوهُمْ هُودٌ أَلَا تَتَّقُونَ ﴿١٢٤﴾\\
\textamh{125.\  } & إِنِّى لَكُمْ رَسُولٌ أَمِينٌۭ ﴿١٢٥﴾\\
\textamh{126.\  } & فَٱتَّقُوا۟ ٱللَّهَ وَأَطِيعُونِ ﴿١٢٦﴾\\
\textamh{127.\  } & وَمَآ أَسْـَٔلُكُمْ عَلَيْهِ مِنْ أَجْرٍ ۖ إِنْ أَجْرِىَ إِلَّا عَلَىٰ رَبِّ ٱلْعَـٰلَمِينَ ﴿١٢٧﴾\\
\textamh{128.\  } & أَتَبْنُونَ بِكُلِّ رِيعٍ ءَايَةًۭ تَعْبَثُونَ ﴿١٢٨﴾\\
\textamh{129.\  } & وَتَتَّخِذُونَ مَصَانِعَ لَعَلَّكُمْ تَخْلُدُونَ ﴿١٢٩﴾\\
\textamh{130.\  } & وَإِذَا بَطَشْتُم بَطَشْتُمْ جَبَّارِينَ ﴿١٣٠﴾\\
\textamh{131.\  } & فَٱتَّقُوا۟ ٱللَّهَ وَأَطِيعُونِ ﴿١٣١﴾\\
\textamh{132.\  } & وَٱتَّقُوا۟ ٱلَّذِىٓ أَمَدَّكُم بِمَا تَعْلَمُونَ ﴿١٣٢﴾\\
\textamh{133.\  } & أَمَدَّكُم بِأَنْعَـٰمٍۢ وَبَنِينَ ﴿١٣٣﴾\\
\textamh{134.\  } & وَجَنَّـٰتٍۢ وَعُيُونٍ ﴿١٣٤﴾\\
\textamh{135.\  } & إِنِّىٓ أَخَافُ عَلَيْكُمْ عَذَابَ يَوْمٍ عَظِيمٍۢ ﴿١٣٥﴾\\
\textamh{136.\  } & قَالُوا۟ سَوَآءٌ عَلَيْنَآ أَوَعَظْتَ أَمْ لَمْ تَكُن مِّنَ ٱلْوَٟعِظِينَ ﴿١٣٦﴾\\
\textamh{137.\  } & إِنْ هَـٰذَآ إِلَّا خُلُقُ ٱلْأَوَّلِينَ ﴿١٣٧﴾\\
\textamh{138.\  } & وَمَا نَحْنُ بِمُعَذَّبِينَ ﴿١٣٨﴾\\
\textamh{139.\  } & فَكَذَّبُوهُ فَأَهْلَكْنَـٰهُمْ ۗ إِنَّ فِى ذَٟلِكَ لَءَايَةًۭ ۖ وَمَا كَانَ أَكْثَرُهُم مُّؤْمِنِينَ ﴿١٣٩﴾\\
\textamh{140.\  } & وَإِنَّ رَبَّكَ لَهُوَ ٱلْعَزِيزُ ٱلرَّحِيمُ ﴿١٤٠﴾\\
\textamh{141.\  } & كَذَّبَتْ ثَمُودُ ٱلْمُرْسَلِينَ ﴿١٤١﴾\\
\textamh{142.\  } & إِذْ قَالَ لَهُمْ أَخُوهُمْ صَـٰلِحٌ أَلَا تَتَّقُونَ ﴿١٤٢﴾\\
\textamh{143.\  } & إِنِّى لَكُمْ رَسُولٌ أَمِينٌۭ ﴿١٤٣﴾\\
\textamh{144.\  } & فَٱتَّقُوا۟ ٱللَّهَ وَأَطِيعُونِ ﴿١٤٤﴾\\
\textamh{145.\  } & وَمَآ أَسْـَٔلُكُمْ عَلَيْهِ مِنْ أَجْرٍ ۖ إِنْ أَجْرِىَ إِلَّا عَلَىٰ رَبِّ ٱلْعَـٰلَمِينَ ﴿١٤٥﴾\\
\textamh{146.\  } & أَتُتْرَكُونَ فِى مَا هَـٰهُنَآ ءَامِنِينَ ﴿١٤٦﴾\\
\textamh{147.\  } & فِى جَنَّـٰتٍۢ وَعُيُونٍۢ ﴿١٤٧﴾\\
\textamh{148.\  } & وَزُرُوعٍۢ وَنَخْلٍۢ طَلْعُهَا هَضِيمٌۭ ﴿١٤٨﴾\\
\textamh{149.\  } & وَتَنْحِتُونَ مِنَ ٱلْجِبَالِ بُيُوتًۭا فَـٰرِهِينَ ﴿١٤٩﴾\\
\textamh{150.\  } & فَٱتَّقُوا۟ ٱللَّهَ وَأَطِيعُونِ ﴿١٥٠﴾\\
\textamh{151.\  } & وَلَا تُطِيعُوٓا۟ أَمْرَ ٱلْمُسْرِفِينَ ﴿١٥١﴾\\
\textamh{152.\  } & ٱلَّذِينَ يُفْسِدُونَ فِى ٱلْأَرْضِ وَلَا يُصْلِحُونَ ﴿١٥٢﴾\\
\textamh{153.\  } & قَالُوٓا۟ إِنَّمَآ أَنتَ مِنَ ٱلْمُسَحَّرِينَ ﴿١٥٣﴾\\
\textamh{154.\  } & مَآ أَنتَ إِلَّا بَشَرٌۭ مِّثْلُنَا فَأْتِ بِـَٔايَةٍ إِن كُنتَ مِنَ ٱلصَّـٰدِقِينَ ﴿١٥٤﴾\\
\textamh{155.\  } & قَالَ هَـٰذِهِۦ نَاقَةٌۭ لَّهَا شِرْبٌۭ وَلَكُمْ شِرْبُ يَوْمٍۢ مَّعْلُومٍۢ ﴿١٥٥﴾\\
\textamh{156.\  } & وَلَا تَمَسُّوهَا بِسُوٓءٍۢ فَيَأْخُذَكُمْ عَذَابُ يَوْمٍ عَظِيمٍۢ ﴿١٥٦﴾\\
\textamh{157.\  } & فَعَقَرُوهَا فَأَصْبَحُوا۟ نَـٰدِمِينَ ﴿١٥٧﴾\\
\textamh{158.\  } & فَأَخَذَهُمُ ٱلْعَذَابُ ۗ إِنَّ فِى ذَٟلِكَ لَءَايَةًۭ ۖ وَمَا كَانَ أَكْثَرُهُم مُّؤْمِنِينَ ﴿١٥٨﴾\\
\textamh{159.\  } & وَإِنَّ رَبَّكَ لَهُوَ ٱلْعَزِيزُ ٱلرَّحِيمُ ﴿١٥٩﴾\\
\textamh{160.\  } & كَذَّبَتْ قَوْمُ لُوطٍ ٱلْمُرْسَلِينَ ﴿١٦٠﴾\\
\textamh{161.\  } & إِذْ قَالَ لَهُمْ أَخُوهُمْ لُوطٌ أَلَا تَتَّقُونَ ﴿١٦١﴾\\
\textamh{162.\  } & إِنِّى لَكُمْ رَسُولٌ أَمِينٌۭ ﴿١٦٢﴾\\
\textamh{163.\  } & فَٱتَّقُوا۟ ٱللَّهَ وَأَطِيعُونِ ﴿١٦٣﴾\\
\textamh{164.\  } & وَمَآ أَسْـَٔلُكُمْ عَلَيْهِ مِنْ أَجْرٍ ۖ إِنْ أَجْرِىَ إِلَّا عَلَىٰ رَبِّ ٱلْعَـٰلَمِينَ ﴿١٦٤﴾\\
\textamh{165.\  } & أَتَأْتُونَ ٱلذُّكْرَانَ مِنَ ٱلْعَـٰلَمِينَ ﴿١٦٥﴾\\
\textamh{166.\  } & وَتَذَرُونَ مَا خَلَقَ لَكُمْ رَبُّكُم مِّنْ أَزْوَٟجِكُم ۚ بَلْ أَنتُمْ قَوْمٌ عَادُونَ ﴿١٦٦﴾\\
\textamh{167.\  } & قَالُوا۟ لَئِن لَّمْ تَنتَهِ يَـٰلُوطُ لَتَكُونَنَّ مِنَ ٱلْمُخْرَجِينَ ﴿١٦٧﴾\\
\textamh{168.\  } & قَالَ إِنِّى لِعَمَلِكُم مِّنَ ٱلْقَالِينَ ﴿١٦٨﴾\\
\textamh{169.\  } & رَبِّ نَجِّنِى وَأَهْلِى مِمَّا يَعْمَلُونَ ﴿١٦٩﴾\\
\textamh{170.\  } & فَنَجَّيْنَـٰهُ وَأَهْلَهُۥٓ أَجْمَعِينَ ﴿١٧٠﴾\\
\textamh{171.\  } & إِلَّا عَجُوزًۭا فِى ٱلْغَٰبِرِينَ ﴿١٧١﴾\\
\textamh{172.\  } & ثُمَّ دَمَّرْنَا ٱلْءَاخَرِينَ ﴿١٧٢﴾\\
\textamh{173.\  } & وَأَمْطَرْنَا عَلَيْهِم مَّطَرًۭا ۖ فَسَآءَ مَطَرُ ٱلْمُنذَرِينَ ﴿١٧٣﴾\\
\textamh{174.\  } & إِنَّ فِى ذَٟلِكَ لَءَايَةًۭ ۖ وَمَا كَانَ أَكْثَرُهُم مُّؤْمِنِينَ ﴿١٧٤﴾\\
\textamh{175.\  } & وَإِنَّ رَبَّكَ لَهُوَ ٱلْعَزِيزُ ٱلرَّحِيمُ ﴿١٧٥﴾\\
\textamh{176.\  } & كَذَّبَ أَصْحَـٰبُ لْـَٔيْكَةِ ٱلْمُرْسَلِينَ ﴿١٧٦﴾\\
\textamh{177.\  } & إِذْ قَالَ لَهُمْ شُعَيْبٌ أَلَا تَتَّقُونَ ﴿١٧٧﴾\\
\textamh{178.\  } & إِنِّى لَكُمْ رَسُولٌ أَمِينٌۭ ﴿١٧٨﴾\\
\textamh{179.\  } & فَٱتَّقُوا۟ ٱللَّهَ وَأَطِيعُونِ ﴿١٧٩﴾\\
\textamh{180.\  } & وَمَآ أَسْـَٔلُكُمْ عَلَيْهِ مِنْ أَجْرٍ ۖ إِنْ أَجْرِىَ إِلَّا عَلَىٰ رَبِّ ٱلْعَـٰلَمِينَ ﴿١٨٠﴾\\
\textamh{181.\  } & ۞ أَوْفُوا۟ ٱلْكَيْلَ وَلَا تَكُونُوا۟ مِنَ ٱلْمُخْسِرِينَ ﴿١٨١﴾\\
\textamh{182.\  } & وَزِنُوا۟ بِٱلْقِسْطَاسِ ٱلْمُسْتَقِيمِ ﴿١٨٢﴾\\
\textamh{183.\  } & وَلَا تَبْخَسُوا۟ ٱلنَّاسَ أَشْيَآءَهُمْ وَلَا تَعْثَوْا۟ فِى ٱلْأَرْضِ مُفْسِدِينَ ﴿١٨٣﴾\\
\textamh{184.\  } & وَٱتَّقُوا۟ ٱلَّذِى خَلَقَكُمْ وَٱلْجِبِلَّةَ ٱلْأَوَّلِينَ ﴿١٨٤﴾\\
\textamh{185.\  } & قَالُوٓا۟ إِنَّمَآ أَنتَ مِنَ ٱلْمُسَحَّرِينَ ﴿١٨٥﴾\\
\textamh{186.\  } & وَمَآ أَنتَ إِلَّا بَشَرٌۭ مِّثْلُنَا وَإِن نَّظُنُّكَ لَمِنَ ٱلْكَـٰذِبِينَ ﴿١٨٦﴾\\
\textamh{187.\  } & فَأَسْقِطْ عَلَيْنَا كِسَفًۭا مِّنَ ٱلسَّمَآءِ إِن كُنتَ مِنَ ٱلصَّـٰدِقِينَ ﴿١٨٧﴾\\
\textamh{188.\  } & قَالَ رَبِّىٓ أَعْلَمُ بِمَا تَعْمَلُونَ ﴿١٨٨﴾\\
\textamh{189.\  } & فَكَذَّبُوهُ فَأَخَذَهُمْ عَذَابُ يَوْمِ ٱلظُّلَّةِ ۚ إِنَّهُۥ كَانَ عَذَابَ يَوْمٍ عَظِيمٍ ﴿١٨٩﴾\\
\textamh{190.\  } & إِنَّ فِى ذَٟلِكَ لَءَايَةًۭ ۖ وَمَا كَانَ أَكْثَرُهُم مُّؤْمِنِينَ ﴿١٩٠﴾\\
\textamh{191.\  } & وَإِنَّ رَبَّكَ لَهُوَ ٱلْعَزِيزُ ٱلرَّحِيمُ ﴿١٩١﴾\\
\textamh{192.\  } & وَإِنَّهُۥ لَتَنزِيلُ رَبِّ ٱلْعَـٰلَمِينَ ﴿١٩٢﴾\\
\textamh{193.\  } & نَزَلَ بِهِ ٱلرُّوحُ ٱلْأَمِينُ ﴿١٩٣﴾\\
\textamh{194.\  } & عَلَىٰ قَلْبِكَ لِتَكُونَ مِنَ ٱلْمُنذِرِينَ ﴿١٩٤﴾\\
\textamh{195.\  } & بِلِسَانٍ عَرَبِىٍّۢ مُّبِينٍۢ ﴿١٩٥﴾\\
\textamh{196.\  } & وَإِنَّهُۥ لَفِى زُبُرِ ٱلْأَوَّلِينَ ﴿١٩٦﴾\\
\textamh{197.\  } & أَوَلَمْ يَكُن لَّهُمْ ءَايَةً أَن يَعْلَمَهُۥ عُلَمَـٰٓؤُا۟ بَنِىٓ إِسْرَٰٓءِيلَ ﴿١٩٧﴾\\
\textamh{198.\  } & وَلَوْ نَزَّلْنَـٰهُ عَلَىٰ بَعْضِ ٱلْأَعْجَمِينَ ﴿١٩٨﴾\\
\textamh{199.\  } & فَقَرَأَهُۥ عَلَيْهِم مَّا كَانُوا۟ بِهِۦ مُؤْمِنِينَ ﴿١٩٩﴾\\
\textamh{200.\  } & كَذَٟلِكَ سَلَكْنَـٰهُ فِى قُلُوبِ ٱلْمُجْرِمِينَ ﴿٢٠٠﴾\\
\textamh{201.\  } & لَا يُؤْمِنُونَ بِهِۦ حَتَّىٰ يَرَوُا۟ ٱلْعَذَابَ ٱلْأَلِيمَ ﴿٢٠١﴾\\
\textamh{202.\  } & فَيَأْتِيَهُم بَغْتَةًۭ وَهُمْ لَا يَشْعُرُونَ ﴿٢٠٢﴾\\
\textamh{203.\  } & فَيَقُولُوا۟ هَلْ نَحْنُ مُنظَرُونَ ﴿٢٠٣﴾\\
\textamh{204.\  } & أَفَبِعَذَابِنَا يَسْتَعْجِلُونَ ﴿٢٠٤﴾\\
\textamh{205.\  } & أَفَرَءَيْتَ إِن مَّتَّعْنَـٰهُمْ سِنِينَ ﴿٢٠٥﴾\\
\textamh{206.\  } & ثُمَّ جَآءَهُم مَّا كَانُوا۟ يُوعَدُونَ ﴿٢٠٦﴾\\
\textamh{207.\  } & مَآ أَغْنَىٰ عَنْهُم مَّا كَانُوا۟ يُمَتَّعُونَ ﴿٢٠٧﴾\\
\textamh{208.\  } & وَمَآ أَهْلَكْنَا مِن قَرْيَةٍ إِلَّا لَهَا مُنذِرُونَ ﴿٢٠٨﴾\\
\textamh{209.\  } & ذِكْرَىٰ وَمَا كُنَّا ظَـٰلِمِينَ ﴿٢٠٩﴾\\
\textamh{210.\  } & وَمَا تَنَزَّلَتْ بِهِ ٱلشَّيَـٰطِينُ ﴿٢١٠﴾\\
\textamh{211.\  } & وَمَا يَنۢبَغِى لَهُمْ وَمَا يَسْتَطِيعُونَ ﴿٢١١﴾\\
\textamh{212.\  } & إِنَّهُمْ عَنِ ٱلسَّمْعِ لَمَعْزُولُونَ ﴿٢١٢﴾\\
\textamh{213.\  } & فَلَا تَدْعُ مَعَ ٱللَّهِ إِلَـٰهًا ءَاخَرَ فَتَكُونَ مِنَ ٱلْمُعَذَّبِينَ ﴿٢١٣﴾\\
\textamh{214.\  } & وَأَنذِرْ عَشِيرَتَكَ ٱلْأَقْرَبِينَ ﴿٢١٤﴾\\
\textamh{215.\  } & وَٱخْفِضْ جَنَاحَكَ لِمَنِ ٱتَّبَعَكَ مِنَ ٱلْمُؤْمِنِينَ ﴿٢١٥﴾\\
\textamh{216.\  } & فَإِنْ عَصَوْكَ فَقُلْ إِنِّى بَرِىٓءٌۭ مِّمَّا تَعْمَلُونَ ﴿٢١٦﴾\\
\textamh{217.\  } & وَتَوَكَّلْ عَلَى ٱلْعَزِيزِ ٱلرَّحِيمِ ﴿٢١٧﴾\\
\textamh{218.\  } & ٱلَّذِى يَرَىٰكَ حِينَ تَقُومُ ﴿٢١٨﴾\\
\textamh{219.\  } & وَتَقَلُّبَكَ فِى ٱلسَّٰجِدِينَ ﴿٢١٩﴾\\
\textamh{220.\  } & إِنَّهُۥ هُوَ ٱلسَّمِيعُ ٱلْعَلِيمُ ﴿٢٢٠﴾\\
\textamh{221.\  } & هَلْ أُنَبِّئُكُمْ عَلَىٰ مَن تَنَزَّلُ ٱلشَّيَـٰطِينُ ﴿٢٢١﴾\\
\textamh{222.\  } & تَنَزَّلُ عَلَىٰ كُلِّ أَفَّاكٍ أَثِيمٍۢ ﴿٢٢٢﴾\\
\textamh{223.\  } & يُلْقُونَ ٱلسَّمْعَ وَأَكْثَرُهُمْ كَـٰذِبُونَ ﴿٢٢٣﴾\\
\textamh{224.\  } & وَٱلشُّعَرَآءُ يَتَّبِعُهُمُ ٱلْغَاوُۥنَ ﴿٢٢٤﴾\\
\textamh{225.\  } & أَلَمْ تَرَ أَنَّهُمْ فِى كُلِّ وَادٍۢ يَهِيمُونَ ﴿٢٢٥﴾\\
\textamh{226.\  } & وَأَنَّهُمْ يَقُولُونَ مَا لَا يَفْعَلُونَ ﴿٢٢٦﴾\\
\textamh{227.\  } & إِلَّا ٱلَّذِينَ ءَامَنُوا۟ وَعَمِلُوا۟ ٱلصَّـٰلِحَـٰتِ وَذَكَرُوا۟ ٱللَّهَ كَثِيرًۭا وَٱنتَصَرُوا۟ مِنۢ بَعْدِ مَا ظُلِمُوا۟ ۗ وَسَيَعْلَمُ ٱلَّذِينَ ظَلَمُوٓا۟ أَىَّ مُنقَلَبٍۢ يَنقَلِبُونَ ﴿٢٢٧﴾\\
\end{longtable} \newpage
