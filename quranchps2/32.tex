%% License: BSD style (Berkley) (i.e. Put the Copyright owner's name always)
%% Writer and Copyright (to): Bewketu(Bilal) Tadilo (2016-17)
\begin{center}\section{\LR{\textamhsec{ሱራቱ አስሰጅደ -}  \textarabic{سوره  السجدة}}}\end{center}
\begin{longtable}{%
  @{}
    p{.5\textwidth}
  @{~~~}
    p{.5\textwidth}
    @{}
}
\textamh{ቢስሚላሂ አራህመኒ ራሂይም } &  \mytextarabic{بِسْمِ ٱللَّهِ ٱلرَّحْمَـٰنِ ٱلرَّحِيمِ}\\
\textamh{1.\  } & \mytextarabic{ الٓمٓ ﴿١﴾}\\
\textamh{2.\  } & \mytextarabic{تَنزِيلُ ٱلْكِتَـٰبِ لَا رَيْبَ فِيهِ مِن رَّبِّ ٱلْعَـٰلَمِينَ ﴿٢﴾}\\
\textamh{3.\  } & \mytextarabic{أَمْ يَقُولُونَ ٱفْتَرَىٰهُ ۚ بَلْ هُوَ ٱلْحَقُّ مِن رَّبِّكَ لِتُنذِرَ قَوْمًۭا مَّآ أَتَىٰهُم مِّن نَّذِيرٍۢ مِّن قَبْلِكَ لَعَلَّهُمْ يَهْتَدُونَ ﴿٣﴾}\\
\textamh{4.\  } & \mytextarabic{ٱللَّهُ ٱلَّذِى خَلَقَ ٱلسَّمَـٰوَٟتِ وَٱلْأَرْضَ وَمَا بَيْنَهُمَا فِى سِتَّةِ أَيَّامٍۢ ثُمَّ ٱسْتَوَىٰ عَلَى ٱلْعَرْشِ ۖ مَا لَكُم مِّن دُونِهِۦ مِن وَلِىٍّۢ وَلَا شَفِيعٍ ۚ أَفَلَا تَتَذَكَّرُونَ ﴿٤﴾}\\
\textamh{5.\  } & \mytextarabic{يُدَبِّرُ ٱلْأَمْرَ مِنَ ٱلسَّمَآءِ إِلَى ٱلْأَرْضِ ثُمَّ يَعْرُجُ إِلَيْهِ فِى يَوْمٍۢ كَانَ مِقْدَارُهُۥٓ أَلْفَ سَنَةٍۢ مِّمَّا تَعُدُّونَ ﴿٥﴾}\\
\textamh{6.\  } & \mytextarabic{ذَٟلِكَ عَـٰلِمُ ٱلْغَيْبِ وَٱلشَّهَـٰدَةِ ٱلْعَزِيزُ ٱلرَّحِيمُ ﴿٦﴾}\\
\textamh{7.\  } & \mytextarabic{ٱلَّذِىٓ أَحْسَنَ كُلَّ شَىْءٍ خَلَقَهُۥ ۖ وَبَدَأَ خَلْقَ ٱلْإِنسَـٰنِ مِن طِينٍۢ ﴿٧﴾}\\
\textamh{8.\  } & \mytextarabic{ثُمَّ جَعَلَ نَسْلَهُۥ مِن سُلَـٰلَةٍۢ مِّن مَّآءٍۢ مَّهِينٍۢ ﴿٨﴾}\\
\textamh{9.\  } & \mytextarabic{ثُمَّ سَوَّىٰهُ وَنَفَخَ فِيهِ مِن رُّوحِهِۦ ۖ وَجَعَلَ لَكُمُ ٱلسَّمْعَ وَٱلْأَبْصَـٰرَ وَٱلْأَفْـِٔدَةَ ۚ قَلِيلًۭا مَّا تَشْكُرُونَ ﴿٩﴾}\\
\textamh{10.\  } & \mytextarabic{وَقَالُوٓا۟ أَءِذَا ضَلَلْنَا فِى ٱلْأَرْضِ أَءِنَّا لَفِى خَلْقٍۢ جَدِيدٍۭ ۚ بَلْ هُم بِلِقَآءِ رَبِّهِمْ كَـٰفِرُونَ ﴿١٠﴾}\\
\textamh{11.\  } & \mytextarabic{۞ قُلْ يَتَوَفَّىٰكُم مَّلَكُ ٱلْمَوْتِ ٱلَّذِى وُكِّلَ بِكُمْ ثُمَّ إِلَىٰ رَبِّكُمْ تُرْجَعُونَ ﴿١١﴾}\\
\textamh{12.\  } & \mytextarabic{وَلَوْ تَرَىٰٓ إِذِ ٱلْمُجْرِمُونَ نَاكِسُوا۟ رُءُوسِهِمْ عِندَ رَبِّهِمْ رَبَّنَآ أَبْصَرْنَا وَسَمِعْنَا فَٱرْجِعْنَا نَعْمَلْ صَـٰلِحًا إِنَّا مُوقِنُونَ ﴿١٢﴾}\\
\textamh{13.\  } & \mytextarabic{وَلَوْ شِئْنَا لَءَاتَيْنَا كُلَّ نَفْسٍ هُدَىٰهَا وَلَـٰكِنْ حَقَّ ٱلْقَوْلُ مِنِّى لَأَمْلَأَنَّ جَهَنَّمَ مِنَ ٱلْجِنَّةِ وَٱلنَّاسِ أَجْمَعِينَ ﴿١٣﴾}\\
\textamh{14.\  } & \mytextarabic{فَذُوقُوا۟ بِمَا نَسِيتُمْ لِقَآءَ يَوْمِكُمْ هَـٰذَآ إِنَّا نَسِينَـٰكُمْ ۖ وَذُوقُوا۟ عَذَابَ ٱلْخُلْدِ بِمَا كُنتُمْ تَعْمَلُونَ ﴿١٤﴾}\\
\textamh{15.\  } & \mytextarabic{إِنَّمَا يُؤْمِنُ بِـَٔايَـٰتِنَا ٱلَّذِينَ إِذَا ذُكِّرُوا۟ بِهَا خَرُّوا۟ سُجَّدًۭا وَسَبَّحُوا۟ بِحَمْدِ رَبِّهِمْ وَهُمْ لَا يَسْتَكْبِرُونَ ۩ ﴿١٥﴾}\\
\textamh{16.\  } & \mytextarabic{تَتَجَافَىٰ جُنُوبُهُمْ عَنِ ٱلْمَضَاجِعِ يَدْعُونَ رَبَّهُمْ خَوْفًۭا وَطَمَعًۭا وَمِمَّا رَزَقْنَـٰهُمْ يُنفِقُونَ ﴿١٦﴾}\\
\textamh{17.\  } & \mytextarabic{فَلَا تَعْلَمُ نَفْسٌۭ مَّآ أُخْفِىَ لَهُم مِّن قُرَّةِ أَعْيُنٍۢ جَزَآءًۢ بِمَا كَانُوا۟ يَعْمَلُونَ ﴿١٧﴾}\\
\textamh{18.\  } & \mytextarabic{أَفَمَن كَانَ مُؤْمِنًۭا كَمَن كَانَ فَاسِقًۭا ۚ لَّا يَسْتَوُۥنَ ﴿١٨﴾}\\
\textamh{19.\  } & \mytextarabic{أَمَّا ٱلَّذِينَ ءَامَنُوا۟ وَعَمِلُوا۟ ٱلصَّـٰلِحَـٰتِ فَلَهُمْ جَنَّـٰتُ ٱلْمَأْوَىٰ نُزُلًۢا بِمَا كَانُوا۟ يَعْمَلُونَ ﴿١٩﴾}\\
\textamh{20.\  } & \mytextarabic{وَأَمَّا ٱلَّذِينَ فَسَقُوا۟ فَمَأْوَىٰهُمُ ٱلنَّارُ ۖ كُلَّمَآ أَرَادُوٓا۟ أَن يَخْرُجُوا۟ مِنْهَآ أُعِيدُوا۟ فِيهَا وَقِيلَ لَهُمْ ذُوقُوا۟ عَذَابَ ٱلنَّارِ ٱلَّذِى كُنتُم بِهِۦ تُكَذِّبُونَ ﴿٢٠﴾}\\
\textamh{21.\  } & \mytextarabic{وَلَنُذِيقَنَّهُم مِّنَ ٱلْعَذَابِ ٱلْأَدْنَىٰ دُونَ ٱلْعَذَابِ ٱلْأَكْبَرِ لَعَلَّهُمْ يَرْجِعُونَ ﴿٢١﴾}\\
\textamh{22.\  } & \mytextarabic{وَمَنْ أَظْلَمُ مِمَّن ذُكِّرَ بِـَٔايَـٰتِ رَبِّهِۦ ثُمَّ أَعْرَضَ عَنْهَآ ۚ إِنَّا مِنَ ٱلْمُجْرِمِينَ مُنتَقِمُونَ ﴿٢٢﴾}\\
\textamh{23.\  } & \mytextarabic{وَلَقَدْ ءَاتَيْنَا مُوسَى ٱلْكِتَـٰبَ فَلَا تَكُن فِى مِرْيَةٍۢ مِّن لِّقَآئِهِۦ ۖ وَجَعَلْنَـٰهُ هُدًۭى لِّبَنِىٓ إِسْرَٰٓءِيلَ ﴿٢٣﴾}\\
\textamh{24.\  } & \mytextarabic{وَجَعَلْنَا مِنْهُمْ أَئِمَّةًۭ يَهْدُونَ بِأَمْرِنَا لَمَّا صَبَرُوا۟ ۖ وَكَانُوا۟ بِـَٔايَـٰتِنَا يُوقِنُونَ ﴿٢٤﴾}\\
\textamh{25.\  } & \mytextarabic{إِنَّ رَبَّكَ هُوَ يَفْصِلُ بَيْنَهُمْ يَوْمَ ٱلْقِيَـٰمَةِ فِيمَا كَانُوا۟ فِيهِ يَخْتَلِفُونَ ﴿٢٥﴾}\\
\textamh{26.\  } & \mytextarabic{أَوَلَمْ يَهْدِ لَهُمْ كَمْ أَهْلَكْنَا مِن قَبْلِهِم مِّنَ ٱلْقُرُونِ يَمْشُونَ فِى مَسَـٰكِنِهِمْ ۚ إِنَّ فِى ذَٟلِكَ لَءَايَـٰتٍ ۖ أَفَلَا يَسْمَعُونَ ﴿٢٦﴾}\\
\textamh{27.\  } & \mytextarabic{أَوَلَمْ يَرَوْا۟ أَنَّا نَسُوقُ ٱلْمَآءَ إِلَى ٱلْأَرْضِ ٱلْجُرُزِ فَنُخْرِجُ بِهِۦ زَرْعًۭا تَأْكُلُ مِنْهُ أَنْعَـٰمُهُمْ وَأَنفُسُهُمْ ۖ أَفَلَا يُبْصِرُونَ ﴿٢٧﴾}\\
\textamh{28.\  } & \mytextarabic{وَيَقُولُونَ مَتَىٰ هَـٰذَا ٱلْفَتْحُ إِن كُنتُمْ صَـٰدِقِينَ ﴿٢٨﴾}\\
\textamh{29.\  } & \mytextarabic{قُلْ يَوْمَ ٱلْفَتْحِ لَا يَنفَعُ ٱلَّذِينَ كَفَرُوٓا۟ إِيمَـٰنُهُمْ وَلَا هُمْ يُنظَرُونَ ﴿٢٩﴾}\\
\textamh{30.\  } & \mytextarabic{فَأَعْرِضْ عَنْهُمْ وَٱنتَظِرْ إِنَّهُم مُّنتَظِرُونَ ﴿٣٠﴾}\\
\end{longtable}
\clearpage