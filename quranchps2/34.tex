%% License: BSD style (Berkley) (i.e. Put the Copyright owner's name always)
%% Writer and Copyright (to): Bewketu(Bilal) Tadilo (2016-17)
\centering\section{\LR{\textamharic{ሱራቱ ሳባ -}  \RL{سوره  سبإ}}}
\begin{longtable}{%
  @{}
    p{.5\textwidth}
  @{~~~~~~~~~~~~}
    p{.5\textwidth}
    @{}
}
\nopagebreak
\textamh{ቢስሚላሂ አራህመኒ ራሂይም } &  بِسْمِ ٱللَّهِ ٱلرَّحْمَـٰنِ ٱلرَّحِيمِ\\
\textamh{1.\  } &  ٱلْحَمْدُ لِلَّهِ ٱلَّذِى لَهُۥ مَا فِى ٱلسَّمَـٰوَٟتِ وَمَا فِى ٱلْأَرْضِ وَلَهُ ٱلْحَمْدُ فِى ٱلْءَاخِرَةِ ۚ وَهُوَ ٱلْحَكِيمُ ٱلْخَبِيرُ ﴿١﴾\\
\textamh{2.\  } & يَعْلَمُ مَا يَلِجُ فِى ٱلْأَرْضِ وَمَا يَخْرُجُ مِنْهَا وَمَا يَنزِلُ مِنَ ٱلسَّمَآءِ وَمَا يَعْرُجُ فِيهَا ۚ وَهُوَ ٱلرَّحِيمُ ٱلْغَفُورُ ﴿٢﴾\\
\textamh{3.\  } & وَقَالَ ٱلَّذِينَ كَفَرُوا۟ لَا تَأْتِينَا ٱلسَّاعَةُ ۖ قُلْ بَلَىٰ وَرَبِّى لَتَأْتِيَنَّكُمْ عَـٰلِمِ ٱلْغَيْبِ ۖ لَا يَعْزُبُ عَنْهُ مِثْقَالُ ذَرَّةٍۢ فِى ٱلسَّمَـٰوَٟتِ وَلَا فِى ٱلْأَرْضِ وَلَآ أَصْغَرُ مِن ذَٟلِكَ وَلَآ أَكْبَرُ إِلَّا فِى كِتَـٰبٍۢ مُّبِينٍۢ ﴿٣﴾\\
\textamh{4.\  } & لِّيَجْزِىَ ٱلَّذِينَ ءَامَنُوا۟ وَعَمِلُوا۟ ٱلصَّـٰلِحَـٰتِ ۚ أُو۟لَـٰٓئِكَ لَهُم مَّغْفِرَةٌۭ وَرِزْقٌۭ كَرِيمٌۭ ﴿٤﴾\\
\textamh{5.\  } & وَٱلَّذِينَ سَعَوْ فِىٓ ءَايَـٰتِنَا مُعَـٰجِزِينَ أُو۟لَـٰٓئِكَ لَهُمْ عَذَابٌۭ مِّن رِّجْزٍ أَلِيمٌۭ ﴿٥﴾\\
\textamh{6.\  } & وَيَرَى ٱلَّذِينَ أُوتُوا۟ ٱلْعِلْمَ ٱلَّذِىٓ أُنزِلَ إِلَيْكَ مِن رَّبِّكَ هُوَ ٱلْحَقَّ وَيَهْدِىٓ إِلَىٰ صِرَٰطِ ٱلْعَزِيزِ ٱلْحَمِيدِ ﴿٦﴾\\
\textamh{7.\  } & وَقَالَ ٱلَّذِينَ كَفَرُوا۟ هَلْ نَدُلُّكُمْ عَلَىٰ رَجُلٍۢ يُنَبِّئُكُمْ إِذَا مُزِّقْتُمْ كُلَّ مُمَزَّقٍ إِنَّكُمْ لَفِى خَلْقٍۢ جَدِيدٍ ﴿٧﴾\\
\textamh{8.\  } & أَفْتَرَىٰ عَلَى ٱللَّهِ كَذِبًا أَم بِهِۦ جِنَّةٌۢ ۗ بَلِ ٱلَّذِينَ لَا يُؤْمِنُونَ بِٱلْءَاخِرَةِ فِى ٱلْعَذَابِ وَٱلضَّلَـٰلِ ٱلْبَعِيدِ ﴿٨﴾\\
\textamh{9.\  } & أَفَلَمْ يَرَوْا۟ إِلَىٰ مَا بَيْنَ أَيْدِيهِمْ وَمَا خَلْفَهُم مِّنَ ٱلسَّمَآءِ وَٱلْأَرْضِ ۚ إِن نَّشَأْ نَخْسِفْ بِهِمُ ٱلْأَرْضَ أَوْ نُسْقِطْ عَلَيْهِمْ كِسَفًۭا مِّنَ ٱلسَّمَآءِ ۚ إِنَّ فِى ذَٟلِكَ لَءَايَةًۭ لِّكُلِّ عَبْدٍۢ مُّنِيبٍۢ ﴿٩﴾\\
\textamh{10.\  } & ۞ وَلَقَدْ ءَاتَيْنَا دَاوُۥدَ مِنَّا فَضْلًۭا ۖ يَـٰجِبَالُ أَوِّبِى مَعَهُۥ وَٱلطَّيْرَ ۖ وَأَلَنَّا لَهُ ٱلْحَدِيدَ ﴿١٠﴾\\
\textamh{11.\  } & أَنِ ٱعْمَلْ سَـٰبِغَٰتٍۢ وَقَدِّرْ فِى ٱلسَّرْدِ ۖ وَٱعْمَلُوا۟ صَـٰلِحًا ۖ إِنِّى بِمَا تَعْمَلُونَ بَصِيرٌۭ ﴿١١﴾\\
\textamh{12.\  } & وَلِسُلَيْمَـٰنَ ٱلرِّيحَ غُدُوُّهَا شَهْرٌۭ وَرَوَاحُهَا شَهْرٌۭ ۖ وَأَسَلْنَا لَهُۥ عَيْنَ ٱلْقِطْرِ ۖ وَمِنَ ٱلْجِنِّ مَن يَعْمَلُ بَيْنَ يَدَيْهِ بِإِذْنِ رَبِّهِۦ ۖ وَمَن يَزِغْ مِنْهُمْ عَنْ أَمْرِنَا نُذِقْهُ مِنْ عَذَابِ ٱلسَّعِيرِ ﴿١٢﴾\\
\textamh{13.\  } & يَعْمَلُونَ لَهُۥ مَا يَشَآءُ مِن مَّحَـٰرِيبَ وَتَمَـٰثِيلَ وَجِفَانٍۢ كَٱلْجَوَابِ وَقُدُورٍۢ رَّاسِيَـٰتٍ ۚ ٱعْمَلُوٓا۟ ءَالَ دَاوُۥدَ شُكْرًۭا ۚ وَقَلِيلٌۭ مِّنْ عِبَادِىَ ٱلشَّكُورُ ﴿١٣﴾\\
\textamh{14.\  } & فَلَمَّا قَضَيْنَا عَلَيْهِ ٱلْمَوْتَ مَا دَلَّهُمْ عَلَىٰ مَوْتِهِۦٓ إِلَّا دَآبَّةُ ٱلْأَرْضِ تَأْكُلُ مِنسَأَتَهُۥ ۖ فَلَمَّا خَرَّ تَبَيَّنَتِ ٱلْجِنُّ أَن لَّوْ كَانُوا۟ يَعْلَمُونَ ٱلْغَيْبَ مَا لَبِثُوا۟ فِى ٱلْعَذَابِ ٱلْمُهِينِ ﴿١٤﴾\\
\textamh{15.\  } & لَقَدْ كَانَ لِسَبَإٍۢ فِى مَسْكَنِهِمْ ءَايَةٌۭ ۖ جَنَّتَانِ عَن يَمِينٍۢ وَشِمَالٍۢ ۖ كُلُوا۟ مِن رِّزْقِ رَبِّكُمْ وَٱشْكُرُوا۟ لَهُۥ ۚ بَلْدَةٌۭ طَيِّبَةٌۭ وَرَبٌّ غَفُورٌۭ ﴿١٥﴾\\
\textamh{16.\  } & فَأَعْرَضُوا۟ فَأَرْسَلْنَا عَلَيْهِمْ سَيْلَ ٱلْعَرِمِ وَبَدَّلْنَـٰهُم بِجَنَّتَيْهِمْ جَنَّتَيْنِ ذَوَاتَىْ أُكُلٍ خَمْطٍۢ وَأَثْلٍۢ وَشَىْءٍۢ مِّن سِدْرٍۢ قَلِيلٍۢ ﴿١٦﴾\\
\textamh{17.\  } & ذَٟلِكَ جَزَيْنَـٰهُم بِمَا كَفَرُوا۟ ۖ وَهَلْ نُجَٰزِىٓ إِلَّا ٱلْكَفُورَ ﴿١٧﴾\\
\textamh{18.\  } & وَجَعَلْنَا بَيْنَهُمْ وَبَيْنَ ٱلْقُرَى ٱلَّتِى بَٰرَكْنَا فِيهَا قُرًۭى ظَـٰهِرَةًۭ وَقَدَّرْنَا فِيهَا ٱلسَّيْرَ ۖ سِيرُوا۟ فِيهَا لَيَالِىَ وَأَيَّامًا ءَامِنِينَ ﴿١٨﴾\\
\textamh{19.\  } & فَقَالُوا۟ رَبَّنَا بَٰعِدْ بَيْنَ أَسْفَارِنَا وَظَلَمُوٓا۟ أَنفُسَهُمْ فَجَعَلْنَـٰهُمْ أَحَادِيثَ وَمَزَّقْنَـٰهُمْ كُلَّ مُمَزَّقٍ ۚ إِنَّ فِى ذَٟلِكَ لَءَايَـٰتٍۢ لِّكُلِّ صَبَّارٍۢ شَكُورٍۢ ﴿١٩﴾\\
\textamh{20.\  } & وَلَقَدْ صَدَّقَ عَلَيْهِمْ إِبْلِيسُ ظَنَّهُۥ فَٱتَّبَعُوهُ إِلَّا فَرِيقًۭا مِّنَ ٱلْمُؤْمِنِينَ ﴿٢٠﴾\\
\textamh{21.\  } & وَمَا كَانَ لَهُۥ عَلَيْهِم مِّن سُلْطَٰنٍ إِلَّا لِنَعْلَمَ مَن يُؤْمِنُ بِٱلْءَاخِرَةِ مِمَّنْ هُوَ مِنْهَا فِى شَكٍّۢ ۗ وَرَبُّكَ عَلَىٰ كُلِّ شَىْءٍ حَفِيظٌۭ ﴿٢١﴾\\
\textamh{22.\  } & قُلِ ٱدْعُوا۟ ٱلَّذِينَ زَعَمْتُم مِّن دُونِ ٱللَّهِ ۖ لَا يَمْلِكُونَ مِثْقَالَ ذَرَّةٍۢ فِى ٱلسَّمَـٰوَٟتِ وَلَا فِى ٱلْأَرْضِ وَمَا لَهُمْ فِيهِمَا مِن شِرْكٍۢ وَمَا لَهُۥ مِنْهُم مِّن ظَهِيرٍۢ ﴿٢٢﴾\\
\textamh{23.\  } & وَلَا تَنفَعُ ٱلشَّفَـٰعَةُ عِندَهُۥٓ إِلَّا لِمَنْ أَذِنَ لَهُۥ ۚ حَتَّىٰٓ إِذَا فُزِّعَ عَن قُلُوبِهِمْ قَالُوا۟ مَاذَا قَالَ رَبُّكُمْ ۖ قَالُوا۟ ٱلْحَقَّ ۖ وَهُوَ ٱلْعَلِىُّ ٱلْكَبِيرُ ﴿٢٣﴾\\
\textamh{24.\  } & ۞ قُلْ مَن يَرْزُقُكُم مِّنَ ٱلسَّمَـٰوَٟتِ وَٱلْأَرْضِ ۖ قُلِ ٱللَّهُ ۖ وَإِنَّآ أَوْ إِيَّاكُمْ لَعَلَىٰ هُدًى أَوْ فِى ضَلَـٰلٍۢ مُّبِينٍۢ ﴿٢٤﴾\\
\textamh{25.\  } & قُل لَّا تُسْـَٔلُونَ عَمَّآ أَجْرَمْنَا وَلَا نُسْـَٔلُ عَمَّا تَعْمَلُونَ ﴿٢٥﴾\\
\textamh{26.\  } & قُلْ يَجْمَعُ بَيْنَنَا رَبُّنَا ثُمَّ يَفْتَحُ بَيْنَنَا بِٱلْحَقِّ وَهُوَ ٱلْفَتَّاحُ ٱلْعَلِيمُ ﴿٢٦﴾\\
\textamh{27.\  } & قُلْ أَرُونِىَ ٱلَّذِينَ أَلْحَقْتُم بِهِۦ شُرَكَآءَ ۖ كَلَّا ۚ بَلْ هُوَ ٱللَّهُ ٱلْعَزِيزُ ٱلْحَكِيمُ ﴿٢٧﴾\\
\textamh{28.\  } & وَمَآ أَرْسَلْنَـٰكَ إِلَّا كَآفَّةًۭ لِّلنَّاسِ بَشِيرًۭا وَنَذِيرًۭا وَلَـٰكِنَّ أَكْثَرَ ٱلنَّاسِ لَا يَعْلَمُونَ ﴿٢٨﴾\\
\textamh{29.\  } & وَيَقُولُونَ مَتَىٰ هَـٰذَا ٱلْوَعْدُ إِن كُنتُمْ صَـٰدِقِينَ ﴿٢٩﴾\\
\textamh{30.\  } & قُل لَّكُم مِّيعَادُ يَوْمٍۢ لَّا تَسْتَـْٔخِرُونَ عَنْهُ سَاعَةًۭ وَلَا تَسْتَقْدِمُونَ ﴿٣٠﴾\\
\textamh{31.\  } & وَقَالَ ٱلَّذِينَ كَفَرُوا۟ لَن نُّؤْمِنَ بِهَـٰذَا ٱلْقُرْءَانِ وَلَا بِٱلَّذِى بَيْنَ يَدَيْهِ ۗ وَلَوْ تَرَىٰٓ إِذِ ٱلظَّـٰلِمُونَ مَوْقُوفُونَ عِندَ رَبِّهِمْ يَرْجِعُ بَعْضُهُمْ إِلَىٰ بَعْضٍ ٱلْقَوْلَ يَقُولُ ٱلَّذِينَ ٱسْتُضْعِفُوا۟ لِلَّذِينَ ٱسْتَكْبَرُوا۟ لَوْلَآ أَنتُمْ لَكُنَّا مُؤْمِنِينَ ﴿٣١﴾\\
\textamh{32.\  } & قَالَ ٱلَّذِينَ ٱسْتَكْبَرُوا۟ لِلَّذِينَ ٱسْتُضْعِفُوٓا۟ أَنَحْنُ صَدَدْنَـٰكُمْ عَنِ ٱلْهُدَىٰ بَعْدَ إِذْ جَآءَكُم ۖ بَلْ كُنتُم مُّجْرِمِينَ ﴿٣٢﴾\\
\textamh{33.\  } & وَقَالَ ٱلَّذِينَ ٱسْتُضْعِفُوا۟ لِلَّذِينَ ٱسْتَكْبَرُوا۟ بَلْ مَكْرُ ٱلَّيْلِ وَٱلنَّهَارِ إِذْ تَأْمُرُونَنَآ أَن نَّكْفُرَ بِٱللَّهِ وَنَجْعَلَ لَهُۥٓ أَندَادًۭا ۚ وَأَسَرُّوا۟ ٱلنَّدَامَةَ لَمَّا رَأَوُا۟ ٱلْعَذَابَ وَجَعَلْنَا ٱلْأَغْلَـٰلَ فِىٓ أَعْنَاقِ ٱلَّذِينَ كَفَرُوا۟ ۚ هَلْ يُجْزَوْنَ إِلَّا مَا كَانُوا۟ يَعْمَلُونَ ﴿٣٣﴾\\
\textamh{34.\  } & وَمَآ أَرْسَلْنَا فِى قَرْيَةٍۢ مِّن نَّذِيرٍ إِلَّا قَالَ مُتْرَفُوهَآ إِنَّا بِمَآ أُرْسِلْتُم بِهِۦ كَـٰفِرُونَ ﴿٣٤﴾\\
\textamh{35.\  } & وَقَالُوا۟ نَحْنُ أَكْثَرُ أَمْوَٟلًۭا وَأَوْلَـٰدًۭا وَمَا نَحْنُ بِمُعَذَّبِينَ ﴿٣٥﴾\\
\textamh{36.\  } & قُلْ إِنَّ رَبِّى يَبْسُطُ ٱلرِّزْقَ لِمَن يَشَآءُ وَيَقْدِرُ وَلَـٰكِنَّ أَكْثَرَ ٱلنَّاسِ لَا يَعْلَمُونَ ﴿٣٦﴾\\
\textamh{37.\  } & وَمَآ أَمْوَٟلُكُمْ وَلَآ أَوْلَـٰدُكُم بِٱلَّتِى تُقَرِّبُكُمْ عِندَنَا زُلْفَىٰٓ إِلَّا مَنْ ءَامَنَ وَعَمِلَ صَـٰلِحًۭا فَأُو۟لَـٰٓئِكَ لَهُمْ جَزَآءُ ٱلضِّعْفِ بِمَا عَمِلُوا۟ وَهُمْ فِى ٱلْغُرُفَـٰتِ ءَامِنُونَ ﴿٣٧﴾\\
\textamh{38.\  } & وَٱلَّذِينَ يَسْعَوْنَ فِىٓ ءَايَـٰتِنَا مُعَـٰجِزِينَ أُو۟لَـٰٓئِكَ فِى ٱلْعَذَابِ مُحْضَرُونَ ﴿٣٨﴾\\
\textamh{39.\  } & قُلْ إِنَّ رَبِّى يَبْسُطُ ٱلرِّزْقَ لِمَن يَشَآءُ مِنْ عِبَادِهِۦ وَيَقْدِرُ لَهُۥ ۚ وَمَآ أَنفَقْتُم مِّن شَىْءٍۢ فَهُوَ يُخْلِفُهُۥ ۖ وَهُوَ خَيْرُ ٱلرَّٟزِقِينَ ﴿٣٩﴾\\
\textamh{40.\  } & وَيَوْمَ يَحْشُرُهُمْ جَمِيعًۭا ثُمَّ يَقُولُ لِلْمَلَـٰٓئِكَةِ أَهَـٰٓؤُلَآءِ إِيَّاكُمْ كَانُوا۟ يَعْبُدُونَ ﴿٤٠﴾\\
\textamh{41.\  } & قَالُوا۟ سُبْحَـٰنَكَ أَنتَ وَلِيُّنَا مِن دُونِهِم ۖ بَلْ كَانُوا۟ يَعْبُدُونَ ٱلْجِنَّ ۖ أَكْثَرُهُم بِهِم مُّؤْمِنُونَ ﴿٤١﴾\\
\textamh{42.\  } & فَٱلْيَوْمَ لَا يَمْلِكُ بَعْضُكُمْ لِبَعْضٍۢ نَّفْعًۭا وَلَا ضَرًّۭا وَنَقُولُ لِلَّذِينَ ظَلَمُوا۟ ذُوقُوا۟ عَذَابَ ٱلنَّارِ ٱلَّتِى كُنتُم بِهَا تُكَذِّبُونَ ﴿٤٢﴾\\
\textamh{43.\  } & وَإِذَا تُتْلَىٰ عَلَيْهِمْ ءَايَـٰتُنَا بَيِّنَـٰتٍۢ قَالُوا۟ مَا هَـٰذَآ إِلَّا رَجُلٌۭ يُرِيدُ أَن يَصُدَّكُمْ عَمَّا كَانَ يَعْبُدُ ءَابَآؤُكُمْ وَقَالُوا۟ مَا هَـٰذَآ إِلَّآ إِفْكٌۭ مُّفْتَرًۭى ۚ وَقَالَ ٱلَّذِينَ كَفَرُوا۟ لِلْحَقِّ لَمَّا جَآءَهُمْ إِنْ هَـٰذَآ إِلَّا سِحْرٌۭ مُّبِينٌۭ ﴿٤٣﴾\\
\textamh{44.\  } & وَمَآ ءَاتَيْنَـٰهُم مِّن كُتُبٍۢ يَدْرُسُونَهَا ۖ وَمَآ أَرْسَلْنَآ إِلَيْهِمْ قَبْلَكَ مِن نَّذِيرٍۢ ﴿٤٤﴾\\
\textamh{45.\  } & وَكَذَّبَ ٱلَّذِينَ مِن قَبْلِهِمْ وَمَا بَلَغُوا۟ مِعْشَارَ مَآ ءَاتَيْنَـٰهُمْ فَكَذَّبُوا۟ رُسُلِى ۖ فَكَيْفَ كَانَ نَكِيرِ ﴿٤٥﴾\\
\textamh{46.\  } & ۞ قُلْ إِنَّمَآ أَعِظُكُم بِوَٟحِدَةٍ ۖ أَن تَقُومُوا۟ لِلَّهِ مَثْنَىٰ وَفُرَٰدَىٰ ثُمَّ تَتَفَكَّرُوا۟ ۚ مَا بِصَاحِبِكُم مِّن جِنَّةٍ ۚ إِنْ هُوَ إِلَّا نَذِيرٌۭ لَّكُم بَيْنَ يَدَىْ عَذَابٍۢ شَدِيدٍۢ ﴿٤٦﴾\\
\textamh{47.\  } & قُلْ مَا سَأَلْتُكُم مِّنْ أَجْرٍۢ فَهُوَ لَكُمْ ۖ إِنْ أَجْرِىَ إِلَّا عَلَى ٱللَّهِ ۖ وَهُوَ عَلَىٰ كُلِّ شَىْءٍۢ شَهِيدٌۭ ﴿٤٧﴾\\
\textamh{48.\  } & قُلْ إِنَّ رَبِّى يَقْذِفُ بِٱلْحَقِّ عَلَّٰمُ ٱلْغُيُوبِ ﴿٤٨﴾\\
\textamh{49.\  } & قُلْ جَآءَ ٱلْحَقُّ وَمَا يُبْدِئُ ٱلْبَٰطِلُ وَمَا يُعِيدُ ﴿٤٩﴾\\
\textamh{50.\  } & قُلْ إِن ضَلَلْتُ فَإِنَّمَآ أَضِلُّ عَلَىٰ نَفْسِى ۖ وَإِنِ ٱهْتَدَيْتُ فَبِمَا يُوحِىٓ إِلَىَّ رَبِّىٓ ۚ إِنَّهُۥ سَمِيعٌۭ قَرِيبٌۭ ﴿٥٠﴾\\
\textamh{51.\  } & وَلَوْ تَرَىٰٓ إِذْ فَزِعُوا۟ فَلَا فَوْتَ وَأُخِذُوا۟ مِن مَّكَانٍۢ قَرِيبٍۢ ﴿٥١﴾\\
\textamh{52.\  } & وَقَالُوٓا۟ ءَامَنَّا بِهِۦ وَأَنَّىٰ لَهُمُ ٱلتَّنَاوُشُ مِن مَّكَانٍۭ بَعِيدٍۢ ﴿٥٢﴾\\
\textamh{53.\  } & وَقَدْ كَفَرُوا۟ بِهِۦ مِن قَبْلُ ۖ وَيَقْذِفُونَ بِٱلْغَيْبِ مِن مَّكَانٍۭ بَعِيدٍۢ ﴿٥٣﴾\\
\textamh{54.\  } & وَحِيلَ بَيْنَهُمْ وَبَيْنَ مَا يَشْتَهُونَ كَمَا فُعِلَ بِأَشْيَاعِهِم مِّن قَبْلُ ۚ إِنَّهُمْ كَانُوا۟ فِى شَكٍّۢ مُّرِيبٍۭ ﴿٥٤﴾\\
\end{longtable}
\clearpage