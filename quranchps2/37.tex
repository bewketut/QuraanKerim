%% License: BSD style (Berkley) (i.e. Put the Copyright owner's name always)
%% Writer and Copyright (to): Bewketu(Bilal) Tadilo (2016-17)
\centering\section{\LR{\textamharic{ሱራቱ አስሳፋት -}  \RL{سوره  الصافات}}}
\begin{longtable}{%
  @{}
    p{.5\textwidth}
  @{~~~~~~~~~~~~~}
    p{.5\textwidth}
    @{}
}
\nopagebreak
\textamh{ቢስሚላሂ አራህመኒ ራሂይም } &  بِسْمِ ٱللَّهِ ٱلرَّحْمَـٰنِ ٱلرَّحِيمِ\\
\textamh{1.\  } &  وَٱلصَّـٰٓفَّٰتِ صَفًّۭا ﴿١﴾\\
\textamh{2.\  } & فَٱلزَّٰجِرَٰتِ زَجْرًۭا ﴿٢﴾\\
\textamh{3.\  } & فَٱلتَّٰلِيَـٰتِ ذِكْرًا ﴿٣﴾\\
\textamh{4.\  } & إِنَّ إِلَـٰهَكُمْ لَوَٟحِدٌۭ ﴿٤﴾\\
\textamh{5.\  } & رَّبُّ ٱلسَّمَـٰوَٟتِ وَٱلْأَرْضِ وَمَا بَيْنَهُمَا وَرَبُّ ٱلْمَشَـٰرِقِ ﴿٥﴾\\
\textamh{6.\  } & إِنَّا زَيَّنَّا ٱلسَّمَآءَ ٱلدُّنْيَا بِزِينَةٍ ٱلْكَوَاكِبِ ﴿٦﴾\\
\textamh{7.\  } & وَحِفْظًۭا مِّن كُلِّ شَيْطَٰنٍۢ مَّارِدٍۢ ﴿٧﴾\\
\textamh{8.\  } & لَّا يَسَّمَّعُونَ إِلَى ٱلْمَلَإِ ٱلْأَعْلَىٰ وَيُقْذَفُونَ مِن كُلِّ جَانِبٍۢ ﴿٨﴾\\
\textamh{9.\  } & دُحُورًۭا ۖ وَلَهُمْ عَذَابٌۭ وَاصِبٌ ﴿٩﴾\\
\textamh{10.\  } & إِلَّا مَنْ خَطِفَ ٱلْخَطْفَةَ فَأَتْبَعَهُۥ شِهَابٌۭ ثَاقِبٌۭ ﴿١٠﴾\\
\textamh{11.\  } & فَٱسْتَفْتِهِمْ أَهُمْ أَشَدُّ خَلْقًا أَم مَّنْ خَلَقْنَآ ۚ إِنَّا خَلَقْنَـٰهُم مِّن طِينٍۢ لَّازِبٍۭ ﴿١١﴾\\
\textamh{12.\  } & بَلْ عَجِبْتَ وَيَسْخَرُونَ ﴿١٢﴾\\
\textamh{13.\  } & وَإِذَا ذُكِّرُوا۟ لَا يَذْكُرُونَ ﴿١٣﴾\\
\textamh{14.\  } & وَإِذَا رَأَوْا۟ ءَايَةًۭ يَسْتَسْخِرُونَ ﴿١٤﴾\\
\textamh{15.\  } & وَقَالُوٓا۟ إِنْ هَـٰذَآ إِلَّا سِحْرٌۭ مُّبِينٌ ﴿١٥﴾\\
\textamh{16.\  } & أَءِذَا مِتْنَا وَكُنَّا تُرَابًۭا وَعِظَـٰمًا أَءِنَّا لَمَبْعُوثُونَ ﴿١٦﴾\\
\textamh{17.\  } & أَوَءَابَآؤُنَا ٱلْأَوَّلُونَ ﴿١٧﴾\\
\textamh{18.\  } & قُلْ نَعَمْ وَأَنتُمْ دَٟخِرُونَ ﴿١٨﴾\\
\textamh{19.\  } & فَإِنَّمَا هِىَ زَجْرَةٌۭ وَٟحِدَةٌۭ فَإِذَا هُمْ يَنظُرُونَ ﴿١٩﴾\\
\textamh{20.\  } & وَقَالُوا۟ يَـٰوَيْلَنَا هَـٰذَا يَوْمُ ٱلدِّينِ ﴿٢٠﴾\\
\textamh{21.\  } & هَـٰذَا يَوْمُ ٱلْفَصْلِ ٱلَّذِى كُنتُم بِهِۦ تُكَذِّبُونَ ﴿٢١﴾\\
\textamh{22.\  } & ۞ ٱحْشُرُوا۟ ٱلَّذِينَ ظَلَمُوا۟ وَأَزْوَٟجَهُمْ وَمَا كَانُوا۟ يَعْبُدُونَ ﴿٢٢﴾\\
\textamh{23.\  } & مِن دُونِ ٱللَّهِ فَٱهْدُوهُمْ إِلَىٰ صِرَٰطِ ٱلْجَحِيمِ ﴿٢٣﴾\\
\textamh{24.\  } & وَقِفُوهُمْ ۖ إِنَّهُم مَّسْـُٔولُونَ ﴿٢٤﴾\\
\textamh{25.\  } & مَا لَكُمْ لَا تَنَاصَرُونَ ﴿٢٥﴾\\
\textamh{26.\  } & بَلْ هُمُ ٱلْيَوْمَ مُسْتَسْلِمُونَ ﴿٢٦﴾\\
\textamh{27.\  } & وَأَقْبَلَ بَعْضُهُمْ عَلَىٰ بَعْضٍۢ يَتَسَآءَلُونَ ﴿٢٧﴾\\
\textamh{28.\  } & قَالُوٓا۟ إِنَّكُمْ كُنتُمْ تَأْتُونَنَا عَنِ ٱلْيَمِينِ ﴿٢٨﴾\\
\textamh{29.\  } & قَالُوا۟ بَل لَّمْ تَكُونُوا۟ مُؤْمِنِينَ ﴿٢٩﴾\\
\textamh{30.\  } & وَمَا كَانَ لَنَا عَلَيْكُم مِّن سُلْطَٰنٍۭ ۖ بَلْ كُنتُمْ قَوْمًۭا طَٰغِينَ ﴿٣٠﴾\\
\textamh{31.\  } & فَحَقَّ عَلَيْنَا قَوْلُ رَبِّنَآ ۖ إِنَّا لَذَآئِقُونَ ﴿٣١﴾\\
\textamh{32.\  } & فَأَغْوَيْنَـٰكُمْ إِنَّا كُنَّا غَٰوِينَ ﴿٣٢﴾\\
\textamh{33.\  } & فَإِنَّهُمْ يَوْمَئِذٍۢ فِى ٱلْعَذَابِ مُشْتَرِكُونَ ﴿٣٣﴾\\
\textamh{34.\  } & إِنَّا كَذَٟلِكَ نَفْعَلُ بِٱلْمُجْرِمِينَ ﴿٣٤﴾\\
\textamh{35.\  } & إِنَّهُمْ كَانُوٓا۟ إِذَا قِيلَ لَهُمْ لَآ إِلَـٰهَ إِلَّا ٱللَّهُ يَسْتَكْبِرُونَ ﴿٣٥﴾\\
\textamh{36.\  } & وَيَقُولُونَ أَئِنَّا لَتَارِكُوٓا۟ ءَالِهَتِنَا لِشَاعِرٍۢ مَّجْنُونٍۭ ﴿٣٦﴾\\
\textamh{37.\  } & بَلْ جَآءَ بِٱلْحَقِّ وَصَدَّقَ ٱلْمُرْسَلِينَ ﴿٣٧﴾\\
\textamh{38.\  } & إِنَّكُمْ لَذَآئِقُوا۟ ٱلْعَذَابِ ٱلْأَلِيمِ ﴿٣٨﴾\\
\textamh{39.\  } & وَمَا تُجْزَوْنَ إِلَّا مَا كُنتُمْ تَعْمَلُونَ ﴿٣٩﴾\\
\textamh{40.\  } & إِلَّا عِبَادَ ٱللَّهِ ٱلْمُخْلَصِينَ ﴿٤٠﴾\\
\textamh{41.\  } & أُو۟لَـٰٓئِكَ لَهُمْ رِزْقٌۭ مَّعْلُومٌۭ ﴿٤١﴾\\
\textamh{42.\  } & فَوَٟكِهُ ۖ وَهُم مُّكْرَمُونَ ﴿٤٢﴾\\
\textamh{43.\  } & فِى جَنَّـٰتِ ٱلنَّعِيمِ ﴿٤٣﴾\\
\textamh{44.\  } & عَلَىٰ سُرُرٍۢ مُّتَقَـٰبِلِينَ ﴿٤٤﴾\\
\textamh{45.\  } & يُطَافُ عَلَيْهِم بِكَأْسٍۢ مِّن مَّعِينٍۭ ﴿٤٥﴾\\
\textamh{46.\  } & بَيْضَآءَ لَذَّةٍۢ لِّلشَّـٰرِبِينَ ﴿٤٦﴾\\
\textamh{47.\  } & لَا فِيهَا غَوْلٌۭ وَلَا هُمْ عَنْهَا يُنزَفُونَ ﴿٤٧﴾\\
\textamh{48.\  } & وَعِندَهُمْ قَـٰصِرَٰتُ ٱلطَّرْفِ عِينٌۭ ﴿٤٨﴾\\
\textamh{49.\  } & كَأَنَّهُنَّ بَيْضٌۭ مَّكْنُونٌۭ ﴿٤٩﴾\\
\textamh{50.\  } & فَأَقْبَلَ بَعْضُهُمْ عَلَىٰ بَعْضٍۢ يَتَسَآءَلُونَ ﴿٥٠﴾\\
\textamh{51.\  } & قَالَ قَآئِلٌۭ مِّنْهُمْ إِنِّى كَانَ لِى قَرِينٌۭ ﴿٥١﴾\\
\textamh{52.\  } & يَقُولُ أَءِنَّكَ لَمِنَ ٱلْمُصَدِّقِينَ ﴿٥٢﴾\\
\textamh{53.\  } & أَءِذَا مِتْنَا وَكُنَّا تُرَابًۭا وَعِظَـٰمًا أَءِنَّا لَمَدِينُونَ ﴿٥٣﴾\\
\textamh{54.\  } & قَالَ هَلْ أَنتُم مُّطَّلِعُونَ ﴿٥٤﴾\\
\textamh{55.\  } & فَٱطَّلَعَ فَرَءَاهُ فِى سَوَآءِ ٱلْجَحِيمِ ﴿٥٥﴾\\
\textamh{56.\  } & قَالَ تَٱللَّهِ إِن كِدتَّ لَتُرْدِينِ ﴿٥٦﴾\\
\textamh{57.\  } & وَلَوْلَا نِعْمَةُ رَبِّى لَكُنتُ مِنَ ٱلْمُحْضَرِينَ ﴿٥٧﴾\\
\textamh{58.\  } & أَفَمَا نَحْنُ بِمَيِّتِينَ ﴿٥٨﴾\\
\textamh{59.\  } & إِلَّا مَوْتَتَنَا ٱلْأُولَىٰ وَمَا نَحْنُ بِمُعَذَّبِينَ ﴿٥٩﴾\\
\textamh{60.\  } & إِنَّ هَـٰذَا لَهُوَ ٱلْفَوْزُ ٱلْعَظِيمُ ﴿٦٠﴾\\
\textamh{61.\  } & لِمِثْلِ هَـٰذَا فَلْيَعْمَلِ ٱلْعَـٰمِلُونَ ﴿٦١﴾\\
\textamh{62.\  } & أَذَٟلِكَ خَيْرٌۭ نُّزُلًا أَمْ شَجَرَةُ ٱلزَّقُّومِ ﴿٦٢﴾\\
\textamh{63.\  } & إِنَّا جَعَلْنَـٰهَا فِتْنَةًۭ لِّلظَّـٰلِمِينَ ﴿٦٣﴾\\
\textamh{64.\  } & إِنَّهَا شَجَرَةٌۭ تَخْرُجُ فِىٓ أَصْلِ ٱلْجَحِيمِ ﴿٦٤﴾\\
\textamh{65.\  } & طَلْعُهَا كَأَنَّهُۥ رُءُوسُ ٱلشَّيَـٰطِينِ ﴿٦٥﴾\\
\textamh{66.\  } & فَإِنَّهُمْ لَءَاكِلُونَ مِنْهَا فَمَالِـُٔونَ مِنْهَا ٱلْبُطُونَ ﴿٦٦﴾\\
\textamh{67.\  } & ثُمَّ إِنَّ لَهُمْ عَلَيْهَا لَشَوْبًۭا مِّنْ حَمِيمٍۢ ﴿٦٧﴾\\
\textamh{68.\  } & ثُمَّ إِنَّ مَرْجِعَهُمْ لَإِلَى ٱلْجَحِيمِ ﴿٦٨﴾\\
\textamh{69.\  } & إِنَّهُمْ أَلْفَوْا۟ ءَابَآءَهُمْ ضَآلِّينَ ﴿٦٩﴾\\
\textamh{70.\  } & فَهُمْ عَلَىٰٓ ءَاثَـٰرِهِمْ يُهْرَعُونَ ﴿٧٠﴾\\
\textamh{71.\  } & وَلَقَدْ ضَلَّ قَبْلَهُمْ أَكْثَرُ ٱلْأَوَّلِينَ ﴿٧١﴾\\
\textamh{72.\  } & وَلَقَدْ أَرْسَلْنَا فِيهِم مُّنذِرِينَ ﴿٧٢﴾\\
\textamh{73.\  } & فَٱنظُرْ كَيْفَ كَانَ عَـٰقِبَةُ ٱلْمُنذَرِينَ ﴿٧٣﴾\\
\textamh{74.\  } & إِلَّا عِبَادَ ٱللَّهِ ٱلْمُخْلَصِينَ ﴿٧٤﴾\\
\textamh{75.\  } & وَلَقَدْ نَادَىٰنَا نُوحٌۭ فَلَنِعْمَ ٱلْمُجِيبُونَ ﴿٧٥﴾\\
\textamh{76.\  } & وَنَجَّيْنَـٰهُ وَأَهْلَهُۥ مِنَ ٱلْكَرْبِ ٱلْعَظِيمِ ﴿٧٦﴾\\
\textamh{77.\  } & وَجَعَلْنَا ذُرِّيَّتَهُۥ هُمُ ٱلْبَاقِينَ ﴿٧٧﴾\\
\textamh{78.\  } & وَتَرَكْنَا عَلَيْهِ فِى ٱلْءَاخِرِينَ ﴿٧٨﴾\\
\textamh{79.\  } & سَلَـٰمٌ عَلَىٰ نُوحٍۢ فِى ٱلْعَـٰلَمِينَ ﴿٧٩﴾\\
\textamh{80.\  } & إِنَّا كَذَٟلِكَ نَجْزِى ٱلْمُحْسِنِينَ ﴿٨٠﴾\\
\textamh{81.\  } & إِنَّهُۥ مِنْ عِبَادِنَا ٱلْمُؤْمِنِينَ ﴿٨١﴾\\
\textamh{82.\  } & ثُمَّ أَغْرَقْنَا ٱلْءَاخَرِينَ ﴿٨٢﴾\\
\textamh{83.\  } & ۞ وَإِنَّ مِن شِيعَتِهِۦ لَإِبْرَٰهِيمَ ﴿٨٣﴾\\
\textamh{84.\  } & إِذْ جَآءَ رَبَّهُۥ بِقَلْبٍۢ سَلِيمٍ ﴿٨٤﴾\\
\textamh{85.\  } & إِذْ قَالَ لِأَبِيهِ وَقَوْمِهِۦ مَاذَا تَعْبُدُونَ ﴿٨٥﴾\\
\textamh{86.\  } & أَئِفْكًا ءَالِهَةًۭ دُونَ ٱللَّهِ تُرِيدُونَ ﴿٨٦﴾\\
\textamh{87.\  } & فَمَا ظَنُّكُم بِرَبِّ ٱلْعَـٰلَمِينَ ﴿٨٧﴾\\
\textamh{88.\  } & فَنَظَرَ نَظْرَةًۭ فِى ٱلنُّجُومِ ﴿٨٨﴾\\
\textamh{89.\  } & فَقَالَ إِنِّى سَقِيمٌۭ ﴿٨٩﴾\\
\textamh{90.\  } & فَتَوَلَّوْا۟ عَنْهُ مُدْبِرِينَ ﴿٩٠﴾\\
\textamh{91.\  } & فَرَاغَ إِلَىٰٓ ءَالِهَتِهِمْ فَقَالَ أَلَا تَأْكُلُونَ ﴿٩١﴾\\
\textamh{92.\  } & مَا لَكُمْ لَا تَنطِقُونَ ﴿٩٢﴾\\
\textamh{93.\  } & فَرَاغَ عَلَيْهِمْ ضَرْبًۢا بِٱلْيَمِينِ ﴿٩٣﴾\\
\textamh{94.\  } & فَأَقْبَلُوٓا۟ إِلَيْهِ يَزِفُّونَ ﴿٩٤﴾\\
\textamh{95.\  } & قَالَ أَتَعْبُدُونَ مَا تَنْحِتُونَ ﴿٩٥﴾\\
\textamh{96.\  } & وَٱللَّهُ خَلَقَكُمْ وَمَا تَعْمَلُونَ ﴿٩٦﴾\\
\textamh{97.\  } & قَالُوا۟ ٱبْنُوا۟ لَهُۥ بُنْيَـٰنًۭا فَأَلْقُوهُ فِى ٱلْجَحِيمِ ﴿٩٧﴾\\
\textamh{98.\  } & فَأَرَادُوا۟ بِهِۦ كَيْدًۭا فَجَعَلْنَـٰهُمُ ٱلْأَسْفَلِينَ ﴿٩٨﴾\\
\textamh{99.\  } & وَقَالَ إِنِّى ذَاهِبٌ إِلَىٰ رَبِّى سَيَهْدِينِ ﴿٩٩﴾\\
\textamh{100.\  } & رَبِّ هَبْ لِى مِنَ ٱلصَّـٰلِحِينَ ﴿١٠٠﴾\\
\textamh{101.\  } & فَبَشَّرْنَـٰهُ بِغُلَـٰمٍ حَلِيمٍۢ ﴿١٠١﴾\\
\textamh{102.\  } & فَلَمَّا بَلَغَ مَعَهُ ٱلسَّعْىَ قَالَ يَـٰبُنَىَّ إِنِّىٓ أَرَىٰ فِى ٱلْمَنَامِ أَنِّىٓ أَذْبَحُكَ فَٱنظُرْ مَاذَا تَرَىٰ ۚ قَالَ يَـٰٓأَبَتِ ٱفْعَلْ مَا تُؤْمَرُ ۖ سَتَجِدُنِىٓ إِن شَآءَ ٱللَّهُ مِنَ ٱلصَّـٰبِرِينَ ﴿١٠٢﴾\\
\textamh{103.\  } & فَلَمَّآ أَسْلَمَا وَتَلَّهُۥ لِلْجَبِينِ ﴿١٠٣﴾\\
\textamh{104.\  } & وَنَـٰدَيْنَـٰهُ أَن يَـٰٓإِبْرَٰهِيمُ ﴿١٠٤﴾\\
\textamh{105.\  } & قَدْ صَدَّقْتَ ٱلرُّءْيَآ ۚ إِنَّا كَذَٟلِكَ نَجْزِى ٱلْمُحْسِنِينَ ﴿١٠٥﴾\\
\textamh{106.\  } & إِنَّ هَـٰذَا لَهُوَ ٱلْبَلَـٰٓؤُا۟ ٱلْمُبِينُ ﴿١٠٦﴾\\
\textamh{107.\  } & وَفَدَيْنَـٰهُ بِذِبْحٍ عَظِيمٍۢ ﴿١٠٧﴾\\
\textamh{108.\  } & وَتَرَكْنَا عَلَيْهِ فِى ٱلْءَاخِرِينَ ﴿١٠٨﴾\\
\textamh{109.\  } & سَلَـٰمٌ عَلَىٰٓ إِبْرَٰهِيمَ ﴿١٠٩﴾\\
\textamh{110.\  } & كَذَٟلِكَ نَجْزِى ٱلْمُحْسِنِينَ ﴿١١٠﴾\\
\textamh{111.\  } & إِنَّهُۥ مِنْ عِبَادِنَا ٱلْمُؤْمِنِينَ ﴿١١١﴾\\
\textamh{112.\  } & وَبَشَّرْنَـٰهُ بِإِسْحَـٰقَ نَبِيًّۭا مِّنَ ٱلصَّـٰلِحِينَ ﴿١١٢﴾\\
\textamh{113.\  } & وَبَٰرَكْنَا عَلَيْهِ وَعَلَىٰٓ إِسْحَـٰقَ ۚ وَمِن ذُرِّيَّتِهِمَا مُحْسِنٌۭ وَظَالِمٌۭ لِّنَفْسِهِۦ مُبِينٌۭ ﴿١١٣﴾\\
\textamh{114.\  } & وَلَقَدْ مَنَنَّا عَلَىٰ مُوسَىٰ وَهَـٰرُونَ ﴿١١٤﴾\\
\textamh{115.\  } & وَنَجَّيْنَـٰهُمَا وَقَوْمَهُمَا مِنَ ٱلْكَرْبِ ٱلْعَظِيمِ ﴿١١٥﴾\\
\textamh{116.\  } & وَنَصَرْنَـٰهُمْ فَكَانُوا۟ هُمُ ٱلْغَٰلِبِينَ ﴿١١٦﴾\\
\textamh{117.\  } & وَءَاتَيْنَـٰهُمَا ٱلْكِتَـٰبَ ٱلْمُسْتَبِينَ ﴿١١٧﴾\\
\textamh{118.\  } & وَهَدَيْنَـٰهُمَا ٱلصِّرَٰطَ ٱلْمُسْتَقِيمَ ﴿١١٨﴾\\
\textamh{119.\  } & وَتَرَكْنَا عَلَيْهِمَا فِى ٱلْءَاخِرِينَ ﴿١١٩﴾\\
\textamh{120.\  } & سَلَـٰمٌ عَلَىٰ مُوسَىٰ وَهَـٰرُونَ ﴿١٢٠﴾\\
\textamh{121.\  } & إِنَّا كَذَٟلِكَ نَجْزِى ٱلْمُحْسِنِينَ ﴿١٢١﴾\\
\textamh{122.\  } & إِنَّهُمَا مِنْ عِبَادِنَا ٱلْمُؤْمِنِينَ ﴿١٢٢﴾\\
\textamh{123.\  } & وَإِنَّ إِلْيَاسَ لَمِنَ ٱلْمُرْسَلِينَ ﴿١٢٣﴾\\
\textamh{124.\  } & إِذْ قَالَ لِقَوْمِهِۦٓ أَلَا تَتَّقُونَ ﴿١٢٤﴾\\
\textamh{125.\  } & أَتَدْعُونَ بَعْلًۭا وَتَذَرُونَ أَحْسَنَ ٱلْخَـٰلِقِينَ ﴿١٢٥﴾\\
\textamh{126.\  } & ٱللَّهَ رَبَّكُمْ وَرَبَّ ءَابَآئِكُمُ ٱلْأَوَّلِينَ ﴿١٢٦﴾\\
\textamh{127.\  } & فَكَذَّبُوهُ فَإِنَّهُمْ لَمُحْضَرُونَ ﴿١٢٧﴾\\
\textamh{128.\  } & إِلَّا عِبَادَ ٱللَّهِ ٱلْمُخْلَصِينَ ﴿١٢٨﴾\\
\textamh{129.\  } & وَتَرَكْنَا عَلَيْهِ فِى ٱلْءَاخِرِينَ ﴿١٢٩﴾\\
\textamh{130.\  } & سَلَـٰمٌ عَلَىٰٓ إِلْ يَاسِينَ ﴿١٣٠﴾\\
\textamh{131.\  } & إِنَّا كَذَٟلِكَ نَجْزِى ٱلْمُحْسِنِينَ ﴿١٣١﴾\\
\textamh{132.\  } & إِنَّهُۥ مِنْ عِبَادِنَا ٱلْمُؤْمِنِينَ ﴿١٣٢﴾\\
\textamh{133.\  } & وَإِنَّ لُوطًۭا لَّمِنَ ٱلْمُرْسَلِينَ ﴿١٣٣﴾\\
\textamh{134.\  } & إِذْ نَجَّيْنَـٰهُ وَأَهْلَهُۥٓ أَجْمَعِينَ ﴿١٣٤﴾\\
\textamh{135.\  } & إِلَّا عَجُوزًۭا فِى ٱلْغَٰبِرِينَ ﴿١٣٥﴾\\
\textamh{136.\  } & ثُمَّ دَمَّرْنَا ٱلْءَاخَرِينَ ﴿١٣٦﴾\\
\textamh{137.\  } & وَإِنَّكُمْ لَتَمُرُّونَ عَلَيْهِم مُّصْبِحِينَ ﴿١٣٧﴾\\
\textamh{138.\  } & وَبِٱلَّيْلِ ۗ أَفَلَا تَعْقِلُونَ ﴿١٣٨﴾\\
\textamh{139.\  } & وَإِنَّ يُونُسَ لَمِنَ ٱلْمُرْسَلِينَ ﴿١٣٩﴾\\
\textamh{140.\  } & إِذْ أَبَقَ إِلَى ٱلْفُلْكِ ٱلْمَشْحُونِ ﴿١٤٠﴾\\
\textamh{141.\  } & فَسَاهَمَ فَكَانَ مِنَ ٱلْمُدْحَضِينَ ﴿١٤١﴾\\
\textamh{142.\  } & فَٱلْتَقَمَهُ ٱلْحُوتُ وَهُوَ مُلِيمٌۭ ﴿١٤٢﴾\\
\textamh{143.\  } & فَلَوْلَآ أَنَّهُۥ كَانَ مِنَ ٱلْمُسَبِّحِينَ ﴿١٤٣﴾\\
\textamh{144.\  } & لَلَبِثَ فِى بَطْنِهِۦٓ إِلَىٰ يَوْمِ يُبْعَثُونَ ﴿١٤٤﴾\\
\textamh{145.\  } & ۞ فَنَبَذْنَـٰهُ بِٱلْعَرَآءِ وَهُوَ سَقِيمٌۭ ﴿١٤٥﴾\\
\textamh{146.\  } & وَأَنۢبَتْنَا عَلَيْهِ شَجَرَةًۭ مِّن يَقْطِينٍۢ ﴿١٤٦﴾\\
\textamh{147.\  } & وَأَرْسَلْنَـٰهُ إِلَىٰ مِا۟ئَةِ أَلْفٍ أَوْ يَزِيدُونَ ﴿١٤٧﴾\\
\textamh{148.\  } & فَـَٔامَنُوا۟ فَمَتَّعْنَـٰهُمْ إِلَىٰ حِينٍۢ ﴿١٤٨﴾\\
\textamh{149.\  } & فَٱسْتَفْتِهِمْ أَلِرَبِّكَ ٱلْبَنَاتُ وَلَهُمُ ٱلْبَنُونَ ﴿١٤٩﴾\\
\textamh{150.\  } & أَمْ خَلَقْنَا ٱلْمَلَـٰٓئِكَةَ إِنَـٰثًۭا وَهُمْ شَـٰهِدُونَ ﴿١٥٠﴾\\
\textamh{151.\  } & أَلَآ إِنَّهُم مِّنْ إِفْكِهِمْ لَيَقُولُونَ ﴿١٥١﴾\\
\textamh{152.\  } & وَلَدَ ٱللَّهُ وَإِنَّهُمْ لَكَـٰذِبُونَ ﴿١٥٢﴾\\
\textamh{153.\  } & أَصْطَفَى ٱلْبَنَاتِ عَلَى ٱلْبَنِينَ ﴿١٥٣﴾\\
\textamh{154.\  } & مَا لَكُمْ كَيْفَ تَحْكُمُونَ ﴿١٥٤﴾\\
\textamh{155.\  } & أَفَلَا تَذَكَّرُونَ ﴿١٥٥﴾\\
\textamh{156.\  } & أَمْ لَكُمْ سُلْطَٰنٌۭ مُّبِينٌۭ ﴿١٥٦﴾\\
\textamh{157.\  } & فَأْتُوا۟ بِكِتَـٰبِكُمْ إِن كُنتُمْ صَـٰدِقِينَ ﴿١٥٧﴾\\
\textamh{158.\  } & وَجَعَلُوا۟ بَيْنَهُۥ وَبَيْنَ ٱلْجِنَّةِ نَسَبًۭا ۚ وَلَقَدْ عَلِمَتِ ٱلْجِنَّةُ إِنَّهُمْ لَمُحْضَرُونَ ﴿١٥٨﴾\\
\textamh{159.\  } & سُبْحَـٰنَ ٱللَّهِ عَمَّا يَصِفُونَ ﴿١٥٩﴾\\
\textamh{160.\  } & إِلَّا عِبَادَ ٱللَّهِ ٱلْمُخْلَصِينَ ﴿١٦٠﴾\\
\textamh{161.\  } & فَإِنَّكُمْ وَمَا تَعْبُدُونَ ﴿١٦١﴾\\
\textamh{162.\  } & مَآ أَنتُمْ عَلَيْهِ بِفَـٰتِنِينَ ﴿١٦٢﴾\\
\textamh{163.\  } & إِلَّا مَنْ هُوَ صَالِ ٱلْجَحِيمِ ﴿١٦٣﴾\\
\textamh{164.\  } & وَمَا مِنَّآ إِلَّا لَهُۥ مَقَامٌۭ مَّعْلُومٌۭ ﴿١٦٤﴾\\
\textamh{165.\  } & وَإِنَّا لَنَحْنُ ٱلصَّآفُّونَ ﴿١٦٥﴾\\
\textamh{166.\  } & وَإِنَّا لَنَحْنُ ٱلْمُسَبِّحُونَ ﴿١٦٦﴾\\
\textamh{167.\  } & وَإِن كَانُوا۟ لَيَقُولُونَ ﴿١٦٧﴾\\
\textamh{168.\  } & لَوْ أَنَّ عِندَنَا ذِكْرًۭا مِّنَ ٱلْأَوَّلِينَ ﴿١٦٨﴾\\
\textamh{169.\  } & لَكُنَّا عِبَادَ ٱللَّهِ ٱلْمُخْلَصِينَ ﴿١٦٩﴾\\
\textamh{170.\  } & فَكَفَرُوا۟ بِهِۦ ۖ فَسَوْفَ يَعْلَمُونَ ﴿١٧٠﴾\\
\textamh{171.\  } & وَلَقَدْ سَبَقَتْ كَلِمَتُنَا لِعِبَادِنَا ٱلْمُرْسَلِينَ ﴿١٧١﴾\\
\textamh{172.\  } & إِنَّهُمْ لَهُمُ ٱلْمَنصُورُونَ ﴿١٧٢﴾\\
\textamh{173.\  } & وَإِنَّ جُندَنَا لَهُمُ ٱلْغَٰلِبُونَ ﴿١٧٣﴾\\
\textamh{174.\  } & فَتَوَلَّ عَنْهُمْ حَتَّىٰ حِينٍۢ ﴿١٧٤﴾\\
\textamh{175.\  } & وَأَبْصِرْهُمْ فَسَوْفَ يُبْصِرُونَ ﴿١٧٥﴾\\
\textamh{176.\  } & أَفَبِعَذَابِنَا يَسْتَعْجِلُونَ ﴿١٧٦﴾\\
\textamh{177.\  } & فَإِذَا نَزَلَ بِسَاحَتِهِمْ فَسَآءَ صَبَاحُ ٱلْمُنذَرِينَ ﴿١٧٧﴾\\
\textamh{178.\  } & وَتَوَلَّ عَنْهُمْ حَتَّىٰ حِينٍۢ ﴿١٧٨﴾\\
\textamh{179.\  } & وَأَبْصِرْ فَسَوْفَ يُبْصِرُونَ ﴿١٧٩﴾\\
\textamh{180.\  } & سُبْحَـٰنَ رَبِّكَ رَبِّ ٱلْعِزَّةِ عَمَّا يَصِفُونَ ﴿١٨٠﴾\\
\textamh{181.\  } & وَسَلَـٰمٌ عَلَى ٱلْمُرْسَلِينَ ﴿١٨١﴾\\
\textamh{182.\  } & وَٱلْحَمْدُ لِلَّهِ رَبِّ ٱلْعَـٰلَمِينَ ﴿١٨٢﴾\\
\end{longtable}
\clearpage