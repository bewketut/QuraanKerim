%% License: BSD style (Berkley) (i.e. Put the Copyright owner's name always)
%% Writer and Copyright (to): Bewketu(Bilal) Tadilo (2016-17)
\begin{center}\section{\LR{\textamhsec{ሱራቱ አልዙመር -}  \textarabic{سوره  الزمر}}}\end{center}
\begin{longtable}{%
  @{}
    p{.5\textwidth}
  @{~~~}
    p{.5\textwidth}
    @{}
}
\textamh{ቢስሚላሂ አራህመኒ ራሂይም } &  \mytextarabic{بِسْمِ ٱللَّهِ ٱلرَّحْمَـٰنِ ٱلرَّحِيمِ}\\
\textamh{1.\  } & \mytextarabic{ تَنزِيلُ ٱلْكِتَـٰبِ مِنَ ٱللَّهِ ٱلْعَزِيزِ ٱلْحَكِيمِ ﴿١﴾}\\
\textamh{2.\  } & \mytextarabic{إِنَّآ أَنزَلْنَآ إِلَيْكَ ٱلْكِتَـٰبَ بِٱلْحَقِّ فَٱعْبُدِ ٱللَّهَ مُخْلِصًۭا لَّهُ ٱلدِّينَ ﴿٢﴾}\\
\textamh{3.\  } & \mytextarabic{أَلَا لِلَّهِ ٱلدِّينُ ٱلْخَالِصُ ۚ وَٱلَّذِينَ ٱتَّخَذُوا۟ مِن دُونِهِۦٓ أَوْلِيَآءَ مَا نَعْبُدُهُمْ إِلَّا لِيُقَرِّبُونَآ إِلَى ٱللَّهِ زُلْفَىٰٓ إِنَّ ٱللَّهَ يَحْكُمُ بَيْنَهُمْ فِى مَا هُمْ فِيهِ يَخْتَلِفُونَ ۗ إِنَّ ٱللَّهَ لَا يَهْدِى مَنْ هُوَ كَـٰذِبٌۭ كَفَّارٌۭ ﴿٣﴾}\\
\textamh{4.\  } & \mytextarabic{لَّوْ أَرَادَ ٱللَّهُ أَن يَتَّخِذَ وَلَدًۭا لَّٱصْطَفَىٰ مِمَّا يَخْلُقُ مَا يَشَآءُ ۚ سُبْحَـٰنَهُۥ ۖ هُوَ ٱللَّهُ ٱلْوَٟحِدُ ٱلْقَهَّارُ ﴿٤﴾}\\
\textamh{5.\  } & \mytextarabic{خَلَقَ ٱلسَّمَـٰوَٟتِ وَٱلْأَرْضَ بِٱلْحَقِّ ۖ يُكَوِّرُ ٱلَّيْلَ عَلَى ٱلنَّهَارِ وَيُكَوِّرُ ٱلنَّهَارَ عَلَى ٱلَّيْلِ ۖ وَسَخَّرَ ٱلشَّمْسَ وَٱلْقَمَرَ ۖ كُلٌّۭ يَجْرِى لِأَجَلٍۢ مُّسَمًّى ۗ أَلَا هُوَ ٱلْعَزِيزُ ٱلْغَفَّٰرُ ﴿٥﴾}\\
\textamh{6.\  } & \mytextarabic{خَلَقَكُم مِّن نَّفْسٍۢ وَٟحِدَةٍۢ ثُمَّ جَعَلَ مِنْهَا زَوْجَهَا وَأَنزَلَ لَكُم مِّنَ ٱلْأَنْعَـٰمِ ثَمَـٰنِيَةَ أَزْوَٟجٍۢ ۚ يَخْلُقُكُمْ فِى بُطُونِ أُمَّهَـٰتِكُمْ خَلْقًۭا مِّنۢ بَعْدِ خَلْقٍۢ فِى ظُلُمَـٰتٍۢ ثَلَـٰثٍۢ ۚ ذَٟلِكُمُ ٱللَّهُ رَبُّكُمْ لَهُ ٱلْمُلْكُ ۖ لَآ إِلَـٰهَ إِلَّا هُوَ ۖ فَأَنَّىٰ تُصْرَفُونَ ﴿٦﴾}\\
\textamh{7.\  } & \mytextarabic{إِن تَكْفُرُوا۟ فَإِنَّ ٱللَّهَ غَنِىٌّ عَنكُمْ ۖ وَلَا يَرْضَىٰ لِعِبَادِهِ ٱلْكُفْرَ ۖ وَإِن تَشْكُرُوا۟ يَرْضَهُ لَكُمْ ۗ وَلَا تَزِرُ وَازِرَةٌۭ وِزْرَ أُخْرَىٰ ۗ ثُمَّ إِلَىٰ رَبِّكُم مَّرْجِعُكُمْ فَيُنَبِّئُكُم بِمَا كُنتُمْ تَعْمَلُونَ ۚ إِنَّهُۥ عَلِيمٌۢ بِذَاتِ ٱلصُّدُورِ ﴿٧﴾}\\
\textamh{8.\  } & \mytextarabic{۞ وَإِذَا مَسَّ ٱلْإِنسَـٰنَ ضُرٌّۭ دَعَا رَبَّهُۥ مُنِيبًا إِلَيْهِ ثُمَّ إِذَا خَوَّلَهُۥ نِعْمَةًۭ مِّنْهُ نَسِىَ مَا كَانَ يَدْعُوٓا۟ إِلَيْهِ مِن قَبْلُ وَجَعَلَ لِلَّهِ أَندَادًۭا لِّيُضِلَّ عَن سَبِيلِهِۦ ۚ قُلْ تَمَتَّعْ بِكُفْرِكَ قَلِيلًا ۖ إِنَّكَ مِنْ أَصْحَـٰبِ ٱلنَّارِ ﴿٨﴾}\\
\textamh{9.\  } & \mytextarabic{أَمَّنْ هُوَ قَـٰنِتٌ ءَانَآءَ ٱلَّيْلِ سَاجِدًۭا وَقَآئِمًۭا يَحْذَرُ ٱلْءَاخِرَةَ وَيَرْجُوا۟ رَحْمَةَ رَبِّهِۦ ۗ قُلْ هَلْ يَسْتَوِى ٱلَّذِينَ يَعْلَمُونَ وَٱلَّذِينَ لَا يَعْلَمُونَ ۗ إِنَّمَا يَتَذَكَّرُ أُو۟لُوا۟ ٱلْأَلْبَٰبِ ﴿٩﴾}\\
\textamh{10.\  } & \mytextarabic{قُلْ يَـٰعِبَادِ ٱلَّذِينَ ءَامَنُوا۟ ٱتَّقُوا۟ رَبَّكُمْ ۚ لِلَّذِينَ أَحْسَنُوا۟ فِى هَـٰذِهِ ٱلدُّنْيَا حَسَنَةٌۭ ۗ وَأَرْضُ ٱللَّهِ وَٟسِعَةٌ ۗ إِنَّمَا يُوَفَّى ٱلصَّـٰبِرُونَ أَجْرَهُم بِغَيْرِ حِسَابٍۢ ﴿١٠﴾}\\
\textamh{11.\  } & \mytextarabic{قُلْ إِنِّىٓ أُمِرْتُ أَنْ أَعْبُدَ ٱللَّهَ مُخْلِصًۭا لَّهُ ٱلدِّينَ ﴿١١﴾}\\
\textamh{12.\  } & \mytextarabic{وَأُمِرْتُ لِأَنْ أَكُونَ أَوَّلَ ٱلْمُسْلِمِينَ ﴿١٢﴾}\\
\textamh{13.\  } & \mytextarabic{قُلْ إِنِّىٓ أَخَافُ إِنْ عَصَيْتُ رَبِّى عَذَابَ يَوْمٍ عَظِيمٍۢ ﴿١٣﴾}\\
\textamh{14.\  } & \mytextarabic{قُلِ ٱللَّهَ أَعْبُدُ مُخْلِصًۭا لَّهُۥ دِينِى ﴿١٤﴾}\\
\textamh{15.\  } & \mytextarabic{فَٱعْبُدُوا۟ مَا شِئْتُم مِّن دُونِهِۦ ۗ قُلْ إِنَّ ٱلْخَـٰسِرِينَ ٱلَّذِينَ خَسِرُوٓا۟ أَنفُسَهُمْ وَأَهْلِيهِمْ يَوْمَ ٱلْقِيَـٰمَةِ ۗ أَلَا ذَٟلِكَ هُوَ ٱلْخُسْرَانُ ٱلْمُبِينُ ﴿١٥﴾}\\
\textamh{16.\  } & \mytextarabic{لَهُم مِّن فَوْقِهِمْ ظُلَلٌۭ مِّنَ ٱلنَّارِ وَمِن تَحْتِهِمْ ظُلَلٌۭ ۚ ذَٟلِكَ يُخَوِّفُ ٱللَّهُ بِهِۦ عِبَادَهُۥ ۚ يَـٰعِبَادِ فَٱتَّقُونِ ﴿١٦﴾}\\
\textamh{17.\  } & \mytextarabic{وَٱلَّذِينَ ٱجْتَنَبُوا۟ ٱلطَّٰغُوتَ أَن يَعْبُدُوهَا وَأَنَابُوٓا۟ إِلَى ٱللَّهِ لَهُمُ ٱلْبُشْرَىٰ ۚ فَبَشِّرْ عِبَادِ ﴿١٧﴾}\\
\textamh{18.\  } & \mytextarabic{ٱلَّذِينَ يَسْتَمِعُونَ ٱلْقَوْلَ فَيَتَّبِعُونَ أَحْسَنَهُۥٓ ۚ أُو۟لَـٰٓئِكَ ٱلَّذِينَ هَدَىٰهُمُ ٱللَّهُ ۖ وَأُو۟لَـٰٓئِكَ هُمْ أُو۟لُوا۟ ٱلْأَلْبَٰبِ ﴿١٨﴾}\\
\textamh{19.\  } & \mytextarabic{أَفَمَنْ حَقَّ عَلَيْهِ كَلِمَةُ ٱلْعَذَابِ أَفَأَنتَ تُنقِذُ مَن فِى ٱلنَّارِ ﴿١٩﴾}\\
\textamh{20.\  } & \mytextarabic{لَـٰكِنِ ٱلَّذِينَ ٱتَّقَوْا۟ رَبَّهُمْ لَهُمْ غُرَفٌۭ مِّن فَوْقِهَا غُرَفٌۭ مَّبْنِيَّةٌۭ تَجْرِى مِن تَحْتِهَا ٱلْأَنْهَـٰرُ ۖ وَعْدَ ٱللَّهِ ۖ لَا يُخْلِفُ ٱللَّهُ ٱلْمِيعَادَ ﴿٢٠﴾}\\
\textamh{21.\  } & \mytextarabic{أَلَمْ تَرَ أَنَّ ٱللَّهَ أَنزَلَ مِنَ ٱلسَّمَآءِ مَآءًۭ فَسَلَكَهُۥ يَنَـٰبِيعَ فِى ٱلْأَرْضِ ثُمَّ يُخْرِجُ بِهِۦ زَرْعًۭا مُّخْتَلِفًا أَلْوَٟنُهُۥ ثُمَّ يَهِيجُ فَتَرَىٰهُ مُصْفَرًّۭا ثُمَّ يَجْعَلُهُۥ حُطَٰمًا ۚ إِنَّ فِى ذَٟلِكَ لَذِكْرَىٰ لِأُو۟لِى ٱلْأَلْبَٰبِ ﴿٢١﴾}\\
\textamh{22.\  } & \mytextarabic{أَفَمَن شَرَحَ ٱللَّهُ صَدْرَهُۥ لِلْإِسْلَـٰمِ فَهُوَ عَلَىٰ نُورٍۢ مِّن رَّبِّهِۦ ۚ فَوَيْلٌۭ لِّلْقَـٰسِيَةِ قُلُوبُهُم مِّن ذِكْرِ ٱللَّهِ ۚ أُو۟لَـٰٓئِكَ فِى ضَلَـٰلٍۢ مُّبِينٍ ﴿٢٢﴾}\\
\textamh{23.\  } & \mytextarabic{ٱللَّهُ نَزَّلَ أَحْسَنَ ٱلْحَدِيثِ كِتَـٰبًۭا مُّتَشَـٰبِهًۭا مَّثَانِىَ تَقْشَعِرُّ مِنْهُ جُلُودُ ٱلَّذِينَ يَخْشَوْنَ رَبَّهُمْ ثُمَّ تَلِينُ جُلُودُهُمْ وَقُلُوبُهُمْ إِلَىٰ ذِكْرِ ٱللَّهِ ۚ ذَٟلِكَ هُدَى ٱللَّهِ يَهْدِى بِهِۦ مَن يَشَآءُ ۚ وَمَن يُضْلِلِ ٱللَّهُ فَمَا لَهُۥ مِنْ هَادٍ ﴿٢٣﴾}\\
\textamh{24.\  } & \mytextarabic{أَفَمَن يَتَّقِى بِوَجْهِهِۦ سُوٓءَ ٱلْعَذَابِ يَوْمَ ٱلْقِيَـٰمَةِ ۚ وَقِيلَ لِلظَّـٰلِمِينَ ذُوقُوا۟ مَا كُنتُمْ تَكْسِبُونَ ﴿٢٤﴾}\\
\textamh{25.\  } & \mytextarabic{كَذَّبَ ٱلَّذِينَ مِن قَبْلِهِمْ فَأَتَىٰهُمُ ٱلْعَذَابُ مِنْ حَيْثُ لَا يَشْعُرُونَ ﴿٢٥﴾}\\
\textamh{26.\  } & \mytextarabic{فَأَذَاقَهُمُ ٱللَّهُ ٱلْخِزْىَ فِى ٱلْحَيَوٰةِ ٱلدُّنْيَا ۖ وَلَعَذَابُ ٱلْءَاخِرَةِ أَكْبَرُ ۚ لَوْ كَانُوا۟ يَعْلَمُونَ ﴿٢٦﴾}\\
\textamh{27.\  } & \mytextarabic{وَلَقَدْ ضَرَبْنَا لِلنَّاسِ فِى هَـٰذَا ٱلْقُرْءَانِ مِن كُلِّ مَثَلٍۢ لَّعَلَّهُمْ يَتَذَكَّرُونَ ﴿٢٧﴾}\\
\textamh{28.\  } & \mytextarabic{قُرْءَانًا عَرَبِيًّا غَيْرَ ذِى عِوَجٍۢ لَّعَلَّهُمْ يَتَّقُونَ ﴿٢٨﴾}\\
\textamh{29.\  } & \mytextarabic{ضَرَبَ ٱللَّهُ مَثَلًۭا رَّجُلًۭا فِيهِ شُرَكَآءُ مُتَشَـٰكِسُونَ وَرَجُلًۭا سَلَمًۭا لِّرَجُلٍ هَلْ يَسْتَوِيَانِ مَثَلًا ۚ ٱلْحَمْدُ لِلَّهِ ۚ بَلْ أَكْثَرُهُمْ لَا يَعْلَمُونَ ﴿٢٩﴾}\\
\textamh{30.\  } & \mytextarabic{إِنَّكَ مَيِّتٌۭ وَإِنَّهُم مَّيِّتُونَ ﴿٣٠﴾}\\
\textamh{31.\  } & \mytextarabic{ثُمَّ إِنَّكُمْ يَوْمَ ٱلْقِيَـٰمَةِ عِندَ رَبِّكُمْ تَخْتَصِمُونَ ﴿٣١﴾}\\
\textamh{32.\  } & \mytextarabic{۞ فَمَنْ أَظْلَمُ مِمَّن كَذَبَ عَلَى ٱللَّهِ وَكَذَّبَ بِٱلصِّدْقِ إِذْ جَآءَهُۥٓ ۚ أَلَيْسَ فِى جَهَنَّمَ مَثْوًۭى لِّلْكَـٰفِرِينَ ﴿٣٢﴾}\\
\textamh{33.\  } & \mytextarabic{وَٱلَّذِى جَآءَ بِٱلصِّدْقِ وَصَدَّقَ بِهِۦٓ ۙ أُو۟لَـٰٓئِكَ هُمُ ٱلْمُتَّقُونَ ﴿٣٣﴾}\\
\textamh{34.\  } & \mytextarabic{لَهُم مَّا يَشَآءُونَ عِندَ رَبِّهِمْ ۚ ذَٟلِكَ جَزَآءُ ٱلْمُحْسِنِينَ ﴿٣٤﴾}\\
\textamh{35.\  } & \mytextarabic{لِيُكَفِّرَ ٱللَّهُ عَنْهُمْ أَسْوَأَ ٱلَّذِى عَمِلُوا۟ وَيَجْزِيَهُمْ أَجْرَهُم بِأَحْسَنِ ٱلَّذِى كَانُوا۟ يَعْمَلُونَ ﴿٣٥﴾}\\
\textamh{36.\  } & \mytextarabic{أَلَيْسَ ٱللَّهُ بِكَافٍ عَبْدَهُۥ ۖ وَيُخَوِّفُونَكَ بِٱلَّذِينَ مِن دُونِهِۦ ۚ وَمَن يُضْلِلِ ٱللَّهُ فَمَا لَهُۥ مِنْ هَادٍۢ ﴿٣٦﴾}\\
\textamh{37.\  } & \mytextarabic{وَمَن يَهْدِ ٱللَّهُ فَمَا لَهُۥ مِن مُّضِلٍّ ۗ أَلَيْسَ ٱللَّهُ بِعَزِيزٍۢ ذِى ٱنتِقَامٍۢ ﴿٣٧﴾}\\
\textamh{38.\  } & \mytextarabic{وَلَئِن سَأَلْتَهُم مَّنْ خَلَقَ ٱلسَّمَـٰوَٟتِ وَٱلْأَرْضَ لَيَقُولُنَّ ٱللَّهُ ۚ قُلْ أَفَرَءَيْتُم مَّا تَدْعُونَ مِن دُونِ ٱللَّهِ إِنْ أَرَادَنِىَ ٱللَّهُ بِضُرٍّ هَلْ هُنَّ كَـٰشِفَـٰتُ ضُرِّهِۦٓ أَوْ أَرَادَنِى بِرَحْمَةٍ هَلْ هُنَّ مُمْسِكَـٰتُ رَحْمَتِهِۦ ۚ قُلْ حَسْبِىَ ٱللَّهُ ۖ عَلَيْهِ يَتَوَكَّلُ ٱلْمُتَوَكِّلُونَ ﴿٣٨﴾}\\
\textamh{39.\  } & \mytextarabic{قُلْ يَـٰقَوْمِ ٱعْمَلُوا۟ عَلَىٰ مَكَانَتِكُمْ إِنِّى عَـٰمِلٌۭ ۖ فَسَوْفَ تَعْلَمُونَ ﴿٣٩﴾}\\
\textamh{40.\  } & \mytextarabic{مَن يَأْتِيهِ عَذَابٌۭ يُخْزِيهِ وَيَحِلُّ عَلَيْهِ عَذَابٌۭ مُّقِيمٌ ﴿٤٠﴾}\\
\textamh{41.\  } & \mytextarabic{إِنَّآ أَنزَلْنَا عَلَيْكَ ٱلْكِتَـٰبَ لِلنَّاسِ بِٱلْحَقِّ ۖ فَمَنِ ٱهْتَدَىٰ فَلِنَفْسِهِۦ ۖ وَمَن ضَلَّ فَإِنَّمَا يَضِلُّ عَلَيْهَا ۖ وَمَآ أَنتَ عَلَيْهِم بِوَكِيلٍ ﴿٤١﴾}\\
\textamh{42.\  } & \mytextarabic{ٱللَّهُ يَتَوَفَّى ٱلْأَنفُسَ حِينَ مَوْتِهَا وَٱلَّتِى لَمْ تَمُتْ فِى مَنَامِهَا ۖ فَيُمْسِكُ ٱلَّتِى قَضَىٰ عَلَيْهَا ٱلْمَوْتَ وَيُرْسِلُ ٱلْأُخْرَىٰٓ إِلَىٰٓ أَجَلٍۢ مُّسَمًّى ۚ إِنَّ فِى ذَٟلِكَ لَءَايَـٰتٍۢ لِّقَوْمٍۢ يَتَفَكَّرُونَ ﴿٤٢﴾}\\
\textamh{43.\  } & \mytextarabic{أَمِ ٱتَّخَذُوا۟ مِن دُونِ ٱللَّهِ شُفَعَآءَ ۚ قُلْ أَوَلَوْ كَانُوا۟ لَا يَمْلِكُونَ شَيْـًۭٔا وَلَا يَعْقِلُونَ ﴿٤٣﴾}\\
\textamh{44.\  } & \mytextarabic{قُل لِّلَّهِ ٱلشَّفَـٰعَةُ جَمِيعًۭا ۖ لَّهُۥ مُلْكُ ٱلسَّمَـٰوَٟتِ وَٱلْأَرْضِ ۖ ثُمَّ إِلَيْهِ تُرْجَعُونَ ﴿٤٤﴾}\\
\textamh{45.\  } & \mytextarabic{وَإِذَا ذُكِرَ ٱللَّهُ وَحْدَهُ ٱشْمَأَزَّتْ قُلُوبُ ٱلَّذِينَ لَا يُؤْمِنُونَ بِٱلْءَاخِرَةِ ۖ وَإِذَا ذُكِرَ ٱلَّذِينَ مِن دُونِهِۦٓ إِذَا هُمْ يَسْتَبْشِرُونَ ﴿٤٥﴾}\\
\textamh{46.\  } & \mytextarabic{قُلِ ٱللَّهُمَّ فَاطِرَ ٱلسَّمَـٰوَٟتِ وَٱلْأَرْضِ عَـٰلِمَ ٱلْغَيْبِ وَٱلشَّهَـٰدَةِ أَنتَ تَحْكُمُ بَيْنَ عِبَادِكَ فِى مَا كَانُوا۟ فِيهِ يَخْتَلِفُونَ ﴿٤٦﴾}\\
\textamh{47.\  } & \mytextarabic{وَلَوْ أَنَّ لِلَّذِينَ ظَلَمُوا۟ مَا فِى ٱلْأَرْضِ جَمِيعًۭا وَمِثْلَهُۥ مَعَهُۥ لَٱفْتَدَوْا۟ بِهِۦ مِن سُوٓءِ ٱلْعَذَابِ يَوْمَ ٱلْقِيَـٰمَةِ ۚ وَبَدَا لَهُم مِّنَ ٱللَّهِ مَا لَمْ يَكُونُوا۟ يَحْتَسِبُونَ ﴿٤٧﴾}\\
\textamh{48.\  } & \mytextarabic{وَبَدَا لَهُمْ سَيِّـَٔاتُ مَا كَسَبُوا۟ وَحَاقَ بِهِم مَّا كَانُوا۟ بِهِۦ يَسْتَهْزِءُونَ ﴿٤٨﴾}\\
\textamh{49.\  } & \mytextarabic{فَإِذَا مَسَّ ٱلْإِنسَـٰنَ ضُرٌّۭ دَعَانَا ثُمَّ إِذَا خَوَّلْنَـٰهُ نِعْمَةًۭ مِّنَّا قَالَ إِنَّمَآ أُوتِيتُهُۥ عَلَىٰ عِلْمٍۭ ۚ بَلْ هِىَ فِتْنَةٌۭ وَلَـٰكِنَّ أَكْثَرَهُمْ لَا يَعْلَمُونَ ﴿٤٩﴾}\\
\textamh{50.\  } & \mytextarabic{قَدْ قَالَهَا ٱلَّذِينَ مِن قَبْلِهِمْ فَمَآ أَغْنَىٰ عَنْهُم مَّا كَانُوا۟ يَكْسِبُونَ ﴿٥٠﴾}\\
\textamh{51.\  } & \mytextarabic{فَأَصَابَهُمْ سَيِّـَٔاتُ مَا كَسَبُوا۟ ۚ وَٱلَّذِينَ ظَلَمُوا۟ مِنْ هَـٰٓؤُلَآءِ سَيُصِيبُهُمْ سَيِّـَٔاتُ مَا كَسَبُوا۟ وَمَا هُم بِمُعْجِزِينَ ﴿٥١﴾}\\
\textamh{52.\  } & \mytextarabic{أَوَلَمْ يَعْلَمُوٓا۟ أَنَّ ٱللَّهَ يَبْسُطُ ٱلرِّزْقَ لِمَن يَشَآءُ وَيَقْدِرُ ۚ إِنَّ فِى ذَٟلِكَ لَءَايَـٰتٍۢ لِّقَوْمٍۢ يُؤْمِنُونَ ﴿٥٢﴾}\\
\textamh{53.\  } & \mytextarabic{۞ قُلْ يَـٰعِبَادِىَ ٱلَّذِينَ أَسْرَفُوا۟ عَلَىٰٓ أَنفُسِهِمْ لَا تَقْنَطُوا۟ مِن رَّحْمَةِ ٱللَّهِ ۚ إِنَّ ٱللَّهَ يَغْفِرُ ٱلذُّنُوبَ جَمِيعًا ۚ إِنَّهُۥ هُوَ ٱلْغَفُورُ ٱلرَّحِيمُ ﴿٥٣﴾}\\
\textamh{54.\  } & \mytextarabic{وَأَنِيبُوٓا۟ إِلَىٰ رَبِّكُمْ وَأَسْلِمُوا۟ لَهُۥ مِن قَبْلِ أَن يَأْتِيَكُمُ ٱلْعَذَابُ ثُمَّ لَا تُنصَرُونَ ﴿٥٤﴾}\\
\textamh{55.\  } & \mytextarabic{وَٱتَّبِعُوٓا۟ أَحْسَنَ مَآ أُنزِلَ إِلَيْكُم مِّن رَّبِّكُم مِّن قَبْلِ أَن يَأْتِيَكُمُ ٱلْعَذَابُ بَغْتَةًۭ وَأَنتُمْ لَا تَشْعُرُونَ ﴿٥٥﴾}\\
\textamh{56.\  } & \mytextarabic{أَن تَقُولَ نَفْسٌۭ يَـٰحَسْرَتَىٰ عَلَىٰ مَا فَرَّطتُ فِى جَنۢبِ ٱللَّهِ وَإِن كُنتُ لَمِنَ ٱلسَّٰخِرِينَ ﴿٥٦﴾}\\
\textamh{57.\  } & \mytextarabic{أَوْ تَقُولَ لَوْ أَنَّ ٱللَّهَ هَدَىٰنِى لَكُنتُ مِنَ ٱلْمُتَّقِينَ ﴿٥٧﴾}\\
\textamh{58.\  } & \mytextarabic{أَوْ تَقُولَ حِينَ تَرَى ٱلْعَذَابَ لَوْ أَنَّ لِى كَرَّةًۭ فَأَكُونَ مِنَ ٱلْمُحْسِنِينَ ﴿٥٨﴾}\\
\textamh{59.\  } & \mytextarabic{بَلَىٰ قَدْ جَآءَتْكَ ءَايَـٰتِى فَكَذَّبْتَ بِهَا وَٱسْتَكْبَرْتَ وَكُنتَ مِنَ ٱلْكَـٰفِرِينَ ﴿٥٩﴾}\\
\textamh{60.\  } & \mytextarabic{وَيَوْمَ ٱلْقِيَـٰمَةِ تَرَى ٱلَّذِينَ كَذَبُوا۟ عَلَى ٱللَّهِ وُجُوهُهُم مُّسْوَدَّةٌ ۚ أَلَيْسَ فِى جَهَنَّمَ مَثْوًۭى لِّلْمُتَكَبِّرِينَ ﴿٦٠﴾}\\
\textamh{61.\  } & \mytextarabic{وَيُنَجِّى ٱللَّهُ ٱلَّذِينَ ٱتَّقَوْا۟ بِمَفَازَتِهِمْ لَا يَمَسُّهُمُ ٱلسُّوٓءُ وَلَا هُمْ يَحْزَنُونَ ﴿٦١﴾}\\
\textamh{62.\  } & \mytextarabic{ٱللَّهُ خَـٰلِقُ كُلِّ شَىْءٍۢ ۖ وَهُوَ عَلَىٰ كُلِّ شَىْءٍۢ وَكِيلٌۭ ﴿٦٢﴾}\\
\textamh{63.\  } & \mytextarabic{لَّهُۥ مَقَالِيدُ ٱلسَّمَـٰوَٟتِ وَٱلْأَرْضِ ۗ وَٱلَّذِينَ كَفَرُوا۟ بِـَٔايَـٰتِ ٱللَّهِ أُو۟لَـٰٓئِكَ هُمُ ٱلْخَـٰسِرُونَ ﴿٦٣﴾}\\
\textamh{64.\  } & \mytextarabic{قُلْ أَفَغَيْرَ ٱللَّهِ تَأْمُرُوٓنِّىٓ أَعْبُدُ أَيُّهَا ٱلْجَٰهِلُونَ ﴿٦٤﴾}\\
\textamh{65.\  } & \mytextarabic{وَلَقَدْ أُوحِىَ إِلَيْكَ وَإِلَى ٱلَّذِينَ مِن قَبْلِكَ لَئِنْ أَشْرَكْتَ لَيَحْبَطَنَّ عَمَلُكَ وَلَتَكُونَنَّ مِنَ ٱلْخَـٰسِرِينَ ﴿٦٥﴾}\\
\textamh{66.\  } & \mytextarabic{بَلِ ٱللَّهَ فَٱعْبُدْ وَكُن مِّنَ ٱلشَّـٰكِرِينَ ﴿٦٦﴾}\\
\textamh{67.\  } & \mytextarabic{وَمَا قَدَرُوا۟ ٱللَّهَ حَقَّ قَدْرِهِۦ وَٱلْأَرْضُ جَمِيعًۭا قَبْضَتُهُۥ يَوْمَ ٱلْقِيَـٰمَةِ وَٱلسَّمَـٰوَٟتُ مَطْوِيَّٰتٌۢ بِيَمِينِهِۦ ۚ سُبْحَـٰنَهُۥ وَتَعَـٰلَىٰ عَمَّا يُشْرِكُونَ ﴿٦٧﴾}\\
\textamh{68.\  } & \mytextarabic{وَنُفِخَ فِى ٱلصُّورِ فَصَعِقَ مَن فِى ٱلسَّمَـٰوَٟتِ وَمَن فِى ٱلْأَرْضِ إِلَّا مَن شَآءَ ٱللَّهُ ۖ ثُمَّ نُفِخَ فِيهِ أُخْرَىٰ فَإِذَا هُمْ قِيَامٌۭ يَنظُرُونَ ﴿٦٨﴾}\\
\textamh{69.\  } & \mytextarabic{وَأَشْرَقَتِ ٱلْأَرْضُ بِنُورِ رَبِّهَا وَوُضِعَ ٱلْكِتَـٰبُ وَجِا۟ىٓءَ بِٱلنَّبِيِّۦنَ وَٱلشُّهَدَآءِ وَقُضِىَ بَيْنَهُم بِٱلْحَقِّ وَهُمْ لَا يُظْلَمُونَ ﴿٦٩﴾}\\
\textamh{70.\  } & \mytextarabic{وَوُفِّيَتْ كُلُّ نَفْسٍۢ مَّا عَمِلَتْ وَهُوَ أَعْلَمُ بِمَا يَفْعَلُونَ ﴿٧٠﴾}\\
\textamh{71.\  } & \mytextarabic{وَسِيقَ ٱلَّذِينَ كَفَرُوٓا۟ إِلَىٰ جَهَنَّمَ زُمَرًا ۖ حَتَّىٰٓ إِذَا جَآءُوهَا فُتِحَتْ أَبْوَٟبُهَا وَقَالَ لَهُمْ خَزَنَتُهَآ أَلَمْ يَأْتِكُمْ رُسُلٌۭ مِّنكُمْ يَتْلُونَ عَلَيْكُمْ ءَايَـٰتِ رَبِّكُمْ وَيُنذِرُونَكُمْ لِقَآءَ يَوْمِكُمْ هَـٰذَا ۚ قَالُوا۟ بَلَىٰ وَلَـٰكِنْ حَقَّتْ كَلِمَةُ ٱلْعَذَابِ عَلَى ٱلْكَـٰفِرِينَ ﴿٧١﴾}\\
\textamh{72.\  } & \mytextarabic{قِيلَ ٱدْخُلُوٓا۟ أَبْوَٟبَ جَهَنَّمَ خَـٰلِدِينَ فِيهَا ۖ فَبِئْسَ مَثْوَى ٱلْمُتَكَبِّرِينَ ﴿٧٢﴾}\\
\textamh{73.\  } & \mytextarabic{وَسِيقَ ٱلَّذِينَ ٱتَّقَوْا۟ رَبَّهُمْ إِلَى ٱلْجَنَّةِ زُمَرًا ۖ حَتَّىٰٓ إِذَا جَآءُوهَا وَفُتِحَتْ أَبْوَٟبُهَا وَقَالَ لَهُمْ خَزَنَتُهَا سَلَـٰمٌ عَلَيْكُمْ طِبْتُمْ فَٱدْخُلُوهَا خَـٰلِدِينَ ﴿٧٣﴾}\\
\textamh{74.\  } & \mytextarabic{وَقَالُوا۟ ٱلْحَمْدُ لِلَّهِ ٱلَّذِى صَدَقَنَا وَعْدَهُۥ وَأَوْرَثَنَا ٱلْأَرْضَ نَتَبَوَّأُ مِنَ ٱلْجَنَّةِ حَيْثُ نَشَآءُ ۖ فَنِعْمَ أَجْرُ ٱلْعَـٰمِلِينَ ﴿٧٤﴾}\\
\textamh{75.\  } & \mytextarabic{وَتَرَى ٱلْمَلَـٰٓئِكَةَ حَآفِّينَ مِنْ حَوْلِ ٱلْعَرْشِ يُسَبِّحُونَ بِحَمْدِ رَبِّهِمْ ۖ وَقُضِىَ بَيْنَهُم بِٱلْحَقِّ وَقِيلَ ٱلْحَمْدُ لِلَّهِ رَبِّ ٱلْعَـٰلَمِينَ ﴿٧٥﴾}\\
\end{longtable}
\clearpage