%% License: BSD style (Berkley) (i.e. Put the Copyright owner's name always)
%% Writer and Copyright (to): Bewketu(Bilal) Tadilo (2016-17)
\begin{center}\section{\LR{\textamhsec{ሱራቱ አልዱኻን -}  \textarabic{سوره  الدخان}}}\end{center}
\begin{longtable}{%
  @{}
    p{.5\textwidth}
  @{~~~}
    p{.5\textwidth}
    @{}
}
\textamh{ቢስሚላሂ አራህመኒ ራሂይም } &  \mytextarabic{بِسْمِ ٱللَّهِ ٱلرَّحْمَـٰنِ ٱلرَّحِيمِ}\\
\textamh{1.\  } & \mytextarabic{ حمٓ ﴿١﴾}\\
\textamh{2.\  } & \mytextarabic{وَٱلْكِتَـٰبِ ٱلْمُبِينِ ﴿٢﴾}\\
\textamh{3.\  } & \mytextarabic{إِنَّآ أَنزَلْنَـٰهُ فِى لَيْلَةٍۢ مُّبَٰرَكَةٍ ۚ إِنَّا كُنَّا مُنذِرِينَ ﴿٣﴾}\\
\textamh{4.\  } & \mytextarabic{فِيهَا يُفْرَقُ كُلُّ أَمْرٍ حَكِيمٍ ﴿٤﴾}\\
\textamh{5.\  } & \mytextarabic{أَمْرًۭا مِّنْ عِندِنَآ ۚ إِنَّا كُنَّا مُرْسِلِينَ ﴿٥﴾}\\
\textamh{6.\  } & \mytextarabic{رَحْمَةًۭ مِّن رَّبِّكَ ۚ إِنَّهُۥ هُوَ ٱلسَّمِيعُ ٱلْعَلِيمُ ﴿٦﴾}\\
\textamh{7.\  } & \mytextarabic{رَبِّ ٱلسَّمَـٰوَٟتِ وَٱلْأَرْضِ وَمَا بَيْنَهُمَآ ۖ إِن كُنتُم مُّوقِنِينَ ﴿٧﴾}\\
\textamh{8.\  } & \mytextarabic{لَآ إِلَـٰهَ إِلَّا هُوَ يُحْىِۦ وَيُمِيتُ ۖ رَبُّكُمْ وَرَبُّ ءَابَآئِكُمُ ٱلْأَوَّلِينَ ﴿٨﴾}\\
\textamh{9.\  } & \mytextarabic{بَلْ هُمْ فِى شَكٍّۢ يَلْعَبُونَ ﴿٩﴾}\\
\textamh{10.\  } & \mytextarabic{فَٱرْتَقِبْ يَوْمَ تَأْتِى ٱلسَّمَآءُ بِدُخَانٍۢ مُّبِينٍۢ ﴿١٠﴾}\\
\textamh{11.\  } & \mytextarabic{يَغْشَى ٱلنَّاسَ ۖ هَـٰذَا عَذَابٌ أَلِيمٌۭ ﴿١١﴾}\\
\textamh{12.\  } & \mytextarabic{رَّبَّنَا ٱكْشِفْ عَنَّا ٱلْعَذَابَ إِنَّا مُؤْمِنُونَ ﴿١٢﴾}\\
\textamh{13.\  } & \mytextarabic{أَنَّىٰ لَهُمُ ٱلذِّكْرَىٰ وَقَدْ جَآءَهُمْ رَسُولٌۭ مُّبِينٌۭ ﴿١٣﴾}\\
\textamh{14.\  } & \mytextarabic{ثُمَّ تَوَلَّوْا۟ عَنْهُ وَقَالُوا۟ مُعَلَّمٌۭ مَّجْنُونٌ ﴿١٤﴾}\\
\textamh{15.\  } & \mytextarabic{إِنَّا كَاشِفُوا۟ ٱلْعَذَابِ قَلِيلًا ۚ إِنَّكُمْ عَآئِدُونَ ﴿١٥﴾}\\
\textamh{16.\  } & \mytextarabic{يَوْمَ نَبْطِشُ ٱلْبَطْشَةَ ٱلْكُبْرَىٰٓ إِنَّا مُنتَقِمُونَ ﴿١٦﴾}\\
\textamh{17.\  } & \mytextarabic{۞ وَلَقَدْ فَتَنَّا قَبْلَهُمْ قَوْمَ فِرْعَوْنَ وَجَآءَهُمْ رَسُولٌۭ كَرِيمٌ ﴿١٧﴾}\\
\textamh{18.\  } & \mytextarabic{أَنْ أَدُّوٓا۟ إِلَىَّ عِبَادَ ٱللَّهِ ۖ إِنِّى لَكُمْ رَسُولٌ أَمِينٌۭ ﴿١٨﴾}\\
\textamh{19.\  } & \mytextarabic{وَأَن لَّا تَعْلُوا۟ عَلَى ٱللَّهِ ۖ إِنِّىٓ ءَاتِيكُم بِسُلْطَٰنٍۢ مُّبِينٍۢ ﴿١٩﴾}\\
\textamh{20.\  } & \mytextarabic{وَإِنِّى عُذْتُ بِرَبِّى وَرَبِّكُمْ أَن تَرْجُمُونِ ﴿٢٠﴾}\\
\textamh{21.\  } & \mytextarabic{وَإِن لَّمْ تُؤْمِنُوا۟ لِى فَٱعْتَزِلُونِ ﴿٢١﴾}\\
\textamh{22.\  } & \mytextarabic{فَدَعَا رَبَّهُۥٓ أَنَّ هَـٰٓؤُلَآءِ قَوْمٌۭ مُّجْرِمُونَ ﴿٢٢﴾}\\
\textamh{23.\  } & \mytextarabic{فَأَسْرِ بِعِبَادِى لَيْلًا إِنَّكُم مُّتَّبَعُونَ ﴿٢٣﴾}\\
\textamh{24.\  } & \mytextarabic{وَٱتْرُكِ ٱلْبَحْرَ رَهْوًا ۖ إِنَّهُمْ جُندٌۭ مُّغْرَقُونَ ﴿٢٤﴾}\\
\textamh{25.\  } & \mytextarabic{كَمْ تَرَكُوا۟ مِن جَنَّـٰتٍۢ وَعُيُونٍۢ ﴿٢٥﴾}\\
\textamh{26.\  } & \mytextarabic{وَزُرُوعٍۢ وَمَقَامٍۢ كَرِيمٍۢ ﴿٢٦﴾}\\
\textamh{27.\  } & \mytextarabic{وَنَعْمَةٍۢ كَانُوا۟ فِيهَا فَـٰكِهِينَ ﴿٢٧﴾}\\
\textamh{28.\  } & \mytextarabic{كَذَٟلِكَ ۖ وَأَوْرَثْنَـٰهَا قَوْمًا ءَاخَرِينَ ﴿٢٨﴾}\\
\textamh{29.\  } & \mytextarabic{فَمَا بَكَتْ عَلَيْهِمُ ٱلسَّمَآءُ وَٱلْأَرْضُ وَمَا كَانُوا۟ مُنظَرِينَ ﴿٢٩﴾}\\
\textamh{30.\  } & \mytextarabic{وَلَقَدْ نَجَّيْنَا بَنِىٓ إِسْرَٰٓءِيلَ مِنَ ٱلْعَذَابِ ٱلْمُهِينِ ﴿٣٠﴾}\\
\textamh{31.\  } & \mytextarabic{مِن فِرْعَوْنَ ۚ إِنَّهُۥ كَانَ عَالِيًۭا مِّنَ ٱلْمُسْرِفِينَ ﴿٣١﴾}\\
\textamh{32.\  } & \mytextarabic{وَلَقَدِ ٱخْتَرْنَـٰهُمْ عَلَىٰ عِلْمٍ عَلَى ٱلْعَـٰلَمِينَ ﴿٣٢﴾}\\
\textamh{33.\  } & \mytextarabic{وَءَاتَيْنَـٰهُم مِّنَ ٱلْءَايَـٰتِ مَا فِيهِ بَلَـٰٓؤٌۭا۟ مُّبِينٌ ﴿٣٣﴾}\\
\textamh{34.\  } & \mytextarabic{إِنَّ هَـٰٓؤُلَآءِ لَيَقُولُونَ ﴿٣٤﴾}\\
\textamh{35.\  } & \mytextarabic{إِنْ هِىَ إِلَّا مَوْتَتُنَا ٱلْأُولَىٰ وَمَا نَحْنُ بِمُنشَرِينَ ﴿٣٥﴾}\\
\textamh{36.\  } & \mytextarabic{فَأْتُوا۟ بِـَٔابَآئِنَآ إِن كُنتُمْ صَـٰدِقِينَ ﴿٣٦﴾}\\
\textamh{37.\  } & \mytextarabic{أَهُمْ خَيْرٌ أَمْ قَوْمُ تُبَّعٍۢ وَٱلَّذِينَ مِن قَبْلِهِمْ ۚ أَهْلَكْنَـٰهُمْ ۖ إِنَّهُمْ كَانُوا۟ مُجْرِمِينَ ﴿٣٧﴾}\\
\textamh{38.\  } & \mytextarabic{وَمَا خَلَقْنَا ٱلسَّمَـٰوَٟتِ وَٱلْأَرْضَ وَمَا بَيْنَهُمَا لَـٰعِبِينَ ﴿٣٨﴾}\\
\textamh{39.\  } & \mytextarabic{مَا خَلَقْنَـٰهُمَآ إِلَّا بِٱلْحَقِّ وَلَـٰكِنَّ أَكْثَرَهُمْ لَا يَعْلَمُونَ ﴿٣٩﴾}\\
\textamh{40.\  } & \mytextarabic{إِنَّ يَوْمَ ٱلْفَصْلِ مِيقَـٰتُهُمْ أَجْمَعِينَ ﴿٤٠﴾}\\
\textamh{41.\  } & \mytextarabic{يَوْمَ لَا يُغْنِى مَوْلًى عَن مَّوْلًۭى شَيْـًۭٔا وَلَا هُمْ يُنصَرُونَ ﴿٤١﴾}\\
\textamh{42.\  } & \mytextarabic{إِلَّا مَن رَّحِمَ ٱللَّهُ ۚ إِنَّهُۥ هُوَ ٱلْعَزِيزُ ٱلرَّحِيمُ ﴿٤٢﴾}\\
\textamh{43.\  } & \mytextarabic{إِنَّ شَجَرَتَ ٱلزَّقُّومِ ﴿٤٣﴾}\\
\textamh{44.\  } & \mytextarabic{طَعَامُ ٱلْأَثِيمِ ﴿٤٤﴾}\\
\textamh{45.\  } & \mytextarabic{كَٱلْمُهْلِ يَغْلِى فِى ٱلْبُطُونِ ﴿٤٥﴾}\\
\textamh{46.\  } & \mytextarabic{كَغَلْىِ ٱلْحَمِيمِ ﴿٤٦﴾}\\
\textamh{47.\  } & \mytextarabic{خُذُوهُ فَٱعْتِلُوهُ إِلَىٰ سَوَآءِ ٱلْجَحِيمِ ﴿٤٧﴾}\\
\textamh{48.\  } & \mytextarabic{ثُمَّ صُبُّوا۟ فَوْقَ رَأْسِهِۦ مِنْ عَذَابِ ٱلْحَمِيمِ ﴿٤٨﴾}\\
\textamh{49.\  } & \mytextarabic{ذُقْ إِنَّكَ أَنتَ ٱلْعَزِيزُ ٱلْكَرِيمُ ﴿٤٩﴾}\\
\textamh{50.\  } & \mytextarabic{إِنَّ هَـٰذَا مَا كُنتُم بِهِۦ تَمْتَرُونَ ﴿٥٠﴾}\\
\textamh{51.\  } & \mytextarabic{إِنَّ ٱلْمُتَّقِينَ فِى مَقَامٍ أَمِينٍۢ ﴿٥١﴾}\\
\textamh{52.\  } & \mytextarabic{فِى جَنَّـٰتٍۢ وَعُيُونٍۢ ﴿٥٢﴾}\\
\textamh{53.\  } & \mytextarabic{يَلْبَسُونَ مِن سُندُسٍۢ وَإِسْتَبْرَقٍۢ مُّتَقَـٰبِلِينَ ﴿٥٣﴾}\\
\textamh{54.\  } & \mytextarabic{كَذَٟلِكَ وَزَوَّجْنَـٰهُم بِحُورٍ عِينٍۢ ﴿٥٤﴾}\\
\textamh{55.\  } & \mytextarabic{يَدْعُونَ فِيهَا بِكُلِّ فَـٰكِهَةٍ ءَامِنِينَ ﴿٥٥﴾}\\
\textamh{56.\  } & \mytextarabic{لَا يَذُوقُونَ فِيهَا ٱلْمَوْتَ إِلَّا ٱلْمَوْتَةَ ٱلْأُولَىٰ ۖ وَوَقَىٰهُمْ عَذَابَ ٱلْجَحِيمِ ﴿٥٦﴾}\\
\textamh{57.\  } & \mytextarabic{فَضْلًۭا مِّن رَّبِّكَ ۚ ذَٟلِكَ هُوَ ٱلْفَوْزُ ٱلْعَظِيمُ ﴿٥٧﴾}\\
\textamh{58.\  } & \mytextarabic{فَإِنَّمَا يَسَّرْنَـٰهُ بِلِسَانِكَ لَعَلَّهُمْ يَتَذَكَّرُونَ ﴿٥٨﴾}\\
\textamh{59.\  } & \mytextarabic{فَٱرْتَقِبْ إِنَّهُم مُّرْتَقِبُونَ ﴿٥٩﴾}\\
\end{longtable}
\clearpage