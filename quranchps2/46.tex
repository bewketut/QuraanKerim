%% License: BSD style (Berkley) (i.e. Put the Copyright owner's name always)
%% Writer and Copyright (to): Bewketu(Bilal) Tadilo (2016-17)
\centering\section{\LR{\textamharic{ሱራቱ አልኣህቃፍ -}  \RL{سوره  الأحقاف}}}
\begin{longtable}{%
  @{}
    p{.5\textwidth}
  @{~~~~~~~~~~~~~}
    p{.5\textwidth}
    @{}
}
\nopagebreak
\textamh{ቢስሚላሂ አራህመኒ ራሂይም } &  بِسْمِ ٱللَّهِ ٱلرَّحْمَـٰنِ ٱلرَّحِيمِ\\
\textamh{1.\  } &  حمٓ ﴿١﴾\\
\textamh{2.\  } & تَنزِيلُ ٱلْكِتَـٰبِ مِنَ ٱللَّهِ ٱلْعَزِيزِ ٱلْحَكِيمِ ﴿٢﴾\\
\textamh{3.\  } & مَا خَلَقْنَا ٱلسَّمَـٰوَٟتِ وَٱلْأَرْضَ وَمَا بَيْنَهُمَآ إِلَّا بِٱلْحَقِّ وَأَجَلٍۢ مُّسَمًّۭى ۚ وَٱلَّذِينَ كَفَرُوا۟ عَمَّآ أُنذِرُوا۟ مُعْرِضُونَ ﴿٣﴾\\
\textamh{4.\  } & قُلْ أَرَءَيْتُم مَّا تَدْعُونَ مِن دُونِ ٱللَّهِ أَرُونِى مَاذَا خَلَقُوا۟ مِنَ ٱلْأَرْضِ أَمْ لَهُمْ شِرْكٌۭ فِى ٱلسَّمَـٰوَٟتِ ۖ ٱئْتُونِى بِكِتَـٰبٍۢ مِّن قَبْلِ هَـٰذَآ أَوْ أَثَـٰرَةٍۢ مِّنْ عِلْمٍ إِن كُنتُمْ صَـٰدِقِينَ ﴿٤﴾\\
\textamh{5.\  } & وَمَنْ أَضَلُّ مِمَّن يَدْعُوا۟ مِن دُونِ ٱللَّهِ مَن لَّا يَسْتَجِيبُ لَهُۥٓ إِلَىٰ يَوْمِ ٱلْقِيَـٰمَةِ وَهُمْ عَن دُعَآئِهِمْ غَٰفِلُونَ ﴿٥﴾\\
\textamh{6.\  } & وَإِذَا حُشِرَ ٱلنَّاسُ كَانُوا۟ لَهُمْ أَعْدَآءًۭ وَكَانُوا۟ بِعِبَادَتِهِمْ كَـٰفِرِينَ ﴿٦﴾\\
\textamh{7.\  } & وَإِذَا تُتْلَىٰ عَلَيْهِمْ ءَايَـٰتُنَا بَيِّنَـٰتٍۢ قَالَ ٱلَّذِينَ كَفَرُوا۟ لِلْحَقِّ لَمَّا جَآءَهُمْ هَـٰذَا سِحْرٌۭ مُّبِينٌ ﴿٧﴾\\
\textamh{8.\  } & أَمْ يَقُولُونَ ٱفْتَرَىٰهُ ۖ قُلْ إِنِ ٱفْتَرَيْتُهُۥ فَلَا تَمْلِكُونَ لِى مِنَ ٱللَّهِ شَيْـًٔا ۖ هُوَ أَعْلَمُ بِمَا تُفِيضُونَ فِيهِ ۖ كَفَىٰ بِهِۦ شَهِيدًۢا بَيْنِى وَبَيْنَكُمْ ۖ وَهُوَ ٱلْغَفُورُ ٱلرَّحِيمُ ﴿٨﴾\\
\textamh{9.\  } & قُلْ مَا كُنتُ بِدْعًۭا مِّنَ ٱلرُّسُلِ وَمَآ أَدْرِى مَا يُفْعَلُ بِى وَلَا بِكُمْ ۖ إِنْ أَتَّبِعُ إِلَّا مَا يُوحَىٰٓ إِلَىَّ وَمَآ أَنَا۠ إِلَّا نَذِيرٌۭ مُّبِينٌۭ ﴿٩﴾\\
\textamh{10.\  } & قُلْ أَرَءَيْتُمْ إِن كَانَ مِنْ عِندِ ٱللَّهِ وَكَفَرْتُم بِهِۦ وَشَهِدَ شَاهِدٌۭ مِّنۢ بَنِىٓ إِسْرَٰٓءِيلَ عَلَىٰ مِثْلِهِۦ فَـَٔامَنَ وَٱسْتَكْبَرْتُمْ ۖ إِنَّ ٱللَّهَ لَا يَهْدِى ٱلْقَوْمَ ٱلظَّـٰلِمِينَ ﴿١٠﴾\\
\textamh{11.\  } & وَقَالَ ٱلَّذِينَ كَفَرُوا۟ لِلَّذِينَ ءَامَنُوا۟ لَوْ كَانَ خَيْرًۭا مَّا سَبَقُونَآ إِلَيْهِ ۚ وَإِذْ لَمْ يَهْتَدُوا۟ بِهِۦ فَسَيَقُولُونَ هَـٰذَآ إِفْكٌۭ قَدِيمٌۭ ﴿١١﴾\\
\textamh{12.\  } & وَمِن قَبْلِهِۦ كِتَـٰبُ مُوسَىٰٓ إِمَامًۭا وَرَحْمَةًۭ ۚ وَهَـٰذَا كِتَـٰبٌۭ مُّصَدِّقٌۭ لِّسَانًا عَرَبِيًّۭا لِّيُنذِرَ ٱلَّذِينَ ظَلَمُوا۟ وَبُشْرَىٰ لِلْمُحْسِنِينَ ﴿١٢﴾\\
\textamh{13.\  } & إِنَّ ٱلَّذِينَ قَالُوا۟ رَبُّنَا ٱللَّهُ ثُمَّ ٱسْتَقَـٰمُوا۟ فَلَا خَوْفٌ عَلَيْهِمْ وَلَا هُمْ يَحْزَنُونَ ﴿١٣﴾\\
\textamh{14.\  } & أُو۟لَـٰٓئِكَ أَصْحَـٰبُ ٱلْجَنَّةِ خَـٰلِدِينَ فِيهَا جَزَآءًۢ بِمَا كَانُوا۟ يَعْمَلُونَ ﴿١٤﴾\\
\textamh{15.\  } & وَوَصَّيْنَا ٱلْإِنسَـٰنَ بِوَٟلِدَيْهِ إِحْسَـٰنًا ۖ حَمَلَتْهُ أُمُّهُۥ كُرْهًۭا وَوَضَعَتْهُ كُرْهًۭا ۖ وَحَمْلُهُۥ وَفِصَـٰلُهُۥ ثَلَـٰثُونَ شَهْرًا ۚ حَتَّىٰٓ إِذَا بَلَغَ أَشُدَّهُۥ وَبَلَغَ أَرْبَعِينَ سَنَةًۭ قَالَ رَبِّ أَوْزِعْنِىٓ أَنْ أَشْكُرَ نِعْمَتَكَ ٱلَّتِىٓ أَنْعَمْتَ عَلَىَّ وَعَلَىٰ وَٟلِدَىَّ وَأَنْ أَعْمَلَ صَـٰلِحًۭا تَرْضَىٰهُ وَأَصْلِحْ لِى فِى ذُرِّيَّتِىٓ ۖ إِنِّى تُبْتُ إِلَيْكَ وَإِنِّى مِنَ ٱلْمُسْلِمِينَ ﴿١٥﴾\\
\textamh{16.\  } & أُو۟لَـٰٓئِكَ ٱلَّذِينَ نَتَقَبَّلُ عَنْهُمْ أَحْسَنَ مَا عَمِلُوا۟ وَنَتَجَاوَزُ عَن سَيِّـَٔاتِهِمْ فِىٓ أَصْحَـٰبِ ٱلْجَنَّةِ ۖ وَعْدَ ٱلصِّدْقِ ٱلَّذِى كَانُوا۟ يُوعَدُونَ ﴿١٦﴾\\
\textamh{17.\  } & وَٱلَّذِى قَالَ لِوَٟلِدَيْهِ أُفٍّۢ لَّكُمَآ أَتَعِدَانِنِىٓ أَنْ أُخْرَجَ وَقَدْ خَلَتِ ٱلْقُرُونُ مِن قَبْلِى وَهُمَا يَسْتَغِيثَانِ ٱللَّهَ وَيْلَكَ ءَامِنْ إِنَّ وَعْدَ ٱللَّهِ حَقٌّۭ فَيَقُولُ مَا هَـٰذَآ إِلَّآ أَسَـٰطِيرُ ٱلْأَوَّلِينَ ﴿١٧﴾\\
\textamh{18.\  } & أُو۟لَـٰٓئِكَ ٱلَّذِينَ حَقَّ عَلَيْهِمُ ٱلْقَوْلُ فِىٓ أُمَمٍۢ قَدْ خَلَتْ مِن قَبْلِهِم مِّنَ ٱلْجِنِّ وَٱلْإِنسِ ۖ إِنَّهُمْ كَانُوا۟ خَـٰسِرِينَ ﴿١٨﴾\\
\textamh{19.\  } & وَلِكُلٍّۢ دَرَجَٰتٌۭ مِّمَّا عَمِلُوا۟ ۖ وَلِيُوَفِّيَهُمْ أَعْمَـٰلَهُمْ وَهُمْ لَا يُظْلَمُونَ ﴿١٩﴾\\
\textamh{20.\  } & وَيَوْمَ يُعْرَضُ ٱلَّذِينَ كَفَرُوا۟ عَلَى ٱلنَّارِ أَذْهَبْتُمْ طَيِّبَٰتِكُمْ فِى حَيَاتِكُمُ ٱلدُّنْيَا وَٱسْتَمْتَعْتُم بِهَا فَٱلْيَوْمَ تُجْزَوْنَ عَذَابَ ٱلْهُونِ بِمَا كُنتُمْ تَسْتَكْبِرُونَ فِى ٱلْأَرْضِ بِغَيْرِ ٱلْحَقِّ وَبِمَا كُنتُمْ تَفْسُقُونَ ﴿٢٠﴾\\
\textamh{21.\  } & ۞ وَٱذْكُرْ أَخَا عَادٍ إِذْ أَنذَرَ قَوْمَهُۥ بِٱلْأَحْقَافِ وَقَدْ خَلَتِ ٱلنُّذُرُ مِنۢ بَيْنِ يَدَيْهِ وَمِنْ خَلْفِهِۦٓ أَلَّا تَعْبُدُوٓا۟ إِلَّا ٱللَّهَ إِنِّىٓ أَخَافُ عَلَيْكُمْ عَذَابَ يَوْمٍ عَظِيمٍۢ ﴿٢١﴾\\
\textamh{22.\  } & قَالُوٓا۟ أَجِئْتَنَا لِتَأْفِكَنَا عَنْ ءَالِهَتِنَا فَأْتِنَا بِمَا تَعِدُنَآ إِن كُنتَ مِنَ ٱلصَّـٰدِقِينَ ﴿٢٢﴾\\
\textamh{23.\  } & قَالَ إِنَّمَا ٱلْعِلْمُ عِندَ ٱللَّهِ وَأُبَلِّغُكُم مَّآ أُرْسِلْتُ بِهِۦ وَلَـٰكِنِّىٓ أَرَىٰكُمْ قَوْمًۭا تَجْهَلُونَ ﴿٢٣﴾\\
\textamh{24.\  } & فَلَمَّا رَأَوْهُ عَارِضًۭا مُّسْتَقْبِلَ أَوْدِيَتِهِمْ قَالُوا۟ هَـٰذَا عَارِضٌۭ مُّمْطِرُنَا ۚ بَلْ هُوَ مَا ٱسْتَعْجَلْتُم بِهِۦ ۖ رِيحٌۭ فِيهَا عَذَابٌ أَلِيمٌۭ ﴿٢٤﴾\\
\textamh{25.\  } & تُدَمِّرُ كُلَّ شَىْءٍۭ بِأَمْرِ رَبِّهَا فَأَصْبَحُوا۟ لَا يُرَىٰٓ إِلَّا مَسَـٰكِنُهُمْ ۚ كَذَٟلِكَ نَجْزِى ٱلْقَوْمَ ٱلْمُجْرِمِينَ ﴿٢٥﴾\\
\textamh{26.\  } & وَلَقَدْ مَكَّنَّـٰهُمْ فِيمَآ إِن مَّكَّنَّـٰكُمْ فِيهِ وَجَعَلْنَا لَهُمْ سَمْعًۭا وَأَبْصَـٰرًۭا وَأَفْـِٔدَةًۭ فَمَآ أَغْنَىٰ عَنْهُمْ سَمْعُهُمْ وَلَآ أَبْصَـٰرُهُمْ وَلَآ أَفْـِٔدَتُهُم مِّن شَىْءٍ إِذْ كَانُوا۟ يَجْحَدُونَ بِـَٔايَـٰتِ ٱللَّهِ وَحَاقَ بِهِم مَّا كَانُوا۟ بِهِۦ يَسْتَهْزِءُونَ ﴿٢٦﴾\\
\textamh{27.\  } & وَلَقَدْ أَهْلَكْنَا مَا حَوْلَكُم مِّنَ ٱلْقُرَىٰ وَصَرَّفْنَا ٱلْءَايَـٰتِ لَعَلَّهُمْ يَرْجِعُونَ ﴿٢٧﴾\\
\textamh{28.\  } & فَلَوْلَا نَصَرَهُمُ ٱلَّذِينَ ٱتَّخَذُوا۟ مِن دُونِ ٱللَّهِ قُرْبَانًا ءَالِهَةًۢ ۖ بَلْ ضَلُّوا۟ عَنْهُمْ ۚ وَذَٟلِكَ إِفْكُهُمْ وَمَا كَانُوا۟ يَفْتَرُونَ ﴿٢٨﴾\\
\textamh{29.\  } & وَإِذْ صَرَفْنَآ إِلَيْكَ نَفَرًۭا مِّنَ ٱلْجِنِّ يَسْتَمِعُونَ ٱلْقُرْءَانَ فَلَمَّا حَضَرُوهُ قَالُوٓا۟ أَنصِتُوا۟ ۖ فَلَمَّا قُضِىَ وَلَّوْا۟ إِلَىٰ قَوْمِهِم مُّنذِرِينَ ﴿٢٩﴾\\
\textamh{30.\  } & قَالُوا۟ يَـٰقَوْمَنَآ إِنَّا سَمِعْنَا كِتَـٰبًا أُنزِلَ مِنۢ بَعْدِ مُوسَىٰ مُصَدِّقًۭا لِّمَا بَيْنَ يَدَيْهِ يَهْدِىٓ إِلَى ٱلْحَقِّ وَإِلَىٰ طَرِيقٍۢ مُّسْتَقِيمٍۢ ﴿٣٠﴾\\
\textamh{31.\  } & يَـٰقَوْمَنَآ أَجِيبُوا۟ دَاعِىَ ٱللَّهِ وَءَامِنُوا۟ بِهِۦ يَغْفِرْ لَكُم مِّن ذُنُوبِكُمْ وَيُجِرْكُم مِّنْ عَذَابٍ أَلِيمٍۢ ﴿٣١﴾\\
\textamh{32.\  } & وَمَن لَّا يُجِبْ دَاعِىَ ٱللَّهِ فَلَيْسَ بِمُعْجِزٍۢ فِى ٱلْأَرْضِ وَلَيْسَ لَهُۥ مِن دُونِهِۦٓ أَوْلِيَآءُ ۚ أُو۟لَـٰٓئِكَ فِى ضَلَـٰلٍۢ مُّبِينٍ ﴿٣٢﴾\\
\textamh{33.\  } & أَوَلَمْ يَرَوْا۟ أَنَّ ٱللَّهَ ٱلَّذِى خَلَقَ ٱلسَّمَـٰوَٟتِ وَٱلْأَرْضَ وَلَمْ يَعْىَ بِخَلْقِهِنَّ بِقَـٰدِرٍ عَلَىٰٓ أَن يُحْۦِىَ ٱلْمَوْتَىٰ ۚ بَلَىٰٓ إِنَّهُۥ عَلَىٰ كُلِّ شَىْءٍۢ قَدِيرٌۭ ﴿٣٣﴾\\
\textamh{34.\  } & وَيَوْمَ يُعْرَضُ ٱلَّذِينَ كَفَرُوا۟ عَلَى ٱلنَّارِ أَلَيْسَ هَـٰذَا بِٱلْحَقِّ ۖ قَالُوا۟ بَلَىٰ وَرَبِّنَا ۚ قَالَ فَذُوقُوا۟ ٱلْعَذَابَ بِمَا كُنتُمْ تَكْفُرُونَ ﴿٣٤﴾\\
\textamh{35.\  } & فَٱصْبِرْ كَمَا صَبَرَ أُو۟لُوا۟ ٱلْعَزْمِ مِنَ ٱلرُّسُلِ وَلَا تَسْتَعْجِل لَّهُمْ ۚ كَأَنَّهُمْ يَوْمَ يَرَوْنَ مَا يُوعَدُونَ لَمْ يَلْبَثُوٓا۟ إِلَّا سَاعَةًۭ مِّن نَّهَارٍۭ ۚ بَلَـٰغٌۭ ۚ فَهَلْ يُهْلَكُ إِلَّا ٱلْقَوْمُ ٱلْفَـٰسِقُونَ ﴿٣٥﴾\\
\end{longtable}
\clearpage