%% License: BSD style (Berkley) (i.e. Put the Copyright owner's name always)
%% Writer and Copyright (to): Bewketu(Bilal) Tadilo (2016-17)
\centering\section{\LR{\textamharic{ሱራቱ አልመአዳ -}  \RL{سوره  المائدة}}}
\begin{longtable}{%
  @{}
    p{.5\textwidth}
  @{~~~~~~~~~~~~~}
    p{.5\textwidth}
    @{}
}
\nopagebreak
\textamh{ቢስሚላሂ አራህመኒ ራሂይም } &  بِسْمِ ٱللَّهِ ٱلرَّحْمَـٰنِ ٱلرَّحِيمِ\\
\textamh{1.\  } &  يَـٰٓأَيُّهَا ٱلَّذِينَ ءَامَنُوٓا۟ أَوْفُوا۟ بِٱلْعُقُودِ ۚ أُحِلَّتْ لَكُم بَهِيمَةُ ٱلْأَنْعَـٰمِ إِلَّا مَا يُتْلَىٰ عَلَيْكُمْ غَيْرَ مُحِلِّى ٱلصَّيْدِ وَأَنتُمْ حُرُمٌ ۗ إِنَّ ٱللَّهَ يَحْكُمُ مَا يُرِيدُ ﴿١﴾\\
\textamh{2.\  } & يَـٰٓأَيُّهَا ٱلَّذِينَ ءَامَنُوا۟ لَا تُحِلُّوا۟ شَعَـٰٓئِرَ ٱللَّهِ وَلَا ٱلشَّهْرَ ٱلْحَرَامَ وَلَا ٱلْهَدْىَ وَلَا ٱلْقَلَـٰٓئِدَ وَلَآ ءَآمِّينَ ٱلْبَيْتَ ٱلْحَرَامَ يَبْتَغُونَ فَضْلًۭا مِّن رَّبِّهِمْ وَرِضْوَٟنًۭا ۚ وَإِذَا حَلَلْتُمْ فَٱصْطَادُوا۟ ۚ وَلَا يَجْرِمَنَّكُمْ شَنَـَٔانُ قَوْمٍ أَن صَدُّوكُمْ عَنِ ٱلْمَسْجِدِ ٱلْحَرَامِ أَن تَعْتَدُوا۟ ۘ وَتَعَاوَنُوا۟ عَلَى ٱلْبِرِّ وَٱلتَّقْوَىٰ ۖ وَلَا تَعَاوَنُوا۟ عَلَى ٱلْإِثْمِ وَٱلْعُدْوَٟنِ ۚ وَٱتَّقُوا۟ ٱللَّهَ ۖ إِنَّ ٱللَّهَ شَدِيدُ ٱلْعِقَابِ ﴿٢﴾\\
\textamh{3.\  } & حُرِّمَتْ عَلَيْكُمُ ٱلْمَيْتَةُ وَٱلدَّمُ وَلَحْمُ ٱلْخِنزِيرِ وَمَآ أُهِلَّ لِغَيْرِ ٱللَّهِ بِهِۦ وَٱلْمُنْخَنِقَةُ وَٱلْمَوْقُوذَةُ وَٱلْمُتَرَدِّيَةُ وَٱلنَّطِيحَةُ وَمَآ أَكَلَ ٱلسَّبُعُ إِلَّا مَا ذَكَّيْتُمْ وَمَا ذُبِحَ عَلَى ٱلنُّصُبِ وَأَن تَسْتَقْسِمُوا۟ بِٱلْأَزْلَـٰمِ ۚ ذَٟلِكُمْ فِسْقٌ ۗ ٱلْيَوْمَ يَئِسَ ٱلَّذِينَ كَفَرُوا۟ مِن دِينِكُمْ فَلَا تَخْشَوْهُمْ وَٱخْشَوْنِ ۚ ٱلْيَوْمَ أَكْمَلْتُ لَكُمْ دِينَكُمْ وَأَتْمَمْتُ عَلَيْكُمْ نِعْمَتِى وَرَضِيتُ لَكُمُ ٱلْإِسْلَـٰمَ دِينًۭا ۚ فَمَنِ ٱضْطُرَّ فِى مَخْمَصَةٍ غَيْرَ مُتَجَانِفٍۢ لِّإِثْمٍۢ ۙ فَإِنَّ ٱللَّهَ غَفُورٌۭ رَّحِيمٌۭ ﴿٣﴾\\
\textamh{4.\  } & يَسْـَٔلُونَكَ مَاذَآ أُحِلَّ لَهُمْ ۖ قُلْ أُحِلَّ لَكُمُ ٱلطَّيِّبَٰتُ ۙ وَمَا عَلَّمْتُم مِّنَ ٱلْجَوَارِحِ مُكَلِّبِينَ تُعَلِّمُونَهُنَّ مِمَّا عَلَّمَكُمُ ٱللَّهُ ۖ فَكُلُوا۟ مِمَّآ أَمْسَكْنَ عَلَيْكُمْ وَٱذْكُرُوا۟ ٱسْمَ ٱللَّهِ عَلَيْهِ ۖ وَٱتَّقُوا۟ ٱللَّهَ ۚ إِنَّ ٱللَّهَ سَرِيعُ ٱلْحِسَابِ ﴿٤﴾\\
\textamh{5.\  } & ٱلْيَوْمَ أُحِلَّ لَكُمُ ٱلطَّيِّبَٰتُ ۖ وَطَعَامُ ٱلَّذِينَ أُوتُوا۟ ٱلْكِتَـٰبَ حِلٌّۭ لَّكُمْ وَطَعَامُكُمْ حِلٌّۭ لَّهُمْ ۖ وَٱلْمُحْصَنَـٰتُ مِنَ ٱلْمُؤْمِنَـٰتِ وَٱلْمُحْصَنَـٰتُ مِنَ ٱلَّذِينَ أُوتُوا۟ ٱلْكِتَـٰبَ مِن قَبْلِكُمْ إِذَآ ءَاتَيْتُمُوهُنَّ أُجُورَهُنَّ مُحْصِنِينَ غَيْرَ مُسَـٰفِحِينَ وَلَا مُتَّخِذِىٓ أَخْدَانٍۢ ۗ وَمَن يَكْفُرْ بِٱلْإِيمَـٰنِ فَقَدْ حَبِطَ عَمَلُهُۥ وَهُوَ فِى ٱلْءَاخِرَةِ مِنَ ٱلْخَـٰسِرِينَ ﴿٥﴾\\
\textamh{6.\  } & يَـٰٓأَيُّهَا ٱلَّذِينَ ءَامَنُوٓا۟ إِذَا قُمْتُمْ إِلَى ٱلصَّلَوٰةِ فَٱغْسِلُوا۟ وُجُوهَكُمْ وَأَيْدِيَكُمْ إِلَى ٱلْمَرَافِقِ وَٱمْسَحُوا۟ بِرُءُوسِكُمْ وَأَرْجُلَكُمْ إِلَى ٱلْكَعْبَيْنِ ۚ وَإِن كُنتُمْ جُنُبًۭا فَٱطَّهَّرُوا۟ ۚ وَإِن كُنتُم مَّرْضَىٰٓ أَوْ عَلَىٰ سَفَرٍ أَوْ جَآءَ أَحَدٌۭ مِّنكُم مِّنَ ٱلْغَآئِطِ أَوْ لَـٰمَسْتُمُ ٱلنِّسَآءَ فَلَمْ تَجِدُوا۟ مَآءًۭ فَتَيَمَّمُوا۟ صَعِيدًۭا طَيِّبًۭا فَٱمْسَحُوا۟ بِوُجُوهِكُمْ وَأَيْدِيكُم مِّنْهُ ۚ مَا يُرِيدُ ٱللَّهُ لِيَجْعَلَ عَلَيْكُم مِّنْ حَرَجٍۢ وَلَـٰكِن يُرِيدُ لِيُطَهِّرَكُمْ وَلِيُتِمَّ نِعْمَتَهُۥ عَلَيْكُمْ لَعَلَّكُمْ تَشْكُرُونَ ﴿٦﴾\\
\textamh{7.\  } & وَٱذْكُرُوا۟ نِعْمَةَ ٱللَّهِ عَلَيْكُمْ وَمِيثَـٰقَهُ ٱلَّذِى وَاثَقَكُم بِهِۦٓ إِذْ قُلْتُمْ سَمِعْنَا وَأَطَعْنَا ۖ وَٱتَّقُوا۟ ٱللَّهَ ۚ إِنَّ ٱللَّهَ عَلِيمٌۢ بِذَاتِ ٱلصُّدُورِ ﴿٧﴾\\
\textamh{8.\  } & يَـٰٓأَيُّهَا ٱلَّذِينَ ءَامَنُوا۟ كُونُوا۟ قَوَّٰمِينَ لِلَّهِ شُهَدَآءَ بِٱلْقِسْطِ ۖ وَلَا يَجْرِمَنَّكُمْ شَنَـَٔانُ قَوْمٍ عَلَىٰٓ أَلَّا تَعْدِلُوا۟ ۚ ٱعْدِلُوا۟ هُوَ أَقْرَبُ لِلتَّقْوَىٰ ۖ وَٱتَّقُوا۟ ٱللَّهَ ۚ إِنَّ ٱللَّهَ خَبِيرٌۢ بِمَا تَعْمَلُونَ ﴿٨﴾\\
\textamh{9.\  } & وَعَدَ ٱللَّهُ ٱلَّذِينَ ءَامَنُوا۟ وَعَمِلُوا۟ ٱلصَّـٰلِحَـٰتِ ۙ لَهُم مَّغْفِرَةٌۭ وَأَجْرٌ عَظِيمٌۭ ﴿٩﴾\\
\textamh{10.\  } & وَٱلَّذِينَ كَفَرُوا۟ وَكَذَّبُوا۟ بِـَٔايَـٰتِنَآ أُو۟لَـٰٓئِكَ أَصْحَـٰبُ ٱلْجَحِيمِ ﴿١٠﴾\\
\textamh{11.\  } & يَـٰٓأَيُّهَا ٱلَّذِينَ ءَامَنُوا۟ ٱذْكُرُوا۟ نِعْمَتَ ٱللَّهِ عَلَيْكُمْ إِذْ هَمَّ قَوْمٌ أَن يَبْسُطُوٓا۟ إِلَيْكُمْ أَيْدِيَهُمْ فَكَفَّ أَيْدِيَهُمْ عَنكُمْ ۖ وَٱتَّقُوا۟ ٱللَّهَ ۚ وَعَلَى ٱللَّهِ فَلْيَتَوَكَّلِ ٱلْمُؤْمِنُونَ ﴿١١﴾\\
\textamh{12.\  } & ۞ وَلَقَدْ أَخَذَ ٱللَّهُ مِيثَـٰقَ بَنِىٓ إِسْرَٰٓءِيلَ وَبَعَثْنَا مِنْهُمُ ٱثْنَىْ عَشَرَ نَقِيبًۭا ۖ وَقَالَ ٱللَّهُ إِنِّى مَعَكُمْ ۖ لَئِنْ أَقَمْتُمُ ٱلصَّلَوٰةَ وَءَاتَيْتُمُ ٱلزَّكَوٰةَ وَءَامَنتُم بِرُسُلِى وَعَزَّرْتُمُوهُمْ وَأَقْرَضْتُمُ ٱللَّهَ قَرْضًا حَسَنًۭا لَّأُكَفِّرَنَّ عَنكُمْ سَيِّـَٔاتِكُمْ وَلَأُدْخِلَنَّكُمْ جَنَّـٰتٍۢ تَجْرِى مِن تَحْتِهَا ٱلْأَنْهَـٰرُ ۚ فَمَن كَفَرَ بَعْدَ ذَٟلِكَ مِنكُمْ فَقَدْ ضَلَّ سَوَآءَ ٱلسَّبِيلِ ﴿١٢﴾\\
\textamh{13.\  } & فَبِمَا نَقْضِهِم مِّيثَـٰقَهُمْ لَعَنَّـٰهُمْ وَجَعَلْنَا قُلُوبَهُمْ قَـٰسِيَةًۭ ۖ يُحَرِّفُونَ ٱلْكَلِمَ عَن مَّوَاضِعِهِۦ ۙ وَنَسُوا۟ حَظًّۭا مِّمَّا ذُكِّرُوا۟ بِهِۦ ۚ وَلَا تَزَالُ تَطَّلِعُ عَلَىٰ خَآئِنَةٍۢ مِّنْهُمْ إِلَّا قَلِيلًۭا مِّنْهُمْ ۖ فَٱعْفُ عَنْهُمْ وَٱصْفَحْ ۚ إِنَّ ٱللَّهَ يُحِبُّ ٱلْمُحْسِنِينَ ﴿١٣﴾\\
\textamh{14.\  } & وَمِنَ ٱلَّذِينَ قَالُوٓا۟ إِنَّا نَصَـٰرَىٰٓ أَخَذْنَا مِيثَـٰقَهُمْ فَنَسُوا۟ حَظًّۭا مِّمَّا ذُكِّرُوا۟ بِهِۦ فَأَغْرَيْنَا بَيْنَهُمُ ٱلْعَدَاوَةَ وَٱلْبَغْضَآءَ إِلَىٰ يَوْمِ ٱلْقِيَـٰمَةِ ۚ وَسَوْفَ يُنَبِّئُهُمُ ٱللَّهُ بِمَا كَانُوا۟ يَصْنَعُونَ ﴿١٤﴾\\
\textamh{15.\  } & يَـٰٓأَهْلَ ٱلْكِتَـٰبِ قَدْ جَآءَكُمْ رَسُولُنَا يُبَيِّنُ لَكُمْ كَثِيرًۭا مِّمَّا كُنتُمْ تُخْفُونَ مِنَ ٱلْكِتَـٰبِ وَيَعْفُوا۟ عَن كَثِيرٍۢ ۚ قَدْ جَآءَكُم مِّنَ ٱللَّهِ نُورٌۭ وَكِتَـٰبٌۭ مُّبِينٌۭ ﴿١٥﴾\\
\textamh{16.\  } & يَهْدِى بِهِ ٱللَّهُ مَنِ ٱتَّبَعَ رِضْوَٟنَهُۥ سُبُلَ ٱلسَّلَـٰمِ وَيُخْرِجُهُم مِّنَ ٱلظُّلُمَـٰتِ إِلَى ٱلنُّورِ بِإِذْنِهِۦ وَيَهْدِيهِمْ إِلَىٰ صِرَٰطٍۢ مُّسْتَقِيمٍۢ ﴿١٦﴾\\
\textamh{17.\  } & لَّقَدْ كَفَرَ ٱلَّذِينَ قَالُوٓا۟ إِنَّ ٱللَّهَ هُوَ ٱلْمَسِيحُ ٱبْنُ مَرْيَمَ ۚ قُلْ فَمَن يَمْلِكُ مِنَ ٱللَّهِ شَيْـًٔا إِنْ أَرَادَ أَن يُهْلِكَ ٱلْمَسِيحَ ٱبْنَ مَرْيَمَ وَأُمَّهُۥ وَمَن فِى ٱلْأَرْضِ جَمِيعًۭا ۗ وَلِلَّهِ مُلْكُ ٱلسَّمَـٰوَٟتِ وَٱلْأَرْضِ وَمَا بَيْنَهُمَا ۚ يَخْلُقُ مَا يَشَآءُ ۚ وَٱللَّهُ عَلَىٰ كُلِّ شَىْءٍۢ قَدِيرٌۭ ﴿١٧﴾\\
\textamh{18.\  } & وَقَالَتِ ٱلْيَهُودُ وَٱلنَّصَـٰرَىٰ نَحْنُ أَبْنَـٰٓؤُا۟ ٱللَّهِ وَأَحِبَّـٰٓؤُهُۥ ۚ قُلْ فَلِمَ يُعَذِّبُكُم بِذُنُوبِكُم ۖ بَلْ أَنتُم بَشَرٌۭ مِّمَّنْ خَلَقَ ۚ يَغْفِرُ لِمَن يَشَآءُ وَيُعَذِّبُ مَن يَشَآءُ ۚ وَلِلَّهِ مُلْكُ ٱلسَّمَـٰوَٟتِ وَٱلْأَرْضِ وَمَا بَيْنَهُمَا ۖ وَإِلَيْهِ ٱلْمَصِيرُ ﴿١٨﴾\\
\textamh{19.\  } & يَـٰٓأَهْلَ ٱلْكِتَـٰبِ قَدْ جَآءَكُمْ رَسُولُنَا يُبَيِّنُ لَكُمْ عَلَىٰ فَتْرَةٍۢ مِّنَ ٱلرُّسُلِ أَن تَقُولُوا۟ مَا جَآءَنَا مِنۢ بَشِيرٍۢ وَلَا نَذِيرٍۢ ۖ فَقَدْ جَآءَكُم بَشِيرٌۭ وَنَذِيرٌۭ ۗ وَٱللَّهُ عَلَىٰ كُلِّ شَىْءٍۢ قَدِيرٌۭ ﴿١٩﴾\\
\textamh{20.\  } & وَإِذْ قَالَ مُوسَىٰ لِقَوْمِهِۦ يَـٰقَوْمِ ٱذْكُرُوا۟ نِعْمَةَ ٱللَّهِ عَلَيْكُمْ إِذْ جَعَلَ فِيكُمْ أَنۢبِيَآءَ وَجَعَلَكُم مُّلُوكًۭا وَءَاتَىٰكُم مَّا لَمْ يُؤْتِ أَحَدًۭا مِّنَ ٱلْعَـٰلَمِينَ ﴿٢٠﴾\\
\textamh{21.\  } & يَـٰقَوْمِ ٱدْخُلُوا۟ ٱلْأَرْضَ ٱلْمُقَدَّسَةَ ٱلَّتِى كَتَبَ ٱللَّهُ لَكُمْ وَلَا تَرْتَدُّوا۟ عَلَىٰٓ أَدْبَارِكُمْ فَتَنقَلِبُوا۟ خَـٰسِرِينَ ﴿٢١﴾\\
\textamh{22.\  } & قَالُوا۟ يَـٰمُوسَىٰٓ إِنَّ فِيهَا قَوْمًۭا جَبَّارِينَ وَإِنَّا لَن نَّدْخُلَهَا حَتَّىٰ يَخْرُجُوا۟ مِنْهَا فَإِن يَخْرُجُوا۟ مِنْهَا فَإِنَّا دَٟخِلُونَ ﴿٢٢﴾\\
\textamh{23.\  } & قَالَ رَجُلَانِ مِنَ ٱلَّذِينَ يَخَافُونَ أَنْعَمَ ٱللَّهُ عَلَيْهِمَا ٱدْخُلُوا۟ عَلَيْهِمُ ٱلْبَابَ فَإِذَا دَخَلْتُمُوهُ فَإِنَّكُمْ غَٰلِبُونَ ۚ وَعَلَى ٱللَّهِ فَتَوَكَّلُوٓا۟ إِن كُنتُم مُّؤْمِنِينَ ﴿٢٣﴾\\
\textamh{24.\  } & قَالُوا۟ يَـٰمُوسَىٰٓ إِنَّا لَن نَّدْخُلَهَآ أَبَدًۭا مَّا دَامُوا۟ فِيهَا ۖ فَٱذْهَبْ أَنتَ وَرَبُّكَ فَقَـٰتِلَآ إِنَّا هَـٰهُنَا قَـٰعِدُونَ ﴿٢٤﴾\\
\textamh{25.\  } & قَالَ رَبِّ إِنِّى لَآ أَمْلِكُ إِلَّا نَفْسِى وَأَخِى ۖ فَٱفْرُقْ بَيْنَنَا وَبَيْنَ ٱلْقَوْمِ ٱلْفَـٰسِقِينَ ﴿٢٥﴾\\
\textamh{26.\  } & قَالَ فَإِنَّهَا مُحَرَّمَةٌ عَلَيْهِمْ ۛ أَرْبَعِينَ سَنَةًۭ ۛ يَتِيهُونَ فِى ٱلْأَرْضِ ۚ فَلَا تَأْسَ عَلَى ٱلْقَوْمِ ٱلْفَـٰسِقِينَ ﴿٢٦﴾\\
\textamh{27.\  } & ۞ وَٱتْلُ عَلَيْهِمْ نَبَأَ ٱبْنَىْ ءَادَمَ بِٱلْحَقِّ إِذْ قَرَّبَا قُرْبَانًۭا فَتُقُبِّلَ مِنْ أَحَدِهِمَا وَلَمْ يُتَقَبَّلْ مِنَ ٱلْءَاخَرِ قَالَ لَأَقْتُلَنَّكَ ۖ قَالَ إِنَّمَا يَتَقَبَّلُ ٱللَّهُ مِنَ ٱلْمُتَّقِينَ ﴿٢٧﴾\\
\textamh{28.\  } & لَئِنۢ بَسَطتَ إِلَىَّ يَدَكَ لِتَقْتُلَنِى مَآ أَنَا۠ بِبَاسِطٍۢ يَدِىَ إِلَيْكَ لِأَقْتُلَكَ ۖ إِنِّىٓ أَخَافُ ٱللَّهَ رَبَّ ٱلْعَـٰلَمِينَ ﴿٢٨﴾\\
\textamh{29.\  } & إِنِّىٓ أُرِيدُ أَن تَبُوٓأَ بِإِثْمِى وَإِثْمِكَ فَتَكُونَ مِنْ أَصْحَـٰبِ ٱلنَّارِ ۚ وَذَٟلِكَ جَزَٰٓؤُا۟ ٱلظَّـٰلِمِينَ ﴿٢٩﴾\\
\textamh{30.\  } & فَطَوَّعَتْ لَهُۥ نَفْسُهُۥ قَتْلَ أَخِيهِ فَقَتَلَهُۥ فَأَصْبَحَ مِنَ ٱلْخَـٰسِرِينَ ﴿٣٠﴾\\
\textamh{31.\  } & فَبَعَثَ ٱللَّهُ غُرَابًۭا يَبْحَثُ فِى ٱلْأَرْضِ لِيُرِيَهُۥ كَيْفَ يُوَٟرِى سَوْءَةَ أَخِيهِ ۚ قَالَ يَـٰوَيْلَتَىٰٓ أَعَجَزْتُ أَنْ أَكُونَ مِثْلَ هَـٰذَا ٱلْغُرَابِ فَأُوَٟرِىَ سَوْءَةَ أَخِى ۖ فَأَصْبَحَ مِنَ ٱلنَّـٰدِمِينَ ﴿٣١﴾\\
\textamh{32.\  } & مِنْ أَجْلِ ذَٟلِكَ كَتَبْنَا عَلَىٰ بَنِىٓ إِسْرَٰٓءِيلَ أَنَّهُۥ مَن قَتَلَ نَفْسًۢا بِغَيْرِ نَفْسٍ أَوْ فَسَادٍۢ فِى ٱلْأَرْضِ فَكَأَنَّمَا قَتَلَ ٱلنَّاسَ جَمِيعًۭا وَمَنْ أَحْيَاهَا فَكَأَنَّمَآ أَحْيَا ٱلنَّاسَ جَمِيعًۭا ۚ وَلَقَدْ جَآءَتْهُمْ رُسُلُنَا بِٱلْبَيِّنَـٰتِ ثُمَّ إِنَّ كَثِيرًۭا مِّنْهُم بَعْدَ ذَٟلِكَ فِى ٱلْأَرْضِ لَمُسْرِفُونَ ﴿٣٢﴾\\
\textamh{33.\  } & إِنَّمَا جَزَٰٓؤُا۟ ٱلَّذِينَ يُحَارِبُونَ ٱللَّهَ وَرَسُولَهُۥ وَيَسْعَوْنَ فِى ٱلْأَرْضِ فَسَادًا أَن يُقَتَّلُوٓا۟ أَوْ يُصَلَّبُوٓا۟ أَوْ تُقَطَّعَ أَيْدِيهِمْ وَأَرْجُلُهُم مِّنْ خِلَـٰفٍ أَوْ يُنفَوْا۟ مِنَ ٱلْأَرْضِ ۚ ذَٟلِكَ لَهُمْ خِزْىٌۭ فِى ٱلدُّنْيَا ۖ وَلَهُمْ فِى ٱلْءَاخِرَةِ عَذَابٌ عَظِيمٌ ﴿٣٣﴾\\
\textamh{34.\  } & إِلَّا ٱلَّذِينَ تَابُوا۟ مِن قَبْلِ أَن تَقْدِرُوا۟ عَلَيْهِمْ ۖ فَٱعْلَمُوٓا۟ أَنَّ ٱللَّهَ غَفُورٌۭ رَّحِيمٌۭ ﴿٣٤﴾\\
\textamh{35.\  } & يَـٰٓأَيُّهَا ٱلَّذِينَ ءَامَنُوا۟ ٱتَّقُوا۟ ٱللَّهَ وَٱبْتَغُوٓا۟ إِلَيْهِ ٱلْوَسِيلَةَ وَجَٰهِدُوا۟ فِى سَبِيلِهِۦ لَعَلَّكُمْ تُفْلِحُونَ ﴿٣٥﴾\\
\textamh{36.\  } & إِنَّ ٱلَّذِينَ كَفَرُوا۟ لَوْ أَنَّ لَهُم مَّا فِى ٱلْأَرْضِ جَمِيعًۭا وَمِثْلَهُۥ مَعَهُۥ لِيَفْتَدُوا۟ بِهِۦ مِنْ عَذَابِ يَوْمِ ٱلْقِيَـٰمَةِ مَا تُقُبِّلَ مِنْهُمْ ۖ وَلَهُمْ عَذَابٌ أَلِيمٌۭ ﴿٣٦﴾\\
\textamh{37.\  } & يُرِيدُونَ أَن يَخْرُجُوا۟ مِنَ ٱلنَّارِ وَمَا هُم بِخَـٰرِجِينَ مِنْهَا ۖ وَلَهُمْ عَذَابٌۭ مُّقِيمٌۭ ﴿٣٧﴾\\
\textamh{38.\  } & وَٱلسَّارِقُ وَٱلسَّارِقَةُ فَٱقْطَعُوٓا۟ أَيْدِيَهُمَا جَزَآءًۢ بِمَا كَسَبَا نَكَـٰلًۭا مِّنَ ٱللَّهِ ۗ وَٱللَّهُ عَزِيزٌ حَكِيمٌۭ ﴿٣٨﴾\\
\textamh{39.\  } & فَمَن تَابَ مِنۢ بَعْدِ ظُلْمِهِۦ وَأَصْلَحَ فَإِنَّ ٱللَّهَ يَتُوبُ عَلَيْهِ ۗ إِنَّ ٱللَّهَ غَفُورٌۭ رَّحِيمٌ ﴿٣٩﴾\\
\textamh{40.\  } & أَلَمْ تَعْلَمْ أَنَّ ٱللَّهَ لَهُۥ مُلْكُ ٱلسَّمَـٰوَٟتِ وَٱلْأَرْضِ يُعَذِّبُ مَن يَشَآءُ وَيَغْفِرُ لِمَن يَشَآءُ ۗ وَٱللَّهُ عَلَىٰ كُلِّ شَىْءٍۢ قَدِيرٌۭ ﴿٤٠﴾\\
\textamh{41.\  } & ۞ يَـٰٓأَيُّهَا ٱلرَّسُولُ لَا يَحْزُنكَ ٱلَّذِينَ يُسَـٰرِعُونَ فِى ٱلْكُفْرِ مِنَ ٱلَّذِينَ قَالُوٓا۟ ءَامَنَّا بِأَفْوَٟهِهِمْ وَلَمْ تُؤْمِن قُلُوبُهُمْ ۛ وَمِنَ ٱلَّذِينَ هَادُوا۟ ۛ سَمَّٰعُونَ لِلْكَذِبِ سَمَّٰعُونَ لِقَوْمٍ ءَاخَرِينَ لَمْ يَأْتُوكَ ۖ يُحَرِّفُونَ ٱلْكَلِمَ مِنۢ بَعْدِ مَوَاضِعِهِۦ ۖ يَقُولُونَ إِنْ أُوتِيتُمْ هَـٰذَا فَخُذُوهُ وَإِن لَّمْ تُؤْتَوْهُ فَٱحْذَرُوا۟ ۚ وَمَن يُرِدِ ٱللَّهُ فِتْنَتَهُۥ فَلَن تَمْلِكَ لَهُۥ مِنَ ٱللَّهِ شَيْـًٔا ۚ أُو۟لَـٰٓئِكَ ٱلَّذِينَ لَمْ يُرِدِ ٱللَّهُ أَن يُطَهِّرَ قُلُوبَهُمْ ۚ لَهُمْ فِى ٱلدُّنْيَا خِزْىٌۭ ۖ وَلَهُمْ فِى ٱلْءَاخِرَةِ عَذَابٌ عَظِيمٌۭ ﴿٤١﴾\\
\textamh{42.\  } & سَمَّٰعُونَ لِلْكَذِبِ أَكَّٰلُونَ لِلسُّحْتِ ۚ فَإِن جَآءُوكَ فَٱحْكُم بَيْنَهُمْ أَوْ أَعْرِضْ عَنْهُمْ ۖ وَإِن تُعْرِضْ عَنْهُمْ فَلَن يَضُرُّوكَ شَيْـًۭٔا ۖ وَإِنْ حَكَمْتَ فَٱحْكُم بَيْنَهُم بِٱلْقِسْطِ ۚ إِنَّ ٱللَّهَ يُحِبُّ ٱلْمُقْسِطِينَ ﴿٤٢﴾\\
\textamh{43.\  } & وَكَيْفَ يُحَكِّمُونَكَ وَعِندَهُمُ ٱلتَّوْرَىٰةُ فِيهَا حُكْمُ ٱللَّهِ ثُمَّ يَتَوَلَّوْنَ مِنۢ بَعْدِ ذَٟلِكَ ۚ وَمَآ أُو۟لَـٰٓئِكَ بِٱلْمُؤْمِنِينَ ﴿٤٣﴾\\
\textamh{44.\  } & إِنَّآ أَنزَلْنَا ٱلتَّوْرَىٰةَ فِيهَا هُدًۭى وَنُورٌۭ ۚ يَحْكُمُ بِهَا ٱلنَّبِيُّونَ ٱلَّذِينَ أَسْلَمُوا۟ لِلَّذِينَ هَادُوا۟ وَٱلرَّبَّـٰنِيُّونَ وَٱلْأَحْبَارُ بِمَا ٱسْتُحْفِظُوا۟ مِن كِتَـٰبِ ٱللَّهِ وَكَانُوا۟ عَلَيْهِ شُهَدَآءَ ۚ فَلَا تَخْشَوُا۟ ٱلنَّاسَ وَٱخْشَوْنِ وَلَا تَشْتَرُوا۟ بِـَٔايَـٰتِى ثَمَنًۭا قَلِيلًۭا ۚ وَمَن لَّمْ يَحْكُم بِمَآ أَنزَلَ ٱللَّهُ فَأُو۟لَـٰٓئِكَ هُمُ ٱلْكَـٰفِرُونَ ﴿٤٤﴾\\
\textamh{45.\  } & وَكَتَبْنَا عَلَيْهِمْ فِيهَآ أَنَّ ٱلنَّفْسَ بِٱلنَّفْسِ وَٱلْعَيْنَ بِٱلْعَيْنِ وَٱلْأَنفَ بِٱلْأَنفِ وَٱلْأُذُنَ بِٱلْأُذُنِ وَٱلسِّنَّ بِٱلسِّنِّ وَٱلْجُرُوحَ قِصَاصٌۭ ۚ فَمَن تَصَدَّقَ بِهِۦ فَهُوَ كَفَّارَةٌۭ لَّهُۥ ۚ وَمَن لَّمْ يَحْكُم بِمَآ أَنزَلَ ٱللَّهُ فَأُو۟لَـٰٓئِكَ هُمُ ٱلظَّـٰلِمُونَ ﴿٤٥﴾\\
\textamh{46.\  } & وَقَفَّيْنَا عَلَىٰٓ ءَاثَـٰرِهِم بِعِيسَى ٱبْنِ مَرْيَمَ مُصَدِّقًۭا لِّمَا بَيْنَ يَدَيْهِ مِنَ ٱلتَّوْرَىٰةِ ۖ وَءَاتَيْنَـٰهُ ٱلْإِنجِيلَ فِيهِ هُدًۭى وَنُورٌۭ وَمُصَدِّقًۭا لِّمَا بَيْنَ يَدَيْهِ مِنَ ٱلتَّوْرَىٰةِ وَهُدًۭى وَمَوْعِظَةًۭ لِّلْمُتَّقِينَ ﴿٤٦﴾\\
\textamh{47.\  } & وَلْيَحْكُمْ أَهْلُ ٱلْإِنجِيلِ بِمَآ أَنزَلَ ٱللَّهُ فِيهِ ۚ وَمَن لَّمْ يَحْكُم بِمَآ أَنزَلَ ٱللَّهُ فَأُو۟لَـٰٓئِكَ هُمُ ٱلْفَـٰسِقُونَ ﴿٤٧﴾\\
\textamh{48.\  } & وَأَنزَلْنَآ إِلَيْكَ ٱلْكِتَـٰبَ بِٱلْحَقِّ مُصَدِّقًۭا لِّمَا بَيْنَ يَدَيْهِ مِنَ ٱلْكِتَـٰبِ وَمُهَيْمِنًا عَلَيْهِ ۖ فَٱحْكُم بَيْنَهُم بِمَآ أَنزَلَ ٱللَّهُ ۖ وَلَا تَتَّبِعْ أَهْوَآءَهُمْ عَمَّا جَآءَكَ مِنَ ٱلْحَقِّ ۚ لِكُلٍّۢ جَعَلْنَا مِنكُمْ شِرْعَةًۭ وَمِنْهَاجًۭا ۚ وَلَوْ شَآءَ ٱللَّهُ لَجَعَلَكُمْ أُمَّةًۭ وَٟحِدَةًۭ وَلَـٰكِن لِّيَبْلُوَكُمْ فِى مَآ ءَاتَىٰكُمْ ۖ فَٱسْتَبِقُوا۟ ٱلْخَيْرَٰتِ ۚ إِلَى ٱللَّهِ مَرْجِعُكُمْ جَمِيعًۭا فَيُنَبِّئُكُم بِمَا كُنتُمْ فِيهِ تَخْتَلِفُونَ ﴿٤٨﴾\\
\textamh{49.\  } & وَأَنِ ٱحْكُم بَيْنَهُم بِمَآ أَنزَلَ ٱللَّهُ وَلَا تَتَّبِعْ أَهْوَآءَهُمْ وَٱحْذَرْهُمْ أَن يَفْتِنُوكَ عَنۢ بَعْضِ مَآ أَنزَلَ ٱللَّهُ إِلَيْكَ ۖ فَإِن تَوَلَّوْا۟ فَٱعْلَمْ أَنَّمَا يُرِيدُ ٱللَّهُ أَن يُصِيبَهُم بِبَعْضِ ذُنُوبِهِمْ ۗ وَإِنَّ كَثِيرًۭا مِّنَ ٱلنَّاسِ لَفَـٰسِقُونَ ﴿٤٩﴾\\
\textamh{50.\  } & أَفَحُكْمَ ٱلْجَٰهِلِيَّةِ يَبْغُونَ ۚ وَمَنْ أَحْسَنُ مِنَ ٱللَّهِ حُكْمًۭا لِّقَوْمٍۢ يُوقِنُونَ ﴿٥٠﴾\\
\textamh{51.\  } & ۞ يَـٰٓأَيُّهَا ٱلَّذِينَ ءَامَنُوا۟ لَا تَتَّخِذُوا۟ ٱلْيَهُودَ وَٱلنَّصَـٰرَىٰٓ أَوْلِيَآءَ ۘ بَعْضُهُمْ أَوْلِيَآءُ بَعْضٍۢ ۚ وَمَن يَتَوَلَّهُم مِّنكُمْ فَإِنَّهُۥ مِنْهُمْ ۗ إِنَّ ٱللَّهَ لَا يَهْدِى ٱلْقَوْمَ ٱلظَّـٰلِمِينَ ﴿٥١﴾\\
\textamh{52.\  } & فَتَرَى ٱلَّذِينَ فِى قُلُوبِهِم مَّرَضٌۭ يُسَـٰرِعُونَ فِيهِمْ يَقُولُونَ نَخْشَىٰٓ أَن تُصِيبَنَا دَآئِرَةٌۭ ۚ فَعَسَى ٱللَّهُ أَن يَأْتِىَ بِٱلْفَتْحِ أَوْ أَمْرٍۢ مِّنْ عِندِهِۦ فَيُصْبِحُوا۟ عَلَىٰ مَآ أَسَرُّوا۟ فِىٓ أَنفُسِهِمْ نَـٰدِمِينَ ﴿٥٢﴾\\
\textamh{53.\  } & وَيَقُولُ ٱلَّذِينَ ءَامَنُوٓا۟ أَهَـٰٓؤُلَآءِ ٱلَّذِينَ أَقْسَمُوا۟ بِٱللَّهِ جَهْدَ أَيْمَـٰنِهِمْ ۙ إِنَّهُمْ لَمَعَكُمْ ۚ حَبِطَتْ أَعْمَـٰلُهُمْ فَأَصْبَحُوا۟ خَـٰسِرِينَ ﴿٥٣﴾\\
\textamh{54.\  } & يَـٰٓأَيُّهَا ٱلَّذِينَ ءَامَنُوا۟ مَن يَرْتَدَّ مِنكُمْ عَن دِينِهِۦ فَسَوْفَ يَأْتِى ٱللَّهُ بِقَوْمٍۢ يُحِبُّهُمْ وَيُحِبُّونَهُۥٓ أَذِلَّةٍ عَلَى ٱلْمُؤْمِنِينَ أَعِزَّةٍ عَلَى ٱلْكَـٰفِرِينَ يُجَٰهِدُونَ فِى سَبِيلِ ٱللَّهِ وَلَا يَخَافُونَ لَوْمَةَ لَآئِمٍۢ ۚ ذَٟلِكَ فَضْلُ ٱللَّهِ يُؤْتِيهِ مَن يَشَآءُ ۚ وَٱللَّهُ وَٟسِعٌ عَلِيمٌ ﴿٥٤﴾\\
\textamh{55.\  } & إِنَّمَا وَلِيُّكُمُ ٱللَّهُ وَرَسُولُهُۥ وَٱلَّذِينَ ءَامَنُوا۟ ٱلَّذِينَ يُقِيمُونَ ٱلصَّلَوٰةَ وَيُؤْتُونَ ٱلزَّكَوٰةَ وَهُمْ رَٰكِعُونَ ﴿٥٥﴾\\
\textamh{56.\  } & وَمَن يَتَوَلَّ ٱللَّهَ وَرَسُولَهُۥ وَٱلَّذِينَ ءَامَنُوا۟ فَإِنَّ حِزْبَ ٱللَّهِ هُمُ ٱلْغَٰلِبُونَ ﴿٥٦﴾\\
\textamh{57.\  } & يَـٰٓأَيُّهَا ٱلَّذِينَ ءَامَنُوا۟ لَا تَتَّخِذُوا۟ ٱلَّذِينَ ٱتَّخَذُوا۟ دِينَكُمْ هُزُوًۭا وَلَعِبًۭا مِّنَ ٱلَّذِينَ أُوتُوا۟ ٱلْكِتَـٰبَ مِن قَبْلِكُمْ وَٱلْكُفَّارَ أَوْلِيَآءَ ۚ وَٱتَّقُوا۟ ٱللَّهَ إِن كُنتُم مُّؤْمِنِينَ ﴿٥٧﴾\\
\textamh{58.\  } & وَإِذَا نَادَيْتُمْ إِلَى ٱلصَّلَوٰةِ ٱتَّخَذُوهَا هُزُوًۭا وَلَعِبًۭا ۚ ذَٟلِكَ بِأَنَّهُمْ قَوْمٌۭ لَّا يَعْقِلُونَ ﴿٥٨﴾\\
\textamh{59.\  } & قُلْ يَـٰٓأَهْلَ ٱلْكِتَـٰبِ هَلْ تَنقِمُونَ مِنَّآ إِلَّآ أَنْ ءَامَنَّا بِٱللَّهِ وَمَآ أُنزِلَ إِلَيْنَا وَمَآ أُنزِلَ مِن قَبْلُ وَأَنَّ أَكْثَرَكُمْ فَـٰسِقُونَ ﴿٥٩﴾\\
\textamh{60.\  } & قُلْ هَلْ أُنَبِّئُكُم بِشَرٍّۢ مِّن ذَٟلِكَ مَثُوبَةً عِندَ ٱللَّهِ ۚ مَن لَّعَنَهُ ٱللَّهُ وَغَضِبَ عَلَيْهِ وَجَعَلَ مِنْهُمُ ٱلْقِرَدَةَ وَٱلْخَنَازِيرَ وَعَبَدَ ٱلطَّٰغُوتَ ۚ أُو۟لَـٰٓئِكَ شَرٌّۭ مَّكَانًۭا وَأَضَلُّ عَن سَوَآءِ ٱلسَّبِيلِ ﴿٦٠﴾\\
\textamh{61.\  } & وَإِذَا جَآءُوكُمْ قَالُوٓا۟ ءَامَنَّا وَقَد دَّخَلُوا۟ بِٱلْكُفْرِ وَهُمْ قَدْ خَرَجُوا۟ بِهِۦ ۚ وَٱللَّهُ أَعْلَمُ بِمَا كَانُوا۟ يَكْتُمُونَ ﴿٦١﴾\\
\textamh{62.\  } & وَتَرَىٰ كَثِيرًۭا مِّنْهُمْ يُسَـٰرِعُونَ فِى ٱلْإِثْمِ وَٱلْعُدْوَٟنِ وَأَكْلِهِمُ ٱلسُّحْتَ ۚ لَبِئْسَ مَا كَانُوا۟ يَعْمَلُونَ ﴿٦٢﴾\\
\textamh{63.\  } & لَوْلَا يَنْهَىٰهُمُ ٱلرَّبَّـٰنِيُّونَ وَٱلْأَحْبَارُ عَن قَوْلِهِمُ ٱلْإِثْمَ وَأَكْلِهِمُ ٱلسُّحْتَ ۚ لَبِئْسَ مَا كَانُوا۟ يَصْنَعُونَ ﴿٦٣﴾\\
\textamh{64.\  } & وَقَالَتِ ٱلْيَهُودُ يَدُ ٱللَّهِ مَغْلُولَةٌ ۚ غُلَّتْ أَيْدِيهِمْ وَلُعِنُوا۟ بِمَا قَالُوا۟ ۘ بَلْ يَدَاهُ مَبْسُوطَتَانِ يُنفِقُ كَيْفَ يَشَآءُ ۚ وَلَيَزِيدَنَّ كَثِيرًۭا مِّنْهُم مَّآ أُنزِلَ إِلَيْكَ مِن رَّبِّكَ طُغْيَـٰنًۭا وَكُفْرًۭا ۚ وَأَلْقَيْنَا بَيْنَهُمُ ٱلْعَدَٟوَةَ وَٱلْبَغْضَآءَ إِلَىٰ يَوْمِ ٱلْقِيَـٰمَةِ ۚ كُلَّمَآ أَوْقَدُوا۟ نَارًۭا لِّلْحَرْبِ أَطْفَأَهَا ٱللَّهُ ۚ وَيَسْعَوْنَ فِى ٱلْأَرْضِ فَسَادًۭا ۚ وَٱللَّهُ لَا يُحِبُّ ٱلْمُفْسِدِينَ ﴿٦٤﴾\\
\textamh{65.\  } & وَلَوْ أَنَّ أَهْلَ ٱلْكِتَـٰبِ ءَامَنُوا۟ وَٱتَّقَوْا۟ لَكَفَّرْنَا عَنْهُمْ سَيِّـَٔاتِهِمْ وَلَأَدْخَلْنَـٰهُمْ جَنَّـٰتِ ٱلنَّعِيمِ ﴿٦٥﴾\\
\textamh{66.\  } & وَلَوْ أَنَّهُمْ أَقَامُوا۟ ٱلتَّوْرَىٰةَ وَٱلْإِنجِيلَ وَمَآ أُنزِلَ إِلَيْهِم مِّن رَّبِّهِمْ لَأَكَلُوا۟ مِن فَوْقِهِمْ وَمِن تَحْتِ أَرْجُلِهِم ۚ مِّنْهُمْ أُمَّةٌۭ مُّقْتَصِدَةٌۭ ۖ وَكَثِيرٌۭ مِّنْهُمْ سَآءَ مَا يَعْمَلُونَ ﴿٦٦﴾\\
\textamh{67.\  } & ۞ يَـٰٓأَيُّهَا ٱلرَّسُولُ بَلِّغْ مَآ أُنزِلَ إِلَيْكَ مِن رَّبِّكَ ۖ وَإِن لَّمْ تَفْعَلْ فَمَا بَلَّغْتَ رِسَالَتَهُۥ ۚ وَٱللَّهُ يَعْصِمُكَ مِنَ ٱلنَّاسِ ۗ إِنَّ ٱللَّهَ لَا يَهْدِى ٱلْقَوْمَ ٱلْكَـٰفِرِينَ ﴿٦٧﴾\\
\textamh{68.\  } & قُلْ يَـٰٓأَهْلَ ٱلْكِتَـٰبِ لَسْتُمْ عَلَىٰ شَىْءٍ حَتَّىٰ تُقِيمُوا۟ ٱلتَّوْرَىٰةَ وَٱلْإِنجِيلَ وَمَآ أُنزِلَ إِلَيْكُم مِّن رَّبِّكُمْ ۗ وَلَيَزِيدَنَّ كَثِيرًۭا مِّنْهُم مَّآ أُنزِلَ إِلَيْكَ مِن رَّبِّكَ طُغْيَـٰنًۭا وَكُفْرًۭا ۖ فَلَا تَأْسَ عَلَى ٱلْقَوْمِ ٱلْكَـٰفِرِينَ ﴿٦٨﴾\\
\textamh{69.\  } & إِنَّ ٱلَّذِينَ ءَامَنُوا۟ وَٱلَّذِينَ هَادُوا۟ وَٱلصَّـٰبِـُٔونَ وَٱلنَّصَـٰرَىٰ مَنْ ءَامَنَ بِٱللَّهِ وَٱلْيَوْمِ ٱلْءَاخِرِ وَعَمِلَ صَـٰلِحًۭا فَلَا خَوْفٌ عَلَيْهِمْ وَلَا هُمْ يَحْزَنُونَ ﴿٦٩﴾\\
\textamh{70.\  } & لَقَدْ أَخَذْنَا مِيثَـٰقَ بَنِىٓ إِسْرَٰٓءِيلَ وَأَرْسَلْنَآ إِلَيْهِمْ رُسُلًۭا ۖ كُلَّمَا جَآءَهُمْ رَسُولٌۢ بِمَا لَا تَهْوَىٰٓ أَنفُسُهُمْ فَرِيقًۭا كَذَّبُوا۟ وَفَرِيقًۭا يَقْتُلُونَ ﴿٧٠﴾\\
\textamh{71.\  } & وَحَسِبُوٓا۟ أَلَّا تَكُونَ فِتْنَةٌۭ فَعَمُوا۟ وَصَمُّوا۟ ثُمَّ تَابَ ٱللَّهُ عَلَيْهِمْ ثُمَّ عَمُوا۟ وَصَمُّوا۟ كَثِيرٌۭ مِّنْهُمْ ۚ وَٱللَّهُ بَصِيرٌۢ بِمَا يَعْمَلُونَ ﴿٧١﴾\\
\textamh{72.\  } & لَقَدْ كَفَرَ ٱلَّذِينَ قَالُوٓا۟ إِنَّ ٱللَّهَ هُوَ ٱلْمَسِيحُ ٱبْنُ مَرْيَمَ ۖ وَقَالَ ٱلْمَسِيحُ يَـٰبَنِىٓ إِسْرَٰٓءِيلَ ٱعْبُدُوا۟ ٱللَّهَ رَبِّى وَرَبَّكُمْ ۖ إِنَّهُۥ مَن يُشْرِكْ بِٱللَّهِ فَقَدْ حَرَّمَ ٱللَّهُ عَلَيْهِ ٱلْجَنَّةَ وَمَأْوَىٰهُ ٱلنَّارُ ۖ وَمَا لِلظَّـٰلِمِينَ مِنْ أَنصَارٍۢ ﴿٧٢﴾\\
\textamh{73.\  } & لَّقَدْ كَفَرَ ٱلَّذِينَ قَالُوٓا۟ إِنَّ ٱللَّهَ ثَالِثُ ثَلَـٰثَةٍۢ ۘ وَمَا مِنْ إِلَـٰهٍ إِلَّآ إِلَـٰهٌۭ وَٟحِدٌۭ ۚ وَإِن لَّمْ يَنتَهُوا۟ عَمَّا يَقُولُونَ لَيَمَسَّنَّ ٱلَّذِينَ كَفَرُوا۟ مِنْهُمْ عَذَابٌ أَلِيمٌ ﴿٧٣﴾\\
\textamh{74.\  } & أَفَلَا يَتُوبُونَ إِلَى ٱللَّهِ وَيَسْتَغْفِرُونَهُۥ ۚ وَٱللَّهُ غَفُورٌۭ رَّحِيمٌۭ ﴿٧٤﴾\\
\textamh{75.\  } & مَّا ٱلْمَسِيحُ ٱبْنُ مَرْيَمَ إِلَّا رَسُولٌۭ قَدْ خَلَتْ مِن قَبْلِهِ ٱلرُّسُلُ وَأُمُّهُۥ صِدِّيقَةٌۭ ۖ كَانَا يَأْكُلَانِ ٱلطَّعَامَ ۗ ٱنظُرْ كَيْفَ نُبَيِّنُ لَهُمُ ٱلْءَايَـٰتِ ثُمَّ ٱنظُرْ أَنَّىٰ يُؤْفَكُونَ ﴿٧٥﴾\\
\textamh{76.\  } & قُلْ أَتَعْبُدُونَ مِن دُونِ ٱللَّهِ مَا لَا يَمْلِكُ لَكُمْ ضَرًّۭا وَلَا نَفْعًۭا ۚ وَٱللَّهُ هُوَ ٱلسَّمِيعُ ٱلْعَلِيمُ ﴿٧٦﴾\\
\textamh{77.\  } & قُلْ يَـٰٓأَهْلَ ٱلْكِتَـٰبِ لَا تَغْلُوا۟ فِى دِينِكُمْ غَيْرَ ٱلْحَقِّ وَلَا تَتَّبِعُوٓا۟ أَهْوَآءَ قَوْمٍۢ قَدْ ضَلُّوا۟ مِن قَبْلُ وَأَضَلُّوا۟ كَثِيرًۭا وَضَلُّوا۟ عَن سَوَآءِ ٱلسَّبِيلِ ﴿٧٧﴾\\
\textamh{78.\  } & لُعِنَ ٱلَّذِينَ كَفَرُوا۟ مِنۢ بَنِىٓ إِسْرَٰٓءِيلَ عَلَىٰ لِسَانِ دَاوُۥدَ وَعِيسَى ٱبْنِ مَرْيَمَ ۚ ذَٟلِكَ بِمَا عَصَوا۟ وَّكَانُوا۟ يَعْتَدُونَ ﴿٧٨﴾\\
\textamh{79.\  } & كَانُوا۟ لَا يَتَنَاهَوْنَ عَن مُّنكَرٍۢ فَعَلُوهُ ۚ لَبِئْسَ مَا كَانُوا۟ يَفْعَلُونَ ﴿٧٩﴾\\
\textamh{80.\  } & تَرَىٰ كَثِيرًۭا مِّنْهُمْ يَتَوَلَّوْنَ ٱلَّذِينَ كَفَرُوا۟ ۚ لَبِئْسَ مَا قَدَّمَتْ لَهُمْ أَنفُسُهُمْ أَن سَخِطَ ٱللَّهُ عَلَيْهِمْ وَفِى ٱلْعَذَابِ هُمْ خَـٰلِدُونَ ﴿٨٠﴾\\
\textamh{81.\  } & وَلَوْ كَانُوا۟ يُؤْمِنُونَ بِٱللَّهِ وَٱلنَّبِىِّ وَمَآ أُنزِلَ إِلَيْهِ مَا ٱتَّخَذُوهُمْ أَوْلِيَآءَ وَلَـٰكِنَّ كَثِيرًۭا مِّنْهُمْ فَـٰسِقُونَ ﴿٨١﴾\\
\textamh{82.\  } & ۞ لَتَجِدَنَّ أَشَدَّ ٱلنَّاسِ عَدَٟوَةًۭ لِّلَّذِينَ ءَامَنُوا۟ ٱلْيَهُودَ وَٱلَّذِينَ أَشْرَكُوا۟ ۖ وَلَتَجِدَنَّ أَقْرَبَهُم مَّوَدَّةًۭ لِّلَّذِينَ ءَامَنُوا۟ ٱلَّذِينَ قَالُوٓا۟ إِنَّا نَصَـٰرَىٰ ۚ ذَٟلِكَ بِأَنَّ مِنْهُمْ قِسِّيسِينَ وَرُهْبَانًۭا وَأَنَّهُمْ لَا يَسْتَكْبِرُونَ ﴿٨٢﴾\\
\textamh{83.\  } & وَإِذَا سَمِعُوا۟ مَآ أُنزِلَ إِلَى ٱلرَّسُولِ تَرَىٰٓ أَعْيُنَهُمْ تَفِيضُ مِنَ ٱلدَّمْعِ مِمَّا عَرَفُوا۟ مِنَ ٱلْحَقِّ ۖ يَقُولُونَ رَبَّنَآ ءَامَنَّا فَٱكْتُبْنَا مَعَ ٱلشَّـٰهِدِينَ ﴿٨٣﴾\\
\textamh{84.\  } & وَمَا لَنَا لَا نُؤْمِنُ بِٱللَّهِ وَمَا جَآءَنَا مِنَ ٱلْحَقِّ وَنَطْمَعُ أَن يُدْخِلَنَا رَبُّنَا مَعَ ٱلْقَوْمِ ٱلصَّـٰلِحِينَ ﴿٨٤﴾\\
\textamh{85.\  } & فَأَثَـٰبَهُمُ ٱللَّهُ بِمَا قَالُوا۟ جَنَّـٰتٍۢ تَجْرِى مِن تَحْتِهَا ٱلْأَنْهَـٰرُ خَـٰلِدِينَ فِيهَا ۚ وَذَٟلِكَ جَزَآءُ ٱلْمُحْسِنِينَ ﴿٨٥﴾\\
\textamh{86.\  } & وَٱلَّذِينَ كَفَرُوا۟ وَكَذَّبُوا۟ بِـَٔايَـٰتِنَآ أُو۟لَـٰٓئِكَ أَصْحَـٰبُ ٱلْجَحِيمِ ﴿٨٦﴾\\
\textamh{87.\  } & يَـٰٓأَيُّهَا ٱلَّذِينَ ءَامَنُوا۟ لَا تُحَرِّمُوا۟ طَيِّبَٰتِ مَآ أَحَلَّ ٱللَّهُ لَكُمْ وَلَا تَعْتَدُوٓا۟ ۚ إِنَّ ٱللَّهَ لَا يُحِبُّ ٱلْمُعْتَدِينَ ﴿٨٧﴾\\
\textamh{88.\  } & وَكُلُوا۟ مِمَّا رَزَقَكُمُ ٱللَّهُ حَلَـٰلًۭا طَيِّبًۭا ۚ وَٱتَّقُوا۟ ٱللَّهَ ٱلَّذِىٓ أَنتُم بِهِۦ مُؤْمِنُونَ ﴿٨٨﴾\\
\textamh{89.\  } & لَا يُؤَاخِذُكُمُ ٱللَّهُ بِٱللَّغْوِ فِىٓ أَيْمَـٰنِكُمْ وَلَـٰكِن يُؤَاخِذُكُم بِمَا عَقَّدتُّمُ ٱلْأَيْمَـٰنَ ۖ فَكَفَّٰرَتُهُۥٓ إِطْعَامُ عَشَرَةِ مَسَـٰكِينَ مِنْ أَوْسَطِ مَا تُطْعِمُونَ أَهْلِيكُمْ أَوْ كِسْوَتُهُمْ أَوْ تَحْرِيرُ رَقَبَةٍۢ ۖ فَمَن لَّمْ يَجِدْ فَصِيَامُ ثَلَـٰثَةِ أَيَّامٍۢ ۚ ذَٟلِكَ كَفَّٰرَةُ أَيْمَـٰنِكُمْ إِذَا حَلَفْتُمْ ۚ وَٱحْفَظُوٓا۟ أَيْمَـٰنَكُمْ ۚ كَذَٟلِكَ يُبَيِّنُ ٱللَّهُ لَكُمْ ءَايَـٰتِهِۦ لَعَلَّكُمْ تَشْكُرُونَ ﴿٨٩﴾\\
\textamh{90.\  } & يَـٰٓأَيُّهَا ٱلَّذِينَ ءَامَنُوٓا۟ إِنَّمَا ٱلْخَمْرُ وَٱلْمَيْسِرُ وَٱلْأَنصَابُ وَٱلْأَزْلَـٰمُ رِجْسٌۭ مِّنْ عَمَلِ ٱلشَّيْطَٰنِ فَٱجْتَنِبُوهُ لَعَلَّكُمْ تُفْلِحُونَ ﴿٩٠﴾\\
\textamh{91.\  } & إِنَّمَا يُرِيدُ ٱلشَّيْطَٰنُ أَن يُوقِعَ بَيْنَكُمُ ٱلْعَدَٟوَةَ وَٱلْبَغْضَآءَ فِى ٱلْخَمْرِ وَٱلْمَيْسِرِ وَيَصُدَّكُمْ عَن ذِكْرِ ٱللَّهِ وَعَنِ ٱلصَّلَوٰةِ ۖ فَهَلْ أَنتُم مُّنتَهُونَ ﴿٩١﴾\\
\textamh{92.\  } & وَأَطِيعُوا۟ ٱللَّهَ وَأَطِيعُوا۟ ٱلرَّسُولَ وَٱحْذَرُوا۟ ۚ فَإِن تَوَلَّيْتُمْ فَٱعْلَمُوٓا۟ أَنَّمَا عَلَىٰ رَسُولِنَا ٱلْبَلَـٰغُ ٱلْمُبِينُ ﴿٩٢﴾\\
\textamh{93.\  } & لَيْسَ عَلَى ٱلَّذِينَ ءَامَنُوا۟ وَعَمِلُوا۟ ٱلصَّـٰلِحَـٰتِ جُنَاحٌۭ فِيمَا طَعِمُوٓا۟ إِذَا مَا ٱتَّقَوا۟ وَّءَامَنُوا۟ وَعَمِلُوا۟ ٱلصَّـٰلِحَـٰتِ ثُمَّ ٱتَّقَوا۟ وَّءَامَنُوا۟ ثُمَّ ٱتَّقَوا۟ وَّأَحْسَنُوا۟ ۗ وَٱللَّهُ يُحِبُّ ٱلْمُحْسِنِينَ ﴿٩٣﴾\\
\textamh{94.\  } & يَـٰٓأَيُّهَا ٱلَّذِينَ ءَامَنُوا۟ لَيَبْلُوَنَّكُمُ ٱللَّهُ بِشَىْءٍۢ مِّنَ ٱلصَّيْدِ تَنَالُهُۥٓ أَيْدِيكُمْ وَرِمَاحُكُمْ لِيَعْلَمَ ٱللَّهُ مَن يَخَافُهُۥ بِٱلْغَيْبِ ۚ فَمَنِ ٱعْتَدَىٰ بَعْدَ ذَٟلِكَ فَلَهُۥ عَذَابٌ أَلِيمٌۭ ﴿٩٤﴾\\
\textamh{95.\  } & يَـٰٓأَيُّهَا ٱلَّذِينَ ءَامَنُوا۟ لَا تَقْتُلُوا۟ ٱلصَّيْدَ وَأَنتُمْ حُرُمٌۭ ۚ وَمَن قَتَلَهُۥ مِنكُم مُّتَعَمِّدًۭا فَجَزَآءٌۭ مِّثْلُ مَا قَتَلَ مِنَ ٱلنَّعَمِ يَحْكُمُ بِهِۦ ذَوَا عَدْلٍۢ مِّنكُمْ هَدْيًۢا بَٰلِغَ ٱلْكَعْبَةِ أَوْ كَفَّٰرَةٌۭ طَعَامُ مَسَـٰكِينَ أَوْ عَدْلُ ذَٟلِكَ صِيَامًۭا لِّيَذُوقَ وَبَالَ أَمْرِهِۦ ۗ عَفَا ٱللَّهُ عَمَّا سَلَفَ ۚ وَمَنْ عَادَ فَيَنتَقِمُ ٱللَّهُ مِنْهُ ۗ وَٱللَّهُ عَزِيزٌۭ ذُو ٱنتِقَامٍ ﴿٩٥﴾\\
\textamh{96.\  } & أُحِلَّ لَكُمْ صَيْدُ ٱلْبَحْرِ وَطَعَامُهُۥ مَتَـٰعًۭا لَّكُمْ وَلِلسَّيَّارَةِ ۖ وَحُرِّمَ عَلَيْكُمْ صَيْدُ ٱلْبَرِّ مَا دُمْتُمْ حُرُمًۭا ۗ وَٱتَّقُوا۟ ٱللَّهَ ٱلَّذِىٓ إِلَيْهِ تُحْشَرُونَ ﴿٩٦﴾\\
\textamh{97.\  } & ۞ جَعَلَ ٱللَّهُ ٱلْكَعْبَةَ ٱلْبَيْتَ ٱلْحَرَامَ قِيَـٰمًۭا لِّلنَّاسِ وَٱلشَّهْرَ ٱلْحَرَامَ وَٱلْهَدْىَ وَٱلْقَلَـٰٓئِدَ ۚ ذَٟلِكَ لِتَعْلَمُوٓا۟ أَنَّ ٱللَّهَ يَعْلَمُ مَا فِى ٱلسَّمَـٰوَٟتِ وَمَا فِى ٱلْأَرْضِ وَأَنَّ ٱللَّهَ بِكُلِّ شَىْءٍ عَلِيمٌ ﴿٩٧﴾\\
\textamh{98.\  } & ٱعْلَمُوٓا۟ أَنَّ ٱللَّهَ شَدِيدُ ٱلْعِقَابِ وَأَنَّ ٱللَّهَ غَفُورٌۭ رَّحِيمٌۭ ﴿٩٨﴾\\
\textamh{99.\  } & مَّا عَلَى ٱلرَّسُولِ إِلَّا ٱلْبَلَـٰغُ ۗ وَٱللَّهُ يَعْلَمُ مَا تُبْدُونَ وَمَا تَكْتُمُونَ ﴿٩٩﴾\\
\textamh{100.\  } & قُل لَّا يَسْتَوِى ٱلْخَبِيثُ وَٱلطَّيِّبُ وَلَوْ أَعْجَبَكَ كَثْرَةُ ٱلْخَبِيثِ ۚ فَٱتَّقُوا۟ ٱللَّهَ يَـٰٓأُو۟لِى ٱلْأَلْبَٰبِ لَعَلَّكُمْ تُفْلِحُونَ ﴿١٠٠﴾\\
\textamh{101.\  } & يَـٰٓأَيُّهَا ٱلَّذِينَ ءَامَنُوا۟ لَا تَسْـَٔلُوا۟ عَنْ أَشْيَآءَ إِن تُبْدَ لَكُمْ تَسُؤْكُمْ وَإِن تَسْـَٔلُوا۟ عَنْهَا حِينَ يُنَزَّلُ ٱلْقُرْءَانُ تُبْدَ لَكُمْ عَفَا ٱللَّهُ عَنْهَا ۗ وَٱللَّهُ غَفُورٌ حَلِيمٌۭ ﴿١٠١﴾\\
\textamh{102.\  } & قَدْ سَأَلَهَا قَوْمٌۭ مِّن قَبْلِكُمْ ثُمَّ أَصْبَحُوا۟ بِهَا كَـٰفِرِينَ ﴿١٠٢﴾\\
\textamh{103.\  } & مَا جَعَلَ ٱللَّهُ مِنۢ بَحِيرَةٍۢ وَلَا سَآئِبَةٍۢ وَلَا وَصِيلَةٍۢ وَلَا حَامٍۢ ۙ وَلَـٰكِنَّ ٱلَّذِينَ كَفَرُوا۟ يَفْتَرُونَ عَلَى ٱللَّهِ ٱلْكَذِبَ ۖ وَأَكْثَرُهُمْ لَا يَعْقِلُونَ ﴿١٠٣﴾\\
\textamh{104.\  } & وَإِذَا قِيلَ لَهُمْ تَعَالَوْا۟ إِلَىٰ مَآ أَنزَلَ ٱللَّهُ وَإِلَى ٱلرَّسُولِ قَالُوا۟ حَسْبُنَا مَا وَجَدْنَا عَلَيْهِ ءَابَآءَنَآ ۚ أَوَلَوْ كَانَ ءَابَآؤُهُمْ لَا يَعْلَمُونَ شَيْـًۭٔا وَلَا يَهْتَدُونَ ﴿١٠٤﴾\\
\textamh{105.\  } & يَـٰٓأَيُّهَا ٱلَّذِينَ ءَامَنُوا۟ عَلَيْكُمْ أَنفُسَكُمْ ۖ لَا يَضُرُّكُم مَّن ضَلَّ إِذَا ٱهْتَدَيْتُمْ ۚ إِلَى ٱللَّهِ مَرْجِعُكُمْ جَمِيعًۭا فَيُنَبِّئُكُم بِمَا كُنتُمْ تَعْمَلُونَ ﴿١٠٥﴾\\
\textamh{106.\  } & يَـٰٓأَيُّهَا ٱلَّذِينَ ءَامَنُوا۟ شَهَـٰدَةُ بَيْنِكُمْ إِذَا حَضَرَ أَحَدَكُمُ ٱلْمَوْتُ حِينَ ٱلْوَصِيَّةِ ٱثْنَانِ ذَوَا عَدْلٍۢ مِّنكُمْ أَوْ ءَاخَرَانِ مِنْ غَيْرِكُمْ إِنْ أَنتُمْ ضَرَبْتُمْ فِى ٱلْأَرْضِ فَأَصَـٰبَتْكُم مُّصِيبَةُ ٱلْمَوْتِ ۚ تَحْبِسُونَهُمَا مِنۢ بَعْدِ ٱلصَّلَوٰةِ فَيُقْسِمَانِ بِٱللَّهِ إِنِ ٱرْتَبْتُمْ لَا نَشْتَرِى بِهِۦ ثَمَنًۭا وَلَوْ كَانَ ذَا قُرْبَىٰ ۙ وَلَا نَكْتُمُ شَهَـٰدَةَ ٱللَّهِ إِنَّآ إِذًۭا لَّمِنَ ٱلْءَاثِمِينَ ﴿١٠٦﴾\\
\textamh{107.\  } & فَإِنْ عُثِرَ عَلَىٰٓ أَنَّهُمَا ٱسْتَحَقَّآ إِثْمًۭا فَـَٔاخَرَانِ يَقُومَانِ مَقَامَهُمَا مِنَ ٱلَّذِينَ ٱسْتَحَقَّ عَلَيْهِمُ ٱلْأَوْلَيَـٰنِ فَيُقْسِمَانِ بِٱللَّهِ لَشَهَـٰدَتُنَآ أَحَقُّ مِن شَهَـٰدَتِهِمَا وَمَا ٱعْتَدَيْنَآ إِنَّآ إِذًۭا لَّمِنَ ٱلظَّـٰلِمِينَ ﴿١٠٧﴾\\
\textamh{108.\  } & ذَٟلِكَ أَدْنَىٰٓ أَن يَأْتُوا۟ بِٱلشَّهَـٰدَةِ عَلَىٰ وَجْهِهَآ أَوْ يَخَافُوٓا۟ أَن تُرَدَّ أَيْمَـٰنٌۢ بَعْدَ أَيْمَـٰنِهِمْ ۗ وَٱتَّقُوا۟ ٱللَّهَ وَٱسْمَعُوا۟ ۗ وَٱللَّهُ لَا يَهْدِى ٱلْقَوْمَ ٱلْفَـٰسِقِينَ ﴿١٠٨﴾\\
\textamh{109.\  } & ۞ يَوْمَ يَجْمَعُ ٱللَّهُ ٱلرُّسُلَ فَيَقُولُ مَاذَآ أُجِبْتُمْ ۖ قَالُوا۟ لَا عِلْمَ لَنَآ ۖ إِنَّكَ أَنتَ عَلَّٰمُ ٱلْغُيُوبِ ﴿١٠٩﴾\\
\textamh{110.\  } & إِذْ قَالَ ٱللَّهُ يَـٰعِيسَى ٱبْنَ مَرْيَمَ ٱذْكُرْ نِعْمَتِى عَلَيْكَ وَعَلَىٰ وَٟلِدَتِكَ إِذْ أَيَّدتُّكَ بِرُوحِ ٱلْقُدُسِ تُكَلِّمُ ٱلنَّاسَ فِى ٱلْمَهْدِ وَكَهْلًۭا ۖ وَإِذْ عَلَّمْتُكَ ٱلْكِتَـٰبَ وَٱلْحِكْمَةَ وَٱلتَّوْرَىٰةَ وَٱلْإِنجِيلَ ۖ وَإِذْ تَخْلُقُ مِنَ ٱلطِّينِ كَهَيْـَٔةِ ٱلطَّيْرِ بِإِذْنِى فَتَنفُخُ فِيهَا فَتَكُونُ طَيْرًۢا بِإِذْنِى ۖ وَتُبْرِئُ ٱلْأَكْمَهَ وَٱلْأَبْرَصَ بِإِذْنِى ۖ وَإِذْ تُخْرِجُ ٱلْمَوْتَىٰ بِإِذْنِى ۖ وَإِذْ كَفَفْتُ بَنِىٓ إِسْرَٰٓءِيلَ عَنكَ إِذْ جِئْتَهُم بِٱلْبَيِّنَـٰتِ فَقَالَ ٱلَّذِينَ كَفَرُوا۟ مِنْهُمْ إِنْ هَـٰذَآ إِلَّا سِحْرٌۭ مُّبِينٌۭ ﴿١١٠﴾\\
\textamh{111.\  } & وَإِذْ أَوْحَيْتُ إِلَى ٱلْحَوَارِيِّۦنَ أَنْ ءَامِنُوا۟ بِى وَبِرَسُولِى قَالُوٓا۟ ءَامَنَّا وَٱشْهَدْ بِأَنَّنَا مُسْلِمُونَ ﴿١١١﴾\\
\textamh{112.\  } & إِذْ قَالَ ٱلْحَوَارِيُّونَ يَـٰعِيسَى ٱبْنَ مَرْيَمَ هَلْ يَسْتَطِيعُ رَبُّكَ أَن يُنَزِّلَ عَلَيْنَا مَآئِدَةًۭ مِّنَ ٱلسَّمَآءِ ۖ قَالَ ٱتَّقُوا۟ ٱللَّهَ إِن كُنتُم مُّؤْمِنِينَ ﴿١١٢﴾\\
\textamh{113.\  } & قَالُوا۟ نُرِيدُ أَن نَّأْكُلَ مِنْهَا وَتَطْمَئِنَّ قُلُوبُنَا وَنَعْلَمَ أَن قَدْ صَدَقْتَنَا وَنَكُونَ عَلَيْهَا مِنَ ٱلشَّـٰهِدِينَ ﴿١١٣﴾\\
\textamh{114.\  } & قَالَ عِيسَى ٱبْنُ مَرْيَمَ ٱللَّهُمَّ رَبَّنَآ أَنزِلْ عَلَيْنَا مَآئِدَةًۭ مِّنَ ٱلسَّمَآءِ تَكُونُ لَنَا عِيدًۭا لِّأَوَّلِنَا وَءَاخِرِنَا وَءَايَةًۭ مِّنكَ ۖ وَٱرْزُقْنَا وَأَنتَ خَيْرُ ٱلرَّٟزِقِينَ ﴿١١٤﴾\\
\textamh{115.\  } & قَالَ ٱللَّهُ إِنِّى مُنَزِّلُهَا عَلَيْكُمْ ۖ فَمَن يَكْفُرْ بَعْدُ مِنكُمْ فَإِنِّىٓ أُعَذِّبُهُۥ عَذَابًۭا لَّآ أُعَذِّبُهُۥٓ أَحَدًۭا مِّنَ ٱلْعَـٰلَمِينَ ﴿١١٥﴾\\
\textamh{116.\  } & وَإِذْ قَالَ ٱللَّهُ يَـٰعِيسَى ٱبْنَ مَرْيَمَ ءَأَنتَ قُلْتَ لِلنَّاسِ ٱتَّخِذُونِى وَأُمِّىَ إِلَـٰهَيْنِ مِن دُونِ ٱللَّهِ ۖ قَالَ سُبْحَـٰنَكَ مَا يَكُونُ لِىٓ أَنْ أَقُولَ مَا لَيْسَ لِى بِحَقٍّ ۚ إِن كُنتُ قُلْتُهُۥ فَقَدْ عَلِمْتَهُۥ ۚ تَعْلَمُ مَا فِى نَفْسِى وَلَآ أَعْلَمُ مَا فِى نَفْسِكَ ۚ إِنَّكَ أَنتَ عَلَّٰمُ ٱلْغُيُوبِ ﴿١١٦﴾\\
\textamh{117.\  } & مَا قُلْتُ لَهُمْ إِلَّا مَآ أَمَرْتَنِى بِهِۦٓ أَنِ ٱعْبُدُوا۟ ٱللَّهَ رَبِّى وَرَبَّكُمْ ۚ وَكُنتُ عَلَيْهِمْ شَهِيدًۭا مَّا دُمْتُ فِيهِمْ ۖ فَلَمَّا تَوَفَّيْتَنِى كُنتَ أَنتَ ٱلرَّقِيبَ عَلَيْهِمْ ۚ وَأَنتَ عَلَىٰ كُلِّ شَىْءٍۢ شَهِيدٌ ﴿١١٧﴾\\
\textamh{118.\  } & إِن تُعَذِّبْهُمْ فَإِنَّهُمْ عِبَادُكَ ۖ وَإِن تَغْفِرْ لَهُمْ فَإِنَّكَ أَنتَ ٱلْعَزِيزُ ٱلْحَكِيمُ ﴿١١٨﴾\\
\textamh{119.\  } & قَالَ ٱللَّهُ هَـٰذَا يَوْمُ يَنفَعُ ٱلصَّـٰدِقِينَ صِدْقُهُمْ ۚ لَهُمْ جَنَّـٰتٌۭ تَجْرِى مِن تَحْتِهَا ٱلْأَنْهَـٰرُ خَـٰلِدِينَ فِيهَآ أَبَدًۭا ۚ رَّضِىَ ٱللَّهُ عَنْهُمْ وَرَضُوا۟ عَنْهُ ۚ ذَٟلِكَ ٱلْفَوْزُ ٱلْعَظِيمُ ﴿١١٩﴾\\
\textamh{120.\  } & لِلَّهِ مُلْكُ ٱلسَّمَـٰوَٟتِ وَٱلْأَرْضِ وَمَا فِيهِنَّ ۚ وَهُوَ عَلَىٰ كُلِّ شَىْءٍۢ قَدِيرٌۢ ﴿١٢٠﴾\\
\end{longtable}
\clearpage
