%% License: BSD style (Berkley) (i.e. Put the Copyright owner's name always)
%% Writer and Copyright (to): Bewketu(Bilal) Tadilo (2016-17)
\centering\section{\LR{\textamharic{ሱራቱ አልዛረያት -}  \RL{سوره  الذاريات}}}
\begin{longtable}{%
  @{}
    p{.5\textwidth}
  @{~~~~~~~~~~~~~}
    p{.5\textwidth}
    @{}
}
\nopagebreak
\textamh{\ \ \ \ \ \  ቢስሚላሂ አራህመኒ ራሂይም } &  بِسْمِ ٱللَّهِ ٱلرَّحْمَـٰنِ ٱلرَّحِيمِ\\
\textamh{1.\  } &  وَٱلذَّٰرِيَـٰتِ ذَرْوًۭا ﴿١﴾\\
\textamh{2.\  } & فَٱلْحَـٰمِلَـٰتِ وِقْرًۭا ﴿٢﴾\\
\textamh{3.\  } & فَٱلْجَٰرِيَـٰتِ يُسْرًۭا ﴿٣﴾\\
\textamh{4.\  } & فَٱلْمُقَسِّمَـٰتِ أَمْرًا ﴿٤﴾\\
\textamh{5.\  } & إِنَّمَا تُوعَدُونَ لَصَادِقٌۭ ﴿٥﴾\\
\textamh{6.\  } & وَإِنَّ ٱلدِّينَ لَوَٟقِعٌۭ ﴿٦﴾\\
\textamh{7.\  } & وَٱلسَّمَآءِ ذَاتِ ٱلْحُبُكِ ﴿٧﴾\\
\textamh{8.\  } & إِنَّكُمْ لَفِى قَوْلٍۢ مُّخْتَلِفٍۢ ﴿٨﴾\\
\textamh{9.\  } & يُؤْفَكُ عَنْهُ مَنْ أُفِكَ ﴿٩﴾\\
\textamh{10.\  } & قُتِلَ ٱلْخَرَّٟصُونَ ﴿١٠﴾\\
\textamh{11.\  } & ٱلَّذِينَ هُمْ فِى غَمْرَةٍۢ سَاهُونَ ﴿١١﴾\\
\textamh{12.\  } & يَسْـَٔلُونَ أَيَّانَ يَوْمُ ٱلدِّينِ ﴿١٢﴾\\
\textamh{13.\  } & يَوْمَ هُمْ عَلَى ٱلنَّارِ يُفْتَنُونَ ﴿١٣﴾\\
\textamh{14.\  } & ذُوقُوا۟ فِتْنَتَكُمْ هَـٰذَا ٱلَّذِى كُنتُم بِهِۦ تَسْتَعْجِلُونَ ﴿١٤﴾\\
\textamh{15.\  } & إِنَّ ٱلْمُتَّقِينَ فِى جَنَّـٰتٍۢ وَعُيُونٍ ﴿١٥﴾\\
\textamh{16.\  } & ءَاخِذِينَ مَآ ءَاتَىٰهُمْ رَبُّهُمْ ۚ إِنَّهُمْ كَانُوا۟ قَبْلَ ذَٟلِكَ مُحْسِنِينَ ﴿١٦﴾\\
\textamh{17.\  } & كَانُوا۟ قَلِيلًۭا مِّنَ ٱلَّيْلِ مَا يَهْجَعُونَ ﴿١٧﴾\\
\textamh{18.\  } & وَبِٱلْأَسْحَارِ هُمْ يَسْتَغْفِرُونَ ﴿١٨﴾\\
\textamh{19.\  } & وَفِىٓ أَمْوَٟلِهِمْ حَقٌّۭ لِّلسَّآئِلِ وَٱلْمَحْرُومِ ﴿١٩﴾\\
\textamh{20.\  } & وَفِى ٱلْأَرْضِ ءَايَـٰتٌۭ لِّلْمُوقِنِينَ ﴿٢٠﴾\\
\textamh{21.\  } & وَفِىٓ أَنفُسِكُمْ ۚ أَفَلَا تُبْصِرُونَ ﴿٢١﴾\\
\textamh{22.\  } & وَفِى ٱلسَّمَآءِ رِزْقُكُمْ وَمَا تُوعَدُونَ ﴿٢٢﴾\\
\textamh{23.\  } & فَوَرَبِّ ٱلسَّمَآءِ وَٱلْأَرْضِ إِنَّهُۥ لَحَقٌّۭ مِّثْلَ مَآ أَنَّكُمْ تَنطِقُونَ ﴿٢٣﴾\\
\textamh{24.\  } & هَلْ أَتَىٰكَ حَدِيثُ ضَيْفِ إِبْرَٰهِيمَ ٱلْمُكْرَمِينَ ﴿٢٤﴾\\
\textamh{25.\  } & إِذْ دَخَلُوا۟ عَلَيْهِ فَقَالُوا۟ سَلَـٰمًۭا ۖ قَالَ سَلَـٰمٌۭ قَوْمٌۭ مُّنكَرُونَ ﴿٢٥﴾\\
\textamh{26.\  } & فَرَاغَ إِلَىٰٓ أَهْلِهِۦ فَجَآءَ بِعِجْلٍۢ سَمِينٍۢ ﴿٢٦﴾\\
\textamh{27.\  } & فَقَرَّبَهُۥٓ إِلَيْهِمْ قَالَ أَلَا تَأْكُلُونَ ﴿٢٧﴾\\
\textamh{28.\  } & فَأَوْجَسَ مِنْهُمْ خِيفَةًۭ ۖ قَالُوا۟ لَا تَخَفْ ۖ وَبَشَّرُوهُ بِغُلَـٰمٍ عَلِيمٍۢ ﴿٢٨﴾\\
\textamh{29.\  } & فَأَقْبَلَتِ ٱمْرَأَتُهُۥ فِى صَرَّةٍۢ فَصَكَّتْ وَجْهَهَا وَقَالَتْ عَجُوزٌ عَقِيمٌۭ ﴿٢٩﴾\\
\textamh{30.\  } & قَالُوا۟ كَذَٟلِكِ قَالَ رَبُّكِ ۖ إِنَّهُۥ هُوَ ٱلْحَكِيمُ ٱلْعَلِيمُ ﴿٣٠﴾\\
\textamh{31.\  } & ۞ قَالَ فَمَا خَطْبُكُمْ أَيُّهَا ٱلْمُرْسَلُونَ ﴿٣١﴾\\
\textamh{32.\  } & قَالُوٓا۟ إِنَّآ أُرْسِلْنَآ إِلَىٰ قَوْمٍۢ مُّجْرِمِينَ ﴿٣٢﴾\\
\textamh{33.\  } & لِنُرْسِلَ عَلَيْهِمْ حِجَارَةًۭ مِّن طِينٍۢ ﴿٣٣﴾\\
\textamh{34.\  } & مُّسَوَّمَةً عِندَ رَبِّكَ لِلْمُسْرِفِينَ ﴿٣٤﴾\\
\textamh{35.\  } & فَأَخْرَجْنَا مَن كَانَ فِيهَا مِنَ ٱلْمُؤْمِنِينَ ﴿٣٥﴾\\
\textamh{36.\  } & فَمَا وَجَدْنَا فِيهَا غَيْرَ بَيْتٍۢ مِّنَ ٱلْمُسْلِمِينَ ﴿٣٦﴾\\
\textamh{37.\  } & وَتَرَكْنَا فِيهَآ ءَايَةًۭ لِّلَّذِينَ يَخَافُونَ ٱلْعَذَابَ ٱلْأَلِيمَ ﴿٣٧﴾\\
\textamh{38.\  } & وَفِى مُوسَىٰٓ إِذْ أَرْسَلْنَـٰهُ إِلَىٰ فِرْعَوْنَ بِسُلْطَٰنٍۢ مُّبِينٍۢ ﴿٣٨﴾\\
\textamh{39.\  } & فَتَوَلَّىٰ بِرُكْنِهِۦ وَقَالَ سَـٰحِرٌ أَوْ مَجْنُونٌۭ ﴿٣٩﴾\\
\textamh{40.\  } & فَأَخَذْنَـٰهُ وَجُنُودَهُۥ فَنَبَذْنَـٰهُمْ فِى ٱلْيَمِّ وَهُوَ مُلِيمٌۭ ﴿٤٠﴾\\
\textamh{41.\  } & وَفِى عَادٍ إِذْ أَرْسَلْنَا عَلَيْهِمُ ٱلرِّيحَ ٱلْعَقِيمَ ﴿٤١﴾\\
\textamh{42.\  } & مَا تَذَرُ مِن شَىْءٍ أَتَتْ عَلَيْهِ إِلَّا جَعَلَتْهُ كَٱلرَّمِيمِ ﴿٤٢﴾\\
\textamh{43.\  } & وَفِى ثَمُودَ إِذْ قِيلَ لَهُمْ تَمَتَّعُوا۟ حَتَّىٰ حِينٍۢ ﴿٤٣﴾\\
\textamh{44.\  } & فَعَتَوْا۟ عَنْ أَمْرِ رَبِّهِمْ فَأَخَذَتْهُمُ ٱلصَّـٰعِقَةُ وَهُمْ يَنظُرُونَ ﴿٤٤﴾\\
\textamh{45.\  } & فَمَا ٱسْتَطَٰعُوا۟ مِن قِيَامٍۢ وَمَا كَانُوا۟ مُنتَصِرِينَ ﴿٤٥﴾\\
\textamh{46.\  } & وَقَوْمَ نُوحٍۢ مِّن قَبْلُ ۖ إِنَّهُمْ كَانُوا۟ قَوْمًۭا فَـٰسِقِينَ ﴿٤٦﴾\\
\textamh{47.\  } & وَٱلسَّمَآءَ بَنَيْنَـٰهَا بِأَيْي۟دٍۢ وَإِنَّا لَمُوسِعُونَ ﴿٤٧﴾\\
\textamh{48.\  } & وَٱلْأَرْضَ فَرَشْنَـٰهَا فَنِعْمَ ٱلْمَـٰهِدُونَ ﴿٤٨﴾\\
\textamh{49.\  } & وَمِن كُلِّ شَىْءٍ خَلَقْنَا زَوْجَيْنِ لَعَلَّكُمْ تَذَكَّرُونَ ﴿٤٩﴾\\
\textamh{50.\  } & فَفِرُّوٓا۟ إِلَى ٱللَّهِ ۖ إِنِّى لَكُم مِّنْهُ نَذِيرٌۭ مُّبِينٌۭ ﴿٥٠﴾\\
\textamh{51.\  } & وَلَا تَجْعَلُوا۟ مَعَ ٱللَّهِ إِلَـٰهًا ءَاخَرَ ۖ إِنِّى لَكُم مِّنْهُ نَذِيرٌۭ مُّبِينٌۭ ﴿٥١﴾\\
\textamh{52.\  } & كَذَٟلِكَ مَآ أَتَى ٱلَّذِينَ مِن قَبْلِهِم مِّن رَّسُولٍ إِلَّا قَالُوا۟ سَاحِرٌ أَوْ مَجْنُونٌ ﴿٥٢﴾\\
\textamh{53.\  } & أَتَوَاصَوْا۟ بِهِۦ ۚ بَلْ هُمْ قَوْمٌۭ طَاغُونَ ﴿٥٣﴾\\
\textamh{54.\  } & فَتَوَلَّ عَنْهُمْ فَمَآ أَنتَ بِمَلُومٍۢ ﴿٥٤﴾\\
\textamh{55.\  } & وَذَكِّرْ فَإِنَّ ٱلذِّكْرَىٰ تَنفَعُ ٱلْمُؤْمِنِينَ ﴿٥٥﴾\\
\textamh{56.\  } & وَمَا خَلَقْتُ ٱلْجِنَّ وَٱلْإِنسَ إِلَّا لِيَعْبُدُونِ ﴿٥٦﴾\\
\textamh{57.\  } & مَآ أُرِيدُ مِنْهُم مِّن رِّزْقٍۢ وَمَآ أُرِيدُ أَن يُطْعِمُونِ ﴿٥٧﴾\\
\textamh{58.\  } & إِنَّ ٱللَّهَ هُوَ ٱلرَّزَّاقُ ذُو ٱلْقُوَّةِ ٱلْمَتِينُ ﴿٥٨﴾\\
\textamh{59.\  } & فَإِنَّ لِلَّذِينَ ظَلَمُوا۟ ذَنُوبًۭا مِّثْلَ ذَنُوبِ أَصْحَـٰبِهِمْ فَلَا يَسْتَعْجِلُونِ ﴿٥٩﴾\\
\textamh{60.\  } & فَوَيْلٌۭ لِّلَّذِينَ كَفَرُوا۟ مِن يَوْمِهِمُ ٱلَّذِى يُوعَدُونَ ﴿٦٠﴾\\
\end{longtable} \newpage
