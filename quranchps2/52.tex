%% License: BSD style (Berkley) (i.e. Put the Copyright owner's name always)
%% Writer and Copyright (to): Bewketu(Bilal) Tadilo (2016-17)
\begin{center}\section{\LR{\textamhsec{ሱራቱ አጥጡር -}  \textarabic{سوره  الطور}}}\end{center}
\begin{longtable}{%
  @{}
    p{.5\textwidth}
  @{~~~}
    p{.5\textwidth}
    @{}
}
\textamh{ቢስሚላሂ አራህመኒ ራሂይም } &  \mytextarabic{بِسْمِ ٱللَّهِ ٱلرَّحْمَـٰنِ ٱلرَّحِيمِ}\\
\textamh{1.\  } & \mytextarabic{ وَٱلطُّورِ ﴿١﴾}\\
\textamh{2.\  } & \mytextarabic{وَكِتَـٰبٍۢ مَّسْطُورٍۢ ﴿٢﴾}\\
\textamh{3.\  } & \mytextarabic{فِى رَقٍّۢ مَّنشُورٍۢ ﴿٣﴾}\\
\textamh{4.\  } & \mytextarabic{وَٱلْبَيْتِ ٱلْمَعْمُورِ ﴿٤﴾}\\
\textamh{5.\  } & \mytextarabic{وَٱلسَّقْفِ ٱلْمَرْفُوعِ ﴿٥﴾}\\
\textamh{6.\  } & \mytextarabic{وَٱلْبَحْرِ ٱلْمَسْجُورِ ﴿٦﴾}\\
\textamh{7.\  } & \mytextarabic{إِنَّ عَذَابَ رَبِّكَ لَوَٟقِعٌۭ ﴿٧﴾}\\
\textamh{8.\  } & \mytextarabic{مَّا لَهُۥ مِن دَافِعٍۢ ﴿٨﴾}\\
\textamh{9.\  } & \mytextarabic{يَوْمَ تَمُورُ ٱلسَّمَآءُ مَوْرًۭا ﴿٩﴾}\\
\textamh{10.\  } & \mytextarabic{وَتَسِيرُ ٱلْجِبَالُ سَيْرًۭا ﴿١٠﴾}\\
\textamh{11.\  } & \mytextarabic{فَوَيْلٌۭ يَوْمَئِذٍۢ لِّلْمُكَذِّبِينَ ﴿١١﴾}\\
\textamh{12.\  } & \mytextarabic{ٱلَّذِينَ هُمْ فِى خَوْضٍۢ يَلْعَبُونَ ﴿١٢﴾}\\
\textamh{13.\  } & \mytextarabic{يَوْمَ يُدَعُّونَ إِلَىٰ نَارِ جَهَنَّمَ دَعًّا ﴿١٣﴾}\\
\textamh{14.\  } & \mytextarabic{هَـٰذِهِ ٱلنَّارُ ٱلَّتِى كُنتُم بِهَا تُكَذِّبُونَ ﴿١٤﴾}\\
\textamh{15.\  } & \mytextarabic{أَفَسِحْرٌ هَـٰذَآ أَمْ أَنتُمْ لَا تُبْصِرُونَ ﴿١٥﴾}\\
\textamh{16.\  } & \mytextarabic{ٱصْلَوْهَا فَٱصْبِرُوٓا۟ أَوْ لَا تَصْبِرُوا۟ سَوَآءٌ عَلَيْكُمْ ۖ إِنَّمَا تُجْزَوْنَ مَا كُنتُمْ تَعْمَلُونَ ﴿١٦﴾}\\
\textamh{17.\  } & \mytextarabic{إِنَّ ٱلْمُتَّقِينَ فِى جَنَّـٰتٍۢ وَنَعِيمٍۢ ﴿١٧﴾}\\
\textamh{18.\  } & \mytextarabic{فَـٰكِهِينَ بِمَآ ءَاتَىٰهُمْ رَبُّهُمْ وَوَقَىٰهُمْ رَبُّهُمْ عَذَابَ ٱلْجَحِيمِ ﴿١٨﴾}\\
\textamh{19.\  } & \mytextarabic{كُلُوا۟ وَٱشْرَبُوا۟ هَنِيٓـًٔۢا بِمَا كُنتُمْ تَعْمَلُونَ ﴿١٩﴾}\\
\textamh{20.\  } & \mytextarabic{مُتَّكِـِٔينَ عَلَىٰ سُرُرٍۢ مَّصْفُوفَةٍۢ ۖ وَزَوَّجْنَـٰهُم بِحُورٍ عِينٍۢ ﴿٢٠﴾}\\
\textamh{21.\  } & \mytextarabic{وَٱلَّذِينَ ءَامَنُوا۟ وَٱتَّبَعَتْهُمْ ذُرِّيَّتُهُم بِإِيمَـٰنٍ أَلْحَقْنَا بِهِمْ ذُرِّيَّتَهُمْ وَمَآ أَلَتْنَـٰهُم مِّنْ عَمَلِهِم مِّن شَىْءٍۢ ۚ كُلُّ ٱمْرِئٍۭ بِمَا كَسَبَ رَهِينٌۭ ﴿٢١﴾}\\
\textamh{22.\  } & \mytextarabic{وَأَمْدَدْنَـٰهُم بِفَـٰكِهَةٍۢ وَلَحْمٍۢ مِّمَّا يَشْتَهُونَ ﴿٢٢﴾}\\
\textamh{23.\  } & \mytextarabic{يَتَنَـٰزَعُونَ فِيهَا كَأْسًۭا لَّا لَغْوٌۭ فِيهَا وَلَا تَأْثِيمٌۭ ﴿٢٣﴾}\\
\textamh{24.\  } & \mytextarabic{۞ وَيَطُوفُ عَلَيْهِمْ غِلْمَانٌۭ لَّهُمْ كَأَنَّهُمْ لُؤْلُؤٌۭ مَّكْنُونٌۭ ﴿٢٤﴾}\\
\textamh{25.\  } & \mytextarabic{وَأَقْبَلَ بَعْضُهُمْ عَلَىٰ بَعْضٍۢ يَتَسَآءَلُونَ ﴿٢٥﴾}\\
\textamh{26.\  } & \mytextarabic{قَالُوٓا۟ إِنَّا كُنَّا قَبْلُ فِىٓ أَهْلِنَا مُشْفِقِينَ ﴿٢٦﴾}\\
\textamh{27.\  } & \mytextarabic{فَمَنَّ ٱللَّهُ عَلَيْنَا وَوَقَىٰنَا عَذَابَ ٱلسَّمُومِ ﴿٢٧﴾}\\
\textamh{28.\  } & \mytextarabic{إِنَّا كُنَّا مِن قَبْلُ نَدْعُوهُ ۖ إِنَّهُۥ هُوَ ٱلْبَرُّ ٱلرَّحِيمُ ﴿٢٨﴾}\\
\textamh{29.\  } & \mytextarabic{فَذَكِّرْ فَمَآ أَنتَ بِنِعْمَتِ رَبِّكَ بِكَاهِنٍۢ وَلَا مَجْنُونٍ ﴿٢٩﴾}\\
\textamh{30.\  } & \mytextarabic{أَمْ يَقُولُونَ شَاعِرٌۭ نَّتَرَبَّصُ بِهِۦ رَيْبَ ٱلْمَنُونِ ﴿٣٠﴾}\\
\textamh{31.\  } & \mytextarabic{قُلْ تَرَبَّصُوا۟ فَإِنِّى مَعَكُم مِّنَ ٱلْمُتَرَبِّصِينَ ﴿٣١﴾}\\
\textamh{32.\  } & \mytextarabic{أَمْ تَأْمُرُهُمْ أَحْلَـٰمُهُم بِهَـٰذَآ ۚ أَمْ هُمْ قَوْمٌۭ طَاغُونَ ﴿٣٢﴾}\\
\textamh{33.\  } & \mytextarabic{أَمْ يَقُولُونَ تَقَوَّلَهُۥ ۚ بَل لَّا يُؤْمِنُونَ ﴿٣٣﴾}\\
\textamh{34.\  } & \mytextarabic{فَلْيَأْتُوا۟ بِحَدِيثٍۢ مِّثْلِهِۦٓ إِن كَانُوا۟ صَـٰدِقِينَ ﴿٣٤﴾}\\
\textamh{35.\  } & \mytextarabic{أَمْ خُلِقُوا۟ مِنْ غَيْرِ شَىْءٍ أَمْ هُمُ ٱلْخَـٰلِقُونَ ﴿٣٥﴾}\\
\textamh{36.\  } & \mytextarabic{أَمْ خَلَقُوا۟ ٱلسَّمَـٰوَٟتِ وَٱلْأَرْضَ ۚ بَل لَّا يُوقِنُونَ ﴿٣٦﴾}\\
\textamh{37.\  } & \mytextarabic{أَمْ عِندَهُمْ خَزَآئِنُ رَبِّكَ أَمْ هُمُ ٱلْمُصَۣيْطِرُونَ ﴿٣٧﴾}\\
\textamh{38.\  } & \mytextarabic{أَمْ لَهُمْ سُلَّمٌۭ يَسْتَمِعُونَ فِيهِ ۖ فَلْيَأْتِ مُسْتَمِعُهُم بِسُلْطَٰنٍۢ مُّبِينٍ ﴿٣٨﴾}\\
\textamh{39.\  } & \mytextarabic{أَمْ لَهُ ٱلْبَنَـٰتُ وَلَكُمُ ٱلْبَنُونَ ﴿٣٩﴾}\\
\textamh{40.\  } & \mytextarabic{أَمْ تَسْـَٔلُهُمْ أَجْرًۭا فَهُم مِّن مَّغْرَمٍۢ مُّثْقَلُونَ ﴿٤٠﴾}\\
\textamh{41.\  } & \mytextarabic{أَمْ عِندَهُمُ ٱلْغَيْبُ فَهُمْ يَكْتُبُونَ ﴿٤١﴾}\\
\textamh{42.\  } & \mytextarabic{أَمْ يُرِيدُونَ كَيْدًۭا ۖ فَٱلَّذِينَ كَفَرُوا۟ هُمُ ٱلْمَكِيدُونَ ﴿٤٢﴾}\\
\textamh{43.\  } & \mytextarabic{أَمْ لَهُمْ إِلَـٰهٌ غَيْرُ ٱللَّهِ ۚ سُبْحَـٰنَ ٱللَّهِ عَمَّا يُشْرِكُونَ ﴿٤٣﴾}\\
\textamh{44.\  } & \mytextarabic{وَإِن يَرَوْا۟ كِسْفًۭا مِّنَ ٱلسَّمَآءِ سَاقِطًۭا يَقُولُوا۟ سَحَابٌۭ مَّرْكُومٌۭ ﴿٤٤﴾}\\
\textamh{45.\  } & \mytextarabic{فَذَرْهُمْ حَتَّىٰ يُلَـٰقُوا۟ يَوْمَهُمُ ٱلَّذِى فِيهِ يُصْعَقُونَ ﴿٤٥﴾}\\
\textamh{46.\  } & \mytextarabic{يَوْمَ لَا يُغْنِى عَنْهُمْ كَيْدُهُمْ شَيْـًۭٔا وَلَا هُمْ يُنصَرُونَ ﴿٤٦﴾}\\
\textamh{47.\  } & \mytextarabic{وَإِنَّ لِلَّذِينَ ظَلَمُوا۟ عَذَابًۭا دُونَ ذَٟلِكَ وَلَـٰكِنَّ أَكْثَرَهُمْ لَا يَعْلَمُونَ ﴿٤٧﴾}\\
\textamh{48.\  } & \mytextarabic{وَٱصْبِرْ لِحُكْمِ رَبِّكَ فَإِنَّكَ بِأَعْيُنِنَا ۖ وَسَبِّحْ بِحَمْدِ رَبِّكَ حِينَ تَقُومُ ﴿٤٨﴾}\\
\textamh{49.\  } & \mytextarabic{وَمِنَ ٱلَّيْلِ فَسَبِّحْهُ وَإِدْبَٰرَ ٱلنُّجُومِ ﴿٤٩﴾}\\
\end{longtable}
\clearpage