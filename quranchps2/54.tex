%% License: BSD style (Berkley) (i.e. Put the Copyright owner's name always)
%% Writer and Copyright (to): Bewketu(Bilal) Tadilo (2016-17)
\centering\section{\LR{\textamharic{ሱራቱ አልቀመር -}  \RL{سوره  القمر}}}
\begin{longtable}{%
  @{}
    p{.5\textwidth}
  @{~~~~~~~~~~~~~}
    p{.5\textwidth}
    @{}
}
\nopagebreak
\textamh{ቢስሚላሂ አራህመኒ ራሂይም } &  بِسْمِ ٱللَّهِ ٱلرَّحْمَـٰنِ ٱلرَّحِيمِ\\
\textamh{1.\  } &  ٱقْتَرَبَتِ ٱلسَّاعَةُ وَٱنشَقَّ ٱلْقَمَرُ ﴿١﴾\\
\textamh{2.\  } & وَإِن يَرَوْا۟ ءَايَةًۭ يُعْرِضُوا۟ وَيَقُولُوا۟ سِحْرٌۭ مُّسْتَمِرٌّۭ ﴿٢﴾\\
\textamh{3.\  } & وَكَذَّبُوا۟ وَٱتَّبَعُوٓا۟ أَهْوَآءَهُمْ ۚ وَكُلُّ أَمْرٍۢ مُّسْتَقِرٌّۭ ﴿٣﴾\\
\textamh{4.\  } & وَلَقَدْ جَآءَهُم مِّنَ ٱلْأَنۢبَآءِ مَا فِيهِ مُزْدَجَرٌ ﴿٤﴾\\
\textamh{5.\  } & حِكْمَةٌۢ بَٰلِغَةٌۭ ۖ فَمَا تُغْنِ ٱلنُّذُرُ ﴿٥﴾\\
\textamh{6.\  } & فَتَوَلَّ عَنْهُمْ ۘ يَوْمَ يَدْعُ ٱلدَّاعِ إِلَىٰ شَىْءٍۢ نُّكُرٍ ﴿٦﴾\\
\textamh{7.\  } & خُشَّعًا أَبْصَـٰرُهُمْ يَخْرُجُونَ مِنَ ٱلْأَجْدَاثِ كَأَنَّهُمْ جَرَادٌۭ مُّنتَشِرٌۭ ﴿٧﴾\\
\textamh{8.\  } & مُّهْطِعِينَ إِلَى ٱلدَّاعِ ۖ يَقُولُ ٱلْكَـٰفِرُونَ هَـٰذَا يَوْمٌ عَسِرٌۭ ﴿٨﴾\\
\textamh{9.\  } & ۞ كَذَّبَتْ قَبْلَهُمْ قَوْمُ نُوحٍۢ فَكَذَّبُوا۟ عَبْدَنَا وَقَالُوا۟ مَجْنُونٌۭ وَٱزْدُجِرَ ﴿٩﴾\\
\textamh{10.\  } & فَدَعَا رَبَّهُۥٓ أَنِّى مَغْلُوبٌۭ فَٱنتَصِرْ ﴿١٠﴾\\
\textamh{11.\  } & فَفَتَحْنَآ أَبْوَٟبَ ٱلسَّمَآءِ بِمَآءٍۢ مُّنْهَمِرٍۢ ﴿١١﴾\\
\textamh{12.\  } & وَفَجَّرْنَا ٱلْأَرْضَ عُيُونًۭا فَٱلْتَقَى ٱلْمَآءُ عَلَىٰٓ أَمْرٍۢ قَدْ قُدِرَ ﴿١٢﴾\\
\textamh{13.\  } & وَحَمَلْنَـٰهُ عَلَىٰ ذَاتِ أَلْوَٟحٍۢ وَدُسُرٍۢ ﴿١٣﴾\\
\textamh{14.\  } & تَجْرِى بِأَعْيُنِنَا جَزَآءًۭ لِّمَن كَانَ كُفِرَ ﴿١٤﴾\\
\textamh{15.\  } & وَلَقَد تَّرَكْنَـٰهَآ ءَايَةًۭ فَهَلْ مِن مُّدَّكِرٍۢ ﴿١٥﴾\\
\textamh{16.\  } & فَكَيْفَ كَانَ عَذَابِى وَنُذُرِ ﴿١٦﴾\\
\textamh{17.\  } & وَلَقَدْ يَسَّرْنَا ٱلْقُرْءَانَ لِلذِّكْرِ فَهَلْ مِن مُّدَّكِرٍۢ ﴿١٧﴾\\
\textamh{18.\  } & كَذَّبَتْ عَادٌۭ فَكَيْفَ كَانَ عَذَابِى وَنُذُرِ ﴿١٨﴾\\
\textamh{19.\  } & إِنَّآ أَرْسَلْنَا عَلَيْهِمْ رِيحًۭا صَرْصَرًۭا فِى يَوْمِ نَحْسٍۢ مُّسْتَمِرٍّۢ ﴿١٩﴾\\
\textamh{20.\  } & تَنزِعُ ٱلنَّاسَ كَأَنَّهُمْ أَعْجَازُ نَخْلٍۢ مُّنقَعِرٍۢ ﴿٢٠﴾\\
\textamh{21.\  } & فَكَيْفَ كَانَ عَذَابِى وَنُذُرِ ﴿٢١﴾\\
\textamh{22.\  } & وَلَقَدْ يَسَّرْنَا ٱلْقُرْءَانَ لِلذِّكْرِ فَهَلْ مِن مُّدَّكِرٍۢ ﴿٢٢﴾\\
\textamh{23.\  } & كَذَّبَتْ ثَمُودُ بِٱلنُّذُرِ ﴿٢٣﴾\\
\textamh{24.\  } & فَقَالُوٓا۟ أَبَشَرًۭا مِّنَّا وَٟحِدًۭا نَّتَّبِعُهُۥٓ إِنَّآ إِذًۭا لَّفِى ضَلَـٰلٍۢ وَسُعُرٍ ﴿٢٤﴾\\
\textamh{25.\  } & أَءُلْقِىَ ٱلذِّكْرُ عَلَيْهِ مِنۢ بَيْنِنَا بَلْ هُوَ كَذَّابٌ أَشِرٌۭ ﴿٢٥﴾\\
\textamh{26.\  } & سَيَعْلَمُونَ غَدًۭا مَّنِ ٱلْكَذَّابُ ٱلْأَشِرُ ﴿٢٦﴾\\
\textamh{27.\  } & إِنَّا مُرْسِلُوا۟ ٱلنَّاقَةِ فِتْنَةًۭ لَّهُمْ فَٱرْتَقِبْهُمْ وَٱصْطَبِرْ ﴿٢٧﴾\\
\textamh{28.\  } & وَنَبِّئْهُمْ أَنَّ ٱلْمَآءَ قِسْمَةٌۢ بَيْنَهُمْ ۖ كُلُّ شِرْبٍۢ مُّحْتَضَرٌۭ ﴿٢٨﴾\\
\textamh{29.\  } & فَنَادَوْا۟ صَاحِبَهُمْ فَتَعَاطَىٰ فَعَقَرَ ﴿٢٩﴾\\
\textamh{30.\  } & فَكَيْفَ كَانَ عَذَابِى وَنُذُرِ ﴿٣٠﴾\\
\textamh{31.\  } & إِنَّآ أَرْسَلْنَا عَلَيْهِمْ صَيْحَةًۭ وَٟحِدَةًۭ فَكَانُوا۟ كَهَشِيمِ ٱلْمُحْتَظِرِ ﴿٣١﴾\\
\textamh{32.\  } & وَلَقَدْ يَسَّرْنَا ٱلْقُرْءَانَ لِلذِّكْرِ فَهَلْ مِن مُّدَّكِرٍۢ ﴿٣٢﴾\\
\textamh{33.\  } & كَذَّبَتْ قَوْمُ لُوطٍۭ بِٱلنُّذُرِ ﴿٣٣﴾\\
\textamh{34.\  } & إِنَّآ أَرْسَلْنَا عَلَيْهِمْ حَاصِبًا إِلَّآ ءَالَ لُوطٍۢ ۖ نَّجَّيْنَـٰهُم بِسَحَرٍۢ ﴿٣٤﴾\\
\textamh{35.\  } & نِّعْمَةًۭ مِّنْ عِندِنَا ۚ كَذَٟلِكَ نَجْزِى مَن شَكَرَ ﴿٣٥﴾\\
\textamh{36.\  } & وَلَقَدْ أَنذَرَهُم بَطْشَتَنَا فَتَمَارَوْا۟ بِٱلنُّذُرِ ﴿٣٦﴾\\
\textamh{37.\  } & وَلَقَدْ رَٰوَدُوهُ عَن ضَيْفِهِۦ فَطَمَسْنَآ أَعْيُنَهُمْ فَذُوقُوا۟ عَذَابِى وَنُذُرِ ﴿٣٧﴾\\
\textamh{38.\  } & وَلَقَدْ صَبَّحَهُم بُكْرَةً عَذَابٌۭ مُّسْتَقِرٌّۭ ﴿٣٨﴾\\
\textamh{39.\  } & فَذُوقُوا۟ عَذَابِى وَنُذُرِ ﴿٣٩﴾\\
\textamh{40.\  } & وَلَقَدْ يَسَّرْنَا ٱلْقُرْءَانَ لِلذِّكْرِ فَهَلْ مِن مُّدَّكِرٍۢ ﴿٤٠﴾\\
\textamh{41.\  } & وَلَقَدْ جَآءَ ءَالَ فِرْعَوْنَ ٱلنُّذُرُ ﴿٤١﴾\\
\textamh{42.\  } & كَذَّبُوا۟ بِـَٔايَـٰتِنَا كُلِّهَا فَأَخَذْنَـٰهُمْ أَخْذَ عَزِيزٍۢ مُّقْتَدِرٍ ﴿٤٢﴾\\
\textamh{43.\  } & أَكُفَّارُكُمْ خَيْرٌۭ مِّنْ أُو۟لَـٰٓئِكُمْ أَمْ لَكُم بَرَآءَةٌۭ فِى ٱلزُّبُرِ ﴿٤٣﴾\\
\textamh{44.\  } & أَمْ يَقُولُونَ نَحْنُ جَمِيعٌۭ مُّنتَصِرٌۭ ﴿٤٤﴾\\
\textamh{45.\  } & سَيُهْزَمُ ٱلْجَمْعُ وَيُوَلُّونَ ٱلدُّبُرَ ﴿٤٥﴾\\
\textamh{46.\  } & بَلِ ٱلسَّاعَةُ مَوْعِدُهُمْ وَٱلسَّاعَةُ أَدْهَىٰ وَأَمَرُّ ﴿٤٦﴾\\
\textamh{47.\  } & إِنَّ ٱلْمُجْرِمِينَ فِى ضَلَـٰلٍۢ وَسُعُرٍۢ ﴿٤٧﴾\\
\textamh{48.\  } & يَوْمَ يُسْحَبُونَ فِى ٱلنَّارِ عَلَىٰ وُجُوهِهِمْ ذُوقُوا۟ مَسَّ سَقَرَ ﴿٤٨﴾\\
\textamh{49.\  } & إِنَّا كُلَّ شَىْءٍ خَلَقْنَـٰهُ بِقَدَرٍۢ ﴿٤٩﴾\\
\textamh{50.\  } & وَمَآ أَمْرُنَآ إِلَّا وَٟحِدَةٌۭ كَلَمْحٍۭ بِٱلْبَصَرِ ﴿٥٠﴾\\
\textamh{51.\  } & وَلَقَدْ أَهْلَكْنَآ أَشْيَاعَكُمْ فَهَلْ مِن مُّدَّكِرٍۢ ﴿٥١﴾\\
\textamh{52.\  } & وَكُلُّ شَىْءٍۢ فَعَلُوهُ فِى ٱلزُّبُرِ ﴿٥٢﴾\\
\textamh{53.\  } & وَكُلُّ صَغِيرٍۢ وَكَبِيرٍۢ مُّسْتَطَرٌ ﴿٥٣﴾\\
\textamh{54.\  } & إِنَّ ٱلْمُتَّقِينَ فِى جَنَّـٰتٍۢ وَنَهَرٍۢ ﴿٥٤﴾\\
\textamh{55.\  } & فِى مَقْعَدِ صِدْقٍ عِندَ مَلِيكٍۢ مُّقْتَدِرٍۭ ﴿٥٥﴾\\
\end{longtable}
\clearpage