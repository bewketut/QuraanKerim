%% License: BSD style (Berkley) (i.e. Put the Copyright owner's name always)
%% Writer and Copyright (to): Bewketu(Bilal) Tadilo (2016-17)
\begin{center}\section{\LR{\textamhsec{ሱራቱ አልዋቂያ -}  \textarabic{سوره  الواقعة}}}\end{center}
\begin{longtable}{%
  @{}
    p{.5\textwidth}
  @{~~~}
    p{.5\textwidth}
    @{}
}
\textamh{ቢስሚላሂ አራህመኒ ራሂይም } &  \mytextarabic{بِسْمِ ٱللَّهِ ٱلرَّحْمَـٰنِ ٱلرَّحِيمِ}\\
\textamh{1.\  } & \mytextarabic{ إِذَا وَقَعَتِ ٱلْوَاقِعَةُ ﴿١﴾}\\
\textamh{2.\  } & \mytextarabic{لَيْسَ لِوَقْعَتِهَا كَاذِبَةٌ ﴿٢﴾}\\
\textamh{3.\  } & \mytextarabic{خَافِضَةٌۭ رَّافِعَةٌ ﴿٣﴾}\\
\textamh{4.\  } & \mytextarabic{إِذَا رُجَّتِ ٱلْأَرْضُ رَجًّۭا ﴿٤﴾}\\
\textamh{5.\  } & \mytextarabic{وَبُسَّتِ ٱلْجِبَالُ بَسًّۭا ﴿٥﴾}\\
\textamh{6.\  } & \mytextarabic{فَكَانَتْ هَبَآءًۭ مُّنۢبَثًّۭا ﴿٦﴾}\\
\textamh{7.\  } & \mytextarabic{وَكُنتُمْ أَزْوَٟجًۭا ثَلَـٰثَةًۭ ﴿٧﴾}\\
\textamh{8.\  } & \mytextarabic{فَأَصْحَـٰبُ ٱلْمَيْمَنَةِ مَآ أَصْحَـٰبُ ٱلْمَيْمَنَةِ ﴿٨﴾}\\
\textamh{9.\  } & \mytextarabic{وَأَصْحَـٰبُ ٱلْمَشْـَٔمَةِ مَآ أَصْحَـٰبُ ٱلْمَشْـَٔمَةِ ﴿٩﴾}\\
\textamh{10.\  } & \mytextarabic{وَٱلسَّٰبِقُونَ ٱلسَّٰبِقُونَ ﴿١٠﴾}\\
\textamh{11.\  } & \mytextarabic{أُو۟لَـٰٓئِكَ ٱلْمُقَرَّبُونَ ﴿١١﴾}\\
\textamh{12.\  } & \mytextarabic{فِى جَنَّـٰتِ ٱلنَّعِيمِ ﴿١٢﴾}\\
\textamh{13.\  } & \mytextarabic{ثُلَّةٌۭ مِّنَ ٱلْأَوَّلِينَ ﴿١٣﴾}\\
\textamh{14.\  } & \mytextarabic{وَقَلِيلٌۭ مِّنَ ٱلْءَاخِرِينَ ﴿١٤﴾}\\
\textamh{15.\  } & \mytextarabic{عَلَىٰ سُرُرٍۢ مَّوْضُونَةٍۢ ﴿١٥﴾}\\
\textamh{16.\  } & \mytextarabic{مُّتَّكِـِٔينَ عَلَيْهَا مُتَقَـٰبِلِينَ ﴿١٦﴾}\\
\textamh{17.\  } & \mytextarabic{يَطُوفُ عَلَيْهِمْ وِلْدَٟنٌۭ مُّخَلَّدُونَ ﴿١٧﴾}\\
\textamh{18.\  } & \mytextarabic{بِأَكْوَابٍۢ وَأَبَارِيقَ وَكَأْسٍۢ مِّن مَّعِينٍۢ ﴿١٨﴾}\\
\textamh{19.\  } & \mytextarabic{لَّا يُصَدَّعُونَ عَنْهَا وَلَا يُنزِفُونَ ﴿١٩﴾}\\
\textamh{20.\  } & \mytextarabic{وَفَـٰكِهَةٍۢ مِّمَّا يَتَخَيَّرُونَ ﴿٢٠﴾}\\
\textamh{21.\  } & \mytextarabic{وَلَحْمِ طَيْرٍۢ مِّمَّا يَشْتَهُونَ ﴿٢١﴾}\\
\textamh{22.\  } & \mytextarabic{وَحُورٌ عِينٌۭ ﴿٢٢﴾}\\
\textamh{23.\  } & \mytextarabic{كَأَمْثَـٰلِ ٱللُّؤْلُؤِ ٱلْمَكْنُونِ ﴿٢٣﴾}\\
\textamh{24.\  } & \mytextarabic{جَزَآءًۢ بِمَا كَانُوا۟ يَعْمَلُونَ ﴿٢٤﴾}\\
\textamh{25.\  } & \mytextarabic{لَا يَسْمَعُونَ فِيهَا لَغْوًۭا وَلَا تَأْثِيمًا ﴿٢٥﴾}\\
\textamh{26.\  } & \mytextarabic{إِلَّا قِيلًۭا سَلَـٰمًۭا سَلَـٰمًۭا ﴿٢٦﴾}\\
\textamh{27.\  } & \mytextarabic{وَأَصْحَـٰبُ ٱلْيَمِينِ مَآ أَصْحَـٰبُ ٱلْيَمِينِ ﴿٢٧﴾}\\
\textamh{28.\  } & \mytextarabic{فِى سِدْرٍۢ مَّخْضُودٍۢ ﴿٢٨﴾}\\
\textamh{29.\  } & \mytextarabic{وَطَلْحٍۢ مَّنضُودٍۢ ﴿٢٩﴾}\\
\textamh{30.\  } & \mytextarabic{وَظِلٍّۢ مَّمْدُودٍۢ ﴿٣٠﴾}\\
\textamh{31.\  } & \mytextarabic{وَمَآءٍۢ مَّسْكُوبٍۢ ﴿٣١﴾}\\
\textamh{32.\  } & \mytextarabic{وَفَـٰكِهَةٍۢ كَثِيرَةٍۢ ﴿٣٢﴾}\\
\textamh{33.\  } & \mytextarabic{لَّا مَقْطُوعَةٍۢ وَلَا مَمْنُوعَةٍۢ ﴿٣٣﴾}\\
\textamh{34.\  } & \mytextarabic{وَفُرُشٍۢ مَّرْفُوعَةٍ ﴿٣٤﴾}\\
\textamh{35.\  } & \mytextarabic{إِنَّآ أَنشَأْنَـٰهُنَّ إِنشَآءًۭ ﴿٣٥﴾}\\
\textamh{36.\  } & \mytextarabic{فَجَعَلْنَـٰهُنَّ أَبْكَارًا ﴿٣٦﴾}\\
\textamh{37.\  } & \mytextarabic{عُرُبًا أَتْرَابًۭا ﴿٣٧﴾}\\
\textamh{38.\  } & \mytextarabic{لِّأَصْحَـٰبِ ٱلْيَمِينِ ﴿٣٨﴾}\\
\textamh{39.\  } & \mytextarabic{ثُلَّةٌۭ مِّنَ ٱلْأَوَّلِينَ ﴿٣٩﴾}\\
\textamh{40.\  } & \mytextarabic{وَثُلَّةٌۭ مِّنَ ٱلْءَاخِرِينَ ﴿٤٠﴾}\\
\textamh{41.\  } & \mytextarabic{وَأَصْحَـٰبُ ٱلشِّمَالِ مَآ أَصْحَـٰبُ ٱلشِّمَالِ ﴿٤١﴾}\\
\textamh{42.\  } & \mytextarabic{فِى سَمُومٍۢ وَحَمِيمٍۢ ﴿٤٢﴾}\\
\textamh{43.\  } & \mytextarabic{وَظِلٍّۢ مِّن يَحْمُومٍۢ ﴿٤٣﴾}\\
\textamh{44.\  } & \mytextarabic{لَّا بَارِدٍۢ وَلَا كَرِيمٍ ﴿٤٤﴾}\\
\textamh{45.\  } & \mytextarabic{إِنَّهُمْ كَانُوا۟ قَبْلَ ذَٟلِكَ مُتْرَفِينَ ﴿٤٥﴾}\\
\textamh{46.\  } & \mytextarabic{وَكَانُوا۟ يُصِرُّونَ عَلَى ٱلْحِنثِ ٱلْعَظِيمِ ﴿٤٦﴾}\\
\textamh{47.\  } & \mytextarabic{وَكَانُوا۟ يَقُولُونَ أَئِذَا مِتْنَا وَكُنَّا تُرَابًۭا وَعِظَـٰمًا أَءِنَّا لَمَبْعُوثُونَ ﴿٤٧﴾}\\
\textamh{48.\  } & \mytextarabic{أَوَءَابَآؤُنَا ٱلْأَوَّلُونَ ﴿٤٨﴾}\\
\textamh{49.\  } & \mytextarabic{قُلْ إِنَّ ٱلْأَوَّلِينَ وَٱلْءَاخِرِينَ ﴿٤٩﴾}\\
\textamh{50.\  } & \mytextarabic{لَمَجْمُوعُونَ إِلَىٰ مِيقَـٰتِ يَوْمٍۢ مَّعْلُومٍۢ ﴿٥٠﴾}\\
\textamh{51.\  } & \mytextarabic{ثُمَّ إِنَّكُمْ أَيُّهَا ٱلضَّآلُّونَ ٱلْمُكَذِّبُونَ ﴿٥١﴾}\\
\textamh{52.\  } & \mytextarabic{لَءَاكِلُونَ مِن شَجَرٍۢ مِّن زَقُّومٍۢ ﴿٥٢﴾}\\
\textamh{53.\  } & \mytextarabic{فَمَالِـُٔونَ مِنْهَا ٱلْبُطُونَ ﴿٥٣﴾}\\
\textamh{54.\  } & \mytextarabic{فَشَـٰرِبُونَ عَلَيْهِ مِنَ ٱلْحَمِيمِ ﴿٥٤﴾}\\
\textamh{55.\  } & \mytextarabic{فَشَـٰرِبُونَ شُرْبَ ٱلْهِيمِ ﴿٥٥﴾}\\
\textamh{56.\  } & \mytextarabic{هَـٰذَا نُزُلُهُمْ يَوْمَ ٱلدِّينِ ﴿٥٦﴾}\\
\textamh{57.\  } & \mytextarabic{نَحْنُ خَلَقْنَـٰكُمْ فَلَوْلَا تُصَدِّقُونَ ﴿٥٧﴾}\\
\textamh{58.\  } & \mytextarabic{أَفَرَءَيْتُم مَّا تُمْنُونَ ﴿٥٨﴾}\\
\textamh{59.\  } & \mytextarabic{ءَأَنتُمْ تَخْلُقُونَهُۥٓ أَمْ نَحْنُ ٱلْخَـٰلِقُونَ ﴿٥٩﴾}\\
\textamh{60.\  } & \mytextarabic{نَحْنُ قَدَّرْنَا بَيْنَكُمُ ٱلْمَوْتَ وَمَا نَحْنُ بِمَسْبُوقِينَ ﴿٦٠﴾}\\
\textamh{61.\  } & \mytextarabic{عَلَىٰٓ أَن نُّبَدِّلَ أَمْثَـٰلَكُمْ وَنُنشِئَكُمْ فِى مَا لَا تَعْلَمُونَ ﴿٦١﴾}\\
\textamh{62.\  } & \mytextarabic{وَلَقَدْ عَلِمْتُمُ ٱلنَّشْأَةَ ٱلْأُولَىٰ فَلَوْلَا تَذَكَّرُونَ ﴿٦٢﴾}\\
\textamh{63.\  } & \mytextarabic{أَفَرَءَيْتُم مَّا تَحْرُثُونَ ﴿٦٣﴾}\\
\textamh{64.\  } & \mytextarabic{ءَأَنتُمْ تَزْرَعُونَهُۥٓ أَمْ نَحْنُ ٱلزَّٰرِعُونَ ﴿٦٤﴾}\\
\textamh{65.\  } & \mytextarabic{لَوْ نَشَآءُ لَجَعَلْنَـٰهُ حُطَٰمًۭا فَظَلْتُمْ تَفَكَّهُونَ ﴿٦٥﴾}\\
\textamh{66.\  } & \mytextarabic{إِنَّا لَمُغْرَمُونَ ﴿٦٦﴾}\\
\textamh{67.\  } & \mytextarabic{بَلْ نَحْنُ مَحْرُومُونَ ﴿٦٧﴾}\\
\textamh{68.\  } & \mytextarabic{أَفَرَءَيْتُمُ ٱلْمَآءَ ٱلَّذِى تَشْرَبُونَ ﴿٦٨﴾}\\
\textamh{69.\  } & \mytextarabic{ءَأَنتُمْ أَنزَلْتُمُوهُ مِنَ ٱلْمُزْنِ أَمْ نَحْنُ ٱلْمُنزِلُونَ ﴿٦٩﴾}\\
\textamh{70.\  } & \mytextarabic{لَوْ نَشَآءُ جَعَلْنَـٰهُ أُجَاجًۭا فَلَوْلَا تَشْكُرُونَ ﴿٧٠﴾}\\
\textamh{71.\  } & \mytextarabic{أَفَرَءَيْتُمُ ٱلنَّارَ ٱلَّتِى تُورُونَ ﴿٧١﴾}\\
\textamh{72.\  } & \mytextarabic{ءَأَنتُمْ أَنشَأْتُمْ شَجَرَتَهَآ أَمْ نَحْنُ ٱلْمُنشِـُٔونَ ﴿٧٢﴾}\\
\textamh{73.\  } & \mytextarabic{نَحْنُ جَعَلْنَـٰهَا تَذْكِرَةًۭ وَمَتَـٰعًۭا لِّلْمُقْوِينَ ﴿٧٣﴾}\\
\textamh{74.\  } & \mytextarabic{فَسَبِّحْ بِٱسْمِ رَبِّكَ ٱلْعَظِيمِ ﴿٧٤﴾}\\
\textamh{75.\  } & \mytextarabic{۞ فَلَآ أُقْسِمُ بِمَوَٟقِعِ ٱلنُّجُومِ ﴿٧٥﴾}\\
\textamh{76.\  } & \mytextarabic{وَإِنَّهُۥ لَقَسَمٌۭ لَّوْ تَعْلَمُونَ عَظِيمٌ ﴿٧٦﴾}\\
\textamh{77.\  } & \mytextarabic{إِنَّهُۥ لَقُرْءَانٌۭ كَرِيمٌۭ ﴿٧٧﴾}\\
\textamh{78.\  } & \mytextarabic{فِى كِتَـٰبٍۢ مَّكْنُونٍۢ ﴿٧٨﴾}\\
\textamh{79.\  } & \mytextarabic{لَّا يَمَسُّهُۥٓ إِلَّا ٱلْمُطَهَّرُونَ ﴿٧٩﴾}\\
\textamh{80.\  } & \mytextarabic{تَنزِيلٌۭ مِّن رَّبِّ ٱلْعَـٰلَمِينَ ﴿٨٠﴾}\\
\textamh{81.\  } & \mytextarabic{أَفَبِهَـٰذَا ٱلْحَدِيثِ أَنتُم مُّدْهِنُونَ ﴿٨١﴾}\\
\textamh{82.\  } & \mytextarabic{وَتَجْعَلُونَ رِزْقَكُمْ أَنَّكُمْ تُكَذِّبُونَ ﴿٨٢﴾}\\
\textamh{83.\  } & \mytextarabic{فَلَوْلَآ إِذَا بَلَغَتِ ٱلْحُلْقُومَ ﴿٨٣﴾}\\
\textamh{84.\  } & \mytextarabic{وَأَنتُمْ حِينَئِذٍۢ تَنظُرُونَ ﴿٨٤﴾}\\
\textamh{85.\  } & \mytextarabic{وَنَحْنُ أَقْرَبُ إِلَيْهِ مِنكُمْ وَلَـٰكِن لَّا تُبْصِرُونَ ﴿٨٥﴾}\\
\textamh{86.\  } & \mytextarabic{فَلَوْلَآ إِن كُنتُمْ غَيْرَ مَدِينِينَ ﴿٨٦﴾}\\
\textamh{87.\  } & \mytextarabic{تَرْجِعُونَهَآ إِن كُنتُمْ صَـٰدِقِينَ ﴿٨٧﴾}\\
\textamh{88.\  } & \mytextarabic{فَأَمَّآ إِن كَانَ مِنَ ٱلْمُقَرَّبِينَ ﴿٨٨﴾}\\
\textamh{89.\  } & \mytextarabic{فَرَوْحٌۭ وَرَيْحَانٌۭ وَجَنَّتُ نَعِيمٍۢ ﴿٨٩﴾}\\
\textamh{90.\  } & \mytextarabic{وَأَمَّآ إِن كَانَ مِنْ أَصْحَـٰبِ ٱلْيَمِينِ ﴿٩٠﴾}\\
\textamh{91.\  } & \mytextarabic{فَسَلَـٰمٌۭ لَّكَ مِنْ أَصْحَـٰبِ ٱلْيَمِينِ ﴿٩١﴾}\\
\textamh{92.\  } & \mytextarabic{وَأَمَّآ إِن كَانَ مِنَ ٱلْمُكَذِّبِينَ ٱلضَّآلِّينَ ﴿٩٢﴾}\\
\textamh{93.\  } & \mytextarabic{فَنُزُلٌۭ مِّنْ حَمِيمٍۢ ﴿٩٣﴾}\\
\textamh{94.\  } & \mytextarabic{وَتَصْلِيَةُ جَحِيمٍ ﴿٩٤﴾}\\
\textamh{95.\  } & \mytextarabic{إِنَّ هَـٰذَا لَهُوَ حَقُّ ٱلْيَقِينِ ﴿٩٥﴾}\\
\textamh{96.\  } & \mytextarabic{فَسَبِّحْ بِٱسْمِ رَبِّكَ ٱلْعَظِيمِ ﴿٩٦﴾}\\
\end{longtable}
\clearpage