%% License: BSD style (Berkley) (i.e. Put the Copyright owner's name always)
%% Writer and Copyright (to): Bewketu(Bilal) Tadilo (2016-17)
\centering\section{\LR{\textamharic{ሱራቱ አልሙልክ -}  \RL{سوره  الملك}}}
\begin{longtable}{%
  @{}
    p{.5\textwidth}
  @{~~~~~~~~~~~~~}
    p{.5\textwidth}
    @{}
}
\nopagebreak
\textamh{ቢስሚላሂ አራህመኒ ራሂይም } &  بِسْمِ ٱللَّهِ ٱلرَّحْمَـٰنِ ٱلرَّحِيمِ\\
\textamh{1.\  } &  تَبَٰرَكَ ٱلَّذِى بِيَدِهِ ٱلْمُلْكُ وَهُوَ عَلَىٰ كُلِّ شَىْءٍۢ قَدِيرٌ ﴿١﴾\\
\textamh{2.\  } & ٱلَّذِى خَلَقَ ٱلْمَوْتَ وَٱلْحَيَوٰةَ لِيَبْلُوَكُمْ أَيُّكُمْ أَحْسَنُ عَمَلًۭا ۚ وَهُوَ ٱلْعَزِيزُ ٱلْغَفُورُ ﴿٢﴾\\
\textamh{3.\  } & ٱلَّذِى خَلَقَ سَبْعَ سَمَـٰوَٟتٍۢ طِبَاقًۭا ۖ مَّا تَرَىٰ فِى خَلْقِ ٱلرَّحْمَـٰنِ مِن تَفَـٰوُتٍۢ ۖ فَٱرْجِعِ ٱلْبَصَرَ هَلْ تَرَىٰ مِن فُطُورٍۢ ﴿٣﴾\\
\textamh{4.\  } & ثُمَّ ٱرْجِعِ ٱلْبَصَرَ كَرَّتَيْنِ يَنقَلِبْ إِلَيْكَ ٱلْبَصَرُ خَاسِئًۭا وَهُوَ حَسِيرٌۭ ﴿٤﴾\\
\textamh{5.\  } & وَلَقَدْ زَيَّنَّا ٱلسَّمَآءَ ٱلدُّنْيَا بِمَصَـٰبِيحَ وَجَعَلْنَـٰهَا رُجُومًۭا لِّلشَّيَـٰطِينِ ۖ وَأَعْتَدْنَا لَهُمْ عَذَابَ ٱلسَّعِيرِ ﴿٥﴾\\
\textamh{6.\  } & وَلِلَّذِينَ كَفَرُوا۟ بِرَبِّهِمْ عَذَابُ جَهَنَّمَ ۖ وَبِئْسَ ٱلْمَصِيرُ ﴿٦﴾\\
\textamh{7.\  } & إِذَآ أُلْقُوا۟ فِيهَا سَمِعُوا۟ لَهَا شَهِيقًۭا وَهِىَ تَفُورُ ﴿٧﴾\\
\textamh{8.\  } & تَكَادُ تَمَيَّزُ مِنَ ٱلْغَيْظِ ۖ كُلَّمَآ أُلْقِىَ فِيهَا فَوْجٌۭ سَأَلَهُمْ خَزَنَتُهَآ أَلَمْ يَأْتِكُمْ نَذِيرٌۭ ﴿٨﴾\\
\textamh{9.\  } & قَالُوا۟ بَلَىٰ قَدْ جَآءَنَا نَذِيرٌۭ فَكَذَّبْنَا وَقُلْنَا مَا نَزَّلَ ٱللَّهُ مِن شَىْءٍ إِنْ أَنتُمْ إِلَّا فِى ضَلَـٰلٍۢ كَبِيرٍۢ ﴿٩﴾\\
\textamh{10.\  } & وَقَالُوا۟ لَوْ كُنَّا نَسْمَعُ أَوْ نَعْقِلُ مَا كُنَّا فِىٓ أَصْحَـٰبِ ٱلسَّعِيرِ ﴿١٠﴾\\
\textamh{11.\  } & فَٱعْتَرَفُوا۟ بِذَنۢبِهِمْ فَسُحْقًۭا لِّأَصْحَـٰبِ ٱلسَّعِيرِ ﴿١١﴾\\
\textamh{12.\  } & إِنَّ ٱلَّذِينَ يَخْشَوْنَ رَبَّهُم بِٱلْغَيْبِ لَهُم مَّغْفِرَةٌۭ وَأَجْرٌۭ كَبِيرٌۭ ﴿١٢﴾\\
\textamh{13.\  } & وَأَسِرُّوا۟ قَوْلَكُمْ أَوِ ٱجْهَرُوا۟ بِهِۦٓ ۖ إِنَّهُۥ عَلِيمٌۢ بِذَاتِ ٱلصُّدُورِ ﴿١٣﴾\\
\textamh{14.\  } & أَلَا يَعْلَمُ مَنْ خَلَقَ وَهُوَ ٱللَّطِيفُ ٱلْخَبِيرُ ﴿١٤﴾\\
\textamh{15.\  } & هُوَ ٱلَّذِى جَعَلَ لَكُمُ ٱلْأَرْضَ ذَلُولًۭا فَٱمْشُوا۟ فِى مَنَاكِبِهَا وَكُلُوا۟ مِن رِّزْقِهِۦ ۖ وَإِلَيْهِ ٱلنُّشُورُ ﴿١٥﴾\\
\textamh{16.\  } & ءَأَمِنتُم مَّن فِى ٱلسَّمَآءِ أَن يَخْسِفَ بِكُمُ ٱلْأَرْضَ فَإِذَا هِىَ تَمُورُ ﴿١٦﴾\\
\textamh{17.\  } & أَمْ أَمِنتُم مَّن فِى ٱلسَّمَآءِ أَن يُرْسِلَ عَلَيْكُمْ حَاصِبًۭا ۖ فَسَتَعْلَمُونَ كَيْفَ نَذِيرِ ﴿١٧﴾\\
\textamh{18.\  } & وَلَقَدْ كَذَّبَ ٱلَّذِينَ مِن قَبْلِهِمْ فَكَيْفَ كَانَ نَكِيرِ ﴿١٨﴾\\
\textamh{19.\  } & أَوَلَمْ يَرَوْا۟ إِلَى ٱلطَّيْرِ فَوْقَهُمْ صَـٰٓفَّٰتٍۢ وَيَقْبِضْنَ ۚ مَا يُمْسِكُهُنَّ إِلَّا ٱلرَّحْمَـٰنُ ۚ إِنَّهُۥ بِكُلِّ شَىْءٍۭ بَصِيرٌ ﴿١٩﴾\\
\textamh{20.\  } & أَمَّنْ هَـٰذَا ٱلَّذِى هُوَ جُندٌۭ لَّكُمْ يَنصُرُكُم مِّن دُونِ ٱلرَّحْمَـٰنِ ۚ إِنِ ٱلْكَـٰفِرُونَ إِلَّا فِى غُرُورٍ ﴿٢٠﴾\\
\textamh{21.\  } & أَمَّنْ هَـٰذَا ٱلَّذِى يَرْزُقُكُمْ إِنْ أَمْسَكَ رِزْقَهُۥ ۚ بَل لَّجُّوا۟ فِى عُتُوٍّۢ وَنُفُورٍ ﴿٢١﴾\\
\textamh{22.\  } & أَفَمَن يَمْشِى مُكِبًّا عَلَىٰ وَجْهِهِۦٓ أَهْدَىٰٓ أَمَّن يَمْشِى سَوِيًّا عَلَىٰ صِرَٰطٍۢ مُّسْتَقِيمٍۢ ﴿٢٢﴾\\
\textamh{23.\  } & قُلْ هُوَ ٱلَّذِىٓ أَنشَأَكُمْ وَجَعَلَ لَكُمُ ٱلسَّمْعَ وَٱلْأَبْصَـٰرَ وَٱلْأَفْـِٔدَةَ ۖ قَلِيلًۭا مَّا تَشْكُرُونَ ﴿٢٣﴾\\
\textamh{24.\  } & قُلْ هُوَ ٱلَّذِى ذَرَأَكُمْ فِى ٱلْأَرْضِ وَإِلَيْهِ تُحْشَرُونَ ﴿٢٤﴾\\
\textamh{25.\  } & وَيَقُولُونَ مَتَىٰ هَـٰذَا ٱلْوَعْدُ إِن كُنتُمْ صَـٰدِقِينَ ﴿٢٥﴾\\
\textamh{26.\  } & قُلْ إِنَّمَا ٱلْعِلْمُ عِندَ ٱللَّهِ وَإِنَّمَآ أَنَا۠ نَذِيرٌۭ مُّبِينٌۭ ﴿٢٦﴾\\
\textamh{27.\  } & فَلَمَّا رَأَوْهُ زُلْفَةًۭ سِيٓـَٔتْ وُجُوهُ ٱلَّذِينَ كَفَرُوا۟ وَقِيلَ هَـٰذَا ٱلَّذِى كُنتُم بِهِۦ تَدَّعُونَ ﴿٢٧﴾\\
\textamh{28.\  } & قُلْ أَرَءَيْتُمْ إِنْ أَهْلَكَنِىَ ٱللَّهُ وَمَن مَّعِىَ أَوْ رَحِمَنَا فَمَن يُجِيرُ ٱلْكَـٰفِرِينَ مِنْ عَذَابٍ أَلِيمٍۢ ﴿٢٨﴾\\
\textamh{29.\  } & قُلْ هُوَ ٱلرَّحْمَـٰنُ ءَامَنَّا بِهِۦ وَعَلَيْهِ تَوَكَّلْنَا ۖ فَسَتَعْلَمُونَ مَنْ هُوَ فِى ضَلَـٰلٍۢ مُّبِينٍۢ ﴿٢٩﴾\\
\textamh{30.\  } & قُلْ أَرَءَيْتُمْ إِنْ أَصْبَحَ مَآؤُكُمْ غَوْرًۭا فَمَن يَأْتِيكُم بِمَآءٍۢ مَّعِينٍۭ ﴿٣٠﴾\\
\end{longtable}
\clearpage