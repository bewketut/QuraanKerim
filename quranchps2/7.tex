%% License: BSD style (Berkley) (i.e. Put the Copyright owner's name always)
%% Writer and Copyright (to): Bewketu(Bilal) Tadilo (2016-17)
\centering\section{\LR{\textamharic{ሱራቱ አልአእራፍ -}  \RL{سوره  الأعراف}}}
\begin{longtable}{%
  @{}
    p{.5\textwidth}
  @{~~~~~~~~~~~~~}
    p{.5\textwidth}
    @{}
}
\nopagebreak
\textamh{ቢስሚላሂ አራህመኒ ራሂይም } &  بِسْمِ ٱللَّهِ ٱلرَّحْمَـٰنِ ٱلرَّحِيمِ\\
\textamh{1.\  } &  الٓمٓصٓ ﴿١﴾\\
\textamh{2.\  } & كِتَـٰبٌ أُنزِلَ إِلَيْكَ فَلَا يَكُن فِى صَدْرِكَ حَرَجٌۭ مِّنْهُ لِتُنذِرَ بِهِۦ وَذِكْرَىٰ لِلْمُؤْمِنِينَ ﴿٢﴾\\
\textamh{3.\  } & ٱتَّبِعُوا۟ مَآ أُنزِلَ إِلَيْكُم مِّن رَّبِّكُمْ وَلَا تَتَّبِعُوا۟ مِن دُونِهِۦٓ أَوْلِيَآءَ ۗ قَلِيلًۭا مَّا تَذَكَّرُونَ ﴿٣﴾\\
\textamh{4.\  } & وَكَم مِّن قَرْيَةٍ أَهْلَكْنَـٰهَا فَجَآءَهَا بَأْسُنَا بَيَـٰتًا أَوْ هُمْ قَآئِلُونَ ﴿٤﴾\\
\textamh{5.\  } & فَمَا كَانَ دَعْوَىٰهُمْ إِذْ جَآءَهُم بَأْسُنَآ إِلَّآ أَن قَالُوٓا۟ إِنَّا كُنَّا ظَـٰلِمِينَ ﴿٥﴾\\
\textamh{6.\  } & فَلَنَسْـَٔلَنَّ ٱلَّذِينَ أُرْسِلَ إِلَيْهِمْ وَلَنَسْـَٔلَنَّ ٱلْمُرْسَلِينَ ﴿٦﴾\\
\textamh{7.\  } & فَلَنَقُصَّنَّ عَلَيْهِم بِعِلْمٍۢ ۖ وَمَا كُنَّا غَآئِبِينَ ﴿٧﴾\\
\textamh{8.\  } & وَٱلْوَزْنُ يَوْمَئِذٍ ٱلْحَقُّ ۚ فَمَن ثَقُلَتْ مَوَٟزِينُهُۥ فَأُو۟لَـٰٓئِكَ هُمُ ٱلْمُفْلِحُونَ ﴿٨﴾\\
\textamh{9.\  } & وَمَنْ خَفَّتْ مَوَٟزِينُهُۥ فَأُو۟لَـٰٓئِكَ ٱلَّذِينَ خَسِرُوٓا۟ أَنفُسَهُم بِمَا كَانُوا۟ بِـَٔايَـٰتِنَا يَظْلِمُونَ ﴿٩﴾\\
\textamh{10.\  } & وَلَقَدْ مَكَّنَّـٰكُمْ فِى ٱلْأَرْضِ وَجَعَلْنَا لَكُمْ فِيهَا مَعَـٰيِشَ ۗ قَلِيلًۭا مَّا تَشْكُرُونَ ﴿١٠﴾\\
\textamh{11.\  } & وَلَقَدْ خَلَقْنَـٰكُمْ ثُمَّ صَوَّرْنَـٰكُمْ ثُمَّ قُلْنَا لِلْمَلَـٰٓئِكَةِ ٱسْجُدُوا۟ لِءَادَمَ فَسَجَدُوٓا۟ إِلَّآ إِبْلِيسَ لَمْ يَكُن مِّنَ ٱلسَّٰجِدِينَ ﴿١١﴾\\
\textamh{12.\  } & قَالَ مَا مَنَعَكَ أَلَّا تَسْجُدَ إِذْ أَمَرْتُكَ ۖ قَالَ أَنَا۠ خَيْرٌۭ مِّنْهُ خَلَقْتَنِى مِن نَّارٍۢ وَخَلَقْتَهُۥ مِن طِينٍۢ ﴿١٢﴾\\
\textamh{13.\  } & قَالَ فَٱهْبِطْ مِنْهَا فَمَا يَكُونُ لَكَ أَن تَتَكَبَّرَ فِيهَا فَٱخْرُجْ إِنَّكَ مِنَ ٱلصَّـٰغِرِينَ ﴿١٣﴾\\
\textamh{14.\  } & قَالَ أَنظِرْنِىٓ إِلَىٰ يَوْمِ يُبْعَثُونَ ﴿١٤﴾\\
\textamh{15.\  } & قَالَ إِنَّكَ مِنَ ٱلْمُنظَرِينَ ﴿١٥﴾\\
\textamh{16.\  } & قَالَ فَبِمَآ أَغْوَيْتَنِى لَأَقْعُدَنَّ لَهُمْ صِرَٰطَكَ ٱلْمُسْتَقِيمَ ﴿١٦﴾\\
\textamh{17.\  } & ثُمَّ لَءَاتِيَنَّهُم مِّنۢ بَيْنِ أَيْدِيهِمْ وَمِنْ خَلْفِهِمْ وَعَنْ أَيْمَـٰنِهِمْ وَعَن شَمَآئِلِهِمْ ۖ وَلَا تَجِدُ أَكْثَرَهُمْ شَـٰكِرِينَ ﴿١٧﴾\\
\textamh{18.\  } & قَالَ ٱخْرُجْ مِنْهَا مَذْءُومًۭا مَّدْحُورًۭا ۖ لَّمَن تَبِعَكَ مِنْهُمْ لَأَمْلَأَنَّ جَهَنَّمَ مِنكُمْ أَجْمَعِينَ ﴿١٨﴾\\
\textamh{19.\  } & وَيَـٰٓـَٔادَمُ ٱسْكُنْ أَنتَ وَزَوْجُكَ ٱلْجَنَّةَ فَكُلَا مِنْ حَيْثُ شِئْتُمَا وَلَا تَقْرَبَا هَـٰذِهِ ٱلشَّجَرَةَ فَتَكُونَا مِنَ ٱلظَّـٰلِمِينَ ﴿١٩﴾\\
\textamh{20.\  } & فَوَسْوَسَ لَهُمَا ٱلشَّيْطَٰنُ لِيُبْدِىَ لَهُمَا مَا وُۥرِىَ عَنْهُمَا مِن سَوْءَٰتِهِمَا وَقَالَ مَا نَهَىٰكُمَا رَبُّكُمَا عَنْ هَـٰذِهِ ٱلشَّجَرَةِ إِلَّآ أَن تَكُونَا مَلَكَيْنِ أَوْ تَكُونَا مِنَ ٱلْخَـٰلِدِينَ ﴿٢٠﴾\\
\textamh{21.\  } & وَقَاسَمَهُمَآ إِنِّى لَكُمَا لَمِنَ ٱلنَّـٰصِحِينَ ﴿٢١﴾\\
\textamh{22.\  } & فَدَلَّىٰهُمَا بِغُرُورٍۢ ۚ فَلَمَّا ذَاقَا ٱلشَّجَرَةَ بَدَتْ لَهُمَا سَوْءَٰتُهُمَا وَطَفِقَا يَخْصِفَانِ عَلَيْهِمَا مِن وَرَقِ ٱلْجَنَّةِ ۖ وَنَادَىٰهُمَا رَبُّهُمَآ أَلَمْ أَنْهَكُمَا عَن تِلْكُمَا ٱلشَّجَرَةِ وَأَقُل لَّكُمَآ إِنَّ ٱلشَّيْطَٰنَ لَكُمَا عَدُوٌّۭ مُّبِينٌۭ ﴿٢٢﴾\\
\textamh{23.\  } & قَالَا رَبَّنَا ظَلَمْنَآ أَنفُسَنَا وَإِن لَّمْ تَغْفِرْ لَنَا وَتَرْحَمْنَا لَنَكُونَنَّ مِنَ ٱلْخَـٰسِرِينَ ﴿٢٣﴾\\
\textamh{24.\  } & قَالَ ٱهْبِطُوا۟ بَعْضُكُمْ لِبَعْضٍ عَدُوٌّۭ ۖ وَلَكُمْ فِى ٱلْأَرْضِ مُسْتَقَرٌّۭ وَمَتَـٰعٌ إِلَىٰ حِينٍۢ ﴿٢٤﴾\\
\textamh{25.\  } & قَالَ فِيهَا تَحْيَوْنَ وَفِيهَا تَمُوتُونَ وَمِنْهَا تُخْرَجُونَ ﴿٢٥﴾\\
\textamh{26.\  } & يَـٰبَنِىٓ ءَادَمَ قَدْ أَنزَلْنَا عَلَيْكُمْ لِبَاسًۭا يُوَٟرِى سَوْءَٰتِكُمْ وَرِيشًۭا ۖ وَلِبَاسُ ٱلتَّقْوَىٰ ذَٟلِكَ خَيْرٌۭ ۚ ذَٟلِكَ مِنْ ءَايَـٰتِ ٱللَّهِ لَعَلَّهُمْ يَذَّكَّرُونَ ﴿٢٦﴾\\
\textamh{27.\  } & يَـٰبَنِىٓ ءَادَمَ لَا يَفْتِنَنَّكُمُ ٱلشَّيْطَٰنُ كَمَآ أَخْرَجَ أَبَوَيْكُم مِّنَ ٱلْجَنَّةِ يَنزِعُ عَنْهُمَا لِبَاسَهُمَا لِيُرِيَهُمَا سَوْءَٰتِهِمَآ ۗ إِنَّهُۥ يَرَىٰكُمْ هُوَ وَقَبِيلُهُۥ مِنْ حَيْثُ لَا تَرَوْنَهُمْ ۗ إِنَّا جَعَلْنَا ٱلشَّيَـٰطِينَ أَوْلِيَآءَ لِلَّذِينَ لَا يُؤْمِنُونَ ﴿٢٧﴾\\
\textamh{28.\  } & وَإِذَا فَعَلُوا۟ فَـٰحِشَةًۭ قَالُوا۟ وَجَدْنَا عَلَيْهَآ ءَابَآءَنَا وَٱللَّهُ أَمَرَنَا بِهَا ۗ قُلْ إِنَّ ٱللَّهَ لَا يَأْمُرُ بِٱلْفَحْشَآءِ ۖ أَتَقُولُونَ عَلَى ٱللَّهِ مَا لَا تَعْلَمُونَ ﴿٢٨﴾\\
\textamh{29.\  } & قُلْ أَمَرَ رَبِّى بِٱلْقِسْطِ ۖ وَأَقِيمُوا۟ وُجُوهَكُمْ عِندَ كُلِّ مَسْجِدٍۢ وَٱدْعُوهُ مُخْلِصِينَ لَهُ ٱلدِّينَ ۚ كَمَا بَدَأَكُمْ تَعُودُونَ ﴿٢٩﴾\\
\textamh{30.\  } & فَرِيقًا هَدَىٰ وَفَرِيقًا حَقَّ عَلَيْهِمُ ٱلضَّلَـٰلَةُ ۗ إِنَّهُمُ ٱتَّخَذُوا۟ ٱلشَّيَـٰطِينَ أَوْلِيَآءَ مِن دُونِ ٱللَّهِ وَيَحْسَبُونَ أَنَّهُم مُّهْتَدُونَ ﴿٣٠﴾\\
\textamh{31.\  } & ۞ يَـٰبَنِىٓ ءَادَمَ خُذُوا۟ زِينَتَكُمْ عِندَ كُلِّ مَسْجِدٍۢ وَكُلُوا۟ وَٱشْرَبُوا۟ وَلَا تُسْرِفُوٓا۟ ۚ إِنَّهُۥ لَا يُحِبُّ ٱلْمُسْرِفِينَ ﴿٣١﴾\\
\textamh{32.\  } & قُلْ مَنْ حَرَّمَ زِينَةَ ٱللَّهِ ٱلَّتِىٓ أَخْرَجَ لِعِبَادِهِۦ وَٱلطَّيِّبَٰتِ مِنَ ٱلرِّزْقِ ۚ قُلْ هِىَ لِلَّذِينَ ءَامَنُوا۟ فِى ٱلْحَيَوٰةِ ٱلدُّنْيَا خَالِصَةًۭ يَوْمَ ٱلْقِيَـٰمَةِ ۗ كَذَٟلِكَ نُفَصِّلُ ٱلْءَايَـٰتِ لِقَوْمٍۢ يَعْلَمُونَ ﴿٣٢﴾\\
\textamh{33.\  } & قُلْ إِنَّمَا حَرَّمَ رَبِّىَ ٱلْفَوَٟحِشَ مَا ظَهَرَ مِنْهَا وَمَا بَطَنَ وَٱلْإِثْمَ وَٱلْبَغْىَ بِغَيْرِ ٱلْحَقِّ وَأَن تُشْرِكُوا۟ بِٱللَّهِ مَا لَمْ يُنَزِّلْ بِهِۦ سُلْطَٰنًۭا وَأَن تَقُولُوا۟ عَلَى ٱللَّهِ مَا لَا تَعْلَمُونَ ﴿٣٣﴾\\
\textamh{34.\  } & وَلِكُلِّ أُمَّةٍ أَجَلٌۭ ۖ فَإِذَا جَآءَ أَجَلُهُمْ لَا يَسْتَأْخِرُونَ سَاعَةًۭ ۖ وَلَا يَسْتَقْدِمُونَ ﴿٣٤﴾\\
\textamh{35.\  } & يَـٰبَنِىٓ ءَادَمَ إِمَّا يَأْتِيَنَّكُمْ رُسُلٌۭ مِّنكُمْ يَقُصُّونَ عَلَيْكُمْ ءَايَـٰتِى ۙ فَمَنِ ٱتَّقَىٰ وَأَصْلَحَ فَلَا خَوْفٌ عَلَيْهِمْ وَلَا هُمْ يَحْزَنُونَ ﴿٣٥﴾\\
\textamh{36.\  } & وَٱلَّذِينَ كَذَّبُوا۟ بِـَٔايَـٰتِنَا وَٱسْتَكْبَرُوا۟ عَنْهَآ أُو۟لَـٰٓئِكَ أَصْحَـٰبُ ٱلنَّارِ ۖ هُمْ فِيهَا خَـٰلِدُونَ ﴿٣٦﴾\\
\textamh{37.\  } & فَمَنْ أَظْلَمُ مِمَّنِ ٱفْتَرَىٰ عَلَى ٱللَّهِ كَذِبًا أَوْ كَذَّبَ بِـَٔايَـٰتِهِۦٓ ۚ أُو۟لَـٰٓئِكَ يَنَالُهُمْ نَصِيبُهُم مِّنَ ٱلْكِتَـٰبِ ۖ حَتَّىٰٓ إِذَا جَآءَتْهُمْ رُسُلُنَا يَتَوَفَّوْنَهُمْ قَالُوٓا۟ أَيْنَ مَا كُنتُمْ تَدْعُونَ مِن دُونِ ٱللَّهِ ۖ قَالُوا۟ ضَلُّوا۟ عَنَّا وَشَهِدُوا۟ عَلَىٰٓ أَنفُسِهِمْ أَنَّهُمْ كَانُوا۟ كَـٰفِرِينَ ﴿٣٧﴾\\
\textamh{38.\  } & قَالَ ٱدْخُلُوا۟ فِىٓ أُمَمٍۢ قَدْ خَلَتْ مِن قَبْلِكُم مِّنَ ٱلْجِنِّ وَٱلْإِنسِ فِى ٱلنَّارِ ۖ كُلَّمَا دَخَلَتْ أُمَّةٌۭ لَّعَنَتْ أُخْتَهَا ۖ حَتَّىٰٓ إِذَا ٱدَّارَكُوا۟ فِيهَا جَمِيعًۭا قَالَتْ أُخْرَىٰهُمْ لِأُولَىٰهُمْ رَبَّنَا هَـٰٓؤُلَآءِ أَضَلُّونَا فَـَٔاتِهِمْ عَذَابًۭا ضِعْفًۭا مِّنَ ٱلنَّارِ ۖ قَالَ لِكُلٍّۢ ضِعْفٌۭ وَلَـٰكِن لَّا تَعْلَمُونَ ﴿٣٨﴾\\
\textamh{39.\  } & وَقَالَتْ أُولَىٰهُمْ لِأُخْرَىٰهُمْ فَمَا كَانَ لَكُمْ عَلَيْنَا مِن فَضْلٍۢ فَذُوقُوا۟ ٱلْعَذَابَ بِمَا كُنتُمْ تَكْسِبُونَ ﴿٣٩﴾\\
\textamh{40.\  } & إِنَّ ٱلَّذِينَ كَذَّبُوا۟ بِـَٔايَـٰتِنَا وَٱسْتَكْبَرُوا۟ عَنْهَا لَا تُفَتَّحُ لَهُمْ أَبْوَٟبُ ٱلسَّمَآءِ وَلَا يَدْخُلُونَ ٱلْجَنَّةَ حَتَّىٰ يَلِجَ ٱلْجَمَلُ فِى سَمِّ ٱلْخِيَاطِ ۚ وَكَذَٟلِكَ نَجْزِى ٱلْمُجْرِمِينَ ﴿٤٠﴾\\
\textamh{41.\  } & لَهُم مِّن جَهَنَّمَ مِهَادٌۭ وَمِن فَوْقِهِمْ غَوَاشٍۢ ۚ وَكَذَٟلِكَ نَجْزِى ٱلظَّـٰلِمِينَ ﴿٤١﴾\\
\textamh{42.\  } & وَٱلَّذِينَ ءَامَنُوا۟ وَعَمِلُوا۟ ٱلصَّـٰلِحَـٰتِ لَا نُكَلِّفُ نَفْسًا إِلَّا وُسْعَهَآ أُو۟لَـٰٓئِكَ أَصْحَـٰبُ ٱلْجَنَّةِ ۖ هُمْ فِيهَا خَـٰلِدُونَ ﴿٤٢﴾\\
\textamh{43.\  } & وَنَزَعْنَا مَا فِى صُدُورِهِم مِّنْ غِلٍّۢ تَجْرِى مِن تَحْتِهِمُ ٱلْأَنْهَـٰرُ ۖ وَقَالُوا۟ ٱلْحَمْدُ لِلَّهِ ٱلَّذِى هَدَىٰنَا لِهَـٰذَا وَمَا كُنَّا لِنَهْتَدِىَ لَوْلَآ أَنْ هَدَىٰنَا ٱللَّهُ ۖ لَقَدْ جَآءَتْ رُسُلُ رَبِّنَا بِٱلْحَقِّ ۖ وَنُودُوٓا۟ أَن تِلْكُمُ ٱلْجَنَّةُ أُورِثْتُمُوهَا بِمَا كُنتُمْ تَعْمَلُونَ ﴿٤٣﴾\\
\textamh{44.\  } & وَنَادَىٰٓ أَصْحَـٰبُ ٱلْجَنَّةِ أَصْحَـٰبَ ٱلنَّارِ أَن قَدْ وَجَدْنَا مَا وَعَدَنَا رَبُّنَا حَقًّۭا فَهَلْ وَجَدتُّم مَّا وَعَدَ رَبُّكُمْ حَقًّۭا ۖ قَالُوا۟ نَعَمْ ۚ فَأَذَّنَ مُؤَذِّنٌۢ بَيْنَهُمْ أَن لَّعْنَةُ ٱللَّهِ عَلَى ٱلظَّـٰلِمِينَ ﴿٤٤﴾\\
\textamh{45.\  } & ٱلَّذِينَ يَصُدُّونَ عَن سَبِيلِ ٱللَّهِ وَيَبْغُونَهَا عِوَجًۭا وَهُم بِٱلْءَاخِرَةِ كَـٰفِرُونَ ﴿٤٥﴾\\
\textamh{46.\  } & وَبَيْنَهُمَا حِجَابٌۭ ۚ وَعَلَى ٱلْأَعْرَافِ رِجَالٌۭ يَعْرِفُونَ كُلًّۢا بِسِيمَىٰهُمْ ۚ وَنَادَوْا۟ أَصْحَـٰبَ ٱلْجَنَّةِ أَن سَلَـٰمٌ عَلَيْكُمْ ۚ لَمْ يَدْخُلُوهَا وَهُمْ يَطْمَعُونَ ﴿٤٦﴾\\
\textamh{47.\  } & ۞ وَإِذَا صُرِفَتْ أَبْصَـٰرُهُمْ تِلْقَآءَ أَصْحَـٰبِ ٱلنَّارِ قَالُوا۟ رَبَّنَا لَا تَجْعَلْنَا مَعَ ٱلْقَوْمِ ٱلظَّـٰلِمِينَ ﴿٤٧﴾\\
\textamh{48.\  } & وَنَادَىٰٓ أَصْحَـٰبُ ٱلْأَعْرَافِ رِجَالًۭا يَعْرِفُونَهُم بِسِيمَىٰهُمْ قَالُوا۟ مَآ أَغْنَىٰ عَنكُمْ جَمْعُكُمْ وَمَا كُنتُمْ تَسْتَكْبِرُونَ ﴿٤٨﴾\\
\textamh{49.\  } & أَهَـٰٓؤُلَآءِ ٱلَّذِينَ أَقْسَمْتُمْ لَا يَنَالُهُمُ ٱللَّهُ بِرَحْمَةٍ ۚ ٱدْخُلُوا۟ ٱلْجَنَّةَ لَا خَوْفٌ عَلَيْكُمْ وَلَآ أَنتُمْ تَحْزَنُونَ ﴿٤٩﴾\\
\textamh{50.\  } & وَنَادَىٰٓ أَصْحَـٰبُ ٱلنَّارِ أَصْحَـٰبَ ٱلْجَنَّةِ أَنْ أَفِيضُوا۟ عَلَيْنَا مِنَ ٱلْمَآءِ أَوْ مِمَّا رَزَقَكُمُ ٱللَّهُ ۚ قَالُوٓا۟ إِنَّ ٱللَّهَ حَرَّمَهُمَا عَلَى ٱلْكَـٰفِرِينَ ﴿٥٠﴾\\
\textamh{51.\  } & ٱلَّذِينَ ٱتَّخَذُوا۟ دِينَهُمْ لَهْوًۭا وَلَعِبًۭا وَغَرَّتْهُمُ ٱلْحَيَوٰةُ ٱلدُّنْيَا ۚ فَٱلْيَوْمَ نَنسَىٰهُمْ كَمَا نَسُوا۟ لِقَآءَ يَوْمِهِمْ هَـٰذَا وَمَا كَانُوا۟ بِـَٔايَـٰتِنَا يَجْحَدُونَ ﴿٥١﴾\\
\textamh{52.\  } & وَلَقَدْ جِئْنَـٰهُم بِكِتَـٰبٍۢ فَصَّلْنَـٰهُ عَلَىٰ عِلْمٍ هُدًۭى وَرَحْمَةًۭ لِّقَوْمٍۢ يُؤْمِنُونَ ﴿٥٢﴾\\
\textamh{53.\  } & هَلْ يَنظُرُونَ إِلَّا تَأْوِيلَهُۥ ۚ يَوْمَ يَأْتِى تَأْوِيلُهُۥ يَقُولُ ٱلَّذِينَ نَسُوهُ مِن قَبْلُ قَدْ جَآءَتْ رُسُلُ رَبِّنَا بِٱلْحَقِّ فَهَل لَّنَا مِن شُفَعَآءَ فَيَشْفَعُوا۟ لَنَآ أَوْ نُرَدُّ فَنَعْمَلَ غَيْرَ ٱلَّذِى كُنَّا نَعْمَلُ ۚ قَدْ خَسِرُوٓا۟ أَنفُسَهُمْ وَضَلَّ عَنْهُم مَّا كَانُوا۟ يَفْتَرُونَ ﴿٥٣﴾\\
\textamh{54.\  } & إِنَّ رَبَّكُمُ ٱللَّهُ ٱلَّذِى خَلَقَ ٱلسَّمَـٰوَٟتِ وَٱلْأَرْضَ فِى سِتَّةِ أَيَّامٍۢ ثُمَّ ٱسْتَوَىٰ عَلَى ٱلْعَرْشِ يُغْشِى ٱلَّيْلَ ٱلنَّهَارَ يَطْلُبُهُۥ حَثِيثًۭا وَٱلشَّمْسَ وَٱلْقَمَرَ وَٱلنُّجُومَ مُسَخَّرَٰتٍۭ بِأَمْرِهِۦٓ ۗ أَلَا لَهُ ٱلْخَلْقُ وَٱلْأَمْرُ ۗ تَبَارَكَ ٱللَّهُ رَبُّ ٱلْعَـٰلَمِينَ ﴿٥٤﴾\\
\textamh{55.\  } & ٱدْعُوا۟ رَبَّكُمْ تَضَرُّعًۭا وَخُفْيَةً ۚ إِنَّهُۥ لَا يُحِبُّ ٱلْمُعْتَدِينَ ﴿٥٥﴾\\
\textamh{56.\  } & وَلَا تُفْسِدُوا۟ فِى ٱلْأَرْضِ بَعْدَ إِصْلَـٰحِهَا وَٱدْعُوهُ خَوْفًۭا وَطَمَعًا ۚ إِنَّ رَحْمَتَ ٱللَّهِ قَرِيبٌۭ مِّنَ ٱلْمُحْسِنِينَ ﴿٥٦﴾\\
\textamh{57.\  } & وَهُوَ ٱلَّذِى يُرْسِلُ ٱلرِّيَـٰحَ بُشْرًۢا بَيْنَ يَدَىْ رَحْمَتِهِۦ ۖ حَتَّىٰٓ إِذَآ أَقَلَّتْ سَحَابًۭا ثِقَالًۭا سُقْنَـٰهُ لِبَلَدٍۢ مَّيِّتٍۢ فَأَنزَلْنَا بِهِ ٱلْمَآءَ فَأَخْرَجْنَا بِهِۦ مِن كُلِّ ٱلثَّمَرَٰتِ ۚ كَذَٟلِكَ نُخْرِجُ ٱلْمَوْتَىٰ لَعَلَّكُمْ تَذَكَّرُونَ ﴿٥٧﴾\\
\textamh{58.\  } & وَٱلْبَلَدُ ٱلطَّيِّبُ يَخْرُجُ نَبَاتُهُۥ بِإِذْنِ رَبِّهِۦ ۖ وَٱلَّذِى خَبُثَ لَا يَخْرُجُ إِلَّا نَكِدًۭا ۚ كَذَٟلِكَ نُصَرِّفُ ٱلْءَايَـٰتِ لِقَوْمٍۢ يَشْكُرُونَ ﴿٥٨﴾\\
\textamh{59.\  } & لَقَدْ أَرْسَلْنَا نُوحًا إِلَىٰ قَوْمِهِۦ فَقَالَ يَـٰقَوْمِ ٱعْبُدُوا۟ ٱللَّهَ مَا لَكُم مِّنْ إِلَـٰهٍ غَيْرُهُۥٓ إِنِّىٓ أَخَافُ عَلَيْكُمْ عَذَابَ يَوْمٍ عَظِيمٍۢ ﴿٥٩﴾\\
\textamh{60.\  } & قَالَ ٱلْمَلَأُ مِن قَوْمِهِۦٓ إِنَّا لَنَرَىٰكَ فِى ضَلَـٰلٍۢ مُّبِينٍۢ ﴿٦٠﴾\\
\textamh{61.\  } & قَالَ يَـٰقَوْمِ لَيْسَ بِى ضَلَـٰلَةٌۭ وَلَـٰكِنِّى رَسُولٌۭ مِّن رَّبِّ ٱلْعَـٰلَمِينَ ﴿٦١﴾\\
\textamh{62.\  } & أُبَلِّغُكُمْ رِسَـٰلَـٰتِ رَبِّى وَأَنصَحُ لَكُمْ وَأَعْلَمُ مِنَ ٱللَّهِ مَا لَا تَعْلَمُونَ ﴿٦٢﴾\\
\textamh{63.\  } & أَوَعَجِبْتُمْ أَن جَآءَكُمْ ذِكْرٌۭ مِّن رَّبِّكُمْ عَلَىٰ رَجُلٍۢ مِّنكُمْ لِيُنذِرَكُمْ وَلِتَتَّقُوا۟ وَلَعَلَّكُمْ تُرْحَمُونَ ﴿٦٣﴾\\
\textamh{64.\  } & فَكَذَّبُوهُ فَأَنجَيْنَـٰهُ وَٱلَّذِينَ مَعَهُۥ فِى ٱلْفُلْكِ وَأَغْرَقْنَا ٱلَّذِينَ كَذَّبُوا۟ بِـَٔايَـٰتِنَآ ۚ إِنَّهُمْ كَانُوا۟ قَوْمًا عَمِينَ ﴿٦٤﴾\\
\textamh{65.\  } & ۞ وَإِلَىٰ عَادٍ أَخَاهُمْ هُودًۭا ۗ قَالَ يَـٰقَوْمِ ٱعْبُدُوا۟ ٱللَّهَ مَا لَكُم مِّنْ إِلَـٰهٍ غَيْرُهُۥٓ ۚ أَفَلَا تَتَّقُونَ ﴿٦٥﴾\\
\textamh{66.\  } & قَالَ ٱلْمَلَأُ ٱلَّذِينَ كَفَرُوا۟ مِن قَوْمِهِۦٓ إِنَّا لَنَرَىٰكَ فِى سَفَاهَةٍۢ وَإِنَّا لَنَظُنُّكَ مِنَ ٱلْكَـٰذِبِينَ ﴿٦٦﴾\\
\textamh{67.\  } & قَالَ يَـٰقَوْمِ لَيْسَ بِى سَفَاهَةٌۭ وَلَـٰكِنِّى رَسُولٌۭ مِّن رَّبِّ ٱلْعَـٰلَمِينَ ﴿٦٧﴾\\
\textamh{68.\  } & أُبَلِّغُكُمْ رِسَـٰلَـٰتِ رَبِّى وَأَنَا۠ لَكُمْ نَاصِحٌ أَمِينٌ ﴿٦٨﴾\\
\textamh{69.\  } & أَوَعَجِبْتُمْ أَن جَآءَكُمْ ذِكْرٌۭ مِّن رَّبِّكُمْ عَلَىٰ رَجُلٍۢ مِّنكُمْ لِيُنذِرَكُمْ ۚ وَٱذْكُرُوٓا۟ إِذْ جَعَلَكُمْ خُلَفَآءَ مِنۢ بَعْدِ قَوْمِ نُوحٍۢ وَزَادَكُمْ فِى ٱلْخَلْقِ بَصْۜطَةًۭ ۖ فَٱذْكُرُوٓا۟ ءَالَآءَ ٱللَّهِ لَعَلَّكُمْ تُفْلِحُونَ ﴿٦٩﴾\\
\textamh{70.\  } & قَالُوٓا۟ أَجِئْتَنَا لِنَعْبُدَ ٱللَّهَ وَحْدَهُۥ وَنَذَرَ مَا كَانَ يَعْبُدُ ءَابَآؤُنَا ۖ فَأْتِنَا بِمَا تَعِدُنَآ إِن كُنتَ مِنَ ٱلصَّـٰدِقِينَ ﴿٧٠﴾\\
\textamh{71.\  } & قَالَ قَدْ وَقَعَ عَلَيْكُم مِّن رَّبِّكُمْ رِجْسٌۭ وَغَضَبٌ ۖ أَتُجَٰدِلُونَنِى فِىٓ أَسْمَآءٍۢ سَمَّيْتُمُوهَآ أَنتُمْ وَءَابَآؤُكُم مَّا نَزَّلَ ٱللَّهُ بِهَا مِن سُلْطَٰنٍۢ ۚ فَٱنتَظِرُوٓا۟ إِنِّى مَعَكُم مِّنَ ٱلْمُنتَظِرِينَ ﴿٧١﴾\\
\textamh{72.\  } & فَأَنجَيْنَـٰهُ وَٱلَّذِينَ مَعَهُۥ بِرَحْمَةٍۢ مِّنَّا وَقَطَعْنَا دَابِرَ ٱلَّذِينَ كَذَّبُوا۟ بِـَٔايَـٰتِنَا ۖ وَمَا كَانُوا۟ مُؤْمِنِينَ ﴿٧٢﴾\\
\textamh{73.\  } & وَإِلَىٰ ثَمُودَ أَخَاهُمْ صَـٰلِحًۭا ۗ قَالَ يَـٰقَوْمِ ٱعْبُدُوا۟ ٱللَّهَ مَا لَكُم مِّنْ إِلَـٰهٍ غَيْرُهُۥ ۖ قَدْ جَآءَتْكُم بَيِّنَةٌۭ مِّن رَّبِّكُمْ ۖ هَـٰذِهِۦ نَاقَةُ ٱللَّهِ لَكُمْ ءَايَةًۭ ۖ فَذَرُوهَا تَأْكُلْ فِىٓ أَرْضِ ٱللَّهِ ۖ وَلَا تَمَسُّوهَا بِسُوٓءٍۢ فَيَأْخُذَكُمْ عَذَابٌ أَلِيمٌۭ ﴿٧٣﴾\\
\textamh{74.\  } & وَٱذْكُرُوٓا۟ إِذْ جَعَلَكُمْ خُلَفَآءَ مِنۢ بَعْدِ عَادٍۢ وَبَوَّأَكُمْ فِى ٱلْأَرْضِ تَتَّخِذُونَ مِن سُهُولِهَا قُصُورًۭا وَتَنْحِتُونَ ٱلْجِبَالَ بُيُوتًۭا ۖ فَٱذْكُرُوٓا۟ ءَالَآءَ ٱللَّهِ وَلَا تَعْثَوْا۟ فِى ٱلْأَرْضِ مُفْسِدِينَ ﴿٧٤﴾\\
\textamh{75.\  } & قَالَ ٱلْمَلَأُ ٱلَّذِينَ ٱسْتَكْبَرُوا۟ مِن قَوْمِهِۦ لِلَّذِينَ ٱسْتُضْعِفُوا۟ لِمَنْ ءَامَنَ مِنْهُمْ أَتَعْلَمُونَ أَنَّ صَـٰلِحًۭا مُّرْسَلٌۭ مِّن رَّبِّهِۦ ۚ قَالُوٓا۟ إِنَّا بِمَآ أُرْسِلَ بِهِۦ مُؤْمِنُونَ ﴿٧٥﴾\\
\textamh{76.\  } & قَالَ ٱلَّذِينَ ٱسْتَكْبَرُوٓا۟ إِنَّا بِٱلَّذِىٓ ءَامَنتُم بِهِۦ كَـٰفِرُونَ ﴿٧٦﴾\\
\textamh{77.\  } & فَعَقَرُوا۟ ٱلنَّاقَةَ وَعَتَوْا۟ عَنْ أَمْرِ رَبِّهِمْ وَقَالُوا۟ يَـٰصَـٰلِحُ ٱئْتِنَا بِمَا تَعِدُنَآ إِن كُنتَ مِنَ ٱلْمُرْسَلِينَ ﴿٧٧﴾\\
\textamh{78.\  } & فَأَخَذَتْهُمُ ٱلرَّجْفَةُ فَأَصْبَحُوا۟ فِى دَارِهِمْ جَٰثِمِينَ ﴿٧٨﴾\\
\textamh{79.\  } & فَتَوَلَّىٰ عَنْهُمْ وَقَالَ يَـٰقَوْمِ لَقَدْ أَبْلَغْتُكُمْ رِسَالَةَ رَبِّى وَنَصَحْتُ لَكُمْ وَلَـٰكِن لَّا تُحِبُّونَ ٱلنَّـٰصِحِينَ ﴿٧٩﴾\\
\textamh{80.\  } & وَلُوطًا إِذْ قَالَ لِقَوْمِهِۦٓ أَتَأْتُونَ ٱلْفَـٰحِشَةَ مَا سَبَقَكُم بِهَا مِنْ أَحَدٍۢ مِّنَ ٱلْعَـٰلَمِينَ ﴿٨٠﴾\\
\textamh{81.\  } & إِنَّكُمْ لَتَأْتُونَ ٱلرِّجَالَ شَهْوَةًۭ مِّن دُونِ ٱلنِّسَآءِ ۚ بَلْ أَنتُمْ قَوْمٌۭ مُّسْرِفُونَ ﴿٨١﴾\\
\textamh{82.\  } & وَمَا كَانَ جَوَابَ قَوْمِهِۦٓ إِلَّآ أَن قَالُوٓا۟ أَخْرِجُوهُم مِّن قَرْيَتِكُمْ ۖ إِنَّهُمْ أُنَاسٌۭ يَتَطَهَّرُونَ ﴿٨٢﴾\\
\textamh{83.\  } & فَأَنجَيْنَـٰهُ وَأَهْلَهُۥٓ إِلَّا ٱمْرَأَتَهُۥ كَانَتْ مِنَ ٱلْغَٰبِرِينَ ﴿٨٣﴾\\
\textamh{84.\  } & وَأَمْطَرْنَا عَلَيْهِم مَّطَرًۭا ۖ فَٱنظُرْ كَيْفَ كَانَ عَـٰقِبَةُ ٱلْمُجْرِمِينَ ﴿٨٤﴾\\
\textamh{85.\  } & وَإِلَىٰ مَدْيَنَ أَخَاهُمْ شُعَيْبًۭا ۗ قَالَ يَـٰقَوْمِ ٱعْبُدُوا۟ ٱللَّهَ مَا لَكُم مِّنْ إِلَـٰهٍ غَيْرُهُۥ ۖ قَدْ جَآءَتْكُم بَيِّنَةٌۭ مِّن رَّبِّكُمْ ۖ فَأَوْفُوا۟ ٱلْكَيْلَ وَٱلْمِيزَانَ وَلَا تَبْخَسُوا۟ ٱلنَّاسَ أَشْيَآءَهُمْ وَلَا تُفْسِدُوا۟ فِى ٱلْأَرْضِ بَعْدَ إِصْلَـٰحِهَا ۚ ذَٟلِكُمْ خَيْرٌۭ لَّكُمْ إِن كُنتُم مُّؤْمِنِينَ ﴿٨٥﴾\\
\textamh{86.\  } & وَلَا تَقْعُدُوا۟ بِكُلِّ صِرَٰطٍۢ تُوعِدُونَ وَتَصُدُّونَ عَن سَبِيلِ ٱللَّهِ مَنْ ءَامَنَ بِهِۦ وَتَبْغُونَهَا عِوَجًۭا ۚ وَٱذْكُرُوٓا۟ إِذْ كُنتُمْ قَلِيلًۭا فَكَثَّرَكُمْ ۖ وَٱنظُرُوا۟ كَيْفَ كَانَ عَـٰقِبَةُ ٱلْمُفْسِدِينَ ﴿٨٦﴾\\
\textamh{87.\  } & وَإِن كَانَ طَآئِفَةٌۭ مِّنكُمْ ءَامَنُوا۟ بِٱلَّذِىٓ أُرْسِلْتُ بِهِۦ وَطَآئِفَةٌۭ لَّمْ يُؤْمِنُوا۟ فَٱصْبِرُوا۟ حَتَّىٰ يَحْكُمَ ٱللَّهُ بَيْنَنَا ۚ وَهُوَ خَيْرُ ٱلْحَـٰكِمِينَ ﴿٨٧﴾\\
\textamh{88.\  } & ۞ قَالَ ٱلْمَلَأُ ٱلَّذِينَ ٱسْتَكْبَرُوا۟ مِن قَوْمِهِۦ لَنُخْرِجَنَّكَ يَـٰشُعَيْبُ وَٱلَّذِينَ ءَامَنُوا۟ مَعَكَ مِن قَرْيَتِنَآ أَوْ لَتَعُودُنَّ فِى مِلَّتِنَا ۚ قَالَ أَوَلَوْ كُنَّا كَـٰرِهِينَ ﴿٨٨﴾\\
\textamh{89.\  } & قَدِ ٱفْتَرَيْنَا عَلَى ٱللَّهِ كَذِبًا إِنْ عُدْنَا فِى مِلَّتِكُم بَعْدَ إِذْ نَجَّىٰنَا ٱللَّهُ مِنْهَا ۚ وَمَا يَكُونُ لَنَآ أَن نَّعُودَ فِيهَآ إِلَّآ أَن يَشَآءَ ٱللَّهُ رَبُّنَا ۚ وَسِعَ رَبُّنَا كُلَّ شَىْءٍ عِلْمًا ۚ عَلَى ٱللَّهِ تَوَكَّلْنَا ۚ رَبَّنَا ٱفْتَحْ بَيْنَنَا وَبَيْنَ قَوْمِنَا بِٱلْحَقِّ وَأَنتَ خَيْرُ ٱلْفَـٰتِحِينَ ﴿٨٩﴾\\
\textamh{90.\  } & وَقَالَ ٱلْمَلَأُ ٱلَّذِينَ كَفَرُوا۟ مِن قَوْمِهِۦ لَئِنِ ٱتَّبَعْتُمْ شُعَيْبًا إِنَّكُمْ إِذًۭا لَّخَـٰسِرُونَ ﴿٩٠﴾\\
\textamh{91.\  } & فَأَخَذَتْهُمُ ٱلرَّجْفَةُ فَأَصْبَحُوا۟ فِى دَارِهِمْ جَٰثِمِينَ ﴿٩١﴾\\
\textamh{92.\  } & ٱلَّذِينَ كَذَّبُوا۟ شُعَيْبًۭا كَأَن لَّمْ يَغْنَوْا۟ فِيهَا ۚ ٱلَّذِينَ كَذَّبُوا۟ شُعَيْبًۭا كَانُوا۟ هُمُ ٱلْخَـٰسِرِينَ ﴿٩٢﴾\\
\textamh{93.\  } & فَتَوَلَّىٰ عَنْهُمْ وَقَالَ يَـٰقَوْمِ لَقَدْ أَبْلَغْتُكُمْ رِسَـٰلَـٰتِ رَبِّى وَنَصَحْتُ لَكُمْ ۖ فَكَيْفَ ءَاسَىٰ عَلَىٰ قَوْمٍۢ كَـٰفِرِينَ ﴿٩٣﴾\\
\textamh{94.\  } & وَمَآ أَرْسَلْنَا فِى قَرْيَةٍۢ مِّن نَّبِىٍّ إِلَّآ أَخَذْنَآ أَهْلَهَا بِٱلْبَأْسَآءِ وَٱلضَّرَّآءِ لَعَلَّهُمْ يَضَّرَّعُونَ ﴿٩٤﴾\\
\textamh{95.\  } & ثُمَّ بَدَّلْنَا مَكَانَ ٱلسَّيِّئَةِ ٱلْحَسَنَةَ حَتَّىٰ عَفَوا۟ وَّقَالُوا۟ قَدْ مَسَّ ءَابَآءَنَا ٱلضَّرَّآءُ وَٱلسَّرَّآءُ فَأَخَذْنَـٰهُم بَغْتَةًۭ وَهُمْ لَا يَشْعُرُونَ ﴿٩٥﴾\\
\textamh{96.\  } & وَلَوْ أَنَّ أَهْلَ ٱلْقُرَىٰٓ ءَامَنُوا۟ وَٱتَّقَوْا۟ لَفَتَحْنَا عَلَيْهِم بَرَكَـٰتٍۢ مِّنَ ٱلسَّمَآءِ وَٱلْأَرْضِ وَلَـٰكِن كَذَّبُوا۟ فَأَخَذْنَـٰهُم بِمَا كَانُوا۟ يَكْسِبُونَ ﴿٩٦﴾\\
\textamh{97.\  } & أَفَأَمِنَ أَهْلُ ٱلْقُرَىٰٓ أَن يَأْتِيَهُم بَأْسُنَا بَيَـٰتًۭا وَهُمْ نَآئِمُونَ ﴿٩٧﴾\\
\textamh{98.\  } & أَوَأَمِنَ أَهْلُ ٱلْقُرَىٰٓ أَن يَأْتِيَهُم بَأْسُنَا ضُحًۭى وَهُمْ يَلْعَبُونَ ﴿٩٨﴾\\
\textamh{99.\  } & أَفَأَمِنُوا۟ مَكْرَ ٱللَّهِ ۚ فَلَا يَأْمَنُ مَكْرَ ٱللَّهِ إِلَّا ٱلْقَوْمُ ٱلْخَـٰسِرُونَ ﴿٩٩﴾\\
\textamh{100.\  } & أَوَلَمْ يَهْدِ لِلَّذِينَ يَرِثُونَ ٱلْأَرْضَ مِنۢ بَعْدِ أَهْلِهَآ أَن لَّوْ نَشَآءُ أَصَبْنَـٰهُم بِذُنُوبِهِمْ ۚ وَنَطْبَعُ عَلَىٰ قُلُوبِهِمْ فَهُمْ لَا يَسْمَعُونَ ﴿١٠٠﴾\\
\textamh{101.\  } & تِلْكَ ٱلْقُرَىٰ نَقُصُّ عَلَيْكَ مِنْ أَنۢبَآئِهَا ۚ وَلَقَدْ جَآءَتْهُمْ رُسُلُهُم بِٱلْبَيِّنَـٰتِ فَمَا كَانُوا۟ لِيُؤْمِنُوا۟ بِمَا كَذَّبُوا۟ مِن قَبْلُ ۚ كَذَٟلِكَ يَطْبَعُ ٱللَّهُ عَلَىٰ قُلُوبِ ٱلْكَـٰفِرِينَ ﴿١٠١﴾\\
\textamh{102.\  } & وَمَا وَجَدْنَا لِأَكْثَرِهِم مِّنْ عَهْدٍۢ ۖ وَإِن وَجَدْنَآ أَكْثَرَهُمْ لَفَـٰسِقِينَ ﴿١٠٢﴾\\
\textamh{103.\  } & ثُمَّ بَعَثْنَا مِنۢ بَعْدِهِم مُّوسَىٰ بِـَٔايَـٰتِنَآ إِلَىٰ فِرْعَوْنَ وَمَلَإِي۟هِۦ فَظَلَمُوا۟ بِهَا ۖ فَٱنظُرْ كَيْفَ كَانَ عَـٰقِبَةُ ٱلْمُفْسِدِينَ ﴿١٠٣﴾\\
\textamh{104.\  } & وَقَالَ مُوسَىٰ يَـٰفِرْعَوْنُ إِنِّى رَسُولٌۭ مِّن رَّبِّ ٱلْعَـٰلَمِينَ ﴿١٠٤﴾\\
\textamh{105.\  } & حَقِيقٌ عَلَىٰٓ أَن لَّآ أَقُولَ عَلَى ٱللَّهِ إِلَّا ٱلْحَقَّ ۚ قَدْ جِئْتُكُم بِبَيِّنَةٍۢ مِّن رَّبِّكُمْ فَأَرْسِلْ مَعِىَ بَنِىٓ إِسْرَٰٓءِيلَ ﴿١٠٥﴾\\
\textamh{106.\  } & قَالَ إِن كُنتَ جِئْتَ بِـَٔايَةٍۢ فَأْتِ بِهَآ إِن كُنتَ مِنَ ٱلصَّـٰدِقِينَ ﴿١٠٦﴾\\
\textamh{107.\  } & فَأَلْقَىٰ عَصَاهُ فَإِذَا هِىَ ثُعْبَانٌۭ مُّبِينٌۭ ﴿١٠٧﴾\\
\textamh{108.\  } & وَنَزَعَ يَدَهُۥ فَإِذَا هِىَ بَيْضَآءُ لِلنَّـٰظِرِينَ ﴿١٠٨﴾\\
\textamh{109.\  } & قَالَ ٱلْمَلَأُ مِن قَوْمِ فِرْعَوْنَ إِنَّ هَـٰذَا لَسَـٰحِرٌ عَلِيمٌۭ ﴿١٠٩﴾\\
\textamh{110.\  } & يُرِيدُ أَن يُخْرِجَكُم مِّنْ أَرْضِكُمْ ۖ فَمَاذَا تَأْمُرُونَ ﴿١١٠﴾\\
\textamh{111.\  } & قَالُوٓا۟ أَرْجِهْ وَأَخَاهُ وَأَرْسِلْ فِى ٱلْمَدَآئِنِ حَـٰشِرِينَ ﴿١١١﴾\\
\textamh{112.\  } & يَأْتُوكَ بِكُلِّ سَـٰحِرٍ عَلِيمٍۢ ﴿١١٢﴾\\
\textamh{113.\  } & وَجَآءَ ٱلسَّحَرَةُ فِرْعَوْنَ قَالُوٓا۟ إِنَّ لَنَا لَأَجْرًا إِن كُنَّا نَحْنُ ٱلْغَٰلِبِينَ ﴿١١٣﴾\\
\textamh{114.\  } & قَالَ نَعَمْ وَإِنَّكُمْ لَمِنَ ٱلْمُقَرَّبِينَ ﴿١١٤﴾\\
\textamh{115.\  } & قَالُوا۟ يَـٰمُوسَىٰٓ إِمَّآ أَن تُلْقِىَ وَإِمَّآ أَن نَّكُونَ نَحْنُ ٱلْمُلْقِينَ ﴿١١٥﴾\\
\textamh{116.\  } & قَالَ أَلْقُوا۟ ۖ فَلَمَّآ أَلْقَوْا۟ سَحَرُوٓا۟ أَعْيُنَ ٱلنَّاسِ وَٱسْتَرْهَبُوهُمْ وَجَآءُو بِسِحْرٍ عَظِيمٍۢ ﴿١١٦﴾\\
\textamh{117.\  } & ۞ وَأَوْحَيْنَآ إِلَىٰ مُوسَىٰٓ أَنْ أَلْقِ عَصَاكَ ۖ فَإِذَا هِىَ تَلْقَفُ مَا يَأْفِكُونَ ﴿١١٧﴾\\
\textamh{118.\  } & فَوَقَعَ ٱلْحَقُّ وَبَطَلَ مَا كَانُوا۟ يَعْمَلُونَ ﴿١١٨﴾\\
\textamh{119.\  } & فَغُلِبُوا۟ هُنَالِكَ وَٱنقَلَبُوا۟ صَـٰغِرِينَ ﴿١١٩﴾\\
\textamh{120.\  } & وَأُلْقِىَ ٱلسَّحَرَةُ سَـٰجِدِينَ ﴿١٢٠﴾\\
\textamh{121.\  } & قَالُوٓا۟ ءَامَنَّا بِرَبِّ ٱلْعَـٰلَمِينَ ﴿١٢١﴾\\
\textamh{122.\  } & رَبِّ مُوسَىٰ وَهَـٰرُونَ ﴿١٢٢﴾\\
\textamh{123.\  } & قَالَ فِرْعَوْنُ ءَامَنتُم بِهِۦ قَبْلَ أَنْ ءَاذَنَ لَكُمْ ۖ إِنَّ هَـٰذَا لَمَكْرٌۭ مَّكَرْتُمُوهُ فِى ٱلْمَدِينَةِ لِتُخْرِجُوا۟ مِنْهَآ أَهْلَهَا ۖ فَسَوْفَ تَعْلَمُونَ ﴿١٢٣﴾\\
\textamh{124.\  } & لَأُقَطِّعَنَّ أَيْدِيَكُمْ وَأَرْجُلَكُم مِّنْ خِلَـٰفٍۢ ثُمَّ لَأُصَلِّبَنَّكُمْ أَجْمَعِينَ ﴿١٢٤﴾\\
\textamh{125.\  } & قَالُوٓا۟ إِنَّآ إِلَىٰ رَبِّنَا مُنقَلِبُونَ ﴿١٢٥﴾\\
\textamh{126.\  } & وَمَا تَنقِمُ مِنَّآ إِلَّآ أَنْ ءَامَنَّا بِـَٔايَـٰتِ رَبِّنَا لَمَّا جَآءَتْنَا ۚ رَبَّنَآ أَفْرِغْ عَلَيْنَا صَبْرًۭا وَتَوَفَّنَا مُسْلِمِينَ ﴿١٢٦﴾\\
\textamh{127.\  } & وَقَالَ ٱلْمَلَأُ مِن قَوْمِ فِرْعَوْنَ أَتَذَرُ مُوسَىٰ وَقَوْمَهُۥ لِيُفْسِدُوا۟ فِى ٱلْأَرْضِ وَيَذَرَكَ وَءَالِهَتَكَ ۚ قَالَ سَنُقَتِّلُ أَبْنَآءَهُمْ وَنَسْتَحْىِۦ نِسَآءَهُمْ وَإِنَّا فَوْقَهُمْ قَـٰهِرُونَ ﴿١٢٧﴾\\
\textamh{128.\  } & قَالَ مُوسَىٰ لِقَوْمِهِ ٱسْتَعِينُوا۟ بِٱللَّهِ وَٱصْبِرُوٓا۟ ۖ إِنَّ ٱلْأَرْضَ لِلَّهِ يُورِثُهَا مَن يَشَآءُ مِنْ عِبَادِهِۦ ۖ وَٱلْعَـٰقِبَةُ لِلْمُتَّقِينَ ﴿١٢٨﴾\\
\textamh{129.\  } & قَالُوٓا۟ أُوذِينَا مِن قَبْلِ أَن تَأْتِيَنَا وَمِنۢ بَعْدِ مَا جِئْتَنَا ۚ قَالَ عَسَىٰ رَبُّكُمْ أَن يُهْلِكَ عَدُوَّكُمْ وَيَسْتَخْلِفَكُمْ فِى ٱلْأَرْضِ فَيَنظُرَ كَيْفَ تَعْمَلُونَ ﴿١٢٩﴾\\
\textamh{130.\  } & وَلَقَدْ أَخَذْنَآ ءَالَ فِرْعَوْنَ بِٱلسِّنِينَ وَنَقْصٍۢ مِّنَ ٱلثَّمَرَٰتِ لَعَلَّهُمْ يَذَّكَّرُونَ ﴿١٣٠﴾\\
\textamh{131.\  } & فَإِذَا جَآءَتْهُمُ ٱلْحَسَنَةُ قَالُوا۟ لَنَا هَـٰذِهِۦ ۖ وَإِن تُصِبْهُمْ سَيِّئَةٌۭ يَطَّيَّرُوا۟ بِمُوسَىٰ وَمَن مَّعَهُۥٓ ۗ أَلَآ إِنَّمَا طَٰٓئِرُهُمْ عِندَ ٱللَّهِ وَلَـٰكِنَّ أَكْثَرَهُمْ لَا يَعْلَمُونَ ﴿١٣١﴾\\
\textamh{132.\  } & وَقَالُوا۟ مَهْمَا تَأْتِنَا بِهِۦ مِنْ ءَايَةٍۢ لِّتَسْحَرَنَا بِهَا فَمَا نَحْنُ لَكَ بِمُؤْمِنِينَ ﴿١٣٢﴾\\
\textamh{133.\  } & فَأَرْسَلْنَا عَلَيْهِمُ ٱلطُّوفَانَ وَٱلْجَرَادَ وَٱلْقُمَّلَ وَٱلضَّفَادِعَ وَٱلدَّمَ ءَايَـٰتٍۢ مُّفَصَّلَـٰتٍۢ فَٱسْتَكْبَرُوا۟ وَكَانُوا۟ قَوْمًۭا مُّجْرِمِينَ ﴿١٣٣﴾\\
\textamh{134.\  } & وَلَمَّا وَقَعَ عَلَيْهِمُ ٱلرِّجْزُ قَالُوا۟ يَـٰمُوسَى ٱدْعُ لَنَا رَبَّكَ بِمَا عَهِدَ عِندَكَ ۖ لَئِن كَشَفْتَ عَنَّا ٱلرِّجْزَ لَنُؤْمِنَنَّ لَكَ وَلَنُرْسِلَنَّ مَعَكَ بَنِىٓ إِسْرَٰٓءِيلَ ﴿١٣٤﴾\\
\textamh{135.\  } & فَلَمَّا كَشَفْنَا عَنْهُمُ ٱلرِّجْزَ إِلَىٰٓ أَجَلٍ هُم بَٰلِغُوهُ إِذَا هُمْ يَنكُثُونَ ﴿١٣٥﴾\\
\textamh{136.\  } & فَٱنتَقَمْنَا مِنْهُمْ فَأَغْرَقْنَـٰهُمْ فِى ٱلْيَمِّ بِأَنَّهُمْ كَذَّبُوا۟ بِـَٔايَـٰتِنَا وَكَانُوا۟ عَنْهَا غَٰفِلِينَ ﴿١٣٦﴾\\
\textamh{137.\  } & وَأَوْرَثْنَا ٱلْقَوْمَ ٱلَّذِينَ كَانُوا۟ يُسْتَضْعَفُونَ مَشَـٰرِقَ ٱلْأَرْضِ وَمَغَٰرِبَهَا ٱلَّتِى بَٰرَكْنَا فِيهَا ۖ وَتَمَّتْ كَلِمَتُ رَبِّكَ ٱلْحُسْنَىٰ عَلَىٰ بَنِىٓ إِسْرَٰٓءِيلَ بِمَا صَبَرُوا۟ ۖ وَدَمَّرْنَا مَا كَانَ يَصْنَعُ فِرْعَوْنُ وَقَوْمُهُۥ وَمَا كَانُوا۟ يَعْرِشُونَ ﴿١٣٧﴾\\
\textamh{138.\  } & وَجَٰوَزْنَا بِبَنِىٓ إِسْرَٰٓءِيلَ ٱلْبَحْرَ فَأَتَوْا۟ عَلَىٰ قَوْمٍۢ يَعْكُفُونَ عَلَىٰٓ أَصْنَامٍۢ لَّهُمْ ۚ قَالُوا۟ يَـٰمُوسَى ٱجْعَل لَّنَآ إِلَـٰهًۭا كَمَا لَهُمْ ءَالِهَةٌۭ ۚ قَالَ إِنَّكُمْ قَوْمٌۭ تَجْهَلُونَ ﴿١٣٨﴾\\
\textamh{139.\  } & إِنَّ هَـٰٓؤُلَآءِ مُتَبَّرٌۭ مَّا هُمْ فِيهِ وَبَٰطِلٌۭ مَّا كَانُوا۟ يَعْمَلُونَ ﴿١٣٩﴾\\
\textamh{140.\  } & قَالَ أَغَيْرَ ٱللَّهِ أَبْغِيكُمْ إِلَـٰهًۭا وَهُوَ فَضَّلَكُمْ عَلَى ٱلْعَـٰلَمِينَ ﴿١٤٠﴾\\
\textamh{141.\  } & وَإِذْ أَنجَيْنَـٰكُم مِّنْ ءَالِ فِرْعَوْنَ يَسُومُونَكُمْ سُوٓءَ ٱلْعَذَابِ ۖ يُقَتِّلُونَ أَبْنَآءَكُمْ وَيَسْتَحْيُونَ نِسَآءَكُمْ ۚ وَفِى ذَٟلِكُم بَلَآءٌۭ مِّن رَّبِّكُمْ عَظِيمٌۭ ﴿١٤١﴾\\
\textamh{142.\  } & ۞ وَوَٟعَدْنَا مُوسَىٰ ثَلَـٰثِينَ لَيْلَةًۭ وَأَتْمَمْنَـٰهَا بِعَشْرٍۢ فَتَمَّ مِيقَـٰتُ رَبِّهِۦٓ أَرْبَعِينَ لَيْلَةًۭ ۚ وَقَالَ مُوسَىٰ لِأَخِيهِ هَـٰرُونَ ٱخْلُفْنِى فِى قَوْمِى وَأَصْلِحْ وَلَا تَتَّبِعْ سَبِيلَ ٱلْمُفْسِدِينَ ﴿١٤٢﴾\\
\textamh{143.\  } & وَلَمَّا جَآءَ مُوسَىٰ لِمِيقَـٰتِنَا وَكَلَّمَهُۥ رَبُّهُۥ قَالَ رَبِّ أَرِنِىٓ أَنظُرْ إِلَيْكَ ۚ قَالَ لَن تَرَىٰنِى وَلَـٰكِنِ ٱنظُرْ إِلَى ٱلْجَبَلِ فَإِنِ ٱسْتَقَرَّ مَكَانَهُۥ فَسَوْفَ تَرَىٰنِى ۚ فَلَمَّا تَجَلَّىٰ رَبُّهُۥ لِلْجَبَلِ جَعَلَهُۥ دَكًّۭا وَخَرَّ مُوسَىٰ صَعِقًۭا ۚ فَلَمَّآ أَفَاقَ قَالَ سُبْحَـٰنَكَ تُبْتُ إِلَيْكَ وَأَنَا۠ أَوَّلُ ٱلْمُؤْمِنِينَ ﴿١٤٣﴾\\
\textamh{144.\  } & قَالَ يَـٰمُوسَىٰٓ إِنِّى ٱصْطَفَيْتُكَ عَلَى ٱلنَّاسِ بِرِسَـٰلَـٰتِى وَبِكَلَـٰمِى فَخُذْ مَآ ءَاتَيْتُكَ وَكُن مِّنَ ٱلشَّـٰكِرِينَ ﴿١٤٤﴾\\
\textamh{145.\  } & وَكَتَبْنَا لَهُۥ فِى ٱلْأَلْوَاحِ مِن كُلِّ شَىْءٍۢ مَّوْعِظَةًۭ وَتَفْصِيلًۭا لِّكُلِّ شَىْءٍۢ فَخُذْهَا بِقُوَّةٍۢ وَأْمُرْ قَوْمَكَ يَأْخُذُوا۟ بِأَحْسَنِهَا ۚ سَأُو۟رِيكُمْ دَارَ ٱلْفَـٰسِقِينَ ﴿١٤٥﴾\\
\textamh{146.\  } & سَأَصْرِفُ عَنْ ءَايَـٰتِىَ ٱلَّذِينَ يَتَكَبَّرُونَ فِى ٱلْأَرْضِ بِغَيْرِ ٱلْحَقِّ وَإِن يَرَوْا۟ كُلَّ ءَايَةٍۢ لَّا يُؤْمِنُوا۟ بِهَا وَإِن يَرَوْا۟ سَبِيلَ ٱلرُّشْدِ لَا يَتَّخِذُوهُ سَبِيلًۭا وَإِن يَرَوْا۟ سَبِيلَ ٱلْغَىِّ يَتَّخِذُوهُ سَبِيلًۭا ۚ ذَٟلِكَ بِأَنَّهُمْ كَذَّبُوا۟ بِـَٔايَـٰتِنَا وَكَانُوا۟ عَنْهَا غَٰفِلِينَ ﴿١٤٦﴾\\
\textamh{147.\  } & وَٱلَّذِينَ كَذَّبُوا۟ بِـَٔايَـٰتِنَا وَلِقَآءِ ٱلْءَاخِرَةِ حَبِطَتْ أَعْمَـٰلُهُمْ ۚ هَلْ يُجْزَوْنَ إِلَّا مَا كَانُوا۟ يَعْمَلُونَ ﴿١٤٧﴾\\
\textamh{148.\  } & وَٱتَّخَذَ قَوْمُ مُوسَىٰ مِنۢ بَعْدِهِۦ مِنْ حُلِيِّهِمْ عِجْلًۭا جَسَدًۭا لَّهُۥ خُوَارٌ ۚ أَلَمْ يَرَوْا۟ أَنَّهُۥ لَا يُكَلِّمُهُمْ وَلَا يَهْدِيهِمْ سَبِيلًا ۘ ٱتَّخَذُوهُ وَكَانُوا۟ ظَـٰلِمِينَ ﴿١٤٨﴾\\
\textamh{149.\  } & وَلَمَّا سُقِطَ فِىٓ أَيْدِيهِمْ وَرَأَوْا۟ أَنَّهُمْ قَدْ ضَلُّوا۟ قَالُوا۟ لَئِن لَّمْ يَرْحَمْنَا رَبُّنَا وَيَغْفِرْ لَنَا لَنَكُونَنَّ مِنَ ٱلْخَـٰسِرِينَ ﴿١٤٩﴾\\
\textamh{150.\  } & وَلَمَّا رَجَعَ مُوسَىٰٓ إِلَىٰ قَوْمِهِۦ غَضْبَٰنَ أَسِفًۭا قَالَ بِئْسَمَا خَلَفْتُمُونِى مِنۢ بَعْدِىٓ ۖ أَعَجِلْتُمْ أَمْرَ رَبِّكُمْ ۖ وَأَلْقَى ٱلْأَلْوَاحَ وَأَخَذَ بِرَأْسِ أَخِيهِ يَجُرُّهُۥٓ إِلَيْهِ ۚ قَالَ ٱبْنَ أُمَّ إِنَّ ٱلْقَوْمَ ٱسْتَضْعَفُونِى وَكَادُوا۟ يَقْتُلُونَنِى فَلَا تُشْمِتْ بِىَ ٱلْأَعْدَآءَ وَلَا تَجْعَلْنِى مَعَ ٱلْقَوْمِ ٱلظَّـٰلِمِينَ ﴿١٥٠﴾\\
\textamh{151.\  } & قَالَ رَبِّ ٱغْفِرْ لِى وَلِأَخِى وَأَدْخِلْنَا فِى رَحْمَتِكَ ۖ وَأَنتَ أَرْحَمُ ٱلرَّٟحِمِينَ ﴿١٥١﴾\\
\textamh{152.\  } & إِنَّ ٱلَّذِينَ ٱتَّخَذُوا۟ ٱلْعِجْلَ سَيَنَالُهُمْ غَضَبٌۭ مِّن رَّبِّهِمْ وَذِلَّةٌۭ فِى ٱلْحَيَوٰةِ ٱلدُّنْيَا ۚ وَكَذَٟلِكَ نَجْزِى ٱلْمُفْتَرِينَ ﴿١٥٢﴾\\
\textamh{153.\  } & وَٱلَّذِينَ عَمِلُوا۟ ٱلسَّيِّـَٔاتِ ثُمَّ تَابُوا۟ مِنۢ بَعْدِهَا وَءَامَنُوٓا۟ إِنَّ رَبَّكَ مِنۢ بَعْدِهَا لَغَفُورٌۭ رَّحِيمٌۭ ﴿١٥٣﴾\\
\textamh{154.\  } & وَلَمَّا سَكَتَ عَن مُّوسَى ٱلْغَضَبُ أَخَذَ ٱلْأَلْوَاحَ ۖ وَفِى نُسْخَتِهَا هُدًۭى وَرَحْمَةٌۭ لِّلَّذِينَ هُمْ لِرَبِّهِمْ يَرْهَبُونَ ﴿١٥٤﴾\\
\textamh{155.\  } & وَٱخْتَارَ مُوسَىٰ قَوْمَهُۥ سَبْعِينَ رَجُلًۭا لِّمِيقَـٰتِنَا ۖ فَلَمَّآ أَخَذَتْهُمُ ٱلرَّجْفَةُ قَالَ رَبِّ لَوْ شِئْتَ أَهْلَكْتَهُم مِّن قَبْلُ وَإِيَّٰىَ ۖ أَتُهْلِكُنَا بِمَا فَعَلَ ٱلسُّفَهَآءُ مِنَّآ ۖ إِنْ هِىَ إِلَّا فِتْنَتُكَ تُضِلُّ بِهَا مَن تَشَآءُ وَتَهْدِى مَن تَشَآءُ ۖ أَنتَ وَلِيُّنَا فَٱغْفِرْ لَنَا وَٱرْحَمْنَا ۖ وَأَنتَ خَيْرُ ٱلْغَٰفِرِينَ ﴿١٥٥﴾\\
\textamh{156.\  } & ۞ وَٱكْتُبْ لَنَا فِى هَـٰذِهِ ٱلدُّنْيَا حَسَنَةًۭ وَفِى ٱلْءَاخِرَةِ إِنَّا هُدْنَآ إِلَيْكَ ۚ قَالَ عَذَابِىٓ أُصِيبُ بِهِۦ مَنْ أَشَآءُ ۖ وَرَحْمَتِى وَسِعَتْ كُلَّ شَىْءٍۢ ۚ فَسَأَكْتُبُهَا لِلَّذِينَ يَتَّقُونَ وَيُؤْتُونَ ٱلزَّكَوٰةَ وَٱلَّذِينَ هُم بِـَٔايَـٰتِنَا يُؤْمِنُونَ ﴿١٥٦﴾\\
\textamh{157.\  } & ٱلَّذِينَ يَتَّبِعُونَ ٱلرَّسُولَ ٱلنَّبِىَّ ٱلْأُمِّىَّ ٱلَّذِى يَجِدُونَهُۥ مَكْتُوبًا عِندَهُمْ فِى ٱلتَّوْرَىٰةِ وَٱلْإِنجِيلِ يَأْمُرُهُم بِٱلْمَعْرُوفِ وَيَنْهَىٰهُمْ عَنِ ٱلْمُنكَرِ وَيُحِلُّ لَهُمُ ٱلطَّيِّبَٰتِ وَيُحَرِّمُ عَلَيْهِمُ ٱلْخَبَٰٓئِثَ وَيَضَعُ عَنْهُمْ إِصْرَهُمْ وَٱلْأَغْلَـٰلَ ٱلَّتِى كَانَتْ عَلَيْهِمْ ۚ فَٱلَّذِينَ ءَامَنُوا۟ بِهِۦ وَعَزَّرُوهُ وَنَصَرُوهُ وَٱتَّبَعُوا۟ ٱلنُّورَ ٱلَّذِىٓ أُنزِلَ مَعَهُۥٓ ۙ أُو۟لَـٰٓئِكَ هُمُ ٱلْمُفْلِحُونَ ﴿١٥٧﴾\\
\textamh{158.\  } & قُلْ يَـٰٓأَيُّهَا ٱلنَّاسُ إِنِّى رَسُولُ ٱللَّهِ إِلَيْكُمْ جَمِيعًا ٱلَّذِى لَهُۥ مُلْكُ ٱلسَّمَـٰوَٟتِ وَٱلْأَرْضِ ۖ لَآ إِلَـٰهَ إِلَّا هُوَ يُحْىِۦ وَيُمِيتُ ۖ فَـَٔامِنُوا۟ بِٱللَّهِ وَرَسُولِهِ ٱلنَّبِىِّ ٱلْأُمِّىِّ ٱلَّذِى يُؤْمِنُ بِٱللَّهِ وَكَلِمَـٰتِهِۦ وَٱتَّبِعُوهُ لَعَلَّكُمْ تَهْتَدُونَ ﴿١٥٨﴾\\
\textamh{159.\  } & وَمِن قَوْمِ مُوسَىٰٓ أُمَّةٌۭ يَهْدُونَ بِٱلْحَقِّ وَبِهِۦ يَعْدِلُونَ ﴿١٥٩﴾\\
\textamh{160.\  } & وَقَطَّعْنَـٰهُمُ ٱثْنَتَىْ عَشْرَةَ أَسْبَاطًا أُمَمًۭا ۚ وَأَوْحَيْنَآ إِلَىٰ مُوسَىٰٓ إِذِ ٱسْتَسْقَىٰهُ قَوْمُهُۥٓ أَنِ ٱضْرِب بِّعَصَاكَ ٱلْحَجَرَ ۖ فَٱنۢبَجَسَتْ مِنْهُ ٱثْنَتَا عَشْرَةَ عَيْنًۭا ۖ قَدْ عَلِمَ كُلُّ أُنَاسٍۢ مَّشْرَبَهُمْ ۚ وَظَلَّلْنَا عَلَيْهِمُ ٱلْغَمَـٰمَ وَأَنزَلْنَا عَلَيْهِمُ ٱلْمَنَّ وَٱلسَّلْوَىٰ ۖ كُلُوا۟ مِن طَيِّبَٰتِ مَا رَزَقْنَـٰكُمْ ۚ وَمَا ظَلَمُونَا وَلَـٰكِن كَانُوٓا۟ أَنفُسَهُمْ يَظْلِمُونَ ﴿١٦٠﴾\\
\textamh{161.\  } & وَإِذْ قِيلَ لَهُمُ ٱسْكُنُوا۟ هَـٰذِهِ ٱلْقَرْيَةَ وَكُلُوا۟ مِنْهَا حَيْثُ شِئْتُمْ وَقُولُوا۟ حِطَّةٌۭ وَٱدْخُلُوا۟ ٱلْبَابَ سُجَّدًۭا نَّغْفِرْ لَكُمْ خَطِيٓـَٰٔتِكُمْ ۚ سَنَزِيدُ ٱلْمُحْسِنِينَ ﴿١٦١﴾\\
\textamh{162.\  } & فَبَدَّلَ ٱلَّذِينَ ظَلَمُوا۟ مِنْهُمْ قَوْلًا غَيْرَ ٱلَّذِى قِيلَ لَهُمْ فَأَرْسَلْنَا عَلَيْهِمْ رِجْزًۭا مِّنَ ٱلسَّمَآءِ بِمَا كَانُوا۟ يَظْلِمُونَ ﴿١٦٢﴾\\
\textamh{163.\  } & وَسْـَٔلْهُمْ عَنِ ٱلْقَرْيَةِ ٱلَّتِى كَانَتْ حَاضِرَةَ ٱلْبَحْرِ إِذْ يَعْدُونَ فِى ٱلسَّبْتِ إِذْ تَأْتِيهِمْ حِيتَانُهُمْ يَوْمَ سَبْتِهِمْ شُرَّعًۭا وَيَوْمَ لَا يَسْبِتُونَ ۙ لَا تَأْتِيهِمْ ۚ كَذَٟلِكَ نَبْلُوهُم بِمَا كَانُوا۟ يَفْسُقُونَ ﴿١٦٣﴾\\
\textamh{164.\  } & وَإِذْ قَالَتْ أُمَّةٌۭ مِّنْهُمْ لِمَ تَعِظُونَ قَوْمًا ۙ ٱللَّهُ مُهْلِكُهُمْ أَوْ مُعَذِّبُهُمْ عَذَابًۭا شَدِيدًۭا ۖ قَالُوا۟ مَعْذِرَةً إِلَىٰ رَبِّكُمْ وَلَعَلَّهُمْ يَتَّقُونَ ﴿١٦٤﴾\\
\textamh{165.\  } & فَلَمَّا نَسُوا۟ مَا ذُكِّرُوا۟ بِهِۦٓ أَنجَيْنَا ٱلَّذِينَ يَنْهَوْنَ عَنِ ٱلسُّوٓءِ وَأَخَذْنَا ٱلَّذِينَ ظَلَمُوا۟ بِعَذَابٍۭ بَـِٔيسٍۭ بِمَا كَانُوا۟ يَفْسُقُونَ ﴿١٦٥﴾\\
\textamh{166.\  } & فَلَمَّا عَتَوْا۟ عَن مَّا نُهُوا۟ عَنْهُ قُلْنَا لَهُمْ كُونُوا۟ قِرَدَةً خَـٰسِـِٔينَ ﴿١٦٦﴾\\
\textamh{167.\  } & وَإِذْ تَأَذَّنَ رَبُّكَ لَيَبْعَثَنَّ عَلَيْهِمْ إِلَىٰ يَوْمِ ٱلْقِيَـٰمَةِ مَن يَسُومُهُمْ سُوٓءَ ٱلْعَذَابِ ۗ إِنَّ رَبَّكَ لَسَرِيعُ ٱلْعِقَابِ ۖ وَإِنَّهُۥ لَغَفُورٌۭ رَّحِيمٌۭ ﴿١٦٧﴾\\
\textamh{168.\  } & وَقَطَّعْنَـٰهُمْ فِى ٱلْأَرْضِ أُمَمًۭا ۖ مِّنْهُمُ ٱلصَّـٰلِحُونَ وَمِنْهُمْ دُونَ ذَٟلِكَ ۖ وَبَلَوْنَـٰهُم بِٱلْحَسَنَـٰتِ وَٱلسَّيِّـَٔاتِ لَعَلَّهُمْ يَرْجِعُونَ ﴿١٦٨﴾\\
\textamh{169.\  } & فَخَلَفَ مِنۢ بَعْدِهِمْ خَلْفٌۭ وَرِثُوا۟ ٱلْكِتَـٰبَ يَأْخُذُونَ عَرَضَ هَـٰذَا ٱلْأَدْنَىٰ وَيَقُولُونَ سَيُغْفَرُ لَنَا وَإِن يَأْتِهِمْ عَرَضٌۭ مِّثْلُهُۥ يَأْخُذُوهُ ۚ أَلَمْ يُؤْخَذْ عَلَيْهِم مِّيثَـٰقُ ٱلْكِتَـٰبِ أَن لَّا يَقُولُوا۟ عَلَى ٱللَّهِ إِلَّا ٱلْحَقَّ وَدَرَسُوا۟ مَا فِيهِ ۗ وَٱلدَّارُ ٱلْءَاخِرَةُ خَيْرٌۭ لِّلَّذِينَ يَتَّقُونَ ۗ أَفَلَا تَعْقِلُونَ ﴿١٦٩﴾\\
\textamh{170.\  } & وَٱلَّذِينَ يُمَسِّكُونَ بِٱلْكِتَـٰبِ وَأَقَامُوا۟ ٱلصَّلَوٰةَ إِنَّا لَا نُضِيعُ أَجْرَ ٱلْمُصْلِحِينَ ﴿١٧٠﴾\\
\textamh{171.\  } & ۞ وَإِذْ نَتَقْنَا ٱلْجَبَلَ فَوْقَهُمْ كَأَنَّهُۥ ظُلَّةٌۭ وَظَنُّوٓا۟ أَنَّهُۥ وَاقِعٌۢ بِهِمْ خُذُوا۟ مَآ ءَاتَيْنَـٰكُم بِقُوَّةٍۢ وَٱذْكُرُوا۟ مَا فِيهِ لَعَلَّكُمْ تَتَّقُونَ ﴿١٧١﴾\\
\textamh{172.\  } & وَإِذْ أَخَذَ رَبُّكَ مِنۢ بَنِىٓ ءَادَمَ مِن ظُهُورِهِمْ ذُرِّيَّتَهُمْ وَأَشْهَدَهُمْ عَلَىٰٓ أَنفُسِهِمْ أَلَسْتُ بِرَبِّكُمْ ۖ قَالُوا۟ بَلَىٰ ۛ شَهِدْنَآ ۛ أَن تَقُولُوا۟ يَوْمَ ٱلْقِيَـٰمَةِ إِنَّا كُنَّا عَنْ هَـٰذَا غَٰفِلِينَ ﴿١٧٢﴾\\
\textamh{173.\  } & أَوْ تَقُولُوٓا۟ إِنَّمَآ أَشْرَكَ ءَابَآؤُنَا مِن قَبْلُ وَكُنَّا ذُرِّيَّةًۭ مِّنۢ بَعْدِهِمْ ۖ أَفَتُهْلِكُنَا بِمَا فَعَلَ ٱلْمُبْطِلُونَ ﴿١٧٣﴾\\
\textamh{174.\  } & وَكَذَٟلِكَ نُفَصِّلُ ٱلْءَايَـٰتِ وَلَعَلَّهُمْ يَرْجِعُونَ ﴿١٧٤﴾\\
\textamh{175.\  } & وَٱتْلُ عَلَيْهِمْ نَبَأَ ٱلَّذِىٓ ءَاتَيْنَـٰهُ ءَايَـٰتِنَا فَٱنسَلَخَ مِنْهَا فَأَتْبَعَهُ ٱلشَّيْطَٰنُ فَكَانَ مِنَ ٱلْغَاوِينَ ﴿١٧٥﴾\\
\textamh{176.\  } & وَلَوْ شِئْنَا لَرَفَعْنَـٰهُ بِهَا وَلَـٰكِنَّهُۥٓ أَخْلَدَ إِلَى ٱلْأَرْضِ وَٱتَّبَعَ هَوَىٰهُ ۚ فَمَثَلُهُۥ كَمَثَلِ ٱلْكَلْبِ إِن تَحْمِلْ عَلَيْهِ يَلْهَثْ أَوْ تَتْرُكْهُ يَلْهَث ۚ ذَّٰلِكَ مَثَلُ ٱلْقَوْمِ ٱلَّذِينَ كَذَّبُوا۟ بِـَٔايَـٰتِنَا ۚ فَٱقْصُصِ ٱلْقَصَصَ لَعَلَّهُمْ يَتَفَكَّرُونَ ﴿١٧٦﴾\\
\textamh{177.\  } & سَآءَ مَثَلًا ٱلْقَوْمُ ٱلَّذِينَ كَذَّبُوا۟ بِـَٔايَـٰتِنَا وَأَنفُسَهُمْ كَانُوا۟ يَظْلِمُونَ ﴿١٧٧﴾\\
\textamh{178.\  } & مَن يَهْدِ ٱللَّهُ فَهُوَ ٱلْمُهْتَدِى ۖ وَمَن يُضْلِلْ فَأُو۟لَـٰٓئِكَ هُمُ ٱلْخَـٰسِرُونَ ﴿١٧٨﴾\\
\textamh{179.\  } & وَلَقَدْ ذَرَأْنَا لِجَهَنَّمَ كَثِيرًۭا مِّنَ ٱلْجِنِّ وَٱلْإِنسِ ۖ لَهُمْ قُلُوبٌۭ لَّا يَفْقَهُونَ بِهَا وَلَهُمْ أَعْيُنٌۭ لَّا يُبْصِرُونَ بِهَا وَلَهُمْ ءَاذَانٌۭ لَّا يَسْمَعُونَ بِهَآ ۚ أُو۟لَـٰٓئِكَ كَٱلْأَنْعَـٰمِ بَلْ هُمْ أَضَلُّ ۚ أُو۟لَـٰٓئِكَ هُمُ ٱلْغَٰفِلُونَ ﴿١٧٩﴾\\
\textamh{180.\  } & وَلِلَّهِ ٱلْأَسْمَآءُ ٱلْحُسْنَىٰ فَٱدْعُوهُ بِهَا ۖ وَذَرُوا۟ ٱلَّذِينَ يُلْحِدُونَ فِىٓ أَسْمَـٰٓئِهِۦ ۚ سَيُجْزَوْنَ مَا كَانُوا۟ يَعْمَلُونَ ﴿١٨٠﴾\\
\textamh{181.\  } & وَمِمَّنْ خَلَقْنَآ أُمَّةٌۭ يَهْدُونَ بِٱلْحَقِّ وَبِهِۦ يَعْدِلُونَ ﴿١٨١﴾\\
\textamh{182.\  } & وَٱلَّذِينَ كَذَّبُوا۟ بِـَٔايَـٰتِنَا سَنَسْتَدْرِجُهُم مِّنْ حَيْثُ لَا يَعْلَمُونَ ﴿١٨٢﴾\\
\textamh{183.\  } & وَأُمْلِى لَهُمْ ۚ إِنَّ كَيْدِى مَتِينٌ ﴿١٨٣﴾\\
\textamh{184.\  } & أَوَلَمْ يَتَفَكَّرُوا۟ ۗ مَا بِصَاحِبِهِم مِّن جِنَّةٍ ۚ إِنْ هُوَ إِلَّا نَذِيرٌۭ مُّبِينٌ ﴿١٨٤﴾\\
\textamh{185.\  } & أَوَلَمْ يَنظُرُوا۟ فِى مَلَكُوتِ ٱلسَّمَـٰوَٟتِ وَٱلْأَرْضِ وَمَا خَلَقَ ٱللَّهُ مِن شَىْءٍۢ وَأَنْ عَسَىٰٓ أَن يَكُونَ قَدِ ٱقْتَرَبَ أَجَلُهُمْ ۖ فَبِأَىِّ حَدِيثٍۭ بَعْدَهُۥ يُؤْمِنُونَ ﴿١٨٥﴾\\
\textamh{186.\  } & مَن يُضْلِلِ ٱللَّهُ فَلَا هَادِىَ لَهُۥ ۚ وَيَذَرُهُمْ فِى طُغْيَـٰنِهِمْ يَعْمَهُونَ ﴿١٨٦﴾\\
\textamh{187.\  } & يَسْـَٔلُونَكَ عَنِ ٱلسَّاعَةِ أَيَّانَ مُرْسَىٰهَا ۖ قُلْ إِنَّمَا عِلْمُهَا عِندَ رَبِّى ۖ لَا يُجَلِّيهَا لِوَقْتِهَآ إِلَّا هُوَ ۚ ثَقُلَتْ فِى ٱلسَّمَـٰوَٟتِ وَٱلْأَرْضِ ۚ لَا تَأْتِيكُمْ إِلَّا بَغْتَةًۭ ۗ يَسْـَٔلُونَكَ كَأَنَّكَ حَفِىٌّ عَنْهَا ۖ قُلْ إِنَّمَا عِلْمُهَا عِندَ ٱللَّهِ وَلَـٰكِنَّ أَكْثَرَ ٱلنَّاسِ لَا يَعْلَمُونَ ﴿١٨٧﴾\\
\textamh{188.\  } & قُل لَّآ أَمْلِكُ لِنَفْسِى نَفْعًۭا وَلَا ضَرًّا إِلَّا مَا شَآءَ ٱللَّهُ ۚ وَلَوْ كُنتُ أَعْلَمُ ٱلْغَيْبَ لَٱسْتَكْثَرْتُ مِنَ ٱلْخَيْرِ وَمَا مَسَّنِىَ ٱلسُّوٓءُ ۚ إِنْ أَنَا۠ إِلَّا نَذِيرٌۭ وَبَشِيرٌۭ لِّقَوْمٍۢ يُؤْمِنُونَ ﴿١٨٨﴾\\
\textamh{189.\  } & ۞ هُوَ ٱلَّذِى خَلَقَكُم مِّن نَّفْسٍۢ وَٟحِدَةٍۢ وَجَعَلَ مِنْهَا زَوْجَهَا لِيَسْكُنَ إِلَيْهَا ۖ فَلَمَّا تَغَشَّىٰهَا حَمَلَتْ حَمْلًا خَفِيفًۭا فَمَرَّتْ بِهِۦ ۖ فَلَمَّآ أَثْقَلَت دَّعَوَا ٱللَّهَ رَبَّهُمَا لَئِنْ ءَاتَيْتَنَا صَـٰلِحًۭا لَّنَكُونَنَّ مِنَ ٱلشَّـٰكِرِينَ ﴿١٨٩﴾\\
\textamh{190.\  } & فَلَمَّآ ءَاتَىٰهُمَا صَـٰلِحًۭا جَعَلَا لَهُۥ شُرَكَآءَ فِيمَآ ءَاتَىٰهُمَا ۚ فَتَعَـٰلَى ٱللَّهُ عَمَّا يُشْرِكُونَ ﴿١٩٠﴾\\
\textamh{191.\  } & أَيُشْرِكُونَ مَا لَا يَخْلُقُ شَيْـًۭٔا وَهُمْ يُخْلَقُونَ ﴿١٩١﴾\\
\textamh{192.\  } & وَلَا يَسْتَطِيعُونَ لَهُمْ نَصْرًۭا وَلَآ أَنفُسَهُمْ يَنصُرُونَ ﴿١٩٢﴾\\
\textamh{193.\  } & وَإِن تَدْعُوهُمْ إِلَى ٱلْهُدَىٰ لَا يَتَّبِعُوكُمْ ۚ سَوَآءٌ عَلَيْكُمْ أَدَعَوْتُمُوهُمْ أَمْ أَنتُمْ صَـٰمِتُونَ ﴿١٩٣﴾\\
\textamh{194.\  } & إِنَّ ٱلَّذِينَ تَدْعُونَ مِن دُونِ ٱللَّهِ عِبَادٌ أَمْثَالُكُمْ ۖ فَٱدْعُوهُمْ فَلْيَسْتَجِيبُوا۟ لَكُمْ إِن كُنتُمْ صَـٰدِقِينَ ﴿١٩٤﴾\\
\textamh{195.\  } & أَلَهُمْ أَرْجُلٌۭ يَمْشُونَ بِهَآ ۖ أَمْ لَهُمْ أَيْدٍۢ يَبْطِشُونَ بِهَآ ۖ أَمْ لَهُمْ أَعْيُنٌۭ يُبْصِرُونَ بِهَآ ۖ أَمْ لَهُمْ ءَاذَانٌۭ يَسْمَعُونَ بِهَا ۗ قُلِ ٱدْعُوا۟ شُرَكَآءَكُمْ ثُمَّ كِيدُونِ فَلَا تُنظِرُونِ ﴿١٩٥﴾\\
\textamh{196.\  } & إِنَّ وَلِۦِّىَ ٱللَّهُ ٱلَّذِى نَزَّلَ ٱلْكِتَـٰبَ ۖ وَهُوَ يَتَوَلَّى ٱلصَّـٰلِحِينَ ﴿١٩٦﴾\\
\textamh{197.\  } & وَٱلَّذِينَ تَدْعُونَ مِن دُونِهِۦ لَا يَسْتَطِيعُونَ نَصْرَكُمْ وَلَآ أَنفُسَهُمْ يَنصُرُونَ ﴿١٩٧﴾\\
\textamh{198.\  } & وَإِن تَدْعُوهُمْ إِلَى ٱلْهُدَىٰ لَا يَسْمَعُوا۟ ۖ وَتَرَىٰهُمْ يَنظُرُونَ إِلَيْكَ وَهُمْ لَا يُبْصِرُونَ ﴿١٩٨﴾\\
\textamh{199.\  } & خُذِ ٱلْعَفْوَ وَأْمُرْ بِٱلْعُرْفِ وَأَعْرِضْ عَنِ ٱلْجَٰهِلِينَ ﴿١٩٩﴾\\
\textamh{200.\  } & وَإِمَّا يَنزَغَنَّكَ مِنَ ٱلشَّيْطَٰنِ نَزْغٌۭ فَٱسْتَعِذْ بِٱللَّهِ ۚ إِنَّهُۥ سَمِيعٌ عَلِيمٌ ﴿٢٠٠﴾\\
\textamh{201.\  } & إِنَّ ٱلَّذِينَ ٱتَّقَوْا۟ إِذَا مَسَّهُمْ طَٰٓئِفٌۭ مِّنَ ٱلشَّيْطَٰنِ تَذَكَّرُوا۟ فَإِذَا هُم مُّبْصِرُونَ ﴿٢٠١﴾\\
\textamh{202.\  } & وَإِخْوَٟنُهُمْ يَمُدُّونَهُمْ فِى ٱلْغَىِّ ثُمَّ لَا يُقْصِرُونَ ﴿٢٠٢﴾\\
\textamh{203.\  } & وَإِذَا لَمْ تَأْتِهِم بِـَٔايَةٍۢ قَالُوا۟ لَوْلَا ٱجْتَبَيْتَهَا ۚ قُلْ إِنَّمَآ أَتَّبِعُ مَا يُوحَىٰٓ إِلَىَّ مِن رَّبِّى ۚ هَـٰذَا بَصَآئِرُ مِن رَّبِّكُمْ وَهُدًۭى وَرَحْمَةٌۭ لِّقَوْمٍۢ يُؤْمِنُونَ ﴿٢٠٣﴾\\
\textamh{204.\  } & وَإِذَا قُرِئَ ٱلْقُرْءَانُ فَٱسْتَمِعُوا۟ لَهُۥ وَأَنصِتُوا۟ لَعَلَّكُمْ تُرْحَمُونَ ﴿٢٠٤﴾\\
\textamh{205.\  } & وَٱذْكُر رَّبَّكَ فِى نَفْسِكَ تَضَرُّعًۭا وَخِيفَةًۭ وَدُونَ ٱلْجَهْرِ مِنَ ٱلْقَوْلِ بِٱلْغُدُوِّ وَٱلْءَاصَالِ وَلَا تَكُن مِّنَ ٱلْغَٰفِلِينَ ﴿٢٠٥﴾\\
\textamh{206.\  } & إِنَّ ٱلَّذِينَ عِندَ رَبِّكَ لَا يَسْتَكْبِرُونَ عَنْ عِبَادَتِهِۦ وَيُسَبِّحُونَهُۥ وَلَهُۥ يَسْجُدُونَ ۩ ﴿٢٠٦﴾\\
\end{longtable}
\clearpage