%% License: BSD style (Berkley) (i.e. Put the Copyright owner's name always)
%% Writer and Copyright (to): Bewketu(Bilal) Tadilo (2016-17)
\begin{center}\section{\LR{\textamhsec{ሱራቱ አልኢንሳን -}  \textarabic{سوره  الانسان}}}\end{center}
\begin{longtable}{%
  @{}
    p{.5\textwidth}
  @{~~~}
    p{.5\textwidth}
    @{}
}
\textamh{ቢስሚላሂ አራህመኒ ራሂይም } &  \mytextarabic{بِسْمِ ٱللَّهِ ٱلرَّحْمَـٰنِ ٱلرَّحِيمِ}\\
\textamh{1.\  } & \mytextarabic{ هَلْ أَتَىٰ عَلَى ٱلْإِنسَـٰنِ حِينٌۭ مِّنَ ٱلدَّهْرِ لَمْ يَكُن شَيْـًۭٔا مَّذْكُورًا ﴿١﴾}\\
\textamh{2.\  } & \mytextarabic{إِنَّا خَلَقْنَا ٱلْإِنسَـٰنَ مِن نُّطْفَةٍ أَمْشَاجٍۢ نَّبْتَلِيهِ فَجَعَلْنَـٰهُ سَمِيعًۢا بَصِيرًا ﴿٢﴾}\\
\textamh{3.\  } & \mytextarabic{إِنَّا هَدَيْنَـٰهُ ٱلسَّبِيلَ إِمَّا شَاكِرًۭا وَإِمَّا كَفُورًا ﴿٣﴾}\\
\textamh{4.\  } & \mytextarabic{إِنَّآ أَعْتَدْنَا لِلْكَـٰفِرِينَ سَلَـٰسِلَا۟ وَأَغْلَـٰلًۭا وَسَعِيرًا ﴿٤﴾}\\
\textamh{5.\  } & \mytextarabic{إِنَّ ٱلْأَبْرَارَ يَشْرَبُونَ مِن كَأْسٍۢ كَانَ مِزَاجُهَا كَافُورًا ﴿٥﴾}\\
\textamh{6.\  } & \mytextarabic{عَيْنًۭا يَشْرَبُ بِهَا عِبَادُ ٱللَّهِ يُفَجِّرُونَهَا تَفْجِيرًۭا ﴿٦﴾}\\
\textamh{7.\  } & \mytextarabic{يُوفُونَ بِٱلنَّذْرِ وَيَخَافُونَ يَوْمًۭا كَانَ شَرُّهُۥ مُسْتَطِيرًۭا ﴿٧﴾}\\
\textamh{8.\  } & \mytextarabic{وَيُطْعِمُونَ ٱلطَّعَامَ عَلَىٰ حُبِّهِۦ مِسْكِينًۭا وَيَتِيمًۭا وَأَسِيرًا ﴿٨﴾}\\
\textamh{9.\  } & \mytextarabic{إِنَّمَا نُطْعِمُكُمْ لِوَجْهِ ٱللَّهِ لَا نُرِيدُ مِنكُمْ جَزَآءًۭ وَلَا شُكُورًا ﴿٩﴾}\\
\textamh{10.\  } & \mytextarabic{إِنَّا نَخَافُ مِن رَّبِّنَا يَوْمًا عَبُوسًۭا قَمْطَرِيرًۭا ﴿١٠﴾}\\
\textamh{11.\  } & \mytextarabic{فَوَقَىٰهُمُ ٱللَّهُ شَرَّ ذَٟلِكَ ٱلْيَوْمِ وَلَقَّىٰهُمْ نَضْرَةًۭ وَسُرُورًۭا ﴿١١﴾}\\
\textamh{12.\  } & \mytextarabic{وَجَزَىٰهُم بِمَا صَبَرُوا۟ جَنَّةًۭ وَحَرِيرًۭا ﴿١٢﴾}\\
\textamh{13.\  } & \mytextarabic{مُّتَّكِـِٔينَ فِيهَا عَلَى ٱلْأَرَآئِكِ ۖ لَا يَرَوْنَ فِيهَا شَمْسًۭا وَلَا زَمْهَرِيرًۭا ﴿١٣﴾}\\
\textamh{14.\  } & \mytextarabic{وَدَانِيَةً عَلَيْهِمْ ظِلَـٰلُهَا وَذُلِّلَتْ قُطُوفُهَا تَذْلِيلًۭا ﴿١٤﴾}\\
\textamh{15.\  } & \mytextarabic{وَيُطَافُ عَلَيْهِم بِـَٔانِيَةٍۢ مِّن فِضَّةٍۢ وَأَكْوَابٍۢ كَانَتْ قَوَارِيرَا۠ ﴿١٥﴾}\\
\textamh{16.\  } & \mytextarabic{قَوَارِيرَا۟ مِن فِضَّةٍۢ قَدَّرُوهَا تَقْدِيرًۭا ﴿١٦﴾}\\
\textamh{17.\  } & \mytextarabic{وَيُسْقَوْنَ فِيهَا كَأْسًۭا كَانَ مِزَاجُهَا زَنجَبِيلًا ﴿١٧﴾}\\
\textamh{18.\  } & \mytextarabic{عَيْنًۭا فِيهَا تُسَمَّىٰ سَلْسَبِيلًۭا ﴿١٨﴾}\\
\textamh{19.\  } & \mytextarabic{۞ وَيَطُوفُ عَلَيْهِمْ وِلْدَٟنٌۭ مُّخَلَّدُونَ إِذَا رَأَيْتَهُمْ حَسِبْتَهُمْ لُؤْلُؤًۭا مَّنثُورًۭا ﴿١٩﴾}\\
\textamh{20.\  } & \mytextarabic{وَإِذَا رَأَيْتَ ثَمَّ رَأَيْتَ نَعِيمًۭا وَمُلْكًۭا كَبِيرًا ﴿٢٠﴾}\\
\textamh{21.\  } & \mytextarabic{عَـٰلِيَهُمْ ثِيَابُ سُندُسٍ خُضْرٌۭ وَإِسْتَبْرَقٌۭ ۖ وَحُلُّوٓا۟ أَسَاوِرَ مِن فِضَّةٍۢ وَسَقَىٰهُمْ رَبُّهُمْ شَرَابًۭا طَهُورًا ﴿٢١﴾}\\
\textamh{22.\  } & \mytextarabic{إِنَّ هَـٰذَا كَانَ لَكُمْ جَزَآءًۭ وَكَانَ سَعْيُكُم مَّشْكُورًا ﴿٢٢﴾}\\
\textamh{23.\  } & \mytextarabic{إِنَّا نَحْنُ نَزَّلْنَا عَلَيْكَ ٱلْقُرْءَانَ تَنزِيلًۭا ﴿٢٣﴾}\\
\textamh{24.\  } & \mytextarabic{فَٱصْبِرْ لِحُكْمِ رَبِّكَ وَلَا تُطِعْ مِنْهُمْ ءَاثِمًا أَوْ كَفُورًۭا ﴿٢٤﴾}\\
\textamh{25.\  } & \mytextarabic{وَٱذْكُرِ ٱسْمَ رَبِّكَ بُكْرَةًۭ وَأَصِيلًۭا ﴿٢٥﴾}\\
\textamh{26.\  } & \mytextarabic{وَمِنَ ٱلَّيْلِ فَٱسْجُدْ لَهُۥ وَسَبِّحْهُ لَيْلًۭا طَوِيلًا ﴿٢٦﴾}\\
\textamh{27.\  } & \mytextarabic{إِنَّ هَـٰٓؤُلَآءِ يُحِبُّونَ ٱلْعَاجِلَةَ وَيَذَرُونَ وَرَآءَهُمْ يَوْمًۭا ثَقِيلًۭا ﴿٢٧﴾}\\
\textamh{28.\  } & \mytextarabic{نَّحْنُ خَلَقْنَـٰهُمْ وَشَدَدْنَآ أَسْرَهُمْ ۖ وَإِذَا شِئْنَا بَدَّلْنَآ أَمْثَـٰلَهُمْ تَبْدِيلًا ﴿٢٨﴾}\\
\textamh{29.\  } & \mytextarabic{إِنَّ هَـٰذِهِۦ تَذْكِرَةٌۭ ۖ فَمَن شَآءَ ٱتَّخَذَ إِلَىٰ رَبِّهِۦ سَبِيلًۭا ﴿٢٩﴾}\\
\textamh{30.\  } & \mytextarabic{وَمَا تَشَآءُونَ إِلَّآ أَن يَشَآءَ ٱللَّهُ ۚ إِنَّ ٱللَّهَ كَانَ عَلِيمًا حَكِيمًۭا ﴿٣٠﴾}\\
\textamh{31.\  } & \mytextarabic{يُدْخِلُ مَن يَشَآءُ فِى رَحْمَتِهِۦ ۚ وَٱلظَّـٰلِمِينَ أَعَدَّ لَهُمْ عَذَابًا أَلِيمًۢا ﴿٣١﴾}\\
\end{longtable}
\clearpage