%% License: BSD style (Berkley) (i.e. Put the Copyright owner's name always)
%% Writer and Copyright (to): Bewketu(Bilal) Tadilo (2016-17)
\begin{center}\section{\LR{\textamhsec{ሱራቱ አልሙጠፊፊይን -}  \textarabic{سوره  المطففين}}}\end{center}
\begin{longtable}{%
  @{}
    p{.5\textwidth}
  @{~~~}
    p{.5\textwidth}
    @{}
}
\textamh{ቢስሚላሂ አራህመኒ ራሂይም } &  \mytextarabic{بِسْمِ ٱللَّهِ ٱلرَّحْمَـٰنِ ٱلرَّحِيمِ}\\
\textamh{1.\  } & \mytextarabic{ وَيْلٌۭ لِّلْمُطَفِّفِينَ ﴿١﴾}\\
\textamh{2.\  } & \mytextarabic{ٱلَّذِينَ إِذَا ٱكْتَالُوا۟ عَلَى ٱلنَّاسِ يَسْتَوْفُونَ ﴿٢﴾}\\
\textamh{3.\  } & \mytextarabic{وَإِذَا كَالُوهُمْ أَو وَّزَنُوهُمْ يُخْسِرُونَ ﴿٣﴾}\\
\textamh{4.\  } & \mytextarabic{أَلَا يَظُنُّ أُو۟لَـٰٓئِكَ أَنَّهُم مَّبْعُوثُونَ ﴿٤﴾}\\
\textamh{5.\  } & \mytextarabic{لِيَوْمٍ عَظِيمٍۢ ﴿٥﴾}\\
\textamh{6.\  } & \mytextarabic{يَوْمَ يَقُومُ ٱلنَّاسُ لِرَبِّ ٱلْعَـٰلَمِينَ ﴿٦﴾}\\
\textamh{7.\  } & \mytextarabic{كَلَّآ إِنَّ كِتَـٰبَ ٱلْفُجَّارِ لَفِى سِجِّينٍۢ ﴿٧﴾}\\
\textamh{8.\  } & \mytextarabic{وَمَآ أَدْرَىٰكَ مَا سِجِّينٌۭ ﴿٨﴾}\\
\textamh{9.\  } & \mytextarabic{كِتَـٰبٌۭ مَّرْقُومٌۭ ﴿٩﴾}\\
\textamh{10.\  } & \mytextarabic{وَيْلٌۭ يَوْمَئِذٍۢ لِّلْمُكَذِّبِينَ ﴿١٠﴾}\\
\textamh{11.\  } & \mytextarabic{ٱلَّذِينَ يُكَذِّبُونَ بِيَوْمِ ٱلدِّينِ ﴿١١﴾}\\
\textamh{12.\  } & \mytextarabic{وَمَا يُكَذِّبُ بِهِۦٓ إِلَّا كُلُّ مُعْتَدٍ أَثِيمٍ ﴿١٢﴾}\\
\textamh{13.\  } & \mytextarabic{إِذَا تُتْلَىٰ عَلَيْهِ ءَايَـٰتُنَا قَالَ أَسَـٰطِيرُ ٱلْأَوَّلِينَ ﴿١٣﴾}\\
\textamh{14.\  } & \mytextarabic{كَلَّا ۖ بَلْ ۜ رَانَ عَلَىٰ قُلُوبِهِم مَّا كَانُوا۟ يَكْسِبُونَ ﴿١٤﴾}\\
\textamh{15.\  } & \mytextarabic{كَلَّآ إِنَّهُمْ عَن رَّبِّهِمْ يَوْمَئِذٍۢ لَّمَحْجُوبُونَ ﴿١٥﴾}\\
\textamh{16.\  } & \mytextarabic{ثُمَّ إِنَّهُمْ لَصَالُوا۟ ٱلْجَحِيمِ ﴿١٦﴾}\\
\textamh{17.\  } & \mytextarabic{ثُمَّ يُقَالُ هَـٰذَا ٱلَّذِى كُنتُم بِهِۦ تُكَذِّبُونَ ﴿١٧﴾}\\
\textamh{18.\  } & \mytextarabic{كَلَّآ إِنَّ كِتَـٰبَ ٱلْأَبْرَارِ لَفِى عِلِّيِّينَ ﴿١٨﴾}\\
\textamh{19.\  } & \mytextarabic{وَمَآ أَدْرَىٰكَ مَا عِلِّيُّونَ ﴿١٩﴾}\\
\textamh{20.\  } & \mytextarabic{كِتَـٰبٌۭ مَّرْقُومٌۭ ﴿٢٠﴾}\\
\textamh{21.\  } & \mytextarabic{يَشْهَدُهُ ٱلْمُقَرَّبُونَ ﴿٢١﴾}\\
\textamh{22.\  } & \mytextarabic{إِنَّ ٱلْأَبْرَارَ لَفِى نَعِيمٍ ﴿٢٢﴾}\\
\textamh{23.\  } & \mytextarabic{عَلَى ٱلْأَرَآئِكِ يَنظُرُونَ ﴿٢٣﴾}\\
\textamh{24.\  } & \mytextarabic{تَعْرِفُ فِى وُجُوهِهِمْ نَضْرَةَ ٱلنَّعِيمِ ﴿٢٤﴾}\\
\textamh{25.\  } & \mytextarabic{يُسْقَوْنَ مِن رَّحِيقٍۢ مَّخْتُومٍ ﴿٢٥﴾}\\
\textamh{26.\  } & \mytextarabic{خِتَـٰمُهُۥ مِسْكٌۭ ۚ وَفِى ذَٟلِكَ فَلْيَتَنَافَسِ ٱلْمُتَنَـٰفِسُونَ ﴿٢٦﴾}\\
\textamh{27.\  } & \mytextarabic{وَمِزَاجُهُۥ مِن تَسْنِيمٍ ﴿٢٧﴾}\\
\textamh{28.\  } & \mytextarabic{عَيْنًۭا يَشْرَبُ بِهَا ٱلْمُقَرَّبُونَ ﴿٢٨﴾}\\
\textamh{29.\  } & \mytextarabic{إِنَّ ٱلَّذِينَ أَجْرَمُوا۟ كَانُوا۟ مِنَ ٱلَّذِينَ ءَامَنُوا۟ يَضْحَكُونَ ﴿٢٩﴾}\\
\textamh{30.\  } & \mytextarabic{وَإِذَا مَرُّوا۟ بِهِمْ يَتَغَامَزُونَ ﴿٣٠﴾}\\
\textamh{31.\  } & \mytextarabic{وَإِذَا ٱنقَلَبُوٓا۟ إِلَىٰٓ أَهْلِهِمُ ٱنقَلَبُوا۟ فَكِهِينَ ﴿٣١﴾}\\
\textamh{32.\  } & \mytextarabic{وَإِذَا رَأَوْهُمْ قَالُوٓا۟ إِنَّ هَـٰٓؤُلَآءِ لَضَآلُّونَ ﴿٣٢﴾}\\
\textamh{33.\  } & \mytextarabic{وَمَآ أُرْسِلُوا۟ عَلَيْهِمْ حَـٰفِظِينَ ﴿٣٣﴾}\\
\textamh{34.\  } & \mytextarabic{فَٱلْيَوْمَ ٱلَّذِينَ ءَامَنُوا۟ مِنَ ٱلْكُفَّارِ يَضْحَكُونَ ﴿٣٤﴾}\\
\textamh{35.\  } & \mytextarabic{عَلَى ٱلْأَرَآئِكِ يَنظُرُونَ ﴿٣٥﴾}\\
\textamh{36.\  } & \mytextarabic{هَلْ ثُوِّبَ ٱلْكُفَّارُ مَا كَانُوا۟ يَفْعَلُونَ ﴿٣٦﴾}\\
\end{longtable}
\clearpage