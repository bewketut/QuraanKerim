%% License: BSD style (Berkley) (i.e. Put the Copyright owner's name always)
%% Writer and Copyright (to): Bewketu(Bilal) Tadilo (2016-17)
\shadowbox{\section{\LR{\textamharic{ሱራቱ አልሙኡሚን -}  \RL{سوره  المؤمنون}}}}
\begin{longtable}{%
  @{}
    p{.5\textwidth}
  @{~~~~~~~~~~~~~}||
    p{.5\textwidth}
    @{}
}
\nopagebreak
\textamh{\ \ \ \ \ \  ቢስሚላሂ አራህመኒ ራሂይም } &  بِسْمِ ٱللَّهِ ٱلرَّحْمَـٰنِ ٱلرَّحِيمِ\\
\textamh{1.\  } &  قَدْ أَفْلَحَ ٱلْمُؤْمِنُونَ ﴿١﴾\\
\textamh{2.\  } & ٱلَّذِينَ هُمْ فِى صَلَاتِهِمْ خَـٰشِعُونَ ﴿٢﴾\\
\textamh{3.\  } & وَٱلَّذِينَ هُمْ عَنِ ٱللَّغْوِ مُعْرِضُونَ ﴿٣﴾\\
\textamh{4.\  } & وَٱلَّذِينَ هُمْ لِلزَّكَوٰةِ فَـٰعِلُونَ ﴿٤﴾\\
\textamh{5.\  } & وَٱلَّذِينَ هُمْ لِفُرُوجِهِمْ حَـٰفِظُونَ ﴿٥﴾\\
\textamh{6.\  } & إِلَّا عَلَىٰٓ أَزْوَٟجِهِمْ أَوْ مَا مَلَكَتْ أَيْمَـٰنُهُمْ فَإِنَّهُمْ غَيْرُ مَلُومِينَ ﴿٦﴾\\
\textamh{7.\  } & فَمَنِ ٱبْتَغَىٰ وَرَآءَ ذَٟلِكَ فَأُو۟لَـٰٓئِكَ هُمُ ٱلْعَادُونَ ﴿٧﴾\\
\textamh{8.\  } & وَٱلَّذِينَ هُمْ لِأَمَـٰنَـٰتِهِمْ وَعَهْدِهِمْ رَٰعُونَ ﴿٨﴾\\
\textamh{9.\  } & وَٱلَّذِينَ هُمْ عَلَىٰ صَلَوَٟتِهِمْ يُحَافِظُونَ ﴿٩﴾\\
\textamh{10.\  } & أُو۟لَـٰٓئِكَ هُمُ ٱلْوَٟرِثُونَ ﴿١٠﴾\\
\textamh{11.\  } & ٱلَّذِينَ يَرِثُونَ ٱلْفِرْدَوْسَ هُمْ فِيهَا خَـٰلِدُونَ ﴿١١﴾\\
\textamh{12.\  } & وَلَقَدْ خَلَقْنَا ٱلْإِنسَـٰنَ مِن سُلَـٰلَةٍۢ مِّن طِينٍۢ ﴿١٢﴾\\
\textamh{13.\  } & ثُمَّ جَعَلْنَـٰهُ نُطْفَةًۭ فِى قَرَارٍۢ مَّكِينٍۢ ﴿١٣﴾\\
\textamh{14.\  } & ثُمَّ خَلَقْنَا ٱلنُّطْفَةَ عَلَقَةًۭ فَخَلَقْنَا ٱلْعَلَقَةَ مُضْغَةًۭ فَخَلَقْنَا ٱلْمُضْغَةَ عِظَـٰمًۭا فَكَسَوْنَا ٱلْعِظَـٰمَ لَحْمًۭا ثُمَّ أَنشَأْنَـٰهُ خَلْقًا ءَاخَرَ ۚ فَتَبَارَكَ ٱللَّهُ أَحْسَنُ ٱلْخَـٰلِقِينَ ﴿١٤﴾\\
\textamh{15.\  } & ثُمَّ إِنَّكُم بَعْدَ ذَٟلِكَ لَمَيِّتُونَ ﴿١٥﴾\\
\textamh{16.\  } & ثُمَّ إِنَّكُمْ يَوْمَ ٱلْقِيَـٰمَةِ تُبْعَثُونَ ﴿١٦﴾\\
\textamh{17.\  } & وَلَقَدْ خَلَقْنَا فَوْقَكُمْ سَبْعَ طَرَآئِقَ وَمَا كُنَّا عَنِ ٱلْخَلْقِ غَٰفِلِينَ ﴿١٧﴾\\
\textamh{18.\  } & وَأَنزَلْنَا مِنَ ٱلسَّمَآءِ مَآءًۢ بِقَدَرٍۢ فَأَسْكَنَّـٰهُ فِى ٱلْأَرْضِ ۖ وَإِنَّا عَلَىٰ ذَهَابٍۭ بِهِۦ لَقَـٰدِرُونَ ﴿١٨﴾\\
\textamh{19.\  } & فَأَنشَأْنَا لَكُم بِهِۦ جَنَّـٰتٍۢ مِّن نَّخِيلٍۢ وَأَعْنَـٰبٍۢ لَّكُمْ فِيهَا فَوَٟكِهُ كَثِيرَةٌۭ وَمِنْهَا تَأْكُلُونَ ﴿١٩﴾\\
\textamh{20.\  } & وَشَجَرَةًۭ تَخْرُجُ مِن طُورِ سَيْنَآءَ تَنۢبُتُ بِٱلدُّهْنِ وَصِبْغٍۢ لِّلْءَاكِلِينَ ﴿٢٠﴾\\
\textamh{21.\  } & وَإِنَّ لَكُمْ فِى ٱلْأَنْعَـٰمِ لَعِبْرَةًۭ ۖ نُّسْقِيكُم مِّمَّا فِى بُطُونِهَا وَلَكُمْ فِيهَا مَنَـٰفِعُ كَثِيرَةٌۭ وَمِنْهَا تَأْكُلُونَ ﴿٢١﴾\\
\textamh{22.\  } & وَعَلَيْهَا وَعَلَى ٱلْفُلْكِ تُحْمَلُونَ ﴿٢٢﴾\\
\textamh{23.\  } & وَلَقَدْ أَرْسَلْنَا نُوحًا إِلَىٰ قَوْمِهِۦ فَقَالَ يَـٰقَوْمِ ٱعْبُدُوا۟ ٱللَّهَ مَا لَكُم مِّنْ إِلَـٰهٍ غَيْرُهُۥٓ ۖ أَفَلَا تَتَّقُونَ ﴿٢٣﴾\\
\textamh{24.\  } & فَقَالَ ٱلْمَلَؤُا۟ ٱلَّذِينَ كَفَرُوا۟ مِن قَوْمِهِۦ مَا هَـٰذَآ إِلَّا بَشَرٌۭ مِّثْلُكُمْ يُرِيدُ أَن يَتَفَضَّلَ عَلَيْكُمْ وَلَوْ شَآءَ ٱللَّهُ لَأَنزَلَ مَلَـٰٓئِكَةًۭ مَّا سَمِعْنَا بِهَـٰذَا فِىٓ ءَابَآئِنَا ٱلْأَوَّلِينَ ﴿٢٤﴾\\
\textamh{25.\  } & إِنْ هُوَ إِلَّا رَجُلٌۢ بِهِۦ جِنَّةٌۭ فَتَرَبَّصُوا۟ بِهِۦ حَتَّىٰ حِينٍۢ ﴿٢٥﴾\\
\textamh{26.\  } & قَالَ رَبِّ ٱنصُرْنِى بِمَا كَذَّبُونِ ﴿٢٦﴾\\
\textamh{27.\  } & فَأَوْحَيْنَآ إِلَيْهِ أَنِ ٱصْنَعِ ٱلْفُلْكَ بِأَعْيُنِنَا وَوَحْيِنَا فَإِذَا جَآءَ أَمْرُنَا وَفَارَ ٱلتَّنُّورُ ۙ فَٱسْلُكْ فِيهَا مِن كُلٍّۢ زَوْجَيْنِ ٱثْنَيْنِ وَأَهْلَكَ إِلَّا مَن سَبَقَ عَلَيْهِ ٱلْقَوْلُ مِنْهُمْ ۖ وَلَا تُخَـٰطِبْنِى فِى ٱلَّذِينَ ظَلَمُوٓا۟ ۖ إِنَّهُم مُّغْرَقُونَ ﴿٢٧﴾\\
\textamh{28.\  } & فَإِذَا ٱسْتَوَيْتَ أَنتَ وَمَن مَّعَكَ عَلَى ٱلْفُلْكِ فَقُلِ ٱلْحَمْدُ لِلَّهِ ٱلَّذِى نَجَّىٰنَا مِنَ ٱلْقَوْمِ ٱلظَّـٰلِمِينَ ﴿٢٨﴾\\
\textamh{29.\  } & وَقُل رَّبِّ أَنزِلْنِى مُنزَلًۭا مُّبَارَكًۭا وَأَنتَ خَيْرُ ٱلْمُنزِلِينَ ﴿٢٩﴾\\
\textamh{30.\  } & إِنَّ فِى ذَٟلِكَ لَءَايَـٰتٍۢ وَإِن كُنَّا لَمُبْتَلِينَ ﴿٣٠﴾\\
\textamh{31.\  } & ثُمَّ أَنشَأْنَا مِنۢ بَعْدِهِمْ قَرْنًا ءَاخَرِينَ ﴿٣١﴾\\
\textamh{32.\  } & فَأَرْسَلْنَا فِيهِمْ رَسُولًۭا مِّنْهُمْ أَنِ ٱعْبُدُوا۟ ٱللَّهَ مَا لَكُم مِّنْ إِلَـٰهٍ غَيْرُهُۥٓ ۖ أَفَلَا تَتَّقُونَ ﴿٣٢﴾\\
\textamh{33.\  } & وَقَالَ ٱلْمَلَأُ مِن قَوْمِهِ ٱلَّذِينَ كَفَرُوا۟ وَكَذَّبُوا۟ بِلِقَآءِ ٱلْءَاخِرَةِ وَأَتْرَفْنَـٰهُمْ فِى ٱلْحَيَوٰةِ ٱلدُّنْيَا مَا هَـٰذَآ إِلَّا بَشَرٌۭ مِّثْلُكُمْ يَأْكُلُ مِمَّا تَأْكُلُونَ مِنْهُ وَيَشْرَبُ مِمَّا تَشْرَبُونَ ﴿٣٣﴾\\
\textamh{34.\  } & وَلَئِنْ أَطَعْتُم بَشَرًۭا مِّثْلَكُمْ إِنَّكُمْ إِذًۭا لَّخَـٰسِرُونَ ﴿٣٤﴾\\
\textamh{35.\  } & أَيَعِدُكُمْ أَنَّكُمْ إِذَا مِتُّمْ وَكُنتُمْ تُرَابًۭا وَعِظَـٰمًا أَنَّكُم مُّخْرَجُونَ ﴿٣٥﴾\\
\textamh{36.\  } & ۞ هَيْهَاتَ هَيْهَاتَ لِمَا تُوعَدُونَ ﴿٣٦﴾\\
\textamh{37.\  } & إِنْ هِىَ إِلَّا حَيَاتُنَا ٱلدُّنْيَا نَمُوتُ وَنَحْيَا وَمَا نَحْنُ بِمَبْعُوثِينَ ﴿٣٧﴾\\
\textamh{38.\  } & إِنْ هُوَ إِلَّا رَجُلٌ ٱفْتَرَىٰ عَلَى ٱللَّهِ كَذِبًۭا وَمَا نَحْنُ لَهُۥ بِمُؤْمِنِينَ ﴿٣٨﴾\\
\textamh{39.\  } & قَالَ رَبِّ ٱنصُرْنِى بِمَا كَذَّبُونِ ﴿٣٩﴾\\
\textamh{40.\  } & قَالَ عَمَّا قَلِيلٍۢ لَّيُصْبِحُنَّ نَـٰدِمِينَ ﴿٤٠﴾\\
\textamh{41.\  } & فَأَخَذَتْهُمُ ٱلصَّيْحَةُ بِٱلْحَقِّ فَجَعَلْنَـٰهُمْ غُثَآءًۭ ۚ فَبُعْدًۭا لِّلْقَوْمِ ٱلظَّـٰلِمِينَ ﴿٤١﴾\\
\textamh{42.\  } & ثُمَّ أَنشَأْنَا مِنۢ بَعْدِهِمْ قُرُونًا ءَاخَرِينَ ﴿٤٢﴾\\
\textamh{43.\  } & مَا تَسْبِقُ مِنْ أُمَّةٍ أَجَلَهَا وَمَا يَسْتَـْٔخِرُونَ ﴿٤٣﴾\\
\textamh{44.\  } & ثُمَّ أَرْسَلْنَا رُسُلَنَا تَتْرَا ۖ كُلَّ مَا جَآءَ أُمَّةًۭ رَّسُولُهَا كَذَّبُوهُ ۚ فَأَتْبَعْنَا بَعْضَهُم بَعْضًۭا وَجَعَلْنَـٰهُمْ أَحَادِيثَ ۚ فَبُعْدًۭا لِّقَوْمٍۢ لَّا يُؤْمِنُونَ ﴿٤٤﴾\\
\textamh{45.\  } & ثُمَّ أَرْسَلْنَا مُوسَىٰ وَأَخَاهُ هَـٰرُونَ بِـَٔايَـٰتِنَا وَسُلْطَٰنٍۢ مُّبِينٍ ﴿٤٥﴾\\
\textamh{46.\  } & إِلَىٰ فِرْعَوْنَ وَمَلَإِي۟هِۦ فَٱسْتَكْبَرُوا۟ وَكَانُوا۟ قَوْمًا عَالِينَ ﴿٤٦﴾\\
\textamh{47.\  } & فَقَالُوٓا۟ أَنُؤْمِنُ لِبَشَرَيْنِ مِثْلِنَا وَقَوْمُهُمَا لَنَا عَـٰبِدُونَ ﴿٤٧﴾\\
\textamh{48.\  } & فَكَذَّبُوهُمَا فَكَانُوا۟ مِنَ ٱلْمُهْلَكِينَ ﴿٤٨﴾\\
\textamh{49.\  } & وَلَقَدْ ءَاتَيْنَا مُوسَى ٱلْكِتَـٰبَ لَعَلَّهُمْ يَهْتَدُونَ ﴿٤٩﴾\\
\textamh{50.\  } & وَجَعَلْنَا ٱبْنَ مَرْيَمَ وَأُمَّهُۥٓ ءَايَةًۭ وَءَاوَيْنَـٰهُمَآ إِلَىٰ رَبْوَةٍۢ ذَاتِ قَرَارٍۢ وَمَعِينٍۢ ﴿٥٠﴾\\
\textamh{51.\  } & يَـٰٓأَيُّهَا ٱلرُّسُلُ كُلُوا۟ مِنَ ٱلطَّيِّبَٰتِ وَٱعْمَلُوا۟ صَـٰلِحًا ۖ إِنِّى بِمَا تَعْمَلُونَ عَلِيمٌۭ ﴿٥١﴾\\
\textamh{52.\  } & وَإِنَّ هَـٰذِهِۦٓ أُمَّتُكُمْ أُمَّةًۭ وَٟحِدَةًۭ وَأَنَا۠ رَبُّكُمْ فَٱتَّقُونِ ﴿٥٢﴾\\
\textamh{53.\  } & فَتَقَطَّعُوٓا۟ أَمْرَهُم بَيْنَهُمْ زُبُرًۭا ۖ كُلُّ حِزْبٍۭ بِمَا لَدَيْهِمْ فَرِحُونَ ﴿٥٣﴾\\
\textamh{54.\  } & فَذَرْهُمْ فِى غَمْرَتِهِمْ حَتَّىٰ حِينٍ ﴿٥٤﴾\\
\textamh{55.\  } & أَيَحْسَبُونَ أَنَّمَا نُمِدُّهُم بِهِۦ مِن مَّالٍۢ وَبَنِينَ ﴿٥٥﴾\\
\textamh{56.\  } & نُسَارِعُ لَهُمْ فِى ٱلْخَيْرَٰتِ ۚ بَل لَّا يَشْعُرُونَ ﴿٥٦﴾\\
\textamh{57.\  } & إِنَّ ٱلَّذِينَ هُم مِّنْ خَشْيَةِ رَبِّهِم مُّشْفِقُونَ ﴿٥٧﴾\\
\textamh{58.\  } & وَٱلَّذِينَ هُم بِـَٔايَـٰتِ رَبِّهِمْ يُؤْمِنُونَ ﴿٥٨﴾\\
\textamh{59.\  } & وَٱلَّذِينَ هُم بِرَبِّهِمْ لَا يُشْرِكُونَ ﴿٥٩﴾\\
\textamh{60.\  } & وَٱلَّذِينَ يُؤْتُونَ مَآ ءَاتَوا۟ وَّقُلُوبُهُمْ وَجِلَةٌ أَنَّهُمْ إِلَىٰ رَبِّهِمْ رَٰجِعُونَ ﴿٦٠﴾\\
\textamh{61.\  } & أُو۟لَـٰٓئِكَ يُسَـٰرِعُونَ فِى ٱلْخَيْرَٰتِ وَهُمْ لَهَا سَـٰبِقُونَ ﴿٦١﴾\\
\textamh{62.\  } & وَلَا نُكَلِّفُ نَفْسًا إِلَّا وُسْعَهَا ۖ وَلَدَيْنَا كِتَـٰبٌۭ يَنطِقُ بِٱلْحَقِّ ۚ وَهُمْ لَا يُظْلَمُونَ ﴿٦٢﴾\\
\textamh{63.\  } & بَلْ قُلُوبُهُمْ فِى غَمْرَةٍۢ مِّنْ هَـٰذَا وَلَهُمْ أَعْمَـٰلٌۭ مِّن دُونِ ذَٟلِكَ هُمْ لَهَا عَـٰمِلُونَ ﴿٦٣﴾\\
\textamh{64.\  } & حَتَّىٰٓ إِذَآ أَخَذْنَا مُتْرَفِيهِم بِٱلْعَذَابِ إِذَا هُمْ يَجْـَٔرُونَ ﴿٦٤﴾\\
\textamh{65.\  } & لَا تَجْـَٔرُوا۟ ٱلْيَوْمَ ۖ إِنَّكُم مِّنَّا لَا تُنصَرُونَ ﴿٦٥﴾\\
\textamh{66.\  } & قَدْ كَانَتْ ءَايَـٰتِى تُتْلَىٰ عَلَيْكُمْ فَكُنتُمْ عَلَىٰٓ أَعْقَـٰبِكُمْ تَنكِصُونَ ﴿٦٦﴾\\
\textamh{67.\  } & مُسْتَكْبِرِينَ بِهِۦ سَـٰمِرًۭا تَهْجُرُونَ ﴿٦٧﴾\\
\textamh{68.\  } & أَفَلَمْ يَدَّبَّرُوا۟ ٱلْقَوْلَ أَمْ جَآءَهُم مَّا لَمْ يَأْتِ ءَابَآءَهُمُ ٱلْأَوَّلِينَ ﴿٦٨﴾\\
\textamh{69.\  } & أَمْ لَمْ يَعْرِفُوا۟ رَسُولَهُمْ فَهُمْ لَهُۥ مُنكِرُونَ ﴿٦٩﴾\\
\textamh{70.\  } & أَمْ يَقُولُونَ بِهِۦ جِنَّةٌۢ ۚ بَلْ جَآءَهُم بِٱلْحَقِّ وَأَكْثَرُهُمْ لِلْحَقِّ كَـٰرِهُونَ ﴿٧٠﴾\\
\textamh{71.\  } & وَلَوِ ٱتَّبَعَ ٱلْحَقُّ أَهْوَآءَهُمْ لَفَسَدَتِ ٱلسَّمَـٰوَٟتُ وَٱلْأَرْضُ وَمَن فِيهِنَّ ۚ بَلْ أَتَيْنَـٰهُم بِذِكْرِهِمْ فَهُمْ عَن ذِكْرِهِم مُّعْرِضُونَ ﴿٧١﴾\\
\textamh{72.\  } & أَمْ تَسْـَٔلُهُمْ خَرْجًۭا فَخَرَاجُ رَبِّكَ خَيْرٌۭ ۖ وَهُوَ خَيْرُ ٱلرَّٟزِقِينَ ﴿٧٢﴾\\
\textamh{73.\  } & وَإِنَّكَ لَتَدْعُوهُمْ إِلَىٰ صِرَٰطٍۢ مُّسْتَقِيمٍۢ ﴿٧٣﴾\\
\textamh{74.\  } & وَإِنَّ ٱلَّذِينَ لَا يُؤْمِنُونَ بِٱلْءَاخِرَةِ عَنِ ٱلصِّرَٰطِ لَنَـٰكِبُونَ ﴿٧٤﴾\\
\textamh{75.\  } & ۞ وَلَوْ رَحِمْنَـٰهُمْ وَكَشَفْنَا مَا بِهِم مِّن ضُرٍّۢ لَّلَجُّوا۟ فِى طُغْيَـٰنِهِمْ يَعْمَهُونَ ﴿٧٥﴾\\
\textamh{76.\  } & وَلَقَدْ أَخَذْنَـٰهُم بِٱلْعَذَابِ فَمَا ٱسْتَكَانُوا۟ لِرَبِّهِمْ وَمَا يَتَضَرَّعُونَ ﴿٧٦﴾\\
\textamh{77.\  } & حَتَّىٰٓ إِذَا فَتَحْنَا عَلَيْهِم بَابًۭا ذَا عَذَابٍۢ شَدِيدٍ إِذَا هُمْ فِيهِ مُبْلِسُونَ ﴿٧٧﴾\\
\textamh{78.\  } & وَهُوَ ٱلَّذِىٓ أَنشَأَ لَكُمُ ٱلسَّمْعَ وَٱلْأَبْصَـٰرَ وَٱلْأَفْـِٔدَةَ ۚ قَلِيلًۭا مَّا تَشْكُرُونَ ﴿٧٨﴾\\
\textamh{79.\  } & وَهُوَ ٱلَّذِى ذَرَأَكُمْ فِى ٱلْأَرْضِ وَإِلَيْهِ تُحْشَرُونَ ﴿٧٩﴾\\
\textamh{80.\  } & وَهُوَ ٱلَّذِى يُحْىِۦ وَيُمِيتُ وَلَهُ ٱخْتِلَـٰفُ ٱلَّيْلِ وَٱلنَّهَارِ ۚ أَفَلَا تَعْقِلُونَ ﴿٨٠﴾\\
\textamh{81.\  } & بَلْ قَالُوا۟ مِثْلَ مَا قَالَ ٱلْأَوَّلُونَ ﴿٨١﴾\\
\textamh{82.\  } & قَالُوٓا۟ أَءِذَا مِتْنَا وَكُنَّا تُرَابًۭا وَعِظَـٰمًا أَءِنَّا لَمَبْعُوثُونَ ﴿٨٢﴾\\
\textamh{83.\  } & لَقَدْ وُعِدْنَا نَحْنُ وَءَابَآؤُنَا هَـٰذَا مِن قَبْلُ إِنْ هَـٰذَآ إِلَّآ أَسَـٰطِيرُ ٱلْأَوَّلِينَ ﴿٨٣﴾\\
\textamh{84.\  } & قُل لِّمَنِ ٱلْأَرْضُ وَمَن فِيهَآ إِن كُنتُمْ تَعْلَمُونَ ﴿٨٤﴾\\
\textamh{85.\  } & سَيَقُولُونَ لِلَّهِ ۚ قُلْ أَفَلَا تَذَكَّرُونَ ﴿٨٥﴾\\
\textamh{86.\  } & قُلْ مَن رَّبُّ ٱلسَّمَـٰوَٟتِ ٱلسَّبْعِ وَرَبُّ ٱلْعَرْشِ ٱلْعَظِيمِ ﴿٨٦﴾\\
\textamh{87.\  } & سَيَقُولُونَ لِلَّهِ ۚ قُلْ أَفَلَا تَتَّقُونَ ﴿٨٧﴾\\
\textamh{88.\  } & قُلْ مَنۢ بِيَدِهِۦ مَلَكُوتُ كُلِّ شَىْءٍۢ وَهُوَ يُجِيرُ وَلَا يُجَارُ عَلَيْهِ إِن كُنتُمْ تَعْلَمُونَ ﴿٨٨﴾\\
\textamh{89.\  } & سَيَقُولُونَ لِلَّهِ ۚ قُلْ فَأَنَّىٰ تُسْحَرُونَ ﴿٨٩﴾\\
\textamh{90.\  } & بَلْ أَتَيْنَـٰهُم بِٱلْحَقِّ وَإِنَّهُمْ لَكَـٰذِبُونَ ﴿٩٠﴾\\
\textamh{91.\  } & مَا ٱتَّخَذَ ٱللَّهُ مِن وَلَدٍۢ وَمَا كَانَ مَعَهُۥ مِنْ إِلَـٰهٍ ۚ إِذًۭا لَّذَهَبَ كُلُّ إِلَـٰهٍۭ بِمَا خَلَقَ وَلَعَلَا بَعْضُهُمْ عَلَىٰ بَعْضٍۢ ۚ سُبْحَـٰنَ ٱللَّهِ عَمَّا يَصِفُونَ ﴿٩١﴾\\
\textamh{92.\  } & عَـٰلِمِ ٱلْغَيْبِ وَٱلشَّهَـٰدَةِ فَتَعَـٰلَىٰ عَمَّا يُشْرِكُونَ ﴿٩٢﴾\\
\textamh{93.\  } & قُل رَّبِّ إِمَّا تُرِيَنِّى مَا يُوعَدُونَ ﴿٩٣﴾\\
\textamh{94.\  } & رَبِّ فَلَا تَجْعَلْنِى فِى ٱلْقَوْمِ ٱلظَّـٰلِمِينَ ﴿٩٤﴾\\
\textamh{95.\  } & وَإِنَّا عَلَىٰٓ أَن نُّرِيَكَ مَا نَعِدُهُمْ لَقَـٰدِرُونَ ﴿٩٥﴾\\
\textamh{96.\  } & ٱدْفَعْ بِٱلَّتِى هِىَ أَحْسَنُ ٱلسَّيِّئَةَ ۚ نَحْنُ أَعْلَمُ بِمَا يَصِفُونَ ﴿٩٦﴾\\
\textamh{97.\  } & وَقُل رَّبِّ أَعُوذُ بِكَ مِنْ هَمَزَٰتِ ٱلشَّيَـٰطِينِ ﴿٩٧﴾\\
\textamh{98.\  } & وَأَعُوذُ بِكَ رَبِّ أَن يَحْضُرُونِ ﴿٩٨﴾\\
\textamh{99.\  } & حَتَّىٰٓ إِذَا جَآءَ أَحَدَهُمُ ٱلْمَوْتُ قَالَ رَبِّ ٱرْجِعُونِ ﴿٩٩﴾\\
\textamh{100.\  } & لَعَلِّىٓ أَعْمَلُ صَـٰلِحًۭا فِيمَا تَرَكْتُ ۚ كَلَّآ ۚ إِنَّهَا كَلِمَةٌ هُوَ قَآئِلُهَا ۖ وَمِن وَرَآئِهِم بَرْزَخٌ إِلَىٰ يَوْمِ يُبْعَثُونَ ﴿١٠٠﴾\\
\textamh{101.\  } & فَإِذَا نُفِخَ فِى ٱلصُّورِ فَلَآ أَنسَابَ بَيْنَهُمْ يَوْمَئِذٍۢ وَلَا يَتَسَآءَلُونَ ﴿١٠١﴾\\
\textamh{102.\  } & فَمَن ثَقُلَتْ مَوَٟزِينُهُۥ فَأُو۟لَـٰٓئِكَ هُمُ ٱلْمُفْلِحُونَ ﴿١٠٢﴾\\
\textamh{103.\  } & وَمَنْ خَفَّتْ مَوَٟزِينُهُۥ فَأُو۟لَـٰٓئِكَ ٱلَّذِينَ خَسِرُوٓا۟ أَنفُسَهُمْ فِى جَهَنَّمَ خَـٰلِدُونَ ﴿١٠٣﴾\\
\textamh{104.\  } & تَلْفَحُ وُجُوهَهُمُ ٱلنَّارُ وَهُمْ فِيهَا كَـٰلِحُونَ ﴿١٠٤﴾\\
\textamh{105.\  } & أَلَمْ تَكُنْ ءَايَـٰتِى تُتْلَىٰ عَلَيْكُمْ فَكُنتُم بِهَا تُكَذِّبُونَ ﴿١٠٥﴾\\
\textamh{106.\  } & قَالُوا۟ رَبَّنَا غَلَبَتْ عَلَيْنَا شِقْوَتُنَا وَكُنَّا قَوْمًۭا ضَآلِّينَ ﴿١٠٦﴾\\
\textamh{107.\  } & رَبَّنَآ أَخْرِجْنَا مِنْهَا فَإِنْ عُدْنَا فَإِنَّا ظَـٰلِمُونَ ﴿١٠٧﴾\\
\textamh{108.\  } & قَالَ ٱخْسَـُٔوا۟ فِيهَا وَلَا تُكَلِّمُونِ ﴿١٠٨﴾\\
\textamh{109.\  } & إِنَّهُۥ كَانَ فَرِيقٌۭ مِّنْ عِبَادِى يَقُولُونَ رَبَّنَآ ءَامَنَّا فَٱغْفِرْ لَنَا وَٱرْحَمْنَا وَأَنتَ خَيْرُ ٱلرَّٟحِمِينَ ﴿١٠٩﴾\\
\textamh{110.\  } & فَٱتَّخَذْتُمُوهُمْ سِخْرِيًّا حَتَّىٰٓ أَنسَوْكُمْ ذِكْرِى وَكُنتُم مِّنْهُمْ تَضْحَكُونَ ﴿١١٠﴾\\
\textamh{111.\  } & إِنِّى جَزَيْتُهُمُ ٱلْيَوْمَ بِمَا صَبَرُوٓا۟ أَنَّهُمْ هُمُ ٱلْفَآئِزُونَ ﴿١١١﴾\\
\textamh{112.\  } & قَـٰلَ كَمْ لَبِثْتُمْ فِى ٱلْأَرْضِ عَدَدَ سِنِينَ ﴿١١٢﴾\\
\textamh{113.\  } & قَالُوا۟ لَبِثْنَا يَوْمًا أَوْ بَعْضَ يَوْمٍۢ فَسْـَٔلِ ٱلْعَآدِّينَ ﴿١١٣﴾\\
\textamh{114.\  } & قَـٰلَ إِن لَّبِثْتُمْ إِلَّا قَلِيلًۭا ۖ لَّوْ أَنَّكُمْ كُنتُمْ تَعْلَمُونَ ﴿١١٤﴾\\
\textamh{115.\  } & أَفَحَسِبْتُمْ أَنَّمَا خَلَقْنَـٰكُمْ عَبَثًۭا وَأَنَّكُمْ إِلَيْنَا لَا تُرْجَعُونَ ﴿١١٥﴾\\
\textamh{116.\  } & فَتَعَـٰلَى ٱللَّهُ ٱلْمَلِكُ ٱلْحَقُّ ۖ لَآ إِلَـٰهَ إِلَّا هُوَ رَبُّ ٱلْعَرْشِ ٱلْكَرِيمِ ﴿١١٦﴾\\
\textamh{117.\  } & وَمَن يَدْعُ مَعَ ٱللَّهِ إِلَـٰهًا ءَاخَرَ لَا بُرْهَـٰنَ لَهُۥ بِهِۦ فَإِنَّمَا حِسَابُهُۥ عِندَ رَبِّهِۦٓ ۚ إِنَّهُۥ لَا يُفْلِحُ ٱلْكَـٰفِرُونَ ﴿١١٧﴾\\
\textamh{118.\  } & وَقُل رَّبِّ ٱغْفِرْ وَٱرْحَمْ وَأَنتَ خَيْرُ ٱلرَّٟحِمِينَ ﴿١١٨﴾\\
\end{longtable} \newpage
