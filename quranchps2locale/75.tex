%% License: BSD style (Berkley) (i.e. Put the Copyright owner's name always)
%% Writer and Copyright (to): Bewketu(Bilal) Tadilo (2016-17)
\shadowbox{\section{\LR{\textamharic{ሱራቱ አልቂያማ -}  \RL{سوره  القيامة}}}}
\begin{longtable}{%
  @{}
    p{.5\textwidth}
  @{~~~~~~~~~~~~~}||
    p{.5\textwidth}
    @{}
}
\nopagebreak
\textamh{\ \ \ \ \ \  ቢስሚላሂ አራህመኒ ራሂይም } &  بِسْمِ ٱللَّهِ ٱلرَّحْمَـٰنِ ٱلرَّحِيمِ\\
\textamh{1.\  } &  لَآ أُقْسِمُ بِيَوْمِ ٱلْقِيَـٰمَةِ ﴿١﴾\\
\textamh{2.\  } & وَلَآ أُقْسِمُ بِٱلنَّفْسِ ٱللَّوَّامَةِ ﴿٢﴾\\
\textamh{3.\  } & أَيَحْسَبُ ٱلْإِنسَـٰنُ أَلَّن نَّجْمَعَ عِظَامَهُۥ ﴿٣﴾\\
\textamh{4.\  } & بَلَىٰ قَـٰدِرِينَ عَلَىٰٓ أَن نُّسَوِّىَ بَنَانَهُۥ ﴿٤﴾\\
\textamh{5.\  } & بَلْ يُرِيدُ ٱلْإِنسَـٰنُ لِيَفْجُرَ أَمَامَهُۥ ﴿٥﴾\\
\textamh{6.\  } & يَسْـَٔلُ أَيَّانَ يَوْمُ ٱلْقِيَـٰمَةِ ﴿٦﴾\\
\textamh{7.\  } & فَإِذَا بَرِقَ ٱلْبَصَرُ ﴿٧﴾\\
\textamh{8.\  } & وَخَسَفَ ٱلْقَمَرُ ﴿٨﴾\\
\textamh{9.\  } & وَجُمِعَ ٱلشَّمْسُ وَٱلْقَمَرُ ﴿٩﴾\\
\textamh{10.\  } & يَقُولُ ٱلْإِنسَـٰنُ يَوْمَئِذٍ أَيْنَ ٱلْمَفَرُّ ﴿١٠﴾\\
\textamh{11.\  } & كَلَّا لَا وَزَرَ ﴿١١﴾\\
\textamh{12.\  } & إِلَىٰ رَبِّكَ يَوْمَئِذٍ ٱلْمُسْتَقَرُّ ﴿١٢﴾\\
\textamh{13.\  } & يُنَبَّؤُا۟ ٱلْإِنسَـٰنُ يَوْمَئِذٍۭ بِمَا قَدَّمَ وَأَخَّرَ ﴿١٣﴾\\
\textamh{14.\  } & بَلِ ٱلْإِنسَـٰنُ عَلَىٰ نَفْسِهِۦ بَصِيرَةٌۭ ﴿١٤﴾\\
\textamh{15.\  } & وَلَوْ أَلْقَىٰ مَعَاذِيرَهُۥ ﴿١٥﴾\\
\textamh{16.\  } & لَا تُحَرِّكْ بِهِۦ لِسَانَكَ لِتَعْجَلَ بِهِۦٓ ﴿١٦﴾\\
\textamh{17.\  } & إِنَّ عَلَيْنَا جَمْعَهُۥ وَقُرْءَانَهُۥ ﴿١٧﴾\\
\textamh{18.\  } & فَإِذَا قَرَأْنَـٰهُ فَٱتَّبِعْ قُرْءَانَهُۥ ﴿١٨﴾\\
\textamh{19.\  } & ثُمَّ إِنَّ عَلَيْنَا بَيَانَهُۥ ﴿١٩﴾\\
\textamh{20.\  } & كَلَّا بَلْ تُحِبُّونَ ٱلْعَاجِلَةَ ﴿٢٠﴾\\
\textamh{21.\  } & وَتَذَرُونَ ٱلْءَاخِرَةَ ﴿٢١﴾\\
\textamh{22.\  } & وُجُوهٌۭ يَوْمَئِذٍۢ نَّاضِرَةٌ ﴿٢٢﴾\\
\textamh{23.\  } & إِلَىٰ رَبِّهَا نَاظِرَةٌۭ ﴿٢٣﴾\\
\textamh{24.\  } & وَوُجُوهٌۭ يَوْمَئِذٍۭ بَاسِرَةٌۭ ﴿٢٤﴾\\
\textamh{25.\  } & تَظُنُّ أَن يُفْعَلَ بِهَا فَاقِرَةٌۭ ﴿٢٥﴾\\
\textamh{26.\  } & كَلَّآ إِذَا بَلَغَتِ ٱلتَّرَاقِىَ ﴿٢٦﴾\\
\textamh{27.\  } & وَقِيلَ مَنْ ۜ رَاقٍۢ ﴿٢٧﴾\\
\textamh{28.\  } & وَظَنَّ أَنَّهُ ٱلْفِرَاقُ ﴿٢٨﴾\\
\textamh{29.\  } & وَٱلْتَفَّتِ ٱلسَّاقُ بِٱلسَّاقِ ﴿٢٩﴾\\
\textamh{30.\  } & إِلَىٰ رَبِّكَ يَوْمَئِذٍ ٱلْمَسَاقُ ﴿٣٠﴾\\
\textamh{31.\  } & فَلَا صَدَّقَ وَلَا صَلَّىٰ ﴿٣١﴾\\
\textamh{32.\  } & وَلَـٰكِن كَذَّبَ وَتَوَلَّىٰ ﴿٣٢﴾\\
\textamh{33.\  } & ثُمَّ ذَهَبَ إِلَىٰٓ أَهْلِهِۦ يَتَمَطَّىٰٓ ﴿٣٣﴾\\
\textamh{34.\  } & أَوْلَىٰ لَكَ فَأَوْلَىٰ ﴿٣٤﴾\\
\textamh{35.\  } & ثُمَّ أَوْلَىٰ لَكَ فَأَوْلَىٰٓ ﴿٣٥﴾\\
\textamh{36.\  } & أَيَحْسَبُ ٱلْإِنسَـٰنُ أَن يُتْرَكَ سُدًى ﴿٣٦﴾\\
\textamh{37.\  } & أَلَمْ يَكُ نُطْفَةًۭ مِّن مَّنِىٍّۢ يُمْنَىٰ ﴿٣٧﴾\\
\textamh{38.\  } & ثُمَّ كَانَ عَلَقَةًۭ فَخَلَقَ فَسَوَّىٰ ﴿٣٨﴾\\
\textamh{39.\  } & فَجَعَلَ مِنْهُ ٱلزَّوْجَيْنِ ٱلذَّكَرَ وَٱلْأُنثَىٰٓ ﴿٣٩﴾\\
\textamh{40.\  } & أَلَيْسَ ذَٟلِكَ بِقَـٰدِرٍ عَلَىٰٓ أَن يُحْۦِىَ ٱلْمَوْتَىٰ ﴿٤٠﴾\\
\end{longtable} \newpage
