%% License: BSD style (Berkley) (i.e. Put the Copyright owner's name always)
%% Writer and Copyright (to): Bewketu(Bilal) Tadilo (2016-17)
\shadowbox{\section{\LR{\textamharic{ሱራቱ አልርኣድ -}  \RL{سوره  الرعد}}}}
\begin{longtable}{%
  @{}
    p{.5\textwidth}
  @{~~~~~~~~~~~~~}||
    p{.5\textwidth}
    @{}
}
\nopagebreak
\textamh{\ \ \ \ \ \  ቢስሚላሂ አራህመኒ ራሂይም } &  بِسمِ ٱللَّهِ ٱلرَّحمَـٰنِ ٱلرَّحِيمِ\\
\textamh{1.\  } &  الٓمٓر ۚ تِلكَ ءَايَـٰتُ ٱلكِتَـٰبِ ۗ وَٱلَّذِىٓ أُنزِلَ إِلَيكَ مِن رَّبِّكَ ٱلحَقُّ وَلَـٰكِنَّ أَكثَرَ ٱلنَّاسِ لَا يُؤمِنُونَ ﴿١﴾\\
\textamh{2.\  } & ٱللَّهُ ٱلَّذِى رَفَعَ ٱلسَّمَـٰوَٟتِ بِغَيرِ عَمَدٍۢ تَرَونَهَا ۖ ثُمَّ ٱستَوَىٰ عَلَى ٱلعَرشِ ۖ وَسَخَّرَ ٱلشَّمسَ وَٱلقَمَرَ ۖ كُلٌّۭ يَجرِى لِأَجَلٍۢ مُّسَمًّۭى ۚ يُدَبِّرُ ٱلأَمرَ يُفَصِّلُ ٱلءَايَـٰتِ لَعَلَّكُم بِلِقَآءِ رَبِّكُم تُوقِنُونَ ﴿٢﴾\\
\textamh{3.\  } & وَهُوَ ٱلَّذِى مَدَّ ٱلأَرضَ وَجَعَلَ فِيهَا رَوَٟسِىَ وَأَنهَـٰرًۭا ۖ وَمِن كُلِّ ٱلثَّمَرَٰتِ جَعَلَ فِيهَا زَوجَينِ ٱثنَينِ ۖ يُغشِى ٱلَّيلَ ٱلنَّهَارَ ۚ إِنَّ فِى ذَٟلِكَ لَءَايَـٰتٍۢ لِّقَومٍۢ يَتَفَكَّرُونَ ﴿٣﴾\\
\textamh{4.\  } & وَفِى ٱلأَرضِ قِطَعٌۭ مُّتَجَٰوِرَٰتٌۭ وَجَنَّـٰتٌۭ مِّن أَعنَـٰبٍۢ وَزَرعٌۭ وَنَخِيلٌۭ صِنوَانٌۭ وَغَيرُ صِنوَانٍۢ يُسقَىٰ بِمَآءٍۢ وَٟحِدٍۢ وَنُفَضِّلُ بَعضَهَا عَلَىٰ بَعضٍۢ فِى ٱلأُكُلِ ۚ إِنَّ فِى ذَٟلِكَ لَءَايَـٰتٍۢ لِّقَومٍۢ يَعقِلُونَ ﴿٤﴾\\
\textamh{5.\  } & ۞ وَإِن تَعجَب فَعَجَبٌۭ قَولُهُم أَءِذَا كُنَّا تُرَٰبًا أَءِنَّا لَفِى خَلقٍۢ جَدِيدٍ ۗ أُو۟لَـٰٓئِكَ ٱلَّذِينَ كَفَرُوا۟ بِرَبِّهِم ۖ وَأُو۟لَـٰٓئِكَ ٱلأَغلَـٰلُ فِىٓ أَعنَاقِهِم ۖ وَأُو۟لَـٰٓئِكَ أَصحَـٰبُ ٱلنَّارِ ۖ هُم فِيهَا خَـٰلِدُونَ ﴿٥﴾\\
\textamh{6.\  } & وَيَستَعجِلُونَكَ بِٱلسَّيِّئَةِ قَبلَ ٱلحَسَنَةِ وَقَد خَلَت مِن قَبلِهِمُ ٱلمَثُلَـٰتُ ۗ وَإِنَّ رَبَّكَ لَذُو مَغفِرَةٍۢ لِّلنَّاسِ عَلَىٰ ظُلمِهِم ۖ وَإِنَّ رَبَّكَ لَشَدِيدُ ٱلعِقَابِ ﴿٦﴾\\
\textamh{7.\  } & وَيَقُولُ ٱلَّذِينَ كَفَرُوا۟ لَولَآ أُنزِلَ عَلَيهِ ءَايَةٌۭ مِّن رَّبِّهِۦٓ ۗ إِنَّمَآ أَنتَ مُنذِرٌۭ ۖ وَلِكُلِّ قَومٍ هَادٍ ﴿٧﴾\\
\textamh{8.\  } & ٱللَّهُ يَعلَمُ مَا تَحمِلُ كُلُّ أُنثَىٰ وَمَا تَغِيضُ ٱلأَرحَامُ وَمَا تَزدَادُ ۖ وَكُلُّ شَىءٍ عِندَهُۥ بِمِقدَارٍ ﴿٨﴾\\
\textamh{9.\  } & عَـٰلِمُ ٱلغَيبِ وَٱلشَّهَـٰدَةِ ٱلكَبِيرُ ٱلمُتَعَالِ ﴿٩﴾\\
\textamh{10.\  } & سَوَآءٌۭ مِّنكُم مَّن أَسَرَّ ٱلقَولَ وَمَن جَهَرَ بِهِۦ وَمَن هُوَ مُستَخفٍۭ بِٱلَّيلِ وَسَارِبٌۢ بِٱلنَّهَارِ ﴿١٠﴾\\
\textamh{11.\  } & لَهُۥ مُعَقِّبَٰتٌۭ مِّنۢ بَينِ يَدَيهِ وَمِن خَلفِهِۦ يَحفَظُونَهُۥ مِن أَمرِ ٱللَّهِ ۗ إِنَّ ٱللَّهَ لَا يُغَيِّرُ مَا بِقَومٍ حَتَّىٰ يُغَيِّرُوا۟ مَا بِأَنفُسِهِم ۗ وَإِذَآ أَرَادَ ٱللَّهُ بِقَومٍۢ سُوٓءًۭا فَلَا مَرَدَّ لَهُۥ ۚ وَمَا لَهُم مِّن دُونِهِۦ مِن وَالٍ ﴿١١﴾\\
\textamh{12.\  } & هُوَ ٱلَّذِى يُرِيكُمُ ٱلبَرقَ خَوفًۭا وَطَمَعًۭا وَيُنشِئُ ٱلسَّحَابَ ٱلثِّقَالَ ﴿١٢﴾\\
\textamh{13.\  } & وَيُسَبِّحُ ٱلرَّعدُ بِحَمدِهِۦ وَٱلمَلَـٰٓئِكَةُ مِن خِيفَتِهِۦ وَيُرسِلُ ٱلصَّوَٟعِقَ فَيُصِيبُ بِهَا مَن يَشَآءُ وَهُم يُجَٰدِلُونَ فِى ٱللَّهِ وَهُوَ شَدِيدُ ٱلمِحَالِ ﴿١٣﴾\\
\textamh{14.\  } & لَهُۥ دَعوَةُ ٱلحَقِّ ۖ وَٱلَّذِينَ يَدعُونَ مِن دُونِهِۦ لَا يَستَجِيبُونَ لَهُم بِشَىءٍ إِلَّا كَبَٰسِطِ كَفَّيهِ إِلَى ٱلمَآءِ لِيَبلُغَ فَاهُ وَمَا هُوَ بِبَٰلِغِهِۦ ۚ وَمَا دُعَآءُ ٱلكَـٰفِرِينَ إِلَّا فِى ضَلَـٰلٍۢ ﴿١٤﴾\\
\textamh{15.\  } & وَلِلَّهِ يَسجُدُ مَن فِى ٱلسَّمَـٰوَٟتِ وَٱلأَرضِ طَوعًۭا وَكَرهًۭا وَظِلَـٰلُهُم بِٱلغُدُوِّ وَٱلءَاصَالِ ۩ ﴿١٥﴾\\
\textamh{16.\  } & قُل مَن رَّبُّ ٱلسَّمَـٰوَٟتِ وَٱلأَرضِ قُلِ ٱللَّهُ ۚ قُل أَفَٱتَّخَذتُم مِّن دُونِهِۦٓ أَولِيَآءَ لَا يَملِكُونَ لِأَنفُسِهِم نَفعًۭا وَلَا ضَرًّۭا ۚ قُل هَل يَستَوِى ٱلأَعمَىٰ وَٱلبَصِيرُ أَم هَل تَستَوِى ٱلظُّلُمَـٰتُ وَٱلنُّورُ ۗ أَم جَعَلُوا۟ لِلَّهِ شُرَكَآءَ خَلَقُوا۟ كَخَلقِهِۦ فَتَشَـٰبَهَ ٱلخَلقُ عَلَيهِم ۚ قُلِ ٱللَّهُ خَـٰلِقُ كُلِّ شَىءٍۢ وَهُوَ ٱلوَٟحِدُ ٱلقَهَّٰرُ ﴿١٦﴾\\
\textamh{17.\  } & أَنزَلَ مِنَ ٱلسَّمَآءِ مَآءًۭ فَسَالَت أَودِيَةٌۢ بِقَدَرِهَا فَٱحتَمَلَ ٱلسَّيلُ زَبَدًۭا رَّابِيًۭا ۚ وَمِمَّا يُوقِدُونَ عَلَيهِ فِى ٱلنَّارِ ٱبتِغَآءَ حِليَةٍ أَو مَتَـٰعٍۢ زَبَدٌۭ مِّثلُهُۥ ۚ كَذَٟلِكَ يَضرِبُ ٱللَّهُ ٱلحَقَّ وَٱلبَٰطِلَ ۚ فَأَمَّا ٱلزَّبَدُ فَيَذهَبُ جُفَآءًۭ ۖ وَأَمَّا مَا يَنفَعُ ٱلنَّاسَ فَيَمكُثُ فِى ٱلأَرضِ ۚ كَذَٟلِكَ يَضرِبُ ٱللَّهُ ٱلأَمثَالَ ﴿١٧﴾\\
\textamh{18.\  } & لِلَّذِينَ ٱستَجَابُوا۟ لِرَبِّهِمُ ٱلحُسنَىٰ ۚ وَٱلَّذِينَ لَم يَستَجِيبُوا۟ لَهُۥ لَو أَنَّ لَهُم مَّا فِى ٱلأَرضِ جَمِيعًۭا وَمِثلَهُۥ مَعَهُۥ لَٱفتَدَوا۟ بِهِۦٓ ۚ أُو۟لَـٰٓئِكَ لَهُم سُوٓءُ ٱلحِسَابِ وَمَأوَىٰهُم جَهَنَّمُ ۖ وَبِئسَ ٱلمِهَادُ ﴿١٨﴾\\
\textamh{19.\  } & ۞ أَفَمَن يَعلَمُ أَنَّمَآ أُنزِلَ إِلَيكَ مِن رَّبِّكَ ٱلحَقُّ كَمَن هُوَ أَعمَىٰٓ ۚ إِنَّمَا يَتَذَكَّرُ أُو۟لُوا۟ ٱلأَلبَٰبِ ﴿١٩﴾\\
\textamh{20.\  } & ٱلَّذِينَ يُوفُونَ بِعَهدِ ٱللَّهِ وَلَا يَنقُضُونَ ٱلمِيثَـٰقَ ﴿٢٠﴾\\
\textamh{21.\  } & وَٱلَّذِينَ يَصِلُونَ مَآ أَمَرَ ٱللَّهُ بِهِۦٓ أَن يُوصَلَ وَيَخشَونَ رَبَّهُم وَيَخَافُونَ سُوٓءَ ٱلحِسَابِ ﴿٢١﴾\\
\textamh{22.\  } & وَٱلَّذِينَ صَبَرُوا۟ ٱبتِغَآءَ وَجهِ رَبِّهِم وَأَقَامُوا۟ ٱلصَّلَوٰةَ وَأَنفَقُوا۟ مِمَّا رَزَقنَـٰهُم سِرًّۭا وَعَلَانِيَةًۭ وَيَدرَءُونَ بِٱلحَسَنَةِ ٱلسَّيِّئَةَ أُو۟لَـٰٓئِكَ لَهُم عُقبَى ٱلدَّارِ ﴿٢٢﴾\\
\textamh{23.\  } & جَنَّـٰتُ عَدنٍۢ يَدخُلُونَهَا وَمَن صَلَحَ مِن ءَابَآئِهِم وَأَزوَٟجِهِم وَذُرِّيَّٰتِهِم ۖ وَٱلمَلَـٰٓئِكَةُ يَدخُلُونَ عَلَيهِم مِّن كُلِّ بَابٍۢ ﴿٢٣﴾\\
\textamh{24.\  } & سَلَـٰمٌ عَلَيكُم بِمَا صَبَرتُم ۚ فَنِعمَ عُقبَى ٱلدَّارِ ﴿٢٤﴾\\
\textamh{25.\  } & وَٱلَّذِينَ يَنقُضُونَ عَهدَ ٱللَّهِ مِنۢ بَعدِ مِيثَـٰقِهِۦ وَيَقطَعُونَ مَآ أَمَرَ ٱللَّهُ بِهِۦٓ أَن يُوصَلَ وَيُفسِدُونَ فِى ٱلأَرضِ ۙ أُو۟لَـٰٓئِكَ لَهُمُ ٱللَّعنَةُ وَلَهُم سُوٓءُ ٱلدَّارِ ﴿٢٥﴾\\
\textamh{26.\  } & ٱللَّهُ يَبسُطُ ٱلرِّزقَ لِمَن يَشَآءُ وَيَقدِرُ ۚ وَفَرِحُوا۟ بِٱلحَيَوٰةِ ٱلدُّنيَا وَمَا ٱلحَيَوٰةُ ٱلدُّنيَا فِى ٱلءَاخِرَةِ إِلَّا مَتَـٰعٌۭ ﴿٢٦﴾\\
\textamh{27.\  } & وَيَقُولُ ٱلَّذِينَ كَفَرُوا۟ لَولَآ أُنزِلَ عَلَيهِ ءَايَةٌۭ مِّن رَّبِّهِۦ ۗ قُل إِنَّ ٱللَّهَ يُضِلُّ مَن يَشَآءُ وَيَهدِىٓ إِلَيهِ مَن أَنَابَ ﴿٢٧﴾\\
\textamh{28.\  } & ٱلَّذِينَ ءَامَنُوا۟ وَتَطمَئِنُّ قُلُوبُهُم بِذِكرِ ٱللَّهِ ۗ أَلَا بِذِكرِ ٱللَّهِ تَطمَئِنُّ ٱلقُلُوبُ ﴿٢٨﴾\\
\textamh{29.\  } & ٱلَّذِينَ ءَامَنُوا۟ وَعَمِلُوا۟ ٱلصَّـٰلِحَـٰتِ طُوبَىٰ لَهُم وَحُسنُ مَـَٔابٍۢ ﴿٢٩﴾\\
\textamh{30.\  } & كَذَٟلِكَ أَرسَلنَـٰكَ فِىٓ أُمَّةٍۢ قَد خَلَت مِن قَبلِهَآ أُمَمٌۭ لِّتَتلُوَا۟ عَلَيهِمُ ٱلَّذِىٓ أَوحَينَآ إِلَيكَ وَهُم يَكفُرُونَ بِٱلرَّحمَـٰنِ ۚ قُل هُوَ رَبِّى لَآ إِلَـٰهَ إِلَّا هُوَ عَلَيهِ تَوَكَّلتُ وَإِلَيهِ مَتَابِ ﴿٣٠﴾\\
\textamh{31.\  } & وَلَو أَنَّ قُرءَانًۭا سُيِّرَت بِهِ ٱلجِبَالُ أَو قُطِّعَت بِهِ ٱلأَرضُ أَو كُلِّمَ بِهِ ٱلمَوتَىٰ ۗ بَل لِّلَّهِ ٱلأَمرُ جَمِيعًا ۗ أَفَلَم يَا۟يـَٔسِ ٱلَّذِينَ ءَامَنُوٓا۟ أَن لَّو يَشَآءُ ٱللَّهُ لَهَدَى ٱلنَّاسَ جَمِيعًۭا ۗ وَلَا يَزَالُ ٱلَّذِينَ كَفَرُوا۟ تُصِيبُهُم بِمَا صَنَعُوا۟ قَارِعَةٌ أَو تَحُلُّ قَرِيبًۭا مِّن دَارِهِم حَتَّىٰ يَأتِىَ وَعدُ ٱللَّهِ ۚ إِنَّ ٱللَّهَ لَا يُخلِفُ ٱلمِيعَادَ ﴿٣١﴾\\
\textamh{32.\  } & وَلَقَدِ ٱستُهزِئَ بِرُسُلٍۢ مِّن قَبلِكَ فَأَملَيتُ لِلَّذِينَ كَفَرُوا۟ ثُمَّ أَخَذتُهُم ۖ فَكَيفَ كَانَ عِقَابِ ﴿٣٢﴾\\
\textamh{33.\  } & أَفَمَن هُوَ قَآئِمٌ عَلَىٰ كُلِّ نَفسٍۭ بِمَا كَسَبَت ۗ وَجَعَلُوا۟ لِلَّهِ شُرَكَآءَ قُل سَمُّوهُم ۚ أَم تُنَبِّـُٔونَهُۥ بِمَا لَا يَعلَمُ فِى ٱلأَرضِ أَم بِظَـٰهِرٍۢ مِّنَ ٱلقَولِ ۗ بَل زُيِّنَ لِلَّذِينَ كَفَرُوا۟ مَكرُهُم وَصُدُّوا۟ عَنِ ٱلسَّبِيلِ ۗ وَمَن يُضلِلِ ٱللَّهُ فَمَا لَهُۥ مِن هَادٍۢ ﴿٣٣﴾\\
\textamh{34.\  } & لَّهُم عَذَابٌۭ فِى ٱلحَيَوٰةِ ٱلدُّنيَا ۖ وَلَعَذَابُ ٱلءَاخِرَةِ أَشَقُّ ۖ وَمَا لَهُم مِّنَ ٱللَّهِ مِن وَاقٍۢ ﴿٣٤﴾\\
\textamh{35.\  } & ۞ مَّثَلُ ٱلجَنَّةِ ٱلَّتِى وُعِدَ ٱلمُتَّقُونَ ۖ تَجرِى مِن تَحتِهَا ٱلأَنهَـٰرُ ۖ أُكُلُهَا دَآئِمٌۭ وَظِلُّهَا ۚ تِلكَ عُقبَى ٱلَّذِينَ ٱتَّقَوا۟ ۖ وَّعُقبَى ٱلكَـٰفِرِينَ ٱلنَّارُ ﴿٣٥﴾\\
\textamh{36.\  } & وَٱلَّذِينَ ءَاتَينَـٰهُمُ ٱلكِتَـٰبَ يَفرَحُونَ بِمَآ أُنزِلَ إِلَيكَ ۖ وَمِنَ ٱلأَحزَابِ مَن يُنكِرُ بَعضَهُۥ ۚ قُل إِنَّمَآ أُمِرتُ أَن أَعبُدَ ٱللَّهَ وَلَآ أُشرِكَ بِهِۦٓ ۚ إِلَيهِ أَدعُوا۟ وَإِلَيهِ مَـَٔابِ ﴿٣٦﴾\\
\textamh{37.\  } & وَكَذَٟلِكَ أَنزَلنَـٰهُ حُكمًا عَرَبِيًّۭا ۚ وَلَئِنِ ٱتَّبَعتَ أَهوَآءَهُم بَعدَمَا جَآءَكَ مِنَ ٱلعِلمِ مَا لَكَ مِنَ ٱللَّهِ مِن وَلِىٍّۢ وَلَا وَاقٍۢ ﴿٣٧﴾\\
\textamh{38.\  } & وَلَقَد أَرسَلنَا رُسُلًۭا مِّن قَبلِكَ وَجَعَلنَا لَهُم أَزوَٟجًۭا وَذُرِّيَّةًۭ ۚ وَمَا كَانَ لِرَسُولٍ أَن يَأتِىَ بِـَٔايَةٍ إِلَّا بِإِذنِ ٱللَّهِ ۗ لِكُلِّ أَجَلٍۢ كِتَابٌۭ ﴿٣٨﴾\\
\textamh{39.\  } & يَمحُوا۟ ٱللَّهُ مَا يَشَآءُ وَيُثبِتُ ۖ وَعِندَهُۥٓ أُمُّ ٱلكِتَـٰبِ ﴿٣٩﴾\\
\textamh{40.\  } & وَإِن مَّا نُرِيَنَّكَ بَعضَ ٱلَّذِى نَعِدُهُم أَو نَتَوَفَّيَنَّكَ فَإِنَّمَا عَلَيكَ ٱلبَلَـٰغُ وَعَلَينَا ٱلحِسَابُ ﴿٤٠﴾\\
\textamh{41.\  } & أَوَلَم يَرَوا۟ أَنَّا نَأتِى ٱلأَرضَ نَنقُصُهَا مِن أَطرَافِهَا ۚ وَٱللَّهُ يَحكُمُ لَا مُعَقِّبَ لِحُكمِهِۦ ۚ وَهُوَ سَرِيعُ ٱلحِسَابِ ﴿٤١﴾\\
\textamh{42.\  } & وَقَد مَكَرَ ٱلَّذِينَ مِن قَبلِهِم فَلِلَّهِ ٱلمَكرُ جَمِيعًۭا ۖ يَعلَمُ مَا تَكسِبُ كُلُّ نَفسٍۢ ۗ وَسَيَعلَمُ ٱلكُفَّٰرُ لِمَن عُقبَى ٱلدَّارِ ﴿٤٢﴾\\
\textamh{43.\  } & وَيَقُولُ ٱلَّذِينَ كَفَرُوا۟ لَستَ مُرسَلًۭا ۚ قُل كَفَىٰ بِٱللَّهِ شَهِيدًۢا بَينِى وَبَينَكُم وَمَن عِندَهُۥ عِلمُ ٱلكِتَـٰبِ ﴿٤٣﴾\\
\end{longtable} \newpage
