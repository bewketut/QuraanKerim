%% License: BSD style (Berkley) (i.e. Put the Copyright owner's name always)
%% Writer and Copyright (to): Bewketu(Bilal) Tadilo (2016-17)
\shadowbox{\section{\LR{\textamharic{ሱራቱ አንነምል -}  \RL{سوره  النمل}}}}
\begin{longtable}{%
  @{}
    p{.5\textwidth}
  @{~~~~~~~~~~~~~}||
    p{.5\textwidth}
    @{}
}
\nopagebreak
\textamh{\ \ \ \ \ \  ቢስሚላሂ አራህመኒ ራሂይም } &  بِسمِ ٱللَّهِ ٱلرَّحمَـٰنِ ٱلرَّحِيمِ\\
\textamh{1.\  } &  طسٓ ۚ تِلكَ ءَايَـٰتُ ٱلقُرءَانِ وَكِتَابٍۢ مُّبِينٍ ﴿١﴾\\
\textamh{2.\  } & هُدًۭى وَبُشرَىٰ لِلمُؤمِنِينَ ﴿٢﴾\\
\textamh{3.\  } & ٱلَّذِينَ يُقِيمُونَ ٱلصَّلَوٰةَ وَيُؤتُونَ ٱلزَّكَوٰةَ وَهُم بِٱلءَاخِرَةِ هُم يُوقِنُونَ ﴿٣﴾\\
\textamh{4.\  } & إِنَّ ٱلَّذِينَ لَا يُؤمِنُونَ بِٱلءَاخِرَةِ زَيَّنَّا لَهُم أَعمَـٰلَهُم فَهُم يَعمَهُونَ ﴿٤﴾\\
\textamh{5.\  } & أُو۟لَـٰٓئِكَ ٱلَّذِينَ لَهُم سُوٓءُ ٱلعَذَابِ وَهُم فِى ٱلءَاخِرَةِ هُمُ ٱلأَخسَرُونَ ﴿٥﴾\\
\textamh{6.\  } & وَإِنَّكَ لَتُلَقَّى ٱلقُرءَانَ مِن لَّدُن حَكِيمٍ عَلِيمٍ ﴿٦﴾\\
\textamh{7.\  } & إِذ قَالَ مُوسَىٰ لِأَهلِهِۦٓ إِنِّىٓ ءَانَستُ نَارًۭا سَـَٔاتِيكُم مِّنهَا بِخَبَرٍ أَو ءَاتِيكُم بِشِهَابٍۢ قَبَسٍۢ لَّعَلَّكُم تَصطَلُونَ ﴿٧﴾\\
\textamh{8.\  } & فَلَمَّا جَآءَهَا نُودِىَ أَنۢ بُورِكَ مَن فِى ٱلنَّارِ وَمَن حَولَهَا وَسُبحَـٰنَ ٱللَّهِ رَبِّ ٱلعَـٰلَمِينَ ﴿٨﴾\\
\textamh{9.\  } & يَـٰمُوسَىٰٓ إِنَّهُۥٓ أَنَا ٱللَّهُ ٱلعَزِيزُ ٱلحَكِيمُ ﴿٩﴾\\
\textamh{10.\  } & وَأَلقِ عَصَاكَ ۚ فَلَمَّا رَءَاهَا تَهتَزُّ كَأَنَّهَا جَآنٌّۭ وَلَّىٰ مُدبِرًۭا وَلَم يُعَقِّب ۚ يَـٰمُوسَىٰ لَا تَخَف إِنِّى لَا يَخَافُ لَدَىَّ ٱلمُرسَلُونَ ﴿١٠﴾\\
\textamh{11.\  } & إِلَّا مَن ظَلَمَ ثُمَّ بَدَّلَ حُسنًۢا بَعدَ سُوٓءٍۢ فَإِنِّى غَفُورٌۭ رَّحِيمٌۭ ﴿١١﴾\\
\textamh{12.\  } & وَأَدخِل يَدَكَ فِى جَيبِكَ تَخرُج بَيضَآءَ مِن غَيرِ سُوٓءٍۢ ۖ فِى تِسعِ ءَايَـٰتٍ إِلَىٰ فِرعَونَ وَقَومِهِۦٓ ۚ إِنَّهُم كَانُوا۟ قَومًۭا فَـٰسِقِينَ ﴿١٢﴾\\
\textamh{13.\  } & فَلَمَّا جَآءَتهُم ءَايَـٰتُنَا مُبصِرَةًۭ قَالُوا۟ هَـٰذَا سِحرٌۭ مُّبِينٌۭ ﴿١٣﴾\\
\textamh{14.\  } & وَجَحَدُوا۟ بِهَا وَٱستَيقَنَتهَآ أَنفُسُهُم ظُلمًۭا وَعُلُوًّۭا ۚ فَٱنظُر كَيفَ كَانَ عَـٰقِبَةُ ٱلمُفسِدِينَ ﴿١٤﴾\\
\textamh{15.\  } & وَلَقَد ءَاتَينَا دَاوُۥدَ وَسُلَيمَـٰنَ عِلمًۭا ۖ وَقَالَا ٱلحَمدُ لِلَّهِ ٱلَّذِى فَضَّلَنَا عَلَىٰ كَثِيرٍۢ مِّن عِبَادِهِ ٱلمُؤمِنِينَ ﴿١٥﴾\\
\textamh{16.\  } & وَوَرِثَ سُلَيمَـٰنُ دَاوُۥدَ ۖ وَقَالَ يَـٰٓأَيُّهَا ٱلنَّاسُ عُلِّمنَا مَنطِقَ ٱلطَّيرِ وَأُوتِينَا مِن كُلِّ شَىءٍ ۖ إِنَّ هَـٰذَا لَهُوَ ٱلفَضلُ ٱلمُبِينُ ﴿١٦﴾\\
\textamh{17.\  } & وَحُشِرَ لِسُلَيمَـٰنَ جُنُودُهُۥ مِنَ ٱلجِنِّ وَٱلإِنسِ وَٱلطَّيرِ فَهُم يُوزَعُونَ ﴿١٧﴾\\
\textamh{18.\  } & حَتَّىٰٓ إِذَآ أَتَوا۟ عَلَىٰ وَادِ ٱلنَّملِ قَالَت نَملَةٌۭ يَـٰٓأَيُّهَا ٱلنَّملُ ٱدخُلُوا۟ مَسَـٰكِنَكُم لَا يَحطِمَنَّكُم سُلَيمَـٰنُ وَجُنُودُهُۥ وَهُم لَا يَشعُرُونَ ﴿١٨﴾\\
\textamh{19.\  } & فَتَبَسَّمَ ضَاحِكًۭا مِّن قَولِهَا وَقَالَ رَبِّ أَوزِعنِىٓ أَن أَشكُرَ نِعمَتَكَ ٱلَّتِىٓ أَنعَمتَ عَلَىَّ وَعَلَىٰ وَٟلِدَىَّ وَأَن أَعمَلَ صَـٰلِحًۭا تَرضَىٰهُ وَأَدخِلنِى بِرَحمَتِكَ فِى عِبَادِكَ ٱلصَّـٰلِحِينَ ﴿١٩﴾\\
\textamh{20.\  } & وَتَفَقَّدَ ٱلطَّيرَ فَقَالَ مَا لِىَ لَآ أَرَى ٱلهُدهُدَ أَم كَانَ مِنَ ٱلغَآئِبِينَ ﴿٢٠﴾\\
\textamh{21.\  } & لَأُعَذِّبَنَّهُۥ عَذَابًۭا شَدِيدًا أَو لَأَا۟ذبَحَنَّهُۥٓ أَو لَيَأتِيَنِّى بِسُلطَٰنٍۢ مُّبِينٍۢ ﴿٢١﴾\\
\textamh{22.\  } & فَمَكَثَ غَيرَ بَعِيدٍۢ فَقَالَ أَحَطتُ بِمَا لَم تُحِط بِهِۦ وَجِئتُكَ مِن سَبَإٍۭ بِنَبَإٍۢ يَقِينٍ ﴿٢٢﴾\\
\textamh{23.\  } & إِنِّى وَجَدتُّ ٱمرَأَةًۭ تَملِكُهُم وَأُوتِيَت مِن كُلِّ شَىءٍۢ وَلَهَا عَرشٌ عَظِيمٌۭ ﴿٢٣﴾\\
\textamh{24.\  } & وَجَدتُّهَا وَقَومَهَا يَسجُدُونَ لِلشَّمسِ مِن دُونِ ٱللَّهِ وَزَيَّنَ لَهُمُ ٱلشَّيطَٰنُ أَعمَـٰلَهُم فَصَدَّهُم عَنِ ٱلسَّبِيلِ فَهُم لَا يَهتَدُونَ ﴿٢٤﴾\\
\textamh{25.\  } & أَلَّا يَسجُدُوا۟ لِلَّهِ ٱلَّذِى يُخرِجُ ٱلخَبءَ فِى ٱلسَّمَـٰوَٟتِ وَٱلأَرضِ وَيَعلَمُ مَا تُخفُونَ وَمَا تُعلِنُونَ ﴿٢٥﴾\\
\textamh{26.\  } & ٱللَّهُ لَآ إِلَـٰهَ إِلَّا هُوَ رَبُّ ٱلعَرشِ ٱلعَظِيمِ ۩ ﴿٢٦﴾\\
\textamh{27.\  } & ۞ قَالَ سَنَنظُرُ أَصَدَقتَ أَم كُنتَ مِنَ ٱلكَـٰذِبِينَ ﴿٢٧﴾\\
\textamh{28.\  } & ٱذهَب بِّكِتَـٰبِى هَـٰذَا فَأَلقِه إِلَيهِم ثُمَّ تَوَلَّ عَنهُم فَٱنظُر مَاذَا يَرجِعُونَ ﴿٢٨﴾\\
\textamh{29.\  } & قَالَت يَـٰٓأَيُّهَا ٱلمَلَؤُا۟ إِنِّىٓ أُلقِىَ إِلَىَّ كِتَـٰبٌۭ كَرِيمٌ ﴿٢٩﴾\\
\textamh{30.\  } & إِنَّهُۥ مِن سُلَيمَـٰنَ وَإِنَّهُۥ  ﴿٣٠﴾\\
\textamh{31.\  } & أَلَّا تَعلُوا۟ عَلَىَّ وَأتُونِى مُسلِمِينَ ﴿٣١﴾\\
\textamh{32.\  } & قَالَت يَـٰٓأَيُّهَا ٱلمَلَؤُا۟ أَفتُونِى فِىٓ أَمرِى مَا كُنتُ قَاطِعَةً أَمرًا حَتَّىٰ تَشهَدُونِ ﴿٣٢﴾\\
\textamh{33.\  } & قَالُوا۟ نَحنُ أُو۟لُوا۟ قُوَّةٍۢ وَأُو۟لُوا۟ بَأسٍۢ شَدِيدٍۢ وَٱلأَمرُ إِلَيكِ فَٱنظُرِى مَاذَا تَأمُرِينَ ﴿٣٣﴾\\
\textamh{34.\  } & قَالَت إِنَّ ٱلمُلُوكَ إِذَا دَخَلُوا۟ قَريَةً أَفسَدُوهَا وَجَعَلُوٓا۟ أَعِزَّةَ أَهلِهَآ أَذِلَّةًۭ ۖ وَكَذَٟلِكَ يَفعَلُونَ ﴿٣٤﴾\\
\textamh{35.\  } & وَإِنِّى مُرسِلَةٌ إِلَيهِم بِهَدِيَّةٍۢ فَنَاظِرَةٌۢ بِمَ يَرجِعُ ٱلمُرسَلُونَ ﴿٣٥﴾\\
\textamh{36.\  } & فَلَمَّا جَآءَ سُلَيمَـٰنَ قَالَ أَتُمِدُّونَنِ بِمَالٍۢ فَمَآ ءَاتَىٰنِۦَ ٱللَّهُ خَيرٌۭ مِّمَّآ ءَاتَىٰكُم بَل أَنتُم بِهَدِيَّتِكُم تَفرَحُونَ ﴿٣٦﴾\\
\textamh{37.\  } & ٱرجِع إِلَيهِم فَلَنَأتِيَنَّهُم بِجُنُودٍۢ لَّا قِبَلَ لَهُم بِهَا وَلَنُخرِجَنَّهُم مِّنهَآ أَذِلَّةًۭ وَهُم صَـٰغِرُونَ ﴿٣٧﴾\\
\textamh{38.\  } & قَالَ يَـٰٓأَيُّهَا ٱلمَلَؤُا۟ أَيُّكُم يَأتِينِى بِعَرشِهَا قَبلَ أَن يَأتُونِى مُسلِمِينَ ﴿٣٨﴾\\
\textamh{39.\  } & قَالَ عِفرِيتٌۭ مِّنَ ٱلجِنِّ أَنَا۠ ءَاتِيكَ بِهِۦ قَبلَ أَن تَقُومَ مِن مَّقَامِكَ ۖ وَإِنِّى عَلَيهِ لَقَوِىٌّ أَمِينٌۭ ﴿٣٩﴾\\
\textamh{40.\  } & قَالَ ٱلَّذِى عِندَهُۥ عِلمٌۭ مِّنَ ٱلكِتَـٰبِ أَنَا۠ ءَاتِيكَ بِهِۦ قَبلَ أَن يَرتَدَّ إِلَيكَ طَرفُكَ ۚ فَلَمَّا رَءَاهُ مُستَقِرًّا عِندَهُۥ قَالَ هَـٰذَا مِن فَضلِ رَبِّى لِيَبلُوَنِىٓ ءَأَشكُرُ أَم أَكفُرُ ۖ وَمَن شَكَرَ فَإِنَّمَا يَشكُرُ لِنَفسِهِۦ ۖ وَمَن كَفَرَ فَإِنَّ رَبِّى غَنِىٌّۭ كَرِيمٌۭ ﴿٤٠﴾\\
\textamh{41.\  } & قَالَ نَكِّرُوا۟ لَهَا عَرشَهَا نَنظُر أَتَهتَدِىٓ أَم تَكُونُ مِنَ ٱلَّذِينَ لَا يَهتَدُونَ ﴿٤١﴾\\
\textamh{42.\  } & فَلَمَّا جَآءَت قِيلَ أَهَـٰكَذَا عَرشُكِ ۖ قَالَت كَأَنَّهُۥ هُوَ ۚ وَأُوتِينَا ٱلعِلمَ مِن قَبلِهَا وَكُنَّا مُسلِمِينَ ﴿٤٢﴾\\
\textamh{43.\  } & وَصَدَّهَا مَا كَانَت تَّعبُدُ مِن دُونِ ٱللَّهِ ۖ إِنَّهَا كَانَت مِن قَومٍۢ كَـٰفِرِينَ ﴿٤٣﴾\\
\textamh{44.\  } & قِيلَ لَهَا ٱدخُلِى ٱلصَّرحَ ۖ فَلَمَّا رَأَتهُ حَسِبَتهُ لُجَّةًۭ وَكَشَفَت عَن سَاقَيهَا ۚ قَالَ إِنَّهُۥ صَرحٌۭ مُّمَرَّدٌۭ مِّن قَوَارِيرَ ۗ قَالَت رَبِّ إِنِّى ظَلَمتُ نَفسِى وَأَسلَمتُ مَعَ سُلَيمَـٰنَ لِلَّهِ رَبِّ ٱلعَـٰلَمِينَ ﴿٤٤﴾\\
\textamh{45.\  } & وَلَقَد أَرسَلنَآ إِلَىٰ ثَمُودَ أَخَاهُم صَـٰلِحًا أَنِ ٱعبُدُوا۟ ٱللَّهَ فَإِذَا هُم فَرِيقَانِ يَختَصِمُونَ ﴿٤٥﴾\\
\textamh{46.\  } & قَالَ يَـٰقَومِ لِمَ تَستَعجِلُونَ بِٱلسَّيِّئَةِ قَبلَ ٱلحَسَنَةِ ۖ لَولَا تَستَغفِرُونَ ٱللَّهَ لَعَلَّكُم تُرحَمُونَ ﴿٤٦﴾\\
\textamh{47.\  } & قَالُوا۟ ٱطَّيَّرنَا بِكَ وَبِمَن مَّعَكَ ۚ قَالَ طَٰٓئِرُكُم عِندَ ٱللَّهِ ۖ بَل أَنتُم قَومٌۭ تُفتَنُونَ ﴿٤٧﴾\\
\textamh{48.\  } & وَكَانَ فِى ٱلمَدِينَةِ تِسعَةُ رَهطٍۢ يُفسِدُونَ فِى ٱلأَرضِ وَلَا يُصلِحُونَ ﴿٤٨﴾\\
\textamh{49.\  } & قَالُوا۟ تَقَاسَمُوا۟ بِٱللَّهِ لَنُبَيِّتَنَّهُۥ وَأَهلَهُۥ ثُمَّ لَنَقُولَنَّ لِوَلِيِّهِۦ مَا شَهِدنَا مَهلِكَ أَهلِهِۦ وَإِنَّا لَصَـٰدِقُونَ ﴿٤٩﴾\\
\textamh{50.\  } & وَمَكَرُوا۟ مَكرًۭا وَمَكَرنَا مَكرًۭا وَهُم لَا يَشعُرُونَ ﴿٥٠﴾\\
\textamh{51.\  } & فَٱنظُر كَيفَ كَانَ عَـٰقِبَةُ مَكرِهِم أَنَّا دَمَّرنَـٰهُم وَقَومَهُم أَجمَعِينَ ﴿٥١﴾\\
\textamh{52.\  } & فَتِلكَ بُيُوتُهُم خَاوِيَةًۢ بِمَا ظَلَمُوٓا۟ ۗ إِنَّ فِى ذَٟلِكَ لَءَايَةًۭ لِّقَومٍۢ يَعلَمُونَ ﴿٥٢﴾\\
\textamh{53.\  } & وَأَنجَينَا ٱلَّذِينَ ءَامَنُوا۟ وَكَانُوا۟ يَتَّقُونَ ﴿٥٣﴾\\
\textamh{54.\  } & وَلُوطًا إِذ قَالَ لِقَومِهِۦٓ أَتَأتُونَ ٱلفَـٰحِشَةَ وَأَنتُم تُبصِرُونَ ﴿٥٤﴾\\
\textamh{55.\  } & أَئِنَّكُم لَتَأتُونَ ٱلرِّجَالَ شَهوَةًۭ مِّن دُونِ ٱلنِّسَآءِ ۚ بَل أَنتُم قَومٌۭ تَجهَلُونَ ﴿٥٥﴾\\
\textamh{56.\  } & ۞ فَمَا كَانَ جَوَابَ قَومِهِۦٓ إِلَّآ أَن قَالُوٓا۟ أَخرِجُوٓا۟ ءَالَ لُوطٍۢ مِّن قَريَتِكُم ۖ إِنَّهُم أُنَاسٌۭ يَتَطَهَّرُونَ ﴿٥٦﴾\\
\textamh{57.\  } & فَأَنجَينَـٰهُ وَأَهلَهُۥٓ إِلَّا ٱمرَأَتَهُۥ قَدَّرنَـٰهَا مِنَ ٱلغَٰبِرِينَ ﴿٥٧﴾\\
\textamh{58.\  } & وَأَمطَرنَا عَلَيهِم مَّطَرًۭا ۖ فَسَآءَ مَطَرُ ٱلمُنذَرِينَ ﴿٥٨﴾\\
\textamh{59.\  } & قُلِ ٱلحَمدُ لِلَّهِ وَسَلَـٰمٌ عَلَىٰ عِبَادِهِ ٱلَّذِينَ ٱصطَفَىٰٓ ۗ ءَآللَّهُ خَيرٌ أَمَّا يُشرِكُونَ ﴿٥٩﴾\\
\textamh{60.\  } & أَمَّن خَلَقَ ٱلسَّمَـٰوَٟتِ وَٱلأَرضَ وَأَنزَلَ لَكُم مِّنَ ٱلسَّمَآءِ مَآءًۭ فَأَنۢبَتنَا بِهِۦ حَدَآئِقَ ذَاتَ بَهجَةٍۢ مَّا كَانَ لَكُم أَن تُنۢبِتُوا۟ شَجَرَهَآ ۗ أَءِلَـٰهٌۭ مَّعَ ٱللَّهِ ۚ بَل هُم قَومٌۭ يَعدِلُونَ ﴿٦٠﴾\\
\textamh{61.\  } & أَمَّن جَعَلَ ٱلأَرضَ قَرَارًۭا وَجَعَلَ خِلَـٰلَهَآ أَنهَـٰرًۭا وَجَعَلَ لَهَا رَوَٟسِىَ وَجَعَلَ بَينَ ٱلبَحرَينِ حَاجِزًا ۗ أَءِلَـٰهٌۭ مَّعَ ٱللَّهِ ۚ بَل أَكثَرُهُم لَا يَعلَمُونَ ﴿٦١﴾\\
\textamh{62.\  } & أَمَّن يُجِيبُ ٱلمُضطَرَّ إِذَا دَعَاهُ وَيَكشِفُ ٱلسُّوٓءَ وَيَجعَلُكُم خُلَفَآءَ ٱلأَرضِ ۗ أَءِلَـٰهٌۭ مَّعَ ٱللَّهِ ۚ قَلِيلًۭا مَّا تَذَكَّرُونَ ﴿٦٢﴾\\
\textamh{63.\  } & أَمَّن يَهدِيكُم فِى ظُلُمَـٰتِ ٱلبَرِّ وَٱلبَحرِ وَمَن يُرسِلُ ٱلرِّيَـٰحَ بُشرًۢا بَينَ يَدَى رَحمَتِهِۦٓ ۗ أَءِلَـٰهٌۭ مَّعَ ٱللَّهِ ۚ تَعَـٰلَى ٱللَّهُ عَمَّا يُشرِكُونَ ﴿٦٣﴾\\
\textamh{64.\  } & أَمَّن يَبدَؤُا۟ ٱلخَلقَ ثُمَّ يُعِيدُهُۥ وَمَن يَرزُقُكُم مِّنَ ٱلسَّمَآءِ وَٱلأَرضِ ۗ أَءِلَـٰهٌۭ مَّعَ ٱللَّهِ ۚ قُل هَاتُوا۟ بُرهَـٰنَكُم إِن كُنتُم صَـٰدِقِينَ ﴿٦٤﴾\\
\textamh{65.\  } & قُل لَّا يَعلَمُ مَن فِى ٱلسَّمَـٰوَٟتِ وَٱلأَرضِ ٱلغَيبَ إِلَّا ٱللَّهُ ۚ وَمَا يَشعُرُونَ أَيَّانَ يُبعَثُونَ ﴿٦٥﴾\\
\textamh{66.\  } & بَلِ ٱدَّٰرَكَ عِلمُهُم فِى ٱلءَاخِرَةِ ۚ بَل هُم فِى شَكٍّۢ مِّنهَا ۖ بَل هُم مِّنهَا عَمُونَ ﴿٦٦﴾\\
\textamh{67.\  } & وَقَالَ ٱلَّذِينَ كَفَرُوٓا۟ أَءِذَا كُنَّا تُرَٰبًۭا وَءَابَآؤُنَآ أَئِنَّا لَمُخرَجُونَ ﴿٦٧﴾\\
\textamh{68.\  } & لَقَد وُعِدنَا هَـٰذَا نَحنُ وَءَابَآؤُنَا مِن قَبلُ إِن هَـٰذَآ إِلَّآ أَسَـٰطِيرُ ٱلأَوَّلِينَ ﴿٦٨﴾\\
\textamh{69.\  } & قُل سِيرُوا۟ فِى ٱلأَرضِ فَٱنظُرُوا۟ كَيفَ كَانَ عَـٰقِبَةُ ٱلمُجرِمِينَ ﴿٦٩﴾\\
\textamh{70.\  } & وَلَا تَحزَن عَلَيهِم وَلَا تَكُن فِى ضَيقٍۢ مِّمَّا يَمكُرُونَ ﴿٧٠﴾\\
\textamh{71.\  } & وَيَقُولُونَ مَتَىٰ هَـٰذَا ٱلوَعدُ إِن كُنتُم صَـٰدِقِينَ ﴿٧١﴾\\
\textamh{72.\  } & قُل عَسَىٰٓ أَن يَكُونَ رَدِفَ لَكُم بَعضُ ٱلَّذِى تَستَعجِلُونَ ﴿٧٢﴾\\
\textamh{73.\  } & وَإِنَّ رَبَّكَ لَذُو فَضلٍ عَلَى ٱلنَّاسِ وَلَـٰكِنَّ أَكثَرَهُم لَا يَشكُرُونَ ﴿٧٣﴾\\
\textamh{74.\  } & وَإِنَّ رَبَّكَ لَيَعلَمُ مَا تُكِنُّ صُدُورُهُم وَمَا يُعلِنُونَ ﴿٧٤﴾\\
\textamh{75.\  } & وَمَا مِن غَآئِبَةٍۢ فِى ٱلسَّمَآءِ وَٱلأَرضِ إِلَّا فِى كِتَـٰبٍۢ مُّبِينٍ ﴿٧٥﴾\\
\textamh{76.\  } & إِنَّ هَـٰذَا ٱلقُرءَانَ يَقُصُّ عَلَىٰ بَنِىٓ إِسرَٰٓءِيلَ أَكثَرَ ٱلَّذِى هُم فِيهِ يَختَلِفُونَ ﴿٧٦﴾\\
\textamh{77.\  } & وَإِنَّهُۥ لَهُدًۭى وَرَحمَةٌۭ لِّلمُؤمِنِينَ ﴿٧٧﴾\\
\textamh{78.\  } & إِنَّ رَبَّكَ يَقضِى بَينَهُم بِحُكمِهِۦ ۚ وَهُوَ ٱلعَزِيزُ ٱلعَلِيمُ ﴿٧٨﴾\\
\textamh{79.\  } & فَتَوَكَّل عَلَى ٱللَّهِ ۖ إِنَّكَ عَلَى ٱلحَقِّ ٱلمُبِينِ ﴿٧٩﴾\\
\textamh{80.\  } & إِنَّكَ لَا تُسمِعُ ٱلمَوتَىٰ وَلَا تُسمِعُ ٱلصُّمَّ ٱلدُّعَآءَ إِذَا وَلَّوا۟ مُدبِرِينَ ﴿٨٠﴾\\
\textamh{81.\  } & وَمَآ أَنتَ بِهَـٰدِى ٱلعُمىِ عَن ضَلَـٰلَتِهِم ۖ إِن تُسمِعُ إِلَّا مَن يُؤمِنُ بِـَٔايَـٰتِنَا فَهُم مُّسلِمُونَ ﴿٨١﴾\\
\textamh{82.\  } & ۞ وَإِذَا وَقَعَ ٱلقَولُ عَلَيهِم أَخرَجنَا لَهُم دَآبَّةًۭ مِّنَ ٱلأَرضِ تُكَلِّمُهُم أَنَّ ٱلنَّاسَ كَانُوا۟ بِـَٔايَـٰتِنَا لَا يُوقِنُونَ ﴿٨٢﴾\\
\textamh{83.\  } & وَيَومَ نَحشُرُ مِن كُلِّ أُمَّةٍۢ فَوجًۭا مِّمَّن يُكَذِّبُ بِـَٔايَـٰتِنَا فَهُم يُوزَعُونَ ﴿٨٣﴾\\
\textamh{84.\  } & حَتَّىٰٓ إِذَا جَآءُو قَالَ أَكَذَّبتُم بِـَٔايَـٰتِى وَلَم تُحِيطُوا۟ بِهَا عِلمًا أَمَّاذَا كُنتُم تَعمَلُونَ ﴿٨٤﴾\\
\textamh{85.\  } & وَوَقَعَ ٱلقَولُ عَلَيهِم بِمَا ظَلَمُوا۟ فَهُم لَا يَنطِقُونَ ﴿٨٥﴾\\
\textamh{86.\  } & أَلَم يَرَوا۟ أَنَّا جَعَلنَا ٱلَّيلَ لِيَسكُنُوا۟ فِيهِ وَٱلنَّهَارَ مُبصِرًا ۚ إِنَّ فِى ذَٟلِكَ لَءَايَـٰتٍۢ لِّقَومٍۢ يُؤمِنُونَ ﴿٨٦﴾\\
\textamh{87.\  } & وَيَومَ يُنفَخُ فِى ٱلصُّورِ فَفَزِعَ مَن فِى ٱلسَّمَـٰوَٟتِ وَمَن فِى ٱلأَرضِ إِلَّا مَن شَآءَ ٱللَّهُ ۚ وَكُلٌّ أَتَوهُ دَٟخِرِينَ ﴿٨٧﴾\\
\textamh{88.\  } & وَتَرَى ٱلجِبَالَ تَحسَبُهَا جَامِدَةًۭ وَهِىَ تَمُرُّ مَرَّ ٱلسَّحَابِ ۚ صُنعَ ٱللَّهِ ٱلَّذِىٓ أَتقَنَ كُلَّ شَىءٍ ۚ إِنَّهُۥ خَبِيرٌۢ بِمَا تَفعَلُونَ ﴿٨٨﴾\\
\textamh{89.\  } & مَن جَآءَ بِٱلحَسَنَةِ فَلَهُۥ خَيرٌۭ مِّنهَا وَهُم مِّن فَزَعٍۢ يَومَئِذٍ ءَامِنُونَ ﴿٨٩﴾\\
\textamh{90.\  } & وَمَن جَآءَ بِٱلسَّيِّئَةِ فَكُبَّت وُجُوهُهُم فِى ٱلنَّارِ هَل تُجزَونَ إِلَّا مَا كُنتُم تَعمَلُونَ ﴿٩٠﴾\\
\textamh{91.\  } & إِنَّمَآ أُمِرتُ أَن أَعبُدَ رَبَّ هَـٰذِهِ ٱلبَلدَةِ ٱلَّذِى حَرَّمَهَا وَلَهُۥ كُلُّ شَىءٍۢ ۖ وَأُمِرتُ أَن أَكُونَ مِنَ ٱلمُسلِمِينَ ﴿٩١﴾\\
\textamh{92.\  } & وَأَن أَتلُوَا۟ ٱلقُرءَانَ ۖ فَمَنِ ٱهتَدَىٰ فَإِنَّمَا يَهتَدِى لِنَفسِهِۦ ۖ وَمَن ضَلَّ فَقُل إِنَّمَآ أَنَا۠ مِنَ ٱلمُنذِرِينَ ﴿٩٢﴾\\
\textamh{93.\  } & وَقُلِ ٱلحَمدُ لِلَّهِ سَيُرِيكُم ءَايَـٰتِهِۦ فَتَعرِفُونَهَا ۚ وَمَا رَبُّكَ بِغَٰفِلٍ عَمَّا تَعمَلُونَ ﴿٩٣﴾\\
\end{longtable} \newpage
