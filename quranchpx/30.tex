%% License: BSD style (Berkley) (i.e. Put the Copyright owner's name always)
%% Writer and Copyright (to): Bewketu(Bilal) Tadilo (2016-17)
\shadowbox{\section{\LR{\textamharic{ሱራቱ አርሩም -}  \RL{سوره  الروم}}}}
\begin{longtable}{%
  @{}
    p{.5\textwidth}
  @{~~~~~~~~~~~~~}||
    p{.5\textwidth}
    @{}
}
\nopagebreak
\textamh{\ \ \ \ \ \  ቢስሚላሂ አራህመኒ ራሂይም } &  بِسمِ ٱللَّهِ ٱلرَّحمَـٰنِ ٱلرَّحِيمِ\\
\textamh{1.\  } &  الٓمٓ ﴿١﴾\\
\textamh{2.\  } & غُلِبَتِ ٱلرُّومُ ﴿٢﴾\\
\textamh{3.\  } & فِىٓ أَدنَى ٱلأَرضِ وَهُم مِّنۢ بَعدِ غَلَبِهِم سَيَغلِبُونَ ﴿٣﴾\\
\textamh{4.\  } & فِى بِضعِ سِنِينَ ۗ لِلَّهِ ٱلأَمرُ مِن قَبلُ وَمِنۢ بَعدُ ۚ وَيَومَئِذٍۢ يَفرَحُ ٱلمُؤمِنُونَ ﴿٤﴾\\
\textamh{5.\  } & بِنَصرِ ٱللَّهِ ۚ يَنصُرُ مَن يَشَآءُ ۖ وَهُوَ ٱلعَزِيزُ ٱلرَّحِيمُ ﴿٥﴾\\
\textamh{6.\  } & وَعدَ ٱللَّهِ ۖ لَا يُخلِفُ ٱللَّهُ وَعدَهُۥ وَلَـٰكِنَّ أَكثَرَ ٱلنَّاسِ لَا يَعلَمُونَ ﴿٦﴾\\
\textamh{7.\  } & يَعلَمُونَ ظَـٰهِرًۭا مِّنَ ٱلحَيَوٰةِ ٱلدُّنيَا وَهُم عَنِ ٱلءَاخِرَةِ هُم غَٰفِلُونَ ﴿٧﴾\\
\textamh{8.\  } & أَوَلَم يَتَفَكَّرُوا۟ فِىٓ أَنفُسِهِم ۗ مَّا خَلَقَ ٱللَّهُ ٱلسَّمَـٰوَٟتِ وَٱلأَرضَ وَمَا بَينَهُمَآ إِلَّا بِٱلحَقِّ وَأَجَلٍۢ مُّسَمًّۭى ۗ وَإِنَّ كَثِيرًۭا مِّنَ ٱلنَّاسِ بِلِقَآئِ رَبِّهِم لَكَـٰفِرُونَ ﴿٨﴾\\
\textamh{9.\  } & أَوَلَم يَسِيرُوا۟ فِى ٱلأَرضِ فَيَنظُرُوا۟ كَيفَ كَانَ عَـٰقِبَةُ ٱلَّذِينَ مِن قَبلِهِم ۚ كَانُوٓا۟ أَشَدَّ مِنهُم قُوَّةًۭ وَأَثَارُوا۟ ٱلأَرضَ وَعَمَرُوهَآ أَكثَرَ مِمَّا عَمَرُوهَا وَجَآءَتهُم رُسُلُهُم بِٱلبَيِّنَـٰتِ ۖ فَمَا كَانَ ٱللَّهُ لِيَظلِمَهُم وَلَـٰكِن كَانُوٓا۟ أَنفُسَهُم يَظلِمُونَ ﴿٩﴾\\
\textamh{10.\  } & ثُمَّ كَانَ عَـٰقِبَةَ ٱلَّذِينَ أَسَـٰٓـُٔوا۟ ٱلسُّوٓأَىٰٓ أَن كَذَّبُوا۟ بِـَٔايَـٰتِ ٱللَّهِ وَكَانُوا۟ بِهَا يَستَهزِءُونَ ﴿١٠﴾\\
\textamh{11.\  } & ٱللَّهُ يَبدَؤُا۟ ٱلخَلقَ ثُمَّ يُعِيدُهُۥ ثُمَّ إِلَيهِ تُرجَعُونَ ﴿١١﴾\\
\textamh{12.\  } & وَيَومَ تَقُومُ ٱلسَّاعَةُ يُبلِسُ ٱلمُجرِمُونَ ﴿١٢﴾\\
\textamh{13.\  } & وَلَم يَكُن لَّهُم مِّن شُرَكَآئِهِم شُفَعَـٰٓؤُا۟ وَكَانُوا۟ بِشُرَكَآئِهِم كَـٰفِرِينَ ﴿١٣﴾\\
\textamh{14.\  } & وَيَومَ تَقُومُ ٱلسَّاعَةُ يَومَئِذٍۢ يَتَفَرَّقُونَ ﴿١٤﴾\\
\textamh{15.\  } & فَأَمَّا ٱلَّذِينَ ءَامَنُوا۟ وَعَمِلُوا۟ ٱلصَّـٰلِحَـٰتِ فَهُم فِى رَوضَةٍۢ يُحبَرُونَ ﴿١٥﴾\\
\textamh{16.\  } & وَأَمَّا ٱلَّذِينَ كَفَرُوا۟ وَكَذَّبُوا۟ بِـَٔايَـٰتِنَا وَلِقَآئِ ٱلءَاخِرَةِ فَأُو۟لَـٰٓئِكَ فِى ٱلعَذَابِ مُحضَرُونَ ﴿١٦﴾\\
\textamh{17.\  } & فَسُبحَـٰنَ ٱللَّهِ حِينَ تُمسُونَ وَحِينَ تُصبِحُونَ ﴿١٧﴾\\
\textamh{18.\  } & وَلَهُ ٱلحَمدُ فِى ٱلسَّمَـٰوَٟتِ وَٱلأَرضِ وَعَشِيًّۭا وَحِينَ تُظهِرُونَ ﴿١٨﴾\\
\textamh{19.\  } & يُخرِجُ ٱلحَىَّ مِنَ ٱلمَيِّتِ وَيُخرِجُ ٱلمَيِّتَ مِنَ ٱلحَىِّ وَيُحىِ ٱلأَرضَ بَعدَ مَوتِهَا ۚ وَكَذَٟلِكَ تُخرَجُونَ ﴿١٩﴾\\
\textamh{20.\  } & وَمِن ءَايَـٰتِهِۦٓ أَن خَلَقَكُم مِّن تُرَابٍۢ ثُمَّ إِذَآ أَنتُم بَشَرٌۭ تَنتَشِرُونَ ﴿٢٠﴾\\
\textamh{21.\  } & وَمِن ءَايَـٰتِهِۦٓ أَن خَلَقَ لَكُم مِّن أَنفُسِكُم أَزوَٟجًۭا لِّتَسكُنُوٓا۟ إِلَيهَا وَجَعَلَ بَينَكُم مَّوَدَّةًۭ وَرَحمَةً ۚ إِنَّ فِى ذَٟلِكَ لَءَايَـٰتٍۢ لِّقَومٍۢ يَتَفَكَّرُونَ ﴿٢١﴾\\
\textamh{22.\  } & وَمِن ءَايَـٰتِهِۦ خَلقُ ٱلسَّمَـٰوَٟتِ وَٱلأَرضِ وَٱختِلَـٰفُ أَلسِنَتِكُم وَأَلوَٟنِكُم ۚ إِنَّ فِى ذَٟلِكَ لَءَايَـٰتٍۢ لِّلعَـٰلِمِينَ ﴿٢٢﴾\\
\textamh{23.\  } & وَمِن ءَايَـٰتِهِۦ مَنَامُكُم بِٱلَّيلِ وَٱلنَّهَارِ وَٱبتِغَآؤُكُم مِّن فَضلِهِۦٓ ۚ إِنَّ فِى ذَٟلِكَ لَءَايَـٰتٍۢ لِّقَومٍۢ يَسمَعُونَ ﴿٢٣﴾\\
\textamh{24.\  } & وَمِن ءَايَـٰتِهِۦ يُرِيكُمُ ٱلبَرقَ خَوفًۭا وَطَمَعًۭا وَيُنَزِّلُ مِنَ ٱلسَّمَآءِ مَآءًۭ فَيُحىِۦ بِهِ ٱلأَرضَ بَعدَ مَوتِهَآ ۚ إِنَّ فِى ذَٟلِكَ لَءَايَـٰتٍۢ لِّقَومٍۢ يَعقِلُونَ ﴿٢٤﴾\\
\textamh{25.\  } & وَمِن ءَايَـٰتِهِۦٓ أَن تَقُومَ ٱلسَّمَآءُ وَٱلأَرضُ بِأَمرِهِۦ ۚ ثُمَّ إِذَا دَعَاكُم دَعوَةًۭ مِّنَ ٱلأَرضِ إِذَآ أَنتُم تَخرُجُونَ ﴿٢٥﴾\\
\textamh{26.\  } & وَلَهُۥ مَن فِى ٱلسَّمَـٰوَٟتِ وَٱلأَرضِ ۖ كُلٌّۭ لَّهُۥ قَـٰنِتُونَ ﴿٢٦﴾\\
\textamh{27.\  } & وَهُوَ ٱلَّذِى يَبدَؤُا۟ ٱلخَلقَ ثُمَّ يُعِيدُهُۥ وَهُوَ أَهوَنُ عَلَيهِ ۚ وَلَهُ ٱلمَثَلُ ٱلأَعلَىٰ فِى ٱلسَّمَـٰوَٟتِ وَٱلأَرضِ ۚ وَهُوَ ٱلعَزِيزُ ٱلحَكِيمُ ﴿٢٧﴾\\
\textamh{28.\  } & ضَرَبَ لَكُم مَّثَلًۭا مِّن أَنفُسِكُم ۖ هَل لَّكُم مِّن مَّا مَلَكَت أَيمَـٰنُكُم مِّن شُرَكَآءَ فِى مَا رَزَقنَـٰكُم فَأَنتُم فِيهِ سَوَآءٌۭ تَخَافُونَهُم كَخِيفَتِكُم أَنفُسَكُم ۚ كَذَٟلِكَ نُفَصِّلُ ٱلءَايَـٰتِ لِقَومٍۢ يَعقِلُونَ ﴿٢٨﴾\\
\textamh{29.\  } & بَلِ ٱتَّبَعَ ٱلَّذِينَ ظَلَمُوٓا۟ أَهوَآءَهُم بِغَيرِ عِلمٍۢ ۖ فَمَن يَهدِى مَن أَضَلَّ ٱللَّهُ ۖ وَمَا لَهُم مِّن نَّـٰصِرِينَ ﴿٢٩﴾\\
\textamh{30.\  } & فَأَقِم وَجهَكَ لِلدِّينِ حَنِيفًۭا ۚ فِطرَتَ ٱللَّهِ ٱلَّتِى فَطَرَ ٱلنَّاسَ عَلَيهَا ۚ لَا تَبدِيلَ لِخَلقِ ٱللَّهِ ۚ ذَٟلِكَ ٱلدِّينُ ٱلقَيِّمُ وَلَـٰكِنَّ أَكثَرَ ٱلنَّاسِ لَا يَعلَمُونَ ﴿٣٠﴾\\
\textamh{31.\  } & ۞ مُنِيبِينَ إِلَيهِ وَٱتَّقُوهُ وَأَقِيمُوا۟ ٱلصَّلَوٰةَ وَلَا تَكُونُوا۟ مِنَ ٱلمُشرِكِينَ ﴿٣١﴾\\
\textamh{32.\  } & مِنَ ٱلَّذِينَ فَرَّقُوا۟ دِينَهُم وَكَانُوا۟ شِيَعًۭا ۖ كُلُّ حِزبٍۭ بِمَا لَدَيهِم فَرِحُونَ ﴿٣٢﴾\\
\textamh{33.\  } & وَإِذَا مَسَّ ٱلنَّاسَ ضُرٌّۭ دَعَوا۟ رَبَّهُم مُّنِيبِينَ إِلَيهِ ثُمَّ إِذَآ أَذَاقَهُم مِّنهُ رَحمَةً إِذَا فَرِيقٌۭ مِّنهُم بِرَبِّهِم يُشرِكُونَ ﴿٣٣﴾\\
\textamh{34.\  } & لِيَكفُرُوا۟ بِمَآ ءَاتَينَـٰهُم ۚ فَتَمَتَّعُوا۟ فَسَوفَ تَعلَمُونَ ﴿٣٤﴾\\
\textamh{35.\  } & أَم أَنزَلنَا عَلَيهِم سُلطَٰنًۭا فَهُوَ يَتَكَلَّمُ بِمَا كَانُوا۟ بِهِۦ يُشرِكُونَ ﴿٣٥﴾\\
\textamh{36.\  } & وَإِذَآ أَذَقنَا ٱلنَّاسَ رَحمَةًۭ فَرِحُوا۟ بِهَا ۖ وَإِن تُصِبهُم سَيِّئَةٌۢ بِمَا قَدَّمَت أَيدِيهِم إِذَا هُم يَقنَطُونَ ﴿٣٦﴾\\
\textamh{37.\  } & أَوَلَم يَرَوا۟ أَنَّ ٱللَّهَ يَبسُطُ ٱلرِّزقَ لِمَن يَشَآءُ وَيَقدِرُ ۚ إِنَّ فِى ذَٟلِكَ لَءَايَـٰتٍۢ لِّقَومٍۢ يُؤمِنُونَ ﴿٣٧﴾\\
\textamh{38.\  } & فَـَٔاتِ ذَا ٱلقُربَىٰ حَقَّهُۥ وَٱلمِسكِينَ وَٱبنَ ٱلسَّبِيلِ ۚ ذَٟلِكَ خَيرٌۭ لِّلَّذِينَ يُرِيدُونَ وَجهَ ٱللَّهِ ۖ وَأُو۟لَـٰٓئِكَ هُمُ ٱلمُفلِحُونَ ﴿٣٨﴾\\
\textamh{39.\  } & وَمَآ ءَاتَيتُم مِّن رِّبًۭا لِّيَربُوَا۟ فِىٓ أَموَٟلِ ٱلنَّاسِ فَلَا يَربُوا۟ عِندَ ٱللَّهِ ۖ وَمَآ ءَاتَيتُم مِّن زَكَوٰةٍۢ تُرِيدُونَ وَجهَ ٱللَّهِ فَأُو۟لَـٰٓئِكَ هُمُ ٱلمُضعِفُونَ ﴿٣٩﴾\\
\textamh{40.\  } & ٱللَّهُ ٱلَّذِى خَلَقَكُم ثُمَّ رَزَقَكُم ثُمَّ يُمِيتُكُم ثُمَّ يُحيِيكُم ۖ هَل مِن شُرَكَآئِكُم مَّن يَفعَلُ مِن ذَٟلِكُم مِّن شَىءٍۢ ۚ سُبحَـٰنَهُۥ وَتَعَـٰلَىٰ عَمَّا يُشرِكُونَ ﴿٤٠﴾\\
\textamh{41.\  } & ظَهَرَ ٱلفَسَادُ فِى ٱلبَرِّ وَٱلبَحرِ بِمَا كَسَبَت أَيدِى ٱلنَّاسِ لِيُذِيقَهُم بَعضَ ٱلَّذِى عَمِلُوا۟ لَعَلَّهُم يَرجِعُونَ ﴿٤١﴾\\
\textamh{42.\  } & قُل سِيرُوا۟ فِى ٱلأَرضِ فَٱنظُرُوا۟ كَيفَ كَانَ عَـٰقِبَةُ ٱلَّذِينَ مِن قَبلُ ۚ كَانَ أَكثَرُهُم مُّشرِكِينَ ﴿٤٢﴾\\
\textamh{43.\  } & فَأَقِم وَجهَكَ لِلدِّينِ ٱلقَيِّمِ مِن قَبلِ أَن يَأتِىَ يَومٌۭ لَّا مَرَدَّ لَهُۥ مِنَ ٱللَّهِ ۖ يَومَئِذٍۢ يَصَّدَّعُونَ ﴿٤٣﴾\\
\textamh{44.\  } & مَن كَفَرَ فَعَلَيهِ كُفرُهُۥ ۖ وَمَن عَمِلَ صَـٰلِحًۭا فَلِأَنفُسِهِم يَمهَدُونَ ﴿٤٤﴾\\
\textamh{45.\  } & لِيَجزِىَ ٱلَّذِينَ ءَامَنُوا۟ وَعَمِلُوا۟ ٱلصَّـٰلِحَـٰتِ مِن فَضلِهِۦٓ ۚ إِنَّهُۥ لَا يُحِبُّ ٱلكَـٰفِرِينَ ﴿٤٥﴾\\
\textamh{46.\  } & وَمِن ءَايَـٰتِهِۦٓ أَن يُرسِلَ ٱلرِّيَاحَ مُبَشِّرَٰتٍۢ وَلِيُذِيقَكُم مِّن رَّحمَتِهِۦ وَلِتَجرِىَ ٱلفُلكُ بِأَمرِهِۦ وَلِتَبتَغُوا۟ مِن فَضلِهِۦ وَلَعَلَّكُم تَشكُرُونَ ﴿٤٦﴾\\
\textamh{47.\  } & وَلَقَد أَرسَلنَا مِن قَبلِكَ رُسُلًا إِلَىٰ قَومِهِم فَجَآءُوهُم بِٱلبَيِّنَـٰتِ فَٱنتَقَمنَا مِنَ ٱلَّذِينَ أَجرَمُوا۟ ۖ وَكَانَ حَقًّا عَلَينَا نَصرُ ٱلمُؤمِنِينَ ﴿٤٧﴾\\
\textamh{48.\  } & ٱللَّهُ ٱلَّذِى يُرسِلُ ٱلرِّيَـٰحَ فَتُثِيرُ سَحَابًۭا فَيَبسُطُهُۥ فِى ٱلسَّمَآءِ كَيفَ يَشَآءُ وَيَجعَلُهُۥ كِسَفًۭا فَتَرَى ٱلوَدقَ يَخرُجُ مِن خِلَـٰلِهِۦ ۖ فَإِذَآ أَصَابَ بِهِۦ مَن يَشَآءُ مِن عِبَادِهِۦٓ إِذَا هُم يَستَبشِرُونَ ﴿٤٨﴾\\
\textamh{49.\  } & وَإِن كَانُوا۟ مِن قَبلِ أَن يُنَزَّلَ عَلَيهِم مِّن قَبلِهِۦ لَمُبلِسِينَ ﴿٤٩﴾\\
\textamh{50.\  } & فَٱنظُر إِلَىٰٓ ءَاثَـٰرِ رَحمَتِ ٱللَّهِ كَيفَ يُحىِ ٱلأَرضَ بَعدَ مَوتِهَآ ۚ إِنَّ ذَٟلِكَ لَمُحىِ ٱلمَوتَىٰ ۖ وَهُوَ عَلَىٰ كُلِّ شَىءٍۢ قَدِيرٌۭ ﴿٥٠﴾\\
\textamh{51.\  } & وَلَئِن أَرسَلنَا رِيحًۭا فَرَأَوهُ مُصفَرًّۭا لَّظَلُّوا۟ مِنۢ بَعدِهِۦ يَكفُرُونَ ﴿٥١﴾\\
\textamh{52.\  } & فَإِنَّكَ لَا تُسمِعُ ٱلمَوتَىٰ وَلَا تُسمِعُ ٱلصُّمَّ ٱلدُّعَآءَ إِذَا وَلَّوا۟ مُدبِرِينَ ﴿٥٢﴾\\
\textamh{53.\  } & وَمَآ أَنتَ بِهَـٰدِ ٱلعُمىِ عَن ضَلَـٰلَتِهِم ۖ إِن تُسمِعُ إِلَّا مَن يُؤمِنُ بِـَٔايَـٰتِنَا فَهُم مُّسلِمُونَ ﴿٥٣﴾\\
\textamh{54.\  } & ۞ ٱللَّهُ ٱلَّذِى خَلَقَكُم مِّن ضَعفٍۢ ثُمَّ جَعَلَ مِنۢ بَعدِ ضَعفٍۢ قُوَّةًۭ ثُمَّ جَعَلَ مِنۢ بَعدِ قُوَّةٍۢ ضَعفًۭا وَشَيبَةًۭ ۚ يَخلُقُ مَا يَشَآءُ ۖ وَهُوَ ٱلعَلِيمُ ٱلقَدِيرُ ﴿٥٤﴾\\
\textamh{55.\  } & وَيَومَ تَقُومُ ٱلسَّاعَةُ يُقسِمُ ٱلمُجرِمُونَ مَا لَبِثُوا۟ غَيرَ سَاعَةٍۢ ۚ كَذَٟلِكَ كَانُوا۟ يُؤفَكُونَ ﴿٥٥﴾\\
\textamh{56.\  } & وَقَالَ ٱلَّذِينَ أُوتُوا۟ ٱلعِلمَ وَٱلإِيمَـٰنَ لَقَد لَبِثتُم فِى كِتَـٰبِ ٱللَّهِ إِلَىٰ يَومِ ٱلبَعثِ ۖ فَهَـٰذَا يَومُ ٱلبَعثِ وَلَـٰكِنَّكُم كُنتُم لَا تَعلَمُونَ ﴿٥٦﴾\\
\textamh{57.\  } & فَيَومَئِذٍۢ لَّا يَنفَعُ ٱلَّذِينَ ظَلَمُوا۟ مَعذِرَتُهُم وَلَا هُم يُستَعتَبُونَ ﴿٥٧﴾\\
\textamh{58.\  } & وَلَقَد ضَرَبنَا لِلنَّاسِ فِى هَـٰذَا ٱلقُرءَانِ مِن كُلِّ مَثَلٍۢ ۚ وَلَئِن جِئتَهُم بِـَٔايَةٍۢ لَّيَقُولَنَّ ٱلَّذِينَ كَفَرُوٓا۟ إِن أَنتُم إِلَّا مُبطِلُونَ ﴿٥٨﴾\\
\textamh{59.\  } & كَذَٟلِكَ يَطبَعُ ٱللَّهُ عَلَىٰ قُلُوبِ ٱلَّذِينَ لَا يَعلَمُونَ ﴿٥٩﴾\\
\textamh{60.\  } & فَٱصبِر إِنَّ وَعدَ ٱللَّهِ حَقٌّۭ ۖ وَلَا يَستَخِفَّنَّكَ ٱلَّذِينَ لَا يُوقِنُونَ ﴿٦٠﴾\\
\end{longtable} \newpage
