%% License: BSD style (Berkley) (i.e. Put the Copyright owner's name always)
%% Writer and Copyright (to): Bewketu(Bilal) Tadilo (2016-17)
\shadowbox{\section{\LR{\textamharic{ሱራቱ ያሲን -}  \RL{سوره  يس}}}}
\begin{longtable}{%
  @{}
    p{.5\textwidth}
  @{~~~~~~~~~~~~~}||
    p{.5\textwidth}
    @{}
}
\nopagebreak
\textamh{\ \ \ \ \ \  ቢስሚላሂ አራህመኒ ራሂይም } &  بِسمِ ٱللَّهِ ٱلرَّحمَـٰنِ ٱلرَّحِيمِ\\
\textamh{1.\  } &  يسٓ ﴿١﴾\\
\textamh{2.\  } & وَٱلقُرءَانِ ٱلحَكِيمِ ﴿٢﴾\\
\textamh{3.\  } & إِنَّكَ لَمِنَ ٱلمُرسَلِينَ ﴿٣﴾\\
\textamh{4.\  } & عَلَىٰ صِرَٰطٍۢ مُّستَقِيمٍۢ ﴿٤﴾\\
\textamh{5.\  } & تَنزِيلَ ٱلعَزِيزِ ٱلرَّحِيمِ ﴿٥﴾\\
\textamh{6.\  } & لِتُنذِرَ قَومًۭا مَّآ أُنذِرَ ءَابَآؤُهُم فَهُم غَٰفِلُونَ ﴿٦﴾\\
\textamh{7.\  } & لَقَد حَقَّ ٱلقَولُ عَلَىٰٓ أَكثَرِهِم فَهُم لَا يُؤمِنُونَ ﴿٧﴾\\
\textamh{8.\  } & إِنَّا جَعَلنَا فِىٓ أَعنَـٰقِهِم أَغلَـٰلًۭا فَهِىَ إِلَى ٱلأَذقَانِ فَهُم مُّقمَحُونَ ﴿٨﴾\\
\textamh{9.\  } & وَجَعَلنَا مِنۢ بَينِ أَيدِيهِم سَدًّۭا وَمِن خَلفِهِم سَدًّۭا فَأَغشَينَـٰهُم فَهُم لَا يُبصِرُونَ ﴿٩﴾\\
\textamh{10.\  } & وَسَوَآءٌ عَلَيهِم ءَأَنذَرتَهُم أَم لَم تُنذِرهُم لَا يُؤمِنُونَ ﴿١٠﴾\\
\textamh{11.\  } & إِنَّمَا تُنذِرُ مَنِ ٱتَّبَعَ ٱلذِّكرَ وَخَشِىَ ٱلرَّحمَـٰنَ بِٱلغَيبِ ۖ فَبَشِّرهُ بِمَغفِرَةٍۢ وَأَجرٍۢ كَرِيمٍ ﴿١١﴾\\
\textamh{12.\  } & إِنَّا نَحنُ نُحىِ ٱلمَوتَىٰ وَنَكتُبُ مَا قَدَّمُوا۟ وَءَاثَـٰرَهُم ۚ وَكُلَّ شَىءٍ أَحصَينَـٰهُ فِىٓ إِمَامٍۢ مُّبِينٍۢ ﴿١٢﴾\\
\textamh{13.\  } & وَٱضرِب لَهُم مَّثَلًا أَصحَـٰبَ ٱلقَريَةِ إِذ جَآءَهَا ٱلمُرسَلُونَ ﴿١٣﴾\\
\textamh{14.\  } & إِذ أَرسَلنَآ إِلَيهِمُ ٱثنَينِ فَكَذَّبُوهُمَا فَعَزَّزنَا بِثَالِثٍۢ فَقَالُوٓا۟ إِنَّآ إِلَيكُم مُّرسَلُونَ ﴿١٤﴾\\
\textamh{15.\  } & قَالُوا۟ مَآ أَنتُم إِلَّا بَشَرٌۭ مِّثلُنَا وَمَآ أَنزَلَ ٱلرَّحمَـٰنُ مِن شَىءٍ إِن أَنتُم إِلَّا تَكذِبُونَ ﴿١٥﴾\\
\textamh{16.\  } & قَالُوا۟ رَبُّنَا يَعلَمُ إِنَّآ إِلَيكُم لَمُرسَلُونَ ﴿١٦﴾\\
\textamh{17.\  } & وَمَا عَلَينَآ إِلَّا ٱلبَلَـٰغُ ٱلمُبِينُ ﴿١٧﴾\\
\textamh{18.\  } & قَالُوٓا۟ إِنَّا تَطَيَّرنَا بِكُم ۖ لَئِن لَّم تَنتَهُوا۟ لَنَرجُمَنَّكُم وَلَيَمَسَّنَّكُم مِّنَّا عَذَابٌ أَلِيمٌۭ ﴿١٨﴾\\
\textamh{19.\  } & قَالُوا۟ طَٰٓئِرُكُم مَّعَكُم ۚ أَئِن ذُكِّرتُم ۚ بَل أَنتُم قَومٌۭ مُّسرِفُونَ ﴿١٩﴾\\
\textamh{20.\  } & وَجَآءَ مِن أَقصَا ٱلمَدِينَةِ رَجُلٌۭ يَسعَىٰ قَالَ يَـٰقَومِ ٱتَّبِعُوا۟ ٱلمُرسَلِينَ ﴿٢٠﴾\\
\textamh{21.\  } & ٱتَّبِعُوا۟ مَن لَّا يَسـَٔلُكُم أَجرًۭا وَهُم مُّهتَدُونَ ﴿٢١﴾\\
\textamh{22.\  } & وَمَا لِىَ لَآ أَعبُدُ ٱلَّذِى فَطَرَنِى وَإِلَيهِ تُرجَعُونَ ﴿٢٢﴾\\
\textamh{23.\  } & ءَأَتَّخِذُ مِن دُونِهِۦٓ ءَالِهَةً إِن يُرِدنِ ٱلرَّحمَـٰنُ بِضُرٍّۢ لَّا تُغنِ عَنِّى شَفَـٰعَتُهُم شَيـًۭٔا وَلَا يُنقِذُونِ ﴿٢٣﴾\\
\textamh{24.\  } & إِنِّىٓ إِذًۭا لَّفِى ضَلَـٰلٍۢ مُّبِينٍ ﴿٢٤﴾\\
\textamh{25.\  } & إِنِّىٓ ءَامَنتُ بِرَبِّكُم فَٱسمَعُونِ ﴿٢٥﴾\\
\textamh{26.\  } & قِيلَ ٱدخُلِ ٱلجَنَّةَ ۖ قَالَ يَـٰلَيتَ قَومِى يَعلَمُونَ ﴿٢٦﴾\\
\textamh{27.\  } & بِمَا غَفَرَ لِى رَبِّى وَجَعَلَنِى مِنَ ٱلمُكرَمِينَ ﴿٢٧﴾\\
\textamh{28.\  } & ۞ وَمَآ أَنزَلنَا عَلَىٰ قَومِهِۦ مِنۢ بَعدِهِۦ مِن جُندٍۢ مِّنَ ٱلسَّمَآءِ وَمَا كُنَّا مُنزِلِينَ ﴿٢٨﴾\\
\textamh{29.\  } & إِن كَانَت إِلَّا صَيحَةًۭ وَٟحِدَةًۭ فَإِذَا هُم خَـٰمِدُونَ ﴿٢٩﴾\\
\textamh{30.\  } & يَـٰحَسرَةً عَلَى ٱلعِبَادِ ۚ مَا يَأتِيهِم مِّن رَّسُولٍ إِلَّا كَانُوا۟ بِهِۦ يَستَهزِءُونَ ﴿٣٠﴾\\
\textamh{31.\  } & أَلَم يَرَوا۟ كَم أَهلَكنَا قَبلَهُم مِّنَ ٱلقُرُونِ أَنَّهُم إِلَيهِم لَا يَرجِعُونَ ﴿٣١﴾\\
\textamh{32.\  } & وَإِن كُلٌّۭ لَّمَّا جَمِيعٌۭ لَّدَينَا مُحضَرُونَ ﴿٣٢﴾\\
\textamh{33.\  } & وَءَايَةٌۭ لَّهُمُ ٱلأَرضُ ٱلمَيتَةُ أَحيَينَـٰهَا وَأَخرَجنَا مِنهَا حَبًّۭا فَمِنهُ يَأكُلُونَ ﴿٣٣﴾\\
\textamh{34.\  } & وَجَعَلنَا فِيهَا جَنَّـٰتٍۢ مِّن نَّخِيلٍۢ وَأَعنَـٰبٍۢ وَفَجَّرنَا فِيهَا مِنَ ٱلعُيُونِ ﴿٣٤﴾\\
\textamh{35.\  } & لِيَأكُلُوا۟ مِن ثَمَرِهِۦ وَمَا عَمِلَتهُ أَيدِيهِم ۖ أَفَلَا يَشكُرُونَ ﴿٣٥﴾\\
\textamh{36.\  } & سُبحَـٰنَ ٱلَّذِى خَلَقَ ٱلأَزوَٟجَ كُلَّهَا مِمَّا تُنۢبِتُ ٱلأَرضُ وَمِن أَنفُسِهِم وَمِمَّا لَا يَعلَمُونَ ﴿٣٦﴾\\
\textamh{37.\  } & وَءَايَةٌۭ لَّهُمُ ٱلَّيلُ نَسلَخُ مِنهُ ٱلنَّهَارَ فَإِذَا هُم مُّظلِمُونَ ﴿٣٧﴾\\
\textamh{38.\  } & وَٱلشَّمسُ تَجرِى لِمُستَقَرٍّۢ لَّهَا ۚ ذَٟلِكَ تَقدِيرُ ٱلعَزِيزِ ٱلعَلِيمِ ﴿٣٨﴾\\
\textamh{39.\  } & وَٱلقَمَرَ قَدَّرنَـٰهُ مَنَازِلَ حَتَّىٰ عَادَ كَٱلعُرجُونِ ٱلقَدِيمِ ﴿٣٩﴾\\
\textamh{40.\  } & لَا ٱلشَّمسُ يَنۢبَغِى لَهَآ أَن تُدرِكَ ٱلقَمَرَ وَلَا ٱلَّيلُ سَابِقُ ٱلنَّهَارِ ۚ وَكُلٌّۭ فِى فَلَكٍۢ يَسبَحُونَ ﴿٤٠﴾\\
\textamh{41.\  } & وَءَايَةٌۭ لَّهُم أَنَّا حَمَلنَا ذُرِّيَّتَهُم فِى ٱلفُلكِ ٱلمَشحُونِ ﴿٤١﴾\\
\textamh{42.\  } & وَخَلَقنَا لَهُم مِّن مِّثلِهِۦ مَا يَركَبُونَ ﴿٤٢﴾\\
\textamh{43.\  } & وَإِن نَّشَأ نُغرِقهُم فَلَا صَرِيخَ لَهُم وَلَا هُم يُنقَذُونَ ﴿٤٣﴾\\
\textamh{44.\  } & إِلَّا رَحمَةًۭ مِّنَّا وَمَتَـٰعًا إِلَىٰ حِينٍۢ ﴿٤٤﴾\\
\textamh{45.\  } & وَإِذَا قِيلَ لَهُمُ ٱتَّقُوا۟ مَا بَينَ أَيدِيكُم وَمَا خَلفَكُم لَعَلَّكُم تُرحَمُونَ ﴿٤٥﴾\\
\textamh{46.\  } & وَمَا تَأتِيهِم مِّن ءَايَةٍۢ مِّن ءَايَـٰتِ رَبِّهِم إِلَّا كَانُوا۟ عَنهَا مُعرِضِينَ ﴿٤٦﴾\\
\textamh{47.\  } & وَإِذَا قِيلَ لَهُم أَنفِقُوا۟ مِمَّا رَزَقَكُمُ ٱللَّهُ قَالَ ٱلَّذِينَ كَفَرُوا۟ لِلَّذِينَ ءَامَنُوٓا۟ أَنُطعِمُ مَن لَّو يَشَآءُ ٱللَّهُ أَطعَمَهُۥٓ إِن أَنتُم إِلَّا فِى ضَلَـٰلٍۢ مُّبِينٍۢ ﴿٤٧﴾\\
\textamh{48.\  } & وَيَقُولُونَ مَتَىٰ هَـٰذَا ٱلوَعدُ إِن كُنتُم صَـٰدِقِينَ ﴿٤٨﴾\\
\textamh{49.\  } & مَا يَنظُرُونَ إِلَّا صَيحَةًۭ وَٟحِدَةًۭ تَأخُذُهُم وَهُم يَخِصِّمُونَ ﴿٤٩﴾\\
\textamh{50.\  } & فَلَا يَستَطِيعُونَ تَوصِيَةًۭ وَلَآ إِلَىٰٓ أَهلِهِم يَرجِعُونَ ﴿٥٠﴾\\
\textamh{51.\  } & وَنُفِخَ فِى ٱلصُّورِ فَإِذَا هُم مِّنَ ٱلأَجدَاثِ إِلَىٰ رَبِّهِم يَنسِلُونَ ﴿٥١﴾\\
\textamh{52.\  } & قَالُوا۟ يَـٰوَيلَنَا مَنۢ بَعَثَنَا مِن مَّرقَدِنَا ۜ ۗ هَـٰذَا مَا وَعَدَ ٱلرَّحمَـٰنُ وَصَدَقَ ٱلمُرسَلُونَ ﴿٥٢﴾\\
\textamh{53.\  } & إِن كَانَت إِلَّا صَيحَةًۭ وَٟحِدَةًۭ فَإِذَا هُم جَمِيعٌۭ لَّدَينَا مُحضَرُونَ ﴿٥٣﴾\\
\textamh{54.\  } & فَٱليَومَ لَا تُظلَمُ نَفسٌۭ شَيـًۭٔا وَلَا تُجزَونَ إِلَّا مَا كُنتُم تَعمَلُونَ ﴿٥٤﴾\\
\textamh{55.\  } & إِنَّ أَصحَـٰبَ ٱلجَنَّةِ ٱليَومَ فِى شُغُلٍۢ فَـٰكِهُونَ ﴿٥٥﴾\\
\textamh{56.\  } & هُم وَأَزوَٟجُهُم فِى ظِلَـٰلٍ عَلَى ٱلأَرَآئِكِ مُتَّكِـُٔونَ ﴿٥٦﴾\\
\textamh{57.\  } & لَهُم فِيهَا فَـٰكِهَةٌۭ وَلَهُم مَّا يَدَّعُونَ ﴿٥٧﴾\\
\textamh{58.\  } & سَلَـٰمٌۭ قَولًۭا مِّن رَّبٍّۢ رَّحِيمٍۢ ﴿٥٨﴾\\
\textamh{59.\  } & وَٱمتَـٰزُوا۟ ٱليَومَ أَيُّهَا ٱلمُجرِمُونَ ﴿٥٩﴾\\
\textamh{60.\  } & ۞ أَلَم أَعهَد إِلَيكُم يَـٰبَنِىٓ ءَادَمَ أَن لَّا تَعبُدُوا۟ ٱلشَّيطَٰنَ ۖ إِنَّهُۥ لَكُم عَدُوٌّۭ مُّبِينٌۭ ﴿٦٠﴾\\
\textamh{61.\  } & وَأَنِ ٱعبُدُونِى ۚ هَـٰذَا صِرَٰطٌۭ مُّستَقِيمٌۭ ﴿٦١﴾\\
\textamh{62.\  } & وَلَقَد أَضَلَّ مِنكُم جِبِلًّۭا كَثِيرًا ۖ أَفَلَم تَكُونُوا۟ تَعقِلُونَ ﴿٦٢﴾\\
\textamh{63.\  } & هَـٰذِهِۦ جَهَنَّمُ ٱلَّتِى كُنتُم تُوعَدُونَ ﴿٦٣﴾\\
\textamh{64.\  } & ٱصلَوهَا ٱليَومَ بِمَا كُنتُم تَكفُرُونَ ﴿٦٤﴾\\
\textamh{65.\  } & ٱليَومَ نَختِمُ عَلَىٰٓ أَفوَٟهِهِم وَتُكَلِّمُنَآ أَيدِيهِم وَتَشهَدُ أَرجُلُهُم بِمَا كَانُوا۟ يَكسِبُونَ ﴿٦٥﴾\\
\textamh{66.\  } & وَلَو نَشَآءُ لَطَمَسنَا عَلَىٰٓ أَعيُنِهِم فَٱستَبَقُوا۟ ٱلصِّرَٰطَ فَأَنَّىٰ يُبصِرُونَ ﴿٦٦﴾\\
\textamh{67.\  } & وَلَو نَشَآءُ لَمَسَخنَـٰهُم عَلَىٰ مَكَانَتِهِم فَمَا ٱستَطَٰعُوا۟ مُضِيًّۭا وَلَا يَرجِعُونَ ﴿٦٧﴾\\
\textamh{68.\  } & وَمَن نُّعَمِّرهُ نُنَكِّسهُ فِى ٱلخَلقِ ۖ أَفَلَا يَعقِلُونَ ﴿٦٨﴾\\
\textamh{69.\  } & وَمَا عَلَّمنَـٰهُ ٱلشِّعرَ وَمَا يَنۢبَغِى لَهُۥٓ ۚ إِن هُوَ إِلَّا ذِكرٌۭ وَقُرءَانٌۭ مُّبِينٌۭ ﴿٦٩﴾\\
\textamh{70.\  } & لِّيُنذِرَ مَن كَانَ حَيًّۭا وَيَحِقَّ ٱلقَولُ عَلَى ٱلكَـٰفِرِينَ ﴿٧٠﴾\\
\textamh{71.\  } & أَوَلَم يَرَوا۟ أَنَّا خَلَقنَا لَهُم مِّمَّا عَمِلَت أَيدِينَآ أَنعَـٰمًۭا فَهُم لَهَا مَـٰلِكُونَ ﴿٧١﴾\\
\textamh{72.\  } & وَذَلَّلنَـٰهَا لَهُم فَمِنهَا رَكُوبُهُم وَمِنهَا يَأكُلُونَ ﴿٧٢﴾\\
\textamh{73.\  } & وَلَهُم فِيهَا مَنَـٰفِعُ وَمَشَارِبُ ۖ أَفَلَا يَشكُرُونَ ﴿٧٣﴾\\
\textamh{74.\  } & وَٱتَّخَذُوا۟ مِن دُونِ ٱللَّهِ ءَالِهَةًۭ لَّعَلَّهُم يُنصَرُونَ ﴿٧٤﴾\\
\textamh{75.\  } & لَا يَستَطِيعُونَ نَصرَهُم وَهُم لَهُم جُندٌۭ مُّحضَرُونَ ﴿٧٥﴾\\
\textamh{76.\  } & فَلَا يَحزُنكَ قَولُهُم ۘ إِنَّا نَعلَمُ مَا يُسِرُّونَ وَمَا يُعلِنُونَ ﴿٧٦﴾\\
\textamh{77.\  } & أَوَلَم يَرَ ٱلإِنسَـٰنُ أَنَّا خَلَقنَـٰهُ مِن نُّطفَةٍۢ فَإِذَا هُوَ خَصِيمٌۭ مُّبِينٌۭ ﴿٧٧﴾\\
\textamh{78.\  } & وَضَرَبَ لَنَا مَثَلًۭا وَنَسِىَ خَلقَهُۥ ۖ قَالَ مَن يُحىِ ٱلعِظَـٰمَ وَهِىَ رَمِيمٌۭ ﴿٧٨﴾\\
\textamh{79.\  } & قُل يُحيِيهَا ٱلَّذِىٓ أَنشَأَهَآ أَوَّلَ مَرَّةٍۢ ۖ وَهُوَ بِكُلِّ خَلقٍ عَلِيمٌ ﴿٧٩﴾\\
\textamh{80.\  } & ٱلَّذِى جَعَلَ لَكُم مِّنَ ٱلشَّجَرِ ٱلأَخضَرِ نَارًۭا فَإِذَآ أَنتُم مِّنهُ تُوقِدُونَ ﴿٨٠﴾\\
\textamh{81.\  } & أَوَلَيسَ ٱلَّذِى خَلَقَ ٱلسَّمَـٰوَٟتِ وَٱلأَرضَ بِقَـٰدِرٍ عَلَىٰٓ أَن يَخلُقَ مِثلَهُم ۚ بَلَىٰ وَهُوَ ٱلخَلَّٰقُ ٱلعَلِيمُ ﴿٨١﴾\\
\textamh{82.\  } & إِنَّمَآ أَمرُهُۥٓ إِذَآ أَرَادَ شَيـًٔا أَن يَقُولَ لَهُۥ كُن فَيَكُونُ ﴿٨٢﴾\\
\textamh{83.\  } & فَسُبحَـٰنَ ٱلَّذِى بِيَدِهِۦ مَلَكُوتُ كُلِّ شَىءٍۢ وَإِلَيهِ تُرجَعُونَ ﴿٨٣﴾\\
\end{longtable} \newpage
