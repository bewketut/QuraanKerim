%% License: BSD style (Berkley) (i.e. Put the Copyright owner's name always)
%% Writer and Copyright (to): Bewketu(Bilal) Tadilo (2016-17)
\shadowbox{\section{\LR{\textamharic{ሱራቱ ፉሲላት -}  \RL{سوره  فصلت}}}}
\begin{longtable}{%
  @{}
    p{.5\textwidth}
  @{~~~~~~~~~~~~~}||
    p{.5\textwidth}
    @{}
}
\nopagebreak
\textamh{\ \ \ \ \ \  ቢስሚላሂ አራህመኒ ራሂይም } &  بِسمِ ٱللَّهِ ٱلرَّحمَـٰنِ ٱلرَّحِيمِ\\
\textamh{1.\  } &  حمٓ ﴿١﴾\\
\textamh{2.\  } & تَنزِيلٌۭ مِّنَ ٱلرَّحمَـٰنِ ٱلرَّحِيمِ ﴿٢﴾\\
\textamh{3.\  } & كِتَـٰبٌۭ فُصِّلَت ءَايَـٰتُهُۥ قُرءَانًا عَرَبِيًّۭا لِّقَومٍۢ يَعلَمُونَ ﴿٣﴾\\
\textamh{4.\  } & بَشِيرًۭا وَنَذِيرًۭا فَأَعرَضَ أَكثَرُهُم فَهُم لَا يَسمَعُونَ ﴿٤﴾\\
\textamh{5.\  } & وَقَالُوا۟ قُلُوبُنَا فِىٓ أَكِنَّةٍۢ مِّمَّا تَدعُونَآ إِلَيهِ وَفِىٓ ءَاذَانِنَا وَقرٌۭ وَمِنۢ بَينِنَا وَبَينِكَ حِجَابٌۭ فَٱعمَل إِنَّنَا عَـٰمِلُونَ ﴿٥﴾\\
\textamh{6.\  } & قُل إِنَّمَآ أَنَا۠ بَشَرٌۭ مِّثلُكُم يُوحَىٰٓ إِلَىَّ أَنَّمَآ إِلَـٰهُكُم إِلَـٰهٌۭ وَٟحِدٌۭ فَٱستَقِيمُوٓا۟ إِلَيهِ وَٱستَغفِرُوهُ ۗ وَوَيلٌۭ لِّلمُشرِكِينَ ﴿٦﴾\\
\textamh{7.\  } & ٱلَّذِينَ لَا يُؤتُونَ ٱلزَّكَوٰةَ وَهُم بِٱلءَاخِرَةِ هُم كَـٰفِرُونَ ﴿٧﴾\\
\textamh{8.\  } & إِنَّ ٱلَّذِينَ ءَامَنُوا۟ وَعَمِلُوا۟ ٱلصَّـٰلِحَـٰتِ لَهُم أَجرٌ غَيرُ مَمنُونٍۢ ﴿٨﴾\\
\textamh{9.\  } & ۞ قُل أَئِنَّكُم لَتَكفُرُونَ بِٱلَّذِى خَلَقَ ٱلأَرضَ فِى يَومَينِ وَتَجعَلُونَ لَهُۥٓ أَندَادًۭا ۚ ذَٟلِكَ رَبُّ ٱلعَـٰلَمِينَ ﴿٩﴾\\
\textamh{10.\  } & وَجَعَلَ فِيهَا رَوَٟسِىَ مِن فَوقِهَا وَبَٰرَكَ فِيهَا وَقَدَّرَ فِيهَآ أَقوَٟتَهَا فِىٓ أَربَعَةِ أَيَّامٍۢ سَوَآءًۭ لِّلسَّآئِلِينَ ﴿١٠﴾\\
\textamh{11.\  } & ثُمَّ ٱستَوَىٰٓ إِلَى ٱلسَّمَآءِ وَهِىَ دُخَانٌۭ فَقَالَ لَهَا وَلِلأَرضِ ٱئتِيَا طَوعًا أَو كَرهًۭا قَالَتَآ أَتَينَا طَآئِعِينَ ﴿١١﴾\\
\textamh{12.\  } & فَقَضَىٰهُنَّ سَبعَ سَمَـٰوَاتٍۢ فِى يَومَينِ وَأَوحَىٰ فِى كُلِّ سَمَآءٍ أَمرَهَا ۚ وَزَيَّنَّا ٱلسَّمَآءَ ٱلدُّنيَا بِمَصَـٰبِيحَ وَحِفظًۭا ۚ ذَٟلِكَ تَقدِيرُ ٱلعَزِيزِ ٱلعَلِيمِ ﴿١٢﴾\\
\textamh{13.\  } & فَإِن أَعرَضُوا۟ فَقُل أَنذَرتُكُم صَـٰعِقَةًۭ مِّثلَ صَـٰعِقَةِ عَادٍۢ وَثَمُودَ ﴿١٣﴾\\
\textamh{14.\  } & إِذ جَآءَتهُمُ ٱلرُّسُلُ مِنۢ بَينِ أَيدِيهِم وَمِن خَلفِهِم أَلَّا تَعبُدُوٓا۟ إِلَّا ٱللَّهَ ۖ قَالُوا۟ لَو شَآءَ رَبُّنَا لَأَنزَلَ مَلَـٰٓئِكَةًۭ فَإِنَّا بِمَآ أُرسِلتُم بِهِۦ كَـٰفِرُونَ ﴿١٤﴾\\
\textamh{15.\  } & فَأَمَّا عَادٌۭ فَٱستَكبَرُوا۟ فِى ٱلأَرضِ بِغَيرِ ٱلحَقِّ وَقَالُوا۟ مَن أَشَدُّ مِنَّا قُوَّةً ۖ أَوَلَم يَرَوا۟ أَنَّ ٱللَّهَ ٱلَّذِى خَلَقَهُم هُوَ أَشَدُّ مِنهُم قُوَّةًۭ ۖ وَكَانُوا۟ بِـَٔايَـٰتِنَا يَجحَدُونَ ﴿١٥﴾\\
\textamh{16.\  } & فَأَرسَلنَا عَلَيهِم رِيحًۭا صَرصَرًۭا فِىٓ أَيَّامٍۢ نَّحِسَاتٍۢ لِّنُذِيقَهُم عَذَابَ ٱلخِزىِ فِى ٱلحَيَوٰةِ ٱلدُّنيَا ۖ وَلَعَذَابُ ٱلءَاخِرَةِ أَخزَىٰ ۖ وَهُم لَا يُنصَرُونَ ﴿١٦﴾\\
\textamh{17.\  } & وَأَمَّا ثَمُودُ فَهَدَينَـٰهُم فَٱستَحَبُّوا۟ ٱلعَمَىٰ عَلَى ٱلهُدَىٰ فَأَخَذَتهُم صَـٰعِقَةُ ٱلعَذَابِ ٱلهُونِ بِمَا كَانُوا۟ يَكسِبُونَ ﴿١٧﴾\\
\textamh{18.\  } & وَنَجَّينَا ٱلَّذِينَ ءَامَنُوا۟ وَكَانُوا۟ يَتَّقُونَ ﴿١٨﴾\\
\textamh{19.\  } & وَيَومَ يُحشَرُ أَعدَآءُ ٱللَّهِ إِلَى ٱلنَّارِ فَهُم يُوزَعُونَ ﴿١٩﴾\\
\textamh{20.\  } & حَتَّىٰٓ إِذَا مَا جَآءُوهَا شَهِدَ عَلَيهِم سَمعُهُم وَأَبصَـٰرُهُم وَجُلُودُهُم بِمَا كَانُوا۟ يَعمَلُونَ ﴿٢٠﴾\\
\textamh{21.\  } & وَقَالُوا۟ لِجُلُودِهِم لِمَ شَهِدتُّم عَلَينَا ۖ قَالُوٓا۟ أَنطَقَنَا ٱللَّهُ ٱلَّذِىٓ أَنطَقَ كُلَّ شَىءٍۢ وَهُوَ خَلَقَكُم أَوَّلَ مَرَّةٍۢ وَإِلَيهِ تُرجَعُونَ ﴿٢١﴾\\
\textamh{22.\  } & وَمَا كُنتُم تَستَتِرُونَ أَن يَشهَدَ عَلَيكُم سَمعُكُم وَلَآ أَبصَـٰرُكُم وَلَا جُلُودُكُم وَلَـٰكِن ظَنَنتُم أَنَّ ٱللَّهَ لَا يَعلَمُ كَثِيرًۭا مِّمَّا تَعمَلُونَ ﴿٢٢﴾\\
\textamh{23.\  } & وَذَٟلِكُم ظَنُّكُمُ ٱلَّذِى ظَنَنتُم بِرَبِّكُم أَردَىٰكُم فَأَصبَحتُم مِّنَ ٱلخَـٰسِرِينَ ﴿٢٣﴾\\
\textamh{24.\  } & فَإِن يَصبِرُوا۟ فَٱلنَّارُ مَثوًۭى لَّهُم ۖ وَإِن يَستَعتِبُوا۟ فَمَا هُم مِّنَ ٱلمُعتَبِينَ ﴿٢٤﴾\\
\textamh{25.\  } & ۞ وَقَيَّضنَا لَهُم قُرَنَآءَ فَزَيَّنُوا۟ لَهُم مَّا بَينَ أَيدِيهِم وَمَا خَلفَهُم وَحَقَّ عَلَيهِمُ ٱلقَولُ فِىٓ أُمَمٍۢ قَد خَلَت مِن قَبلِهِم مِّنَ ٱلجِنِّ وَٱلإِنسِ ۖ إِنَّهُم كَانُوا۟ خَـٰسِرِينَ ﴿٢٥﴾\\
\textamh{26.\  } & وَقَالَ ٱلَّذِينَ كَفَرُوا۟ لَا تَسمَعُوا۟ لِهَـٰذَا ٱلقُرءَانِ وَٱلغَوا۟ فِيهِ لَعَلَّكُم تَغلِبُونَ ﴿٢٦﴾\\
\textamh{27.\  } & فَلَنُذِيقَنَّ ٱلَّذِينَ كَفَرُوا۟ عَذَابًۭا شَدِيدًۭا وَلَنَجزِيَنَّهُم أَسوَأَ ٱلَّذِى كَانُوا۟ يَعمَلُونَ ﴿٢٧﴾\\
\textamh{28.\  } & ذَٟلِكَ جَزَآءُ أَعدَآءِ ٱللَّهِ ٱلنَّارُ ۖ لَهُم فِيهَا دَارُ ٱلخُلدِ ۖ جَزَآءًۢ بِمَا كَانُوا۟ بِـَٔايَـٰتِنَا يَجحَدُونَ ﴿٢٨﴾\\
\textamh{29.\  } & وَقَالَ ٱلَّذِينَ كَفَرُوا۟ رَبَّنَآ أَرِنَا ٱلَّذَينِ أَضَلَّانَا مِنَ ٱلجِنِّ وَٱلإِنسِ نَجعَلهُمَا تَحتَ أَقدَامِنَا لِيَكُونَا مِنَ ٱلأَسفَلِينَ ﴿٢٩﴾\\
\textamh{30.\  } & إِنَّ ٱلَّذِينَ قَالُوا۟ رَبُّنَا ٱللَّهُ ثُمَّ ٱستَقَـٰمُوا۟ تَتَنَزَّلُ عَلَيهِمُ ٱلمَلَـٰٓئِكَةُ أَلَّا تَخَافُوا۟ وَلَا تَحزَنُوا۟ وَأَبشِرُوا۟ بِٱلجَنَّةِ ٱلَّتِى كُنتُم تُوعَدُونَ ﴿٣٠﴾\\
\textamh{31.\  } & نَحنُ أَولِيَآؤُكُم فِى ٱلحَيَوٰةِ ٱلدُّنيَا وَفِى ٱلءَاخِرَةِ ۖ وَلَكُم فِيهَا مَا تَشتَهِىٓ أَنفُسُكُم وَلَكُم فِيهَا مَا تَدَّعُونَ ﴿٣١﴾\\
\textamh{32.\  } & نُزُلًۭا مِّن غَفُورٍۢ رَّحِيمٍۢ ﴿٣٢﴾\\
\textamh{33.\  } & وَمَن أَحسَنُ قَولًۭا مِّمَّن دَعَآ إِلَى ٱللَّهِ وَعَمِلَ صَـٰلِحًۭا وَقَالَ إِنَّنِى مِنَ ٱلمُسلِمِينَ ﴿٣٣﴾\\
\textamh{34.\  } & وَلَا تَستَوِى ٱلحَسَنَةُ وَلَا ٱلسَّيِّئَةُ ۚ ٱدفَع بِٱلَّتِى هِىَ أَحسَنُ فَإِذَا ٱلَّذِى بَينَكَ وَبَينَهُۥ عَدَٟوَةٌۭ كَأَنَّهُۥ وَلِىٌّ حَمِيمٌۭ ﴿٣٤﴾\\
\textamh{35.\  } & وَمَا يُلَقَّىٰهَآ إِلَّا ٱلَّذِينَ صَبَرُوا۟ وَمَا يُلَقَّىٰهَآ إِلَّا ذُو حَظٍّ عَظِيمٍۢ ﴿٣٥﴾\\
\textamh{36.\  } & وَإِمَّا يَنزَغَنَّكَ مِنَ ٱلشَّيطَٰنِ نَزغٌۭ فَٱستَعِذ بِٱللَّهِ ۖ إِنَّهُۥ هُوَ ٱلسَّمِيعُ ٱلعَلِيمُ ﴿٣٦﴾\\
\textamh{37.\  } & وَمِن ءَايَـٰتِهِ ٱلَّيلُ وَٱلنَّهَارُ وَٱلشَّمسُ وَٱلقَمَرُ ۚ لَا تَسجُدُوا۟ لِلشَّمسِ وَلَا لِلقَمَرِ وَٱسجُدُوا۟ لِلَّهِ ٱلَّذِى خَلَقَهُنَّ إِن كُنتُم إِيَّاهُ تَعبُدُونَ ﴿٣٧﴾\\
\textamh{38.\  } & فَإِنِ ٱستَكبَرُوا۟ فَٱلَّذِينَ عِندَ رَبِّكَ يُسَبِّحُونَ لَهُۥ بِٱلَّيلِ وَٱلنَّهَارِ وَهُم لَا يَسـَٔمُونَ ۩ ﴿٣٨﴾\\
\textamh{39.\  } & وَمِن ءَايَـٰتِهِۦٓ أَنَّكَ تَرَى ٱلأَرضَ خَـٰشِعَةًۭ فَإِذَآ أَنزَلنَا عَلَيهَا ٱلمَآءَ ٱهتَزَّت وَرَبَت ۚ إِنَّ ٱلَّذِىٓ أَحيَاهَا لَمُحىِ ٱلمَوتَىٰٓ ۚ إِنَّهُۥ عَلَىٰ كُلِّ شَىءٍۢ قَدِيرٌ ﴿٣٩﴾\\
\textamh{40.\  } & إِنَّ ٱلَّذِينَ يُلحِدُونَ فِىٓ ءَايَـٰتِنَا لَا يَخفَونَ عَلَينَآ ۗ أَفَمَن يُلقَىٰ فِى ٱلنَّارِ خَيرٌ أَم مَّن يَأتِىٓ ءَامِنًۭا يَومَ ٱلقِيَـٰمَةِ ۚ ٱعمَلُوا۟ مَا شِئتُم ۖ إِنَّهُۥ بِمَا تَعمَلُونَ بَصِيرٌ ﴿٤٠﴾\\
\textamh{41.\  } & إِنَّ ٱلَّذِينَ كَفَرُوا۟ بِٱلذِّكرِ لَمَّا جَآءَهُم ۖ وَإِنَّهُۥ لَكِتَـٰبٌ عَزِيزٌۭ ﴿٤١﴾\\
\textamh{42.\  } & لَّا يَأتِيهِ ٱلبَٰطِلُ مِنۢ بَينِ يَدَيهِ وَلَا مِن خَلفِهِۦ ۖ تَنزِيلٌۭ مِّن حَكِيمٍ حَمِيدٍۢ ﴿٤٢﴾\\
\textamh{43.\  } & مَّا يُقَالُ لَكَ إِلَّا مَا قَد قِيلَ لِلرُّسُلِ مِن قَبلِكَ ۚ إِنَّ رَبَّكَ لَذُو مَغفِرَةٍۢ وَذُو عِقَابٍ أَلِيمٍۢ ﴿٤٣﴾\\
\textamh{44.\  } & وَلَو جَعَلنَـٰهُ قُرءَانًا أَعجَمِيًّۭا لَّقَالُوا۟ لَولَا فُصِّلَت ءَايَـٰتُهُۥٓ ۖ ءَا۬عجَمِىٌّۭ وَعَرَبِىٌّۭ ۗ قُل هُوَ لِلَّذِينَ ءَامَنُوا۟ هُدًۭى وَشِفَآءٌۭ ۖ وَٱلَّذِينَ لَا يُؤمِنُونَ فِىٓ ءَاذَانِهِم وَقرٌۭ وَهُوَ عَلَيهِم عَمًى ۚ أُو۟لَـٰٓئِكَ يُنَادَونَ مِن مَّكَانٍۭ بَعِيدٍۢ ﴿٤٤﴾\\
\textamh{45.\  } & وَلَقَد ءَاتَينَا مُوسَى ٱلكِتَـٰبَ فَٱختُلِفَ فِيهِ ۗ وَلَولَا كَلِمَةٌۭ سَبَقَت مِن رَّبِّكَ لَقُضِىَ بَينَهُم ۚ وَإِنَّهُم لَفِى شَكٍّۢ مِّنهُ مُرِيبٍۢ ﴿٤٥﴾\\
\textamh{46.\  } & مَّن عَمِلَ صَـٰلِحًۭا فَلِنَفسِهِۦ ۖ وَمَن أَسَآءَ فَعَلَيهَا ۗ وَمَا رَبُّكَ بِظَلَّٰمٍۢ لِّلعَبِيدِ ﴿٤٦﴾\\
\textamh{47.\  } & ۞ إِلَيهِ يُرَدُّ عِلمُ ٱلسَّاعَةِ ۚ وَمَا تَخرُجُ مِن ثَمَرَٰتٍۢ مِّن أَكمَامِهَا وَمَا تَحمِلُ مِن أُنثَىٰ وَلَا تَضَعُ إِلَّا بِعِلمِهِۦ ۚ وَيَومَ يُنَادِيهِم أَينَ شُرَكَآءِى قَالُوٓا۟ ءَاذَنَّـٰكَ مَا مِنَّا مِن شَهِيدٍۢ ﴿٤٧﴾\\
\textamh{48.\  } & وَضَلَّ عَنهُم مَّا كَانُوا۟ يَدعُونَ مِن قَبلُ ۖ وَظَنُّوا۟ مَا لَهُم مِّن مَّحِيصٍۢ ﴿٤٨﴾\\
\textamh{49.\  } & لَّا يَسـَٔمُ ٱلإِنسَـٰنُ مِن دُعَآءِ ٱلخَيرِ وَإِن مَّسَّهُ ٱلشَّرُّ فَيَـُٔوسٌۭ قَنُوطٌۭ ﴿٤٩﴾\\
\textamh{50.\  } & وَلَئِن أَذَقنَـٰهُ رَحمَةًۭ مِّنَّا مِنۢ بَعدِ ضَرَّآءَ مَسَّتهُ لَيَقُولَنَّ هَـٰذَا لِى وَمَآ أَظُنُّ ٱلسَّاعَةَ قَآئِمَةًۭ وَلَئِن رُّجِعتُ إِلَىٰ رَبِّىٓ إِنَّ لِى عِندَهُۥ لَلحُسنَىٰ ۚ فَلَنُنَبِّئَنَّ ٱلَّذِينَ كَفَرُوا۟ بِمَا عَمِلُوا۟ وَلَنُذِيقَنَّهُم مِّن عَذَابٍ غَلِيظٍۢ ﴿٥٠﴾\\
\textamh{51.\  } & وَإِذَآ أَنعَمنَا عَلَى ٱلإِنسَـٰنِ أَعرَضَ وَنَـَٔا بِجَانِبِهِۦ وَإِذَا مَسَّهُ ٱلشَّرُّ فَذُو دُعَآءٍ عَرِيضٍۢ ﴿٥١﴾\\
\textamh{52.\  } & قُل أَرَءَيتُم إِن كَانَ مِن عِندِ ٱللَّهِ ثُمَّ كَفَرتُم بِهِۦ مَن أَضَلُّ مِمَّن هُوَ فِى شِقَاقٍۭ بَعِيدٍۢ ﴿٥٢﴾\\
\textamh{53.\  } & سَنُرِيهِم ءَايَـٰتِنَا فِى ٱلءَافَاقِ وَفِىٓ أَنفُسِهِم حَتَّىٰ يَتَبَيَّنَ لَهُم أَنَّهُ ٱلحَقُّ ۗ أَوَلَم يَكفِ بِرَبِّكَ أَنَّهُۥ عَلَىٰ كُلِّ شَىءٍۢ شَهِيدٌ ﴿٥٣﴾\\
\textamh{54.\  } & أَلَآ إِنَّهُم فِى مِريَةٍۢ مِّن لِّقَآءِ رَبِّهِم ۗ أَلَآ إِنَّهُۥ بِكُلِّ شَىءٍۢ مُّحِيطٌۢ ﴿٥٤﴾\\
\end{longtable} \newpage
