%% License: BSD style (Berkley) (i.e. Put the Copyright owner's name always)
%% Writer and Copyright (to): Bewketu(Bilal) Tadilo (2016-17)
\shadowbox{\section{\LR{\textamharic{ሱራቱ አልዙኽሩፍ -}  \RL{سوره  الزخرف}}}}
\begin{longtable}{%
  @{}
    p{.5\textwidth}
  @{~~~~~~~~~~~~~}||
    p{.5\textwidth}
    @{}
}
\nopagebreak
\textamh{\ \ \ \ \ \  ቢስሚላሂ አራህመኒ ራሂይም } &  بِسمِ ٱللَّهِ ٱلرَّحمَـٰنِ ٱلرَّحِيمِ\\
\textamh{1.\  } &  حمٓ ﴿١﴾\\
\textamh{2.\  } & وَٱلكِتَـٰبِ ٱلمُبِينِ ﴿٢﴾\\
\textamh{3.\  } & إِنَّا جَعَلنَـٰهُ قُرءَٰنًا عَرَبِيًّۭا لَّعَلَّكُم تَعقِلُونَ ﴿٣﴾\\
\textamh{4.\  } & وَإِنَّهُۥ فِىٓ أُمِّ ٱلكِتَـٰبِ لَدَينَا لَعَلِىٌّ حَكِيمٌ ﴿٤﴾\\
\textamh{5.\  } & أَفَنَضرِبُ عَنكُمُ ٱلذِّكرَ صَفحًا أَن كُنتُم قَومًۭا مُّسرِفِينَ ﴿٥﴾\\
\textamh{6.\  } & وَكَم أَرسَلنَا مِن نَّبِىٍّۢ فِى ٱلأَوَّلِينَ ﴿٦﴾\\
\textamh{7.\  } & وَمَا يَأتِيهِم مِّن نَّبِىٍّ إِلَّا كَانُوا۟ بِهِۦ يَستَهزِءُونَ ﴿٧﴾\\
\textamh{8.\  } & فَأَهلَكنَآ أَشَدَّ مِنهُم بَطشًۭا وَمَضَىٰ مَثَلُ ٱلأَوَّلِينَ ﴿٨﴾\\
\textamh{9.\  } & وَلَئِن سَأَلتَهُم مَّن خَلَقَ ٱلسَّمَـٰوَٟتِ وَٱلأَرضَ لَيَقُولُنَّ خَلَقَهُنَّ ٱلعَزِيزُ ٱلعَلِيمُ ﴿٩﴾\\
\textamh{10.\  } & ٱلَّذِى جَعَلَ لَكُمُ ٱلأَرضَ مَهدًۭا وَجَعَلَ لَكُم فِيهَا سُبُلًۭا لَّعَلَّكُم تَهتَدُونَ ﴿١٠﴾\\
\textamh{11.\  } & وَٱلَّذِى نَزَّلَ مِنَ ٱلسَّمَآءِ مَآءًۢ بِقَدَرٍۢ فَأَنشَرنَا بِهِۦ بَلدَةًۭ مَّيتًۭا ۚ كَذَٟلِكَ تُخرَجُونَ ﴿١١﴾\\
\textamh{12.\  } & وَٱلَّذِى خَلَقَ ٱلأَزوَٟجَ كُلَّهَا وَجَعَلَ لَكُم مِّنَ ٱلفُلكِ وَٱلأَنعَـٰمِ مَا تَركَبُونَ ﴿١٢﴾\\
\textamh{13.\  } & لِتَستَوُۥا۟ عَلَىٰ ظُهُورِهِۦ ثُمَّ تَذكُرُوا۟ نِعمَةَ رَبِّكُم إِذَا ٱستَوَيتُم عَلَيهِ وَتَقُولُوا۟ سُبحَـٰنَ ٱلَّذِى سَخَّرَ لَنَا هَـٰذَا وَمَا كُنَّا لَهُۥ مُقرِنِينَ ﴿١٣﴾\\
\textamh{14.\  } & وَإِنَّآ إِلَىٰ رَبِّنَا لَمُنقَلِبُونَ ﴿١٤﴾\\
\textamh{15.\  } & وَجَعَلُوا۟ لَهُۥ مِن عِبَادِهِۦ جُزءًا ۚ إِنَّ ٱلإِنسَـٰنَ لَكَفُورٌۭ مُّبِينٌ ﴿١٥﴾\\
\textamh{16.\  } & أَمِ ٱتَّخَذَ مِمَّا يَخلُقُ بَنَاتٍۢ وَأَصفَىٰكُم بِٱلبَنِينَ ﴿١٦﴾\\
\textamh{17.\  } & وَإِذَا بُشِّرَ أَحَدُهُم بِمَا ضَرَبَ لِلرَّحمَـٰنِ مَثَلًۭا ظَلَّ وَجهُهُۥ مُسوَدًّۭا وَهُوَ كَظِيمٌ ﴿١٧﴾\\
\textamh{18.\  } & أَوَمَن يُنَشَّؤُا۟ فِى ٱلحِليَةِ وَهُوَ فِى ٱلخِصَامِ غَيرُ مُبِينٍۢ ﴿١٨﴾\\
\textamh{19.\  } & وَجَعَلُوا۟ ٱلمَلَـٰٓئِكَةَ ٱلَّذِينَ هُم عِبَٰدُ ٱلرَّحمَـٰنِ إِنَـٰثًا ۚ أَشَهِدُوا۟ خَلقَهُم ۚ سَتُكتَبُ شَهَـٰدَتُهُم وَيُسـَٔلُونَ ﴿١٩﴾\\
\textamh{20.\  } & وَقَالُوا۟ لَو شَآءَ ٱلرَّحمَـٰنُ مَا عَبَدنَـٰهُم ۗ مَّا لَهُم بِذَٟلِكَ مِن عِلمٍ ۖ إِن هُم إِلَّا يَخرُصُونَ ﴿٢٠﴾\\
\textamh{21.\  } & أَم ءَاتَينَـٰهُم كِتَـٰبًۭا مِّن قَبلِهِۦ فَهُم بِهِۦ مُستَمسِكُونَ ﴿٢١﴾\\
\textamh{22.\  } & بَل قَالُوٓا۟ إِنَّا وَجَدنَآ ءَابَآءَنَا عَلَىٰٓ أُمَّةٍۢ وَإِنَّا عَلَىٰٓ ءَاثَـٰرِهِم مُّهتَدُونَ ﴿٢٢﴾\\
\textamh{23.\  } & وَكَذَٟلِكَ مَآ أَرسَلنَا مِن قَبلِكَ فِى قَريَةٍۢ مِّن نَّذِيرٍ إِلَّا قَالَ مُترَفُوهَآ إِنَّا وَجَدنَآ ءَابَآءَنَا عَلَىٰٓ أُمَّةٍۢ وَإِنَّا عَلَىٰٓ ءَاثَـٰرِهِم مُّقتَدُونَ ﴿٢٣﴾\\
\textamh{24.\  } & ۞ قَـٰلَ أَوَلَو جِئتُكُم بِأَهدَىٰ مِمَّا وَجَدتُّم عَلَيهِ ءَابَآءَكُم ۖ قَالُوٓا۟ إِنَّا بِمَآ أُرسِلتُم بِهِۦ كَـٰفِرُونَ ﴿٢٤﴾\\
\textamh{25.\  } & فَٱنتَقَمنَا مِنهُم ۖ فَٱنظُر كَيفَ كَانَ عَـٰقِبَةُ ٱلمُكَذِّبِينَ ﴿٢٥﴾\\
\textamh{26.\  } & وَإِذ قَالَ إِبرَٰهِيمُ لِأَبِيهِ وَقَومِهِۦٓ إِنَّنِى بَرَآءٌۭ مِّمَّا تَعبُدُونَ ﴿٢٦﴾\\
\textamh{27.\  } & إِلَّا ٱلَّذِى فَطَرَنِى فَإِنَّهُۥ سَيَهدِينِ ﴿٢٧﴾\\
\textamh{28.\  } & وَجَعَلَهَا كَلِمَةًۢ بَاقِيَةًۭ فِى عَقِبِهِۦ لَعَلَّهُم يَرجِعُونَ ﴿٢٨﴾\\
\textamh{29.\  } & بَل مَتَّعتُ هَـٰٓؤُلَآءِ وَءَابَآءَهُم حَتَّىٰ جَآءَهُمُ ٱلحَقُّ وَرَسُولٌۭ مُّبِينٌۭ ﴿٢٩﴾\\
\textamh{30.\  } & وَلَمَّا جَآءَهُمُ ٱلحَقُّ قَالُوا۟ هَـٰذَا سِحرٌۭ وَإِنَّا بِهِۦ كَـٰفِرُونَ ﴿٣٠﴾\\
\textamh{31.\  } & وَقَالُوا۟ لَولَا نُزِّلَ هَـٰذَا ٱلقُرءَانُ عَلَىٰ رَجُلٍۢ مِّنَ ٱلقَريَتَينِ عَظِيمٍ ﴿٣١﴾\\
\textamh{32.\  } & أَهُم يَقسِمُونَ رَحمَتَ رَبِّكَ ۚ نَحنُ قَسَمنَا بَينَهُم مَّعِيشَتَهُم فِى ٱلحَيَوٰةِ ٱلدُّنيَا ۚ وَرَفَعنَا بَعضَهُم فَوقَ بَعضٍۢ دَرَجَٰتٍۢ لِّيَتَّخِذَ بَعضُهُم بَعضًۭا سُخرِيًّۭا ۗ وَرَحمَتُ رَبِّكَ خَيرٌۭ مِّمَّا يَجمَعُونَ ﴿٣٢﴾\\
\textamh{33.\  } & وَلَولَآ أَن يَكُونَ ٱلنَّاسُ أُمَّةًۭ وَٟحِدَةًۭ لَّجَعَلنَا لِمَن يَكفُرُ بِٱلرَّحمَـٰنِ لِبُيُوتِهِم سُقُفًۭا مِّن فِضَّةٍۢ وَمَعَارِجَ عَلَيهَا يَظهَرُونَ ﴿٣٣﴾\\
\textamh{34.\  } & وَلِبُيُوتِهِم أَبوَٟبًۭا وَسُرُرًا عَلَيهَا يَتَّكِـُٔونَ ﴿٣٤﴾\\
\textamh{35.\  } & وَزُخرُفًۭا ۚ وَإِن كُلُّ ذَٟلِكَ لَمَّا مَتَـٰعُ ٱلحَيَوٰةِ ٱلدُّنيَا ۚ وَٱلءَاخِرَةُ عِندَ رَبِّكَ لِلمُتَّقِينَ ﴿٣٥﴾\\
\textamh{36.\  } & وَمَن يَعشُ عَن ذِكرِ ٱلرَّحمَـٰنِ نُقَيِّض لَهُۥ شَيطَٰنًۭا فَهُوَ لَهُۥ قَرِينٌۭ ﴿٣٦﴾\\
\textamh{37.\  } & وَإِنَّهُم لَيَصُدُّونَهُم عَنِ ٱلسَّبِيلِ وَيَحسَبُونَ أَنَّهُم مُّهتَدُونَ ﴿٣٧﴾\\
\textamh{38.\  } & حَتَّىٰٓ إِذَا جَآءَنَا قَالَ يَـٰلَيتَ بَينِى وَبَينَكَ بُعدَ ٱلمَشرِقَينِ فَبِئسَ ٱلقَرِينُ ﴿٣٨﴾\\
\textamh{39.\  } & وَلَن يَنفَعَكُمُ ٱليَومَ إِذ ظَّلَمتُم أَنَّكُم فِى ٱلعَذَابِ مُشتَرِكُونَ ﴿٣٩﴾\\
\textamh{40.\  } & أَفَأَنتَ تُسمِعُ ٱلصُّمَّ أَو تَهدِى ٱلعُمىَ وَمَن كَانَ فِى ضَلَـٰلٍۢ مُّبِينٍۢ ﴿٤٠﴾\\
\textamh{41.\  } & فَإِمَّا نَذهَبَنَّ بِكَ فَإِنَّا مِنهُم مُّنتَقِمُونَ ﴿٤١﴾\\
\textamh{42.\  } & أَو نُرِيَنَّكَ ٱلَّذِى وَعَدنَـٰهُم فَإِنَّا عَلَيهِم مُّقتَدِرُونَ ﴿٤٢﴾\\
\textamh{43.\  } & فَٱستَمسِك بِٱلَّذِىٓ أُوحِىَ إِلَيكَ ۖ إِنَّكَ عَلَىٰ صِرَٰطٍۢ مُّستَقِيمٍۢ ﴿٤٣﴾\\
\textamh{44.\  } & وَإِنَّهُۥ لَذِكرٌۭ لَّكَ وَلِقَومِكَ ۖ وَسَوفَ تُسـَٔلُونَ ﴿٤٤﴾\\
\textamh{45.\  } & وَسـَٔل مَن أَرسَلنَا مِن قَبلِكَ مِن رُّسُلِنَآ أَجَعَلنَا مِن دُونِ ٱلرَّحمَـٰنِ ءَالِهَةًۭ يُعبَدُونَ ﴿٤٥﴾\\
\textamh{46.\  } & وَلَقَد أَرسَلنَا مُوسَىٰ بِـَٔايَـٰتِنَآ إِلَىٰ فِرعَونَ وَمَلَإِي۟هِۦ فَقَالَ إِنِّى رَسُولُ رَبِّ ٱلعَـٰلَمِينَ ﴿٤٦﴾\\
\textamh{47.\  } & فَلَمَّا جَآءَهُم بِـَٔايَـٰتِنَآ إِذَا هُم مِّنهَا يَضحَكُونَ ﴿٤٧﴾\\
\textamh{48.\  } & وَمَا نُرِيهِم مِّن ءَايَةٍ إِلَّا هِىَ أَكبَرُ مِن أُختِهَا ۖ وَأَخَذنَـٰهُم بِٱلعَذَابِ لَعَلَّهُم يَرجِعُونَ ﴿٤٨﴾\\
\textamh{49.\  } & وَقَالُوا۟ يَـٰٓأَيُّهَ ٱلسَّاحِرُ ٱدعُ لَنَا رَبَّكَ بِمَا عَهِدَ عِندَكَ إِنَّنَا لَمُهتَدُونَ ﴿٤٩﴾\\
\textamh{50.\  } & فَلَمَّا كَشَفنَا عَنهُمُ ٱلعَذَابَ إِذَا هُم يَنكُثُونَ ﴿٥٠﴾\\
\textamh{51.\  } & وَنَادَىٰ فِرعَونُ فِى قَومِهِۦ قَالَ يَـٰقَومِ أَلَيسَ لِى مُلكُ مِصرَ وَهَـٰذِهِ ٱلأَنهَـٰرُ تَجرِى مِن تَحتِىٓ ۖ أَفَلَا تُبصِرُونَ ﴿٥١﴾\\
\textamh{52.\  } & أَم أَنَا۠ خَيرٌۭ مِّن هَـٰذَا ٱلَّذِى هُوَ مَهِينٌۭ وَلَا يَكَادُ يُبِينُ ﴿٥٢﴾\\
\textamh{53.\  } & فَلَولَآ أُلقِىَ عَلَيهِ أَسوِرَةٌۭ مِّن ذَهَبٍ أَو جَآءَ مَعَهُ ٱلمَلَـٰٓئِكَةُ مُقتَرِنِينَ ﴿٥٣﴾\\
\textamh{54.\  } & فَٱستَخَفَّ قَومَهُۥ فَأَطَاعُوهُ ۚ إِنَّهُم كَانُوا۟ قَومًۭا فَـٰسِقِينَ ﴿٥٤﴾\\
\textamh{55.\  } & فَلَمَّآ ءَاسَفُونَا ٱنتَقَمنَا مِنهُم فَأَغرَقنَـٰهُم أَجمَعِينَ ﴿٥٥﴾\\
\textamh{56.\  } & فَجَعَلنَـٰهُم سَلَفًۭا وَمَثَلًۭا لِّلءَاخِرِينَ ﴿٥٦﴾\\
\textamh{57.\  } & ۞ وَلَمَّا ضُرِبَ ٱبنُ مَريَمَ مَثَلًا إِذَا قَومُكَ مِنهُ يَصِدُّونَ ﴿٥٧﴾\\
\textamh{58.\  } & وَقَالُوٓا۟ ءَأَٰلِهَتُنَا خَيرٌ أَم هُوَ ۚ مَا ضَرَبُوهُ لَكَ إِلَّا جَدَلًۢا ۚ بَل هُم قَومٌ خَصِمُونَ ﴿٥٨﴾\\
\textamh{59.\  } & إِن هُوَ إِلَّا عَبدٌ أَنعَمنَا عَلَيهِ وَجَعَلنَـٰهُ مَثَلًۭا لِّبَنِىٓ إِسرَٰٓءِيلَ ﴿٥٩﴾\\
\textamh{60.\  } & وَلَو نَشَآءُ لَجَعَلنَا مِنكُم مَّلَـٰٓئِكَةًۭ فِى ٱلأَرضِ يَخلُفُونَ ﴿٦٠﴾\\
\textamh{61.\  } & وَإِنَّهُۥ لَعِلمٌۭ لِّلسَّاعَةِ فَلَا تَمتَرُنَّ بِهَا وَٱتَّبِعُونِ ۚ هَـٰذَا صِرَٰطٌۭ مُّستَقِيمٌۭ ﴿٦١﴾\\
\textamh{62.\  } & وَلَا يَصُدَّنَّكُمُ ٱلشَّيطَٰنُ ۖ إِنَّهُۥ لَكُم عَدُوٌّۭ مُّبِينٌۭ ﴿٦٢﴾\\
\textamh{63.\  } & وَلَمَّا جَآءَ عِيسَىٰ بِٱلبَيِّنَـٰتِ قَالَ قَد جِئتُكُم بِٱلحِكمَةِ وَلِأُبَيِّنَ لَكُم بَعضَ ٱلَّذِى تَختَلِفُونَ فِيهِ ۖ فَٱتَّقُوا۟ ٱللَّهَ وَأَطِيعُونِ ﴿٦٣﴾\\
\textamh{64.\  } & إِنَّ ٱللَّهَ هُوَ رَبِّى وَرَبُّكُم فَٱعبُدُوهُ ۚ هَـٰذَا صِرَٰطٌۭ مُّستَقِيمٌۭ ﴿٦٤﴾\\
\textamh{65.\  } & فَٱختَلَفَ ٱلأَحزَابُ مِنۢ بَينِهِم ۖ فَوَيلٌۭ لِّلَّذِينَ ظَلَمُوا۟ مِن عَذَابِ يَومٍ أَلِيمٍ ﴿٦٥﴾\\
\textamh{66.\  } & هَل يَنظُرُونَ إِلَّا ٱلسَّاعَةَ أَن تَأتِيَهُم بَغتَةًۭ وَهُم لَا يَشعُرُونَ ﴿٦٦﴾\\
\textamh{67.\  } & ٱلأَخِلَّآءُ يَومَئِذٍۭ بَعضُهُم لِبَعضٍ عَدُوٌّ إِلَّا ٱلمُتَّقِينَ ﴿٦٧﴾\\
\textamh{68.\  } & يَـٰعِبَادِ لَا خَوفٌ عَلَيكُمُ ٱليَومَ وَلَآ أَنتُم تَحزَنُونَ ﴿٦٨﴾\\
\textamh{69.\  } & ٱلَّذِينَ ءَامَنُوا۟ بِـَٔايَـٰتِنَا وَكَانُوا۟ مُسلِمِينَ ﴿٦٩﴾\\
\textamh{70.\  } & ٱدخُلُوا۟ ٱلجَنَّةَ أَنتُم وَأَزوَٟجُكُم تُحبَرُونَ ﴿٧٠﴾\\
\textamh{71.\  } & يُطَافُ عَلَيهِم بِصِحَافٍۢ مِّن ذَهَبٍۢ وَأَكوَابٍۢ ۖ وَفِيهَا مَا تَشتَهِيهِ ٱلأَنفُسُ وَتَلَذُّ ٱلأَعيُنُ ۖ وَأَنتُم فِيهَا خَـٰلِدُونَ ﴿٧١﴾\\
\textamh{72.\  } & وَتِلكَ ٱلجَنَّةُ ٱلَّتِىٓ أُورِثتُمُوهَا بِمَا كُنتُم تَعمَلُونَ ﴿٧٢﴾\\
\textamh{73.\  } & لَكُم فِيهَا فَـٰكِهَةٌۭ كَثِيرَةٌۭ مِّنهَا تَأكُلُونَ ﴿٧٣﴾\\
\textamh{74.\  } & إِنَّ ٱلمُجرِمِينَ فِى عَذَابِ جَهَنَّمَ خَـٰلِدُونَ ﴿٧٤﴾\\
\textamh{75.\  } & لَا يُفَتَّرُ عَنهُم وَهُم فِيهِ مُبلِسُونَ ﴿٧٥﴾\\
\textamh{76.\  } & وَمَا ظَلَمنَـٰهُم وَلَـٰكِن كَانُوا۟ هُمُ ٱلظَّـٰلِمِينَ ﴿٧٦﴾\\
\textamh{77.\  } & وَنَادَوا۟ يَـٰمَـٰلِكُ لِيَقضِ عَلَينَا رَبُّكَ ۖ قَالَ إِنَّكُم مَّٰكِثُونَ ﴿٧٧﴾\\
\textamh{78.\  } & لَقَد جِئنَـٰكُم بِٱلحَقِّ وَلَـٰكِنَّ أَكثَرَكُم لِلحَقِّ كَـٰرِهُونَ ﴿٧٨﴾\\
\textamh{79.\  } & أَم أَبرَمُوٓا۟ أَمرًۭا فَإِنَّا مُبرِمُونَ ﴿٧٩﴾\\
\textamh{80.\  } & أَم يَحسَبُونَ أَنَّا لَا نَسمَعُ سِرَّهُم وَنَجوَىٰهُم ۚ بَلَىٰ وَرُسُلُنَا لَدَيهِم يَكتُبُونَ ﴿٨٠﴾\\
\textamh{81.\  } & قُل إِن كَانَ لِلرَّحمَـٰنِ وَلَدٌۭ فَأَنَا۠ أَوَّلُ ٱلعَـٰبِدِينَ ﴿٨١﴾\\
\textamh{82.\  } & سُبحَـٰنَ رَبِّ ٱلسَّمَـٰوَٟتِ وَٱلأَرضِ رَبِّ ٱلعَرشِ عَمَّا يَصِفُونَ ﴿٨٢﴾\\
\textamh{83.\  } & فَذَرهُم يَخُوضُوا۟ وَيَلعَبُوا۟ حَتَّىٰ يُلَـٰقُوا۟ يَومَهُمُ ٱلَّذِى يُوعَدُونَ ﴿٨٣﴾\\
\textamh{84.\  } & وَهُوَ ٱلَّذِى فِى ٱلسَّمَآءِ إِلَـٰهٌۭ وَفِى ٱلأَرضِ إِلَـٰهٌۭ ۚ وَهُوَ ٱلحَكِيمُ ٱلعَلِيمُ ﴿٨٤﴾\\
\textamh{85.\  } & وَتَبَارَكَ ٱلَّذِى لَهُۥ مُلكُ ٱلسَّمَـٰوَٟتِ وَٱلأَرضِ وَمَا بَينَهُمَا وَعِندَهُۥ عِلمُ ٱلسَّاعَةِ وَإِلَيهِ تُرجَعُونَ ﴿٨٥﴾\\
\textamh{86.\  } & وَلَا يَملِكُ ٱلَّذِينَ يَدعُونَ مِن دُونِهِ ٱلشَّفَـٰعَةَ إِلَّا مَن شَهِدَ بِٱلحَقِّ وَهُم يَعلَمُونَ ﴿٨٦﴾\\
\textamh{87.\  } & وَلَئِن سَأَلتَهُم مَّن خَلَقَهُم لَيَقُولُنَّ ٱللَّهُ ۖ فَأَنَّىٰ يُؤفَكُونَ ﴿٨٧﴾\\
\textamh{88.\  } & وَقِيلِهِۦ يَـٰرَبِّ إِنَّ هَـٰٓؤُلَآءِ قَومٌۭ لَّا يُؤمِنُونَ ﴿٨٨﴾\\
\textamh{89.\  } & فَٱصفَح عَنهُم وَقُل سَلَـٰمٌۭ ۚ فَسَوفَ يَعلَمُونَ ﴿٨٩﴾\\
\end{longtable} \newpage
