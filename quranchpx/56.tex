%% License: BSD style (Berkley) (i.e. Put the Copyright owner's name always)
%% Writer and Copyright (to): Bewketu(Bilal) Tadilo (2016-17)
\shadowbox{\section{\LR{\textamharic{ሱራቱ አልዋቂያ -}  \RL{سوره  الواقعة}}}}
\begin{longtable}{%
  @{}
    p{.5\textwidth}
  @{~~~~~~~~~~~~~}||
    p{.5\textwidth}
    @{}
}
\nopagebreak
\textamh{\ \ \ \ \ \  ቢስሚላሂ አራህመኒ ራሂይም } &  بِسمِ ٱللَّهِ ٱلرَّحمَـٰنِ ٱلرَّحِيمِ\\
\textamh{1.\  } &  إِذَا وَقَعَتِ ٱلوَاقِعَةُ ﴿١﴾\\
\textamh{2.\  } & لَيسَ لِوَقعَتِهَا كَاذِبَةٌ ﴿٢﴾\\
\textamh{3.\  } & خَافِضَةٌۭ رَّافِعَةٌ ﴿٣﴾\\
\textamh{4.\  } & إِذَا رُجَّتِ ٱلأَرضُ رَجًّۭا ﴿٤﴾\\
\textamh{5.\  } & وَبُسَّتِ ٱلجِبَالُ بَسًّۭا ﴿٥﴾\\
\textamh{6.\  } & فَكَانَت هَبَآءًۭ مُّنۢبَثًّۭا ﴿٦﴾\\
\textamh{7.\  } & وَكُنتُم أَزوَٟجًۭا ثَلَـٰثَةًۭ ﴿٧﴾\\
\textamh{8.\  } & فَأَصحَـٰبُ ٱلمَيمَنَةِ مَآ أَصحَـٰبُ ٱلمَيمَنَةِ ﴿٨﴾\\
\textamh{9.\  } & وَأَصحَـٰبُ ٱلمَشـَٔمَةِ مَآ أَصحَـٰبُ ٱلمَشـَٔمَةِ ﴿٩﴾\\
\textamh{10.\  } & وَٱلسَّٰبِقُونَ ٱلسَّٰبِقُونَ ﴿١٠﴾\\
\textamh{11.\  } & أُو۟لَـٰٓئِكَ ٱلمُقَرَّبُونَ ﴿١١﴾\\
\textamh{12.\  } & فِى جَنَّـٰتِ ٱلنَّعِيمِ ﴿١٢﴾\\
\textamh{13.\  } & ثُلَّةٌۭ مِّنَ ٱلأَوَّلِينَ ﴿١٣﴾\\
\textamh{14.\  } & وَقَلِيلٌۭ مِّنَ ٱلءَاخِرِينَ ﴿١٤﴾\\
\textamh{15.\  } & عَلَىٰ سُرُرٍۢ مَّوضُونَةٍۢ ﴿١٥﴾\\
\textamh{16.\  } & مُّتَّكِـِٔينَ عَلَيهَا مُتَقَـٰبِلِينَ ﴿١٦﴾\\
\textamh{17.\  } & يَطُوفُ عَلَيهِم وِلدَٟنٌۭ مُّخَلَّدُونَ ﴿١٧﴾\\
\textamh{18.\  } & بِأَكوَابٍۢ وَأَبَارِيقَ وَكَأسٍۢ مِّن مَّعِينٍۢ ﴿١٨﴾\\
\textamh{19.\  } & لَّا يُصَدَّعُونَ عَنهَا وَلَا يُنزِفُونَ ﴿١٩﴾\\
\textamh{20.\  } & وَفَـٰكِهَةٍۢ مِّمَّا يَتَخَيَّرُونَ ﴿٢٠﴾\\
\textamh{21.\  } & وَلَحمِ طَيرٍۢ مِّمَّا يَشتَهُونَ ﴿٢١﴾\\
\textamh{22.\  } & وَحُورٌ عِينٌۭ ﴿٢٢﴾\\
\textamh{23.\  } & كَأَمثَـٰلِ ٱللُّؤلُؤِ ٱلمَكنُونِ ﴿٢٣﴾\\
\textamh{24.\  } & جَزَآءًۢ بِمَا كَانُوا۟ يَعمَلُونَ ﴿٢٤﴾\\
\textamh{25.\  } & لَا يَسمَعُونَ فِيهَا لَغوًۭا وَلَا تَأثِيمًا ﴿٢٥﴾\\
\textamh{26.\  } & إِلَّا قِيلًۭا سَلَـٰمًۭا سَلَـٰمًۭا ﴿٢٦﴾\\
\textamh{27.\  } & وَأَصحَـٰبُ ٱليَمِينِ مَآ أَصحَـٰبُ ٱليَمِينِ ﴿٢٧﴾\\
\textamh{28.\  } & فِى سِدرٍۢ مَّخضُودٍۢ ﴿٢٨﴾\\
\textamh{29.\  } & وَطَلحٍۢ مَّنضُودٍۢ ﴿٢٩﴾\\
\textamh{30.\  } & وَظِلٍّۢ مَّمدُودٍۢ ﴿٣٠﴾\\
\textamh{31.\  } & وَمَآءٍۢ مَّسكُوبٍۢ ﴿٣١﴾\\
\textamh{32.\  } & وَفَـٰكِهَةٍۢ كَثِيرَةٍۢ ﴿٣٢﴾\\
\textamh{33.\  } & لَّا مَقطُوعَةٍۢ وَلَا مَمنُوعَةٍۢ ﴿٣٣﴾\\
\textamh{34.\  } & وَفُرُشٍۢ مَّرفُوعَةٍ ﴿٣٤﴾\\
\textamh{35.\  } & إِنَّآ أَنشَأنَـٰهُنَّ إِنشَآءًۭ ﴿٣٥﴾\\
\textamh{36.\  } & فَجَعَلنَـٰهُنَّ أَبكَارًا ﴿٣٦﴾\\
\textamh{37.\  } & عُرُبًا أَترَابًۭا ﴿٣٧﴾\\
\textamh{38.\  } & لِّأَصحَـٰبِ ٱليَمِينِ ﴿٣٨﴾\\
\textamh{39.\  } & ثُلَّةٌۭ مِّنَ ٱلأَوَّلِينَ ﴿٣٩﴾\\
\textamh{40.\  } & وَثُلَّةٌۭ مِّنَ ٱلءَاخِرِينَ ﴿٤٠﴾\\
\textamh{41.\  } & وَأَصحَـٰبُ ٱلشِّمَالِ مَآ أَصحَـٰبُ ٱلشِّمَالِ ﴿٤١﴾\\
\textamh{42.\  } & فِى سَمُومٍۢ وَحَمِيمٍۢ ﴿٤٢﴾\\
\textamh{43.\  } & وَظِلٍّۢ مِّن يَحمُومٍۢ ﴿٤٣﴾\\
\textamh{44.\  } & لَّا بَارِدٍۢ وَلَا كَرِيمٍ ﴿٤٤﴾\\
\textamh{45.\  } & إِنَّهُم كَانُوا۟ قَبلَ ذَٟلِكَ مُترَفِينَ ﴿٤٥﴾\\
\textamh{46.\  } & وَكَانُوا۟ يُصِرُّونَ عَلَى ٱلحِنثِ ٱلعَظِيمِ ﴿٤٦﴾\\
\textamh{47.\  } & وَكَانُوا۟ يَقُولُونَ أَئِذَا مِتنَا وَكُنَّا تُرَابًۭا وَعِظَـٰمًا أَءِنَّا لَمَبعُوثُونَ ﴿٤٧﴾\\
\textamh{48.\  } & أَوَءَابَآؤُنَا ٱلأَوَّلُونَ ﴿٤٨﴾\\
\textamh{49.\  } & قُل إِنَّ ٱلأَوَّلِينَ وَٱلءَاخِرِينَ ﴿٤٩﴾\\
\textamh{50.\  } & لَمَجمُوعُونَ إِلَىٰ مِيقَـٰتِ يَومٍۢ مَّعلُومٍۢ ﴿٥٠﴾\\
\textamh{51.\  } & ثُمَّ إِنَّكُم أَيُّهَا ٱلضَّآلُّونَ ٱلمُكَذِّبُونَ ﴿٥١﴾\\
\textamh{52.\  } & لَءَاكِلُونَ مِن شَجَرٍۢ مِّن زَقُّومٍۢ ﴿٥٢﴾\\
\textamh{53.\  } & فَمَالِـُٔونَ مِنهَا ٱلبُطُونَ ﴿٥٣﴾\\
\textamh{54.\  } & فَشَـٰرِبُونَ عَلَيهِ مِنَ ٱلحَمِيمِ ﴿٥٤﴾\\
\textamh{55.\  } & فَشَـٰرِبُونَ شُربَ ٱلهِيمِ ﴿٥٥﴾\\
\textamh{56.\  } & هَـٰذَا نُزُلُهُم يَومَ ٱلدِّينِ ﴿٥٦﴾\\
\textamh{57.\  } & نَحنُ خَلَقنَـٰكُم فَلَولَا تُصَدِّقُونَ ﴿٥٧﴾\\
\textamh{58.\  } & أَفَرَءَيتُم مَّا تُمنُونَ ﴿٥٨﴾\\
\textamh{59.\  } & ءَأَنتُم تَخلُقُونَهُۥٓ أَم نَحنُ ٱلخَـٰلِقُونَ ﴿٥٩﴾\\
\textamh{60.\  } & نَحنُ قَدَّرنَا بَينَكُمُ ٱلمَوتَ وَمَا نَحنُ بِمَسبُوقِينَ ﴿٦٠﴾\\
\textamh{61.\  } & عَلَىٰٓ أَن نُّبَدِّلَ أَمثَـٰلَكُم وَنُنشِئَكُم فِى مَا لَا تَعلَمُونَ ﴿٦١﴾\\
\textamh{62.\  } & وَلَقَد عَلِمتُمُ ٱلنَّشأَةَ ٱلأُولَىٰ فَلَولَا تَذَكَّرُونَ ﴿٦٢﴾\\
\textamh{63.\  } & أَفَرَءَيتُم مَّا تَحرُثُونَ ﴿٦٣﴾\\
\textamh{64.\  } & ءَأَنتُم تَزرَعُونَهُۥٓ أَم نَحنُ ٱلزَّٰرِعُونَ ﴿٦٤﴾\\
\textamh{65.\  } & لَو نَشَآءُ لَجَعَلنَـٰهُ حُطَٰمًۭا فَظَلتُم تَفَكَّهُونَ ﴿٦٥﴾\\
\textamh{66.\  } & إِنَّا لَمُغرَمُونَ ﴿٦٦﴾\\
\textamh{67.\  } & بَل نَحنُ مَحرُومُونَ ﴿٦٧﴾\\
\textamh{68.\  } & أَفَرَءَيتُمُ ٱلمَآءَ ٱلَّذِى تَشرَبُونَ ﴿٦٨﴾\\
\textamh{69.\  } & ءَأَنتُم أَنزَلتُمُوهُ مِنَ ٱلمُزنِ أَم نَحنُ ٱلمُنزِلُونَ ﴿٦٩﴾\\
\textamh{70.\  } & لَو نَشَآءُ جَعَلنَـٰهُ أُجَاجًۭا فَلَولَا تَشكُرُونَ ﴿٧٠﴾\\
\textamh{71.\  } & أَفَرَءَيتُمُ ٱلنَّارَ ٱلَّتِى تُورُونَ ﴿٧١﴾\\
\textamh{72.\  } & ءَأَنتُم أَنشَأتُم شَجَرَتَهَآ أَم نَحنُ ٱلمُنشِـُٔونَ ﴿٧٢﴾\\
\textamh{73.\  } & نَحنُ جَعَلنَـٰهَا تَذكِرَةًۭ وَمَتَـٰعًۭا لِّلمُقوِينَ ﴿٧٣﴾\\
\textamh{74.\  } & فَسَبِّح بِٱسمِ رَبِّكَ ٱلعَظِيمِ ﴿٧٤﴾\\
\textamh{75.\  } & ۞ فَلَآ أُقسِمُ بِمَوَٟقِعِ ٱلنُّجُومِ ﴿٧٥﴾\\
\textamh{76.\  } & وَإِنَّهُۥ لَقَسَمٌۭ لَّو تَعلَمُونَ عَظِيمٌ ﴿٧٦﴾\\
\textamh{77.\  } & إِنَّهُۥ لَقُرءَانٌۭ كَرِيمٌۭ ﴿٧٧﴾\\
\textamh{78.\  } & فِى كِتَـٰبٍۢ مَّكنُونٍۢ ﴿٧٨﴾\\
\textamh{79.\  } & لَّا يَمَسُّهُۥٓ إِلَّا ٱلمُطَهَّرُونَ ﴿٧٩﴾\\
\textamh{80.\  } & تَنزِيلٌۭ مِّن رَّبِّ ٱلعَـٰلَمِينَ ﴿٨٠﴾\\
\textamh{81.\  } & أَفَبِهَـٰذَا ٱلحَدِيثِ أَنتُم مُّدهِنُونَ ﴿٨١﴾\\
\textamh{82.\  } & وَتَجعَلُونَ رِزقَكُم أَنَّكُم تُكَذِّبُونَ ﴿٨٢﴾\\
\textamh{83.\  } & فَلَولَآ إِذَا بَلَغَتِ ٱلحُلقُومَ ﴿٨٣﴾\\
\textamh{84.\  } & وَأَنتُم حِينَئِذٍۢ تَنظُرُونَ ﴿٨٤﴾\\
\textamh{85.\  } & وَنَحنُ أَقرَبُ إِلَيهِ مِنكُم وَلَـٰكِن لَّا تُبصِرُونَ ﴿٨٥﴾\\
\textamh{86.\  } & فَلَولَآ إِن كُنتُم غَيرَ مَدِينِينَ ﴿٨٦﴾\\
\textamh{87.\  } & تَرجِعُونَهَآ إِن كُنتُم صَـٰدِقِينَ ﴿٨٧﴾\\
\textamh{88.\  } & فَأَمَّآ إِن كَانَ مِنَ ٱلمُقَرَّبِينَ ﴿٨٨﴾\\
\textamh{89.\  } & فَرَوحٌۭ وَرَيحَانٌۭ وَجَنَّتُ نَعِيمٍۢ ﴿٨٩﴾\\
\textamh{90.\  } & وَأَمَّآ إِن كَانَ مِن أَصحَـٰبِ ٱليَمِينِ ﴿٩٠﴾\\
\textamh{91.\  } & فَسَلَـٰمٌۭ لَّكَ مِن أَصحَـٰبِ ٱليَمِينِ ﴿٩١﴾\\
\textamh{92.\  } & وَأَمَّآ إِن كَانَ مِنَ ٱلمُكَذِّبِينَ ٱلضَّآلِّينَ ﴿٩٢﴾\\
\textamh{93.\  } & فَنُزُلٌۭ مِّن حَمِيمٍۢ ﴿٩٣﴾\\
\textamh{94.\  } & وَتَصلِيَةُ جَحِيمٍ ﴿٩٤﴾\\
\textamh{95.\  } & إِنَّ هَـٰذَا لَهُوَ حَقُّ ٱليَقِينِ ﴿٩٥﴾\\
\textamh{96.\  } & فَسَبِّح بِٱسمِ رَبِّكَ ٱلعَظِيمِ ﴿٩٦﴾\\
\end{longtable} \newpage
