%% License: BSD style (Berkley) (i.e. Put the Copyright owner's name always)
%% Writer and Copyright (to): Bewketu(Bilal) Tadilo (2016-17)
\shadowbox{\section{\LR{\textamharic{ሱራቱ አልሀዲይድ -}  \RL{سوره  الحديد}}}}
\begin{longtable}{%
  @{}
    p{.5\textwidth}
  @{~~~~~~~~~~~~~}||
    p{.5\textwidth}
    @{}
}
\nopagebreak
\textamh{\ \ \ \ \ \  ቢስሚላሂ አራህመኒ ራሂይም } &  بِسمِ ٱللَّهِ ٱلرَّحمَـٰنِ ٱلرَّحِيمِ\\
\textamh{1.\  } &  سَبَّحَ لِلَّهِ مَا فِى ٱلسَّمَـٰوَٟتِ وَٱلأَرضِ ۖ وَهُوَ ٱلعَزِيزُ ٱلحَكِيمُ ﴿١﴾\\
\textamh{2.\  } & لَهُۥ مُلكُ ٱلسَّمَـٰوَٟتِ وَٱلأَرضِ ۖ يُحىِۦ وَيُمِيتُ ۖ وَهُوَ عَلَىٰ كُلِّ شَىءٍۢ قَدِيرٌ ﴿٢﴾\\
\textamh{3.\  } & هُوَ ٱلأَوَّلُ وَٱلءَاخِرُ وَٱلظَّـٰهِرُ وَٱلبَاطِنُ ۖ وَهُوَ بِكُلِّ شَىءٍ عَلِيمٌ ﴿٣﴾\\
\textamh{4.\  } & هُوَ ٱلَّذِى خَلَقَ ٱلسَّمَـٰوَٟتِ وَٱلأَرضَ فِى سِتَّةِ أَيَّامٍۢ ثُمَّ ٱستَوَىٰ عَلَى ٱلعَرشِ ۚ يَعلَمُ مَا يَلِجُ فِى ٱلأَرضِ وَمَا يَخرُجُ مِنهَا وَمَا يَنزِلُ مِنَ ٱلسَّمَآءِ وَمَا يَعرُجُ فِيهَا ۖ وَهُوَ مَعَكُم أَينَ مَا كُنتُم ۚ وَٱللَّهُ بِمَا تَعمَلُونَ بَصِيرٌۭ ﴿٤﴾\\
\textamh{5.\  } & لَّهُۥ مُلكُ ٱلسَّمَـٰوَٟتِ وَٱلأَرضِ ۚ وَإِلَى ٱللَّهِ تُرجَعُ ٱلأُمُورُ ﴿٥﴾\\
\textamh{6.\  } & يُولِجُ ٱلَّيلَ فِى ٱلنَّهَارِ وَيُولِجُ ٱلنَّهَارَ فِى ٱلَّيلِ ۚ وَهُوَ عَلِيمٌۢ بِذَاتِ ٱلصُّدُورِ ﴿٦﴾\\
\textamh{7.\  } & ءَامِنُوا۟ بِٱللَّهِ وَرَسُولِهِۦ وَأَنفِقُوا۟ مِمَّا جَعَلَكُم مُّستَخلَفِينَ فِيهِ ۖ فَٱلَّذِينَ ءَامَنُوا۟ مِنكُم وَأَنفَقُوا۟ لَهُم أَجرٌۭ كَبِيرٌۭ ﴿٧﴾\\
\textamh{8.\  } & وَمَا لَكُم لَا تُؤمِنُونَ بِٱللَّهِ ۙ وَٱلرَّسُولُ يَدعُوكُم لِتُؤمِنُوا۟ بِرَبِّكُم وَقَد أَخَذَ مِيثَـٰقَكُم إِن كُنتُم مُّؤمِنِينَ ﴿٨﴾\\
\textamh{9.\  } & هُوَ ٱلَّذِى يُنَزِّلُ عَلَىٰ عَبدِهِۦٓ ءَايَـٰتٍۭ بَيِّنَـٰتٍۢ لِّيُخرِجَكُم مِّنَ ٱلظُّلُمَـٰتِ إِلَى ٱلنُّورِ ۚ وَإِنَّ ٱللَّهَ بِكُم لَرَءُوفٌۭ رَّحِيمٌۭ ﴿٩﴾\\
\textamh{10.\  } & وَمَا لَكُم أَلَّا تُنفِقُوا۟ فِى سَبِيلِ ٱللَّهِ وَلِلَّهِ مِيرَٰثُ ٱلسَّمَـٰوَٟتِ وَٱلأَرضِ ۚ لَا يَستَوِى مِنكُم مَّن أَنفَقَ مِن قَبلِ ٱلفَتحِ وَقَـٰتَلَ ۚ أُو۟لَـٰٓئِكَ أَعظَمُ دَرَجَةًۭ مِّنَ ٱلَّذِينَ أَنفَقُوا۟ مِنۢ بَعدُ وَقَـٰتَلُوا۟ ۚ وَكُلًّۭا وَعَدَ ٱللَّهُ ٱلحُسنَىٰ ۚ وَٱللَّهُ بِمَا تَعمَلُونَ خَبِيرٌۭ ﴿١٠﴾\\
\textamh{11.\  } & مَّن ذَا ٱلَّذِى يُقرِضُ ٱللَّهَ قَرضًا حَسَنًۭا فَيُضَٰعِفَهُۥ لَهُۥ وَلَهُۥٓ أَجرٌۭ كَرِيمٌۭ ﴿١١﴾\\
\textamh{12.\  } & يَومَ تَرَى ٱلمُؤمِنِينَ وَٱلمُؤمِنَـٰتِ يَسعَىٰ نُورُهُم بَينَ أَيدِيهِم وَبِأَيمَـٰنِهِم بُشرَىٰكُمُ ٱليَومَ جَنَّـٰتٌۭ تَجرِى مِن تَحتِهَا ٱلأَنهَـٰرُ خَـٰلِدِينَ فِيهَا ۚ ذَٟلِكَ هُوَ ٱلفَوزُ ٱلعَظِيمُ ﴿١٢﴾\\
\textamh{13.\  } & يَومَ يَقُولُ ٱلمُنَـٰفِقُونَ وَٱلمُنَـٰفِقَـٰتُ لِلَّذِينَ ءَامَنُوا۟ ٱنظُرُونَا نَقتَبِس مِن نُّورِكُم قِيلَ ٱرجِعُوا۟ وَرَآءَكُم فَٱلتَمِسُوا۟ نُورًۭا فَضُرِبَ بَينَهُم بِسُورٍۢ لَّهُۥ بَابٌۢ بَاطِنُهُۥ فِيهِ ٱلرَّحمَةُ وَظَـٰهِرُهُۥ مِن قِبَلِهِ ٱلعَذَابُ ﴿١٣﴾\\
\textamh{14.\  } & يُنَادُونَهُم أَلَم نَكُن مَّعَكُم ۖ قَالُوا۟ بَلَىٰ وَلَـٰكِنَّكُم فَتَنتُم أَنفُسَكُم وَتَرَبَّصتُم وَٱرتَبتُم وَغَرَّتكُمُ ٱلأَمَانِىُّ حَتَّىٰ جَآءَ أَمرُ ٱللَّهِ وَغَرَّكُم بِٱللَّهِ ٱلغَرُورُ ﴿١٤﴾\\
\textamh{15.\  } & فَٱليَومَ لَا يُؤخَذُ مِنكُم فِديَةٌۭ وَلَا مِنَ ٱلَّذِينَ كَفَرُوا۟ ۚ مَأوَىٰكُمُ ٱلنَّارُ ۖ هِىَ مَولَىٰكُم ۖ وَبِئسَ ٱلمَصِيرُ ﴿١٥﴾\\
\textamh{16.\  } & ۞ أَلَم يَأنِ لِلَّذِينَ ءَامَنُوٓا۟ أَن تَخشَعَ قُلُوبُهُم لِذِكرِ ٱللَّهِ وَمَا نَزَلَ مِنَ ٱلحَقِّ وَلَا يَكُونُوا۟ كَٱلَّذِينَ أُوتُوا۟ ٱلكِتَـٰبَ مِن قَبلُ فَطَالَ عَلَيهِمُ ٱلأَمَدُ فَقَسَت قُلُوبُهُم ۖ وَكَثِيرٌۭ مِّنهُم فَـٰسِقُونَ ﴿١٦﴾\\
\textamh{17.\  } & ٱعلَمُوٓا۟ أَنَّ ٱللَّهَ يُحىِ ٱلأَرضَ بَعدَ مَوتِهَا ۚ قَد بَيَّنَّا لَكُمُ ٱلءَايَـٰتِ لَعَلَّكُم تَعقِلُونَ ﴿١٧﴾\\
\textamh{18.\  } & إِنَّ ٱلمُصَّدِّقِينَ وَٱلمُصَّدِّقَـٰتِ وَأَقرَضُوا۟ ٱللَّهَ قَرضًا حَسَنًۭا يُضَٰعَفُ لَهُم وَلَهُم أَجرٌۭ كَرِيمٌۭ ﴿١٨﴾\\
\textamh{19.\  } & وَٱلَّذِينَ ءَامَنُوا۟ بِٱللَّهِ وَرُسُلِهِۦٓ أُو۟لَـٰٓئِكَ هُمُ ٱلصِّدِّيقُونَ ۖ وَٱلشُّهَدَآءُ عِندَ رَبِّهِم لَهُم أَجرُهُم وَنُورُهُم ۖ وَٱلَّذِينَ كَفَرُوا۟ وَكَذَّبُوا۟ بِـَٔايَـٰتِنَآ أُو۟لَـٰٓئِكَ أَصحَـٰبُ ٱلجَحِيمِ ﴿١٩﴾\\
\textamh{20.\  } & ٱعلَمُوٓا۟ أَنَّمَا ٱلحَيَوٰةُ ٱلدُّنيَا لَعِبٌۭ وَلَهوٌۭ وَزِينَةٌۭ وَتَفَاخُرٌۢ بَينَكُم وَتَكَاثُرٌۭ فِى ٱلأَموَٟلِ وَٱلأَولَـٰدِ ۖ كَمَثَلِ غَيثٍ أَعجَبَ ٱلكُفَّارَ نَبَاتُهُۥ ثُمَّ يَهِيجُ فَتَرَىٰهُ مُصفَرًّۭا ثُمَّ يَكُونُ حُطَٰمًۭا ۖ وَفِى ٱلءَاخِرَةِ عَذَابٌۭ شَدِيدٌۭ وَمَغفِرَةٌۭ مِّنَ ٱللَّهِ وَرِضوَٟنٌۭ ۚ وَمَا ٱلحَيَوٰةُ ٱلدُّنيَآ إِلَّا مَتَـٰعُ ٱلغُرُورِ ﴿٢٠﴾\\
\textamh{21.\  } & سَابِقُوٓا۟ إِلَىٰ مَغفِرَةٍۢ مِّن رَّبِّكُم وَجَنَّةٍ عَرضُهَا كَعَرضِ ٱلسَّمَآءِ وَٱلأَرضِ أُعِدَّت لِلَّذِينَ ءَامَنُوا۟ بِٱللَّهِ وَرُسُلِهِۦ ۚ ذَٟلِكَ فَضلُ ٱللَّهِ يُؤتِيهِ مَن يَشَآءُ ۚ وَٱللَّهُ ذُو ٱلفَضلِ ٱلعَظِيمِ ﴿٢١﴾\\
\textamh{22.\  } & مَآ أَصَابَ مِن مُّصِيبَةٍۢ فِى ٱلأَرضِ وَلَا فِىٓ أَنفُسِكُم إِلَّا فِى كِتَـٰبٍۢ مِّن قَبلِ أَن نَّبرَأَهَآ ۚ إِنَّ ذَٟلِكَ عَلَى ٱللَّهِ يَسِيرٌۭ ﴿٢٢﴾\\
\textamh{23.\  } & لِّكَيلَا تَأسَوا۟ عَلَىٰ مَا فَاتَكُم وَلَا تَفرَحُوا۟ بِمَآ ءَاتَىٰكُم ۗ وَٱللَّهُ لَا يُحِبُّ كُلَّ مُختَالٍۢ فَخُورٍ ﴿٢٣﴾\\
\textamh{24.\  } & ٱلَّذِينَ يَبخَلُونَ وَيَأمُرُونَ ٱلنَّاسَ بِٱلبُخلِ ۗ وَمَن يَتَوَلَّ فَإِنَّ ٱللَّهَ هُوَ ٱلغَنِىُّ ٱلحَمِيدُ ﴿٢٤﴾\\
\textamh{25.\  } & لَقَد أَرسَلنَا رُسُلَنَا بِٱلبَيِّنَـٰتِ وَأَنزَلنَا مَعَهُمُ ٱلكِتَـٰبَ وَٱلمِيزَانَ لِيَقُومَ ٱلنَّاسُ بِٱلقِسطِ ۖ وَأَنزَلنَا ٱلحَدِيدَ فِيهِ بَأسٌۭ شَدِيدٌۭ وَمَنَـٰفِعُ لِلنَّاسِ وَلِيَعلَمَ ٱللَّهُ مَن يَنصُرُهُۥ وَرُسُلَهُۥ بِٱلغَيبِ ۚ إِنَّ ٱللَّهَ قَوِىٌّ عَزِيزٌۭ ﴿٢٥﴾\\
\textamh{26.\  } & وَلَقَد أَرسَلنَا نُوحًۭا وَإِبرَٰهِيمَ وَجَعَلنَا فِى ذُرِّيَّتِهِمَا ٱلنُّبُوَّةَ وَٱلكِتَـٰبَ ۖ فَمِنهُم مُّهتَدٍۢ ۖ وَكَثِيرٌۭ مِّنهُم فَـٰسِقُونَ ﴿٢٦﴾\\
\textamh{27.\  } & ثُمَّ قَفَّينَا عَلَىٰٓ ءَاثَـٰرِهِم بِرُسُلِنَا وَقَفَّينَا بِعِيسَى ٱبنِ مَريَمَ وَءَاتَينَـٰهُ ٱلإِنجِيلَ وَجَعَلنَا فِى قُلُوبِ ٱلَّذِينَ ٱتَّبَعُوهُ رَأفَةًۭ وَرَحمَةًۭ وَرَهبَانِيَّةً ٱبتَدَعُوهَا مَا كَتَبنَـٰهَا عَلَيهِم إِلَّا ٱبتِغَآءَ رِضوَٟنِ ٱللَّهِ فَمَا رَعَوهَا حَقَّ رِعَايَتِهَا ۖ فَـَٔاتَينَا ٱلَّذِينَ ءَامَنُوا۟ مِنهُم أَجرَهُم ۖ وَكَثِيرٌۭ مِّنهُم فَـٰسِقُونَ ﴿٢٧﴾\\
\textamh{28.\  } & يَـٰٓأَيُّهَا ٱلَّذِينَ ءَامَنُوا۟ ٱتَّقُوا۟ ٱللَّهَ وَءَامِنُوا۟ بِرَسُولِهِۦ يُؤتِكُم كِفلَينِ مِن رَّحمَتِهِۦ وَيَجعَل لَّكُم نُورًۭا تَمشُونَ بِهِۦ وَيَغفِر لَكُم ۚ وَٱللَّهُ غَفُورٌۭ رَّحِيمٌۭ ﴿٢٨﴾\\
\textamh{29.\  } & لِّئَلَّا يَعلَمَ أَهلُ ٱلكِتَـٰبِ أَلَّا يَقدِرُونَ عَلَىٰ شَىءٍۢ مِّن فَضلِ ٱللَّهِ ۙ وَأَنَّ ٱلفَضلَ بِيَدِ ٱللَّهِ يُؤتِيهِ مَن يَشَآءُ ۚ وَٱللَّهُ ذُو ٱلفَضلِ ٱلعَظِيمِ ﴿٢٩﴾\\
\end{longtable} \newpage
