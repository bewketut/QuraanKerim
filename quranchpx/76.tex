%% License: BSD style (Berkley) (i.e. Put the Copyright owner's name always)
%% Writer and Copyright (to): Bewketu(Bilal) Tadilo (2016-17)
\shadowbox{\section{\LR{\textamharic{ሱራቱ አልኢንሳን -}  \RL{سوره  الانسان}}}}
\begin{longtable}{%
  @{}
    p{.5\textwidth}
  @{~~~~~~~~~~~~~}||
    p{.5\textwidth}
    @{}
}
\nopagebreak
\textamh{\ \ \ \ \ \  ቢስሚላሂ አራህመኒ ራሂይም } &  بِسمِ ٱللَّهِ ٱلرَّحمَـٰنِ ٱلرَّحِيمِ\\
\textamh{1.\  } &  هَل أَتَىٰ عَلَى ٱلإِنسَـٰنِ حِينٌۭ مِّنَ ٱلدَّهرِ لَم يَكُن شَيـًۭٔا مَّذكُورًا ﴿١﴾\\
\textamh{2.\  } & إِنَّا خَلَقنَا ٱلإِنسَـٰنَ مِن نُّطفَةٍ أَمشَاجٍۢ نَّبتَلِيهِ فَجَعَلنَـٰهُ سَمِيعًۢا بَصِيرًا ﴿٢﴾\\
\textamh{3.\  } & إِنَّا هَدَينَـٰهُ ٱلسَّبِيلَ إِمَّا شَاكِرًۭا وَإِمَّا كَفُورًا ﴿٣﴾\\
\textamh{4.\  } & إِنَّآ أَعتَدنَا لِلكَـٰفِرِينَ سَلَـٰسِلَا۟ وَأَغلَـٰلًۭا وَسَعِيرًا ﴿٤﴾\\
\textamh{5.\  } & إِنَّ ٱلأَبرَارَ يَشرَبُونَ مِن كَأسٍۢ كَانَ مِزَاجُهَا كَافُورًا ﴿٥﴾\\
\textamh{6.\  } & عَينًۭا يَشرَبُ بِهَا عِبَادُ ٱللَّهِ يُفَجِّرُونَهَا تَفجِيرًۭا ﴿٦﴾\\
\textamh{7.\  } & يُوفُونَ بِٱلنَّذرِ وَيَخَافُونَ يَومًۭا كَانَ شَرُّهُۥ مُستَطِيرًۭا ﴿٧﴾\\
\textamh{8.\  } & وَيُطعِمُونَ ٱلطَّعَامَ عَلَىٰ حُبِّهِۦ مِسكِينًۭا وَيَتِيمًۭا وَأَسِيرًا ﴿٨﴾\\
\textamh{9.\  } & إِنَّمَا نُطعِمُكُم لِوَجهِ ٱللَّهِ لَا نُرِيدُ مِنكُم جَزَآءًۭ وَلَا شُكُورًا ﴿٩﴾\\
\textamh{10.\  } & إِنَّا نَخَافُ مِن رَّبِّنَا يَومًا عَبُوسًۭا قَمطَرِيرًۭا ﴿١٠﴾\\
\textamh{11.\  } & فَوَقَىٰهُمُ ٱللَّهُ شَرَّ ذَٟلِكَ ٱليَومِ وَلَقَّىٰهُم نَضرَةًۭ وَسُرُورًۭا ﴿١١﴾\\
\textamh{12.\  } & وَجَزَىٰهُم بِمَا صَبَرُوا۟ جَنَّةًۭ وَحَرِيرًۭا ﴿١٢﴾\\
\textamh{13.\  } & مُّتَّكِـِٔينَ فِيهَا عَلَى ٱلأَرَآئِكِ ۖ لَا يَرَونَ فِيهَا شَمسًۭا وَلَا زَمهَرِيرًۭا ﴿١٣﴾\\
\textamh{14.\  } & وَدَانِيَةً عَلَيهِم ظِلَـٰلُهَا وَذُلِّلَت قُطُوفُهَا تَذلِيلًۭا ﴿١٤﴾\\
\textamh{15.\  } & وَيُطَافُ عَلَيهِم بِـَٔانِيَةٍۢ مِّن فِضَّةٍۢ وَأَكوَابٍۢ كَانَت قَوَارِيرَا۠ ﴿١٥﴾\\
\textamh{16.\  } & قَوَارِيرَا۟ مِن فِضَّةٍۢ قَدَّرُوهَا تَقدِيرًۭا ﴿١٦﴾\\
\textamh{17.\  } & وَيُسقَونَ فِيهَا كَأسًۭا كَانَ مِزَاجُهَا زَنجَبِيلًا ﴿١٧﴾\\
\textamh{18.\  } & عَينًۭا فِيهَا تُسَمَّىٰ سَلسَبِيلًۭا ﴿١٨﴾\\
\textamh{19.\  } & ۞ وَيَطُوفُ عَلَيهِم وِلدَٟنٌۭ مُّخَلَّدُونَ إِذَا رَأَيتَهُم حَسِبتَهُم لُؤلُؤًۭا مَّنثُورًۭا ﴿١٩﴾\\
\textamh{20.\  } & وَإِذَا رَأَيتَ ثَمَّ رَأَيتَ نَعِيمًۭا وَمُلكًۭا كَبِيرًا ﴿٢٠﴾\\
\textamh{21.\  } & عَـٰلِيَهُم ثِيَابُ سُندُسٍ خُضرٌۭ وَإِستَبرَقٌۭ ۖ وَحُلُّوٓا۟ أَسَاوِرَ مِن فِضَّةٍۢ وَسَقَىٰهُم رَبُّهُم شَرَابًۭا طَهُورًا ﴿٢١﴾\\
\textamh{22.\  } & إِنَّ هَـٰذَا كَانَ لَكُم جَزَآءًۭ وَكَانَ سَعيُكُم مَّشكُورًا ﴿٢٢﴾\\
\textamh{23.\  } & إِنَّا نَحنُ نَزَّلنَا عَلَيكَ ٱلقُرءَانَ تَنزِيلًۭا ﴿٢٣﴾\\
\textamh{24.\  } & فَٱصبِر لِحُكمِ رَبِّكَ وَلَا تُطِع مِنهُم ءَاثِمًا أَو كَفُورًۭا ﴿٢٤﴾\\
\textamh{25.\  } & وَٱذكُرِ ٱسمَ رَبِّكَ بُكرَةًۭ وَأَصِيلًۭا ﴿٢٥﴾\\
\textamh{26.\  } & وَمِنَ ٱلَّيلِ فَٱسجُد لَهُۥ وَسَبِّحهُ لَيلًۭا طَوِيلًا ﴿٢٦﴾\\
\textamh{27.\  } & إِنَّ هَـٰٓؤُلَآءِ يُحِبُّونَ ٱلعَاجِلَةَ وَيَذَرُونَ وَرَآءَهُم يَومًۭا ثَقِيلًۭا ﴿٢٧﴾\\
\textamh{28.\  } & نَّحنُ خَلَقنَـٰهُم وَشَدَدنَآ أَسرَهُم ۖ وَإِذَا شِئنَا بَدَّلنَآ أَمثَـٰلَهُم تَبدِيلًا ﴿٢٨﴾\\
\textamh{29.\  } & إِنَّ هَـٰذِهِۦ تَذكِرَةٌۭ ۖ فَمَن شَآءَ ٱتَّخَذَ إِلَىٰ رَبِّهِۦ سَبِيلًۭا ﴿٢٩﴾\\
\textamh{30.\  } & وَمَا تَشَآءُونَ إِلَّآ أَن يَشَآءَ ٱللَّهُ ۚ إِنَّ ٱللَّهَ كَانَ عَلِيمًا حَكِيمًۭا ﴿٣٠﴾\\
\textamh{31.\  } & يُدخِلُ مَن يَشَآءُ فِى رَحمَتِهِۦ ۚ وَٱلظَّـٰلِمِينَ أَعَدَّ لَهُم عَذَابًا أَلِيمًۢا ﴿٣١﴾\\
\end{longtable} \newpage
