%% License: BSD style (Berkley) (i.e. Put the Copyright owner's name always)
%% Writer and Copyright (to): Bewketu(Bilal) Tadilo (2016-17)
\shadowbox{\section{\LR{\textamharic{ሱራቱ አልኢስራኣ -}  \RL{سوره  الإسراء}}}}
\begin{longtable}{%
  @{}
    p{.5\textwidth}
  @{~~~~~~~~~~~~~}||
    p{.5\textwidth}
    @{}
}
\nopagebreak
\textamh{\ \ \ \ \ \  ቢስሚላሂ አራህመኒ ራሂይም } &  بِسمِ ٱللَّهِ ٱلرَّحمَـٰنِ ٱلرَّحِيمِ\\
\textamh{1.\  } &  سُبحَـٰنَ ٱلَّذِىٓ أَسرَىٰ بِعَبدِهِۦ لَيلًۭا مِّنَ ٱلمَسجِدِ ٱلحَرَامِ إِلَى ٱلمَسجِدِ ٱلأَقصَا ٱلَّذِى بَٰرَكنَا حَولَهُۥ لِنُرِيَهُۥ مِن ءَايَـٰتِنَآ ۚ إِنَّهُۥ هُوَ ٱلسَّمِيعُ ٱلبَصِيرُ ﴿١﴾\\
\textamh{2.\  } & وَءَاتَينَا مُوسَى ٱلكِتَـٰبَ وَجَعَلنَـٰهُ هُدًۭى لِّبَنِىٓ إِسرَٰٓءِيلَ أَلَّا تَتَّخِذُوا۟ مِن دُونِى وَكِيلًۭا ﴿٢﴾\\
\textamh{3.\  } & ذُرِّيَّةَ مَن حَمَلنَا مَعَ نُوحٍ ۚ إِنَّهُۥ كَانَ عَبدًۭا شَكُورًۭا ﴿٣﴾\\
\textamh{4.\  } & وَقَضَينَآ إِلَىٰ بَنِىٓ إِسرَٰٓءِيلَ فِى ٱلكِتَـٰبِ لَتُفسِدُنَّ فِى ٱلأَرضِ مَرَّتَينِ وَلَتَعلُنَّ عُلُوًّۭا كَبِيرًۭا ﴿٤﴾\\
\textamh{5.\  } & فَإِذَا جَآءَ وَعدُ أُولَىٰهُمَا بَعَثنَا عَلَيكُم عِبَادًۭا لَّنَآ أُو۟لِى بَأسٍۢ شَدِيدٍۢ فَجَاسُوا۟ خِلَـٰلَ ٱلدِّيَارِ ۚ وَكَانَ وَعدًۭا مَّفعُولًۭا ﴿٥﴾\\
\textamh{6.\  } & ثُمَّ رَدَدنَا لَكُمُ ٱلكَرَّةَ عَلَيهِم وَأَمدَدنَـٰكُم بِأَموَٟلٍۢ وَبَنِينَ وَجَعَلنَـٰكُم أَكثَرَ نَفِيرًا ﴿٦﴾\\
\textamh{7.\  } & إِن أَحسَنتُم أَحسَنتُم لِأَنفُسِكُم ۖ وَإِن أَسَأتُم فَلَهَا ۚ فَإِذَا جَآءَ وَعدُ ٱلءَاخِرَةِ لِيَسُۥٓـُٔوا۟ وُجُوهَكُم وَلِيَدخُلُوا۟ ٱلمَسجِدَ كَمَا دَخَلُوهُ أَوَّلَ مَرَّةٍۢ وَلِيُتَبِّرُوا۟ مَا عَلَوا۟ تَتبِيرًا ﴿٧﴾\\
\textamh{8.\  } & عَسَىٰ رَبُّكُم أَن يَرحَمَكُم ۚ وَإِن عُدتُّم عُدنَا ۘ وَجَعَلنَا جَهَنَّمَ لِلكَـٰفِرِينَ حَصِيرًا ﴿٨﴾\\
\textamh{9.\  } & إِنَّ هَـٰذَا ٱلقُرءَانَ يَهدِى لِلَّتِى هِىَ أَقوَمُ وَيُبَشِّرُ ٱلمُؤمِنِينَ ٱلَّذِينَ يَعمَلُونَ ٱلصَّـٰلِحَـٰتِ أَنَّ لَهُم أَجرًۭا كَبِيرًۭا ﴿٩﴾\\
\textamh{10.\  } & وَأَنَّ ٱلَّذِينَ لَا يُؤمِنُونَ بِٱلءَاخِرَةِ أَعتَدنَا لَهُم عَذَابًا أَلِيمًۭا ﴿١٠﴾\\
\textamh{11.\  } & وَيَدعُ ٱلإِنسَـٰنُ بِٱلشَّرِّ دُعَآءَهُۥ بِٱلخَيرِ ۖ وَكَانَ ٱلإِنسَـٰنُ عَجُولًۭا ﴿١١﴾\\
\textamh{12.\  } & وَجَعَلنَا ٱلَّيلَ وَٱلنَّهَارَ ءَايَتَينِ ۖ فَمَحَونَآ ءَايَةَ ٱلَّيلِ وَجَعَلنَآ ءَايَةَ ٱلنَّهَارِ مُبصِرَةًۭ لِّتَبتَغُوا۟ فَضلًۭا مِّن رَّبِّكُم وَلِتَعلَمُوا۟ عَدَدَ ٱلسِّنِينَ وَٱلحِسَابَ ۚ وَكُلَّ شَىءٍۢ فَصَّلنَـٰهُ تَفصِيلًۭا ﴿١٢﴾\\
\textamh{13.\  } & وَكُلَّ إِنسَـٰنٍ أَلزَمنَـٰهُ طَٰٓئِرَهُۥ فِى عُنُقِهِۦ ۖ وَنُخرِجُ لَهُۥ يَومَ ٱلقِيَـٰمَةِ كِتَـٰبًۭا يَلقَىٰهُ مَنشُورًا ﴿١٣﴾\\
\textamh{14.\  } & ٱقرَأ كِتَـٰبَكَ كَفَىٰ بِنَفسِكَ ٱليَومَ عَلَيكَ حَسِيبًۭا ﴿١٤﴾\\
\textamh{15.\  } & مَّنِ ٱهتَدَىٰ فَإِنَّمَا يَهتَدِى لِنَفسِهِۦ ۖ وَمَن ضَلَّ فَإِنَّمَا يَضِلُّ عَلَيهَا ۚ وَلَا تَزِرُ وَازِرَةٌۭ وِزرَ أُخرَىٰ ۗ وَمَا كُنَّا مُعَذِّبِينَ حَتَّىٰ نَبعَثَ رَسُولًۭا ﴿١٥﴾\\
\textamh{16.\  } & وَإِذَآ أَرَدنَآ أَن نُّهلِكَ قَريَةً أَمَرنَا مُترَفِيهَا فَفَسَقُوا۟ فِيهَا فَحَقَّ عَلَيهَا ٱلقَولُ فَدَمَّرنَـٰهَا تَدمِيرًۭا ﴿١٦﴾\\
\textamh{17.\  } & وَكَم أَهلَكنَا مِنَ ٱلقُرُونِ مِنۢ بَعدِ نُوحٍۢ ۗ وَكَفَىٰ بِرَبِّكَ بِذُنُوبِ عِبَادِهِۦ خَبِيرًۢا بَصِيرًۭا ﴿١٧﴾\\
\textamh{18.\  } & مَّن كَانَ يُرِيدُ ٱلعَاجِلَةَ عَجَّلنَا لَهُۥ فِيهَا مَا نَشَآءُ لِمَن نُّرِيدُ ثُمَّ جَعَلنَا لَهُۥ جَهَنَّمَ يَصلَىٰهَا مَذمُومًۭا مَّدحُورًۭا ﴿١٨﴾\\
\textamh{19.\  } & وَمَن أَرَادَ ٱلءَاخِرَةَ وَسَعَىٰ لَهَا سَعيَهَا وَهُوَ مُؤمِنٌۭ فَأُو۟لَـٰٓئِكَ كَانَ سَعيُهُم مَّشكُورًۭا ﴿١٩﴾\\
\textamh{20.\  } & كُلًّۭا نُّمِدُّ هَـٰٓؤُلَآءِ وَهَـٰٓؤُلَآءِ مِن عَطَآءِ رَبِّكَ ۚ وَمَا كَانَ عَطَآءُ رَبِّكَ مَحظُورًا ﴿٢٠﴾\\
\textamh{21.\  } & ٱنظُر كَيفَ فَضَّلنَا بَعضَهُم عَلَىٰ بَعضٍۢ ۚ وَلَلءَاخِرَةُ أَكبَرُ دَرَجَٰتٍۢ وَأَكبَرُ تَفضِيلًۭا ﴿٢١﴾\\
\textamh{22.\  } & لَّا تَجعَل مَعَ ٱللَّهِ إِلَـٰهًا ءَاخَرَ فَتَقعُدَ مَذمُومًۭا مَّخذُولًۭا ﴿٢٢﴾\\
\textamh{23.\  } & ۞ وَقَضَىٰ رَبُّكَ أَلَّا تَعبُدُوٓا۟ إِلَّآ إِيَّاهُ وَبِٱلوَٟلِدَينِ إِحسَـٰنًا ۚ إِمَّا يَبلُغَنَّ عِندَكَ ٱلكِبَرَ أَحَدُهُمَآ أَو كِلَاهُمَا فَلَا تَقُل لَّهُمَآ أُفٍّۢ وَلَا تَنهَرهُمَا وَقُل لَّهُمَا قَولًۭا كَرِيمًۭا ﴿٢٣﴾\\
\textamh{24.\  } & وَٱخفِض لَهُمَا جَنَاحَ ٱلذُّلِّ مِنَ ٱلرَّحمَةِ وَقُل رَّبِّ ٱرحَمهُمَا كَمَا رَبَّيَانِى صَغِيرًۭا ﴿٢٤﴾\\
\textamh{25.\  } & رَّبُّكُم أَعلَمُ بِمَا فِى نُفُوسِكُم ۚ إِن تَكُونُوا۟ صَـٰلِحِينَ فَإِنَّهُۥ كَانَ لِلأَوَّٰبِينَ غَفُورًۭا ﴿٢٥﴾\\
\textamh{26.\  } & وَءَاتِ ذَا ٱلقُربَىٰ حَقَّهُۥ وَٱلمِسكِينَ وَٱبنَ ٱلسَّبِيلِ وَلَا تُبَذِّر تَبذِيرًا ﴿٢٦﴾\\
\textamh{27.\  } & إِنَّ ٱلمُبَذِّرِينَ كَانُوٓا۟ إِخوَٟنَ ٱلشَّيَـٰطِينِ ۖ وَكَانَ ٱلشَّيطَٰنُ لِرَبِّهِۦ كَفُورًۭا ﴿٢٧﴾\\
\textamh{28.\  } & وَإِمَّا تُعرِضَنَّ عَنهُمُ ٱبتِغَآءَ رَحمَةٍۢ مِّن رَّبِّكَ تَرجُوهَا فَقُل لَّهُم قَولًۭا مَّيسُورًۭا ﴿٢٨﴾\\
\textamh{29.\  } & وَلَا تَجعَل يَدَكَ مَغلُولَةً إِلَىٰ عُنُقِكَ وَلَا تَبسُطهَا كُلَّ ٱلبَسطِ فَتَقعُدَ مَلُومًۭا مَّحسُورًا ﴿٢٩﴾\\
\textamh{30.\  } & إِنَّ رَبَّكَ يَبسُطُ ٱلرِّزقَ لِمَن يَشَآءُ وَيَقدِرُ ۚ إِنَّهُۥ كَانَ بِعِبَادِهِۦ خَبِيرًۢا بَصِيرًۭا ﴿٣٠﴾\\
\textamh{31.\  } & وَلَا تَقتُلُوٓا۟ أَولَـٰدَكُم خَشيَةَ إِملَـٰقٍۢ ۖ نَّحنُ نَرزُقُهُم وَإِيَّاكُم ۚ إِنَّ قَتلَهُم كَانَ خِطـًۭٔا كَبِيرًۭا ﴿٣١﴾\\
\textamh{32.\  } & وَلَا تَقرَبُوا۟ ٱلزِّنَىٰٓ ۖ إِنَّهُۥ كَانَ فَـٰحِشَةًۭ وَسَآءَ سَبِيلًۭا ﴿٣٢﴾\\
\textamh{33.\  } & وَلَا تَقتُلُوا۟ ٱلنَّفسَ ٱلَّتِى حَرَّمَ ٱللَّهُ إِلَّا بِٱلحَقِّ ۗ وَمَن قُتِلَ مَظلُومًۭا فَقَد جَعَلنَا لِوَلِيِّهِۦ سُلطَٰنًۭا فَلَا يُسرِف فِّى ٱلقَتلِ ۖ إِنَّهُۥ كَانَ مَنصُورًۭا ﴿٣٣﴾\\
\textamh{34.\  } & وَلَا تَقرَبُوا۟ مَالَ ٱليَتِيمِ إِلَّا بِٱلَّتِى هِىَ أَحسَنُ حَتَّىٰ يَبلُغَ أَشُدَّهُۥ ۚ وَأَوفُوا۟ بِٱلعَهدِ ۖ إِنَّ ٱلعَهدَ كَانَ مَسـُٔولًۭا ﴿٣٤﴾\\
\textamh{35.\  } & وَأَوفُوا۟ ٱلكَيلَ إِذَا كِلتُم وَزِنُوا۟ بِٱلقِسطَاسِ ٱلمُستَقِيمِ ۚ ذَٟلِكَ خَيرٌۭ وَأَحسَنُ تَأوِيلًۭا ﴿٣٥﴾\\
\textamh{36.\  } & وَلَا تَقفُ مَا لَيسَ لَكَ بِهِۦ عِلمٌ ۚ إِنَّ ٱلسَّمعَ وَٱلبَصَرَ وَٱلفُؤَادَ كُلُّ أُو۟لَـٰٓئِكَ كَانَ عَنهُ مَسـُٔولًۭا ﴿٣٦﴾\\
\textamh{37.\  } & وَلَا تَمشِ فِى ٱلأَرضِ مَرَحًا ۖ إِنَّكَ لَن تَخرِقَ ٱلأَرضَ وَلَن تَبلُغَ ٱلجِبَالَ طُولًۭا ﴿٣٧﴾\\
\textamh{38.\  } & كُلُّ ذَٟلِكَ كَانَ سَيِّئُهُۥ عِندَ رَبِّكَ مَكرُوهًۭا ﴿٣٨﴾\\
\textamh{39.\  } & ذَٟلِكَ مِمَّآ أَوحَىٰٓ إِلَيكَ رَبُّكَ مِنَ ٱلحِكمَةِ ۗ وَلَا تَجعَل مَعَ ٱللَّهِ إِلَـٰهًا ءَاخَرَ فَتُلقَىٰ فِى جَهَنَّمَ مَلُومًۭا مَّدحُورًا ﴿٣٩﴾\\
\textamh{40.\  } & أَفَأَصفَىٰكُم رَبُّكُم بِٱلبَنِينَ وَٱتَّخَذَ مِنَ ٱلمَلَـٰٓئِكَةِ إِنَـٰثًا ۚ إِنَّكُم لَتَقُولُونَ قَولًا عَظِيمًۭا ﴿٤٠﴾\\
\textamh{41.\  } & وَلَقَد صَرَّفنَا فِى هَـٰذَا ٱلقُرءَانِ لِيَذَّكَّرُوا۟ وَمَا يَزِيدُهُم إِلَّا نُفُورًۭا ﴿٤١﴾\\
\textamh{42.\  } & قُل لَّو كَانَ مَعَهُۥٓ ءَالِهَةٌۭ كَمَا يَقُولُونَ إِذًۭا لَّٱبتَغَوا۟ إِلَىٰ ذِى ٱلعَرشِ سَبِيلًۭا ﴿٤٢﴾\\
\textamh{43.\  } & سُبحَـٰنَهُۥ وَتَعَـٰلَىٰ عَمَّا يَقُولُونَ عُلُوًّۭا كَبِيرًۭا ﴿٤٣﴾\\
\textamh{44.\  } & تُسَبِّحُ لَهُ ٱلسَّمَـٰوَٟتُ ٱلسَّبعُ وَٱلأَرضُ وَمَن فِيهِنَّ ۚ وَإِن مِّن شَىءٍ إِلَّا يُسَبِّحُ بِحَمدِهِۦ وَلَـٰكِن لَّا تَفقَهُونَ تَسبِيحَهُم ۗ إِنَّهُۥ كَانَ حَلِيمًا غَفُورًۭا ﴿٤٤﴾\\
\textamh{45.\  } & وَإِذَا قَرَأتَ ٱلقُرءَانَ جَعَلنَا بَينَكَ وَبَينَ ٱلَّذِينَ لَا يُؤمِنُونَ بِٱلءَاخِرَةِ حِجَابًۭا مَّستُورًۭا ﴿٤٥﴾\\
\textamh{46.\  } & وَجَعَلنَا عَلَىٰ قُلُوبِهِم أَكِنَّةً أَن يَفقَهُوهُ وَفِىٓ ءَاذَانِهِم وَقرًۭا ۚ وَإِذَا ذَكَرتَ رَبَّكَ فِى ٱلقُرءَانِ وَحدَهُۥ وَلَّوا۟ عَلَىٰٓ أَدبَٰرِهِم نُفُورًۭا ﴿٤٦﴾\\
\textamh{47.\  } & نَّحنُ أَعلَمُ بِمَا يَستَمِعُونَ بِهِۦٓ إِذ يَستَمِعُونَ إِلَيكَ وَإِذ هُم نَجوَىٰٓ إِذ يَقُولُ ٱلظَّـٰلِمُونَ إِن تَتَّبِعُونَ إِلَّا رَجُلًۭا مَّسحُورًا ﴿٤٧﴾\\
\textamh{48.\  } & ٱنظُر كَيفَ ضَرَبُوا۟ لَكَ ٱلأَمثَالَ فَضَلُّوا۟ فَلَا يَستَطِيعُونَ سَبِيلًۭا ﴿٤٨﴾\\
\textamh{49.\  } & وَقَالُوٓا۟ أَءِذَا كُنَّا عِظَـٰمًۭا وَرُفَـٰتًا أَءِنَّا لَمَبعُوثُونَ خَلقًۭا جَدِيدًۭا ﴿٤٩﴾\\
\textamh{50.\  } & ۞ قُل كُونُوا۟ حِجَارَةً أَو حَدِيدًا ﴿٥٠﴾\\
\textamh{51.\  } & أَو خَلقًۭا مِّمَّا يَكبُرُ فِى صُدُورِكُم ۚ فَسَيَقُولُونَ مَن يُعِيدُنَا ۖ قُلِ ٱلَّذِى فَطَرَكُم أَوَّلَ مَرَّةٍۢ ۚ فَسَيُنغِضُونَ إِلَيكَ رُءُوسَهُم وَيَقُولُونَ مَتَىٰ هُوَ ۖ قُل عَسَىٰٓ أَن يَكُونَ قَرِيبًۭا ﴿٥١﴾\\
\textamh{52.\  } & يَومَ يَدعُوكُم فَتَستَجِيبُونَ بِحَمدِهِۦ وَتَظُنُّونَ إِن لَّبِثتُم إِلَّا قَلِيلًۭا ﴿٥٢﴾\\
\textamh{53.\  } & وَقُل لِّعِبَادِى يَقُولُوا۟ ٱلَّتِى هِىَ أَحسَنُ ۚ إِنَّ ٱلشَّيطَٰنَ يَنزَغُ بَينَهُم ۚ إِنَّ ٱلشَّيطَٰنَ كَانَ لِلإِنسَـٰنِ عَدُوًّۭا مُّبِينًۭا ﴿٥٣﴾\\
\textamh{54.\  } & رَّبُّكُم أَعلَمُ بِكُم ۖ إِن يَشَأ يَرحَمكُم أَو إِن يَشَأ يُعَذِّبكُم ۚ وَمَآ أَرسَلنَـٰكَ عَلَيهِم وَكِيلًۭا ﴿٥٤﴾\\
\textamh{55.\  } & وَرَبُّكَ أَعلَمُ بِمَن فِى ٱلسَّمَـٰوَٟتِ وَٱلأَرضِ ۗ وَلَقَد فَضَّلنَا بَعضَ ٱلنَّبِيِّۦنَ عَلَىٰ بَعضٍۢ ۖ وَءَاتَينَا دَاوُۥدَ زَبُورًۭا ﴿٥٥﴾\\
\textamh{56.\  } & قُلِ ٱدعُوا۟ ٱلَّذِينَ زَعَمتُم مِّن دُونِهِۦ فَلَا يَملِكُونَ كَشفَ ٱلضُّرِّ عَنكُم وَلَا تَحوِيلًا ﴿٥٦﴾\\
\textamh{57.\  } & أُو۟لَـٰٓئِكَ ٱلَّذِينَ يَدعُونَ يَبتَغُونَ إِلَىٰ رَبِّهِمُ ٱلوَسِيلَةَ أَيُّهُم أَقرَبُ وَيَرجُونَ رَحمَتَهُۥ وَيَخَافُونَ عَذَابَهُۥٓ ۚ إِنَّ عَذَابَ رَبِّكَ كَانَ مَحذُورًۭا ﴿٥٧﴾\\
\textamh{58.\  } & وَإِن مِّن قَريَةٍ إِلَّا نَحنُ مُهلِكُوهَا قَبلَ يَومِ ٱلقِيَـٰمَةِ أَو مُعَذِّبُوهَا عَذَابًۭا شَدِيدًۭا ۚ كَانَ ذَٟلِكَ فِى ٱلكِتَـٰبِ مَسطُورًۭا ﴿٥٨﴾\\
\textamh{59.\  } & وَمَا مَنَعَنَآ أَن نُّرسِلَ بِٱلءَايَـٰتِ إِلَّآ أَن كَذَّبَ بِهَا ٱلأَوَّلُونَ ۚ وَءَاتَينَا ثَمُودَ ٱلنَّاقَةَ مُبصِرَةًۭ فَظَلَمُوا۟ بِهَا ۚ وَمَا نُرسِلُ بِٱلءَايَـٰتِ إِلَّا تَخوِيفًۭا ﴿٥٩﴾\\
\textamh{60.\  } & وَإِذ قُلنَا لَكَ إِنَّ رَبَّكَ أَحَاطَ بِٱلنَّاسِ ۚ وَمَا جَعَلنَا ٱلرُّءيَا ٱلَّتِىٓ أَرَينَـٰكَ إِلَّا فِتنَةًۭ لِّلنَّاسِ وَٱلشَّجَرَةَ ٱلمَلعُونَةَ فِى ٱلقُرءَانِ ۚ وَنُخَوِّفُهُم فَمَا يَزِيدُهُم إِلَّا طُغيَـٰنًۭا كَبِيرًۭا ﴿٦٠﴾\\
\textamh{61.\  } & وَإِذ قُلنَا لِلمَلَـٰٓئِكَةِ ٱسجُدُوا۟ لِءَادَمَ فَسَجَدُوٓا۟ إِلَّآ إِبلِيسَ قَالَ ءَأَسجُدُ لِمَن خَلَقتَ طِينًۭا ﴿٦١﴾\\
\textamh{62.\  } & قَالَ أَرَءَيتَكَ هَـٰذَا ٱلَّذِى كَرَّمتَ عَلَىَّ لَئِن أَخَّرتَنِ إِلَىٰ يَومِ ٱلقِيَـٰمَةِ لَأَحتَنِكَنَّ ذُرِّيَّتَهُۥٓ إِلَّا قَلِيلًۭا ﴿٦٢﴾\\
\textamh{63.\  } & قَالَ ٱذهَب فَمَن تَبِعَكَ مِنهُم فَإِنَّ جَهَنَّمَ جَزَآؤُكُم جَزَآءًۭ مَّوفُورًۭا ﴿٦٣﴾\\
\textamh{64.\  } & وَٱستَفزِز مَنِ ٱستَطَعتَ مِنهُم بِصَوتِكَ وَأَجلِب عَلَيهِم بِخَيلِكَ وَرَجِلِكَ وَشَارِكهُم فِى ٱلأَموَٟلِ وَٱلأَولَـٰدِ وَعِدهُم ۚ وَمَا يَعِدُهُمُ ٱلشَّيطَٰنُ إِلَّا غُرُورًا ﴿٦٤﴾\\
\textamh{65.\  } & إِنَّ عِبَادِى لَيسَ لَكَ عَلَيهِم سُلطَٰنٌۭ ۚ وَكَفَىٰ بِرَبِّكَ وَكِيلًۭا ﴿٦٥﴾\\
\textamh{66.\  } & رَّبُّكُمُ ٱلَّذِى يُزجِى لَكُمُ ٱلفُلكَ فِى ٱلبَحرِ لِتَبتَغُوا۟ مِن فَضلِهِۦٓ ۚ إِنَّهُۥ كَانَ بِكُم رَحِيمًۭا ﴿٦٦﴾\\
\textamh{67.\  } & وَإِذَا مَسَّكُمُ ٱلضُّرُّ فِى ٱلبَحرِ ضَلَّ مَن تَدعُونَ إِلَّآ إِيَّاهُ ۖ فَلَمَّا نَجَّىٰكُم إِلَى ٱلبَرِّ أَعرَضتُم ۚ وَكَانَ ٱلإِنسَـٰنُ كَفُورًا ﴿٦٧﴾\\
\textamh{68.\  } & أَفَأَمِنتُم أَن يَخسِفَ بِكُم جَانِبَ ٱلبَرِّ أَو يُرسِلَ عَلَيكُم حَاصِبًۭا ثُمَّ لَا تَجِدُوا۟ لَكُم وَكِيلًا ﴿٦٨﴾\\
\textamh{69.\  } & أَم أَمِنتُم أَن يُعِيدَكُم فِيهِ تَارَةً أُخرَىٰ فَيُرسِلَ عَلَيكُم قَاصِفًۭا مِّنَ ٱلرِّيحِ فَيُغرِقَكُم بِمَا كَفَرتُم ۙ ثُمَّ لَا تَجِدُوا۟ لَكُم عَلَينَا بِهِۦ تَبِيعًۭا ﴿٦٩﴾\\
\textamh{70.\  } & ۞ وَلَقَد كَرَّمنَا بَنِىٓ ءَادَمَ وَحَمَلنَـٰهُم فِى ٱلبَرِّ وَٱلبَحرِ وَرَزَقنَـٰهُم مِّنَ ٱلطَّيِّبَٰتِ وَفَضَّلنَـٰهُم عَلَىٰ كَثِيرٍۢ مِّمَّن خَلَقنَا تَفضِيلًۭا ﴿٧٠﴾\\
\textamh{71.\  } & يَومَ نَدعُوا۟ كُلَّ أُنَاسٍۭ بِإِمَـٰمِهِم ۖ فَمَن أُوتِىَ كِتَـٰبَهُۥ بِيَمِينِهِۦ فَأُو۟لَـٰٓئِكَ يَقرَءُونَ كِتَـٰبَهُم وَلَا يُظلَمُونَ فَتِيلًۭا ﴿٧١﴾\\
\textamh{72.\  } & وَمَن كَانَ فِى هَـٰذِهِۦٓ أَعمَىٰ فَهُوَ فِى ٱلءَاخِرَةِ أَعمَىٰ وَأَضَلُّ سَبِيلًۭا ﴿٧٢﴾\\
\textamh{73.\  } & وَإِن كَادُوا۟ لَيَفتِنُونَكَ عَنِ ٱلَّذِىٓ أَوحَينَآ إِلَيكَ لِتَفتَرِىَ عَلَينَا غَيرَهُۥ ۖ وَإِذًۭا لَّٱتَّخَذُوكَ خَلِيلًۭا ﴿٧٣﴾\\
\textamh{74.\  } & وَلَولَآ أَن ثَبَّتنَـٰكَ لَقَد كِدتَّ تَركَنُ إِلَيهِم شَيـًۭٔا قَلِيلًا ﴿٧٤﴾\\
\textamh{75.\  } & إِذًۭا لَّأَذَقنَـٰكَ ضِعفَ ٱلحَيَوٰةِ وَضِعفَ ٱلمَمَاتِ ثُمَّ لَا تَجِدُ لَكَ عَلَينَا نَصِيرًۭا ﴿٧٥﴾\\
\textamh{76.\  } & وَإِن كَادُوا۟ لَيَستَفِزُّونَكَ مِنَ ٱلأَرضِ لِيُخرِجُوكَ مِنهَا ۖ وَإِذًۭا لَّا يَلبَثُونَ خِلَـٰفَكَ إِلَّا قَلِيلًۭا ﴿٧٦﴾\\
\textamh{77.\  } & سُنَّةَ مَن قَد أَرسَلنَا قَبلَكَ مِن رُّسُلِنَا ۖ وَلَا تَجِدُ لِسُنَّتِنَا تَحوِيلًا ﴿٧٧﴾\\
\textamh{78.\  } & أَقِمِ ٱلصَّلَوٰةَ لِدُلُوكِ ٱلشَّمسِ إِلَىٰ غَسَقِ ٱلَّيلِ وَقُرءَانَ ٱلفَجرِ ۖ إِنَّ قُرءَانَ ٱلفَجرِ كَانَ مَشهُودًۭا ﴿٧٨﴾\\
\textamh{79.\  } & وَمِنَ ٱلَّيلِ فَتَهَجَّد بِهِۦ نَافِلَةًۭ لَّكَ عَسَىٰٓ أَن يَبعَثَكَ رَبُّكَ مَقَامًۭا مَّحمُودًۭا ﴿٧٩﴾\\
\textamh{80.\  } & وَقُل رَّبِّ أَدخِلنِى مُدخَلَ صِدقٍۢ وَأَخرِجنِى مُخرَجَ صِدقٍۢ وَٱجعَل لِّى مِن لَّدُنكَ سُلطَٰنًۭا نَّصِيرًۭا ﴿٨٠﴾\\
\textamh{81.\  } & وَقُل جَآءَ ٱلحَقُّ وَزَهَقَ ٱلبَٰطِلُ ۚ إِنَّ ٱلبَٰطِلَ كَانَ زَهُوقًۭا ﴿٨١﴾\\
\textamh{82.\  } & وَنُنَزِّلُ مِنَ ٱلقُرءَانِ مَا هُوَ شِفَآءٌۭ وَرَحمَةٌۭ لِّلمُؤمِنِينَ ۙ وَلَا يَزِيدُ ٱلظَّـٰلِمِينَ إِلَّا خَسَارًۭا ﴿٨٢﴾\\
\textamh{83.\  } & وَإِذَآ أَنعَمنَا عَلَى ٱلإِنسَـٰنِ أَعرَضَ وَنَـَٔا بِجَانِبِهِۦ ۖ وَإِذَا مَسَّهُ ٱلشَّرُّ كَانَ يَـُٔوسًۭا ﴿٨٣﴾\\
\textamh{84.\  } & قُل كُلٌّۭ يَعمَلُ عَلَىٰ شَاكِلَتِهِۦ فَرَبُّكُم أَعلَمُ بِمَن هُوَ أَهدَىٰ سَبِيلًۭا ﴿٨٤﴾\\
\textamh{85.\  } & وَيَسـَٔلُونَكَ عَنِ ٱلرُّوحِ ۖ قُلِ ٱلرُّوحُ مِن أَمرِ رَبِّى وَمَآ أُوتِيتُم مِّنَ ٱلعِلمِ إِلَّا قَلِيلًۭا ﴿٨٥﴾\\
\textamh{86.\  } & وَلَئِن شِئنَا لَنَذهَبَنَّ بِٱلَّذِىٓ أَوحَينَآ إِلَيكَ ثُمَّ لَا تَجِدُ لَكَ بِهِۦ عَلَينَا وَكِيلًا ﴿٨٦﴾\\
\textamh{87.\  } & إِلَّا رَحمَةًۭ مِّن رَّبِّكَ ۚ إِنَّ فَضلَهُۥ كَانَ عَلَيكَ كَبِيرًۭا ﴿٨٧﴾\\
\textamh{88.\  } & قُل لَّئِنِ ٱجتَمَعَتِ ٱلإِنسُ وَٱلجِنُّ عَلَىٰٓ أَن يَأتُوا۟ بِمِثلِ هَـٰذَا ٱلقُرءَانِ لَا يَأتُونَ بِمِثلِهِۦ وَلَو كَانَ بَعضُهُم لِبَعضٍۢ ظَهِيرًۭا ﴿٨٨﴾\\
\textamh{89.\  } & وَلَقَد صَرَّفنَا لِلنَّاسِ فِى هَـٰذَا ٱلقُرءَانِ مِن كُلِّ مَثَلٍۢ فَأَبَىٰٓ أَكثَرُ ٱلنَّاسِ إِلَّا كُفُورًۭا ﴿٨٩﴾\\
\textamh{90.\  } & وَقَالُوا۟ لَن نُّؤمِنَ لَكَ حَتَّىٰ تَفجُرَ لَنَا مِنَ ٱلأَرضِ يَنۢبُوعًا ﴿٩٠﴾\\
\textamh{91.\  } & أَو تَكُونَ لَكَ جَنَّةٌۭ مِّن نَّخِيلٍۢ وَعِنَبٍۢ فَتُفَجِّرَ ٱلأَنهَـٰرَ خِلَـٰلَهَا تَفجِيرًا ﴿٩١﴾\\
\textamh{92.\  } & أَو تُسقِطَ ٱلسَّمَآءَ كَمَا زَعَمتَ عَلَينَا كِسَفًا أَو تَأتِىَ بِٱللَّهِ وَٱلمَلَـٰٓئِكَةِ قَبِيلًا ﴿٩٢﴾\\
\textamh{93.\  } & أَو يَكُونَ لَكَ بَيتٌۭ مِّن زُخرُفٍ أَو تَرقَىٰ فِى ٱلسَّمَآءِ وَلَن نُّؤمِنَ لِرُقِيِّكَ حَتَّىٰ تُنَزِّلَ عَلَينَا كِتَـٰبًۭا نَّقرَؤُهُۥ ۗ قُل سُبحَانَ رَبِّى هَل كُنتُ إِلَّا بَشَرًۭا رَّسُولًۭا ﴿٩٣﴾\\
\textamh{94.\  } & وَمَا مَنَعَ ٱلنَّاسَ أَن يُؤمِنُوٓا۟ إِذ جَآءَهُمُ ٱلهُدَىٰٓ إِلَّآ أَن قَالُوٓا۟ أَبَعَثَ ٱللَّهُ بَشَرًۭا رَّسُولًۭا ﴿٩٤﴾\\
\textamh{95.\  } & قُل لَّو كَانَ فِى ٱلأَرضِ مَلَـٰٓئِكَةٌۭ يَمشُونَ مُطمَئِنِّينَ لَنَزَّلنَا عَلَيهِم مِّنَ ٱلسَّمَآءِ مَلَكًۭا رَّسُولًۭا ﴿٩٥﴾\\
\textamh{96.\  } & قُل كَفَىٰ بِٱللَّهِ شَهِيدًۢا بَينِى وَبَينَكُم ۚ إِنَّهُۥ كَانَ بِعِبَادِهِۦ خَبِيرًۢا بَصِيرًۭا ﴿٩٦﴾\\
\textamh{97.\  } & وَمَن يَهدِ ٱللَّهُ فَهُوَ ٱلمُهتَدِ ۖ وَمَن يُضلِل فَلَن تَجِدَ لَهُم أَولِيَآءَ مِن دُونِهِۦ ۖ وَنَحشُرُهُم يَومَ ٱلقِيَـٰمَةِ عَلَىٰ وُجُوهِهِم عُميًۭا وَبُكمًۭا وَصُمًّۭا ۖ مَّأوَىٰهُم جَهَنَّمُ ۖ كُلَّمَا خَبَت زِدنَـٰهُم سَعِيرًۭا ﴿٩٧﴾\\
\textamh{98.\  } & ذَٟلِكَ جَزَآؤُهُم بِأَنَّهُم كَفَرُوا۟ بِـَٔايَـٰتِنَا وَقَالُوٓا۟ أَءِذَا كُنَّا عِظَـٰمًۭا وَرُفَـٰتًا أَءِنَّا لَمَبعُوثُونَ خَلقًۭا جَدِيدًا ﴿٩٨﴾\\
\textamh{99.\  } & ۞ أَوَلَم يَرَوا۟ أَنَّ ٱللَّهَ ٱلَّذِى خَلَقَ ٱلسَّمَـٰوَٟتِ وَٱلأَرضَ قَادِرٌ عَلَىٰٓ أَن يَخلُقَ مِثلَهُم وَجَعَلَ لَهُم أَجَلًۭا لَّا رَيبَ فِيهِ فَأَبَى ٱلظَّـٰلِمُونَ إِلَّا كُفُورًۭا ﴿٩٩﴾\\
\textamh{100.\  } & قُل لَّو أَنتُم تَملِكُونَ خَزَآئِنَ رَحمَةِ رَبِّىٓ إِذًۭا لَّأَمسَكتُم خَشيَةَ ٱلإِنفَاقِ ۚ وَكَانَ ٱلإِنسَـٰنُ قَتُورًۭا ﴿١٠٠﴾\\
\textamh{101.\  } & وَلَقَد ءَاتَينَا مُوسَىٰ تِسعَ ءَايَـٰتٍۭ بَيِّنَـٰتٍۢ ۖ فَسـَٔل بَنِىٓ إِسرَٰٓءِيلَ إِذ جَآءَهُم فَقَالَ لَهُۥ فِرعَونُ إِنِّى لَأَظُنُّكَ يَـٰمُوسَىٰ مَسحُورًۭا ﴿١٠١﴾\\
\textamh{102.\  } & قَالَ لَقَد عَلِمتَ مَآ أَنزَلَ هَـٰٓؤُلَآءِ إِلَّا رَبُّ ٱلسَّمَـٰوَٟتِ وَٱلأَرضِ بَصَآئِرَ وَإِنِّى لَأَظُنُّكَ يَـٰفِرعَونُ مَثبُورًۭا ﴿١٠٢﴾\\
\textamh{103.\  } & فَأَرَادَ أَن يَستَفِزَّهُم مِّنَ ٱلأَرضِ فَأَغرَقنَـٰهُ وَمَن مَّعَهُۥ جَمِيعًۭا ﴿١٠٣﴾\\
\textamh{104.\  } & وَقُلنَا مِنۢ بَعدِهِۦ لِبَنِىٓ إِسرَٰٓءِيلَ ٱسكُنُوا۟ ٱلأَرضَ فَإِذَا جَآءَ وَعدُ ٱلءَاخِرَةِ جِئنَا بِكُم لَفِيفًۭا ﴿١٠٤﴾\\
\textamh{105.\  } & وَبِٱلحَقِّ أَنزَلنَـٰهُ وَبِٱلحَقِّ نَزَلَ ۗ وَمَآ أَرسَلنَـٰكَ إِلَّا مُبَشِّرًۭا وَنَذِيرًۭا ﴿١٠٥﴾\\
\textamh{106.\  } & وَقُرءَانًۭا فَرَقنَـٰهُ لِتَقرَأَهُۥ عَلَى ٱلنَّاسِ عَلَىٰ مُكثٍۢ وَنَزَّلنَـٰهُ تَنزِيلًۭا ﴿١٠٦﴾\\
\textamh{107.\  } & قُل ءَامِنُوا۟ بِهِۦٓ أَو لَا تُؤمِنُوٓا۟ ۚ إِنَّ ٱلَّذِينَ أُوتُوا۟ ٱلعِلمَ مِن قَبلِهِۦٓ إِذَا يُتلَىٰ عَلَيهِم يَخِرُّونَ لِلأَذقَانِ سُجَّدًۭا ﴿١٠٧﴾\\
\textamh{108.\  } & وَيَقُولُونَ سُبحَـٰنَ رَبِّنَآ إِن كَانَ وَعدُ رَبِّنَا لَمَفعُولًۭا ﴿١٠٨﴾\\
\textamh{109.\  } & وَيَخِرُّونَ لِلأَذقَانِ يَبكُونَ وَيَزِيدُهُم خُشُوعًۭا ۩ ﴿١٠٩﴾\\
\textamh{110.\  } & قُلِ ٱدعُوا۟ ٱللَّهَ أَوِ ٱدعُوا۟ ٱلرَّحمَـٰنَ ۖ أَيًّۭا مَّا تَدعُوا۟ فَلَهُ ٱلأَسمَآءُ ٱلحُسنَىٰ ۚ وَلَا تَجهَر بِصَلَاتِكَ وَلَا تُخَافِت بِهَا وَٱبتَغِ بَينَ ذَٟلِكَ سَبِيلًۭا ﴿١١٠﴾\\
\textamh{111.\  } & وَقُلِ ٱلحَمدُ لِلَّهِ ٱلَّذِى لَم يَتَّخِذ وَلَدًۭا وَلَم يَكُن لَّهُۥ شَرِيكٌۭ فِى ٱلمُلكِ وَلَم يَكُن لَّهُۥ وَلِىٌّۭ مِّنَ ٱلذُّلِّ ۖ وَكَبِّرهُ تَكبِيرًۢا ﴿١١١﴾\\
\end{longtable} \newpage
