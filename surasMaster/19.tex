%% License: BSD style (Berkley) (i.e. Put the Copyright owner's name always)
%% Writer and Copyright (to): Bewketu(Bilal) Tadilo (2016-17)
\shadowbox{\section{\LR{\textamharic{ሱራቱ ማሪያም -}  \RL{سوره  مريم}}}}
\begin{longtable}{%
  @{}
    p{.5\textwidth}
  @{~~~~~~~~~~~~~}||
    p{.5\textwidth}
    @{}
}
\nopagebreak
\textamh{\ \ \ \ \ \  ቢስሚላሂ አራህመኒ ራሂይም } &  بِسمِ ٱللَّهِ ٱلرَّحمَـٰنِ ٱلرَّحِيمِ\\
\textamh{1.\  } &  كٓهيعٓصٓ ﴿١﴾\\
\textamh{2.\  } & ذِكرُ رَحمَتِ رَبِّكَ عَبدَهُۥ زَكَرِيَّآ ﴿٢﴾\\
\textamh{3.\  } & إِذ نَادَىٰ رَبَّهُۥ نِدَآءً خَفِيًّۭا ﴿٣﴾\\
\textamh{4.\  } & قَالَ رَبِّ إِنِّى وَهَنَ ٱلعَظمُ مِنِّى وَٱشتَعَلَ ٱلرَّأسُ شَيبًۭا وَلَم أَكُنۢ بِدُعَآئِكَ رَبِّ شَقِيًّۭا ﴿٤﴾\\
\textamh{5.\  } & وَإِنِّى خِفتُ ٱلمَوَٟلِىَ مِن وَرَآءِى وَكَانَتِ ٱمرَأَتِى عَاقِرًۭا فَهَب لِى مِن لَّدُنكَ وَلِيًّۭا ﴿٥﴾\\
\textamh{6.\  } & يَرِثُنِى وَيَرِثُ مِن ءَالِ يَعقُوبَ ۖ وَٱجعَلهُ رَبِّ رَضِيًّۭا ﴿٦﴾\\
\textamh{7.\  } & يَـٰزَكَرِيَّآ إِنَّا نُبَشِّرُكَ بِغُلَـٰمٍ ٱسمُهُۥ يَحيَىٰ لَم نَجعَل لَّهُۥ مِن قَبلُ سَمِيًّۭا ﴿٧﴾\\
\textamh{8.\  } & قَالَ رَبِّ أَنَّىٰ يَكُونُ لِى غُلَـٰمٌۭ وَكَانَتِ ٱمرَأَتِى عَاقِرًۭا وَقَد بَلَغتُ مِنَ ٱلكِبَرِ عِتِيًّۭا ﴿٨﴾\\
\textamh{9.\  } & قَالَ كَذَٟلِكَ قَالَ رَبُّكَ هُوَ عَلَىَّ هَيِّنٌۭ وَقَد خَلَقتُكَ مِن قَبلُ وَلَم تَكُ شَيـًۭٔا ﴿٩﴾\\
\textamh{10.\  } & قَالَ رَبِّ ٱجعَل لِّىٓ ءَايَةًۭ ۚ قَالَ ءَايَتُكَ أَلَّا تُكَلِّمَ ٱلنَّاسَ ثَلَـٰثَ لَيَالٍۢ سَوِيًّۭا ﴿١٠﴾\\
\textamh{11.\  } & فَخَرَجَ عَلَىٰ قَومِهِۦ مِنَ ٱلمِحرَابِ فَأَوحَىٰٓ إِلَيهِم أَن سَبِّحُوا۟ بُكرَةًۭ وَعَشِيًّۭا ﴿١١﴾\\
\textamh{12.\  } & يَـٰيَحيَىٰ خُذِ ٱلكِتَـٰبَ بِقُوَّةٍۢ ۖ وَءَاتَينَـٰهُ ٱلحُكمَ صَبِيًّۭا ﴿١٢﴾\\
\textamh{13.\  } & وَحَنَانًۭا مِّن لَّدُنَّا وَزَكَوٰةًۭ ۖ وَكَانَ تَقِيًّۭا ﴿١٣﴾\\
\textamh{14.\  } & وَبَرًّۢا بِوَٟلِدَيهِ وَلَم يَكُن جَبَّارًا عَصِيًّۭا ﴿١٤﴾\\
\textamh{15.\  } & وَسَلَـٰمٌ عَلَيهِ يَومَ وُلِدَ وَيَومَ يَمُوتُ وَيَومَ يُبعَثُ حَيًّۭا ﴿١٥﴾\\
\textamh{16.\  } & وَٱذكُر فِى ٱلكِتَـٰبِ مَريَمَ إِذِ ٱنتَبَذَت مِن أَهلِهَا مَكَانًۭا شَرقِيًّۭا ﴿١٦﴾\\
\textamh{17.\  } & فَٱتَّخَذَت مِن دُونِهِم حِجَابًۭا فَأَرسَلنَآ إِلَيهَا رُوحَنَا فَتَمَثَّلَ لَهَا بَشَرًۭا سَوِيًّۭا ﴿١٧﴾\\
\textamh{18.\  } & قَالَت إِنِّىٓ أَعُوذُ بِٱلرَّحمَـٰنِ مِنكَ إِن كُنتَ تَقِيًّۭا ﴿١٨﴾\\
\textamh{19.\  } & قَالَ إِنَّمَآ أَنَا۠ رَسُولُ رَبِّكِ لِأَهَبَ لَكِ غُلَـٰمًۭا زَكِيًّۭا ﴿١٩﴾\\
\textamh{20.\  } & قَالَت أَنَّىٰ يَكُونُ لِى غُلَـٰمٌۭ وَلَم يَمسَسنِى بَشَرٌۭ وَلَم أَكُ بَغِيًّۭا ﴿٢٠﴾\\
\textamh{21.\  } & قَالَ كَذَٟلِكِ قَالَ رَبُّكِ هُوَ عَلَىَّ هَيِّنٌۭ ۖ وَلِنَجعَلَهُۥٓ ءَايَةًۭ لِّلنَّاسِ وَرَحمَةًۭ مِّنَّا ۚ وَكَانَ أَمرًۭا مَّقضِيًّۭا ﴿٢١﴾\\
\textamh{22.\  } & ۞ فَحَمَلَتهُ فَٱنتَبَذَت بِهِۦ مَكَانًۭا قَصِيًّۭا ﴿٢٢﴾\\
\textamh{23.\  } & فَأَجَآءَهَا ٱلمَخَاضُ إِلَىٰ جِذعِ ٱلنَّخلَةِ قَالَت يَـٰلَيتَنِى مِتُّ قَبلَ هَـٰذَا وَكُنتُ نَسيًۭا مَّنسِيًّۭا ﴿٢٣﴾\\
\textamh{24.\  } & فَنَادَىٰهَا مِن تَحتِهَآ أَلَّا تَحزَنِى قَد جَعَلَ رَبُّكِ تَحتَكِ سَرِيًّۭا ﴿٢٤﴾\\
\textamh{25.\  } & وَهُزِّىٓ إِلَيكِ بِجِذعِ ٱلنَّخلَةِ تُسَـٰقِط عَلَيكِ رُطَبًۭا جَنِيًّۭا ﴿٢٥﴾\\
\textamh{26.\  } & فَكُلِى وَٱشرَبِى وَقَرِّى عَينًۭا ۖ فَإِمَّا تَرَيِنَّ مِنَ ٱلبَشَرِ أَحَدًۭا فَقُولِىٓ إِنِّى نَذَرتُ لِلرَّحمَـٰنِ صَومًۭا فَلَن أُكَلِّمَ ٱليَومَ إِنسِيًّۭا ﴿٢٦﴾\\
\textamh{27.\  } & فَأَتَت بِهِۦ قَومَهَا تَحمِلُهُۥ ۖ قَالُوا۟ يَـٰمَريَمُ لَقَد جِئتِ شَيـًۭٔا فَرِيًّۭا ﴿٢٧﴾\\
\textamh{28.\  } & يَـٰٓأُختَ هَـٰرُونَ مَا كَانَ أَبُوكِ ٱمرَأَ سَوءٍۢ وَمَا كَانَت أُمُّكِ بَغِيًّۭا ﴿٢٨﴾\\
\textamh{29.\  } & فَأَشَارَت إِلَيهِ ۖ قَالُوا۟ كَيفَ نُكَلِّمُ مَن كَانَ فِى ٱلمَهدِ صَبِيًّۭا ﴿٢٩﴾\\
\textamh{30.\  } & قَالَ إِنِّى عَبدُ ٱللَّهِ ءَاتَىٰنِىَ ٱلكِتَـٰبَ وَجَعَلَنِى نَبِيًّۭا ﴿٣٠﴾\\
\textamh{31.\  } & وَجَعَلَنِى مُبَارَكًا أَينَ مَا كُنتُ وَأَوصَـٰنِى بِٱلصَّلَوٰةِ وَٱلزَّكَوٰةِ مَا دُمتُ حَيًّۭا ﴿٣١﴾\\
\textamh{32.\  } & وَبَرًّۢا بِوَٟلِدَتِى وَلَم يَجعَلنِى جَبَّارًۭا شَقِيًّۭا ﴿٣٢﴾\\
\textamh{33.\  } & وَٱلسَّلَـٰمُ عَلَىَّ يَومَ وُلِدتُّ وَيَومَ أَمُوتُ وَيَومَ أُبعَثُ حَيًّۭا ﴿٣٣﴾\\
\textamh{34.\  } & ذَٟلِكَ عِيسَى ٱبنُ مَريَمَ ۚ قَولَ ٱلحَقِّ ٱلَّذِى فِيهِ يَمتَرُونَ ﴿٣٤﴾\\
\textamh{35.\  } & مَا كَانَ لِلَّهِ أَن يَتَّخِذَ مِن وَلَدٍۢ ۖ سُبحَـٰنَهُۥٓ ۚ إِذَا قَضَىٰٓ أَمرًۭا فَإِنَّمَا يَقُولُ لَهُۥ كُن فَيَكُونُ ﴿٣٥﴾\\
\textamh{36.\  } & وَإِنَّ ٱللَّهَ رَبِّى وَرَبُّكُم فَٱعبُدُوهُ ۚ هَـٰذَا صِرَٰطٌۭ مُّستَقِيمٌۭ ﴿٣٦﴾\\
\textamh{37.\  } & فَٱختَلَفَ ٱلأَحزَابُ مِنۢ بَينِهِم ۖ فَوَيلٌۭ لِّلَّذِينَ كَفَرُوا۟ مِن مَّشهَدِ يَومٍ عَظِيمٍ ﴿٣٧﴾\\
\textamh{38.\  } & أَسمِع بِهِم وَأَبصِر يَومَ يَأتُونَنَا ۖ لَـٰكِنِ ٱلظَّـٰلِمُونَ ٱليَومَ فِى ضَلَـٰلٍۢ مُّبِينٍۢ ﴿٣٨﴾\\
\textamh{39.\  } & وَأَنذِرهُم يَومَ ٱلحَسرَةِ إِذ قُضِىَ ٱلأَمرُ وَهُم فِى غَفلَةٍۢ وَهُم لَا يُؤمِنُونَ ﴿٣٩﴾\\
\textamh{40.\  } & إِنَّا نَحنُ نَرِثُ ٱلأَرضَ وَمَن عَلَيهَا وَإِلَينَا يُرجَعُونَ ﴿٤٠﴾\\
\textamh{41.\  } & وَٱذكُر فِى ٱلكِتَـٰبِ إِبرَٰهِيمَ ۚ إِنَّهُۥ كَانَ صِدِّيقًۭا نَّبِيًّا ﴿٤١﴾\\
\textamh{42.\  } & إِذ قَالَ لِأَبِيهِ يَـٰٓأَبَتِ لِمَ تَعبُدُ مَا لَا يَسمَعُ وَلَا يُبصِرُ وَلَا يُغنِى عَنكَ شَيـًۭٔا ﴿٤٢﴾\\
\textamh{43.\  } & يَـٰٓأَبَتِ إِنِّى قَد جَآءَنِى مِنَ ٱلعِلمِ مَا لَم يَأتِكَ فَٱتَّبِعنِىٓ أَهدِكَ صِرَٰطًۭا سَوِيًّۭا ﴿٤٣﴾\\
\textamh{44.\  } & يَـٰٓأَبَتِ لَا تَعبُدِ ٱلشَّيطَٰنَ ۖ إِنَّ ٱلشَّيطَٰنَ كَانَ لِلرَّحمَـٰنِ عَصِيًّۭا ﴿٤٤﴾\\
\textamh{45.\  } & يَـٰٓأَبَتِ إِنِّىٓ أَخَافُ أَن يَمَسَّكَ عَذَابٌۭ مِّنَ ٱلرَّحمَـٰنِ فَتَكُونَ لِلشَّيطَٰنِ وَلِيًّۭا ﴿٤٥﴾\\
\textamh{46.\  } & قَالَ أَرَاغِبٌ أَنتَ عَن ءَالِهَتِى يَـٰٓإِبرَٰهِيمُ ۖ لَئِن لَّم تَنتَهِ لَأَرجُمَنَّكَ ۖ وَٱهجُرنِى مَلِيًّۭا ﴿٤٦﴾\\
\textamh{47.\  } & قَالَ سَلَـٰمٌ عَلَيكَ ۖ سَأَستَغفِرُ لَكَ رَبِّىٓ ۖ إِنَّهُۥ كَانَ بِى حَفِيًّۭا ﴿٤٧﴾\\
\textamh{48.\  } & وَأَعتَزِلُكُم وَمَا تَدعُونَ مِن دُونِ ٱللَّهِ وَأَدعُوا۟ رَبِّى عَسَىٰٓ أَلَّآ أَكُونَ بِدُعَآءِ رَبِّى شَقِيًّۭا ﴿٤٨﴾\\
\textamh{49.\  } & فَلَمَّا ٱعتَزَلَهُم وَمَا يَعبُدُونَ مِن دُونِ ٱللَّهِ وَهَبنَا لَهُۥٓ إِسحَـٰقَ وَيَعقُوبَ ۖ وَكُلًّۭا جَعَلنَا نَبِيًّۭا ﴿٤٩﴾\\
\textamh{50.\  } & وَوَهَبنَا لَهُم مِّن رَّحمَتِنَا وَجَعَلنَا لَهُم لِسَانَ صِدقٍ عَلِيًّۭا ﴿٥٠﴾\\
\textamh{51.\  } & وَٱذكُر فِى ٱلكِتَـٰبِ مُوسَىٰٓ ۚ إِنَّهُۥ كَانَ مُخلَصًۭا وَكَانَ رَسُولًۭا نَّبِيًّۭا ﴿٥١﴾\\
\textamh{52.\  } & وَنَـٰدَينَـٰهُ مِن جَانِبِ ٱلطُّورِ ٱلأَيمَنِ وَقَرَّبنَـٰهُ نَجِيًّۭا ﴿٥٢﴾\\
\textamh{53.\  } & وَوَهَبنَا لَهُۥ مِن رَّحمَتِنَآ أَخَاهُ هَـٰرُونَ نَبِيًّۭا ﴿٥٣﴾\\
\textamh{54.\  } & وَٱذكُر فِى ٱلكِتَـٰبِ إِسمَـٰعِيلَ ۚ إِنَّهُۥ كَانَ صَادِقَ ٱلوَعدِ وَكَانَ رَسُولًۭا نَّبِيًّۭا ﴿٥٤﴾\\
\textamh{55.\  } & وَكَانَ يَأمُرُ أَهلَهُۥ بِٱلصَّلَوٰةِ وَٱلزَّكَوٰةِ وَكَانَ عِندَ رَبِّهِۦ مَرضِيًّۭا ﴿٥٥﴾\\
\textamh{56.\  } & وَٱذكُر فِى ٱلكِتَـٰبِ إِدرِيسَ ۚ إِنَّهُۥ كَانَ صِدِّيقًۭا نَّبِيًّۭا ﴿٥٦﴾\\
\textamh{57.\  } & وَرَفَعنَـٰهُ مَكَانًا عَلِيًّا ﴿٥٧﴾\\
\textamh{58.\  } & أُو۟لَـٰٓئِكَ ٱلَّذِينَ أَنعَمَ ٱللَّهُ عَلَيهِم مِّنَ ٱلنَّبِيِّۦنَ مِن ذُرِّيَّةِ ءَادَمَ وَمِمَّن حَمَلنَا مَعَ نُوحٍۢ وَمِن ذُرِّيَّةِ إِبرَٰهِيمَ وَإِسرَٰٓءِيلَ وَمِمَّن هَدَينَا وَٱجتَبَينَآ ۚ إِذَا تُتلَىٰ عَلَيهِم ءَايَـٰتُ ٱلرَّحمَـٰنِ خَرُّوا۟ سُجَّدًۭا وَبُكِيًّۭا ۩ ﴿٥٨﴾\\
\textamh{59.\  } & ۞ فَخَلَفَ مِنۢ بَعدِهِم خَلفٌ أَضَاعُوا۟ ٱلصَّلَوٰةَ وَٱتَّبَعُوا۟ ٱلشَّهَوَٟتِ ۖ فَسَوفَ يَلقَونَ غَيًّا ﴿٥٩﴾\\
\textamh{60.\  } & إِلَّا مَن تَابَ وَءَامَنَ وَعَمِلَ صَـٰلِحًۭا فَأُو۟لَـٰٓئِكَ يَدخُلُونَ ٱلجَنَّةَ وَلَا يُظلَمُونَ شَيـًۭٔا ﴿٦٠﴾\\
\textamh{61.\  } & جَنَّـٰتِ عَدنٍ ٱلَّتِى وَعَدَ ٱلرَّحمَـٰنُ عِبَادَهُۥ بِٱلغَيبِ ۚ إِنَّهُۥ كَانَ وَعدُهُۥ مَأتِيًّۭا ﴿٦١﴾\\
\textamh{62.\  } & لَّا يَسمَعُونَ فِيهَا لَغوًا إِلَّا سَلَـٰمًۭا ۖ وَلَهُم رِزقُهُم فِيهَا بُكرَةًۭ وَعَشِيًّۭا ﴿٦٢﴾\\
\textamh{63.\  } & تِلكَ ٱلجَنَّةُ ٱلَّتِى نُورِثُ مِن عِبَادِنَا مَن كَانَ تَقِيًّۭا ﴿٦٣﴾\\
\textamh{64.\  } & وَمَا نَتَنَزَّلُ إِلَّا بِأَمرِ رَبِّكَ ۖ لَهُۥ مَا بَينَ أَيدِينَا وَمَا خَلفَنَا وَمَا بَينَ ذَٟلِكَ ۚ وَمَا كَانَ رَبُّكَ نَسِيًّۭا ﴿٦٤﴾\\
\textamh{65.\  } & رَّبُّ ٱلسَّمَـٰوَٟتِ وَٱلأَرضِ وَمَا بَينَهُمَا فَٱعبُدهُ وَٱصطَبِر لِعِبَٰدَتِهِۦ ۚ هَل تَعلَمُ لَهُۥ سَمِيًّۭا ﴿٦٥﴾\\
\textamh{66.\  } & وَيَقُولُ ٱلإِنسَـٰنُ أَءِذَا مَا مِتُّ لَسَوفَ أُخرَجُ حَيًّا ﴿٦٦﴾\\
\textamh{67.\  } & أَوَلَا يَذكُرُ ٱلإِنسَـٰنُ أَنَّا خَلَقنَـٰهُ مِن قَبلُ وَلَم يَكُ شَيـًۭٔا ﴿٦٧﴾\\
\textamh{68.\  } & فَوَرَبِّكَ لَنَحشُرَنَّهُم وَٱلشَّيَـٰطِينَ ثُمَّ لَنُحضِرَنَّهُم حَولَ جَهَنَّمَ جِثِيًّۭا ﴿٦٨﴾\\
\textamh{69.\  } & ثُمَّ لَنَنزِعَنَّ مِن كُلِّ شِيعَةٍ أَيُّهُم أَشَدُّ عَلَى ٱلرَّحمَـٰنِ عِتِيًّۭا ﴿٦٩﴾\\
\textamh{70.\  } & ثُمَّ لَنَحنُ أَعلَمُ بِٱلَّذِينَ هُم أَولَىٰ بِهَا صِلِيًّۭا ﴿٧٠﴾\\
\textamh{71.\  } & وَإِن مِّنكُم إِلَّا وَارِدُهَا ۚ كَانَ عَلَىٰ رَبِّكَ حَتمًۭا مَّقضِيًّۭا ﴿٧١﴾\\
\textamh{72.\  } & ثُمَّ نُنَجِّى ٱلَّذِينَ ٱتَّقَوا۟ وَّنَذَرُ ٱلظَّـٰلِمِينَ فِيهَا جِثِيًّۭا ﴿٧٢﴾\\
\textamh{73.\  } & وَإِذَا تُتلَىٰ عَلَيهِم ءَايَـٰتُنَا بَيِّنَـٰتٍۢ قَالَ ٱلَّذِينَ كَفَرُوا۟ لِلَّذِينَ ءَامَنُوٓا۟ أَىُّ ٱلفَرِيقَينِ خَيرٌۭ مَّقَامًۭا وَأَحسَنُ نَدِيًّۭا ﴿٧٣﴾\\
\textamh{74.\  } & وَكَم أَهلَكنَا قَبلَهُم مِّن قَرنٍ هُم أَحسَنُ أَثَـٰثًۭا وَرِءيًۭا ﴿٧٤﴾\\
\textamh{75.\  } & قُل مَن كَانَ فِى ٱلضَّلَـٰلَةِ فَليَمدُد لَهُ ٱلرَّحمَـٰنُ مَدًّا ۚ حَتَّىٰٓ إِذَا رَأَوا۟ مَا يُوعَدُونَ إِمَّا ٱلعَذَابَ وَإِمَّا ٱلسَّاعَةَ فَسَيَعلَمُونَ مَن هُوَ شَرٌّۭ مَّكَانًۭا وَأَضعَفُ جُندًۭا ﴿٧٥﴾\\
\textamh{76.\  } & وَيَزِيدُ ٱللَّهُ ٱلَّذِينَ ٱهتَدَوا۟ هُدًۭى ۗ وَٱلبَٰقِيَـٰتُ ٱلصَّـٰلِحَـٰتُ خَيرٌ عِندَ رَبِّكَ ثَوَابًۭا وَخَيرٌۭ مَّرَدًّا ﴿٧٦﴾\\
\textamh{77.\  } & أَفَرَءَيتَ ٱلَّذِى كَفَرَ بِـَٔايَـٰتِنَا وَقَالَ لَأُوتَيَنَّ مَالًۭا وَوَلَدًا ﴿٧٧﴾\\
\textamh{78.\  } & أَطَّلَعَ ٱلغَيبَ أَمِ ٱتَّخَذَ عِندَ ٱلرَّحمَـٰنِ عَهدًۭا ﴿٧٨﴾\\
\textamh{79.\  } & كَلَّا ۚ سَنَكتُبُ مَا يَقُولُ وَنَمُدُّ لَهُۥ مِنَ ٱلعَذَابِ مَدًّۭا ﴿٧٩﴾\\
\textamh{80.\  } & وَنَرِثُهُۥ مَا يَقُولُ وَيَأتِينَا فَردًۭا ﴿٨٠﴾\\
\textamh{81.\  } & وَٱتَّخَذُوا۟ مِن دُونِ ٱللَّهِ ءَالِهَةًۭ لِّيَكُونُوا۟ لَهُم عِزًّۭا ﴿٨١﴾\\
\textamh{82.\  } & كَلَّا ۚ سَيَكفُرُونَ بِعِبَادَتِهِم وَيَكُونُونَ عَلَيهِم ضِدًّا ﴿٨٢﴾\\
\textamh{83.\  } & أَلَم تَرَ أَنَّآ أَرسَلنَا ٱلشَّيَـٰطِينَ عَلَى ٱلكَـٰفِرِينَ تَؤُزُّهُم أَزًّۭا ﴿٨٣﴾\\
\textamh{84.\  } & فَلَا تَعجَل عَلَيهِم ۖ إِنَّمَا نَعُدُّ لَهُم عَدًّۭا ﴿٨٤﴾\\
\textamh{85.\  } & يَومَ نَحشُرُ ٱلمُتَّقِينَ إِلَى ٱلرَّحمَـٰنِ وَفدًۭا ﴿٨٥﴾\\
\textamh{86.\  } & وَنَسُوقُ ٱلمُجرِمِينَ إِلَىٰ جَهَنَّمَ وِردًۭا ﴿٨٦﴾\\
\textamh{87.\  } & لَّا يَملِكُونَ ٱلشَّفَـٰعَةَ إِلَّا مَنِ ٱتَّخَذَ عِندَ ٱلرَّحمَـٰنِ عَهدًۭا ﴿٨٧﴾\\
\textamh{88.\  } & وَقَالُوا۟ ٱتَّخَذَ ٱلرَّحمَـٰنُ وَلَدًۭا ﴿٨٨﴾\\
\textamh{89.\  } & لَّقَد جِئتُم شَيـًٔا إِدًّۭا ﴿٨٩﴾\\
\textamh{90.\  } & تَكَادُ ٱلسَّمَـٰوَٟتُ يَتَفَطَّرنَ مِنهُ وَتَنشَقُّ ٱلأَرضُ وَتَخِرُّ ٱلجِبَالُ هَدًّا ﴿٩٠﴾\\
\textamh{91.\  } & أَن دَعَوا۟ لِلرَّحمَـٰنِ وَلَدًۭا ﴿٩١﴾\\
\textamh{92.\  } & وَمَا يَنۢبَغِى لِلرَّحمَـٰنِ أَن يَتَّخِذَ وَلَدًا ﴿٩٢﴾\\
\textamh{93.\  } & إِن كُلُّ مَن فِى ٱلسَّمَـٰوَٟتِ وَٱلأَرضِ إِلَّآ ءَاتِى ٱلرَّحمَـٰنِ عَبدًۭا ﴿٩٣﴾\\
\textamh{94.\  } & لَّقَد أَحصَىٰهُم وَعَدَّهُم عَدًّۭا ﴿٩٤﴾\\
\textamh{95.\  } & وَكُلُّهُم ءَاتِيهِ يَومَ ٱلقِيَـٰمَةِ فَردًا ﴿٩٥﴾\\
\textamh{96.\  } & إِنَّ ٱلَّذِينَ ءَامَنُوا۟ وَعَمِلُوا۟ ٱلصَّـٰلِحَـٰتِ سَيَجعَلُ لَهُمُ ٱلرَّحمَـٰنُ وُدًّۭا ﴿٩٦﴾\\
\textamh{97.\  } & فَإِنَّمَا يَسَّرنَـٰهُ بِلِسَانِكَ لِتُبَشِّرَ بِهِ ٱلمُتَّقِينَ وَتُنذِرَ بِهِۦ قَومًۭا لُّدًّۭا ﴿٩٧﴾\\
\textamh{98.\  } & وَكَم أَهلَكنَا قَبلَهُم مِّن قَرنٍ هَل تُحِسُّ مِنهُم مِّن أَحَدٍ أَو تَسمَعُ لَهُم رِكزًۢا ﴿٩٨﴾\\
\end{longtable} \newpage
