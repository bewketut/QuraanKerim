%% License: BSD style (Berkley) (i.e. Put the Copyright owner's name always)
%% Writer and Copyright (to): Bewketu(Bilal) Tadilo (2016-17)
\shadowbox{\section{\LR{\textamharic{ሱራቱ አልፉርቃን -}  \RL{سوره  الفرقان}}}}
\begin{longtable}{%
  @{}
    p{.5\textwidth}
  @{~~~~~~~~~~~~~}||
    p{.5\textwidth}
    @{}
}
\nopagebreak
\textamh{\ \ \ \ \ \  ቢስሚላሂ አራህመኒ ራሂይም } &  بِسمِ ٱللَّهِ ٱلرَّحمَـٰنِ ٱلرَّحِيمِ\\
\textamh{1.\  } &  تَبَارَكَ ٱلَّذِى نَزَّلَ ٱلفُرقَانَ عَلَىٰ عَبدِهِۦ لِيَكُونَ لِلعَـٰلَمِينَ نَذِيرًا ﴿١﴾\\
\textamh{2.\  } & ٱلَّذِى لَهُۥ مُلكُ ٱلسَّمَـٰوَٟتِ وَٱلأَرضِ وَلَم يَتَّخِذ وَلَدًۭا وَلَم يَكُن لَّهُۥ شَرِيكٌۭ فِى ٱلمُلكِ وَخَلَقَ كُلَّ شَىءٍۢ فَقَدَّرَهُۥ تَقدِيرًۭا ﴿٢﴾\\
\textamh{3.\  } & وَٱتَّخَذُوا۟ مِن دُونِهِۦٓ ءَالِهَةًۭ لَّا يَخلُقُونَ شَيـًۭٔا وَهُم يُخلَقُونَ وَلَا يَملِكُونَ لِأَنفُسِهِم ضَرًّۭا وَلَا نَفعًۭا وَلَا يَملِكُونَ مَوتًۭا وَلَا حَيَوٰةًۭ وَلَا نُشُورًۭا ﴿٣﴾\\
\textamh{4.\  } & وَقَالَ ٱلَّذِينَ كَفَرُوٓا۟ إِن هَـٰذَآ إِلَّآ إِفكٌ ٱفتَرَىٰهُ وَأَعَانَهُۥ عَلَيهِ قَومٌ ءَاخَرُونَ ۖ فَقَد جَآءُو ظُلمًۭا وَزُورًۭا ﴿٤﴾\\
\textamh{5.\  } & وَقَالُوٓا۟ أَسَـٰطِيرُ ٱلأَوَّلِينَ ٱكتَتَبَهَا فَهِىَ تُملَىٰ عَلَيهِ بُكرَةًۭ وَأَصِيلًۭا ﴿٥﴾\\
\textamh{6.\  } & قُل أَنزَلَهُ ٱلَّذِى يَعلَمُ ٱلسِّرَّ فِى ٱلسَّمَـٰوَٟتِ وَٱلأَرضِ ۚ إِنَّهُۥ كَانَ غَفُورًۭا رَّحِيمًۭا ﴿٦﴾\\
\textamh{7.\  } & وَقَالُوا۟ مَالِ هَـٰذَا ٱلرَّسُولِ يَأكُلُ ٱلطَّعَامَ وَيَمشِى فِى ٱلأَسوَاقِ ۙ لَولَآ أُنزِلَ إِلَيهِ مَلَكٌۭ فَيَكُونَ مَعَهُۥ نَذِيرًا ﴿٧﴾\\
\textamh{8.\  } & أَو يُلقَىٰٓ إِلَيهِ كَنزٌ أَو تَكُونُ لَهُۥ جَنَّةٌۭ يَأكُلُ مِنهَا ۚ وَقَالَ ٱلظَّـٰلِمُونَ إِن تَتَّبِعُونَ إِلَّا رَجُلًۭا مَّسحُورًا ﴿٨﴾\\
\textamh{9.\  } & ٱنظُر كَيفَ ضَرَبُوا۟ لَكَ ٱلأَمثَـٰلَ فَضَلُّوا۟ فَلَا يَستَطِيعُونَ سَبِيلًۭا ﴿٩﴾\\
\textamh{10.\  } & تَبَارَكَ ٱلَّذِىٓ إِن شَآءَ جَعَلَ لَكَ خَيرًۭا مِّن ذَٟلِكَ جَنَّـٰتٍۢ تَجرِى مِن تَحتِهَا ٱلأَنهَـٰرُ وَيَجعَل لَّكَ قُصُورًۢا ﴿١٠﴾\\
\textamh{11.\  } & بَل كَذَّبُوا۟ بِٱلسَّاعَةِ ۖ وَأَعتَدنَا لِمَن كَذَّبَ بِٱلسَّاعَةِ سَعِيرًا ﴿١١﴾\\
\textamh{12.\  } & إِذَا رَأَتهُم مِّن مَّكَانٍۭ بَعِيدٍۢ سَمِعُوا۟ لَهَا تَغَيُّظًۭا وَزَفِيرًۭا ﴿١٢﴾\\
\textamh{13.\  } & وَإِذَآ أُلقُوا۟ مِنهَا مَكَانًۭا ضَيِّقًۭا مُّقَرَّنِينَ دَعَوا۟ هُنَالِكَ ثُبُورًۭا ﴿١٣﴾\\
\textamh{14.\  } & لَّا تَدعُوا۟ ٱليَومَ ثُبُورًۭا وَٟحِدًۭا وَٱدعُوا۟ ثُبُورًۭا كَثِيرًۭا ﴿١٤﴾\\
\textamh{15.\  } & قُل أَذَٟلِكَ خَيرٌ أَم جَنَّةُ ٱلخُلدِ ٱلَّتِى وُعِدَ ٱلمُتَّقُونَ ۚ كَانَت لَهُم جَزَآءًۭ وَمَصِيرًۭا ﴿١٥﴾\\
\textamh{16.\  } & لَّهُم فِيهَا مَا يَشَآءُونَ خَـٰلِدِينَ ۚ كَانَ عَلَىٰ رَبِّكَ وَعدًۭا مَّسـُٔولًۭا ﴿١٦﴾\\
\textamh{17.\  } & وَيَومَ يَحشُرُهُم وَمَا يَعبُدُونَ مِن دُونِ ٱللَّهِ فَيَقُولُ ءَأَنتُم أَضلَلتُم عِبَادِى هَـٰٓؤُلَآءِ أَم هُم ضَلُّوا۟ ٱلسَّبِيلَ ﴿١٧﴾\\
\textamh{18.\  } & قَالُوا۟ سُبحَـٰنَكَ مَا كَانَ يَنۢبَغِى لَنَآ أَن نَّتَّخِذَ مِن دُونِكَ مِن أَولِيَآءَ وَلَـٰكِن مَّتَّعتَهُم وَءَابَآءَهُم حَتَّىٰ نَسُوا۟ ٱلذِّكرَ وَكَانُوا۟ قَومًۢا بُورًۭا ﴿١٨﴾\\
\textamh{19.\  } & فَقَد كَذَّبُوكُم بِمَا تَقُولُونَ فَمَا تَستَطِيعُونَ صَرفًۭا وَلَا نَصرًۭا ۚ وَمَن يَظلِم مِّنكُم نُذِقهُ عَذَابًۭا كَبِيرًۭا ﴿١٩﴾\\
\textamh{20.\  } & وَمَآ أَرسَلنَا قَبلَكَ مِنَ ٱلمُرسَلِينَ إِلَّآ إِنَّهُم لَيَأكُلُونَ ٱلطَّعَامَ وَيَمشُونَ فِى ٱلأَسوَاقِ ۗ وَجَعَلنَا بَعضَكُم لِبَعضٍۢ فِتنَةً أَتَصبِرُونَ ۗ وَكَانَ رَبُّكَ بَصِيرًۭا ﴿٢٠﴾\\
\textamh{21.\  } & ۞ وَقَالَ ٱلَّذِينَ لَا يَرجُونَ لِقَآءَنَا لَولَآ أُنزِلَ عَلَينَا ٱلمَلَـٰٓئِكَةُ أَو نَرَىٰ رَبَّنَا ۗ لَقَدِ ٱستَكبَرُوا۟ فِىٓ أَنفُسِهِم وَعَتَو عُتُوًّۭا كَبِيرًۭا ﴿٢١﴾\\
\textamh{22.\  } & يَومَ يَرَونَ ٱلمَلَـٰٓئِكَةَ لَا بُشرَىٰ يَومَئِذٍۢ لِّلمُجرِمِينَ وَيَقُولُونَ حِجرًۭا مَّحجُورًۭا ﴿٢٢﴾\\
\textamh{23.\  } & وَقَدِمنَآ إِلَىٰ مَا عَمِلُوا۟ مِن عَمَلٍۢ فَجَعَلنَـٰهُ هَبَآءًۭ مَّنثُورًا ﴿٢٣﴾\\
\textamh{24.\  } & أَصحَـٰبُ ٱلجَنَّةِ يَومَئِذٍ خَيرٌۭ مُّستَقَرًّۭا وَأَحسَنُ مَقِيلًۭا ﴿٢٤﴾\\
\textamh{25.\  } & وَيَومَ تَشَقَّقُ ٱلسَّمَآءُ بِٱلغَمَـٰمِ وَنُزِّلَ ٱلمَلَـٰٓئِكَةُ تَنزِيلًا ﴿٢٥﴾\\
\textamh{26.\  } & ٱلمُلكُ يَومَئِذٍ ٱلحَقُّ لِلرَّحمَـٰنِ ۚ وَكَانَ يَومًا عَلَى ٱلكَـٰفِرِينَ عَسِيرًۭا ﴿٢٦﴾\\
\textamh{27.\  } & وَيَومَ يَعَضُّ ٱلظَّالِمُ عَلَىٰ يَدَيهِ يَقُولُ يَـٰلَيتَنِى ٱتَّخَذتُ مَعَ ٱلرَّسُولِ سَبِيلًۭا ﴿٢٧﴾\\
\textamh{28.\  } & يَـٰوَيلَتَىٰ لَيتَنِى لَم أَتَّخِذ فُلَانًا خَلِيلًۭا ﴿٢٨﴾\\
\textamh{29.\  } & لَّقَد أَضَلَّنِى عَنِ ٱلذِّكرِ بَعدَ إِذ جَآءَنِى ۗ وَكَانَ ٱلشَّيطَٰنُ لِلإِنسَـٰنِ خَذُولًۭا ﴿٢٩﴾\\
\textamh{30.\  } & وَقَالَ ٱلرَّسُولُ يَـٰرَبِّ إِنَّ قَومِى ٱتَّخَذُوا۟ هَـٰذَا ٱلقُرءَانَ مَهجُورًۭا ﴿٣٠﴾\\
\textamh{31.\  } & وَكَذَٟلِكَ جَعَلنَا لِكُلِّ نَبِىٍّ عَدُوًّۭا مِّنَ ٱلمُجرِمِينَ ۗ وَكَفَىٰ بِرَبِّكَ هَادِيًۭا وَنَصِيرًۭا ﴿٣١﴾\\
\textamh{32.\  } & وَقَالَ ٱلَّذِينَ كَفَرُوا۟ لَولَا نُزِّلَ عَلَيهِ ٱلقُرءَانُ جُملَةًۭ وَٟحِدَةًۭ ۚ كَذَٟلِكَ لِنُثَبِّتَ بِهِۦ فُؤَادَكَ ۖ وَرَتَّلنَـٰهُ تَرتِيلًۭا ﴿٣٢﴾\\
\textamh{33.\  } & وَلَا يَأتُونَكَ بِمَثَلٍ إِلَّا جِئنَـٰكَ بِٱلحَقِّ وَأَحسَنَ تَفسِيرًا ﴿٣٣﴾\\
\textamh{34.\  } & ٱلَّذِينَ يُحشَرُونَ عَلَىٰ وُجُوهِهِم إِلَىٰ جَهَنَّمَ أُو۟لَـٰٓئِكَ شَرٌّۭ مَّكَانًۭا وَأَضَلُّ سَبِيلًۭا ﴿٣٤﴾\\
\textamh{35.\  } & وَلَقَد ءَاتَينَا مُوسَى ٱلكِتَـٰبَ وَجَعَلنَا مَعَهُۥٓ أَخَاهُ هَـٰرُونَ وَزِيرًۭا ﴿٣٥﴾\\
\textamh{36.\  } & فَقُلنَا ٱذهَبَآ إِلَى ٱلقَومِ ٱلَّذِينَ كَذَّبُوا۟ بِـَٔايَـٰتِنَا فَدَمَّرنَـٰهُم تَدمِيرًۭا ﴿٣٦﴾\\
\textamh{37.\  } & وَقَومَ نُوحٍۢ لَّمَّا كَذَّبُوا۟ ٱلرُّسُلَ أَغرَقنَـٰهُم وَجَعَلنَـٰهُم لِلنَّاسِ ءَايَةًۭ ۖ وَأَعتَدنَا لِلظَّـٰلِمِينَ عَذَابًا أَلِيمًۭا ﴿٣٧﴾\\
\textamh{38.\  } & وَعَادًۭا وَثَمُودَا۟ وَأَصحَـٰبَ ٱلرَّسِّ وَقُرُونًۢا بَينَ ذَٟلِكَ كَثِيرًۭا ﴿٣٨﴾\\
\textamh{39.\  } & وَكُلًّۭا ضَرَبنَا لَهُ ٱلأَمثَـٰلَ ۖ وَكُلًّۭا تَبَّرنَا تَتبِيرًۭا ﴿٣٩﴾\\
\textamh{40.\  } & وَلَقَد أَتَوا۟ عَلَى ٱلقَريَةِ ٱلَّتِىٓ أُمطِرَت مَطَرَ ٱلسَّوءِ ۚ أَفَلَم يَكُونُوا۟ يَرَونَهَا ۚ بَل كَانُوا۟ لَا يَرجُونَ نُشُورًۭا ﴿٤٠﴾\\
\textamh{41.\  } & وَإِذَا رَأَوكَ إِن يَتَّخِذُونَكَ إِلَّا هُزُوًا أَهَـٰذَا ٱلَّذِى بَعَثَ ٱللَّهُ رَسُولًا ﴿٤١﴾\\
\textamh{42.\  } & إِن كَادَ لَيُضِلُّنَا عَن ءَالِهَتِنَا لَولَآ أَن صَبَرنَا عَلَيهَا ۚ وَسَوفَ يَعلَمُونَ حِينَ يَرَونَ ٱلعَذَابَ مَن أَضَلُّ سَبِيلًا ﴿٤٢﴾\\
\textamh{43.\  } & أَرَءَيتَ مَنِ ٱتَّخَذَ إِلَـٰهَهُۥ هَوَىٰهُ أَفَأَنتَ تَكُونُ عَلَيهِ وَكِيلًا ﴿٤٣﴾\\
\textamh{44.\  } & أَم تَحسَبُ أَنَّ أَكثَرَهُم يَسمَعُونَ أَو يَعقِلُونَ ۚ إِن هُم إِلَّا كَٱلأَنعَـٰمِ ۖ بَل هُم أَضَلُّ سَبِيلًا ﴿٤٤﴾\\
\textamh{45.\  } & أَلَم تَرَ إِلَىٰ رَبِّكَ كَيفَ مَدَّ ٱلظِّلَّ وَلَو شَآءَ لَجَعَلَهُۥ سَاكِنًۭا ثُمَّ جَعَلنَا ٱلشَّمسَ عَلَيهِ دَلِيلًۭا ﴿٤٥﴾\\
\textamh{46.\  } & ثُمَّ قَبَضنَـٰهُ إِلَينَا قَبضًۭا يَسِيرًۭا ﴿٤٦﴾\\
\textamh{47.\  } & وَهُوَ ٱلَّذِى جَعَلَ لَكُمُ ٱلَّيلَ لِبَاسًۭا وَٱلنَّومَ سُبَاتًۭا وَجَعَلَ ٱلنَّهَارَ نُشُورًۭا ﴿٤٧﴾\\
\textamh{48.\  } & وَهُوَ ٱلَّذِىٓ أَرسَلَ ٱلرِّيَـٰحَ بُشرًۢا بَينَ يَدَى رَحمَتِهِۦ ۚ وَأَنزَلنَا مِنَ ٱلسَّمَآءِ مَآءًۭ طَهُورًۭا ﴿٤٨﴾\\
\textamh{49.\  } & لِّنُحۦِىَ بِهِۦ بَلدَةًۭ مَّيتًۭا وَنُسقِيَهُۥ مِمَّا خَلَقنَآ أَنعَـٰمًۭا وَأَنَاسِىَّ كَثِيرًۭا ﴿٤٩﴾\\
\textamh{50.\  } & وَلَقَد صَرَّفنَـٰهُ بَينَهُم لِيَذَّكَّرُوا۟ فَأَبَىٰٓ أَكثَرُ ٱلنَّاسِ إِلَّا كُفُورًۭا ﴿٥٠﴾\\
\textamh{51.\  } & وَلَو شِئنَا لَبَعَثنَا فِى كُلِّ قَريَةٍۢ نَّذِيرًۭا ﴿٥١﴾\\
\textamh{52.\  } & فَلَا تُطِعِ ٱلكَـٰفِرِينَ وَجَٰهِدهُم بِهِۦ جِهَادًۭا كَبِيرًۭا ﴿٥٢﴾\\
\textamh{53.\  } & ۞ وَهُوَ ٱلَّذِى مَرَجَ ٱلبَحرَينِ هَـٰذَا عَذبٌۭ فُرَاتٌۭ وَهَـٰذَا مِلحٌ أُجَاجٌۭ وَجَعَلَ بَينَهُمَا بَرزَخًۭا وَحِجرًۭا مَّحجُورًۭا ﴿٥٣﴾\\
\textamh{54.\  } & وَهُوَ ٱلَّذِى خَلَقَ مِنَ ٱلمَآءِ بَشَرًۭا فَجَعَلَهُۥ نَسَبًۭا وَصِهرًۭا ۗ وَكَانَ رَبُّكَ قَدِيرًۭا ﴿٥٤﴾\\
\textamh{55.\  } & وَيَعبُدُونَ مِن دُونِ ٱللَّهِ مَا لَا يَنفَعُهُم وَلَا يَضُرُّهُم ۗ وَكَانَ ٱلكَافِرُ عَلَىٰ رَبِّهِۦ ظَهِيرًۭا ﴿٥٥﴾\\
\textamh{56.\  } & وَمَآ أَرسَلنَـٰكَ إِلَّا مُبَشِّرًۭا وَنَذِيرًۭا ﴿٥٦﴾\\
\textamh{57.\  } & قُل مَآ أَسـَٔلُكُم عَلَيهِ مِن أَجرٍ إِلَّا مَن شَآءَ أَن يَتَّخِذَ إِلَىٰ رَبِّهِۦ سَبِيلًۭا ﴿٥٧﴾\\
\textamh{58.\  } & وَتَوَكَّل عَلَى ٱلحَىِّ ٱلَّذِى لَا يَمُوتُ وَسَبِّح بِحَمدِهِۦ ۚ وَكَفَىٰ بِهِۦ بِذُنُوبِ عِبَادِهِۦ خَبِيرًا ﴿٥٨﴾\\
\textamh{59.\  } & ٱلَّذِى خَلَقَ ٱلسَّمَـٰوَٟتِ وَٱلأَرضَ وَمَا بَينَهُمَا فِى سِتَّةِ أَيَّامٍۢ ثُمَّ ٱستَوَىٰ عَلَى ٱلعَرشِ ۚ ٱلرَّحمَـٰنُ فَسـَٔل بِهِۦ خَبِيرًۭا ﴿٥٩﴾\\
\textamh{60.\  } & وَإِذَا قِيلَ لَهُمُ ٱسجُدُوا۟ لِلرَّحمَـٰنِ قَالُوا۟ وَمَا ٱلرَّحمَـٰنُ أَنَسجُدُ لِمَا تَأمُرُنَا وَزَادَهُم نُفُورًۭا ۩ ﴿٦٠﴾\\
\textamh{61.\  } & تَبَارَكَ ٱلَّذِى جَعَلَ فِى ٱلسَّمَآءِ بُرُوجًۭا وَجَعَلَ فِيهَا سِرَٰجًۭا وَقَمَرًۭا مُّنِيرًۭا ﴿٦١﴾\\
\textamh{62.\  } & وَهُوَ ٱلَّذِى جَعَلَ ٱلَّيلَ وَٱلنَّهَارَ خِلفَةًۭ لِّمَن أَرَادَ أَن يَذَّكَّرَ أَو أَرَادَ شُكُورًۭا ﴿٦٢﴾\\
\textamh{63.\  } & وَعِبَادُ ٱلرَّحمَـٰنِ ٱلَّذِينَ يَمشُونَ عَلَى ٱلأَرضِ هَونًۭا وَإِذَا خَاطَبَهُمُ ٱلجَٰهِلُونَ قَالُوا۟ سَلَـٰمًۭا ﴿٦٣﴾\\
\textamh{64.\  } & وَٱلَّذِينَ يَبِيتُونَ لِرَبِّهِم سُجَّدًۭا وَقِيَـٰمًۭا ﴿٦٤﴾\\
\textamh{65.\  } & وَٱلَّذِينَ يَقُولُونَ رَبَّنَا ٱصرِف عَنَّا عَذَابَ جَهَنَّمَ ۖ إِنَّ عَذَابَهَا كَانَ غَرَامًا ﴿٦٥﴾\\
\textamh{66.\  } & إِنَّهَا سَآءَت مُستَقَرًّۭا وَمُقَامًۭا ﴿٦٦﴾\\
\textamh{67.\  } & وَٱلَّذِينَ إِذَآ أَنفَقُوا۟ لَم يُسرِفُوا۟ وَلَم يَقتُرُوا۟ وَكَانَ بَينَ ذَٟلِكَ قَوَامًۭا ﴿٦٧﴾\\
\textamh{68.\  } & وَٱلَّذِينَ لَا يَدعُونَ مَعَ ٱللَّهِ إِلَـٰهًا ءَاخَرَ وَلَا يَقتُلُونَ ٱلنَّفسَ ٱلَّتِى حَرَّمَ ٱللَّهُ إِلَّا بِٱلحَقِّ وَلَا يَزنُونَ ۚ وَمَن يَفعَل ذَٟلِكَ يَلقَ أَثَامًۭا ﴿٦٨﴾\\
\textamh{69.\  } & يُضَٰعَف لَهُ ٱلعَذَابُ يَومَ ٱلقِيَـٰمَةِ وَيَخلُد فِيهِۦ مُهَانًا ﴿٦٩﴾\\
\textamh{70.\  } & إِلَّا مَن تَابَ وَءَامَنَ وَعَمِلَ عَمَلًۭا صَـٰلِحًۭا فَأُو۟لَـٰٓئِكَ يُبَدِّلُ ٱللَّهُ سَيِّـَٔاتِهِم حَسَنَـٰتٍۢ ۗ وَكَانَ ٱللَّهُ غَفُورًۭا رَّحِيمًۭا ﴿٧٠﴾\\
\textamh{71.\  } & وَمَن تَابَ وَعَمِلَ صَـٰلِحًۭا فَإِنَّهُۥ يَتُوبُ إِلَى ٱللَّهِ مَتَابًۭا ﴿٧١﴾\\
\textamh{72.\  } & وَٱلَّذِينَ لَا يَشهَدُونَ ٱلزُّورَ وَإِذَا مَرُّوا۟ بِٱللَّغوِ مَرُّوا۟ كِرَامًۭا ﴿٧٢﴾\\
\textamh{73.\  } & وَٱلَّذِينَ إِذَا ذُكِّرُوا۟ بِـَٔايَـٰتِ رَبِّهِم لَم يَخِرُّوا۟ عَلَيهَا صُمًّۭا وَعُميَانًۭا ﴿٧٣﴾\\
\textamh{74.\  } & وَٱلَّذِينَ يَقُولُونَ رَبَّنَا هَب لَنَا مِن أَزوَٟجِنَا وَذُرِّيَّٰتِنَا قُرَّةَ أَعيُنٍۢ وَٱجعَلنَا لِلمُتَّقِينَ إِمَامًا ﴿٧٤﴾\\
\textamh{75.\  } & أُو۟لَـٰٓئِكَ يُجزَونَ ٱلغُرفَةَ بِمَا صَبَرُوا۟ وَيُلَقَّونَ فِيهَا تَحِيَّةًۭ وَسَلَـٰمًا ﴿٧٥﴾\\
\textamh{76.\  } & خَـٰلِدِينَ فِيهَا ۚ حَسُنَت مُستَقَرًّۭا وَمُقَامًۭا ﴿٧٦﴾\\
\textamh{77.\  } & قُل مَا يَعبَؤُا۟ بِكُم رَبِّى لَولَا دُعَآؤُكُم ۖ فَقَد كَذَّبتُم فَسَوفَ يَكُونُ لِزَامًۢا ﴿٧٧﴾\\
\end{longtable} \newpage
