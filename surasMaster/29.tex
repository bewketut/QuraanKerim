%% License: BSD style (Berkley) (i.e. Put the Copyright owner's name always)
%% Writer and Copyright (to): Bewketu(Bilal) Tadilo (2016-17)
\shadowbox{\section{\LR{\textamharic{ሱራቱ አልአንከቡት -}  \RL{سوره  العنكبوت}}}}
\begin{longtable}{%
  @{}
    p{.5\textwidth}
  @{~~~~~~~~~~~~~}||
    p{.5\textwidth}
    @{}
}
\nopagebreak
\textamh{\ \ \ \ \ \  ቢስሚላሂ አራህመኒ ራሂይም } &  بِسمِ ٱللَّهِ ٱلرَّحمَـٰنِ ٱلرَّحِيمِ\\
\textamh{1.\  } &  الٓمٓ ﴿١﴾\\
\textamh{2.\  } & أَحَسِبَ ٱلنَّاسُ أَن يُترَكُوٓا۟ أَن يَقُولُوٓا۟ ءَامَنَّا وَهُم لَا يُفتَنُونَ ﴿٢﴾\\
\textamh{3.\  } & وَلَقَد فَتَنَّا ٱلَّذِينَ مِن قَبلِهِم ۖ فَلَيَعلَمَنَّ ٱللَّهُ ٱلَّذِينَ صَدَقُوا۟ وَلَيَعلَمَنَّ ٱلكَـٰذِبِينَ ﴿٣﴾\\
\textamh{4.\  } & أَم حَسِبَ ٱلَّذِينَ يَعمَلُونَ ٱلسَّيِّـَٔاتِ أَن يَسبِقُونَا ۚ سَآءَ مَا يَحكُمُونَ ﴿٤﴾\\
\textamh{5.\  } & مَن كَانَ يَرجُوا۟ لِقَآءَ ٱللَّهِ فَإِنَّ أَجَلَ ٱللَّهِ لَءَاتٍۢ ۚ وَهُوَ ٱلسَّمِيعُ ٱلعَلِيمُ ﴿٥﴾\\
\textamh{6.\  } & وَمَن جَٰهَدَ فَإِنَّمَا يُجَٰهِدُ لِنَفسِهِۦٓ ۚ إِنَّ ٱللَّهَ لَغَنِىٌّ عَنِ ٱلعَـٰلَمِينَ ﴿٦﴾\\
\textamh{7.\  } & وَٱلَّذِينَ ءَامَنُوا۟ وَعَمِلُوا۟ ٱلصَّـٰلِحَـٰتِ لَنُكَفِّرَنَّ عَنهُم سَيِّـَٔاتِهِم وَلَنَجزِيَنَّهُم أَحسَنَ ٱلَّذِى كَانُوا۟ يَعمَلُونَ ﴿٧﴾\\
\textamh{8.\  } & وَوَصَّينَا ٱلإِنسَـٰنَ بِوَٟلِدَيهِ حُسنًۭا ۖ وَإِن جَٰهَدَاكَ لِتُشرِكَ بِى مَا لَيسَ لَكَ بِهِۦ عِلمٌۭ فَلَا تُطِعهُمَآ ۚ إِلَىَّ مَرجِعُكُم فَأُنَبِّئُكُم بِمَا كُنتُم تَعمَلُونَ ﴿٨﴾\\
\textamh{9.\  } & وَٱلَّذِينَ ءَامَنُوا۟ وَعَمِلُوا۟ ٱلصَّـٰلِحَـٰتِ لَنُدخِلَنَّهُم فِى ٱلصَّـٰلِحِينَ ﴿٩﴾\\
\textamh{10.\  } & وَمِنَ ٱلنَّاسِ مَن يَقُولُ ءَامَنَّا بِٱللَّهِ فَإِذَآ أُوذِىَ فِى ٱللَّهِ جَعَلَ فِتنَةَ ٱلنَّاسِ كَعَذَابِ ٱللَّهِ وَلَئِن جَآءَ نَصرٌۭ مِّن رَّبِّكَ لَيَقُولُنَّ إِنَّا كُنَّا مَعَكُم ۚ أَوَلَيسَ ٱللَّهُ بِأَعلَمَ بِمَا فِى صُدُورِ ٱلعَـٰلَمِينَ ﴿١٠﴾\\
\textamh{11.\  } & وَلَيَعلَمَنَّ ٱللَّهُ ٱلَّذِينَ ءَامَنُوا۟ وَلَيَعلَمَنَّ ٱلمُنَـٰفِقِينَ ﴿١١﴾\\
\textamh{12.\  } & وَقَالَ ٱلَّذِينَ كَفَرُوا۟ لِلَّذِينَ ءَامَنُوا۟ ٱتَّبِعُوا۟ سَبِيلَنَا وَلنَحمِل خَطَٰيَـٰكُم وَمَا هُم بِحَـٰمِلِينَ مِن خَطَٰيَـٰهُم مِّن شَىءٍ ۖ إِنَّهُم لَكَـٰذِبُونَ ﴿١٢﴾\\
\textamh{13.\  } & وَلَيَحمِلُنَّ أَثقَالَهُم وَأَثقَالًۭا مَّعَ أَثقَالِهِم ۖ وَلَيُسـَٔلُنَّ يَومَ ٱلقِيَـٰمَةِ عَمَّا كَانُوا۟ يَفتَرُونَ ﴿١٣﴾\\
\textamh{14.\  } & وَلَقَد أَرسَلنَا نُوحًا إِلَىٰ قَومِهِۦ فَلَبِثَ فِيهِم أَلفَ سَنَةٍ إِلَّا خَمسِينَ عَامًۭا فَأَخَذَهُمُ ٱلطُّوفَانُ وَهُم ظَـٰلِمُونَ ﴿١٤﴾\\
\textamh{15.\  } & فَأَنجَينَـٰهُ وَأَصحَـٰبَ ٱلسَّفِينَةِ وَجَعَلنَـٰهَآ ءَايَةًۭ لِّلعَـٰلَمِينَ ﴿١٥﴾\\
\textamh{16.\  } & وَإِبرَٰهِيمَ إِذ قَالَ لِقَومِهِ ٱعبُدُوا۟ ٱللَّهَ وَٱتَّقُوهُ ۖ ذَٟلِكُم خَيرٌۭ لَّكُم إِن كُنتُم تَعلَمُونَ ﴿١٦﴾\\
\textamh{17.\  } & إِنَّمَا تَعبُدُونَ مِن دُونِ ٱللَّهِ أَوثَـٰنًۭا وَتَخلُقُونَ إِفكًا ۚ إِنَّ ٱلَّذِينَ تَعبُدُونَ مِن دُونِ ٱللَّهِ لَا يَملِكُونَ لَكُم رِزقًۭا فَٱبتَغُوا۟ عِندَ ٱللَّهِ ٱلرِّزقَ وَٱعبُدُوهُ وَٱشكُرُوا۟ لَهُۥٓ ۖ إِلَيهِ تُرجَعُونَ ﴿١٧﴾\\
\textamh{18.\  } & وَإِن تُكَذِّبُوا۟ فَقَد كَذَّبَ أُمَمٌۭ مِّن قَبلِكُم ۖ وَمَا عَلَى ٱلرَّسُولِ إِلَّا ٱلبَلَـٰغُ ٱلمُبِينُ ﴿١٨﴾\\
\textamh{19.\  } & أَوَلَم يَرَوا۟ كَيفَ يُبدِئُ ٱللَّهُ ٱلخَلقَ ثُمَّ يُعِيدُهُۥٓ ۚ إِنَّ ذَٟلِكَ عَلَى ٱللَّهِ يَسِيرٌۭ ﴿١٩﴾\\
\textamh{20.\  } & قُل سِيرُوا۟ فِى ٱلأَرضِ فَٱنظُرُوا۟ كَيفَ بَدَأَ ٱلخَلقَ ۚ ثُمَّ ٱللَّهُ يُنشِئُ ٱلنَّشأَةَ ٱلءَاخِرَةَ ۚ إِنَّ ٱللَّهَ عَلَىٰ كُلِّ شَىءٍۢ قَدِيرٌۭ ﴿٢٠﴾\\
\textamh{21.\  } & يُعَذِّبُ مَن يَشَآءُ وَيَرحَمُ مَن يَشَآءُ ۖ وَإِلَيهِ تُقلَبُونَ ﴿٢١﴾\\
\textamh{22.\  } & وَمَآ أَنتُم بِمُعجِزِينَ فِى ٱلأَرضِ وَلَا فِى ٱلسَّمَآءِ ۖ وَمَا لَكُم مِّن دُونِ ٱللَّهِ مِن وَلِىٍّۢ وَلَا نَصِيرٍۢ ﴿٢٢﴾\\
\textamh{23.\  } & وَٱلَّذِينَ كَفَرُوا۟ بِـَٔايَـٰتِ ٱللَّهِ وَلِقَآئِهِۦٓ أُو۟لَـٰٓئِكَ يَئِسُوا۟ مِن رَّحمَتِى وَأُو۟لَـٰٓئِكَ لَهُم عَذَابٌ أَلِيمٌۭ ﴿٢٣﴾\\
\textamh{24.\  } & فَمَا كَانَ جَوَابَ قَومِهِۦٓ إِلَّآ أَن قَالُوا۟ ٱقتُلُوهُ أَو حَرِّقُوهُ فَأَنجَىٰهُ ٱللَّهُ مِنَ ٱلنَّارِ ۚ إِنَّ فِى ذَٟلِكَ لَءَايَـٰتٍۢ لِّقَومٍۢ يُؤمِنُونَ ﴿٢٤﴾\\
\textamh{25.\  } & وَقَالَ إِنَّمَا ٱتَّخَذتُم مِّن دُونِ ٱللَّهِ أَوثَـٰنًۭا مَّوَدَّةَ بَينِكُم فِى ٱلحَيَوٰةِ ٱلدُّنيَا ۖ ثُمَّ يَومَ ٱلقِيَـٰمَةِ يَكفُرُ بَعضُكُم بِبَعضٍۢ وَيَلعَنُ بَعضُكُم بَعضًۭا وَمَأوَىٰكُمُ ٱلنَّارُ وَمَا لَكُم مِّن نَّـٰصِرِينَ ﴿٢٥﴾\\
\textamh{26.\  } & ۞ فَـَٔامَنَ لَهُۥ لُوطٌۭ ۘ وَقَالَ إِنِّى مُهَاجِرٌ إِلَىٰ رَبِّىٓ ۖ إِنَّهُۥ هُوَ ٱلعَزِيزُ ٱلحَكِيمُ ﴿٢٦﴾\\
\textamh{27.\  } & وَوَهَبنَا لَهُۥٓ إِسحَـٰقَ وَيَعقُوبَ وَجَعَلنَا فِى ذُرِّيَّتِهِ ٱلنُّبُوَّةَ وَٱلكِتَـٰبَ وَءَاتَينَـٰهُ أَجرَهُۥ فِى ٱلدُّنيَا ۖ وَإِنَّهُۥ فِى ٱلءَاخِرَةِ لَمِنَ ٱلصَّـٰلِحِينَ ﴿٢٧﴾\\
\textamh{28.\  } & وَلُوطًا إِذ قَالَ لِقَومِهِۦٓ إِنَّكُم لَتَأتُونَ ٱلفَـٰحِشَةَ مَا سَبَقَكُم بِهَا مِن أَحَدٍۢ مِّنَ ٱلعَـٰلَمِينَ ﴿٢٨﴾\\
\textamh{29.\  } & أَئِنَّكُم لَتَأتُونَ ٱلرِّجَالَ وَتَقطَعُونَ ٱلسَّبِيلَ وَتَأتُونَ فِى نَادِيكُمُ ٱلمُنكَرَ ۖ فَمَا كَانَ جَوَابَ قَومِهِۦٓ إِلَّآ أَن قَالُوا۟ ٱئتِنَا بِعَذَابِ ٱللَّهِ إِن كُنتَ مِنَ ٱلصَّـٰدِقِينَ ﴿٢٩﴾\\
\textamh{30.\  } & قَالَ رَبِّ ٱنصُرنِى عَلَى ٱلقَومِ ٱلمُفسِدِينَ ﴿٣٠﴾\\
\textamh{31.\  } & وَلَمَّا جَآءَت رُسُلُنَآ إِبرَٰهِيمَ بِٱلبُشرَىٰ قَالُوٓا۟ إِنَّا مُهلِكُوٓا۟ أَهلِ هَـٰذِهِ ٱلقَريَةِ ۖ إِنَّ أَهلَهَا كَانُوا۟ ظَـٰلِمِينَ ﴿٣١﴾\\
\textamh{32.\  } & قَالَ إِنَّ فِيهَا لُوطًۭا ۚ قَالُوا۟ نَحنُ أَعلَمُ بِمَن فِيهَا ۖ لَنُنَجِّيَنَّهُۥ وَأَهلَهُۥٓ إِلَّا ٱمرَأَتَهُۥ كَانَت مِنَ ٱلغَٰبِرِينَ ﴿٣٢﴾\\
\textamh{33.\  } & وَلَمَّآ أَن جَآءَت رُسُلُنَا لُوطًۭا سِىٓءَ بِهِم وَضَاقَ بِهِم ذَرعًۭا وَقَالُوا۟ لَا تَخَف وَلَا تَحزَن ۖ إِنَّا مُنَجُّوكَ وَأَهلَكَ إِلَّا ٱمرَأَتَكَ كَانَت مِنَ ٱلغَٰبِرِينَ ﴿٣٣﴾\\
\textamh{34.\  } & إِنَّا مُنزِلُونَ عَلَىٰٓ أَهلِ هَـٰذِهِ ٱلقَريَةِ رِجزًۭا مِّنَ ٱلسَّمَآءِ بِمَا كَانُوا۟ يَفسُقُونَ ﴿٣٤﴾\\
\textamh{35.\  } & وَلَقَد تَّرَكنَا مِنهَآ ءَايَةًۢ بَيِّنَةًۭ لِّقَومٍۢ يَعقِلُونَ ﴿٣٥﴾\\
\textamh{36.\  } & وَإِلَىٰ مَديَنَ أَخَاهُم شُعَيبًۭا فَقَالَ يَـٰقَومِ ٱعبُدُوا۟ ٱللَّهَ وَٱرجُوا۟ ٱليَومَ ٱلءَاخِرَ وَلَا تَعثَوا۟ فِى ٱلأَرضِ مُفسِدِينَ ﴿٣٦﴾\\
\textamh{37.\  } & فَكَذَّبُوهُ فَأَخَذَتهُمُ ٱلرَّجفَةُ فَأَصبَحُوا۟ فِى دَارِهِم جَٰثِمِينَ ﴿٣٧﴾\\
\textamh{38.\  } & وَعَادًۭا وَثَمُودَا۟ وَقَد تَّبَيَّنَ لَكُم مِّن مَّسَـٰكِنِهِم ۖ وَزَيَّنَ لَهُمُ ٱلشَّيطَٰنُ أَعمَـٰلَهُم فَصَدَّهُم عَنِ ٱلسَّبِيلِ وَكَانُوا۟ مُستَبصِرِينَ ﴿٣٨﴾\\
\textamh{39.\  } & وَقَـٰرُونَ وَفِرعَونَ وَهَـٰمَـٰنَ ۖ وَلَقَد جَآءَهُم مُّوسَىٰ بِٱلبَيِّنَـٰتِ فَٱستَكبَرُوا۟ فِى ٱلأَرضِ وَمَا كَانُوا۟ سَـٰبِقِينَ ﴿٣٩﴾\\
\textamh{40.\  } & فَكُلًّا أَخَذنَا بِذَنۢبِهِۦ ۖ فَمِنهُم مَّن أَرسَلنَا عَلَيهِ حَاصِبًۭا وَمِنهُم مَّن أَخَذَتهُ ٱلصَّيحَةُ وَمِنهُم مَّن خَسَفنَا بِهِ ٱلأَرضَ وَمِنهُم مَّن أَغرَقنَا ۚ وَمَا كَانَ ٱللَّهُ لِيَظلِمَهُم وَلَـٰكِن كَانُوٓا۟ أَنفُسَهُم يَظلِمُونَ ﴿٤٠﴾\\
\textamh{41.\  } & مَثَلُ ٱلَّذِينَ ٱتَّخَذُوا۟ مِن دُونِ ٱللَّهِ أَولِيَآءَ كَمَثَلِ ٱلعَنكَبُوتِ ٱتَّخَذَت بَيتًۭا ۖ وَإِنَّ أَوهَنَ ٱلبُيُوتِ لَبَيتُ ٱلعَنكَبُوتِ ۖ لَو كَانُوا۟ يَعلَمُونَ ﴿٤١﴾\\
\textamh{42.\  } & إِنَّ ٱللَّهَ يَعلَمُ مَا يَدعُونَ مِن دُونِهِۦ مِن شَىءٍۢ ۚ وَهُوَ ٱلعَزِيزُ ٱلحَكِيمُ ﴿٤٢﴾\\
\textamh{43.\  } & وَتِلكَ ٱلأَمثَـٰلُ نَضرِبُهَا لِلنَّاسِ ۖ وَمَا يَعقِلُهَآ إِلَّا ٱلعَـٰلِمُونَ ﴿٤٣﴾\\
\textamh{44.\  } & خَلَقَ ٱللَّهُ ٱلسَّمَـٰوَٟتِ وَٱلأَرضَ بِٱلحَقِّ ۚ إِنَّ فِى ذَٟلِكَ لَءَايَةًۭ لِّلمُؤمِنِينَ ﴿٤٤﴾\\
\textamh{45.\  } & ٱتلُ مَآ أُوحِىَ إِلَيكَ مِنَ ٱلكِتَـٰبِ وَأَقِمِ ٱلصَّلَوٰةَ ۖ إِنَّ ٱلصَّلَوٰةَ تَنهَىٰ عَنِ ٱلفَحشَآءِ وَٱلمُنكَرِ ۗ وَلَذِكرُ ٱللَّهِ أَكبَرُ ۗ وَٱللَّهُ يَعلَمُ مَا تَصنَعُونَ ﴿٤٥﴾\\
\textamh{46.\  } & ۞ وَلَا تُجَٰدِلُوٓا۟ أَهلَ ٱلكِتَـٰبِ إِلَّا بِٱلَّتِى هِىَ أَحسَنُ إِلَّا ٱلَّذِينَ ظَلَمُوا۟ مِنهُم ۖ وَقُولُوٓا۟ ءَامَنَّا بِٱلَّذِىٓ أُنزِلَ إِلَينَا وَأُنزِلَ إِلَيكُم وَإِلَـٰهُنَا وَإِلَـٰهُكُم وَٟحِدٌۭ وَنَحنُ لَهُۥ مُسلِمُونَ ﴿٤٦﴾\\
\textamh{47.\  } & وَكَذَٟلِكَ أَنزَلنَآ إِلَيكَ ٱلكِتَـٰبَ ۚ فَٱلَّذِينَ ءَاتَينَـٰهُمُ ٱلكِتَـٰبَ يُؤمِنُونَ بِهِۦ ۖ وَمِن هَـٰٓؤُلَآءِ مَن يُؤمِنُ بِهِۦ ۚ وَمَا يَجحَدُ بِـَٔايَـٰتِنَآ إِلَّا ٱلكَـٰفِرُونَ ﴿٤٧﴾\\
\textamh{48.\  } & وَمَا كُنتَ تَتلُوا۟ مِن قَبلِهِۦ مِن كِتَـٰبٍۢ وَلَا تَخُطُّهُۥ بِيَمِينِكَ ۖ إِذًۭا لَّٱرتَابَ ٱلمُبطِلُونَ ﴿٤٨﴾\\
\textamh{49.\  } & بَل هُوَ ءَايَـٰتٌۢ بَيِّنَـٰتٌۭ فِى صُدُورِ ٱلَّذِينَ أُوتُوا۟ ٱلعِلمَ ۚ وَمَا يَجحَدُ بِـَٔايَـٰتِنَآ إِلَّا ٱلظَّـٰلِمُونَ ﴿٤٩﴾\\
\textamh{50.\  } & وَقَالُوا۟ لَولَآ أُنزِلَ عَلَيهِ ءَايَـٰتٌۭ مِّن رَّبِّهِۦ ۖ قُل إِنَّمَا ٱلءَايَـٰتُ عِندَ ٱللَّهِ وَإِنَّمَآ أَنَا۠ نَذِيرٌۭ مُّبِينٌ ﴿٥٠﴾\\
\textamh{51.\  } & أَوَلَم يَكفِهِم أَنَّآ أَنزَلنَا عَلَيكَ ٱلكِتَـٰبَ يُتلَىٰ عَلَيهِم ۚ إِنَّ فِى ذَٟلِكَ لَرَحمَةًۭ وَذِكرَىٰ لِقَومٍۢ يُؤمِنُونَ ﴿٥١﴾\\
\textamh{52.\  } & قُل كَفَىٰ بِٱللَّهِ بَينِى وَبَينَكُم شَهِيدًۭا ۖ يَعلَمُ مَا فِى ٱلسَّمَـٰوَٟتِ وَٱلأَرضِ ۗ وَٱلَّذِينَ ءَامَنُوا۟ بِٱلبَٰطِلِ وَكَفَرُوا۟ بِٱللَّهِ أُو۟لَـٰٓئِكَ هُمُ ٱلخَـٰسِرُونَ ﴿٥٢﴾\\
\textamh{53.\  } & وَيَستَعجِلُونَكَ بِٱلعَذَابِ ۚ وَلَولَآ أَجَلٌۭ مُّسَمًّۭى لَّجَآءَهُمُ ٱلعَذَابُ وَلَيَأتِيَنَّهُم بَغتَةًۭ وَهُم لَا يَشعُرُونَ ﴿٥٣﴾\\
\textamh{54.\  } & يَستَعجِلُونَكَ بِٱلعَذَابِ وَإِنَّ جَهَنَّمَ لَمُحِيطَةٌۢ بِٱلكَـٰفِرِينَ ﴿٥٤﴾\\
\textamh{55.\  } & يَومَ يَغشَىٰهُمُ ٱلعَذَابُ مِن فَوقِهِم وَمِن تَحتِ أَرجُلِهِم وَيَقُولُ ذُوقُوا۟ مَا كُنتُم تَعمَلُونَ ﴿٥٥﴾\\
\textamh{56.\  } & يَـٰعِبَادِىَ ٱلَّذِينَ ءَامَنُوٓا۟ إِنَّ أَرضِى وَٟسِعَةٌۭ فَإِيَّٰىَ فَٱعبُدُونِ ﴿٥٦﴾\\
\textamh{57.\  } & كُلُّ نَفسٍۢ ذَآئِقَةُ ٱلمَوتِ ۖ ثُمَّ إِلَينَا تُرجَعُونَ ﴿٥٧﴾\\
\textamh{58.\  } & وَٱلَّذِينَ ءَامَنُوا۟ وَعَمِلُوا۟ ٱلصَّـٰلِحَـٰتِ لَنُبَوِّئَنَّهُم مِّنَ ٱلجَنَّةِ غُرَفًۭا تَجرِى مِن تَحتِهَا ٱلأَنهَـٰرُ خَـٰلِدِينَ فِيهَا ۚ نِعمَ أَجرُ ٱلعَـٰمِلِينَ ﴿٥٨﴾\\
\textamh{59.\  } & ٱلَّذِينَ صَبَرُوا۟ وَعَلَىٰ رَبِّهِم يَتَوَكَّلُونَ ﴿٥٩﴾\\
\textamh{60.\  } & وَكَأَيِّن مِّن دَآبَّةٍۢ لَّا تَحمِلُ رِزقَهَا ٱللَّهُ يَرزُقُهَا وَإِيَّاكُم ۚ وَهُوَ ٱلسَّمِيعُ ٱلعَلِيمُ ﴿٦٠﴾\\
\textamh{61.\  } & وَلَئِن سَأَلتَهُم مَّن خَلَقَ ٱلسَّمَـٰوَٟتِ وَٱلأَرضَ وَسَخَّرَ ٱلشَّمسَ وَٱلقَمَرَ لَيَقُولُنَّ ٱللَّهُ ۖ فَأَنَّىٰ يُؤفَكُونَ ﴿٦١﴾\\
\textamh{62.\  } & ٱللَّهُ يَبسُطُ ٱلرِّزقَ لِمَن يَشَآءُ مِن عِبَادِهِۦ وَيَقدِرُ لَهُۥٓ ۚ إِنَّ ٱللَّهَ بِكُلِّ شَىءٍ عَلِيمٌۭ ﴿٦٢﴾\\
\textamh{63.\  } & وَلَئِن سَأَلتَهُم مَّن نَّزَّلَ مِنَ ٱلسَّمَآءِ مَآءًۭ فَأَحيَا بِهِ ٱلأَرضَ مِنۢ بَعدِ مَوتِهَا لَيَقُولُنَّ ٱللَّهُ ۚ قُلِ ٱلحَمدُ لِلَّهِ ۚ بَل أَكثَرُهُم لَا يَعقِلُونَ ﴿٦٣﴾\\
\textamh{64.\  } & وَمَا هَـٰذِهِ ٱلحَيَوٰةُ ٱلدُّنيَآ إِلَّا لَهوٌۭ وَلَعِبٌۭ ۚ وَإِنَّ ٱلدَّارَ ٱلءَاخِرَةَ لَهِىَ ٱلحَيَوَانُ ۚ لَو كَانُوا۟ يَعلَمُونَ ﴿٦٤﴾\\
\textamh{65.\  } & فَإِذَا رَكِبُوا۟ فِى ٱلفُلكِ دَعَوُا۟ ٱللَّهَ مُخلِصِينَ لَهُ ٱلدِّينَ فَلَمَّا نَجَّىٰهُم إِلَى ٱلبَرِّ إِذَا هُم يُشرِكُونَ ﴿٦٥﴾\\
\textamh{66.\  } & لِيَكفُرُوا۟ بِمَآ ءَاتَينَـٰهُم وَلِيَتَمَتَّعُوا۟ ۖ فَسَوفَ يَعلَمُونَ ﴿٦٦﴾\\
\textamh{67.\  } & أَوَلَم يَرَوا۟ أَنَّا جَعَلنَا حَرَمًا ءَامِنًۭا وَيُتَخَطَّفُ ٱلنَّاسُ مِن حَولِهِم ۚ أَفَبِٱلبَٰطِلِ يُؤمِنُونَ وَبِنِعمَةِ ٱللَّهِ يَكفُرُونَ ﴿٦٧﴾\\
\textamh{68.\  } & وَمَن أَظلَمُ مِمَّنِ ٱفتَرَىٰ عَلَى ٱللَّهِ كَذِبًا أَو كَذَّبَ بِٱلحَقِّ لَمَّا جَآءَهُۥٓ ۚ أَلَيسَ فِى جَهَنَّمَ مَثوًۭى لِّلكَـٰفِرِينَ ﴿٦٨﴾\\
\textamh{69.\  } & وَٱلَّذِينَ جَٰهَدُوا۟ فِينَا لَنَهدِيَنَّهُم سُبُلَنَا ۚ وَإِنَّ ٱللَّهَ لَمَعَ ٱلمُحسِنِينَ ﴿٦٩﴾\\
\end{longtable} \newpage
