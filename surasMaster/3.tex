%% License: BSD style (Berkley) (i.e. Put the Copyright owner's name always)
%% Writer and Copyright (to): Bewketu(Bilal) Tadilo (2016-17)
\shadowbox{\section{\LR{\textamharic{ሱራቱ አልኢምራን -}  \RL{سوره  عمران}}}}
\begin{longtable}{%
  @{}
    p{.5\textwidth}
  @{~~~~~~~~~~~~~}||
    p{.5\textwidth}
    @{}
}
\nopagebreak
\textamh{\ \ \ \ \ \  ቢስሚላሂ አራህመኒ ራሂይም } &  بِسمِ ٱللَّهِ ٱلرَّحمَـٰنِ ٱلرَّحِيمِ\\
\textamh{1.\  } &  الٓمٓ ﴿١﴾\\
\textamh{2.\  } & ٱللَّهُ لَآ إِلَـٰهَ إِلَّا هُوَ ٱلحَىُّ ٱلقَيُّومُ ﴿٢﴾\\
\textamh{3.\  } & نَزَّلَ عَلَيكَ ٱلكِتَـٰبَ بِٱلحَقِّ مُصَدِّقًۭا لِّمَا بَينَ يَدَيهِ وَأَنزَلَ ٱلتَّورَىٰةَ وَٱلإِنجِيلَ ﴿٣﴾\\
\textamh{4.\  } & مِن قَبلُ هُدًۭى لِّلنَّاسِ وَأَنزَلَ ٱلفُرقَانَ ۗ إِنَّ ٱلَّذِينَ كَفَرُوا۟ بِـَٔايَـٰتِ ٱللَّهِ لَهُم عَذَابٌۭ شَدِيدٌۭ ۗ وَٱللَّهُ عَزِيزٌۭ ذُو ٱنتِقَامٍ ﴿٤﴾\\
\textamh{5.\  } & إِنَّ ٱللَّهَ لَا يَخفَىٰ عَلَيهِ شَىءٌۭ فِى ٱلأَرضِ وَلَا فِى ٱلسَّمَآءِ ﴿٥﴾\\
\textamh{6.\  } & هُوَ ٱلَّذِى يُصَوِّرُكُم فِى ٱلأَرحَامِ كَيفَ يَشَآءُ ۚ لَآ إِلَـٰهَ إِلَّا هُوَ ٱلعَزِيزُ ٱلحَكِيمُ ﴿٦﴾\\
\textamh{7.\  } & هُوَ ٱلَّذِىٓ أَنزَلَ عَلَيكَ ٱلكِتَـٰبَ مِنهُ ءَايَـٰتٌۭ مُّحكَمَـٰتٌ هُنَّ أُمُّ ٱلكِتَـٰبِ وَأُخَرُ مُتَشَـٰبِهَـٰتٌۭ ۖ فَأَمَّا ٱلَّذِينَ فِى قُلُوبِهِم زَيغٌۭ فَيَتَّبِعُونَ مَا تَشَـٰبَهَ مِنهُ ٱبتِغَآءَ ٱلفِتنَةِ وَٱبتِغَآءَ تَأوِيلِهِۦ ۗ وَمَا يَعلَمُ تَأوِيلَهُۥٓ إِلَّا ٱللَّهُ ۗ وَٱلرَّٟسِخُونَ فِى ٱلعِلمِ يَقُولُونَ ءَامَنَّا بِهِۦ كُلٌّۭ مِّن عِندِ رَبِّنَا ۗ وَمَا يَذَّكَّرُ إِلَّآ أُو۟لُوا۟ ٱلأَلبَٰبِ ﴿٧﴾\\
\textamh{8.\  } & رَبَّنَا لَا تُزِغ قُلُوبَنَا بَعدَ إِذ هَدَيتَنَا وَهَب لَنَا مِن لَّدُنكَ رَحمَةً ۚ إِنَّكَ أَنتَ ٱلوَهَّابُ ﴿٨﴾\\
\textamh{9.\  } & رَبَّنَآ إِنَّكَ جَامِعُ ٱلنَّاسِ لِيَومٍۢ لَّا رَيبَ فِيهِ ۚ إِنَّ ٱللَّهَ لَا يُخلِفُ ٱلمِيعَادَ ﴿٩﴾\\
\textamh{10.\  } & إِنَّ ٱلَّذِينَ كَفَرُوا۟ لَن تُغنِىَ عَنهُم أَموَٟلُهُم وَلَآ أَولَـٰدُهُم مِّنَ ٱللَّهِ شَيـًۭٔا ۖ وَأُو۟لَـٰٓئِكَ هُم وَقُودُ ٱلنَّارِ ﴿١٠﴾\\
\textamh{11.\  } & كَدَأبِ ءَالِ فِرعَونَ وَٱلَّذِينَ مِن قَبلِهِم ۚ كَذَّبُوا۟ بِـَٔايَـٰتِنَا فَأَخَذَهُمُ ٱللَّهُ بِذُنُوبِهِم ۗ وَٱللَّهُ شَدِيدُ ٱلعِقَابِ ﴿١١﴾\\
\textamh{12.\  } & قُل لِّلَّذِينَ كَفَرُوا۟ سَتُغلَبُونَ وَتُحشَرُونَ إِلَىٰ جَهَنَّمَ ۚ وَبِئسَ ٱلمِهَادُ ﴿١٢﴾\\
\textamh{13.\  } & قَد كَانَ لَكُم ءَايَةٌۭ فِى فِئَتَينِ ٱلتَقَتَا ۖ فِئَةٌۭ تُقَـٰتِلُ فِى سَبِيلِ ٱللَّهِ وَأُخرَىٰ كَافِرَةٌۭ يَرَونَهُم مِّثلَيهِم رَأىَ ٱلعَينِ ۚ وَٱللَّهُ يُؤَيِّدُ بِنَصرِهِۦ مَن يَشَآءُ ۗ إِنَّ فِى ذَٟلِكَ لَعِبرَةًۭ لِّأُو۟لِى ٱلأَبصَـٰرِ ﴿١٣﴾\\
\textamh{14.\  } & زُيِّنَ لِلنَّاسِ حُبُّ ٱلشَّهَوَٟتِ مِنَ ٱلنِّسَآءِ وَٱلبَنِينَ وَٱلقَنَـٰطِيرِ ٱلمُقَنطَرَةِ مِنَ ٱلذَّهَبِ وَٱلفِضَّةِ وَٱلخَيلِ ٱلمُسَوَّمَةِ وَٱلأَنعَـٰمِ وَٱلحَرثِ ۗ ذَٟلِكَ مَتَـٰعُ ٱلحَيَوٰةِ ٱلدُّنيَا ۖ وَٱللَّهُ عِندَهُۥ حُسنُ ٱلمَـَٔابِ ﴿١٤﴾\\
\textamh{15.\  } & ۞ قُل أَؤُنَبِّئُكُم بِخَيرٍۢ مِّن ذَٟلِكُم ۚ لِلَّذِينَ ٱتَّقَوا۟ عِندَ رَبِّهِم جَنَّـٰتٌۭ تَجرِى مِن تَحتِهَا ٱلأَنهَـٰرُ خَـٰلِدِينَ فِيهَا وَأَزوَٟجٌۭ مُّطَهَّرَةٌۭ وَرِضوَٟنٌۭ مِّنَ ٱللَّهِ ۗ وَٱللَّهُ بَصِيرٌۢ بِٱلعِبَادِ ﴿١٥﴾\\
\textamh{16.\  } & ٱلَّذِينَ يَقُولُونَ رَبَّنَآ إِنَّنَآ ءَامَنَّا فَٱغفِر لَنَا ذُنُوبَنَا وَقِنَا عَذَابَ ٱلنَّارِ ﴿١٦﴾\\
\textamh{17.\  } & ٱلصَّـٰبِرِينَ وَٱلصَّـٰدِقِينَ وَٱلقَـٰنِتِينَ وَٱلمُنفِقِينَ وَٱلمُستَغفِرِينَ بِٱلأَسحَارِ ﴿١٧﴾\\
\textamh{18.\  } & شَهِدَ ٱللَّهُ أَنَّهُۥ لَآ إِلَـٰهَ إِلَّا هُوَ وَٱلمَلَـٰٓئِكَةُ وَأُو۟لُوا۟ ٱلعِلمِ قَآئِمًۢا بِٱلقِسطِ ۚ لَآ إِلَـٰهَ إِلَّا هُوَ ٱلعَزِيزُ ٱلحَكِيمُ ﴿١٨﴾\\
\textamh{19.\  } & إِنَّ ٱلدِّينَ عِندَ ٱللَّهِ ٱلإِسلَـٰمُ ۗ وَمَا ٱختَلَفَ ٱلَّذِينَ أُوتُوا۟ ٱلكِتَـٰبَ إِلَّا مِنۢ بَعدِ مَا جَآءَهُمُ ٱلعِلمُ بَغيًۢا بَينَهُم ۗ وَمَن يَكفُر بِـَٔايَـٰتِ ٱللَّهِ فَإِنَّ ٱللَّهَ سَرِيعُ ٱلحِسَابِ ﴿١٩﴾\\
\textamh{20.\  } & فَإِن حَآجُّوكَ فَقُل أَسلَمتُ وَجهِىَ لِلَّهِ وَمَنِ ٱتَّبَعَنِ ۗ وَقُل لِّلَّذِينَ أُوتُوا۟ ٱلكِتَـٰبَ وَٱلأُمِّيِّۦنَ ءَأَسلَمتُم ۚ فَإِن أَسلَمُوا۟ فَقَدِ ٱهتَدَوا۟ ۖ وَّإِن تَوَلَّوا۟ فَإِنَّمَا عَلَيكَ ٱلبَلَـٰغُ ۗ وَٱللَّهُ بَصِيرٌۢ بِٱلعِبَادِ ﴿٢٠﴾\\
\textamh{21.\  } & إِنَّ ٱلَّذِينَ يَكفُرُونَ بِـَٔايَـٰتِ ٱللَّهِ وَيَقتُلُونَ ٱلنَّبِيِّۦنَ بِغَيرِ حَقٍّۢ وَيَقتُلُونَ ٱلَّذِينَ يَأمُرُونَ بِٱلقِسطِ مِنَ ٱلنَّاسِ فَبَشِّرهُم بِعَذَابٍ أَلِيمٍ ﴿٢١﴾\\
\textamh{22.\  } & أُو۟لَـٰٓئِكَ ٱلَّذِينَ حَبِطَت أَعمَـٰلُهُم فِى ٱلدُّنيَا وَٱلءَاخِرَةِ وَمَا لَهُم مِّن نَّـٰصِرِينَ ﴿٢٢﴾\\
\textamh{23.\  } & أَلَم تَرَ إِلَى ٱلَّذِينَ أُوتُوا۟ نَصِيبًۭا مِّنَ ٱلكِتَـٰبِ يُدعَونَ إِلَىٰ كِتَـٰبِ ٱللَّهِ لِيَحكُمَ بَينَهُم ثُمَّ يَتَوَلَّىٰ فَرِيقٌۭ مِّنهُم وَهُم مُّعرِضُونَ ﴿٢٣﴾\\
\textamh{24.\  } & ذَٟلِكَ بِأَنَّهُم قَالُوا۟ لَن تَمَسَّنَا ٱلنَّارُ إِلَّآ أَيَّامًۭا مَّعدُودَٟتٍۢ ۖ وَغَرَّهُم فِى دِينِهِم مَّا كَانُوا۟ يَفتَرُونَ ﴿٢٤﴾\\
\textamh{25.\  } & فَكَيفَ إِذَا جَمَعنَـٰهُم لِيَومٍۢ لَّا رَيبَ فِيهِ وَوُفِّيَت كُلُّ نَفسٍۢ مَّا كَسَبَت وَهُم لَا يُظلَمُونَ ﴿٢٥﴾\\
\textamh{26.\  } & قُلِ ٱللَّهُمَّ مَـٰلِكَ ٱلمُلكِ تُؤتِى ٱلمُلكَ مَن تَشَآءُ وَتَنزِعُ ٱلمُلكَ مِمَّن تَشَآءُ وَتُعِزُّ مَن تَشَآءُ وَتُذِلُّ مَن تَشَآءُ ۖ بِيَدِكَ ٱلخَيرُ ۖ إِنَّكَ عَلَىٰ كُلِّ شَىءٍۢ قَدِيرٌۭ ﴿٢٦﴾\\
\textamh{27.\  } & تُولِجُ ٱلَّيلَ فِى ٱلنَّهَارِ وَتُولِجُ ٱلنَّهَارَ فِى ٱلَّيلِ ۖ وَتُخرِجُ ٱلحَىَّ مِنَ ٱلمَيِّتِ وَتُخرِجُ ٱلمَيِّتَ مِنَ ٱلحَىِّ ۖ وَتَرزُقُ مَن تَشَآءُ بِغَيرِ حِسَابٍۢ ﴿٢٧﴾\\
\textamh{28.\  } & لَّا يَتَّخِذِ ٱلمُؤمِنُونَ ٱلكَـٰفِرِينَ أَولِيَآءَ مِن دُونِ ٱلمُؤمِنِينَ ۖ وَمَن يَفعَل ذَٟلِكَ فَلَيسَ مِنَ ٱللَّهِ فِى شَىءٍ إِلَّآ أَن تَتَّقُوا۟ مِنهُم تُقَىٰةًۭ ۗ وَيُحَذِّرُكُمُ ٱللَّهُ نَفسَهُۥ ۗ وَإِلَى ٱللَّهِ ٱلمَصِيرُ ﴿٢٨﴾\\
\textamh{29.\  } & قُل إِن تُخفُوا۟ مَا فِى صُدُورِكُم أَو تُبدُوهُ يَعلَمهُ ٱللَّهُ ۗ وَيَعلَمُ مَا فِى ٱلسَّمَـٰوَٟتِ وَمَا فِى ٱلأَرضِ ۗ وَٱللَّهُ عَلَىٰ كُلِّ شَىءٍۢ قَدِيرٌۭ ﴿٢٩﴾\\
\textamh{30.\  } & يَومَ تَجِدُ كُلُّ نَفسٍۢ مَّا عَمِلَت مِن خَيرٍۢ مُّحضَرًۭا وَمَا عَمِلَت مِن سُوٓءٍۢ تَوَدُّ لَو أَنَّ بَينَهَا وَبَينَهُۥٓ أَمَدًۢا بَعِيدًۭا ۗ وَيُحَذِّرُكُمُ ٱللَّهُ نَفسَهُۥ ۗ وَٱللَّهُ رَءُوفٌۢ بِٱلعِبَادِ ﴿٣٠﴾\\
\textamh{31.\  } & قُل إِن كُنتُم تُحِبُّونَ ٱللَّهَ فَٱتَّبِعُونِى يُحبِبكُمُ ٱللَّهُ وَيَغفِر لَكُم ذُنُوبَكُم ۗ وَٱللَّهُ غَفُورٌۭ رَّحِيمٌۭ ﴿٣١﴾\\
\textamh{32.\  } & قُل أَطِيعُوا۟ ٱللَّهَ وَٱلرَّسُولَ ۖ فَإِن تَوَلَّوا۟ فَإِنَّ ٱللَّهَ لَا يُحِبُّ ٱلكَـٰفِرِينَ ﴿٣٢﴾\\
\textamh{33.\  } & ۞ إِنَّ ٱللَّهَ ٱصطَفَىٰٓ ءَادَمَ وَنُوحًۭا وَءَالَ إِبرَٰهِيمَ وَءَالَ عِمرَٰنَ عَلَى ٱلعَـٰلَمِينَ ﴿٣٣﴾\\
\textamh{34.\  } & ذُرِّيَّةًۢ بَعضُهَا مِنۢ بَعضٍۢ ۗ وَٱللَّهُ سَمِيعٌ عَلِيمٌ ﴿٣٤﴾\\
\textamh{35.\  } & إِذ قَالَتِ ٱمرَأَتُ عِمرَٰنَ رَبِّ إِنِّى نَذَرتُ لَكَ مَا فِى بَطنِى مُحَرَّرًۭا فَتَقَبَّل مِنِّىٓ ۖ إِنَّكَ أَنتَ ٱلسَّمِيعُ ٱلعَلِيمُ ﴿٣٥﴾\\
\textamh{36.\  } & فَلَمَّا وَضَعَتهَا قَالَت رَبِّ إِنِّى وَضَعتُهَآ أُنثَىٰ وَٱللَّهُ أَعلَمُ بِمَا وَضَعَت وَلَيسَ ٱلذَّكَرُ كَٱلأُنثَىٰ ۖ وَإِنِّى سَمَّيتُهَا مَريَمَ وَإِنِّىٓ أُعِيذُهَا بِكَ وَذُرِّيَّتَهَا مِنَ ٱلشَّيطَٰنِ ٱلرَّجِيمِ ﴿٣٦﴾\\
\textamh{37.\  } & فَتَقَبَّلَهَا رَبُّهَا بِقَبُولٍ حَسَنٍۢ وَأَنۢبَتَهَا نَبَاتًا حَسَنًۭا وَكَفَّلَهَا زَكَرِيَّا ۖ كُلَّمَا دَخَلَ عَلَيهَا زَكَرِيَّا ٱلمِحرَابَ وَجَدَ عِندَهَا رِزقًۭا ۖ قَالَ يَـٰمَريَمُ أَنَّىٰ لَكِ هَـٰذَا ۖ قَالَت هُوَ مِن عِندِ ٱللَّهِ ۖ إِنَّ ٱللَّهَ يَرزُقُ مَن يَشَآءُ بِغَيرِ حِسَابٍ ﴿٣٧﴾\\
\textamh{38.\  } & هُنَالِكَ دَعَا زَكَرِيَّا رَبَّهُۥ ۖ قَالَ رَبِّ هَب لِى مِن لَّدُنكَ ذُرِّيَّةًۭ طَيِّبَةً ۖ إِنَّكَ سَمِيعُ ٱلدُّعَآءِ ﴿٣٨﴾\\
\textamh{39.\  } & فَنَادَتهُ ٱلمَلَـٰٓئِكَةُ وَهُوَ قَآئِمٌۭ يُصَلِّى فِى ٱلمِحرَابِ أَنَّ ٱللَّهَ يُبَشِّرُكَ بِيَحيَىٰ مُصَدِّقًۢا بِكَلِمَةٍۢ مِّنَ ٱللَّهِ وَسَيِّدًۭا وَحَصُورًۭا وَنَبِيًّۭا مِّنَ ٱلصَّـٰلِحِينَ ﴿٣٩﴾\\
\textamh{40.\  } & قَالَ رَبِّ أَنَّىٰ يَكُونُ لِى غُلَـٰمٌۭ وَقَد بَلَغَنِىَ ٱلكِبَرُ وَٱمرَأَتِى عَاقِرٌۭ ۖ قَالَ كَذَٟلِكَ ٱللَّهُ يَفعَلُ مَا يَشَآءُ ﴿٤٠﴾\\
\textamh{41.\  } & قَالَ رَبِّ ٱجعَل لِّىٓ ءَايَةًۭ ۖ قَالَ ءَايَتُكَ أَلَّا تُكَلِّمَ ٱلنَّاسَ ثَلَـٰثَةَ أَيَّامٍ إِلَّا رَمزًۭا ۗ وَٱذكُر رَّبَّكَ كَثِيرًۭا وَسَبِّح بِٱلعَشِىِّ وَٱلإِبكَـٰرِ ﴿٤١﴾\\
\textamh{42.\  } & وَإِذ قَالَتِ ٱلمَلَـٰٓئِكَةُ يَـٰمَريَمُ إِنَّ ٱللَّهَ ٱصطَفَىٰكِ وَطَهَّرَكِ وَٱصطَفَىٰكِ عَلَىٰ نِسَآءِ ٱلعَـٰلَمِينَ ﴿٤٢﴾\\
\textamh{43.\  } & يَـٰمَريَمُ ٱقنُتِى لِرَبِّكِ وَٱسجُدِى وَٱركَعِى مَعَ ٱلرَّٟكِعِينَ ﴿٤٣﴾\\
\textamh{44.\  } & ذَٟلِكَ مِن أَنۢبَآءِ ٱلغَيبِ نُوحِيهِ إِلَيكَ ۚ وَمَا كُنتَ لَدَيهِم إِذ يُلقُونَ أَقلَـٰمَهُم أَيُّهُم يَكفُلُ مَريَمَ وَمَا كُنتَ لَدَيهِم إِذ يَختَصِمُونَ ﴿٤٤﴾\\
\textamh{45.\  } & إِذ قَالَتِ ٱلمَلَـٰٓئِكَةُ يَـٰمَريَمُ إِنَّ ٱللَّهَ يُبَشِّرُكِ بِكَلِمَةٍۢ مِّنهُ ٱسمُهُ ٱلمَسِيحُ عِيسَى ٱبنُ مَريَمَ وَجِيهًۭا فِى ٱلدُّنيَا وَٱلءَاخِرَةِ وَمِنَ ٱلمُقَرَّبِينَ ﴿٤٥﴾\\
\textamh{46.\  } & وَيُكَلِّمُ ٱلنَّاسَ فِى ٱلمَهدِ وَكَهلًۭا وَمِنَ ٱلصَّـٰلِحِينَ ﴿٤٦﴾\\
\textamh{47.\  } & قَالَت رَبِّ أَنَّىٰ يَكُونُ لِى وَلَدٌۭ وَلَم يَمسَسنِى بَشَرٌۭ ۖ قَالَ كَذَٟلِكِ ٱللَّهُ يَخلُقُ مَا يَشَآءُ ۚ إِذَا قَضَىٰٓ أَمرًۭا فَإِنَّمَا يَقُولُ لَهُۥ كُن فَيَكُونُ ﴿٤٧﴾\\
\textamh{48.\  } & وَيُعَلِّمُهُ ٱلكِتَـٰبَ وَٱلحِكمَةَ وَٱلتَّورَىٰةَ وَٱلإِنجِيلَ ﴿٤٨﴾\\
\textamh{49.\  } & وَرَسُولًا إِلَىٰ بَنِىٓ إِسرَٰٓءِيلَ أَنِّى قَد جِئتُكُم بِـَٔايَةٍۢ مِّن رَّبِّكُم ۖ أَنِّىٓ أَخلُقُ لَكُم مِّنَ ٱلطِّينِ كَهَيـَٔةِ ٱلطَّيرِ فَأَنفُخُ فِيهِ فَيَكُونُ طَيرًۢا بِإِذنِ ٱللَّهِ ۖ وَأُبرِئُ ٱلأَكمَهَ وَٱلأَبرَصَ وَأُحىِ ٱلمَوتَىٰ بِإِذنِ ٱللَّهِ ۖ وَأُنَبِّئُكُم بِمَا تَأكُلُونَ وَمَا تَدَّخِرُونَ فِى بُيُوتِكُم ۚ إِنَّ فِى ذَٟلِكَ لَءَايَةًۭ لَّكُم إِن كُنتُم مُّؤمِنِينَ ﴿٤٩﴾\\
\textamh{50.\  } & وَمُصَدِّقًۭا لِّمَا بَينَ يَدَىَّ مِنَ ٱلتَّورَىٰةِ وَلِأُحِلَّ لَكُم بَعضَ ٱلَّذِى حُرِّمَ عَلَيكُم ۚ وَجِئتُكُم بِـَٔايَةٍۢ مِّن رَّبِّكُم فَٱتَّقُوا۟ ٱللَّهَ وَأَطِيعُونِ ﴿٥٠﴾\\
\textamh{51.\  } & إِنَّ ٱللَّهَ رَبِّى وَرَبُّكُم فَٱعبُدُوهُ ۗ هَـٰذَا صِرَٰطٌۭ مُّستَقِيمٌۭ ﴿٥١﴾\\
\textamh{52.\  } & ۞ فَلَمَّآ أَحَسَّ عِيسَىٰ مِنهُمُ ٱلكُفرَ قَالَ مَن أَنصَارِىٓ إِلَى ٱللَّهِ ۖ قَالَ ٱلحَوَارِيُّونَ نَحنُ أَنصَارُ ٱللَّهِ ءَامَنَّا بِٱللَّهِ وَٱشهَد بِأَنَّا مُسلِمُونَ ﴿٥٢﴾\\
\textamh{53.\  } & رَبَّنَآ ءَامَنَّا بِمَآ أَنزَلتَ وَٱتَّبَعنَا ٱلرَّسُولَ فَٱكتُبنَا مَعَ ٱلشَّـٰهِدِينَ ﴿٥٣﴾\\
\textamh{54.\  } & وَمَكَرُوا۟ وَمَكَرَ ٱللَّهُ ۖ وَٱللَّهُ خَيرُ ٱلمَـٰكِرِينَ ﴿٥٤﴾\\
\textamh{55.\  } & إِذ قَالَ ٱللَّهُ يَـٰعِيسَىٰٓ إِنِّى مُتَوَفِّيكَ وَرَافِعُكَ إِلَىَّ وَمُطَهِّرُكَ مِنَ ٱلَّذِينَ كَفَرُوا۟ وَجَاعِلُ ٱلَّذِينَ ٱتَّبَعُوكَ فَوقَ ٱلَّذِينَ كَفَرُوٓا۟ إِلَىٰ يَومِ ٱلقِيَـٰمَةِ ۖ ثُمَّ إِلَىَّ مَرجِعُكُم فَأَحكُمُ بَينَكُم فِيمَا كُنتُم فِيهِ تَختَلِفُونَ ﴿٥٥﴾\\
\textamh{56.\  } & فَأَمَّا ٱلَّذِينَ كَفَرُوا۟ فَأُعَذِّبُهُم عَذَابًۭا شَدِيدًۭا فِى ٱلدُّنيَا وَٱلءَاخِرَةِ وَمَا لَهُم مِّن نَّـٰصِرِينَ ﴿٥٦﴾\\
\textamh{57.\  } & وَأَمَّا ٱلَّذِينَ ءَامَنُوا۟ وَعَمِلُوا۟ ٱلصَّـٰلِحَـٰتِ فَيُوَفِّيهِم أُجُورَهُم ۗ وَٱللَّهُ لَا يُحِبُّ ٱلظَّـٰلِمِينَ ﴿٥٧﴾\\
\textamh{58.\  } & ذَٟلِكَ نَتلُوهُ عَلَيكَ مِنَ ٱلءَايَـٰتِ وَٱلذِّكرِ ٱلحَكِيمِ ﴿٥٨﴾\\
\textamh{59.\  } & إِنَّ مَثَلَ عِيسَىٰ عِندَ ٱللَّهِ كَمَثَلِ ءَادَمَ ۖ خَلَقَهُۥ مِن تُرَابٍۢ ثُمَّ قَالَ لَهُۥ كُن فَيَكُونُ ﴿٥٩﴾\\
\textamh{60.\  } & ٱلحَقُّ مِن رَّبِّكَ فَلَا تَكُن مِّنَ ٱلمُمتَرِينَ ﴿٦٠﴾\\
\textamh{61.\  } & فَمَن حَآجَّكَ فِيهِ مِنۢ بَعدِ مَا جَآءَكَ مِنَ ٱلعِلمِ فَقُل تَعَالَوا۟ نَدعُ أَبنَآءَنَا وَأَبنَآءَكُم وَنِسَآءَنَا وَنِسَآءَكُم وَأَنفُسَنَا وَأَنفُسَكُم ثُمَّ نَبتَهِل فَنَجعَل لَّعنَتَ ٱللَّهِ عَلَى ٱلكَـٰذِبِينَ ﴿٦١﴾\\
\textamh{62.\  } & إِنَّ هَـٰذَا لَهُوَ ٱلقَصَصُ ٱلحَقُّ ۚ وَمَا مِن إِلَـٰهٍ إِلَّا ٱللَّهُ ۚ وَإِنَّ ٱللَّهَ لَهُوَ ٱلعَزِيزُ ٱلحَكِيمُ ﴿٦٢﴾\\
\textamh{63.\  } & فَإِن تَوَلَّوا۟ فَإِنَّ ٱللَّهَ عَلِيمٌۢ بِٱلمُفسِدِينَ ﴿٦٣﴾\\
\textamh{64.\  } & قُل يَـٰٓأَهلَ ٱلكِتَـٰبِ تَعَالَوا۟ إِلَىٰ كَلِمَةٍۢ سَوَآءٍۭ بَينَنَا وَبَينَكُم أَلَّا نَعبُدَ إِلَّا ٱللَّهَ وَلَا نُشرِكَ بِهِۦ شَيـًۭٔا وَلَا يَتَّخِذَ بَعضُنَا بَعضًا أَربَابًۭا مِّن دُونِ ٱللَّهِ ۚ فَإِن تَوَلَّوا۟ فَقُولُوا۟ ٱشهَدُوا۟ بِأَنَّا مُسلِمُونَ ﴿٦٤﴾\\
\textamh{65.\  } & يَـٰٓأَهلَ ٱلكِتَـٰبِ لِمَ تُحَآجُّونَ فِىٓ إِبرَٰهِيمَ وَمَآ أُنزِلَتِ ٱلتَّورَىٰةُ وَٱلإِنجِيلُ إِلَّا مِنۢ بَعدِهِۦٓ ۚ أَفَلَا تَعقِلُونَ ﴿٦٥﴾\\
\textamh{66.\  } & هَـٰٓأَنتُم هَـٰٓؤُلَآءِ حَـٰجَجتُم فِيمَا لَكُم بِهِۦ عِلمٌۭ فَلِمَ تُحَآجُّونَ فِيمَا لَيسَ لَكُم بِهِۦ عِلمٌۭ ۚ وَٱللَّهُ يَعلَمُ وَأَنتُم لَا تَعلَمُونَ ﴿٦٦﴾\\
\textamh{67.\  } & مَا كَانَ إِبرَٰهِيمُ يَهُودِيًّۭا وَلَا نَصرَانِيًّۭا وَلَـٰكِن كَانَ حَنِيفًۭا مُّسلِمًۭا وَمَا كَانَ مِنَ ٱلمُشرِكِينَ ﴿٦٧﴾\\
\textamh{68.\  } & إِنَّ أَولَى ٱلنَّاسِ بِإِبرَٰهِيمَ لَلَّذِينَ ٱتَّبَعُوهُ وَهَـٰذَا ٱلنَّبِىُّ وَٱلَّذِينَ ءَامَنُوا۟ ۗ وَٱللَّهُ وَلِىُّ ٱلمُؤمِنِينَ ﴿٦٨﴾\\
\textamh{69.\  } & وَدَّت طَّآئِفَةٌۭ مِّن أَهلِ ٱلكِتَـٰبِ لَو يُضِلُّونَكُم وَمَا يُضِلُّونَ إِلَّآ أَنفُسَهُم وَمَا يَشعُرُونَ ﴿٦٩﴾\\
\textamh{70.\  } & يَـٰٓأَهلَ ٱلكِتَـٰبِ لِمَ تَكفُرُونَ بِـَٔايَـٰتِ ٱللَّهِ وَأَنتُم تَشهَدُونَ ﴿٧٠﴾\\
\textamh{71.\  } & يَـٰٓأَهلَ ٱلكِتَـٰبِ لِمَ تَلبِسُونَ ٱلحَقَّ بِٱلبَٰطِلِ وَتَكتُمُونَ ٱلحَقَّ وَأَنتُم تَعلَمُونَ ﴿٧١﴾\\
\textamh{72.\  } & وَقَالَت طَّآئِفَةٌۭ مِّن أَهلِ ٱلكِتَـٰبِ ءَامِنُوا۟ بِٱلَّذِىٓ أُنزِلَ عَلَى ٱلَّذِينَ ءَامَنُوا۟ وَجهَ ٱلنَّهَارِ وَٱكفُرُوٓا۟ ءَاخِرَهُۥ لَعَلَّهُم يَرجِعُونَ ﴿٧٢﴾\\
\textamh{73.\  } & وَلَا تُؤمِنُوٓا۟ إِلَّا لِمَن تَبِعَ دِينَكُم قُل إِنَّ ٱلهُدَىٰ هُدَى ٱللَّهِ أَن يُؤتَىٰٓ أَحَدٌۭ مِّثلَ مَآ أُوتِيتُم أَو يُحَآجُّوكُم عِندَ رَبِّكُم ۗ قُل إِنَّ ٱلفَضلَ بِيَدِ ٱللَّهِ يُؤتِيهِ مَن يَشَآءُ ۗ وَٱللَّهُ وَٟسِعٌ عَلِيمٌۭ ﴿٧٣﴾\\
\textamh{74.\  } & يَختَصُّ بِرَحمَتِهِۦ مَن يَشَآءُ ۗ وَٱللَّهُ ذُو ٱلفَضلِ ٱلعَظِيمِ ﴿٧٤﴾\\
\textamh{75.\  } & ۞ وَمِن أَهلِ ٱلكِتَـٰبِ مَن إِن تَأمَنهُ بِقِنطَارٍۢ يُؤَدِّهِۦٓ إِلَيكَ وَمِنهُم مَّن إِن تَأمَنهُ بِدِينَارٍۢ لَّا يُؤَدِّهِۦٓ إِلَيكَ إِلَّا مَا دُمتَ عَلَيهِ قَآئِمًۭا ۗ ذَٟلِكَ بِأَنَّهُم قَالُوا۟ لَيسَ عَلَينَا فِى ٱلأُمِّيِّۦنَ سَبِيلٌۭ وَيَقُولُونَ عَلَى ٱللَّهِ ٱلكَذِبَ وَهُم يَعلَمُونَ ﴿٧٥﴾\\
\textamh{76.\  } & بَلَىٰ مَن أَوفَىٰ بِعَهدِهِۦ وَٱتَّقَىٰ فَإِنَّ ٱللَّهَ يُحِبُّ ٱلمُتَّقِينَ ﴿٧٦﴾\\
\textamh{77.\  } & إِنَّ ٱلَّذِينَ يَشتَرُونَ بِعَهدِ ٱللَّهِ وَأَيمَـٰنِهِم ثَمَنًۭا قَلِيلًا أُو۟لَـٰٓئِكَ لَا خَلَـٰقَ لَهُم فِى ٱلءَاخِرَةِ وَلَا يُكَلِّمُهُمُ ٱللَّهُ وَلَا يَنظُرُ إِلَيهِم يَومَ ٱلقِيَـٰمَةِ وَلَا يُزَكِّيهِم وَلَهُم عَذَابٌ أَلِيمٌۭ ﴿٧٧﴾\\
\textamh{78.\  } & وَإِنَّ مِنهُم لَفَرِيقًۭا يَلوُۥنَ أَلسِنَتَهُم بِٱلكِتَـٰبِ لِتَحسَبُوهُ مِنَ ٱلكِتَـٰبِ وَمَا هُوَ مِنَ ٱلكِتَـٰبِ وَيَقُولُونَ هُوَ مِن عِندِ ٱللَّهِ وَمَا هُوَ مِن عِندِ ٱللَّهِ وَيَقُولُونَ عَلَى ٱللَّهِ ٱلكَذِبَ وَهُم يَعلَمُونَ ﴿٧٨﴾\\
\textamh{79.\  } & مَا كَانَ لِبَشَرٍ أَن يُؤتِيَهُ ٱللَّهُ ٱلكِتَـٰبَ وَٱلحُكمَ وَٱلنُّبُوَّةَ ثُمَّ يَقُولَ لِلنَّاسِ كُونُوا۟ عِبَادًۭا لِّى مِن دُونِ ٱللَّهِ وَلَـٰكِن كُونُوا۟ رَبَّـٰنِيِّۦنَ بِمَا كُنتُم تُعَلِّمُونَ ٱلكِتَـٰبَ وَبِمَا كُنتُم تَدرُسُونَ ﴿٧٩﴾\\
\textamh{80.\  } & وَلَا يَأمُرَكُم أَن تَتَّخِذُوا۟ ٱلمَلَـٰٓئِكَةَ وَٱلنَّبِيِّۦنَ أَربَابًا ۗ أَيَأمُرُكُم بِٱلكُفرِ بَعدَ إِذ أَنتُم مُّسلِمُونَ ﴿٨٠﴾\\
\textamh{81.\  } & وَإِذ أَخَذَ ٱللَّهُ مِيثَـٰقَ ٱلنَّبِيِّۦنَ لَمَآ ءَاتَيتُكُم مِّن كِتَـٰبٍۢ وَحِكمَةٍۢ ثُمَّ جَآءَكُم رَسُولٌۭ مُّصَدِّقٌۭ لِّمَا مَعَكُم لَتُؤمِنُنَّ بِهِۦ وَلَتَنصُرُنَّهُۥ ۚ قَالَ ءَأَقرَرتُم وَأَخَذتُم عَلَىٰ ذَٟلِكُم إِصرِى ۖ قَالُوٓا۟ أَقرَرنَا ۚ قَالَ فَٱشهَدُوا۟ وَأَنَا۠ مَعَكُم مِّنَ ٱلشَّـٰهِدِينَ ﴿٨١﴾\\
\textamh{82.\  } & فَمَن تَوَلَّىٰ بَعدَ ذَٟلِكَ فَأُو۟لَـٰٓئِكَ هُمُ ٱلفَـٰسِقُونَ ﴿٨٢﴾\\
\textamh{83.\  } & أَفَغَيرَ دِينِ ٱللَّهِ يَبغُونَ وَلَهُۥٓ أَسلَمَ مَن فِى ٱلسَّمَـٰوَٟتِ وَٱلأَرضِ طَوعًۭا وَكَرهًۭا وَإِلَيهِ يُرجَعُونَ ﴿٨٣﴾\\
\textamh{84.\  } & قُل ءَامَنَّا بِٱللَّهِ وَمَآ أُنزِلَ عَلَينَا وَمَآ أُنزِلَ عَلَىٰٓ إِبرَٰهِيمَ وَإِسمَـٰعِيلَ وَإِسحَـٰقَ وَيَعقُوبَ وَٱلأَسبَاطِ وَمَآ أُوتِىَ مُوسَىٰ وَعِيسَىٰ وَٱلنَّبِيُّونَ مِن رَّبِّهِم لَا نُفَرِّقُ بَينَ أَحَدٍۢ مِّنهُم وَنَحنُ لَهُۥ مُسلِمُونَ ﴿٨٤﴾\\
\textamh{85.\  } & وَمَن يَبتَغِ غَيرَ ٱلإِسلَـٰمِ دِينًۭا فَلَن يُقبَلَ مِنهُ وَهُوَ فِى ٱلءَاخِرَةِ مِنَ ٱلخَـٰسِرِينَ ﴿٨٥﴾\\
\textamh{86.\  } & كَيفَ يَهدِى ٱللَّهُ قَومًۭا كَفَرُوا۟ بَعدَ إِيمَـٰنِهِم وَشَهِدُوٓا۟ أَنَّ ٱلرَّسُولَ حَقٌّۭ وَجَآءَهُمُ ٱلبَيِّنَـٰتُ ۚ وَٱللَّهُ لَا يَهدِى ٱلقَومَ ٱلظَّـٰلِمِينَ ﴿٨٦﴾\\
\textamh{87.\  } & أُو۟لَـٰٓئِكَ جَزَآؤُهُم أَنَّ عَلَيهِم لَعنَةَ ٱللَّهِ وَٱلمَلَـٰٓئِكَةِ وَٱلنَّاسِ أَجمَعِينَ ﴿٨٧﴾\\
\textamh{88.\  } & خَـٰلِدِينَ فِيهَا لَا يُخَفَّفُ عَنهُمُ ٱلعَذَابُ وَلَا هُم يُنظَرُونَ ﴿٨٨﴾\\
\textamh{89.\  } & إِلَّا ٱلَّذِينَ تَابُوا۟ مِنۢ بَعدِ ذَٟلِكَ وَأَصلَحُوا۟ فَإِنَّ ٱللَّهَ غَفُورٌۭ رَّحِيمٌ ﴿٨٩﴾\\
\textamh{90.\  } & إِنَّ ٱلَّذِينَ كَفَرُوا۟ بَعدَ إِيمَـٰنِهِم ثُمَّ ٱزدَادُوا۟ كُفرًۭا لَّن تُقبَلَ تَوبَتُهُم وَأُو۟لَـٰٓئِكَ هُمُ ٱلضَّآلُّونَ ﴿٩٠﴾\\
\textamh{91.\  } & إِنَّ ٱلَّذِينَ كَفَرُوا۟ وَمَاتُوا۟ وَهُم كُفَّارٌۭ فَلَن يُقبَلَ مِن أَحَدِهِم مِّلءُ ٱلأَرضِ ذَهَبًۭا وَلَوِ ٱفتَدَىٰ بِهِۦٓ ۗ أُو۟لَـٰٓئِكَ لَهُم عَذَابٌ أَلِيمٌۭ وَمَا لَهُم مِّن نَّـٰصِرِينَ ﴿٩١﴾\\
\textamh{92.\  } & لَن تَنَالُوا۟ ٱلبِرَّ حَتَّىٰ تُنفِقُوا۟ مِمَّا تُحِبُّونَ ۚ وَمَا تُنفِقُوا۟ مِن شَىءٍۢ فَإِنَّ ٱللَّهَ بِهِۦ عَلِيمٌۭ ﴿٩٢﴾\\
\textamh{93.\  } & ۞ كُلُّ ٱلطَّعَامِ كَانَ حِلًّۭا لِّبَنِىٓ إِسرَٰٓءِيلَ إِلَّا مَا حَرَّمَ إِسرَٰٓءِيلُ عَلَىٰ نَفسِهِۦ مِن قَبلِ أَن تُنَزَّلَ ٱلتَّورَىٰةُ ۗ قُل فَأتُوا۟ بِٱلتَّورَىٰةِ فَٱتلُوهَآ إِن كُنتُم صَـٰدِقِينَ ﴿٩٣﴾\\
\textamh{94.\  } & فَمَنِ ٱفتَرَىٰ عَلَى ٱللَّهِ ٱلكَذِبَ مِنۢ بَعدِ ذَٟلِكَ فَأُو۟لَـٰٓئِكَ هُمُ ٱلظَّـٰلِمُونَ ﴿٩٤﴾\\
\textamh{95.\  } & قُل صَدَقَ ٱللَّهُ ۗ فَٱتَّبِعُوا۟ مِلَّةَ إِبرَٰهِيمَ حَنِيفًۭا وَمَا كَانَ مِنَ ٱلمُشرِكِينَ ﴿٩٥﴾\\
\textamh{96.\  } & إِنَّ أَوَّلَ بَيتٍۢ وُضِعَ لِلنَّاسِ لَلَّذِى بِبَكَّةَ مُبَارَكًۭا وَهُدًۭى لِّلعَـٰلَمِينَ ﴿٩٦﴾\\
\textamh{97.\  } & فِيهِ ءَايَـٰتٌۢ بَيِّنَـٰتٌۭ مَّقَامُ إِبرَٰهِيمَ ۖ وَمَن دَخَلَهُۥ كَانَ ءَامِنًۭا ۗ وَلِلَّهِ عَلَى ٱلنَّاسِ حِجُّ ٱلبَيتِ مَنِ ٱستَطَاعَ إِلَيهِ سَبِيلًۭا ۚ وَمَن كَفَرَ فَإِنَّ ٱللَّهَ غَنِىٌّ عَنِ ٱلعَـٰلَمِينَ ﴿٩٧﴾\\
\textamh{98.\  } & قُل يَـٰٓأَهلَ ٱلكِتَـٰبِ لِمَ تَكفُرُونَ بِـَٔايَـٰتِ ٱللَّهِ وَٱللَّهُ شَهِيدٌ عَلَىٰ مَا تَعمَلُونَ ﴿٩٨﴾\\
\textamh{99.\  } & قُل يَـٰٓأَهلَ ٱلكِتَـٰبِ لِمَ تَصُدُّونَ عَن سَبِيلِ ٱللَّهِ مَن ءَامَنَ تَبغُونَهَا عِوَجًۭا وَأَنتُم شُهَدَآءُ ۗ وَمَا ٱللَّهُ بِغَٰفِلٍ عَمَّا تَعمَلُونَ ﴿٩٩﴾\\
\textamh{100.\  } & يَـٰٓأَيُّهَا ٱلَّذِينَ ءَامَنُوٓا۟ إِن تُطِيعُوا۟ فَرِيقًۭا مِّنَ ٱلَّذِينَ أُوتُوا۟ ٱلكِتَـٰبَ يَرُدُّوكُم بَعدَ إِيمَـٰنِكُم كَـٰفِرِينَ ﴿١٠٠﴾\\
\textamh{101.\  } & وَكَيفَ تَكفُرُونَ وَأَنتُم تُتلَىٰ عَلَيكُم ءَايَـٰتُ ٱللَّهِ وَفِيكُم رَسُولُهُۥ ۗ وَمَن يَعتَصِم بِٱللَّهِ فَقَد هُدِىَ إِلَىٰ صِرَٰطٍۢ مُّستَقِيمٍۢ ﴿١٠١﴾\\
\textamh{102.\  } & يَـٰٓأَيُّهَا ٱلَّذِينَ ءَامَنُوا۟ ٱتَّقُوا۟ ٱللَّهَ حَقَّ تُقَاتِهِۦ وَلَا تَمُوتُنَّ إِلَّا وَأَنتُم مُّسلِمُونَ ﴿١٠٢﴾\\
\textamh{103.\  } & وَٱعتَصِمُوا۟ بِحَبلِ ٱللَّهِ جَمِيعًۭا وَلَا تَفَرَّقُوا۟ ۚ وَٱذكُرُوا۟ نِعمَتَ ٱللَّهِ عَلَيكُم إِذ كُنتُم أَعدَآءًۭ فَأَلَّفَ بَينَ قُلُوبِكُم فَأَصبَحتُم بِنِعمَتِهِۦٓ إِخوَٟنًۭا وَكُنتُم عَلَىٰ شَفَا حُفرَةٍۢ مِّنَ ٱلنَّارِ فَأَنقَذَكُم مِّنهَا ۗ كَذَٟلِكَ يُبَيِّنُ ٱللَّهُ لَكُم ءَايَـٰتِهِۦ لَعَلَّكُم تَهتَدُونَ ﴿١٠٣﴾\\
\textamh{104.\  } & وَلتَكُن مِّنكُم أُمَّةٌۭ يَدعُونَ إِلَى ٱلخَيرِ وَيَأمُرُونَ بِٱلمَعرُوفِ وَيَنهَونَ عَنِ ٱلمُنكَرِ ۚ وَأُو۟لَـٰٓئِكَ هُمُ ٱلمُفلِحُونَ ﴿١٠٤﴾\\
\textamh{105.\  } & وَلَا تَكُونُوا۟ كَٱلَّذِينَ تَفَرَّقُوا۟ وَٱختَلَفُوا۟ مِنۢ بَعدِ مَا جَآءَهُمُ ٱلبَيِّنَـٰتُ ۚ وَأُو۟لَـٰٓئِكَ لَهُم عَذَابٌ عَظِيمٌۭ ﴿١٠٥﴾\\
\textamh{106.\  } & يَومَ تَبيَضُّ وُجُوهٌۭ وَتَسوَدُّ وُجُوهٌۭ ۚ فَأَمَّا ٱلَّذِينَ ٱسوَدَّت وُجُوهُهُم أَكَفَرتُم بَعدَ إِيمَـٰنِكُم فَذُوقُوا۟ ٱلعَذَابَ بِمَا كُنتُم تَكفُرُونَ ﴿١٠٦﴾\\
\textamh{107.\  } & وَأَمَّا ٱلَّذِينَ ٱبيَضَّت وُجُوهُهُم فَفِى رَحمَةِ ٱللَّهِ هُم فِيهَا خَـٰلِدُونَ ﴿١٠٧﴾\\
\textamh{108.\  } & تِلكَ ءَايَـٰتُ ٱللَّهِ نَتلُوهَا عَلَيكَ بِٱلحَقِّ ۗ وَمَا ٱللَّهُ يُرِيدُ ظُلمًۭا لِّلعَـٰلَمِينَ ﴿١٠٨﴾\\
\textamh{109.\  } & وَلِلَّهِ مَا فِى ٱلسَّمَـٰوَٟتِ وَمَا فِى ٱلأَرضِ ۚ وَإِلَى ٱللَّهِ تُرجَعُ ٱلأُمُورُ ﴿١٠٩﴾\\
\textamh{110.\  } & كُنتُم خَيرَ أُمَّةٍ أُخرِجَت لِلنَّاسِ تَأمُرُونَ بِٱلمَعرُوفِ وَتَنهَونَ عَنِ ٱلمُنكَرِ وَتُؤمِنُونَ بِٱللَّهِ ۗ وَلَو ءَامَنَ أَهلُ ٱلكِتَـٰبِ لَكَانَ خَيرًۭا لَّهُم ۚ مِّنهُمُ ٱلمُؤمِنُونَ وَأَكثَرُهُمُ ٱلفَـٰسِقُونَ ﴿١١٠﴾\\
\textamh{111.\  } & لَن يَضُرُّوكُم إِلَّآ أَذًۭى ۖ وَإِن يُقَـٰتِلُوكُم يُوَلُّوكُمُ ٱلأَدبَارَ ثُمَّ لَا يُنصَرُونَ ﴿١١١﴾\\
\textamh{112.\  } & ضُرِبَت عَلَيهِمُ ٱلذِّلَّةُ أَينَ مَا ثُقِفُوٓا۟ إِلَّا بِحَبلٍۢ مِّنَ ٱللَّهِ وَحَبلٍۢ مِّنَ ٱلنَّاسِ وَبَآءُو بِغَضَبٍۢ مِّنَ ٱللَّهِ وَضُرِبَت عَلَيهِمُ ٱلمَسكَنَةُ ۚ ذَٟلِكَ بِأَنَّهُم كَانُوا۟ يَكفُرُونَ بِـَٔايَـٰتِ ٱللَّهِ وَيَقتُلُونَ ٱلأَنۢبِيَآءَ بِغَيرِ حَقٍّۢ ۚ ذَٟلِكَ بِمَا عَصَوا۟ وَّكَانُوا۟ يَعتَدُونَ ﴿١١٢﴾\\
\textamh{113.\  } & ۞ لَيسُوا۟ سَوَآءًۭ ۗ مِّن أَهلِ ٱلكِتَـٰبِ أُمَّةٌۭ قَآئِمَةٌۭ يَتلُونَ ءَايَـٰتِ ٱللَّهِ ءَانَآءَ ٱلَّيلِ وَهُم يَسجُدُونَ ﴿١١٣﴾\\
\textamh{114.\  } & يُؤمِنُونَ بِٱللَّهِ وَٱليَومِ ٱلءَاخِرِ وَيَأمُرُونَ بِٱلمَعرُوفِ وَيَنهَونَ عَنِ ٱلمُنكَرِ وَيُسَـٰرِعُونَ فِى ٱلخَيرَٰتِ وَأُو۟لَـٰٓئِكَ مِنَ ٱلصَّـٰلِحِينَ ﴿١١٤﴾\\
\textamh{115.\  } & وَمَا يَفعَلُوا۟ مِن خَيرٍۢ فَلَن يُكفَرُوهُ ۗ وَٱللَّهُ عَلِيمٌۢ بِٱلمُتَّقِينَ ﴿١١٥﴾\\
\textamh{116.\  } & إِنَّ ٱلَّذِينَ كَفَرُوا۟ لَن تُغنِىَ عَنهُم أَموَٟلُهُم وَلَآ أَولَـٰدُهُم مِّنَ ٱللَّهِ شَيـًۭٔا ۖ وَأُو۟لَـٰٓئِكَ أَصحَـٰبُ ٱلنَّارِ ۚ هُم فِيهَا خَـٰلِدُونَ ﴿١١٦﴾\\
\textamh{117.\  } & مَثَلُ مَا يُنفِقُونَ فِى هَـٰذِهِ ٱلحَيَوٰةِ ٱلدُّنيَا كَمَثَلِ رِيحٍۢ فِيهَا صِرٌّ أَصَابَت حَرثَ قَومٍۢ ظَلَمُوٓا۟ أَنفُسَهُم فَأَهلَكَتهُ ۚ وَمَا ظَلَمَهُمُ ٱللَّهُ وَلَـٰكِن أَنفُسَهُم يَظلِمُونَ ﴿١١٧﴾\\
\textamh{118.\  } & يَـٰٓأَيُّهَا ٱلَّذِينَ ءَامَنُوا۟ لَا تَتَّخِذُوا۟ بِطَانَةًۭ مِّن دُونِكُم لَا يَألُونَكُم خَبَالًۭا وَدُّوا۟ مَا عَنِتُّم قَد بَدَتِ ٱلبَغضَآءُ مِن أَفوَٟهِهِم وَمَا تُخفِى صُدُورُهُم أَكبَرُ ۚ قَد بَيَّنَّا لَكُمُ ٱلءَايَـٰتِ ۖ إِن كُنتُم تَعقِلُونَ ﴿١١٨﴾\\
\textamh{119.\  } & هَـٰٓأَنتُم أُو۟لَآءِ تُحِبُّونَهُم وَلَا يُحِبُّونَكُم وَتُؤمِنُونَ بِٱلكِتَـٰبِ كُلِّهِۦ وَإِذَا لَقُوكُم قَالُوٓا۟ ءَامَنَّا وَإِذَا خَلَوا۟ عَضُّوا۟ عَلَيكُمُ ٱلأَنَامِلَ مِنَ ٱلغَيظِ ۚ قُل مُوتُوا۟ بِغَيظِكُم ۗ إِنَّ ٱللَّهَ عَلِيمٌۢ بِذَاتِ ٱلصُّدُورِ ﴿١١٩﴾\\
\textamh{120.\  } & إِن تَمسَسكُم حَسَنَةٌۭ تَسُؤهُم وَإِن تُصِبكُم سَيِّئَةٌۭ يَفرَحُوا۟ بِهَا ۖ وَإِن تَصبِرُوا۟ وَتَتَّقُوا۟ لَا يَضُرُّكُم كَيدُهُم شَيـًٔا ۗ إِنَّ ٱللَّهَ بِمَا يَعمَلُونَ مُحِيطٌۭ ﴿١٢٠﴾\\
\textamh{121.\  } & وَإِذ غَدَوتَ مِن أَهلِكَ تُبَوِّئُ ٱلمُؤمِنِينَ مَقَـٰعِدَ لِلقِتَالِ ۗ وَٱللَّهُ سَمِيعٌ عَلِيمٌ ﴿١٢١﴾\\
\textamh{122.\  } & إِذ هَمَّت طَّآئِفَتَانِ مِنكُم أَن تَفشَلَا وَٱللَّهُ وَلِيُّهُمَا ۗ وَعَلَى ٱللَّهِ فَليَتَوَكَّلِ ٱلمُؤمِنُونَ ﴿١٢٢﴾\\
\textamh{123.\  } & وَلَقَد نَصَرَكُمُ ٱللَّهُ بِبَدرٍۢ وَأَنتُم أَذِلَّةٌۭ ۖ فَٱتَّقُوا۟ ٱللَّهَ لَعَلَّكُم تَشكُرُونَ ﴿١٢٣﴾\\
\textamh{124.\  } & إِذ تَقُولُ لِلمُؤمِنِينَ أَلَن يَكفِيَكُم أَن يُمِدَّكُم رَبُّكُم بِثَلَـٰثَةِ ءَالَـٰفٍۢ مِّنَ ٱلمَلَـٰٓئِكَةِ مُنزَلِينَ ﴿١٢٤﴾\\
\textamh{125.\  } & بَلَىٰٓ ۚ إِن تَصبِرُوا۟ وَتَتَّقُوا۟ وَيَأتُوكُم مِّن فَورِهِم هَـٰذَا يُمدِدكُم رَبُّكُم بِخَمسَةِ ءَالَـٰفٍۢ مِّنَ ٱلمَلَـٰٓئِكَةِ مُسَوِّمِينَ ﴿١٢٥﴾\\
\textamh{126.\  } & وَمَا جَعَلَهُ ٱللَّهُ إِلَّا بُشرَىٰ لَكُم وَلِتَطمَئِنَّ قُلُوبُكُم بِهِۦ ۗ وَمَا ٱلنَّصرُ إِلَّا مِن عِندِ ٱللَّهِ ٱلعَزِيزِ ٱلحَكِيمِ ﴿١٢٦﴾\\
\textamh{127.\  } & لِيَقطَعَ طَرَفًۭا مِّنَ ٱلَّذِينَ كَفَرُوٓا۟ أَو يَكبِتَهُم فَيَنقَلِبُوا۟ خَآئِبِينَ ﴿١٢٧﴾\\
\textamh{128.\  } & لَيسَ لَكَ مِنَ ٱلأَمرِ شَىءٌ أَو يَتُوبَ عَلَيهِم أَو يُعَذِّبَهُم فَإِنَّهُم ظَـٰلِمُونَ ﴿١٢٨﴾\\
\textamh{129.\  } & وَلِلَّهِ مَا فِى ٱلسَّمَـٰوَٟتِ وَمَا فِى ٱلأَرضِ ۚ يَغفِرُ لِمَن يَشَآءُ وَيُعَذِّبُ مَن يَشَآءُ ۚ وَٱللَّهُ غَفُورٌۭ رَّحِيمٌۭ ﴿١٢٩﴾\\
\textamh{130.\  } & يَـٰٓأَيُّهَا ٱلَّذِينَ ءَامَنُوا۟ لَا تَأكُلُوا۟ ٱلرِّبَوٰٓا۟ أَضعَـٰفًۭا مُّضَٰعَفَةًۭ ۖ وَٱتَّقُوا۟ ٱللَّهَ لَعَلَّكُم تُفلِحُونَ ﴿١٣٠﴾\\
\textamh{131.\  } & وَٱتَّقُوا۟ ٱلنَّارَ ٱلَّتِىٓ أُعِدَّت لِلكَـٰفِرِينَ ﴿١٣١﴾\\
\textamh{132.\  } & وَأَطِيعُوا۟ ٱللَّهَ وَٱلرَّسُولَ لَعَلَّكُم تُرحَمُونَ ﴿١٣٢﴾\\
\textamh{133.\  } & ۞ وَسَارِعُوٓا۟ إِلَىٰ مَغفِرَةٍۢ مِّن رَّبِّكُم وَجَنَّةٍ عَرضُهَا ٱلسَّمَـٰوَٟتُ وَٱلأَرضُ أُعِدَّت لِلمُتَّقِينَ ﴿١٣٣﴾\\
\textamh{134.\  } & ٱلَّذِينَ يُنفِقُونَ فِى ٱلسَّرَّآءِ وَٱلضَّرَّآءِ وَٱلكَـٰظِمِينَ ٱلغَيظَ وَٱلعَافِينَ عَنِ ٱلنَّاسِ ۗ وَٱللَّهُ يُحِبُّ ٱلمُحسِنِينَ ﴿١٣٤﴾\\
\textamh{135.\  } & وَٱلَّذِينَ إِذَا فَعَلُوا۟ فَـٰحِشَةً أَو ظَلَمُوٓا۟ أَنفُسَهُم ذَكَرُوا۟ ٱللَّهَ فَٱستَغفَرُوا۟ لِذُنُوبِهِم وَمَن يَغفِرُ ٱلذُّنُوبَ إِلَّا ٱللَّهُ وَلَم يُصِرُّوا۟ عَلَىٰ مَا فَعَلُوا۟ وَهُم يَعلَمُونَ ﴿١٣٥﴾\\
\textamh{136.\  } & أُو۟لَـٰٓئِكَ جَزَآؤُهُم مَّغفِرَةٌۭ مِّن رَّبِّهِم وَجَنَّـٰتٌۭ تَجرِى مِن تَحتِهَا ٱلأَنهَـٰرُ خَـٰلِدِينَ فِيهَا ۚ وَنِعمَ أَجرُ ٱلعَـٰمِلِينَ ﴿١٣٦﴾\\
\textamh{137.\  } & قَد خَلَت مِن قَبلِكُم سُنَنٌۭ فَسِيرُوا۟ فِى ٱلأَرضِ فَٱنظُرُوا۟ كَيفَ كَانَ عَـٰقِبَةُ ٱلمُكَذِّبِينَ ﴿١٣٧﴾\\
\textamh{138.\  } & هَـٰذَا بَيَانٌۭ لِّلنَّاسِ وَهُدًۭى وَمَوعِظَةٌۭ لِّلمُتَّقِينَ ﴿١٣٨﴾\\
\textamh{139.\  } & وَلَا تَهِنُوا۟ وَلَا تَحزَنُوا۟ وَأَنتُمُ ٱلأَعلَونَ إِن كُنتُم مُّؤمِنِينَ ﴿١٣٩﴾\\
\textamh{140.\  } & إِن يَمسَسكُم قَرحٌۭ فَقَد مَسَّ ٱلقَومَ قَرحٌۭ مِّثلُهُۥ ۚ وَتِلكَ ٱلأَيَّامُ نُدَاوِلُهَا بَينَ ٱلنَّاسِ وَلِيَعلَمَ ٱللَّهُ ٱلَّذِينَ ءَامَنُوا۟ وَيَتَّخِذَ مِنكُم شُهَدَآءَ ۗ وَٱللَّهُ لَا يُحِبُّ ٱلظَّـٰلِمِينَ ﴿١٤٠﴾\\
\textamh{141.\  } & وَلِيُمَحِّصَ ٱللَّهُ ٱلَّذِينَ ءَامَنُوا۟ وَيَمحَقَ ٱلكَـٰفِرِينَ ﴿١٤١﴾\\
\textamh{142.\  } & أَم حَسِبتُم أَن تَدخُلُوا۟ ٱلجَنَّةَ وَلَمَّا يَعلَمِ ٱللَّهُ ٱلَّذِينَ جَٰهَدُوا۟ مِنكُم وَيَعلَمَ ٱلصَّـٰبِرِينَ ﴿١٤٢﴾\\
\textamh{143.\  } & وَلَقَد كُنتُم تَمَنَّونَ ٱلمَوتَ مِن قَبلِ أَن تَلقَوهُ فَقَد رَأَيتُمُوهُ وَأَنتُم تَنظُرُونَ ﴿١٤٣﴾\\
\textamh{144.\  } & وَمَا مُحَمَّدٌ إِلَّا رَسُولٌۭ قَد خَلَت مِن قَبلِهِ ٱلرُّسُلُ ۚ أَفَإِي۟ن مَّاتَ أَو قُتِلَ ٱنقَلَبتُم عَلَىٰٓ أَعقَـٰبِكُم ۚ وَمَن يَنقَلِب عَلَىٰ عَقِبَيهِ فَلَن يَضُرَّ ٱللَّهَ شَيـًۭٔا ۗ وَسَيَجزِى ٱللَّهُ ٱلشَّـٰكِرِينَ ﴿١٤٤﴾\\
\textamh{145.\  } & وَمَا كَانَ لِنَفسٍ أَن تَمُوتَ إِلَّا بِإِذنِ ٱللَّهِ كِتَـٰبًۭا مُّؤَجَّلًۭا ۗ وَمَن يُرِد ثَوَابَ ٱلدُّنيَا نُؤتِهِۦ مِنهَا وَمَن يُرِد ثَوَابَ ٱلءَاخِرَةِ نُؤتِهِۦ مِنهَا ۚ وَسَنَجزِى ٱلشَّـٰكِرِينَ ﴿١٤٥﴾\\
\textamh{146.\  } & وَكَأَيِّن مِّن نَّبِىٍّۢ قَـٰتَلَ مَعَهُۥ رِبِّيُّونَ كَثِيرٌۭ فَمَا وَهَنُوا۟ لِمَآ أَصَابَهُم فِى سَبِيلِ ٱللَّهِ وَمَا ضَعُفُوا۟ وَمَا ٱستَكَانُوا۟ ۗ وَٱللَّهُ يُحِبُّ ٱلصَّـٰبِرِينَ ﴿١٤٦﴾\\
\textamh{147.\  } & وَمَا كَانَ قَولَهُم إِلَّآ أَن قَالُوا۟ رَبَّنَا ٱغفِر لَنَا ذُنُوبَنَا وَإِسرَافَنَا فِىٓ أَمرِنَا وَثَبِّت أَقدَامَنَا وَٱنصُرنَا عَلَى ٱلقَومِ ٱلكَـٰفِرِينَ ﴿١٤٧﴾\\
\textamh{148.\  } & فَـَٔاتَىٰهُمُ ٱللَّهُ ثَوَابَ ٱلدُّنيَا وَحُسنَ ثَوَابِ ٱلءَاخِرَةِ ۗ وَٱللَّهُ يُحِبُّ ٱلمُحسِنِينَ ﴿١٤٨﴾\\
\textamh{149.\  } & يَـٰٓأَيُّهَا ٱلَّذِينَ ءَامَنُوٓا۟ إِن تُطِيعُوا۟ ٱلَّذِينَ كَفَرُوا۟ يَرُدُّوكُم عَلَىٰٓ أَعقَـٰبِكُم فَتَنقَلِبُوا۟ خَـٰسِرِينَ ﴿١٤٩﴾\\
\textamh{150.\  } & بَلِ ٱللَّهُ مَولَىٰكُم ۖ وَهُوَ خَيرُ ٱلنَّـٰصِرِينَ ﴿١٥٠﴾\\
\textamh{151.\  } & سَنُلقِى فِى قُلُوبِ ٱلَّذِينَ كَفَرُوا۟ ٱلرُّعبَ بِمَآ أَشرَكُوا۟ بِٱللَّهِ مَا لَم يُنَزِّل بِهِۦ سُلطَٰنًۭا ۖ وَمَأوَىٰهُمُ ٱلنَّارُ ۚ وَبِئسَ مَثوَى ٱلظَّـٰلِمِينَ ﴿١٥١﴾\\
\textamh{152.\  } & وَلَقَد صَدَقَكُمُ ٱللَّهُ وَعدَهُۥٓ إِذ تَحُسُّونَهُم بِإِذنِهِۦ ۖ حَتَّىٰٓ إِذَا فَشِلتُم وَتَنَـٰزَعتُم فِى ٱلأَمرِ وَعَصَيتُم مِّنۢ بَعدِ مَآ أَرَىٰكُم مَّا تُحِبُّونَ ۚ مِنكُم مَّن يُرِيدُ ٱلدُّنيَا وَمِنكُم مَّن يُرِيدُ ٱلءَاخِرَةَ ۚ ثُمَّ صَرَفَكُم عَنهُم لِيَبتَلِيَكُم ۖ وَلَقَد عَفَا عَنكُم ۗ وَٱللَّهُ ذُو فَضلٍ عَلَى ٱلمُؤمِنِينَ ﴿١٥٢﴾\\
\textamh{153.\  } & ۞ إِذ تُصعِدُونَ وَلَا تَلوُۥنَ عَلَىٰٓ أَحَدٍۢ وَٱلرَّسُولُ يَدعُوكُم فِىٓ أُخرَىٰكُم فَأَثَـٰبَكُم غَمًّۢا بِغَمٍّۢ لِّكَيلَا تَحزَنُوا۟ عَلَىٰ مَا فَاتَكُم وَلَا مَآ أَصَـٰبَكُم ۗ وَٱللَّهُ خَبِيرٌۢ بِمَا تَعمَلُونَ ﴿١٥٣﴾\\
\textamh{154.\  } & ثُمَّ أَنزَلَ عَلَيكُم مِّنۢ بَعدِ ٱلغَمِّ أَمَنَةًۭ نُّعَاسًۭا يَغشَىٰ طَآئِفَةًۭ مِّنكُم ۖ وَطَآئِفَةٌۭ قَد أَهَمَّتهُم أَنفُسُهُم يَظُنُّونَ بِٱللَّهِ غَيرَ ٱلحَقِّ ظَنَّ ٱلجَٰهِلِيَّةِ ۖ يَقُولُونَ هَل لَّنَا مِنَ ٱلأَمرِ مِن شَىءٍۢ ۗ قُل إِنَّ ٱلأَمرَ كُلَّهُۥ لِلَّهِ ۗ يُخفُونَ فِىٓ أَنفُسِهِم مَّا لَا يُبدُونَ لَكَ ۖ يَقُولُونَ لَو كَانَ لَنَا مِنَ ٱلأَمرِ شَىءٌۭ مَّا قُتِلنَا هَـٰهُنَا ۗ قُل لَّو كُنتُم فِى بُيُوتِكُم لَبَرَزَ ٱلَّذِينَ كُتِبَ عَلَيهِمُ ٱلقَتلُ إِلَىٰ مَضَاجِعِهِم ۖ وَلِيَبتَلِىَ ٱللَّهُ مَا فِى صُدُورِكُم وَلِيُمَحِّصَ مَا فِى قُلُوبِكُم ۗ وَٱللَّهُ عَلِيمٌۢ بِذَاتِ ٱلصُّدُورِ ﴿١٥٤﴾\\
\textamh{155.\  } & إِنَّ ٱلَّذِينَ تَوَلَّوا۟ مِنكُم يَومَ ٱلتَقَى ٱلجَمعَانِ إِنَّمَا ٱستَزَلَّهُمُ ٱلشَّيطَٰنُ بِبَعضِ مَا كَسَبُوا۟ ۖ وَلَقَد عَفَا ٱللَّهُ عَنهُم ۗ إِنَّ ٱللَّهَ غَفُورٌ حَلِيمٌۭ ﴿١٥٥﴾\\
\textamh{156.\  } & يَـٰٓأَيُّهَا ٱلَّذِينَ ءَامَنُوا۟ لَا تَكُونُوا۟ كَٱلَّذِينَ كَفَرُوا۟ وَقَالُوا۟ لِإِخوَٟنِهِم إِذَا ضَرَبُوا۟ فِى ٱلأَرضِ أَو كَانُوا۟ غُزًّۭى لَّو كَانُوا۟ عِندَنَا مَا مَاتُوا۟ وَمَا قُتِلُوا۟ لِيَجعَلَ ٱللَّهُ ذَٟلِكَ حَسرَةًۭ فِى قُلُوبِهِم ۗ وَٱللَّهُ يُحىِۦ وَيُمِيتُ ۗ وَٱللَّهُ بِمَا تَعمَلُونَ بَصِيرٌۭ ﴿١٥٦﴾\\
\textamh{157.\  } & وَلَئِن قُتِلتُم فِى سَبِيلِ ٱللَّهِ أَو مُتُّم لَمَغفِرَةٌۭ مِّنَ ٱللَّهِ وَرَحمَةٌ خَيرٌۭ مِّمَّا يَجمَعُونَ ﴿١٥٧﴾\\
\textamh{158.\  } & وَلَئِن مُّتُّم أَو قُتِلتُم لَإِلَى ٱللَّهِ تُحشَرُونَ ﴿١٥٨﴾\\
\textamh{159.\  } & فَبِمَا رَحمَةٍۢ مِّنَ ٱللَّهِ لِنتَ لَهُم ۖ وَلَو كُنتَ فَظًّا غَلِيظَ ٱلقَلبِ لَٱنفَضُّوا۟ مِن حَولِكَ ۖ فَٱعفُ عَنهُم وَٱستَغفِر لَهُم وَشَاوِرهُم فِى ٱلأَمرِ ۖ فَإِذَا عَزَمتَ فَتَوَكَّل عَلَى ٱللَّهِ ۚ إِنَّ ٱللَّهَ يُحِبُّ ٱلمُتَوَكِّلِينَ ﴿١٥٩﴾\\
\textamh{160.\  } & إِن يَنصُركُمُ ٱللَّهُ فَلَا غَالِبَ لَكُم ۖ وَإِن يَخذُلكُم فَمَن ذَا ٱلَّذِى يَنصُرُكُم مِّنۢ بَعدِهِۦ ۗ وَعَلَى ٱللَّهِ فَليَتَوَكَّلِ ٱلمُؤمِنُونَ ﴿١٦٠﴾\\
\textamh{161.\  } & وَمَا كَانَ لِنَبِىٍّ أَن يَغُلَّ ۚ وَمَن يَغلُل يَأتِ بِمَا غَلَّ يَومَ ٱلقِيَـٰمَةِ ۚ ثُمَّ تُوَفَّىٰ كُلُّ نَفسٍۢ مَّا كَسَبَت وَهُم لَا يُظلَمُونَ ﴿١٦١﴾\\
\textamh{162.\  } & أَفَمَنِ ٱتَّبَعَ رِضوَٟنَ ٱللَّهِ كَمَنۢ بَآءَ بِسَخَطٍۢ مِّنَ ٱللَّهِ وَمَأوَىٰهُ جَهَنَّمُ ۚ وَبِئسَ ٱلمَصِيرُ ﴿١٦٢﴾\\
\textamh{163.\  } & هُم دَرَجَٰتٌ عِندَ ٱللَّهِ ۗ وَٱللَّهُ بَصِيرٌۢ بِمَا يَعمَلُونَ ﴿١٦٣﴾\\
\textamh{164.\  } & لَقَد مَنَّ ٱللَّهُ عَلَى ٱلمُؤمِنِينَ إِذ بَعَثَ فِيهِم رَسُولًۭا مِّن أَنفُسِهِم يَتلُوا۟ عَلَيهِم ءَايَـٰتِهِۦ وَيُزَكِّيهِم وَيُعَلِّمُهُمُ ٱلكِتَـٰبَ وَٱلحِكمَةَ وَإِن كَانُوا۟ مِن قَبلُ لَفِى ضَلَـٰلٍۢ مُّبِينٍ ﴿١٦٤﴾\\
\textamh{165.\  } & أَوَلَمَّآ أَصَـٰبَتكُم مُّصِيبَةٌۭ قَد أَصَبتُم مِّثلَيهَا قُلتُم أَنَّىٰ هَـٰذَا ۖ قُل هُوَ مِن عِندِ أَنفُسِكُم ۗ إِنَّ ٱللَّهَ عَلَىٰ كُلِّ شَىءٍۢ قَدِيرٌۭ ﴿١٦٥﴾\\
\textamh{166.\  } & وَمَآ أَصَـٰبَكُم يَومَ ٱلتَقَى ٱلجَمعَانِ فَبِإِذنِ ٱللَّهِ وَلِيَعلَمَ ٱلمُؤمِنِينَ ﴿١٦٦﴾\\
\textamh{167.\  } & وَلِيَعلَمَ ٱلَّذِينَ نَافَقُوا۟ ۚ وَقِيلَ لَهُم تَعَالَوا۟ قَـٰتِلُوا۟ فِى سَبِيلِ ٱللَّهِ أَوِ ٱدفَعُوا۟ ۖ قَالُوا۟ لَو نَعلَمُ قِتَالًۭا لَّٱتَّبَعنَـٰكُم ۗ هُم لِلكُفرِ يَومَئِذٍ أَقرَبُ مِنهُم لِلإِيمَـٰنِ ۚ يَقُولُونَ بِأَفوَٟهِهِم مَّا لَيسَ فِى قُلُوبِهِم ۗ وَٱللَّهُ أَعلَمُ بِمَا يَكتُمُونَ ﴿١٦٧﴾\\
\textamh{168.\  } & ٱلَّذِينَ قَالُوا۟ لِإِخوَٟنِهِم وَقَعَدُوا۟ لَو أَطَاعُونَا مَا قُتِلُوا۟ ۗ قُل فَٱدرَءُوا۟ عَن أَنفُسِكُمُ ٱلمَوتَ إِن كُنتُم صَـٰدِقِينَ ﴿١٦٨﴾\\
\textamh{169.\  } & وَلَا تَحسَبَنَّ ٱلَّذِينَ قُتِلُوا۟ فِى سَبِيلِ ٱللَّهِ أَموَٟتًۢا ۚ بَل أَحيَآءٌ عِندَ رَبِّهِم يُرزَقُونَ ﴿١٦٩﴾\\
\textamh{170.\  } & فَرِحِينَ بِمَآ ءَاتَىٰهُمُ ٱللَّهُ مِن فَضلِهِۦ وَيَستَبشِرُونَ بِٱلَّذِينَ لَم يَلحَقُوا۟ بِهِم مِّن خَلفِهِم أَلَّا خَوفٌ عَلَيهِم وَلَا هُم يَحزَنُونَ ﴿١٧٠﴾\\
\textamh{171.\  } & ۞ يَستَبشِرُونَ بِنِعمَةٍۢ مِّنَ ٱللَّهِ وَفَضلٍۢ وَأَنَّ ٱللَّهَ لَا يُضِيعُ أَجرَ ٱلمُؤمِنِينَ ﴿١٧١﴾\\
\textamh{172.\  } & ٱلَّذِينَ ٱستَجَابُوا۟ لِلَّهِ وَٱلرَّسُولِ مِنۢ بَعدِ مَآ أَصَابَهُمُ ٱلقَرحُ ۚ لِلَّذِينَ أَحسَنُوا۟ مِنهُم وَٱتَّقَوا۟ أَجرٌ عَظِيمٌ ﴿١٧٢﴾\\
\textamh{173.\  } & ٱلَّذِينَ قَالَ لَهُمُ ٱلنَّاسُ إِنَّ ٱلنَّاسَ قَد جَمَعُوا۟ لَكُم فَٱخشَوهُم فَزَادَهُم إِيمَـٰنًۭا وَقَالُوا۟ حَسبُنَا ٱللَّهُ وَنِعمَ ٱلوَكِيلُ ﴿١٧٣﴾\\
\textamh{174.\  } & فَٱنقَلَبُوا۟ بِنِعمَةٍۢ مِّنَ ٱللَّهِ وَفَضلٍۢ لَّم يَمسَسهُم سُوٓءٌۭ وَٱتَّبَعُوا۟ رِضوَٟنَ ٱللَّهِ ۗ وَٱللَّهُ ذُو فَضلٍ عَظِيمٍ ﴿١٧٤﴾\\
\textamh{175.\  } & إِنَّمَا ذَٟلِكُمُ ٱلشَّيطَٰنُ يُخَوِّفُ أَولِيَآءَهُۥ فَلَا تَخَافُوهُم وَخَافُونِ إِن كُنتُم مُّؤمِنِينَ ﴿١٧٥﴾\\
\textamh{176.\  } & وَلَا يَحزُنكَ ٱلَّذِينَ يُسَـٰرِعُونَ فِى ٱلكُفرِ ۚ إِنَّهُم لَن يَضُرُّوا۟ ٱللَّهَ شَيـًۭٔا ۗ يُرِيدُ ٱللَّهُ أَلَّا يَجعَلَ لَهُم حَظًّۭا فِى ٱلءَاخِرَةِ ۖ وَلَهُم عَذَابٌ عَظِيمٌ ﴿١٧٦﴾\\
\textamh{177.\  } & إِنَّ ٱلَّذِينَ ٱشتَرَوُا۟ ٱلكُفرَ بِٱلإِيمَـٰنِ لَن يَضُرُّوا۟ ٱللَّهَ شَيـًۭٔا وَلَهُم عَذَابٌ أَلِيمٌۭ ﴿١٧٧﴾\\
\textamh{178.\  } & وَلَا يَحسَبَنَّ ٱلَّذِينَ كَفَرُوٓا۟ أَنَّمَا نُملِى لَهُم خَيرٌۭ لِّأَنفُسِهِم ۚ إِنَّمَا نُملِى لَهُم لِيَزدَادُوٓا۟ إِثمًۭا ۚ وَلَهُم عَذَابٌۭ مُّهِينٌۭ ﴿١٧٨﴾\\
\textamh{179.\  } & مَّا كَانَ ٱللَّهُ لِيَذَرَ ٱلمُؤمِنِينَ عَلَىٰ مَآ أَنتُم عَلَيهِ حَتَّىٰ يَمِيزَ ٱلخَبِيثَ مِنَ ٱلطَّيِّبِ ۗ وَمَا كَانَ ٱللَّهُ لِيُطلِعَكُم عَلَى ٱلغَيبِ وَلَـٰكِنَّ ٱللَّهَ يَجتَبِى مِن رُّسُلِهِۦ مَن يَشَآءُ ۖ فَـَٔامِنُوا۟ بِٱللَّهِ وَرُسُلِهِۦ ۚ وَإِن تُؤمِنُوا۟ وَتَتَّقُوا۟ فَلَكُم أَجرٌ عَظِيمٌۭ ﴿١٧٩﴾\\
\textamh{180.\  } & وَلَا يَحسَبَنَّ ٱلَّذِينَ يَبخَلُونَ بِمَآ ءَاتَىٰهُمُ ٱللَّهُ مِن فَضلِهِۦ هُوَ خَيرًۭا لَّهُم ۖ بَل هُوَ شَرٌّۭ لَّهُم ۖ سَيُطَوَّقُونَ مَا بَخِلُوا۟ بِهِۦ يَومَ ٱلقِيَـٰمَةِ ۗ وَلِلَّهِ مِيرَٰثُ ٱلسَّمَـٰوَٟتِ وَٱلأَرضِ ۗ وَٱللَّهُ بِمَا تَعمَلُونَ خَبِيرٌۭ ﴿١٨٠﴾\\
\textamh{181.\  } & لَّقَد سَمِعَ ٱللَّهُ قَولَ ٱلَّذِينَ قَالُوٓا۟ إِنَّ ٱللَّهَ فَقِيرٌۭ وَنَحنُ أَغنِيَآءُ ۘ سَنَكتُبُ مَا قَالُوا۟ وَقَتلَهُمُ ٱلأَنۢبِيَآءَ بِغَيرِ حَقٍّۢ وَنَقُولُ ذُوقُوا۟ عَذَابَ ٱلحَرِيقِ ﴿١٨١﴾\\
\textamh{182.\  } & ذَٟلِكَ بِمَا قَدَّمَت أَيدِيكُم وَأَنَّ ٱللَّهَ لَيسَ بِظَلَّامٍۢ لِّلعَبِيدِ ﴿١٨٢﴾\\
\textamh{183.\  } & ٱلَّذِينَ قَالُوٓا۟ إِنَّ ٱللَّهَ عَهِدَ إِلَينَآ أَلَّا نُؤمِنَ لِرَسُولٍ حَتَّىٰ يَأتِيَنَا بِقُربَانٍۢ تَأكُلُهُ ٱلنَّارُ ۗ قُل قَد جَآءَكُم رُسُلٌۭ مِّن قَبلِى بِٱلبَيِّنَـٰتِ وَبِٱلَّذِى قُلتُم فَلِمَ قَتَلتُمُوهُم إِن كُنتُم صَـٰدِقِينَ ﴿١٨٣﴾\\
\textamh{184.\  } & فَإِن كَذَّبُوكَ فَقَد كُذِّبَ رُسُلٌۭ مِّن قَبلِكَ جَآءُو بِٱلبَيِّنَـٰتِ وَٱلزُّبُرِ وَٱلكِتَـٰبِ ٱلمُنِيرِ ﴿١٨٤﴾\\
\textamh{185.\  } & كُلُّ نَفسٍۢ ذَآئِقَةُ ٱلمَوتِ ۗ وَإِنَّمَا تُوَفَّونَ أُجُورَكُم يَومَ ٱلقِيَـٰمَةِ ۖ فَمَن زُحزِحَ عَنِ ٱلنَّارِ وَأُدخِلَ ٱلجَنَّةَ فَقَد فَازَ ۗ وَمَا ٱلحَيَوٰةُ ٱلدُّنيَآ إِلَّا مَتَـٰعُ ٱلغُرُورِ ﴿١٨٥﴾\\
\textamh{186.\  } & ۞ لَتُبلَوُنَّ فِىٓ أَموَٟلِكُم وَأَنفُسِكُم وَلَتَسمَعُنَّ مِنَ ٱلَّذِينَ أُوتُوا۟ ٱلكِتَـٰبَ مِن قَبلِكُم وَمِنَ ٱلَّذِينَ أَشرَكُوٓا۟ أَذًۭى كَثِيرًۭا ۚ وَإِن تَصبِرُوا۟ وَتَتَّقُوا۟ فَإِنَّ ذَٟلِكَ مِن عَزمِ ٱلأُمُورِ ﴿١٨٦﴾\\
\textamh{187.\  } & وَإِذ أَخَذَ ٱللَّهُ مِيثَـٰقَ ٱلَّذِينَ أُوتُوا۟ ٱلكِتَـٰبَ لَتُبَيِّنُنَّهُۥ لِلنَّاسِ وَلَا تَكتُمُونَهُۥ فَنَبَذُوهُ وَرَآءَ ظُهُورِهِم وَٱشتَرَوا۟ بِهِۦ ثَمَنًۭا قَلِيلًۭا ۖ فَبِئسَ مَا يَشتَرُونَ ﴿١٨٧﴾\\
\textamh{188.\  } & لَا تَحسَبَنَّ ٱلَّذِينَ يَفرَحُونَ بِمَآ أَتَوا۟ وَّيُحِبُّونَ أَن يُحمَدُوا۟ بِمَا لَم يَفعَلُوا۟ فَلَا تَحسَبَنَّهُم بِمَفَازَةٍۢ مِّنَ ٱلعَذَابِ ۖ وَلَهُم عَذَابٌ أَلِيمٌۭ ﴿١٨٨﴾\\
\textamh{189.\  } & وَلِلَّهِ مُلكُ ٱلسَّمَـٰوَٟتِ وَٱلأَرضِ ۗ وَٱللَّهُ عَلَىٰ كُلِّ شَىءٍۢ قَدِيرٌ ﴿١٨٩﴾\\
\textamh{190.\  } & إِنَّ فِى خَلقِ ٱلسَّمَـٰوَٟتِ وَٱلأَرضِ وَٱختِلَـٰفِ ٱلَّيلِ وَٱلنَّهَارِ لَءَايَـٰتٍۢ لِّأُو۟لِى ٱلأَلبَٰبِ ﴿١٩٠﴾\\
\textamh{191.\  } & ٱلَّذِينَ يَذكُرُونَ ٱللَّهَ قِيَـٰمًۭا وَقُعُودًۭا وَعَلَىٰ جُنُوبِهِم وَيَتَفَكَّرُونَ فِى خَلقِ ٱلسَّمَـٰوَٟتِ وَٱلأَرضِ رَبَّنَا مَا خَلَقتَ هَـٰذَا بَٰطِلًۭا سُبحَـٰنَكَ فَقِنَا عَذَابَ ٱلنَّارِ ﴿١٩١﴾\\
\textamh{192.\  } & رَبَّنَآ إِنَّكَ مَن تُدخِلِ ٱلنَّارَ فَقَد أَخزَيتَهُۥ ۖ وَمَا لِلظَّـٰلِمِينَ مِن أَنصَارٍۢ ﴿١٩٢﴾\\
\textamh{193.\  } & رَّبَّنَآ إِنَّنَا سَمِعنَا مُنَادِيًۭا يُنَادِى لِلإِيمَـٰنِ أَن ءَامِنُوا۟ بِرَبِّكُم فَـَٔامَنَّا ۚ رَبَّنَا فَٱغفِر لَنَا ذُنُوبَنَا وَكَفِّر عَنَّا سَيِّـَٔاتِنَا وَتَوَفَّنَا مَعَ ٱلأَبرَارِ ﴿١٩٣﴾\\
\textamh{194.\  } & رَبَّنَا وَءَاتِنَا مَا وَعَدتَّنَا عَلَىٰ رُسُلِكَ وَلَا تُخزِنَا يَومَ ٱلقِيَـٰمَةِ ۗ إِنَّكَ لَا تُخلِفُ ٱلمِيعَادَ ﴿١٩٤﴾\\
\textamh{195.\  } & فَٱستَجَابَ لَهُم رَبُّهُم أَنِّى لَآ أُضِيعُ عَمَلَ عَـٰمِلٍۢ مِّنكُم مِّن ذَكَرٍ أَو أُنثَىٰ ۖ بَعضُكُم مِّنۢ بَعضٍۢ ۖ فَٱلَّذِينَ هَاجَرُوا۟ وَأُخرِجُوا۟ مِن دِيَـٰرِهِم وَأُوذُوا۟ فِى سَبِيلِى وَقَـٰتَلُوا۟ وَقُتِلُوا۟ لَأُكَفِّرَنَّ عَنهُم سَيِّـَٔاتِهِم وَلَأُدخِلَنَّهُم جَنَّـٰتٍۢ تَجرِى مِن تَحتِهَا ٱلأَنهَـٰرُ ثَوَابًۭا مِّن عِندِ ٱللَّهِ ۗ وَٱللَّهُ عِندَهُۥ حُسنُ ٱلثَّوَابِ ﴿١٩٥﴾\\
\textamh{196.\  } & لَا يَغُرَّنَّكَ تَقَلُّبُ ٱلَّذِينَ كَفَرُوا۟ فِى ٱلبِلَـٰدِ ﴿١٩٦﴾\\
\textamh{197.\  } & مَتَـٰعٌۭ قَلِيلٌۭ ثُمَّ مَأوَىٰهُم جَهَنَّمُ ۚ وَبِئسَ ٱلمِهَادُ ﴿١٩٧﴾\\
\textamh{198.\  } & لَـٰكِنِ ٱلَّذِينَ ٱتَّقَوا۟ رَبَّهُم لَهُم جَنَّـٰتٌۭ تَجرِى مِن تَحتِهَا ٱلأَنهَـٰرُ خَـٰلِدِينَ فِيهَا نُزُلًۭا مِّن عِندِ ٱللَّهِ ۗ وَمَا عِندَ ٱللَّهِ خَيرٌۭ لِّلأَبرَارِ ﴿١٩٨﴾\\
\textamh{199.\  } & وَإِنَّ مِن أَهلِ ٱلكِتَـٰبِ لَمَن يُؤمِنُ بِٱللَّهِ وَمَآ أُنزِلَ إِلَيكُم وَمَآ أُنزِلَ إِلَيهِم خَـٰشِعِينَ لِلَّهِ لَا يَشتَرُونَ بِـَٔايَـٰتِ ٱللَّهِ ثَمَنًۭا قَلِيلًا ۗ أُو۟لَـٰٓئِكَ لَهُم أَجرُهُم عِندَ رَبِّهِم ۗ إِنَّ ٱللَّهَ سَرِيعُ ٱلحِسَابِ ﴿١٩٩﴾\\
\textamh{200.\  } & يَـٰٓأَيُّهَا ٱلَّذِينَ ءَامَنُوا۟ ٱصبِرُوا۟ وَصَابِرُوا۟ وَرَابِطُوا۟ وَٱتَّقُوا۟ ٱللَّهَ لَعَلَّكُم تُفلِحُونَ ﴿٢٠٠﴾
\end{longtable} \newpage

