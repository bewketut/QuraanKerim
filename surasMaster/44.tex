%% License: BSD style (Berkley) (i.e. Put the Copyright owner's name always)
%% Writer and Copyright (to): Bewketu(Bilal) Tadilo (2016-17)
\shadowbox{\section{\LR{\textamharic{ሱራቱ አልዱኻን -}  \RL{سوره  الدخان}}}}
\begin{longtable}{%
  @{}
    p{.5\textwidth}
  @{~~~~~~~~~~~~~}||
    p{.5\textwidth}
    @{}
}
\nopagebreak
\textamh{\ \ \ \ \ \  ቢስሚላሂ አራህመኒ ራሂይም } &  بِسمِ ٱللَّهِ ٱلرَّحمَـٰنِ ٱلرَّحِيمِ\\
\textamh{1.\  } &  حمٓ ﴿١﴾\\
\textamh{2.\  } & وَٱلكِتَـٰبِ ٱلمُبِينِ ﴿٢﴾\\
\textamh{3.\  } & إِنَّآ أَنزَلنَـٰهُ فِى لَيلَةٍۢ مُّبَٰرَكَةٍ ۚ إِنَّا كُنَّا مُنذِرِينَ ﴿٣﴾\\
\textamh{4.\  } & فِيهَا يُفرَقُ كُلُّ أَمرٍ حَكِيمٍ ﴿٤﴾\\
\textamh{5.\  } & أَمرًۭا مِّن عِندِنَآ ۚ إِنَّا كُنَّا مُرسِلِينَ ﴿٥﴾\\
\textamh{6.\  } & رَحمَةًۭ مِّن رَّبِّكَ ۚ إِنَّهُۥ هُوَ ٱلسَّمِيعُ ٱلعَلِيمُ ﴿٦﴾\\
\textamh{7.\  } & رَبِّ ٱلسَّمَـٰوَٟتِ وَٱلأَرضِ وَمَا بَينَهُمَآ ۖ إِن كُنتُم مُّوقِنِينَ ﴿٧﴾\\
\textamh{8.\  } & لَآ إِلَـٰهَ إِلَّا هُوَ يُحىِۦ وَيُمِيتُ ۖ رَبُّكُم وَرَبُّ ءَابَآئِكُمُ ٱلأَوَّلِينَ ﴿٨﴾\\
\textamh{9.\  } & بَل هُم فِى شَكٍّۢ يَلعَبُونَ ﴿٩﴾\\
\textamh{10.\  } & فَٱرتَقِب يَومَ تَأتِى ٱلسَّمَآءُ بِدُخَانٍۢ مُّبِينٍۢ ﴿١٠﴾\\
\textamh{11.\  } & يَغشَى ٱلنَّاسَ ۖ هَـٰذَا عَذَابٌ أَلِيمٌۭ ﴿١١﴾\\
\textamh{12.\  } & رَّبَّنَا ٱكشِف عَنَّا ٱلعَذَابَ إِنَّا مُؤمِنُونَ ﴿١٢﴾\\
\textamh{13.\  } & أَنَّىٰ لَهُمُ ٱلذِّكرَىٰ وَقَد جَآءَهُم رَسُولٌۭ مُّبِينٌۭ ﴿١٣﴾\\
\textamh{14.\  } & ثُمَّ تَوَلَّوا۟ عَنهُ وَقَالُوا۟ مُعَلَّمٌۭ مَّجنُونٌ ﴿١٤﴾\\
\textamh{15.\  } & إِنَّا كَاشِفُوا۟ ٱلعَذَابِ قَلِيلًا ۚ إِنَّكُم عَآئِدُونَ ﴿١٥﴾\\
\textamh{16.\  } & يَومَ نَبطِشُ ٱلبَطشَةَ ٱلكُبرَىٰٓ إِنَّا مُنتَقِمُونَ ﴿١٦﴾\\
\textamh{17.\  } & ۞ وَلَقَد فَتَنَّا قَبلَهُم قَومَ فِرعَونَ وَجَآءَهُم رَسُولٌۭ كَرِيمٌ ﴿١٧﴾\\
\textamh{18.\  } & أَن أَدُّوٓا۟ إِلَىَّ عِبَادَ ٱللَّهِ ۖ إِنِّى لَكُم رَسُولٌ أَمِينٌۭ ﴿١٨﴾\\
\textamh{19.\  } & وَأَن لَّا تَعلُوا۟ عَلَى ٱللَّهِ ۖ إِنِّىٓ ءَاتِيكُم بِسُلطَٰنٍۢ مُّبِينٍۢ ﴿١٩﴾\\
\textamh{20.\  } & وَإِنِّى عُذتُ بِرَبِّى وَرَبِّكُم أَن تَرجُمُونِ ﴿٢٠﴾\\
\textamh{21.\  } & وَإِن لَّم تُؤمِنُوا۟ لِى فَٱعتَزِلُونِ ﴿٢١﴾\\
\textamh{22.\  } & فَدَعَا رَبَّهُۥٓ أَنَّ هَـٰٓؤُلَآءِ قَومٌۭ مُّجرِمُونَ ﴿٢٢﴾\\
\textamh{23.\  } & فَأَسرِ بِعِبَادِى لَيلًا إِنَّكُم مُّتَّبَعُونَ ﴿٢٣﴾\\
\textamh{24.\  } & وَٱترُكِ ٱلبَحرَ رَهوًا ۖ إِنَّهُم جُندٌۭ مُّغرَقُونَ ﴿٢٤﴾\\
\textamh{25.\  } & كَم تَرَكُوا۟ مِن جَنَّـٰتٍۢ وَعُيُونٍۢ ﴿٢٥﴾\\
\textamh{26.\  } & وَزُرُوعٍۢ وَمَقَامٍۢ كَرِيمٍۢ ﴿٢٦﴾\\
\textamh{27.\  } & وَنَعمَةٍۢ كَانُوا۟ فِيهَا فَـٰكِهِينَ ﴿٢٧﴾\\
\textamh{28.\  } & كَذَٟلِكَ ۖ وَأَورَثنَـٰهَا قَومًا ءَاخَرِينَ ﴿٢٨﴾\\
\textamh{29.\  } & فَمَا بَكَت عَلَيهِمُ ٱلسَّمَآءُ وَٱلأَرضُ وَمَا كَانُوا۟ مُنظَرِينَ ﴿٢٩﴾\\
\textamh{30.\  } & وَلَقَد نَجَّينَا بَنِىٓ إِسرَٰٓءِيلَ مِنَ ٱلعَذَابِ ٱلمُهِينِ ﴿٣٠﴾\\
\textamh{31.\  } & مِن فِرعَونَ ۚ إِنَّهُۥ كَانَ عَالِيًۭا مِّنَ ٱلمُسرِفِينَ ﴿٣١﴾\\
\textamh{32.\  } & وَلَقَدِ ٱختَرنَـٰهُم عَلَىٰ عِلمٍ عَلَى ٱلعَـٰلَمِينَ ﴿٣٢﴾\\
\textamh{33.\  } & وَءَاتَينَـٰهُم مِّنَ ٱلءَايَـٰتِ مَا فِيهِ بَلَـٰٓؤٌۭا۟ مُّبِينٌ ﴿٣٣﴾\\
\textamh{34.\  } & إِنَّ هَـٰٓؤُلَآءِ لَيَقُولُونَ ﴿٣٤﴾\\
\textamh{35.\  } & إِن هِىَ إِلَّا مَوتَتُنَا ٱلأُولَىٰ وَمَا نَحنُ بِمُنشَرِينَ ﴿٣٥﴾\\
\textamh{36.\  } & فَأتُوا۟ بِـَٔابَآئِنَآ إِن كُنتُم صَـٰدِقِينَ ﴿٣٦﴾\\
\textamh{37.\  } & أَهُم خَيرٌ أَم قَومُ تُبَّعٍۢ وَٱلَّذِينَ مِن قَبلِهِم ۚ أَهلَكنَـٰهُم ۖ إِنَّهُم كَانُوا۟ مُجرِمِينَ ﴿٣٧﴾\\
\textamh{38.\  } & وَمَا خَلَقنَا ٱلسَّمَـٰوَٟتِ وَٱلأَرضَ وَمَا بَينَهُمَا لَـٰعِبِينَ ﴿٣٨﴾\\
\textamh{39.\  } & مَا خَلَقنَـٰهُمَآ إِلَّا بِٱلحَقِّ وَلَـٰكِنَّ أَكثَرَهُم لَا يَعلَمُونَ ﴿٣٩﴾\\
\textamh{40.\  } & إِنَّ يَومَ ٱلفَصلِ مِيقَـٰتُهُم أَجمَعِينَ ﴿٤٠﴾\\
\textamh{41.\  } & يَومَ لَا يُغنِى مَولًى عَن مَّولًۭى شَيـًۭٔا وَلَا هُم يُنصَرُونَ ﴿٤١﴾\\
\textamh{42.\  } & إِلَّا مَن رَّحِمَ ٱللَّهُ ۚ إِنَّهُۥ هُوَ ٱلعَزِيزُ ٱلرَّحِيمُ ﴿٤٢﴾\\
\textamh{43.\  } & إِنَّ شَجَرَتَ ٱلزَّقُّومِ ﴿٤٣﴾\\
\textamh{44.\  } & طَعَامُ ٱلأَثِيمِ ﴿٤٤﴾\\
\textamh{45.\  } & كَٱلمُهلِ يَغلِى فِى ٱلبُطُونِ ﴿٤٥﴾\\
\textamh{46.\  } & كَغَلىِ ٱلحَمِيمِ ﴿٤٦﴾\\
\textamh{47.\  } & خُذُوهُ فَٱعتِلُوهُ إِلَىٰ سَوَآءِ ٱلجَحِيمِ ﴿٤٧﴾\\
\textamh{48.\  } & ثُمَّ صُبُّوا۟ فَوقَ رَأسِهِۦ مِن عَذَابِ ٱلحَمِيمِ ﴿٤٨﴾\\
\textamh{49.\  } & ذُق إِنَّكَ أَنتَ ٱلعَزِيزُ ٱلكَرِيمُ ﴿٤٩﴾\\
\textamh{50.\  } & إِنَّ هَـٰذَا مَا كُنتُم بِهِۦ تَمتَرُونَ ﴿٥٠﴾\\
\textamh{51.\  } & إِنَّ ٱلمُتَّقِينَ فِى مَقَامٍ أَمِينٍۢ ﴿٥١﴾\\
\textamh{52.\  } & فِى جَنَّـٰتٍۢ وَعُيُونٍۢ ﴿٥٢﴾\\
\textamh{53.\  } & يَلبَسُونَ مِن سُندُسٍۢ وَإِستَبرَقٍۢ مُّتَقَـٰبِلِينَ ﴿٥٣﴾\\
\textamh{54.\  } & كَذَٟلِكَ وَزَوَّجنَـٰهُم بِحُورٍ عِينٍۢ ﴿٥٤﴾\\
\textamh{55.\  } & يَدعُونَ فِيهَا بِكُلِّ فَـٰكِهَةٍ ءَامِنِينَ ﴿٥٥﴾\\
\textamh{56.\  } & لَا يَذُوقُونَ فِيهَا ٱلمَوتَ إِلَّا ٱلمَوتَةَ ٱلأُولَىٰ ۖ وَوَقَىٰهُم عَذَابَ ٱلجَحِيمِ ﴿٥٦﴾\\
\textamh{57.\  } & فَضلًۭا مِّن رَّبِّكَ ۚ ذَٟلِكَ هُوَ ٱلفَوزُ ٱلعَظِيمُ ﴿٥٧﴾\\
\textamh{58.\  } & فَإِنَّمَا يَسَّرنَـٰهُ بِلِسَانِكَ لَعَلَّهُم يَتَذَكَّرُونَ ﴿٥٨﴾\\
\textamh{59.\  } & فَٱرتَقِب إِنَّهُم مُّرتَقِبُونَ ﴿٥٩﴾\\
\end{longtable} \newpage
