%% License: BSD style (Berkley) (i.e. Put the Copyright owner's name always)
%% Writer and Copyright (to): Bewketu(Bilal) Tadilo (2016-17)
\shadowbox{\section{\LR{\textamharic{ሱራቱ ቃፍ -}  \RL{سوره  ق}}}}
\begin{longtable}{%
  @{}
    p{.5\textwidth}
  @{~~~~~~~~~~~~~}||
    p{.5\textwidth}
    @{}
}
\nopagebreak
\textamh{\ \ \ \ \ \  ቢስሚላሂ አራህመኒ ራሂይም } &  بِسمِ ٱللَّهِ ٱلرَّحمَـٰنِ ٱلرَّحِيمِ\\
\textamh{1.\  } &  قٓ ۚ وَٱلقُرءَانِ ٱلمَجِيدِ ﴿١﴾\\
\textamh{2.\  } & بَل عَجِبُوٓا۟ أَن جَآءَهُم مُّنذِرٌۭ مِّنهُم فَقَالَ ٱلكَـٰفِرُونَ هَـٰذَا شَىءٌ عَجِيبٌ ﴿٢﴾\\
\textamh{3.\  } & أَءِذَا مِتنَا وَكُنَّا تُرَابًۭا ۖ ذَٟلِكَ رَجعٌۢ بَعِيدٌۭ ﴿٣﴾\\
\textamh{4.\  } & قَد عَلِمنَا مَا تَنقُصُ ٱلأَرضُ مِنهُم ۖ وَعِندَنَا كِتَـٰبٌ حَفِيظٌۢ ﴿٤﴾\\
\textamh{5.\  } & بَل كَذَّبُوا۟ بِٱلحَقِّ لَمَّا جَآءَهُم فَهُم فِىٓ أَمرٍۢ مَّرِيجٍ ﴿٥﴾\\
\textamh{6.\  } & أَفَلَم يَنظُرُوٓا۟ إِلَى ٱلسَّمَآءِ فَوقَهُم كَيفَ بَنَينَـٰهَا وَزَيَّنَّـٰهَا وَمَا لَهَا مِن فُرُوجٍۢ ﴿٦﴾\\
\textamh{7.\  } & وَٱلأَرضَ مَدَدنَـٰهَا وَأَلقَينَا فِيهَا رَوَٟسِىَ وَأَنۢبَتنَا فِيهَا مِن كُلِّ زَوجٍۭ بَهِيجٍۢ ﴿٧﴾\\
\textamh{8.\  } & تَبصِرَةًۭ وَذِكرَىٰ لِكُلِّ عَبدٍۢ مُّنِيبٍۢ ﴿٨﴾\\
\textamh{9.\  } & وَنَزَّلنَا مِنَ ٱلسَّمَآءِ مَآءًۭ مُّبَٰرَكًۭا فَأَنۢبَتنَا بِهِۦ جَنَّـٰتٍۢ وَحَبَّ ٱلحَصِيدِ ﴿٩﴾\\
\textamh{10.\  } & وَٱلنَّخلَ بَاسِقَـٰتٍۢ لَّهَا طَلعٌۭ نَّضِيدٌۭ ﴿١٠﴾\\
\textamh{11.\  } & رِّزقًۭا لِّلعِبَادِ ۖ وَأَحيَينَا بِهِۦ بَلدَةًۭ مَّيتًۭا ۚ كَذَٟلِكَ ٱلخُرُوجُ ﴿١١﴾\\
\textamh{12.\  } & كَذَّبَت قَبلَهُم قَومُ نُوحٍۢ وَأَصحَـٰبُ ٱلرَّسِّ وَثَمُودُ ﴿١٢﴾\\
\textamh{13.\  } & وَعَادٌۭ وَفِرعَونُ وَإِخوَٟنُ لُوطٍۢ ﴿١٣﴾\\
\textamh{14.\  } & وَأَصحَـٰبُ ٱلأَيكَةِ وَقَومُ تُبَّعٍۢ ۚ كُلٌّۭ كَذَّبَ ٱلرُّسُلَ فَحَقَّ وَعِيدِ ﴿١٤﴾\\
\textamh{15.\  } & أَفَعَيِينَا بِٱلخَلقِ ٱلأَوَّلِ ۚ بَل هُم فِى لَبسٍۢ مِّن خَلقٍۢ جَدِيدٍۢ ﴿١٥﴾\\
\textamh{16.\  } & وَلَقَد خَلَقنَا ٱلإِنسَـٰنَ وَنَعلَمُ مَا تُوَسوِسُ بِهِۦ نَفسُهُۥ ۖ وَنَحنُ أَقرَبُ إِلَيهِ مِن حَبلِ ٱلوَرِيدِ ﴿١٦﴾\\
\textamh{17.\  } & إِذ يَتَلَقَّى ٱلمُتَلَقِّيَانِ عَنِ ٱليَمِينِ وَعَنِ ٱلشِّمَالِ قَعِيدٌۭ ﴿١٧﴾\\
\textamh{18.\  } & مَّا يَلفِظُ مِن قَولٍ إِلَّا لَدَيهِ رَقِيبٌ عَتِيدٌۭ ﴿١٨﴾\\
\textamh{19.\  } & وَجَآءَت سَكرَةُ ٱلمَوتِ بِٱلحَقِّ ۖ ذَٟلِكَ مَا كُنتَ مِنهُ تَحِيدُ ﴿١٩﴾\\
\textamh{20.\  } & وَنُفِخَ فِى ٱلصُّورِ ۚ ذَٟلِكَ يَومُ ٱلوَعِيدِ ﴿٢٠﴾\\
\textamh{21.\  } & وَجَآءَت كُلُّ نَفسٍۢ مَّعَهَا سَآئِقٌۭ وَشَهِيدٌۭ ﴿٢١﴾\\
\textamh{22.\  } & لَّقَد كُنتَ فِى غَفلَةٍۢ مِّن هَـٰذَا فَكَشَفنَا عَنكَ غِطَآءَكَ فَبَصَرُكَ ٱليَومَ حَدِيدٌۭ ﴿٢٢﴾\\
\textamh{23.\  } & وَقَالَ قَرِينُهُۥ هَـٰذَا مَا لَدَىَّ عَتِيدٌ ﴿٢٣﴾\\
\textamh{24.\  } & أَلقِيَا فِى جَهَنَّمَ كُلَّ كَفَّارٍ عَنِيدٍۢ ﴿٢٤﴾\\
\textamh{25.\  } & مَّنَّاعٍۢ لِّلخَيرِ مُعتَدٍۢ مُّرِيبٍ ﴿٢٥﴾\\
\textamh{26.\  } & ٱلَّذِى جَعَلَ مَعَ ٱللَّهِ إِلَـٰهًا ءَاخَرَ فَأَلقِيَاهُ فِى ٱلعَذَابِ ٱلشَّدِيدِ ﴿٢٦﴾\\
\textamh{27.\  } & ۞ قَالَ قَرِينُهُۥ رَبَّنَا مَآ أَطغَيتُهُۥ وَلَـٰكِن كَانَ فِى ضَلَـٰلٍۭ بَعِيدٍۢ ﴿٢٧﴾\\
\textamh{28.\  } & قَالَ لَا تَختَصِمُوا۟ لَدَىَّ وَقَد قَدَّمتُ إِلَيكُم بِٱلوَعِيدِ ﴿٢٨﴾\\
\textamh{29.\  } & مَا يُبَدَّلُ ٱلقَولُ لَدَىَّ وَمَآ أَنَا۠ بِظَلَّٰمٍۢ لِّلعَبِيدِ ﴿٢٩﴾\\
\textamh{30.\  } & يَومَ نَقُولُ لِجَهَنَّمَ هَلِ ٱمتَلَأتِ وَتَقُولُ هَل مِن مَّزِيدٍۢ ﴿٣٠﴾\\
\textamh{31.\  } & وَأُزلِفَتِ ٱلجَنَّةُ لِلمُتَّقِينَ غَيرَ بَعِيدٍ ﴿٣١﴾\\
\textamh{32.\  } & هَـٰذَا مَا تُوعَدُونَ لِكُلِّ أَوَّابٍ حَفِيظٍۢ ﴿٣٢﴾\\
\textamh{33.\  } & مَّن خَشِىَ ٱلرَّحمَـٰنَ بِٱلغَيبِ وَجَآءَ بِقَلبٍۢ مُّنِيبٍ ﴿٣٣﴾\\
\textamh{34.\  } & ٱدخُلُوهَا بِسَلَـٰمٍۢ ۖ ذَٟلِكَ يَومُ ٱلخُلُودِ ﴿٣٤﴾\\
\textamh{35.\  } & لَهُم مَّا يَشَآءُونَ فِيهَا وَلَدَينَا مَزِيدٌۭ ﴿٣٥﴾\\
\textamh{36.\  } & وَكَم أَهلَكنَا قَبلَهُم مِّن قَرنٍ هُم أَشَدُّ مِنهُم بَطشًۭا فَنَقَّبُوا۟ فِى ٱلبِلَـٰدِ هَل مِن مَّحِيصٍ ﴿٣٦﴾\\
\textamh{37.\  } & إِنَّ فِى ذَٟلِكَ لَذِكرَىٰ لِمَن كَانَ لَهُۥ قَلبٌ أَو أَلقَى ٱلسَّمعَ وَهُوَ شَهِيدٌۭ ﴿٣٧﴾\\
\textamh{38.\  } & وَلَقَد خَلَقنَا ٱلسَّمَـٰوَٟتِ وَٱلأَرضَ وَمَا بَينَهُمَا فِى سِتَّةِ أَيَّامٍۢ وَمَا مَسَّنَا مِن لُّغُوبٍۢ ﴿٣٨﴾\\
\textamh{39.\  } & فَٱصبِر عَلَىٰ مَا يَقُولُونَ وَسَبِّح بِحَمدِ رَبِّكَ قَبلَ طُلُوعِ ٱلشَّمسِ وَقَبلَ ٱلغُرُوبِ ﴿٣٩﴾\\
\textamh{40.\  } & وَمِنَ ٱلَّيلِ فَسَبِّحهُ وَأَدبَٰرَ ٱلسُّجُودِ ﴿٤٠﴾\\
\textamh{41.\  } & وَٱستَمِع يَومَ يُنَادِ ٱلمُنَادِ مِن مَّكَانٍۢ قَرِيبٍۢ ﴿٤١﴾\\
\textamh{42.\  } & يَومَ يَسمَعُونَ ٱلصَّيحَةَ بِٱلحَقِّ ۚ ذَٟلِكَ يَومُ ٱلخُرُوجِ ﴿٤٢﴾\\
\textamh{43.\  } & إِنَّا نَحنُ نُحىِۦ وَنُمِيتُ وَإِلَينَا ٱلمَصِيرُ ﴿٤٣﴾\\
\textamh{44.\  } & يَومَ تَشَقَّقُ ٱلأَرضُ عَنهُم سِرَاعًۭا ۚ ذَٟلِكَ حَشرٌ عَلَينَا يَسِيرٌۭ ﴿٤٤﴾\\
\textamh{45.\  } & نَّحنُ أَعلَمُ بِمَا يَقُولُونَ ۖ وَمَآ أَنتَ عَلَيهِم بِجَبَّارٍۢ ۖ فَذَكِّر بِٱلقُرءَانِ مَن يَخَافُ وَعِيدِ ﴿٤٥﴾\\
\end{longtable} \newpage
